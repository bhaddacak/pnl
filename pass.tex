\chapter{A book \headhl{is read} by me}\label{chap:pass}

\phantomsection
\addcontentsline{toc}{section}{Introduction to Passive Voice}
\section*{Introduction to Passive Voice}

In this chapter we will extend our understanding to passive structure in P\=ali. The topic is quite complicated but really important. So, tighten your seatbelt and drive through carefully. Simply put, `passive voice' is a kind of inverted version of normal way of saying. Basically, when we say things in English we form a structure like subject-verb-object (SVO). This is called active structure: someone does something to an object. For example, ``I kick a ball'' means I apply some force with my foot to an elastic round-shaped object. Technically we call `I' in this sentence agent, and `ball' patient (see Chapter \ref{chap:cases} for some more information). We can revert this to ``A ball is kicked by me'' meaning the ball receives a kick applied by me. Now patient turns to be (grammatical) subject of the sentence. That is quite easy. But, in P\=ali it is more complicated than that.

To understand the crux of this, let us make clear some basic things first. Generally, we divide verbs into transitive (those that need object, e.g.\ ``I eat food'' and the kicking example above) and intransitive (those that do not need object, e.g.\ ``I sleep''). It is so in P\=ali. When we talk about object, it is obvious that we are talking about transitive verbs. And passive structure in English has things to do only with transitive verbs and their object. Unfortunately, in P\=ali it is not quite so. We can even make a passive sentence from intransitive verbs. That is amazing (or you might think it is terrible). So, please prepare your mind for this weird thing.

Before you can understand passive voice in P\=ali, you have to tackle `middle' voice first. If you are not a learner of Greek, you are likely to be baffled by this. Although the use of middle voice in P\=ali was out of fashion long time ago, even before the first P\=ali prose was composed, it left remnants in the system. That is why we have to learn it, but in a less rigorous manner. That is the very reason we touch on this matter in later part of our lessons. In practice, speaking P\=ali in daily basis, if there is such thing, does not require any use of middle voice. But for a scholastic purpose, we can find its uses in grammar textbooks, and so do we in our lessons after you know how to use it.

Then, what is middle voice? Let us go step by step. First, what is `voice' after all? Simply put, it can be defined in this way: ``[T]he question of whether the subject performs or receives the verb's action is called \emph{voice}.''\footnote{\citealp[p.~105]{fairbairn:understanding}} That is straightforward on practical level. On conceptual level, voice has things to do with perspectives from which a situation is presented.\footnote{\citealp[p.~466]{brownmiller:dict}} That is to say, in active structure like ``I kick a ball,'' the focus of the event is on the action of agent `I.' On the other hand, in passive structure like ``A ball is kicked (by me),'' the focus is now moved to patient `ball' which has received the action. The doer of the action in latter case is optional. Without that information provided, the sentence is still valid in form. Still, the agent is implied but not informed. Middle voice goes between these two perspectives.

In some situations, agent and patient can be the same person, for example ``I get myself delighted by reading books.'' Although the meaning does not go far from ``I am delighted by reading books'' (passive) or ``Reading books delights me'' (active), the implication of these sentences are quite different. Grammatically, we can call the structure of ``I get myself delighted by reading books'' a kind of middle voice, because the subject is performing the action on itself.\footnote{\citealp[p.~114]{fairbairn:understanding}. See also a discussion on middle voice in \citealp[pp.~105--6]{pinker:stuff}.} A more unusual example is ``A ball gets itself kicked.'' The only focus in this sentence is on the patient, and the agent is completely absent or put aside.\footnote{It is explained that the subject in middle structure is neither patient nor agent but the `participant' that controls the situation \citep[pp.~466--7]{brownmiller:dict}.} You can even say this, not before your English teachers, ``The ball is kicking perfectly.'' This does not mean the ball is kicking itself, but it is being kicked well regardless of what or who the kicker is. That is typical middle voice in English. Here are some other examples given by a reliable source:\footnote{\citealp[p.~466]{brownmiller:dict}} 

\begin{quote}
``This sweater washes well.''\\
(It means the sweater is normally in clean condition.)\\[1.5mm]
``One bomb didn't guide and crashed'' (Army communiqu\`e)\\
(This means the bomb itself is to be blamed for not being guided.)\\[1.5mm]
``The course is jumping well'' (TV presenter)\\
(This perhaps means ``The racers in the course are performing well.'')
\end{quote}

Now we come to P\=ali. If you explore conjugation tables depicted in Appendix \ref{chap:conj}, you can see that each tense and mood in P\=ali verb classes has two \pali{pada}s: \pali{parassapada} and \pali{attanopada}. Scholars translate this \pali{pada} as `voice.' Hence, they are `active' voice and `middle' voice respectively. Literally, \pali{parassapada} means `term for other.' It denotes that verbs in this form are the actions done to others. And \pali{attanopada} means `term for oneself'---the actions done to one's own self.\footnote{It is also worth reading Warder on `middle' conjugation \citep[pp.~314--6]{warder:intro}. From a study of the use of these reflexive forms in the Collection of Long Discourses (D\=ighanik\=aya), he concludes that ``the shade of meaning they carry is simply a poetic, dramatic or elevated one, adding emphasis or dignity'' (p.~316).} That fits to our definition of middle voice above. By principle, this means you should use \pali{parassapada} forms in active structure, and \pali{attanopada} forms in passive and middle structure, even though evidence from the scriptures tells you otherwise. Here is an example of middle voice in use:\footnote{This is suggested by Vito Perniola. It is also worth reading his explanation on middle voice. See \citealp[pp.~339--41]{perniola:grammar}.}

\begin{quote}
\pali{kacci, samma s\=arathi, kum\=aro uyy\=anabh\=umiy\=a abhiramittha}\footnote{D2\,45 (DN\,14). The verb \pali{abhiramittha} is in perfect tense, middle voice.}\\
``Mr.\,driver, did the prince enjoy himself in the garden?''
\end{quote}

As you might notice, I carefully use `voice' here because it may cause a confusion. Voice in English and P\=ali may share some aspects, but they are not exactly the same thing, once you follow the scholars' definition. Voice in P\=ali denotes certain verb forms. It has only two kinds, active and middle voice as described above. But when we talk about structure of sentences, it can be active, causative, or passive structure and so on (more about these later). Some teachers mix these notions up causing a lot of headache in students. In Chapter \ref{chap:vform} I call what I use `structure' here `\emph{stance}' to differentiate it from `voice.' However, when I talk generally about voice, the English notion of voice may be applied. Sometimes, I cause myself a headache too.

That is all you need to know about middle voice in P\=ali. In fact, that is the only information we have on this obsolete verb form. For some more information, see Chapter \ref{chap:vclass}. I summarize practical rules on using voice in P\=ali as follows:
\begin{enumerate}
\item Use active voice most of the time in all structures.
\item Use middle voice in a classroom or other learning context, and in poetic works (if necessary).
\item Use verbal \pali{kita}s instead in passive structure, if possible. For past tense, for example, it is recommended to use verbs in \pali{ta} form. For imp.\ and opt., if the meaning is applicable, using \pali{tabba} and \pali{an\=iya} form is easier.
\item When reading texts, you have to recognize both active and middle voice. So, do not just throw middle voice away. That is the main reason we learn all of these.
\end{enumerate}

The next thing you need to know is what I call structure or stance. In P\=ali we call it \pali{v\=acaka}. There are five types of structure: (1) active structure, (2) causative structure, (3) passive structure, (4) impersonal passive structure, and (5) causal passive structure (for more information see Chapter \ref{chap:vform}). In this chapter we try to tackle two of them: passive and impersonal passive structure. We have done already a lot on active structure, and we will learn both causative and casusal passive structure in Chapter \ref{chap:caus}.

Unlike English, which you can easily use `be' or `get' plus a verb in past participle to form a passive sentence, in P\=ali it is a little more complicated. I summarize a guideline on composing a passive sentence as follows:
\begin{enumerate}
\item Choose a verb to use, be aware of its root and possible variation. Roots are listed in Appendix \ref{chap:roots}. For common verbs, you can see in the vocaburary (page \pageref{vocab:verb} onwards). From present forms, you can determine the root or stem by reversed processing.
\item Apply \pali{ya} (\pali{paccaya}) to the verb stem. Sometimes \pali{i} or \pali{\=i} is also added before that. This is the (real) marker of passive voice. To learn how \pali{ya} works, see page \pageref{pacca:ya2}.
\item Apply a \pali{vibhatti} of \pali{attanopada} after that, corresponding to the intended tense or mood, as well as person and number of the subject (\pali{parassapada} can be optionally used).
\item Apply nominative case to patient, the receiver of the action. This is the subject.
\item Apply instrumental case to agent, the doer of the action (if any). This is equivalent to `by \ldots' phrase in English.\footnote{Occasionally, we can find that genitive case can be used in this position.}
\item Compose all components in a proper order.
\end{enumerate}

For example, \pali{gacchati} ([One] goes) comes from root \pali{gamu}, but the stem we normally use is \pali{gacch}. To make this passive, we get \pali{gacch + \=i + ya}, hence \pali{gacch\=iya}.\footnote{As far as I know, there is no explicit rule whatsoever when \pali{\=i} or \pali{i} should be added. Textbooks just say sometimes it is so (Kacc\,442, R\=upa\,448, Sadd\,922, Mogg\,6.37). You have to observe these for a while, then you will get a knack. Practically, if there is no typical form to follow, just use whatever sounds best to you.} Then we finish this with \pali{attanopada} ending (see Appendix \ref{chap:conj}). Theoretically, here are examples of passive voice in some variety. Please note carefully on subject-verb agreement.

\begin{quote}
\pali{maggo tena/t\=aya gacch\=iyate}\\
``A path is gone by him/her.''\\[1.5mm]
\pali{magg\=a tehi/t\=ahi gacch\=iyante}\\
``Paths are gone by them.''\\[1.5mm]
\pali{tva\d m janena gacch\=iyase}\\
``You are gone [to] by a person.''\\[1.5mm]
\pali{tumhe janehi gacch\=iyavhe}\\
``You [all] are gone [to] by people.''\\[1.5mm]
\pali{aha\d m janena gacch\=iye}\\
``I am gone [to] by a person.''\\[1.5mm]
\pali{maya\d m janena gacch\=iy\=amhe}\\
``We are gone [to] by a person.''\\[1.5mm]
\pali{maggo ma\d m janena gacch\=iyate}\\
``A path is gone by a person to me.''\\[1.5mm]
\pali{magg\=a ma\d m janena gacch\=iyante}\\
``Paths are gone by a person to me.''\\[1.5mm]
\end{quote}

Let us do our heading task together. Here is how to say ``A book is read by me'' step be step:
\begin{enumerate}
\item We find \pali{pa\d thati} that means `to read.' The root of this is \pali{pa\d tha}.
\item Adding \pali{ya} to it, we get \pali{pa\d thaya}.\footnote{Although \pali{pa\d th\=iya} may sound a little better, let us follow a straight way of doing.}
\item For present tense, 3rd person, singular, we apply \pali{te} to this, hence \pali{pa\d thayate}.
\item Applying nominative case to `book,' we get \pali{potthako} (m.).
\item Applying instrumental case to `me,' we get \pali{may\=a} or \pali{me}.
\end{enumerate}

Finally, we get this sentence:

\palisample{potthako may\=a/me pa\d thayate.\sampleor[or, alternatively]potthako may\=a/me pa\d thayati.}

The only difficulty of forming a passive verb is when \pali{ya} is applied, several unexpected things can happen, as you can see on page \pageref{pacca:ya2} onwards. That makes the outcome of \pali{ya} not easily recognized sometimes. And unfortunately, you hardly find verbs with \pali{ya} in a normal dictionary. And worst, verbs having \pali{ya} near the end are not necessary to be passive. Some are of verb group 3 (\pali{diva})\footnote{About verb groups, see Chapter \ref{chap:vform}.} which have \pali{ya} as their group \pali{paccaya} (see page \pageref{pacca:ya1}), e.g.\ \pali{gh\=ayati} ([One] smells).\footnote{With shared \pali{ya} forms, it becomes difficult to tell active from passive structure of this verb group. See \citealp[p.~63]{warder:intro}.} And some verbs are created from nouns with a help of \pali{\=aya} (see page \pageref{pacca:aaya}), e.g.\ \pali{nidd\=ayati} ([One] sleeps). Your only viable treatment is to remember peculiar passive forms as many as possible.

Let us go into this for a while for better understanding. When I say ``I give a book to you,'' I put it like this:

\palisample{aha\d m te/tuyha\d m potthaka\d m dad\=ami.}

Changing this to passive sentence, we get ``A book is given to you by me.'' The passive form of \pali{d\=a} is \pali{d\=iyati}.\footnote{Kacc\,502, R\=upa\,493, Sadd\,1014, Mogg\,5.137} Then we get this:

\palisample{may\=a te potthako d\=iyati.}

Some teachers might protest me why I do not use \pali{d\=iyate}. The reason I want to emphasize is that \pali{d\=iyati} has more uses in the canon. The only instance I find \pali{d\=iyate} in use is ``\pali{Bhojana\d m d\=iyate nicca\d m}''\footnote{Pv\,306} (Food is given constantly). That is in a verse.

Let us try a little more challenging one. Suppose, we are in an ancient society and you owe me as a slave. Then you give me to a king. I describe the event as ``I am given to a king by you.'' The P\=ali equivalent of this will be:

\palisample{aha\d m tay\=a ra\~n\~no d\=iy\=ami. \sampleor[or with middle voice] aha\d m tay\=a ra\~n\~no d\=iye.}

Using \pali{te} instead of \pali{tay\=a} in this sentence may cause an ambiguity, for it can be read as ``I am given to you (and) to king.'' A thing to remember here is you have to maintain the agreement between subject and verb.

Another verb that is often found in the texts is `to say,' \pali{vadati}\footnote{This term comes from \pali{vada}, but \pali{vuccati} is from \pali{vaca} of the same meaning. There is no use of present form of \pali{vaca}, see PTSD in `vatti.'} and its passive \pali{vuccati}.\footnote{Kacc\,487, R\=upa\,478, Sadd\,978} When I say ``I call this thing `a book','' I put it as:

\palisample{aha\d m ima\d m vatthu\d m `potthako'ti vad\=ami.}

And ``This thing is called `a book'\,'' can be said as this:

\palisample{ida\d m vatthu\d m `potthako'ti vuccati.}

We will find similar uses of this in the texts, particularly when terms are defined. Here is an example from the Vinaya.

\begin{quote}
\pali{Ogu\d n\d thitas\=iso n\=ama sas\=isa\d m p\=aruto vuccati.}\footnote{Buv2\,644}\\
``[What is] called \pali{ogu\d n\d thitas\=isa} is said [to be one who was] veiled over the head.''
\end{quote}

We can find \pali{vuccate} mostly in verses, for example:

\begin{quote}
\pali{Sabbe bhog\=a vinassanti, ra\~n\~no ta\d m vuccate agha\d m.}\footnote{Ja\,16:335}\\
``All possessions perish. That is said to be a king's pain.''
\end{quote}

Now you can see that why middle voice is not necessary for creating passive sentences. The key factor of passive verb forms is in fact \pali{ya} regardless of whatever voice we use. From now on, if I say \emph{passive verb form}, it means a verb with \pali{ya} applied, ending with either active or middle voice \pali{vibhatti}. So, for \emph{active verb form} I just mean a verb without \pali{ya} regardless of its voice.

Now we move to a bizarre aspect of passive verb form. In English, we do not use intransitive verbs in passive voice. Have you ever tried this? Changing ``I stand'' into a passive sentence will dumbfound you. At best, you get this ``It is stood by me.'' That sounds weird nevertheless. In P\=ali, however, it is natural to do so, even it is less common in use. We call this structure \emph{impersonal passive}\footnote{See e.g.\ \citealp[p.~146]{collins:grammar}; \citealp[p.~42]{warder:intro}.}, because it shows only the state of being, not showing that someone is doing something. Technically, we call this \pali{bh\=avav\=acaka}. When we say ``I stand'' actively, we use this:

\palisample{aha\d m ti\d t\d th\=ami.}

And when it is converted to passive form, we get this:

\palisample{may\=a \d th\=iyate. \sampleor may\=a \d th\=iyati.}

Because there is no subject for the verb to agree with, we use 3rd person singular in this structure. This sentence is a little difficult to translate literally into English. My method is we change the verb to its verbal noun form (-ing) and compose it into a passive structure. Hence, we get ``Standing is done by me.''\footnote{It is far better than ``It is stood by me.''} That is the closest way, because `standing' expresses a state of being exactly what we call \pali{bh\=ava} in P\=ali.\footnote{In Sadd-Pad Ch.\,1, Aggava\d msa explains that \pali{\d th\=iyate} means the same as \pali{\d th\=ana\d m} (\pali{Yath\=a ca \d th\=ana\d m \d thiti \ldots}).} In practice, however, you can translate it simply as ``I (by myself) stand,'' but this does not reflect the original structure of the language.

\phantomsection
\addcontentsline{toc}{section}{Using \pali{Kita} in Passive Voice}
\section*{Using \pali{Kita} in Passive Voice}

If only present tense is what you say, things will go without any problem. In real life you have to say many things in various tenses and moods. In principle is quite simple when you construct a passive sentence: just add \pali{ya} before verbal \pali{vibhatti} is applied. In practice, however, it is not that easy or preferable to do with other tenses and moods. So, passive verb forms in other verb classes than present tense are rarely found. Here are some examples from my searching:

\begin{quote}
\pali{amh\=aka\d m \=av\=ase uposatho kar\=iyatu}\footnote{Mv\,2.142}\\
``The Vinaya recital must be done in our temple.''\\[1.5mm]
\pali{kattha v\=a ajjuposatho kar\=iyissati}\footnote{Mv\,2.141}\\
``Where will the Vinaya recital be held today?''\\[1.5mm]
\pali{yo by\=ap\=ado so pah\=iyissati}\footnote{M2\,120 (MN\,62)}\\
``Which malevolence [exists], that will be destroyed.''\\[1.5mm]
\pali{\=Ak\=ase pupphachadana\d m, dh\=arayissati sabbad\=a.}\footnote{Apad\=a 1.633}\\
``A roof of flower will be held all the time in the air.''\\[1.5mm]
\pali{Tasmi\d m kho, br\=ahma\d na, ya\~n\~ne neva g\=avo ha\~n\~ni\d msu}\footnote{D1\,345 (DN\,5)}\\
``In that sacrifice, brahman, oxen were not killed.''\\
\end{quote}

In imperative mood and future tense, we can get the job done without a great difficulty, because these verb forms use the model of present tense. I have no idea what passive optative will look like. In past tense, as shown in the last one, the verb used also mimics the present model. I am not sure what to do with other verbs if I use them in past tense.\footnote{Sometimes the line between active and passive verbs in past tense is unclear. See \citealp[pp.~155-6]{warder:intro}.} To soothe this difficulty, verbal \pali{kita} comes into play. In the meaning of requests, invitation, permission, or advices, verbs in \pali{tabba} and \pali{an\=iya} form can be used.\footnote{Kacc\,635, R\=upa\,559, Sadd\,1244. Scholars call this \emph{future passive participle} (\citealp[p.~104]{warder:intro}; \citealp[p.~110]{collins:grammar}).}. That can be a good alternative to imp.\ and opt.\ mood. There are other some \pali{paccaya}s can do this job as well. For more information, see page \pageref{par:passpaccaya}. Here are examples given by textbooks:

\begin{quote}
\pali{sayitabba\d m tay\=a.}\\
``Sleeping should be done by you.''\\[1.5mm]
\pali{kattabba\d m kamma\d m tay\=a.}\\
``Work should be done by you.''\\[1.5mm]
\pali{kara\d n\=iya\d m kicca\d m tay\=a.}\\
``Duty should be done by you.''\\[1.5mm]
\pali{bhottabba\d m/bhojan\=iya\d m bhojana\d m tay\=a.}\\
``Food may be eaten by you.''\\[1.5mm]
\pali{bhottabbo odano tay\=a.}\\
``Boiled rice may be eaten by you.''\\[1.5mm]
\pali{bhottabbo odano amhehi.}\\
``Boiled rice may be eaten by us.''\\
= ``Let's eat boiled rice.''\\[1.5mm]
\pali{ajjhayitabba\d m/ajjhayan\=iya\d m ajjheyya\d m tay\=a.}\footnote{The terms are from \pali{adhi + i} (to go over = to learn by heart). It is rare to be found in main verb form. It is often found as \pali{ajjhayana} [\pali{adhi + i + yu}] (learning).}\\
``A thing to study should be learned by you.''\\[1.5mm]
\pali{upasamp\=adetabba\d m tay\=a.}\\
``Ordination should be given by you.''\\
= ``May you ordain me, please.''\\[1.5mm]
\end{quote}

In addition, \pali{tabba} and \pali{an\=iya} can also imply inevitability or obligation\footnote{Kacc\,636, R\=upa\,659, Sadd\,1245. See also \pali{\d n\=i} on page \pageref{pacck1:dnii}.}, for example:
\begin{quote}
\pali{kattabba\d m me tay\=a geha\d m.}\\
``A house has to be built by you for me.''\\[1.5mm]
\pali{d\=atabba\d m me tay\=a sata\d m i\d na\d m.}\\
``Debt of 100 has to be paid to me by you.''\\[1.5mm]
\pali{dh\=aritabba\d m me tay\=a sahassa\d m i\d na\d m.}\\
``Debt of 1,000 is obligatorily held by you for me.''\\[1.5mm]
\end{quote}

As you may see, these verbal \pali{kita}s do not really behave like verbs. They look more like adjective because their ending agrees with the subject in the same way as adjectives do. In fact, product of \pali{tabba} and \pali{an\=iya} can be used as a noun or adjective, for example, \pali{p\=an\=iya\d m} (thing should be drunk = water), \pali{kara\d n\=iya\d m/kattabba\d m} (thing should be done = duty). When you see these in a sentence with a normal verb, it is likely to be a noun or adjective. Even the verb is absent, like we normally leave out \pali{hoti} or \pali{bhavati}, they can still be seen as such (see below). Some teachers say these can work like a kind of verb. This is reasonable too, because they also has modal meaning apart from their lexical meaning. That is to say, sentences composed with these \pali{kita}s are complete by themselves. They can stand alone without any \pali{\=akhy\=ata} (verb).\footnote{Warder notices that \pali{tabba} is more used as sentence verb, whereas \pali{an\=iya} is more as adjective \citep[p.~104]{warder:intro}.}

Verbs in \pali{tabba} form can be found accompanied with \pali{ma\~n\~nati} (to deem, to think). See these examples for the idea:\footnote{Thanks to Perniola \citep[p.~371]{perniola:grammar} for pointing these out.}

\begin{quote}
\pali{Appeva n\=ama appasadda\d m parisa\d m viditv\=a \\upasa\.nkamitabba\d m ma\~n\~neyya}\footnote{D1\,409 (DN\,9)}\\
``Having seen the silent assembly, [he] might think [it is worth] coming [here].''\\[1.5mm]
\pali{tath\=agate arahante samm\=asambuddhe \=as\=adetabba\d m ma\~n\~nasi}\footnote{D3\,28 (DN\,24)}\\
``[You] think [that] insulting the Buddha, the Fully Enlightened One, might be done.''\\[1.5mm]
\pali{upagat\=ana\d m pi\d n\d daka\d m d\=atabba\d m ma\~n\~neyy\=asi}\footnote{M2\,68 (MN\,56)}\\
``[You] should think [that] giving food to whom coming should be done.''\\[1.5mm]
\end{quote}

Let us do some example for more understanding. To say ``This book should be read by you,'' you can go like this:

\palisample{aya\d m potthako pa\d thatabbo/pa\d than\=iyo tay\=a. \sampleor[or, used as nt.] ida\d m potthaka\d m pa\d thatabba\d m/pa\d than\=iya\d m tay\=a.}

If you add \pali{hoti} to this sentence, hence it becomes ``\pali{ida\d m potthaka\d m pa\d thatabba\d m/pa\d than\=iya\d m tay\=a hoti}.'' It is logical to translate the sentence as ``This book is advisable to read by you.'' Even if \pali{hoti} is left out, it can be read as such. Now let us see some examples from the scriptures:

\begin{quote}
\pali{Eva\~nca pana, bhikkhave, pav\=aretabba\d m. \\Byattena bhikkhun\=a pa\d tibalena sa\.ngho \~n\=apetabbo}\footnote{Mv\,4.209}\\
``As such, monks, the Invitation should be done. The Sangha should be made know by a learned monk \ldots''\\[1.5mm]
\pali{Nanu n\=ama sannipatitehi dhammo bh\=asitabbo}\footnote{Mv\,2.132}\\
``The teaching should be preached by those assembled, shouldn't it?''\\[1.5mm]
\pali{Asantiy\=a \=apattiy\=a tu\d nh\=i bhavitabba\d m.}\footnote{Mv\,2.134}\\
``Being in silence should be done by [one] who has no offense.''\\[1.5mm]
\pali{Parima\d n\d dala\d m niv\=asess\=am\=i'ti sikkh\=a kara\d n\=iy\=a.}\footnote{Buv2\,576}\\
``\,`I will dress myself properly,' thus a discipline should be done.''\\[1.5mm]
\pali{Gaman\=iyo sampar\=ayo, mant\=aya\d m boddhabba\d m, \\kattabba\d m kusala\d m, caritabba\d m brahmacariya\d m, \\natthi j\=atassa amara\d na\d m.}\footnote{D2\,323 (DN\,19)}\\
``The next world is to be gone; knowing should be done by wisdom; wholesomeness should be done; \\religious life should be practiced; there is no deathlessness of the already-born.''
\end{quote}

Apart from these, you can find many more, because these verb forms are quite easy to use and expressive. For those who have not yet caught on how to use these by examples illustrated. I conclude this with a simple guideline as follows:

\begin{enumerate}
\item Determine the verb to use whether it is transitive or intransitive.
\item If a transitive verb is used, apply the patient of the verb with nom.\ corresponding to its gender and number. This is the subject. For intransitive verbs, there is none.
\item Apply \pali{tabba} or \pali{an\=iya}, or others with the same function if you like, to the verb. There are not many irregular forms of these to remember, fortunately (see page \pageref{sec:irrprod}). Then apply it with an ending agreeable to the subject of the previous item in the same manner as you do with a regular adjective. If there is no subject because an intransitive verb is used, make it nt.\ sg., hence \pali{-tabba\d m} or \pali{-an\=iya\d m}.
\item If the agent of the action is present, apply it with ins.
\item Put all these together in a proper order.
\end{enumerate}

How about passive past tense, then? This is good news. Using aorist or other past verbs is headachy enough by itself. Putting past verbs into passive form can be a challenging task, even to P\=ali experts. In this situation, we can use verbs in \pali{ta} form.\footnote{Kacc\,625, R\=upa\,605, Sadd\,1232} This verbal \pali{kita} is more versatile than \pali{tabba} and \pali{an\=iya} because it can be used in both active and passive structure. You have learned to use active \pali{ta} in Chapter \ref{chap:pp}. Now we will focus only on passive side of it. Let us see examples given by textbooks first:

\begin{quote}
\pali{sayita\d m tay\=a.}\\
``Sleeping was done by you.''\\
\pali{sayita\d m sayana\d m tay\=a.}\\
``A bed has been slept by you.''\\
\pali{pacito odano tay\=a.}\\
``Rice has been cooked by you.''\\
\end{quote}

In P\=ali there is no (longer a) distinction between past and perfect tense, so you have to decide what is suitable to the context. Like \pali{tabba} and \pali{an\=iya} mentioned above, we can interpret \pali{ta} as a noun, adjective, or verb, and the same guideline can be applied here. You can suppose there is \pali{hoti} in ``sayita\d m tay\=a'' and read it as ``There was a sleep done by you.'' And you can read ``pacito odano tay\=a'' as ``There is boiled rice done by you.'' You know now why verbs in \pali{ta} form are called \emph{past participle}. To find an instance with \pali{ta} in the texts is extremely easy, because it is used extensively. Here are what I take from the very first part of the canon:

\begin{quote}
\pali{Ta\d m kho pana bhavanta\d m gotama\d m eva\d m kaly\=a\d no kittisaddo \textbf{abbhuggato}}\footnote{Buv1\,1}\\
``A charming reputation has been spread that Venerable Gotama \ldots''\\[1.5mm]
\pali{Ye te, br\=ahma\d na, r\=uparas\=a saddaras\=a gandharas\=a rasaras\=a pho\d t\d thabbaras\=a te tath\=agatassa \textbf{pah\=in\=a}}\footnote{Buv1\,3}\\
``Brahman, which tastes in sight, tastes in sound, tastes in smell, tastes in flavor, tastes in contact, those are destroyed by the Tath\=agata''\footnote{In this instance, \pali{tath\=agatassa} is used as instrumental.}\\[1.5mm]
\end{quote}

We can also find \pali{ta} forms frequently in compounds, for example just after the last example above, \pali{\textbf{ucchinna}m\=ul\=a} (having root destroyed). Another ubiquitous phrase with \pali{ta} found throughout the texts is ``\pali{eva\d m me suta\d m}'' (Thus it was heard by me; Hearing was done my me in this way). As you now realize, verbs in \pali{ta} are very important. Without knowing this, you barely understand what is said in the texts. The only difficulty is when \pali{ta} is applied, a variety of outcome can be produced. You have to master it first (see page \pageref{sec:irrprod} onwards; and in our vocabulary verbs in \pali{ta} form are also listed, see page \pageref{vocab:verb}).

Another \pali{paccaya} that has a passive sense is \pali{kha}.\footnote{Kacc\,625, R\=upa\,605, Sadd\,1232. See page \pageref{pacck4:kha} for more information.} This can be used like the aforementioned. Its forms look more like adjectives or nouns than others. Here are some given examples:\footnote{I have not found any use of these in the canon.}

\begin{quote}
\pali{ki\~ncissayo tay\=a.}\\
``Some sleep is done by you.''\\
\pali{\=isassayo tay\=a.}\\
``Little sleep is done by you.''\\
\pali{dussayo tay\=a.}\\
``Difficult sleep is done by you.''\\
\pali{sussayo tay\=a.}\\
``Easy sleep is done by you.''\\
\end{quote}

Yet another \pali{paccaya} able to use in passive structure is \pali{m\=ana}, but in a limited way. As we have seen in Chapter \ref{chap:prp}, together with \pali{anta}, \pali{m\=ana} can help us create subordinate clauses or adjective phrases like present participles in English. The only function that \pali{m\=ana} outdoes \pali{anta} is it can be used in passive structure as well. You can use \pali{m\=ana} only in relative clauses or as a modifier. Here are some examples from the canon:

\begin{quote}
\pali{kariyam\=ane aru\d na\d m u\d t\d thahati}\footnote{Pvr\,412}\\
``While [the robe] is being made, the dawn breaks.''\\[1.5mm]
\pali{ki\d m me kar\=iyam\=ana\d m d\=igharatta\d m ahit\=aya dukkh\=aya assa}\footnote{D3\,84 (DN\,26)}\\
``What is being done by me is for non-benefit, for suffering for a long time?''\\[1.5mm]
\pali{T\=ani ce sutte os\=ariyam\=an\=ani vinaye sandassiyam\=an\=ani na ceva sutte osaranti}\footnote{D2\,188 (DN\,16)}\\
``Being compared with the Sutta, being checked with the Vinaya, if those [teachings] do not comply \ldots''\\[1.5mm]
\end{quote}

As you have seen, to make \pali{m\=ana} verbs passive \pali{ya} has to be applied, unlike \pali{tabba}, \pali{an\=iya}, and \pali{ta} which are inherently passive. To see a clearer picture, let us say this sentence: ``There is a book being read by me.''

\palisample{potthako may\=a paṭhayam\=ano hoti.}

Then you see that if we leave out \pali{hoti} as we often do, it looks like \pali{paṭhayam\=ano} ends the sentence. You can see it in this way, even if P\=ali teachers generally say \pali{m\=ana} cannot make a sentence. So, it is better to treat it like an adjective in this structure.

And let us try this with a relative clause using absolute construction: ``When this book is being read, I am studying it at school.''

\palisample{[yassa] imassa potthakassa pa\d thayam\=anassa, aha\d m p\=a\d thas\=al\=aya ta\d m sikkh\=ami.}

Another verb form that can be used in passive structure is \pali{tv\=a}. This marks the succession of events (see Chapter \ref{chap:pp}). Here is an example from the canon:

\begin{quote}
\pali{Sa\.nghassa kh\=adan\=iye bh\=aj\=iyam\=ane sabbesa\d m pa\d tivis\=a \textbf{\=aharitv\=a} upanikkhitt\=a honti.}\footnote{Buv1\,147}\\
``When sweetmeats were being distributed to monks, the portions of all, having been brought, were kept [by each monk].''\\
\end{quote}

\section*{Exercise \ref{chap:pass}}
Say these in P\=ali.
\begin{compactenum}
\item In a previous life of the Buddha as Vessantara, his son and daughter are given to other and punished. Is that unethical to do so?
\item The vision of the Buddha cannot be known by us. It is explained that enlightenment is more important than one's belongings, including children and wife.
\item By that time, he was not enlightened yet. How did he know that? It might be a kind of superstitious belief. If everything known by him was true at that time, the later life would not be needed.
\item By the religion's point of view, thinking in that way is not permissible. Otherwise, the foundation of the religion would be undermined.
\item If that happens today, it will be immoral because children and wife do note belong to a man. They cannot be given away just for the man's benefit.
\item The present days and the former days have different norms. It might not be seen as wrong at that time.
\item Is natural moral principle timeless or not? Or is there an exception for a particular person?
\item The decision of the Buddha should not be judged.
\item You are arguing in circle.
\item You must believe in order to understand.\footnote{This sentence comes from Paul Ricoeur in \emph{The Conflict of Interpretations} (Northwestern University Press, 1974, \url{https://books.google.com/books?id=0QuXVWzxoLIC}). His idea goes like this: ``to understand the text, it is necessary to believe in what the text announces to me; but what the text announces to me is given nowhere but in the text. This is why it is necessary to understand the text in order to believe.'' (p.~390). You can see a circle here. Technically, we call this \emph{hermeneutic circle}. You have to start somewhere, pre-understanding or pre-belief, and let the circle runs to gain better understanding and belief. That is hermeneutics in a nutshell.}
\item I think Buddhism is a reasonable religion.
\end{compactenum}
