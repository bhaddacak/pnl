\chapter{I go \headhl{where} you go}\label{chap:yata}

In this chapter a new pronoun will be introduced, an important one. We have talked about demonstrative pronouns in Chapter \ref{chap:pron-demon}, personal pronouns in Chapter \ref{chap:pron-person}, and interrogative pronoun in Chapter \ref{chap:vockim}. The next one to be addressed here is also used frequently, and often paired with \pali{ta} (that). In English, we call this relative pronoun. P\=ali has only one term of that kind---\pali{ya} (which). From now on I will not show the table of terms' declension, because we already have full list of them in Appendix \ref{chap:decl}. For all pronouns, see \ref{decl:pron}, page \pageref{decl:pron} onwards. For \pali{ya} see page \pageref{decl:ya}. If you can decline \pali{ki\d m}, you can do it with \pali{ya} in a similar manner, maybe a bit easier.

\phantomsection
\addcontentsline{toc}{section}{Correlative Sentences}
\section*{Correlative Sentences}

In Chapter \ref{chap:vockim} we learned to use \pali{ki\d m} to make questions. If you understand that, \pali{ya} will be easy. Like \pali{ki\d m}, \pali{ya} also represents question words, but in relative sense not interrogative sense. This word help us compose complex sentences like ``Those who go to school are students.'' In P\=ali you cannot put that straight. You have to change the sentence to ``Who go to school, they are students.'' The `who' in the sentence is relative pronoun, i.e.\ \pali{ya}, which relate to `those,' i.e.\ \pali{ta}. That is why we often see \pali{ya} comes together with \pali{ta}.\footnote{In some cases, however, \pali{ya} can pair with other words, such as \pali{eva\d m}.} Here is its P\=ali equivalent.

\palisample{ye p\=a\d thas\=ala\d m gacchanti, te siss\=a honti.}

You might protest that teachers go to school as well. Then I change the English sentence to ``Children who go to school are students.'' When you transform this sentence, if you never have learned this kind of language before, you may get an awkward moment. It should come out as ``Which children go to school, they are students.'' In P\=ali, it fits the meaning perfectly:

\palisample{ye d\=arak\=a p\=a\d thas\=ala\d m gacchanti, te siss\=a honti.}

When \pali{ya} come with a noun, it functions as a pronominal adjective, unlike `who' in English to which that function is not allowed. If you want to go smoothly, you have to think in P\=ali. I mean in P\=ali's terms not in P\=ali language. That is to say, you have to think in terms of cases and try to match \pali{ya} with \pali{ta}. Let us tackle the sentence posted as the title of this chapter: ``I go where you go.'' You have to restructure it to ``Where you go, I go there.'' Then you have a \pali{ya-ta} pair, \emph{where-there} in this case. After that, you think which case will be appropriate to this context. Accusative case is obvious here. Therefore we get the sentence in P\=ali:

\palisample{(tva\d m) ya\d m gacchasi, (aha\d m) ta\d m gacch\=ami.}

Is that simple? Do not mix up \pali{ya} and \pali{ta} clauses. Question words go with \pali{ya}, whereas demonstrative or personal pronouns go with \pali{ta}. In P\=ali sentences, you put the \pali{ya} clause first. In English, relative pronouns are often left out. So, you have to really understand what you will say first. Here is another sentence: ``The one (who) I give a book to is my friend.'' It should be transformed to ``To whom I give a book, he/she is my friend.'' Which case? Dative. That's right. So, we get this:

\palisample{(aha\d m) yassa potthaka\d m demi, so/s\=a mama mitto hoti.}

As you have seen, \pali{ya} and \pali{ta} do not need to take the same case. It depends on the context. In the following sentence \pali{ya} and \pali{ta} take the same case: ``I give a pen to the one (who) I give a book.'' This yields ``To whom I give a book, I give a pen to him/her.''

\palisample{(aha\d m) yassa potthaka\d m demi, tassa/t\=aya lekhani\d m demi.}

Do you remember that I have left one riddle to you in the chapter concerning genitive case (Chapter \ref{chap:gen})? It is how to say ``You have my book.'' If you use the method learned in that chapter, you go nowhere. You just get a gibberish ``Your my book exists.'' The logic of this is that you cannot really have my book for it does not belong to you. A provisional solution is to use another verb to express the idea. For example, you can say ``You hold my book'' as ``\pali{tva\d m mama potthaka\d m dh\=aresi}'' or ``\pali{tva\d m mama potthaka\d m ga\d nh\=asi.}'' But this is not the right way to do in P\=ali. We normally use \pali{ya-ta} structure in such a case.

First, we transform the sentence to ``Which book you have, it is mine.''  Then we change it to gen.\ sentence: ``Your which book exists, it is mine.'' So, we get the final solution as follows:

\palisample{tuyha\d m ya\d m potthaka\d m atthi, ta\d m mayha\d m (potthaka\d m hoti).}

Let us try another case. Figure out how to say this: ``The pen which whose book is lost is lost (too).'' Now you change this ugly sentence to ``Whose book is lost, his/her pen is lost (too).'' This sentence clearly uses gen. For the verb, we normally use \pali{nassati} or \pali{vinassati} (perish) in this sense. Hence, we get this:

\palisample{yassa potthaka\d m (vi)nassati, tassa/t\=aya lekhan\=i (ca) vinassati.}

Comparing this P\=ali sentence to the English one, you will realize that how beautifully the \pali{ya-ta} structure transforms our (ugly) complex sentence. Do not worry about particle \pali{ca} now. We will learn this later in Chapter \ref{chap:ind-intro}.

Let us try this tricky one: ``You say like I do.'' This sentence can be said in several ways. To use \pali{ya-ta}, we transform it to ``How I say, you say (by) that.'' Which case? Instrumental. Well done. And here how it comes out:

\palisample{(aha\d m) yena bh\=as\=ami, (tva\d m) tena bh\=asasi.}

How about this: ``(The reason that) why we eat is (the same as) why we sleep.'' We transform this to ``From what reason we eat, we sleep from that reason.'' Then, we put it tersely as:

\palisample{(maya\d m) yasm\=a bhu\~nj\=ama, tasm\=a say\=ama.}

That is ablative case. However, causes of action can be other cases as well, e.g.\ ins.\ and loc. You can use whatever you feel right.

Here is the last one: ``I go when you come.'' We reform this to ``When you come, in that (time) I go.'' So, we get this:

\palisample{(tva\d m) yasmi\d m \=agacchasi, (aha\d m) tasmi\d m gacch\=ami.}

Practically, to make this unambiguous a pair of particles (\pali{yad\=a-tad\=a}) is often used instead of loc. So, normally we use ``\pali{yad\=a \=agacchasi, tad\=a gacch\=ami}.'' We will talk about these particles in Chapter \ref{chap:ind-to}.

If you feel you barely grasp the lesson, it means you need to review all fundamentals we have learned so far again (and again, if necessary). And please do that before you proceed. The understanding of this chapter is really important.

How about this exercise?

\section*{Exercise \ref{chap:yata}}
Say these in P\=ali using \pali{ya-ta} structure.
\begin{compactenum}
\item The book I read is yours.
\item I live where my parents live.
\item Students repeat (words) after the teacher.
\item She and you come from the same country.
\item I go to town by the car you give me.
\item A thief steals a car of one who has a big house.
\end{compactenum}
