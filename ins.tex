\chapter{I go to school \headhl{by bus}}\label{chap:ins}

Now we will add another case of declension, an important one, which is used extensively in a variety of contexts. We are going to talk about \emph{instrumental} case.\footnote{By its modern name, this case is not used by Greek and Latin \citep[p.~61, 68]{fairbairn:understanding}. However, its function can be achieved by using other cases instead \citep[p.~67]{fairbairn:understanding}.} 

\phantomsection
\addcontentsline{toc}{section}{Declension of Instrumental Case}
\section*{Declension of Instrumental Case}

The main function of this case is to mark the \emph{means} or \emph{instrument}, as its name implies, of the action. In English, this function is simply performed by prepositions `by' and `through' and `via' and, to some extent, `with.' Table \ref{tab:insreg} summarizes the rule of the declension for regular nouns.

\begin{table}[!hbt]
\centering
\caption{Instrumental case endings of regular nouns}
\label{tab:insreg}
\bigskip
\begin{tabular}{@{}>{\bfseries}l*{5}{>{\itshape}l}@{}} \toprule
\multirow{2}{*}{G. Num.} & \multicolumn{5}{c}{\bfseries Endings} \\
\cmidrule(l){2-6}
& a & i & \=i & u & \=u\\
\midrule
m. sg. & \texthl{\replacewith{a}{ena}} & \texthl{in\=a} & \texthl{\replacewith{\=i}{in\=a}} & \texthl{un\=a} & \texthl{\replacewith{\=u}{un\=a}} \\
m. pl. & \replacewith{a}{ehi} & \replacewith{i}{\=ihi} & \=ihi & \replacewith{u}{\=uhi} & \=uhi \\
& \replacewith{a}{ebhi} & \replacewith{i}{\=ibhi} & \=ibhi & \replacewith{u}{\=ubhi} & \=ubhi \\
\midrule
nt. sg. & \texthl{\replacewith{a}{ena}} & \texthl{in\=a} & & \texthl{un\=a} & \\
nt. pl. & \replacewith{a}{ehi} & \replacewith{i}{\=ihi} & & \replacewith{u}{\=uhi} & \\
& \replacewith{a}{ebhi} & \replacewith{i}{\=ibhi} & & \replacewith{u}{\=ubhi} & \\
\midrule
& \=a & i & \=i & u & \=u\\
\midrule
f. sg. & \=aya & iy\=a & \replacewith{\=i}{iy\=a} & uy\=a & \replacewith{\=u}{uy\=a} \\
f. pl. & \=ahi & \replacewith{i}{\=ihi} & \=ihi & \replacewith{u}{\=uhi} & \=uhi \\
& \=abhi & \replacewith{i}{\=ibhi} & \=ibhi & \replacewith{u}{\=ubhi} & \=ubhi \\
\bottomrule
\end{tabular}
\end{table}

For m.\ and nt.\ nouns, instrumental case in P\=ali is easy to recognize, particularly in singular forms. This case shares plural forms with ablatives, so it can be confusing to new students. For f.\ nouns, instrumentals and ablatives share totally the same forms. For translating texts, this can puzzle us to tell the cases apart. But for composing, it makes things easier, because we do not need to remember a lot of forms. For pronouns, Table \ref{tab:inspron} shows the declension of this case.

\begin{table}[!hbt]
\centering
\caption{Instrumental case of pronouns}
\label{tab:inspron}
\bigskip
\begin{tabular}{@{}*{5}{>{\itshape}l}@{}} \toprule
\multirow{2}{*}{\bfseries\upshape Pron.} & \multicolumn{2}{c}{\bfseries\upshape m./nt.} & \multicolumn{2}{c}{\bfseries\upshape f.} \\
\cmidrule(lr){2-3} \cmidrule(lr){4-5}
& \bfseries\upshape sg. & \bfseries\upshape pl. & \bfseries\upshape sg. & \bfseries\upshape pl. \\
\midrule
amha & may\=a & amhehi & & \\
& me & no & & \\
tumha & tay\=a & tumhehi & & \\
& te & vo & & \\
ta & tena & tehi & t\=aya & t\=ahi\\
& & tebhi & & t\=abhi \\
eta & etena & etehi & et\=aya & et\=ahi \\
& & etebhi & & et\=abhi \\
ima & imin\=a & imehi & im\=aya & im\=ahi \\
& anena & imebhi & & im\=abhi \\
amu & amun\=a & am\=uhi & amuy\=a & am\=uhi \\
& & am\=ubhi & & am\=ubhi \\
\bottomrule
\end{tabular}
\end{table}

Now we can say ``I go to school by bus'' as follows:

\palisample{aha\d m mah\=arathena p\=a\d thas\=ala\d m gacch\=ami.}

Again, be aware of case agreement of modifiers. If the sentence is modified to ``I go to school by a big bus,'' its P\=ali now is:

\palisample{aha\d m mahantena mah\=arathena p\=a\d thas\=ala\d m gacch\=ami.}

Instrumental case is often used with certain particles as I summarize in Table \ref{tab:indins}.

\begin{table}[!hbt]
\centering
\caption{Particles often used with ins.}
\label{tab:indins}
\bigskip
\begin{tabular}{@{}>{\itshape}ll@{}} \toprule
\bfseries\upshape Particle & \bfseries Description \\ \midrule
saddhi\d m & accompanied by/with, together with \\
saha & accompanied by/with, together with \\
vin\=a & without, by the absence of \\
\bottomrule
\end{tabular}
\end{table}

So, we can say ``I go to school by bus with you'' as:

\palisample{aha\d m tay\=a saddhi\d m mah\=arathena p\=a\d thas\=ala\d m gacch\=ami.}

Alternatively, \pali{saha} can replace \pali{saddhi\d m} in the sentence. In negative sense, we use \pali{vin\=a}. For example, if I say ``\pali{aha\d m tay\=a vin\=a mah\=arathena p\=a\d thas\=ala\d m gacch\=ami},'' I mean I go to school without you. For more particles that are used with instrumental case, see Appendix \ref{chap:nipata}, page \pageref{nip:saha} onwards.

\pali{Saddhi\d m} and \pali{saha} can also be used with verb `to be' to mean that someone is of the same type or have the same quality of the other. For example, ``\pali{aya\d m ka\~n\~n\=a mittehi saddhi\d m sur\=up\=a hoti}'' means ``This girl together with friends is beautiful'' or ``This girl, as well as (her) friends, is beautiful.''

If you ponder more about the sense of instrumental case, you can find that it can also express the cause of the action. For example, to answer the question ``How do you come here?'' you normally think in terms of the method that you use to move there. But you can also think that the question is asked for the cause or the reason of your coming---you can read `why' from `how,' so to speak. Therefore, instrumental case can be used to identify the cause of the action as well. For example, ``He becomes a thief because he is poor'' can be put tersely as:

\palisample{so da\d liddena coro hoti.}

Our exercise in this chapter asks for new verbs that I have not mentioned before. You can find the verbs unknown to you in Appendix \ref{vocab:verb}, page \pageref{vocab:verb}. Only their dictionary form is used for now. Remember that sometimes P\=ali terms do not exactly mean as their English counterparts do, and sometimes P\=ali has an idiomatic way to say things. In a real situation, if some verbs do not come to your mind, you can compose a new one from its manner. For example, you can say ``\pali{kamma\d m karomi}'' (I do a work) to mean ``I work,'' or ``\pali{p\=adena gacch\=ami}'' (I go by foot) to mean ``I walk.''

The tradition really uses this kind of verb formation. Some idioms with \pali{karoti} you can find in the texts are, for example, ``\pali{n\=ama\d m karoti}'' (to give a name), ``\pali{garukaroti}'' (to respect), ``\pali{manasi karoti}'' (to keep in mind), ``\pali{vin\=akaroti}'' (to separate), and ``\pali{k\=ala\d m karoti}'' (to make time = to die).

I would like to remind you more that when you are learning to speak P\=ali, there is no grammar policeman to give you a ticket if you say something wrong grammatically. You can speak in any way as long as it is understandable in an acceptable way. P\=ali conversation is a reconstruction of the past. There is no `good' P\=ali in this regard, only intelligible P\=ali. Learning to translate texts is a different story. We have to listen to authority otherwise we hardly make sense out of cryptic scriptures. Once you understand the language well enough, you can argue with authority.

\section*{Exercise \ref{chap:ins}}
Say these in P\=ali.
\begin{compactenum}
\item I hear with ears, see with eyes, eat with mouth.
\item I live without you because of poorness.
\item By train, those women go from their village to the city.
\item I buy many things from that merchant with my money.
\item They (m.) see this beautiful image with their eyes.
\item I, together with friends, go to a theater by my small car.
\item You (f.), a smart teacher, carry a big tree with hands together with many boys, your students.
\end{compactenum}
