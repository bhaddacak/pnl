\chapter{\headhl{This} (is) a book}\label{chap:pron-demon}

\phantomsection
\addcontentsline{toc}{section}{Demonstrative Pronouns}
\section*{Demonstrative Pronouns}

In this chapter we will learn how to locate an object with indicators like `this' or `that.' These are called \emph{demonstrative pronouns}, which are ``used to point to entities, locating them as near to or remote from the speaker.''\footnote{\citealp[p.~126]{brownmiller:dict}}

Like adjectives, pronouns (\pali{sabban\=ama}---name of everything) in P\=ali is a kind of noun. Pronouns stand for nouns or noun phrases. In western terms, pronouns can be divided to \emph{personal}, \emph{demonstrative}, \emph{relative}, \emph{interrogative}, and \emph{indefinite} pronouns. Here we focus only on demonstrative ones, and we will come to the rest later. In P\=ali, similar to adjectives, pronouns take gender and number from the noun they represent. Our task is to remember forms of declension, only nominative for now, as shown in Table \ref{tab:nomdemon} (for full paradigms see Appendix \ref{decl:pron}). Pay more attention on the words highlighted.

\begin{table}[!hbt]
\centering
\caption{Nominative case of demonstrative pronouns}
\label{tab:nomdemon}
\bigskip
\begin{tabular}{@{}l*{6}{>{\itshape}l}@{}} \toprule
\multirow{2}{*}{\bfseries\upshape pron.} & \multicolumn{2}{c}{\bfseries\upshape m.} & \multicolumn{2}{c}{\bfseries\upshape f.} & \multicolumn{2}{c}{\bfseries\upshape nt.} \\
\cmidrule(lr){2-3} \cmidrule(lr){4-5} \cmidrule(lr){6-7} 
& \bfseries\upshape sg. & \bfseries\upshape pl. & \bfseries\upshape sg. & \bfseries\upshape pl. & \bfseries\upshape sg. & \bfseries\upshape pl. \\
\midrule
\pali{ta} (that) & \texthl{so} & te & \texthl{s\=a} & t\=a & ta\d m & t\=ani \\
\pali{eta} (this/that) & \texthl{eso} & ete & \texthl{es\=a} & et\=a & eta\d m & et\=ani \\
\pali{ima} (this) & \texthl{aya\d m} & ime & \texthl{aya\d m} & im\=a & \texthl{ida\d m} & im\=ani \\
\pali{amu} (yonder) & \texthl{asu} & am\=u & \texthl{asu} & am\=u & \texthl{adu\d m} & am\=uni \\
\bottomrule
\end{tabular}
\end{table}

Distinguishing \pali{ta} and \pali{eta} might be difficult at first. By traditional explanation, \pali{ta}\footnote{This is also used as third personal pronoun, e.g.\ he, she, it, and they (see Chapter \ref{chap:pron-person}).} refers to things absent at the moment (\pali{parammukh\=a}), \pali{eta} refers to things nearby (\pali{sam\=ipa}), \pali{ima} refers to things very close (\pali{accantasam\=ipa}), and \pali{amu} refers to things far away (\pali{d\=ura}). How close is \pali{eta} and how far is \pali{amu} are a relative matter. By intuition, \pali{ima} can be close at hand, \pali{eta} can be a little out of reach, \pali{amu} can be seen far away but not out of sight. You can use \pali{asuka} or \pali{amuka} instead of \pali{amu} (see declension of the term on page \pageref{decl:asuka}). The only difference between the two is that \pali{asuka} is adjective but \pali{amu} is pronoun. Both use different paradigms to decline, but when used they go in the same manner.

In conversation or direct speech, \pali{ta} can be used to refer to the thing (or person) mentioned earlier\footnote{Linguists call this \emph{anaphora}. \citealp[See also][p.~29]{warder:intro}.}, whereas \pali{eta} is used to point to the thing (or person) that is present at the moment.\footnote{Linguists call this \emph{deixis}.} When you and a friend are in a pet store, you point to a puppy and say ``That dog is chubby.'' And your friend say to you ``That/It is cute.'' The fist `that' is \pali{eta}, the second is \pali{ta}. In P\=ali they go like this: ``\pali{eso sunakho th\=ulo}'' and ``\pali{so sundaro}.'' In Chapter \ref{chap:yata} we will learn to pair \pali{ta} with \pali{ya} (which) to form correlative sentences.

Then we can say ``This (is) a book'' as follows:

\palisample{aya\d m potthako.\sampleor ida\d m  potthaka\d m.}

Here is for ``These (are) books.''

\palisample{ime potthak\=a.\sampleor im\=ani potthak\=ani.}

And these are for, ``This (is) a girl'' and ``These (are) girls'':

\palisample{aya\d m ka\~n\~n\=a. im\=a ka\~n\~n\=a(yo).}

All demonstrative pronouns mentioned here can be used as pronominal adjectives to modify a noun, for example, \pali{so puriso} (that man), \pali{s\=a ka\~n\~n\=a} (that girl), \pali{aya\d m bh\=as\=a} (this language), \pali{im\=ani kul\=ani} (these clans). They look alike in form but different in function. For the examples above, when the terms are used as a pronominal adjective, they form a noun phrase. But when they are used as a pronoun, they form a complete sentence with verb `to be' or `to exist' left out.

As an adjective, \pali{so puriso} means \emph{that man} not anyone else. As a pronoun, \pali{so puriso} means \emph{that being is a man} not any other being.

How to say ``This (is) a big book'' then? As you may guess, we can go bluntly as ``\pali{ida\d m th\=ula\d m potthaka\d m}'' (nt.). This sounds very much like a noun phrase (`this big book') if a verb is not explicitly specified. It is better to say ``\pali{ida\d m potthaka\d m th\=ula\d m}'' (``This book (is) big''). In this sentence `\pali{th\=ula\d m}' is the subject complement and \pali{ida\d m} can be seen as both an adjective modifying the subject or a pronoun standing for the subject. Word order here plays a clarifying role.

Another translation of ``\pali{ida\d m th\=ula\d m potthaka\d m}'' is ``\pali{ida\d m th\=ula\d m (vatthu) potthaka\d m (hoti)}'' (This fat thing is a book) which has slightly different meaning. Here is a lesson from this pondering. Although word order in P\=ali has no strict rule, there are typical uses of the order that help clarify sentences. Moreover, word order can reflect the style of P\=ali compositions.\footnote{For a comprehensive study of word order in early texts, see \citealp{bodhiprasiddhinand:order}.}

Before you leave this chapter, please beat the exercise first.

\section*{Exercise \ref{chap:pron-demon}}
Say these in P\=ali.
\begin{compactenum}
\item That (is) a fire.
\item Over there (is) a lightning.
\item Those (are) people.
\item This (is) a fat elephant. That/It is high.
\item This season is hot. That/It is summer.
\item Those geckoes (are) many. Those/They are ugly.
\item These quick beasts (are) horses.
\item Many fruits (are) over there.
\item This old man (is) wise.
\item Those young foreign girls (are) beautiful.
\end{compactenum}
