\chapter{Principle of Verb Formation}\label{chap:vform}

Now we come to the crux. For me, this is the hardest part of all when we learn about verbs. Without knowledge about this matter, you cannot fully understand cases as I explain in Chapter \ref{chap:cases}. And if you cannot understand cases in P\=ali, you cannot understand the language at all. So, drive through carefully here.

Why verb formation is crucial in P\=ali? To remind you, in case you have forgotten, P\=ali is a highly inflectional language. When you use a word, you cannot take it from a dictionary and put it into a sentence. You can do that with English to some extent, but not with P\=ali. As you may realize when you learn about nouns, adjectives, and pronouns, the heart of the learning is to know how terms are formed. It is true about verbs as well. To be clear, when I use `verb' here I mean only \pali{\=akhy\=ata} not verbal \pali{kita} (primary derivation). And you are supposed to read Chapter \ref{chap:vclass} before you come to this.

When we talk about verb formation, things involved are \pali{dh\=atu} (root), \pali{paccaya} (suffix/infix), and \pali{v\=acaka} (expressing stance). Like other Indo-European languages, root is the fundamental part of verbs. In Kacc/Sadd school a description goes ``suchlike \pali{bh\=u} and so on are \pali{dh\=atu}.''\footnote{Kacc\,457, R\=upa\,424, Sadd\,938} In Mogg it has a terse but more sensible formula, ``\pali{kriyatth\=a}''\footnote{Mogg\,5.14. In the explanation part it goes ``\pali{Kriy\=a attho yassa so kriyattho dh\=atu}.''} (Those [sound] having meaning of action). 

As I have discussed once in Chapter \ref{chap:ind-intro}, \pali{paccaya} is a process that transforms root into a meaningful term. As I say elsewhere, learning how \pali{paccaya}s work is central to the traditional approach to the language. There are numerous of \pali{paccaya}s. The majority of them are used in derivation. In verb formation they are less to be dealt with. In traditional account, \pali{paccaya} can be added to \pali{dh\=atu} and \pali{li\.nga}.\footnote{Kacc\,432, R\=upa\,362, Sadd\,905} This means that verbs can be created from a root, the normal process, e.g.\ \pali{kara + o + ti} = \pali{karoti} ([One] does). Or they can be created from nouns (\pali{li\.nga}), e.g.\ \pali{pabbata + \=aya + ti} = \pali{pabbat\=ayati} ([One] does like a mountain). They can also be created in an onomatopoetic way, e.g.\ \pali{cicci\d ta + \=aya + ti} = \pali{cicci\d t\=ayati} ([One] makes chit-chit sound).\footnote{In Sadd\,905, the base part is called \pali{anukara\d na}.} We will learn all of these in due course.

To clarify a bit more, we distinguish between \pali{paccaya} and \pali{vibhatti}. The former is more generic. It means any dependent part that marks the transforming process. We normally see it as suffixes, or infixes if you like. In the examples above, they are `\pali{o}' and `\pali{\=aya}' for instance. Whereas \pali{vibhatti}, exemplified by \pali{ti}, is the final part of the terms that marks tense, mood, person, number, and voice.\footnote{There is also nominal \pali{vibhatti} that marks case and number for nouns. In here we only focus on verbal \pali{vibhatti}.} If you still feel confused with \pali{vibhatti}, revisit Chapter \ref{chap:ind-intro} and \ref{chap:vclass} again. If this does not help much, you may need a grand review from the beginning.

Once a verb is formed by composition of a root, \pali{paccaya}, and \pali{vibhatti}, it is a meaningful entity that can be one of five expressing stances\footnote{Some may call these `voices,' but I reserve the word for `\pali{pada}' which can be `active' and `middle' voice. I do not take this so seriously and consistently though. Sometimes I use `voice' in English sense including what I call `stance' here.} (\pali{v\=acaka}) as follows:

\paragraph*{(1) Active stance (\pali{kattuv\=acaka})} \ \par
This verb form expresses that the subject of the sentence is the active actor of it, for example, ``\pali{s\=udo odana\d m \textbf{pacati}}'' (A chef cooks boiled rice). In this sentence \pali{pacati} (\pali{paca + a + ti}) is an active verb, verb that takes active expressing stance.

\paragraph*{(2) Causative stance (\pali{hetukattuv\=acaka})} \ \par
This verb form shows that the subject of the sentence is not the direct actor of the action, but an indirect actor who causes the real actor to do the action, for example, ``\pali{s\=amiko s\=uda\d m odana\d m \textbf{p\=aceti}}'' (A master has a chef cook boiled rice). In this sentence, \pali{p\=aceti} (\pali{paca + \d ne + ti}) is a causative verb.

\paragraph*{(3) Passive stance (\pali{kammav\=acaka})} \ \par
This verb form shows that the subject of the sentence is not the actor but the patient of the action, for example, ``\pali{odano s\=udena \textbf{paciyate}}'' (Boiled rice is cooked by a chef). In this sentence, \pali{paciyate} (\pali{paca + ya + i + te}) is a passive verb.

\paragraph*{(4) Impersonal passive stance (\pali{bh\=avav\=acaka})} \ \par
This is a bit hard to understand and explain. It is the verb that expresses itself to show certain state-of-being. This normally occurs to intransitive verbs but in passive form, for example, ``\pali{tena \textbf{bh\=uyate}}'' (existing [is done] by him).\footnote{Translating this kind of sentence into English is awkward. To make it sensible, I change verb into noun.} In this sentence, \pali{bh\=uyate} (\pali{bh\=u + ya + te}) is an impersonal passive verb.

\paragraph*{(5) Causal passive stance (\pali{hetukammav\=acaka})} \ \par
This is rather complicated. Verb form of this expresses that the subject of the sentence is the patient of the action that is done by an actor who is caused by other actor, for example, ``\pali{odano s\=amikena s\=udena \textbf{p\=ac\=apiyate}}'' (Boiled rice is cooked by a chef [who is ordered] by a master). In this sentence, \pali{p\=ac\=apiyate} (\pali{paca + \d n\=ape + ya + i + te}) is a causal passive verb.

\bigskip
As you have seen from the mentioned examples, \pali{vibhatti} and \pali{paccaya} have different function. The former marks tense or mood, person, number, and voice, whereas the latter marks the relation between verb and subject of sentences.\footnote{From the examples, \pali{i} is not counted as a \pali{paccaya} but just an insertion.} You can see the distinction between `voice' and `stance' here. The former is marked by \pali{vibhatti}, whereas the latter is marked by \pali{paccaya}.

\phantomsection
\addcontentsline{toc}{section}{Active Verb Forms}
\section*{Active Verb Forms}

Now we will learn how to create active verb forms by application of \pali{paccaya}. There are three groups of \pali{paccaya} to learn here: for root-group (\pali{vikara\d napaccaya}), for root (\pali{dh\=atupaccaya}), and for transforming noun into denominative verbs.

\subsection*{\pali{Paccaya} for root-group}

According to Kacc/Sadd school, roots can be classified into eight groups. Each group has its own \pali{paccaya}. I summarize this in Table \ref{tab:rgroupk}.

\begin{table}[!hbt]
\centering
\caption{Root group according to Kacc/Sadd}
\label{tab:rgroupk}
\bigskip
\begin{tabular}{@{}ll>{\itshape}l@{}} \toprule
\bfseries No. & \bfseries Root & \bfseries\upshape \pali{Vikara\d napaccaya} \\
\midrule
1. & \pali{bh\=u}, etc. & a \\
2. & \pali{rudha}, etc. & a, i, \=i, e, o \upshape[with \d m insertion] \\
3. & \pali{diva}, etc. & ya \\
4. & \pali{su}, etc. & \d nu, \d n\=a, u\d n\=a \\
5. & \pali{k\=i}, etc. & n\=a \\
6. & \pali{gaha}, etc. & ppa, \d nh\=a \\
7. & \pali{tana}, etc. & o, yira \\
8. & \pali{cura}, etc. & \d ne, \d naya \\
\bottomrule
\end{tabular}
\end{table}

With a different perspective, in Mogg nine groups of roots are presented as shown in Table \ref{tab:rgroupm}. There are things worth noting in this scheme. In Mogg, \pali{gaha} and its peers is grouped with \pali{rudha}, so there is no group for this. The group of \pali{tuda}, which belongs to \pali{bh\=u} group in Kacc/Sadd scheme, is added. The difference is that this group does not undergo \pali{vuddhi} operation, whereas some of \pali{bh\=u} group in Kacc/Sadd do. This is marked by \pali{k-anubandha} in its \pali{paccaya}. To simplify our learning, we will follow Kacc/Sadd way of grouping.

\begin{table}[!hbt]
\centering
\caption{Root group according to Mogg}
\label{tab:rgroupm}
\bigskip
\begin{tabular}{@{}lll@{}} \toprule
\bfseries No. & \bfseries Root & \bfseries\upshape \pali{Vikara\d napaccaya} \\
\midrule
1. & \pali{bh\=u}, etc. & \pali{la} (= \pali{a}) \\
2. & \pali{rudha}, etc. & \pali{la} (= \pali{a}) [with \d m insertion] \\
3. & \pali{diva}, etc. & \pali{yaka} (= \pali{ya}) \\
4. & \pali{su}, etc. & \pali{k\d no} \\
5. & \pali{k\=i}, etc. & \pali{k\d n\=a} \\
6. & \pali{ji}, etc. & \pali{kn\=a} \\
7. & \pali{tana}, etc. & \pali{o} \\
8. & \pali{cura}, etc. & \pali{\d ni} (= \pali{\d ne, \d naya}) \\
9. & \pali{tuda}, etc. & \pali{ka} (= \pali{a}) \\
\bottomrule
\end{tabular}
\end{table}

For new students, before we go further, if you feel baffled with \pali{vuddhi} and \pali{anubandha}, because you just see them in first time here or you cannot remember it, I have a brief treatment for you. \pali{Vuddhi} is the top degree of vowel strength. The lesser one is called \pali{gu\d na}. And the least one has no name. We may call it zero strength. At this point, please refer to the last part of Chapter \ref{chap:nuts}. When certain \pali{paccaya} is in operation, it can cause, normally, the first vowel of root to be in \pali{vuddhi} strength. For example, \pali{i} can become \pali{e}, and \pali{u} can become \pali{o}. When you read on, you will find this kind of transformation a lot. A well-known marker of \pali{vuddhi} is \pali{\d n}. This means when you see \pali{\d n} in \pali{paccaya}, most of the time \pali{vuddhi} process will be involved. But sometimes \pali{vuddhi} can occur without \pali{\d n}-marker. We call \pali{\d n} and the like \pali{anubandha}. It is a marker in \pali{paccaya} to show that certain operation is needed apart from alphabet addition. That explains why you do not see \pali{\d n} in the product of \pali{\d n-anubandha}. It is in fact \pali{vuddhi + a}. 

However, as you will see below in \pali{su} and \pali{gaha} group, \pali{\d n} in the \pali{paccaya} of these is not \pali{anubandha}. It is the body of them, the character to be added, so to speak. But in \pali{cura} group, \pali{\d n} is \pali{anubandha} and \pali{vuddhi} is entailed. You are supposed to be confused now. That is the main reason why Moggall\=ana names \pali{paccaya}s differently in a more precise way. You will see a lot of \pali{paccaya}s behaving in various ways when you study derivations in Appendix \ref{chap:kita} and \ref{chap:taddhita}. I also summarize all \pali{paccaya}s in Appendix \ref{chap:paccaya}. You can also find discussions about certain \pali{paccaya}s there. If you have not seen those yet, do not haste into them, please finish this chapter first. It is far more important.

\paragraph*{(1) \pali{Bh\=u} group} (Kacc\,445, R\=upa\,433, Sadd\,925, Mogg\,5.18)\label{pacca:a1}\label{pacca:la1}\par
The number of roots in this group is far more numerous than other groups. It has only \pali{a} as group \pali{paccaya}. Some roots can undergo vowel \pali{vuddhi}. In Mogg, it is said to have \pali{la} instead. Both are identical in practice, but in Mogg it is more precise marked by \pali{l-anubandha} meaning that other thing can happen, such as \pali{vuddhi}. Here are some examples:\par
- \pali{bh\=u + a + ti} = \palibf{bhavati/bhoti}\footnote{Kacc\,513, R\=upa\,435, Sadd\,1027. See also Kacc\,485, R\=upa\,434, Sadd\,975, Mogg\,5.82.} ([One] exists)\par
- \pali{cu + a + ti} = \palibf{cavati} ([One] shifts/dies)\par
- \pali{h\=u + a + ti} = \palibf{hoti} ([One] exists)\par
- \pali{ikkha + a + ti} = \palibf{ikkhati} ([One] sees)\par
- \pali{labha + a + ti} = \palibf{labhati} ([One] gets)\par
- \pali{gamu + a + ti} = \palibf{gacchati}\footnote{Kacc\,476, R\=upa\,472, Sadd\,957, Mogg\,5.173} ([One] goes)\par
- \pali{gamu + a + ti} = \palibf{ghammati}\footnote{Kacc\,501, R\=upa\,443, Sadd\,1013, Mogg\,5.176} ([One] goes)\par
- \pali{gamu + a + si} = \palibf{gagghasi}\footnote{Sadd\,1013} ([You] go)\par
- \pali{y\=a + a + ti} = \palibf{y\=ati} ([One] goes)\par
- \pali{p\=a + a + ti} = \palibf{p\=ati} ([One] drinks)\par
- \pali{p\=a + a + ti} = \palibf{pivati/pipati}\footnote{Kacc\,469, R\=upa\,494, Sadd\,949, 1057, Mogg\,5.175} ([One] drinks)\par
- \pali{ji + a + ti} = \palibf{jayati}\footnote{Kacc\,514, R\=upa\,491, Sadd\,1028, Mogg\,5.89} ([One] wins)\par
- \pali{s\=i + a + ti} = \palibf{seti/sayati} ([One] lies down)\par
- \pali{n\=i + a + ti} = \palibf{neti/nayati} ([One] leads)\par
- \pali{d\=a + a + ti} = \palibf{dad\=ati/deti/dajjati}\footnote{Kacc\,499, R\=upa\,507, Sadd\,1005, Mogg\,5.176} ([One] gives)\par
- \pali{vada + a + ti} = \palibf{vadati/vadeti/vajjeti/vajjati}\footnote{In Kacc\,510, R\=upa\,487, Sadd\,1023, it is said that sometimes \pali{a} is deleted or changed to \pali{e}; see also Mogg\,5.161, 5.163, 5.176. For \pali{vajja} form, see Kacc\,500, R\=upa\,486, Sadd\,1006.} ([One] speaks)\par
- \pali{hana + a + ti} = \palibf{hanati/hanti/vadhati}\footnote{Mogg\,5.161, \pali{a} can be deleted sometimes. In Kacc\,592, R\=upa\,503, Sadd\,1058, \pali{hana} can change to \pali{vadha}.} ([One] kills)\par
- \pali{\=asa + a + ti} = \palibf{acchati}\footnote{Sadd\,1042} ([One] waits)\par
- \pali{\d th\=a + a + ti} = \palibf{ti\d t\d thati}\footnote{Kacc\,468, R\=upa\,492, Sadd\,949, Mogg\,5.175} ([One] stands)\par
- \pali{sa\d m + \d th\=a + a + ti} = \palibf{sa\d n\d thahati/sa\d n\d th\=ati}\footnote{Sadd\,1055, Mogg\,5.131} ([One] remains)\par
- \pali{pati + \d th\=a + a + ti} = \palibf{pati\d t\d thahati/pati\d t\d th\=ati}\footnote{Sadd\,1056} ([One] establishes)\par
- \pali{\~n\=a + a + ti} = \palibf{j\=an\=ati}\footnote{Kacc\,470, R\=upa\,514, Sadd\,950, Mogg\,5.120. In passive form \pali{\~n\=a} is retained, hence \pali{\~n\=ayati} ([A thing] is known).} ([One] knows)\par
- \pali{disa + a + ti} = \palibf{passati/dakkhati}\footnote{Kacc\,471, R\=upa\,483, Sadd\,951, Mogg\,5.124} ([One] sees)\par
- \pali{jara + a + ti} = \palibf{j\=irati/jiyyati/j\=iyati}\footnote{Kacc\,505, R\=upa\,482, Sadd\,1018, Mogg\,5.174} ([One] gets old)\par
- \pali{mara + a + ti} = \palibf{marati/miyyati/m\=iyati}\footnote{Kacc\,505, R\=upa\,482, Sadd\,1018, Mogg\,5.174} ([One] dies)\par
- \pali{ni + sada + a + ti} = \palibf{nis\=idati}\footnote{Kacc\,505, R\=upa\,482, Sadd\,1018, Mogg\,5.123} ([One] sits down)\par

- \pali{tuda + a + ti} = \palibf{tudati}\footnote{In Mogg\,5.22, this is treated as another group. The \pali{paccaya} is \pali{ka} (= \pali{a} without \pali{vuddhi}).} ([One] pricks)\par
Like \pali{tuda}, the following verbs are also rendered without \pali{vuddhi}. In Mogg's perspective, these can be grouped with \pali{tuda} and take \pali{ka-paccaya}.\par
- \pali{vi + kira + a + ti} = \palibf{vikirati} ([One] scatters)\par
- \pali{khipa + a + ti} = \palibf{khipati} ([One] throws)\par
- \pali{ni + gira + a + ti} = \palibf{nigirati} ([One] swallows)\par
- \pali{gila + a + ti} = \palibf{gilati} ([One] swallows)\par
- \pali{nuda + a + ti} = \palibf{nudati} ([One] expels)\par
- \pali{phusa + a + ti} = \palibf{phusati} ([One] touches)\par
- \pali{musa + a + ti} = \palibf{musati} ([One] steals)\par
- \pali{likha + a + ti} = \palibf{likhati} ([One] writes)\par
- \pali{vida + a + ti} = \palibf{vidati} ([One] knows)\par
- \pali{visa + a + ti} = \palibf{visati} ([It] diffuses)\par
- \pali{supa + a + ti} = \palibf{supati} ([One] sleeps)\par

\paragraph*{(2) \pali{Rudha} group} (Kacc\,446, R\=upa\,509, Sadd\,926, Mogg\,5.19, 5.93)\label{pacca:a2}\label{pacca:i}\label{pacca:ii}\label{pacca:e}\label{pacca:o1}\label{pacca:la2}\par
This group has \pali{a} etc.\ as as its \pali{paccaya} plus a special treatment of \pali{\d m} insertion after the first vowel of the roots. To illustrate, when \pali{rudha} is inserted with \pali{\d m}, it becomes \pali{ru + \d m + dha}. Then \pali{\d m} is assimilated by being changed to the nasal character of the following, thus \pali{n}. Hence we get \pali{rundha} as the product of the insertion. If you are still confused, see Appendix \ref{chap:sandhi}. From now on, I will not show \pali{\d m} in the decomposition, because it is not \pali{paccaya}. Some examples of this group are shown as follows:\par
- \pali{rudha + a + ti} = \palibf{rundhati} ([One] obstructs)\par
- \pali{chidi + a + ti} = \palibf{chindati} ([One] cuts)\par
- \pali{bhidi + a + ti} = \palibf{bhindati} ([One] breaks)\par
- \pali{bhuja + a + ti} = \palibf{bhu\~njati} ([One] eats)\par
In Sadd\,927, it is said that \pali{i, \=i, e,} and \pali{o} can be used as \pali{paccaya} sometimes, for example, \pali{rundhiti, rundh\=iti, rundheti,} and \pali{subha + o = sumbhoti} ([One] strikes).

\paragraph*{(3) \pali{Diva} group} (Kacc\,447, R\=upa\,510, Sadd\,928, Mogg\,5.21)\label{pacca:ya1}\label{pacca:yaka}\par
The \pali{paccaya} of this group is \pali{ya}. In Mogg it is called \pali{yaka}. With \pali{ka}, it stresses that no \pali{vuddhi} will be applied. I call \pali{k-anubandha} in Mogg's sense as `\pali{vuddhi} preventer.' Among the most used \pali{paccaya}s, \pali{ya} is one of them. It is used in a variety of contexts. It is noteworthy because of its unique characteristic. When the root has more than one character, under \pali{ya} operation the last character will undergo duplication like passive verb forms.\footnote{Kacc\,444, R\=upa\,511, Sadd\,924} Here are some examples:\par
- \pali{kh\=i + ya + ti} = \palibf{kh\=iyati} ([One] is exhausted)\par
- \pali{gh\=a + ya + ti} = \palibf{gh\=ayati} ([One] smells)\par
- \pali{divu + ya + ti} = \palibf{dibbati} ([One] plays)\par
- \pali{budha + ya + ti} = \palibf{bujjhati} ([One] knows)\par
- \pali{mana + ya + ti} = \palibf{ma\~n\~nati} ([One] deems)\par
- \pali{yudha + ya + ti} = \palibf{yujjhati} ([One] fights)\par
- \pali{ruca + ya + ti} = \palibf{ruccati} ([One] likes)\par
- \pali{lubha + ya + ti} = \palibf{lubbhati} ([One] desires)\par
- \pali{sivu + ya + ti} = \palibf{sibbati} ([One] sews)\par
- \pali{sudha + ya + ti} = \palibf{sujjhati} ([One] is purified)\par
- \pali{hana + ya + ti} = \palibf{ha\~n\~nati} ([One] kills)\par

\paragraph*{(4) \pali{Su} group} (Kacc\,448, R\=upa\,512, Sadd\,929, Mogg\,5.25)\label{pacca:dnu}\label{pacca:dnaa}\label{pacca:udnaa}\label{pacca:kdno}\par
In this group, \pali{paccaya}s used are \pali{\d nu, \d n\=a,} and \pali{u\d n\=a}. In these \pali{\d n} is not \pali{anubandha}, so it is added to the root under the process. In Mogg, the \pali{paccaya} is called \pali{k\d no}. With \pali{k-anubandha}, the \pali{vuddhi} process is prevented here. So, you just add \pali{\d no} to the root. Here are some examples:\par
- \pali{su} $\Rightarrow$ \palibf{su\d noti/su\d n\=ati}\footnote{In Mogg, \pali{su\d n\=ati} is a product of \pali{k\d n\=a} in \pali{k\=i} group.} ([One] listens)\par
- \pali{sa\d m + vu} $\Rightarrow$ \palibf{sa\d mvu\d noti/sa\d mvu\d n\=ati}\footnote{In Sadd\,976, it is said that \pali{sa\d mvu\d noti} has \pali{vuddhi} done to the \pali{paccaya} itself, thus \pali{\d nu} becomes \pali{\d no}.} ([One] restrains)\par
- \pali{saka} $\Rightarrow$ \palibf{sakku\d noti/sakku\d n\=ati}\footnote{Mogg\,5.121} ([One] is capable [of])\par
- \pali{pa + apa} $\Rightarrow$ \palibf{p\=apu\d noti/p\=apu\d n\=ati}\footnote{Mogg\,5.121} ([One] attains)\par

\paragraph*{(5) \pali{K\=i} group} (Kacc\,449, R\=upa\,513, Sadd\,930, Mogg\,5.23--4)\label{pacca:naa}\label{pacca:knaa}\label{pacca:kdnaa}\par
The \pali{paccaya} in this group is \pali{n\=a}. In Mogg this group is split into \pali{k\=i} and \pali{ji} group. The former uses \pali{k\d n\=a} (\pali{\d n\=a} without \pali{vuddhi}), and the latter \pali{kn\=a}. Examples are:\par
- \pali{k\=i + n\=a/k\d n\=a + ti} = \palibf{k\=i\d n\=ati/ki\d n\=ati}\footnote{The first vowel can be shortened (Sadd\,1074, Mogg\,6.32). See also Sadd\,1066.} ([One] buys)\par
- \pali{vi + k\=i + n\=a + ti} = \palibf{vikki\d n\=ati} ([One] sells)\par
- \pali{ji + n\=a + ti} = \palibf{jin\=ati} ([One] wins)\par
- \pali{dh\=u + n\=a + ti} = \palibf{dhun\=ati} ([One] removes)\par
- \pali{mu + n\=a + ti} = \palibf{mun\=ati} ([One] ties)\par
- \pali{l\=u + n\=a + ti} = \palibf{lun\=ati} ([One] cuts)\par
- \pali{p\=u + n\=a + ti} = \palibf{pun\=ati} ([One] cleanses)\par
- \pali{vi + ci + n\=a + ti} = \palibf{vicin\=ati} ([One] selects)\par
- \pali{m\=a + n\=a + ti} = \palibf{min\=ati}\footnote{Sadd\,1073} ([One] measures)\par
- \pali{\~n\=a + n\=a + ti} = \palibf{j\=an\=ati}\footnote{Kacc\,509, R\=upa\,516, Sadd\,1022, Mogg\,6.61} ([One] knows)\par
- \pali{\~n\=a + n\=a + ya + ti} = \palibf{n\=ayati}\footnote{Kacc\,509, R\=upa\,516, Sadd\,1022, Mogg\,6.61} ([One] is known)\par
- \pali{\~n\=a + n\=a + eyya} = \palibf{ja\~n\~n\=a}\footnote{Kacc\,509, R\=upa\,516, Sadd\,1022, Mogg\,6.62} ([One] should know)\par

\paragraph*{(6) \pali{Gaha} group} (Kacc\,450, R\=upa\,517, Sadd\,931)\label{pacca:ppa}\label{pacca:dnhaa}\par
This group has \pali{ppa} and \pali{\d nh\=a} as \pali{paccaya}. In Mogg this is grouped with \pali{rudha}. Examples are:\par
- \pali{gaha + ppa + ti} = \palibf{gheppati}\footnote{Kacc\,489, R\=upa\,519, Sadd\,981. In Mogg\,5.178, it is said that \pali{gaha} is transformed to \pali{gheppa}.} ([One] seizes)\par
- \pali{gaha + \d nh\=a + ti} = \palibf{ga\d nh\=ati}\footnote{In Kacc\,490, R\=upa\,518, Sadd\,982, \pali{ha} is deleted. In Mogg\,5.179, \pali{\d n} comes from \pali{\d m} insertion.} ([One] seizes)\par

\paragraph*{(7) \pali{Tana} group} (Kacc\,451, R\=upa\,520, Sadd\,932, Mogg\,5.26)\label{pacca:o2}\label{pacca:yira}\par
This group has \pali{o} and \pali{yira} as \pali{paccaya}, for example:\par
- \pali{tana + o + ti} = \palibf{tanoti} ([It] spreads)\par
- \pali{tana + o + te} = \palibf{tanute}\footnote{Kacc\,511, R\=upa\,521, Sadd\,1024, Mogg\,6.76} ([It] spreads)\par
- \pali{j\=agara + o + ti} = \palibf{j\=agaroti} ([One] is awake)\par
- \pali{saka + o + ti} = \palibf{sakoti} ([One] is capable [of])\par
- \pali{kara + o + ti} = \palibf{karoti} ([One] does)\par
- \pali{kara + o + te} = \palibf{kurute}\footnote{Kacc\,511, R\=upa\,521, Sadd\,1024} ([One] does)\par
- \pali{kara + yira + ti} = \palibf{kayirati}\footnote{Only \pali{kara} takes \pali{yira}.} ([One] does)\par

\paragraph*{(8) \pali{Cura} group} (Kacc\,452, R\=upa\,525, Sadd\,933, Mogg\,5.15)\label{pacca:dne1}\label{pacca:dnaya1}\label{pacca:dni1}\par
Two \pali{paccaya}s in this group are \pali{\d ne} and \pali{\d naya}. In these \pali{\d n} is \pali{vuddhi} marker. In Mogg, the two are seen as one, \pali{\d ni} which its \pali{i} can be changed to \pali{e} or \pali{aya}. Second to \pali{bh\=u} group, this group has a considerable number of roots. Here are some examples:\par
- \pali{cura + \d ne/\d naya + ti} = \palibf{coreti/corayati} ([One] steals)\par
- \pali{cinta + \d ne/\d naya + ti} = \palibf{cinteti/cintayati} ([One] thinks)\par
- \pali{ga\d na + \d ne/\d naya + ti} = \palibf{ga\d neti/ga\d nayati} ([One] counts)\par
- \pali{manta + \d ne/\d naya + ti} = \palibf{manteti/mantayati} ([One] consults)\par
- \pali{disa + \d ne/\d naya + ti} = \palibf{deseti/desayati} ([One] preaches)\par
- \pali{vanda + \d ne/\d naya + ti} = \palibf{vandeti/vandayati} ([One] salutes)\par

\bigskip
As you might realize, one meaning can be derived from multiple roots of different groups. Even the roots look alike, they are treated as different roots. For example, \pali{saka} (to be capable) can be of \pali{su} group, thus \pali{sakku\d noti} or \pali{sakku\d n\=ati} is rendered. It can also be of \pali{tana} group, thus \pali{sakoti}\footnote{The term is widely used as \pali{sakkoti}.} is rendered.

\subsection*{\pali{Paccaya} for roots}\label{pacca:kha}\label{pacca:cha}\label{pacca:sa}

There are three \pali{paccaya}s, i.e.\ \pali{kha, cha,} and \pali{sa}, that can change the meaning of certain roots under their operation. Reduplication (see below) can also be seen with these. \pali{Vibhatti} is also applied.\footnote{Kacc\,455, R\=upa\,530, Sadd\,936}

\paragraph*{(1) With \pali{tija, gupa, kita, m\=ana}} (Kacc\,433, R\=upa\,528, Sadd\,906--9, Mogg\,5.1--3)\par
In examples below, verbs with normal \pali{paccaya} are also shown for comparison.\par
- \pali{tija + a + ti} = \palibf{tejati} ([One] sharpens)\par
- \pali{tija + kha + ti} = \palibf{titikkhati} ([One] endures)\par
- \pali{gupa + a + ti} = \palibf{gopati} ([One] protects)\par
- \pali{gupa + cha + ti} = \palibf{jigucchati} ([One] loathes)\par
- \pali{badha + \d ni + ti} = \palibf{b\=adheti} ([One] binds)\par
- \pali{badha + cha + ti} = \palibf{b\=ibhacchati}\footnote{This instance is proposed in Mogg\,5.3.} ([One] loathes)\par
- \pali{kita + a + ti} = \palibf{ketati} ([One] notes)\par
- \pali{kita + cha + ti} = \palibf{tikicchati} ([One] cures)\par
- \pali{m\=ana + \d ne + ti} = \palibf{m\=aneti} ([One] honors)\par
- \pali{m\=ana + sa + ti} = \palibf{v\=ima\d msati} ([One] investigates)\par

\paragraph*{(2) With \pali{bhuja, ghasa, hara, su, p\=a}} (Kacc\,434, R\=upa\,534, Sadd\,910, Mogg\,5.4)\par
This group relates to \pali{tu\d m-paccaya} of verbal \pali{kita} by its meaning. It denotes the intention or desire to do something. Here are examples:\par
- \pali{bhuja + kha + ti} = \palibf{bubhukkhati}\footnote{This is equal to ``\pali{bhottu\d m icchati}.''} ([One] wishes to eat)\par
- \pali{ghasa + cha + ti} = \palibf{jighacchati}\footnote{This is equal to ``\pali{ghasitu\d m icchati}.''} ([One] wishes to eat)\par
- \pali{hara + sa + ti} = \palibf{jigi\d msati}\footnote{This is equal to ``\pali{haritu\d m icchati}.''} ([One] wishes to acquire)\par
- \pali{su + sa + ti} = \palibf{suss\=usati}\footnote{This is equal to ``\pali{sotu\d m icchati}.''} ([One] wishes to hear)\par
- \pali{p\=a + sa + ti} = \palibf{pip\=asati}\footnote{This is equal to ``\pali{p\=atu\d m icchati}.''} ([One] wishes to drink)\par

\subsection*{\pali{Paccaya} for denominative verbs}\label{sec:denomverbs}

There are \pali{paccaya}s that can magically change nouns into verbs. In Kacc/Sadd, three are mentioned, \pali{\=aya, \=iya,} and \pali{\d naya}. In Mogg, five are mentioned, \pali{\=aya, assa, \=iya, \d naya,} and \pali{\=api}. There are uses to be concerned as follows:

\paragraph*{(1) \pali{\=Aya} on imitating agents} (Kacc\,435, R\=upa\,536, Sadd\,911, Mogg\,5.8)\label{pacca:aaya}\par
- \pali{pabbata + \=aya + ti} = \palibf{pabbat\=ayati}\footnote{\pali{sa\d mgho pabbato iva, att\=anam\=acarati pabbat\=ayati.}} ([One] acts like a mountain)\par
- \pali{samudda + \=aya + ti} = \palibf{samudd\=ayati} ([One] acts like an ocean)\par
- \pali{cicci\d ta + \=aya + ti} = \palibf{cicci\d t\=ayati} ([One] makes chit-chit sound)\par

\paragraph*{(2) \pali{\=Aya} on becoming} (Mogg\,5.9)\par
This means something happening unexpectedly, for example:\par
- \pali{bhusa + \=aya + ti} = \palibf{bhus\=ayati} ([It] becomes chaff)\par
- \pali{pa\d tapa\d ta + \=aya + ti} = \palibf{pa\d tapa\d t\=ayati} ([It] sounds like pat-pat)\footnote{Perhaps it sounds like a cloth waving in wind.}\par
- \pali{lohita + \=aya + ti} = \palibf{lohit\=ayati} ([It] becomes red)\par

\paragraph*{(3) \pali{\=Aya} on producing something} (Mogg\,5.10)\par
- \pali{sadda + \=aya + ti} = \palibf{sadd\=ayati} ([One] makes sound)\par
- \pali{vera + \=aya + ti} = \palibf{ver\=ayati} ([One] makes enmity)\par
- \pali{kalaha + \=aya + ti} = \palibf{kalah\=ayati} ([One] makes a quarrel)\par

\paragraph*{(4) \pali{\=Iya} on imitated patients} (Kacc\,436, R\=upa\,537, Sadd\,912, Mogg\,5.6)\label{pacca:iiya}\par
- \pali{chatta + \=iya + ti} = \palibf{chatt\=iyati}\footnote{\pali{achatta\d m chattamiva, \=acarati chatt\=iyati.}} ([One] treats [something] as if it is an umbrella)\par
- \pali{putta + \=iya + ti} = \palibf{putt\=iyati} ([One] treats [someone] as if he/she is one's own child)\par

\paragraph*{(5) \pali{\=Iya} on acting in place} (Mogg\,5.7)\par
- \pali{ku\d ti + \=iya + ti} = \palibf{ku\d t\=iyati [p\=as\=ade]} ([One] acts in a mansion as if it is a hut)\par
- \pali{p\=as\=ada + \=iya + ti} = \palibf{p\=as\=ad\=iyati [ku\d tiya\d m]} ([One] acts in a hut as if it is a mansion)\par

\paragraph*{(6) \pali{\=Iya} on object of desire for oneself} (Kacc\,437, R\=upa\,538, Sadd\,913, Mogg\,5.5)\par
- \pali{putta + \=iya + ti} = \palibf{putt\=iyati}\footnote{\pali{attano puttamicchati putt\=iyati.}} ([One] wishes a child for oneself)\par
- \pali{patta + \=iya + ti} = \palibf{patt\=iyati} ([One] wishes a bowl for oneself)\par
- \pali{c\=ivara + \=iya + ti} = \palibf{c\=ivar\=iyati} ([One] wishes a robe for oneself)\par

\paragraph*{(7) \pali{\d Naya} on noun as root} (Kacc\,439, R\=upa\,539, Sadd\,919, Mogg\,5.12)\label{pacca:dnaya2}\par
- \pali{ati + hatthi + \d naya + ti} = \palibf{atihatthayati}\footnote{\pali{hatthin\=a atikkamati atihatthayati.}} ([One] overcomes with an elephant)\par
- \pali{upa + v\=i\d n\=a + \d naya + ti} = \palibf{upav\=i\d nayati}\footnote{\pali{v\=i\d n\=aya upag\=ayati upav\=i\d nayati.}} ([One] goes for singing with a lute)\par
- \pali{da\d lha + \d naya + ti} = \palibf{da\d lhayati}\footnote{\pali{da\d lha\d m karoti v\=iriya\d m da\d lhayati.}} ([One] strengthens)\par
- \pali{kusala + \d naya + ti} = \palibf{kusalayati}\footnote{\pali{kusala\d m pucchati kusalayati.}} ([One] asks for goodness)\par

\paragraph*{(8) \pali{Assa} with \pali{namo}} (Mogg\,5.11)\label{pacca:assa}\par
- \pali{namo + assa + ti} = \palibf{namassati} ([One] venerates)\par

\paragraph*{(9) \pali{\=Api} with \pali{sacca}, etc.} (Mogg\,5.13)\label{pacca:aapi}\par
- \pali{sacca + \=api + ti} = \palibf{sacc\=apeti} ([One] tells the truth)\par
- \pali{sukha + \=api + ti} = \palibf{sukh\=apeti} ([One] makes happy)\par
- \pali{dukkha + \=api + ti} = \palibf{dukkh\=apeti} ([One] makes unhappy)\par
- \pali{veda + \=api + ti} = \palibf{ved\=apeti} ([One] makes knowledge [learn?])\par

\phantomsection
\addcontentsline{toc}{section}{Causative Verb Forms}
\section*{Causative Verb Forms}\label{pacca:dne2}\label{pacca:dnaya3}\label{pacca:dnaape}\label{pacca:dnaapaya}\label{pacca:dni2}\label{pacca:dnaapi}

In English when we create a causative sentence, we just use some verbs that have a meaning contributing to that condition. For example, we use `have', `get', `make', or the like to denote that someone causes another one to do something. That is quite easy. In P\=ali it is not that simple. We have to use a different verb form to mark causative condition. In Kacc/Sadd, there are four \pali{paccaya}s that mark causative form, i.e.\ \pali{\d ne, \d naya, \d n\=ape,} and \pali{\d n\=apaya}. In Mogg, they are named differently, so only two are mentioned, \pali{\d ni} and \pali{\d n\=api}. We will follow Kacc/Sadd naming scheme here. In all these, \pali{\d n} is deleted when applied.\footnote{Kacc\,523, R\=upa\,526} It is a \pali{vuddhi} marker that causes the first vowel to be in \pali{vuddhi} strength, if it is not followed by double consonants.\footnote{Kacc\,483, R\=upa\,527, Sadd\,973, Mogg\,5.84} And when we compose these into a sentence, \pali{vibhatti} has to be applied too. For more detail on the use of the causative, see Chapter \ref{chap:caus}.

\paragraph*{(1) \pali{\d Ne, \d naya, \d n\=ape, \d n\=apaya} on causative verbs} (Kacc\,438, R\=upa\,540, Sadd\,914, Mogg\,5.16)\par
This is a general use of these \pali{paccaya}s. In Sadd\,917, it is stressed that these can be used with multi-syllabled roots, for example:\par
- \pali{kara + \d ne + ti} = \palibf{k\=areti} ([One] causes another to do)\par
- \pali{kara + \d naya + ti} = \palibf{k\=arayati} ([One] causes another to do)\par
- \pali{kara + \d n\=ape + ti} = \palibf{k\=ar\=apeti} ([One] causes another to do)\par
- \pali{kara + \d n\=apaya + ti} = \palibf{k\=ar\=apayati} ([One] causes another to do)\par
- \pali{o + bh\=asa + \d ne + ti} = \palibf{obh\=aseti} ([One] causes [a thing] to illuminate)\par
- \pali{o + bh\=asa + \d naya + ti} = \palibf{obh\=asayati} ([One] causes [a thing] to illuminate)\par

\paragraph*{(2) \pali{\d Ne, \d naya} on verbs ending with \pali{u, \=u}} (Sadd\,915)\par
- \pali{su + \d ne + ti} = \palibf{s\=aveti} ([One] causes another to listen)\par
- \pali{su + \d naya + ti} = \palibf{s\=avayati} ([One] causes another to listen)\par
- \pali{bh\=u + \d ne + ti} = \palibf{bh\=aveti} ([One] causes another to be)\par
- \pali{bh\=u + \d naya + ti} = \palibf{bh\=avayati} ([One] causes another to be)\par

\paragraph*{(3) \pali{\d N\=ape, \d n\=apaya} on verbs ending with \pali{\=a}} (Sadd\,916)\par
- \pali{d\=a + \d n\=ape + ti} = \palibf{d\=apeti} ([One] causes another to give)\par
- \pali{d\=a + \d n\=apaya + ti} = \palibf{d\=apayati} ([One] causes another to give)\par

\paragraph*{(4) \pali{\d N\=ape, \d n\=apaya} on verbs in \pali{cura} group} (Sadd\,918)\par
This is reasonable, because this verb-group already has \pali{\d ne} and \pali{\d naya} as its group \pali{paccaya}.\par
- \pali{cura + \d n\=ape + ti} = \palibf{cor\=apeti} ([One] causes another to steal)\par
- \pali{cura + \d n\=apaya + ti} = \palibf{cor\=apayati} ([One] causes another to steal)\par
- \pali{cinta + \d n\=ape + ti} = \palibf{cint\=apeti} ([One] causes another to think)\par
- \pali{cinta + \d n\=apaya + ti} = \palibf{cint\=apayati} ([One] causes another to think)\par

\paragraph*{(5) Other specific concerns} \ \par
Sometimes \pali{vuddhi} is optional (Kacc\,484, R\=upa\,542, Sadd\,974), for example:\par
- \pali{gha\d ta + \d ne + ti} = \palibf{gh\=a\d teti/gha\d teti} ([One] causes another to strive)\par
- \pali{gamu + \d ne + ti} = \palibf{g\=ameti/gameti} ([One] causes another to go)\par
Sometimes the first vowel is just lengthened (Kacc\,486, R\=upa\,543, Sadd\,977, Mogg\,5.104--5), for example:\par
- \pali{guha + \d naya + ti} = \palibf{g\=uhayati} ([One] causes another to cover)\par
- \pali{dusa + \d naya + ti} = \palibf{d\=usayati} ([One] causes another to offend)\par
After \pali{vuddhi} is applied, the verb can be transformed further (Kacc\,515, R\=upa\,541, Sadd\,1029, 1100, Mogg\,5.90), for example:\par
- \pali{l\=u + \d ne + ti} = \palibf{l\=aveti}\footnote{\pali{l\=u} $\rightarrow$ \pali{lo} $\rightarrow$ \pali{l\=ava}} ([One] causes another to cut)\par
- \pali{n\=i + \d ne + ti} = \palibf{n\=ayeti}\footnote{\pali{n\=i} $\rightarrow$ \pali{ne} $\rightarrow$ \pali{n\=aya}} ([One] causes another to lead)\par
Sometimes \pali{\d ne} and \pali{\d n\=ape} are applied together (Sadd\,1101), for example:\par
- \pali{pari + ava + so + \d ne + \d n\=ape + ti} = \palibf{pariyos\=av\=apeti} ([One] causes another to finish [some task])\par
Sometimes \pali{\=i} is transformed to \pali{\=a} (Sadd\,1040), for example:\par
- \pali{ni + sada + \d ne + ti} = \palibf{nis\=adeti}\footnote{This means instead of \pali{nis\=ideti} it becomes \pali{nis\=adeti}. Also \pali{nis\=id\=apeti} can be found.} ([One] causes another to sit down)\par

\phantomsection
\addcontentsline{toc}{section}{Passive Verb Forms}
\section*{Passive Verb Forms}\label{pacca:ya2}\label{pacca:kya}

This section also includes impersonal passive stance, for they use the same \pali{paccaya}. The only one to use here is \pali{ya}, or \pali{kya} (\pali{ya} without \pali{vuddhi}) in Mogg.\footnote{Kacc\,440, R\=upa\,445, Sadd\,920, Mogg\,5.17} The difference between passive and impersonal passive is the former uses transitive verbs, whereas the latter uses intransitive verbs. When passive verbs are composed in sentences, \pali{vibhatti} is also applied. In some examples below, it is shown that the middle voice (\pali{attanopada}) is used. However, evidence shows that normal active voice (\pali{parassapada}) can be used as well.\footnote{Kacc\,518, R\=upa\,446, Sadd\,1031} For more about passive voice, see Chapter \ref{chap:pass}. Here are some examples:\par
- \pali{kara + ya + te} = \palibf{kar\=iyate/kayyate}\footnote{For \pali{\=i} insertion see below. For \pali{kayyate} see Sadd\,1068.} ([A thing] is being done/Doing [is being done by one])\par
- \pali{yuja + ya + te} = \palibf{yujjate} ([A thing] is being composed)\par
- \pali{labha + ya + te} = \palibf{labbhate} ([A thing] is being obtained)\par
- \pali{bh\=u + ya + te} = \palibf{bhuyyate} (Existing [is being done by one])\par
- \pali{\d th\=a + ya + te} = \palibf{\d th\=iyate} (Standing [by one])\par
- \pali{su + ya + te} = \palibf{s\=uyate} (Hearing [by one]/[Sound] is being heard)\par
- \pali{\=a + d\=a + ya + ti} = \palibf{\=adiyati}\footnote{Sadd\,1063, Mogg\,5.132} ([A thing] is taken)\par
- \pali{sa\d m + \=a + d\=a + ya + ti} = \palibf{sam\=adiyati} ([A thing] is taken upon)\par
- \pali{jana + ya + ti} = \palibf{j\=ayati}\footnote{Sadd\,1064} ([One] is born)\par

\bigskip
When \pali{ya} is applied, there are things to be concerned as follows:

\paragraph*{(1) \pali{Ya} and the last consonant are changed to \pali{ca, cha, ja, jha, \~na, ya, va}} (Kacc\,441, R\=upa\,447, Sadd\,921)\par
- \pali{vaca + ya + te} = \palibf{vuccate} (Saying [by one])\par
- \pali{mada + ya + te} = \palibf{majjate} (Intoxicating [by one])\par
- \pali{budha + ya + te} = \palibf{bujjhate} (Knowing [by one])\par
- \pali{hana + ya + te} = \palibf{ha\~n\~nate} ([One] is being hurt)\par
- \pali{kara + ya + te} = \palibf{kayyate} ([A thing] is being done/Doing [by one])\par
- \pali{divu + ya + te} = \palibf{dibbate}\footnote{In P\=ali sometimes \pali{v} and \pali{b} can be interchanged.} (Playing [by one])\par

\paragraph*{(2) Sometimes \pali{i} or \pali{\=i} is inserted} (Kacc\,442, R\=upa\,448, Sadd\,922, Mogg\,6.37)\par
- \pali{kara + ya + te} = \palibf{kariyyate/kar\=iyate} ([A thing] is being done/Doing [by one])\par
- \pali{gamu + ya + te} = \palibf{gacchiyyate/gacch\=iyate} (Going [by one])\par
- \pali{paca + ya + ti} = \palibf{pac\=iyati}\footnote{From Mogg\,6.37, it is said \pali{\=i\~na} is inserted.} (Cooking [by one]/[Food] is being cooked)\par

\paragraph*{(3) Sometimes \pali{ya} assimilates the preceding consonant} (Kacc\,443, R\=upa\,449, Sadd\,923)\par
- \pali{va\d d\d dha + ya + te} = \palibf{vu\d d\d dhate} (Growing [by one])\par
- \pali{damu + ya + te} = \palibf{dammate} ([One] is being tamed/trained)\par
- \pali{phala + ya + te} = \palibf{phallate} ([Fruit] is being produced)\par
- \pali{labha + ya + te} = \palibf{labbhate} ([A thing] is being got)\par
- \pali{disa + ya + te} = \palibf{dissate} ([A thing] is being seen/Seeing [is being done by one])\par

\paragraph*{(4) Other specific concerns} \ \par
For some roots, the first \pali{a} is changed to \pali{u} (Kacc\,487, R\=upa\,478, Sadd\,978), for example:\par
- \pali{vaca + ya + ti} = \palibf{vuccati/uccati} (Saying [by one])\par
- \pali{vasa + ya + ti} = \palibf{vussati} (Living [by one])\par
- \pali{vaha + ya + ti} = \palibf{vuyhati/vulhati}\footnote{See also Kacc\,488, R\=upa\,481, Sadd\,980, 1048, 1050} ([A thing] is carried away [by water])\par
For some roots, the last vowel is changed to \pali{\=i} (Kacc\,502, R\=upa\,493, Sadd\,1014, Mogg\,5.137), for example:\par
- \pali{d\=a + ya + ti} = \palibf{d\=iyati} (Giving [by one])\par
- \pali{dh\=a + ya + ti} = \palibf{dh\=iyati} (Holding [by one])\par
- \pali{m\=a + ya + ti} = \palibf{m\=iyati} (Measuring [by one])\par
- \pali{\d th\=a + ya + ti} = \palibf{\d th\=iyati} (Standing [by one])\par
- \pali{h\=a + ya + ti} = \palibf{h\=iyati} (Abandoning [by one])\par
- \pali{p\=a + ya + ti} = \palibf{p\=iyati} (Drinking [by one])\par
- \pali{maha + ya + ti} = \palibf{mah\=iyati} (Honoring [by one])\par
- \pali{matha + ya + ti} = \palibf{math\=iyati} (Disturbing [by one])\par
Sometimes the last vowel is lengthened (Mogg\,5.139), for example:\par
- \pali{ci + ya + te} = \palibf{c\=iyate} (Collecting [by one])\par
- \pali{su + ya + te} = \palibf{s\=uyate} (Listening [by one])\par
Specifically for \pali{yaja}, \pali{ya} is changed to \pali{i} (Kacc\,503, R\=upa\,485, Sadd\,1015).\par
- \pali{yaja + ya + te} = \palibf{ijjate} (Sacrificing [by one])\par
Specifically for \pali{\~n\=a}, sometimes \pali{\=a} is changed to \pali{e} (Sadd\,1069).\par
- \pali{\~n\=a + ya + ti} = \palibf{\~neyyati}\footnote{Typically, we use \pali{\~n\=ayati}.} (Knowing [by one])\par
Specific treatment for \pali{tana} (Mogg\,5.138).\par
- \pali{tana + ya + te} = \palibf{t\=ayate/ta\~n\~nate} ([A thing] is spread)\par

\phantomsection
\addcontentsline{toc}{section}{Reduplication}
\section*{Reduplication}\label{sec:redup}

This topic is quite advanced. With a second thought, I decide to add this rather than omit it. The merit of this topic is on deeper understanding in P\=ali word formation, but less on using. Reduplication is an ancient technique in creating new words from existing roots. In Greek, it is ``the addition of a syllable to the front of the root, and this syllable consists of the initial consonant of the root (sometimes slightly modified).''\footnote{\citealp[p.~134]{fairbairn:understanding}} In P\=ali it is called \pali{abbh\=asa}.\footnote{Kacc\,459, R\=upa\,462, Sadd\,940. To be precise, just the newly added part is called \pali{abbh\=asa}.} Reduplication in P\=ali can happen when certain \pali{paccaya}s are applied, i.e.\ \pali{kha, cha,} and \pali{sa}\footnote{Some outcome of these can be called \emph{desiderative} denoting certain wishes (see also \citealp[pp.~352--3]{warder:intro}). And some can be called, by Warder, \emph{intensive conjugation} (p.~331).}; when perfect (\pali{parokkh\=a}) verbs are formed; and when it is needed for certain roots. To ease our learning, I will just list reduplicated instances, for we can be familiar with them. Rules posited by the textbooks will be mentioned if necessary in footnotes. Here are the examples:

- \pali{tija + kha + ti} = \palibf{titikkhati}\footnote{In Kacc\,458, R\=upa\,461, Sadd\,939, it is said that the first character of the root is duplicated and it takes the same vowel. In Mogg\,5.69 and 5.75, it is said that the whole root is duplicated and the ending is deleted, thus \pali{tija} $\rightarrow$ \pali{tik} $\rightarrow$ \pali{tiktik} $\rightarrow$ \pali{titik}} ([One] endures)\par
- \pali{asa + sa + ti} = \palibf{asisisati}\footnote{In Mogg\,5.71, sometimes two syllables are duplicated.} ([One] wishes to eat)\par
- \pali{tija + kha + sa + ti} = \palibf{titikkhisati}\footnote{This is a double reduplication. In Mogg\,5.72, it is said that if the form is reduplicated, no further duplication will be applied.} ([One] wishes to endure)\par
- \pali{d\=a + a + ti} = \palibf{dad\=ati}\footnote{Make the first vowel short (Mogg\,5.74).} ([One] gives)\par
- \pali{chidi + a + a} = \palibf{cicchida}\footnote{In Kacc\,461, R\=upa\,464, Sadd\,942, Mogg\,5.78, it is said that when being duplicated, aspirated character is changed to its unaspirated pair (see Chapter \ref{chap:nuts} for more detail), for example, \pali{cha}$\rightarrow$\pali{ca}, \pali{dha}$\rightarrow$\pali{da}, \pali{bha}$\rightarrow$\pali{ba}. This instance is a perfect verb.} ([One] cut)\par
- \pali{bhuja + kha + ti} = \palibf{bubhukkhati} ([One] wishes to eat)\par
- \pali{dh\=a + a + ti} = \palibf{dadh\=ati} ([One] holds)\par
- \pali{kita + cha + ti} = \palibf{cikicchati}\footnote{A guttural character is changed to palatal one (Kacc\,462, R\=upa\,467, Sadd\,943, Mogg\,5.79).} ([One] cures)\par
- \pali{gamu + a + ti} = \palibf{ja\.ngamati} ([One] travels)\par
- \pali{h\=a + a + ti} = \palibf{jah\=ati}\footnote{\pali{Ha} is changed to \pali{ja} (Kacc\,464, R\=upa\,504, Sadd\,945, Mogg\,5.79).} ([One] abandons)\par
- \pali{hu + a + ti} = \palibf{juhoti} ([One] honors)\par
- \pali{m\=ana + sa + ti} = \palibf{v\=ima\d msati}\footnote{Kacc\,463, R\=upa\,532, Sadd\,944, Mogg\,5.80. See also Kacc\,467, R\=upa\,533, Sadd\,948.} ([One] investigates)\par
- \pali{kita + cha + ti} = \palibf{tikicchati}\footnote{Kacc\,463, R\=upa\,532, Sadd\,944, Mogg\,5.81. Also \pali{cikicchati} is valid.} ([One] cures)\par
- \pali{gupa + cha + ti} = \palibf{jigucchati}\footnote{Sometimes the first vowel will be \pali{i} or \pali{\=i} or \pali{a} (Kacc\,465, R\=upa\,463, Sadd\,946, see also Mogg\,5.76--7).} ([One] loathes)\par
- \pali{ghasa + cha + ti} = \palibf{jighacchati} ([One] wishes to eat)\par
- \pali{bh\=u + a + a} = \palibf{babh\=uva} ([One] was)\par
- \pali{kamu + a + ti} = \palibf{ca\.nkamati}\footnote{Insertion of \pali{\d m} can be appied (Kacc\,466, R\=upa\,489, Sadd\,947).} ([One] walks about)\par
- \pali{cala + a + ti} = \palibf{ca\~ncalati} ([One] moves)\par
- \pali{api + dh\=a + a + ti} = \palibf{pidahati}\footnote{Sadd\,1059} ([One] closes)\par
- \pali{dh\=a + a + ti} = \palibf{dahati}\footnote{Mogg\,5.103} ([One] accepts)\par
- \pali{p\=a + sa + ti} = \palibf{piv\=asati}\footnote{Kacc\,467, R\=upa\,533, Sadd\,948} ([One] wishes to drink)\par
- \pali{hara + sa + ti} = \palibf{jig\=isati}\footnote{Kacc\,474, R\=upa\,535, Sadd\,954, Mogg\,5.102} ([One] wishes to acquire)\par
- \pali{vi + ji + sa + ti} = \palibf{vijig\=isati}\footnote{Sadd\,955, Mogg\,5.102} ([One] wishes to win)\par
- \pali{hana + sa + ti} = \palibf{jigha\d msati}\footnote{Mogg\,5.101} ([One] wishes to kill)\par
