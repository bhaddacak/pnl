\chapter{\headhl{Going to town}, I buy you a book}\label{chap:prp}

\phantomsection
\addcontentsline{toc}{section}{Introduction to Present Participles}
\section*{Introduction to Present Participles}

P\=ali has nice ways for constructing a complex sentence. In Chapter \ref{chap:yata} we have learned how to make a complex structure by correlation. That is quite an elegant way to do, from my view. In this chapter, we will learn about the present participle in P\=ali. The main tool used to achieve this is verbal \pali{kita}, namely verbs in \pali{anta} and \pali{m\=ana} (also \pali{\=ana}) form. In principle, you can refer to Appendix \ref{chap:kita}, page \pageref{pacck10:maana}. Here our focus is on how to use them in practice.

In English, we make a distinction between \emph{finite} and \emph{non-finite} verb. The former can complete sentences with information of tense, person, and number. In a simple sentence, there must be only one finite verb. That is the general idea when we think of a verb---the action that the subject does. On the other hand, non-finite verbs cannot complete sentences, and they do not provide information about tense, person, and number. In English, we have \emph{infinitives} and \pali{participles} as non-finite verbs. That is to say, non-finite verbs only appear as a part in sentences, mostly as a subordinate or relative clause. For example, in our heading task we have ``Going to town, I buy you a book.'' In this sentence, `buy' is finite, whereas `going' is non-finite. This can mean ``I go to town to buy you a book'' or ``I go to town, also I buy you a book.''

When we use English grammatical terms to explain P\=ali grammar, often the terms do not fit well. For example, some of several verbal \pali{kita}s we have can be of finite kind, i.e.\ \pali{ta, tabba,} and \pali{an\=iya}. Some are loosely closer to non-finite kind, such as \pali{anta} and \pali{m\=ana} in our concern here. Scholars call products of these \pali{kita} \emph{present participles}. The name sounds not suitable enough, because present participles can commonly appear in a past sentence, for example, ``Going to town, I bought you a book.'' In P\=ali it can be used in a similar way.

Here is good news. Verbs in \pali{anta} and \pali{m\=ana} form are easy to deal with. New students love these because they have only a few irregular forms. If you can figure out a present verb, say, \pali{gacchati}, you can render the result at ease, hence \pali{gacchanta} and \pali{gaccham\=ana} (going). Only common irregular terms we should be aware of are \pali{m\=ana} forms of \pali{karoti}, thus \pali{kurum\=ana} (doing) and in rare case \pali{kar\=ana}.

Now, here is a guideline when we use present participles in P\=ali.
\begin{enumerate}
\item Specify the subordinate verb to use by its root, or easier by the stem of its present form. Then apply \pali{anta} or \pali{m\=ana} to the stem. For active structure, both forms can be used interchangeably. For passive structure, only \pali{m\=ana} can be used (see Chapter \ref{chap:pass}).
\item Identify the doer of that action. It can be the same as the main verb, the subject of the sentence. Or it can be an other noun. If the doer takes the subject position, apply it with nominative case agreeable to gender and number of the subject, hence \pali{anto/ant\=a} (m.), \pali{ant\=i/ant\=iyo} (f.), \pali{anta\d m/ant\=ani} (nt.); \pali{m\=ano/m\=an\=a} (m.), \pali{m\=an\=a/m\=an\=ayo} (f.), \pali{m\=ana\d m/m\=an\=ani} (nt.). Please note on feminine forms. If the doer of the subordinate action is a noun other than the subject, apply it with the case agreeable to that noun.
\item Apply proper cases to other components related to the participle, if any. For example, if the action has an object, make it accusative as usual.
\item Compose the clause to the sentence in a proper order. Remember that it cannot finish the sentence, so the main verb with a proper ending has to be present too, if not understood.
\end{enumerate}

If you are ready, here we go for our heading task in the case that the speaker is a male:

\palisample{aha\d m nagara\d m gacchanto tuyha\d m potthaka\d m ki\d n\=ami. \sampleor aha\d m nagara\d m gaccham\=ano \ldots}

And if the speaker is a female, we get this instead:

\palisample{aha\d m nagara\d m gacchant\=i tuyha\d m potthaka\d m ki\d n\=ami. \sampleor aha\d m nagara\d m gaccham\=an\=a \ldots}

And here are some examples from the canon:

\begin{quote}
\pali{Kalandag\=ame sapad\=ana\d m pi\d n\d d\=aya \textbf{caram\=ano} yena sakapitu nivesana\d m tenupasa\.nkami.}\footnote{Buv1\,30}\\
``Walking for alms house by house in the village of Kalanda, [Sudinna] approached his father's house.''\\[1.5mm]
\pali{Ek\=a g\=amantara\d m \textbf{gacchant\=i} tisso \=apattiyo \=apajjati.}\footnote{Pvr\,229}\\
``Going to a village alone, [a bhikkhun\=i] gets into three offenses.''\\[1.5mm]
\pali{Tena kho pana samayena chabbaggiy\=a bhikkh\=u ucc\=asadda\d m mah\=asadda\d m \textbf{karont\=a} antaraghare gacchanti.}\footnote{Buv2\,588}\\
``By that occasion, the six monks, making a loud noise, go to the village.''\\[1.5mm]
\pali{ekacco puggalo \ldots k\=ala\d m \textbf{kurum\=ano} \=ak\=as\=ana\~c\=ayatan\=upag\=ana\d m dev\=ana\d m sahabyata\d m upapajjati.}\footnote{A3\,117}\\
``Some person, [after] dying, is reborn as a companion of deities in the Realm of Infnite Space.''\\[1.5mm]
\end{quote}

And here is an example that the actor of participle is not the subject of the sentence:

\begin{quote}
\pali{Addasa\d msu kho gop\=alak\=a pasup\=alak\=a kassak\=a path\=avino bhagavanta\d m d\=uratova \textbf{\=agacchanta\d m}.}\footnote{Buv2\,326}\\
``Cowherds, cattlemen, farmers, and travellers saw the Buddha coming from a faraway [place].''\\
\end{quote}

In this instance \pali{addasa\d msu} ([They] saw) is the main verb in aorist.\footnote{It is worth noting that \pali{addas\=a} and its variation are often placed at the beginning.} The object of the main verb is \pali{bhagavanta\d m}, the doer of \pali{\=agacchanta\d m}. That is why they take accusative case. As a part of the subordinate clause, \pali{d\=uratova} is a chunk of particles, so no declension is needed. If you ponder on this example, you can see that terms with \pali{anta} or \pali{m\=ana} work really like a modifier. It is logical to translate \pali{\=agacchanta} as ``one who is coming.'' This blurs the distinct line between verbal and nominal status of P\=ali participles. It is true to other product of verbal \pali{kita} as well. You can read it either as a verb or a noun (adjective included), so to speak.

There is a thing to be aware of here. When you treat terms in \pali{anta} form as a noun, you have to use its declensional paradigm, which is a little irregular. See the paradigm of \pali{gacchanta} in Appendix \ref{chap:decl}, page \pageref{decl:gacchanta}. Here is an example of this:

\begin{quote}
\pali{Atha pan\=aya\d m sama\d no \textbf{gaccha\d m} yev\=aha \d thito aha\d m}\footnote{M2\,348 (MN\,86)}\\
``This ascetic who was going but said `I stood'.''\\
\end{quote}

Let us play around with this for a while. To say ``You will get a book from me who is going to town,'' in P\=ali we can put it like this:

\palisample{tva\d m may\=a nagara\d m gacchantasm\=a potthaka\d m labhissasi.}

If you ask why ablative case is used here, you need a big review of the early lessons. As the paradigm tells us, \pali{gacchant\=a} or \pali{gacchat\=a} can do the job as well. In this example, you may realize that in fact word order in P\=ali is not entirely arbitrary. Certain placement is required so that a proper meaning can be rendered. However, you can rearrange the sentence to ``\pali{tva\d m potthaka\d m labhissasi may\=a nagara\d m gacchantasm\=a}.'' Even, I think, ``\pali{tva\d m may\=a potthaka\d m labhissasi nagara\d m gacchantasm\=a}'' is fine. But when you break \pali{nagara\d m} from \pali{gacchantasm\=a}, it becomes clueless.

Here is another example, ``You give money to me who is going to town.'' We can render this as follows:

\palisample{tva\d m mayha\d m nagara\d m gacchantassa m\=ula\d m dad\=asi.}

Let us keep this example in mind for a while.

Now I move to another aspect of present participles. In P\=ali it can be used to construct relative clauses that express a simultaneous action, like we mark a clause with `when' or `while.' Here is the principle. When we talk about a relative action which occurs at the same time with the main action, we can use \emph{absolute construction} in both \emph{genitive} form or \emph{locative} form to mark the relative clause.\footnote{Kacc\,305, R\=upa\,323, Sadd\,633, Mogg\,2.35. Accusative absolute can also be found, but very rarely.} For more information, see Chapter \ref{chap:cases} to find out what all cases can do, including absolute construction. Here is a guideline of how to compose a relative clause.

\begin{enumerate}
\item Specify the subject and verb of the relative clause to be composed.
\item For the verb, apply \pali{anta} or \pali{m\=ana} to it.
\item Apply genitive case or locative case to the subject and the verb of relative clause. Retain the case of other components of the clause, if any.
\item Adding this clause to the main sentence in a proper position.
\end{enumerate}

For example, if I want to say ``When I am going to town, you give me money,'' I can put it in this way:

\palisample{mayha\d m nagara\d m gacchantassa, tva\d m me m\=ula\d m dad\=asi. \sampleor mayi nagara\d m gacchantasmi\d m, \ldots}

Now let us go back to the example you have just kept in mind. You can see that the structure of that sentence and this example (the first one) looks very similar. But they are not the same. In that example, as modifier the case is dative. In this example, as in relative clause the case is genitive. They just happen to look alike. To clarify a little more, in ``\pali{mayha\d m nagara\d m gacchantassa, tva\d m me m\=ula\d m dad\=asi},'' \pali{mayha\d m} is in gen.\ but \pali{me} is dat. Although, in principle they can be identical, it is better to make them look different.

Here is another example to strengthen your understanding. To say ``You give me money, while I am sitting in a car,'' we can put it like this:

\palisample{tva\d m me m\=ula\d m dad\=asi, mayha\d m rathe nis\=idam\=anassa. \sampleor \ldots, mayi rathe nis\=idam\=ane.}

For comparison, this is for ``You give money to me who is sitting in a car.''

\palisample{tva\d m mayha\d m rathe nis\=idam\=anassa m\=ula\d m dad\=asi.}

Here are some examples of absolute construction as relative clauses used in the canon:

\begin{quote}
\pali{Buddhassa gaccham\=anassa, duss\=a dh\=avanti pacchato}\footnote{Apad\=a 17.40}\\
``While the Buddha is going, the clothes are blown from [his] back.''\\[1.5mm]
\pali{Ya\d m j\=ata\d m ta\d m sa\.nghamajjhe pucchante santa\d m atth\=i'ti vattabba\d m}\footnote{Mv\,1.126. For verbs in \pali{tabba} form, see Chapter \ref{chap:pass}.}\\
``When [they] ask among the Sangha about which thing that arose, [if] that exists `\pali{atthi}' should be said.''\\[1.5mm]
\pali{Atha kho tassa bhikkhuno g\=amak\=a kosambi\d m gacchantassa antar\=amagge nadi\d m tarantassa s\=ukarik\=ana\d m hatthato mutt\=a medava\d t\d ti p\=ade lagg\=a hoti.}\footnote{Buv1\,160}\\
``When that monk is going from a village to Kosamb\=i, on the way when he is crossing the river, there is a lump of fat, fallen from a pig-killer's hand, stuck to [his] foot.''\\[1.5mm]
\end{quote}

\section*{Exercise \ref{chap:prp}}
Say these in P\=ali.
\begin{compactenum}
\item Madam, what was you doing when the thief broke into your house?
\item I was sleeping upstairs when the thief came in, officer.
\item As you know now, what is lost?
\item I think, let me see, it is not obvious. When I came down in the morning, I found the front door was opened, as well as my refrigerator.
\item Maybe he is hungry or something.
\item That's ridiculous. I will not break into someone's house, when I just want something to eat.
\item Maybe someone you know. Where's your husband when the incident occurred?
\item He told me he had to work all night and he would not come home. If it is him why did he leave the door opened? It must be a thief.
\item (Another officer) Madam, we find a man, looking like your husband, drunk, sleeping in the garage.
\item (The first officer) This [information] explains all these thing.
\end{compactenum}
