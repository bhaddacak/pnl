\chapter{I \headhl{will} go to school}\label{chap:fut}

\phantomsection
\addcontentsline{toc}{section}{Future Tense}
\section*{Future Tense}

To the lesson concerning past tense, future tense is a big relief. It is far more easy to deal with, very much like present tense. When you know the rule you can apply it widely with very few variations. So, I reproduce the endings of future tense in Table \ref{tab:conjfut}. Traditionally, this tense is called \pali{Bhavissanti} ([They] will be).

\begin{table}[!hbt]
\centering
\caption{Endings of future tense conjugation}
\label{tab:conjfut}
\bigskip
\begin{tabular}{l*{2}{>{\itshape}l}} \toprule
\bfseries Person & \bfseries\upshape Singular & \bfseries\upshape Plural \\ \midrule
3rd & ssati & ssanti \\
2nd & ssasi & ssatha \\
1st & ss\=ami & ss\=ama \\
\bottomrule
\end{tabular}
\end{table}

To use these endings, you have to extract verb stem from its dictionary form (see Chapter \ref{chap:verb-go}), remove the ending vowel to get the bare stem, add \pali{i}\footnote{Kacc\,516, R\=upa\,466, Sadd\,1030, Mogg\,6.35, Niru\,588.} and annex it with the endings.  For \pali{gacchati} we normally use \pali{gam} as stem, but \pali{gacch} is still found in the texts.\footnote{For maximum cases as shown in \textsc{P\=ali\,Platform} 3 using CSTR, 240 occurrences are found for \pali{gamiss\=ami}, 24 for \pali{gacchissanti}.} Thus, to say ``I will go to school'' we simply go like this:

\palisample{(aha\d m) p\=a\d thas\=ala\d m gamiss\=ami/gacchiss\=ami.\sampleor[and ``You go to school''](tva\d m) p\=a\d thas\=ala\d m gamissasi/gacchissasi.\sampleor[and ``He/She goes to school''](so/s\=a) p\=a\d thas\=ala\d m gamissati/gacchissati.}

However, there are some verbs that have slightly different rendition. I list them in Table \ref{tab:futirr}. So, it is worth remembering these.

\bigskip
\begin{longtable}[c]{@{}>{\raggedright\arraybackslash}p{0.2\linewidth}%
	>{\raggedright\arraybackslash}p{0.14\linewidth}%
	>{\raggedright\arraybackslash\itshape}p{0.24\linewidth}%
	>{\raggedright\arraybackslash\itshape}p{0.27\linewidth}@{}}
\caption{Some irregular future verb forms}\label{tab:futirr}\\
\toprule
\bfseries Verb &\bfseries Person & \bfseries\upshape Singular & \bfseries\upshape Plural \\ \midrule
\endfirsthead
\multicolumn{4}{c}{\tablename\ \thetable: Some irregular future verb forms (contd\ldots)}\\
\toprule
\bfseries Verb &\bfseries Person & \bfseries\upshape Singular & \bfseries\upshape Plural \\ \midrule
\endhead
\bottomrule
\ltblcontinuedbreak{4}
\endfoot
\bottomrule
\endlastfoot
%%
\pali{dad\=ati/deti} & 3rd & dassati & dassanti \\*
(to give) & 2nd & dassasi & dassatha \\*
(R\=upa\,508) & 1st & dass\=ami & dass\=ama \\
\midrule
\pali{sakkoti} & 3rd & sakkhissati & sakkhissanti \\*
(to be able) & 2nd & sakkhissasi & sakkhissatha \\*
(R\=upa\,512) & 1st & sakkhiss\=ami & sakkhiss\=ama \\
\midrule
\pali{karoti}\footnote{It is more common to use \pali{karissati}, etc.} & 3rd & k\=ahati, k\=ahiti & k\=ahanti, k\=ahinti \\*
(to do) & 2nd & \mbox{k\=ahasi, k\=ahisi} & \mbox{k\=ahatha, k\=ahitha} \\*
(R\=upa\,524) & 1st & \mbox{k\=ah\=ami, k\=ahimi} & \mbox{k\=ah\=ama, k\=ahima} \\
\midrule
\pali{labhati}\footnote{It is more common to use \pali{labhissati}, etc.} & 3rd & lacchati & lacchanti \\*
(to get) & 2nd & lacchasi & lacchatha \\*
(R\=upa\,477) & 1st & lacch\=ami & lacch\=ama \\
\midrule
\pali{su\d n\=ati}\footnote{It is more common to use \pali{su\d nissati}, etc.} & 3rd & sossati & sossanti \\*
(to listen) & 2nd & sossasi & sossatha \\*
(R\=upa\,512) & 1st & soss\=ami & soss\=ama \\
\end{longtable}

Apart from speculating on events in the future, \pali{Bhavissanti} also has a few other uses. With \pali{katha\~nhi n\=ama}, it can refer to an action in the past\footnote{Sadd\,893}, often as a rebuke. In this sense, \pali{katha\~nhi n\=ama} means ``for such a reason?'' rather than a straight question, ``why?'' or ``for what reason?'' Here is an example from the scriptures.

\begin{quote}
\pali{katha\~nhi n\=ama tva\d m, moghapurisa, eva\d m sv\=akkh\=ate dhammavinaye udarassa k\=ara\d n\=a pabbajissasi.}\footnote{Mv\,1.73}\\
``For such a reason, useless man, you (will go) went forth for stomach's reason into this well-preached teaching?''
\end{quote}

The structure of the sentence above is a kind of stock phrases. It is often used when the Buddha gives admonitions to monks. In the example, `stomach' (\pali{udara}) is a metonym representing ``making a living.''

Another use is to insult or make a doubtful or sarcastic or ridiculous remark, for example:

\begin{quote}
\pali{acchariya\d m andho n\=ama pabbatam\=arohissati, badhiro n\=ama sadda\d m sossati.}\footnote{Mogg\,6.3}\\
``Amazing!, [one] called blind will climb the mountain, [one] called deaf will listen to the sound.''
\end{quote}

Now it is the time for practicing.

\section*{Exercise \ref{chap:fut}}
Say these in P\=ali.
\begin{compactenum}
\item Where will you go tomorrow?
\item I will buy new clothes at the market tomorrow.
\item You had a lot. What will you get those for?
\item I will give them to my sister. She wanted new clothes, but she has no time for shopping.
\item Will your sister like them? (Will the clothes satisfy your sister?)
\item Yes, we dress in the same way. She will put them on.
\end{compactenum}
