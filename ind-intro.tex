\chapter{I \headhl{and} you do \headhl{not} go to school}\label{chap:ind-intro}

\phantomsection
\addcontentsline{toc}{section}{Introduction to Indeclinables}
\section*{Introduction to Indeclinables}

It is a proper time to introduce P\=ali indeclinables now. By linguistic definition, this class of words can be called \emph{particle}.\footnote{``[A]ny uninflected word or word that does not change its form'' \citep[p.~332]{brownmiller:dict}.} In P\=ali we have roughly two classes of this category: \pali{upasagga} (prefixes), and \pali{nip\=ata} (particles). We already have met a few of \pali{upasagga}s, e.g.\ \pali{\=a} (near to) in \pali{\=agacchati} (to go near to = to come). You can learn more about \pali{upasagga} in Appendix \ref{chap:upasagga}. For particles, in modern English grammar's terms, many of them work very much like adverbs. We will learn more about adverbs in Chapter \ref{chap:adv}. In this chapter I will introduce you to the world of particles and to meet the top-five. Particles in P\=ali are numerous, if not countless for we can create some form of them at will. You can find the full account of particles in Appendix \ref{chap:nipata}.

Indeclinables can be formed in a few ways. First, they can be individual words that are always used in the same form, e.g.\ \pali{ca, v\=a, iti}, etc. These terms can be found normally in dictionaries. Second, they can be composed from certain nouns and pronouns with particular suffixes. When composed, they stay unchanged in all their life. And third, they can be inflected terms that are used in an idiomatic way all the time, so they look as if they are immutable, even though they are inflected once. For instance, \pali{op\=ayika\d m} and \pali{patir\=apa\d m}, both mean `proper' or	`suitable' or something like ``that's right.'' In this chapter we will talk only about some of the first group. We will talk about the second group in Chapter \ref{chap:ind-to}, and some of the third group can be found in Appendix \ref{chap:nipata}.

Are you curious about what the most frequent P\=ali term is? In the past, it was impossible to count individual terms in the whole collection, but nowadays it is just a mouse click in a suitable software. I reproduce the result of the top-five of P\=ali terms in the whole corpus of Cha\d t\d tha Sa\.ng\=ayana Tipi\d taka Restructured counted by \textsc{P\=ali\,Platform} 3 in Table \ref{tab:topfive}.\footnote{Even though the numbers are actually counted, they are a close approximation at best. For several reasons, the exact occurrence count is impossible in the collection we have.}

\begin{table}[!hbt]
\centering\small
\caption{Top-five of the most frequent P\=ali terms}
\label{tab:topfive}
\bigskip
\begin{tabular}{>{\itshape}l*{4}r} \toprule
\upshape Term & Total Freq & G\=ath\=a Freq & Length \\
\midrule
ti & 203,632 & 11,840 & 2 \\
ca & 147,637 & 18,472 & 2 \\
na & 141,210 & 10,336 & 2 \\
v\=a & 109,133 & 3,543 & 2 \\
pana & 70,180 & 1,998 & 4 \\
\bottomrule
\end{tabular}
\end{table}

From the table you can see that all the top-five are particles. Because of their common use, particles are therefore important. But it is not so urgent to know them earlier, because several of them add nothing to the meaning. Before I go to each word, it is better to know indeclinables in principle first. 

From the start, I am reluctant to introduce grammatical terms used by the tradition, for they tend to be confusing and distracting to new students rather than illuminating. By this reason, I thus use Western grammatical explanations to help students be familiar with the language first. But at some point when we go deeper, Western grammatical terminology seems unable to capture all of the traditional mentality. We have to return to the traditional terminology eventually. However, I have to admit that in some respect they are too many and irrelevant. So, I have to compromise here by piecemeal introducing you the traditional terms when they are really necessary. Knowing grammatical terms is essential in the case that you study the traditional textbooks by yourselves. That is one of the objectives of my writing this book. If you are very new to the traditional P\=ali textbooks, you should take a look at Appendix \ref{chap:textbook} before you go further. The following explanation is heavily theoretical. This will prepare the readers to the tone of the coming lessons.

Following Saddan\=iti, the most fundamental unit of P\=ali language is \pali{sadda} (sound, noise). Aggava\d msa's first formula is this:

\begin{quote} 
Sadd\,1: \pali{Appabhutekat\=al\=isa sadd\=a va\d n\d n\=a.}\footnote{\citealp[p.~604]{smith:sadd3}}\\
``Beginning with \pali{a}, 41 sounds [are] \pali{va\d n\d na} (letters)''
\end{quote}
 
That is to say, at alphabet level, they are \pali{sadda}.\footnote{I try to think this in terms of \emph{phoneme}, but it does not really fit.} Also when they form a combination but not yet get any specific meaning, only certain potential, they are \pali{sadda}\footnote{This can be called \pali{li\.nga}, according to Sadd\,196. But Sadd\,192 seems to imply that \pali{li\.nga} indeed has meaning for it is composed with \pali{vibhatti}. Furthermore, in Sadd\,197 Aggava\d msa adds that also \pali{upasagga} and \pali{nip\=ata} are \pali{li\.nga}. All these accounts render \pali{li\.nga} as a problematic term. We usually use it to mean `gender,' but it turns to mean many things. So, I suggest we avoid using this term altogether.}, for example, \pali{purisa}, \pali{satthu}, and \pali{ka\~n\~n\=a}. Traditionally, these are called \pali{purisasadda} (Sadd-Pad Ch.\,5), \pali{satthusadda} (Sadd-Pad Ch.\,6), and \pali{ka\~n\~n\=asadda} (Sadd-Pad Ch.\,8). They are just sounds, albeit complex, but they have no specific meaning yet because they are not composed in a sentence, no relation to other sounds. However, \pali{sadda}s have categories, some become nouns, some become verbs (i.e.\ roots), some have other functions. Those that help others \pali{sadda} form a word unit are called \pali{paccaya}.

\begin{quote}
\pali{Ye r\=upanipphattiy\=a upak\=arak\=a atthavisesassa jotak\=a v\=a ajotak\=a v\=a lopan\=iy\=a v\=a alopan\=ay\=a v\=a, te sadd\=a paccay\=a.}\footnote{Sadd-Pad Ch.\,1; \citealp[p.~3]{smith:sadd1}}\\
``Sounds that are helpful to word formation, illuminating distinct meaning or not, elided or not, [are] \pali{paccaya}s.''
\end{quote}

We can see \pali{paccaya} as suffixes in general. Learning \pali{paccaya}s is the main approach to all traditional schools of P\=ali grammar. So, they are really important. But so far we did not follow that path, at least not yet. One \pali{paccaya} may not make a sound completely meaningful. In verb formation, for instance, it has to use with others in combination. A subset of \pali{paccaya} we have met before that make nouns and verbs meaningful is \emph{inflectional suffixes} (see Chapter \ref{chap:nom}). Precisely, for nouns we call \emph{declensional suffixes} which mark cases, and for verbs \emph{conjugational suffixes} which mark tenses and moods. Grammatically, these are called \pali{vibhatti} (division, classification) in P\=ali.

\begin{quote}
Sadd\,198: \pali{Sy\=adayo ty\=adayo ca vibhattiyo.}\footnote{\citealp[p.~641]{smith:sadd3}}\\
``Suchlike \pali{si} and suchlike \pali{ti} [are] \pali{vibhatti}.''\\[1.5mm]
Sadd\,199: \pali{Sy\=adayo n\=ame, ty\=adayo \=akhy\=ate.}\footnote{\citealp[p.~642]{smith:sadd3}}\\
``Suchlike \pali{si} [is used] in nouns, suchlike \pali{ti} [is used] in verbs.''
\end{quote}

We did not talk about \pali{si}, the sign of singular nominative case (\pali{pa\d tham\=avibhatti}), but we have already done a lot on \pali{ti} (\pali{vattam\=an\=avibhatti}) as we use \pali{hoti, bhavati}, or \pali{gacchati}. Even \pali{atthi} also has something to do with \pali{ti}, but in an irregular way. It is safe to put in this way: \pali{si} and \pali{ti} represent distinct formation processes. In most cases we can recognize which process is operated by seeing their name as the sign, e.g.\ \pali{ti}. But many are difficult to detect. That is the reason why we have never seen \pali{si}, even though it is always in process when we use singular nominative case. And this is the very reason I did not follow traditional approach at the beginning. It is really confusing when you say you use a \pali{vibhatti/paccaya} and then you delete it so that it can not be seen, or it causes certain transformation so that the word looks like a new one, or it undergoes certain process but the word stays the same.

When a \pali{sadda} is operated under a \pali{vibhatti/paccaya} process, finally it becomes a meaningful term. Normally we call this term \pali{pada}. Aggava\d msa puts it in this way:

\begin{quote}
Sadd\,27: \pali{Vibhatyantamavibhatyanta\d m v\=a atthajotaka\d m pada\d m.}\footnote{\citealp[p.~610]{smith:sadd3}. In R\=upa\,11, there is ``\pali{Vibhatyanta\d m pada\d m}.''}\\
``Illuminating meaning, [term] with \pali{vibhatti} or without \pali{vibhatti} [is] \pali{pada}.'' 
\end{quote}

I think by term without \pali{vibhatti} here Aggava\d msa means particles. But as we shall see below, he is somewhat inconsistent. Distinction between \pali{sadda} and \pali{pada} seems blurred when he uses \pali{atthiy\=anatthiy\=asadd\=ana\d m} (of \pali{atthiy\=a} and \pali{natthiy\=a} sounds) in Sadd-Pad Ch.\,13. These should be \pali{pada} not \pali{sadda} in our definition.\footnote{It seems that to Aggava\d msa, \pali{sadda} means anything uttered, meaningful or not. It means sound or word in general, so to speak.} I suggest that we should stick with the notion that \pali{pada} has meaning whereas \pali{sadda} has not (yet). This use is technical to this context only. Both terms can have other specific meaning in other contexts.

I give you some examples here: \pali{purisa} + \pali{si} = \pali{puriso} (a man), \pali{satthu} + \pali{si} = \pali{satth\=a} (a teacher), \pali{ka\~n\~n\=a} + \pali{si} = \pali{ka\~n\~n\=a} (a girl), \pali{gam} + \pali{a} + \pali{ti} = \pali{gacchati}. In the first instance, \pali{purisa} is \pali{sadda}, \pali{si} is \pali{vibhatti}, and \pali{puriso} is \pali{pada}. Only \pali{puriso} has meaning because it has a sign of present nominative case which make it the subject of a sentence. We can write the general formula of this as:

\begingroup
\large
\medskip
(\pali{upasagga})+\pali{sadda}+\pali{paccaya}(s)+\pali{vibhatti} = \pali{pada}
\medskip
\endgroup

For verbs, \pali{sadda} is their root, whereas for nouns it is \pali{n\=amasadda} like \pali{purisa} mentioned above. Operating under multiple \pali{paccaya}s can be the case, particularly in verb formation. A \pali{vibhatti} has to be present to finalize the term. \pali{Upasagga} is optional, but the main \pali{sadda} and \pali{paccaya/vibhatti} are always present.\footnote{In rare cases, certain nouns are used without \pali{vibhatti} (see \pali{Avibhattikaniddeso} in Sadd-Pad Ch.\,2). I see this as an anomaly (perhaps, typo or memory lapse), whereas Aggava\d msa sees that everything in the canon is from the Buddha (\pali{tath\=agatamukhato}), so he thinks it has to be a reason of that.} In a sentence we see only \pali{pada} because it is ready for certain meaning due to relation to other \pali{pada} marked by \pali{vibhatti}. This formula is the basis of all classes of word formation in P\=ali, including particles!

You may protest that by definition particles or indeclinables do not undergo any process that changes their form, and Aggava\d msa himself maintains that meaningful term can be present without \pali{vibhatti}. That is to say, \pali{sadda} and \pali{pada} of particles are the same. We just use them as they are. However, that is not the way the tradition sees them. In grammarians' point of view, including Aggava\d msa himself, every \pali{sadda}\footnote{Except the \pali{paccaya} itself, otherwise it will be an endless recursion.} has to be processed, but the process can be invisible to us. The \pali{paccaya} used can be elided as mentioned in the excerpt concerning \pali{paccaya} above.\footnote{According to Kacc\,221, ``\pali{Sabb\=asm\=avusopasagganip\=at\=ad\=ihi ca},'' R\=upa\,282 explains that: ``\pali{\=Avusosaddato, upasagganip\=atehi ca sabb\=asa\d m par\=asa\d m vibhatt\=ina\d m lopo hoti}'' (There is elision of all ending \pali{vibhatti} from \pali{\=avuso} and prefixes and particles). See also Mogg\,2.118, Niru\,288, Sadd\,448.}

We can wrap up the point in this way. Given by Western scholars, `indeclinables' is misnomer in P\=ali grammarians' view. It might be better to call them `unchangeables' because they do decline but invisibly or stay the same.\footnote{It is even not exactly the case to say as such. In some rare cases, you can see inflected particles. When Aggava\d msa discusses about \pali{atthi-natthi} as particle or \pali{nip\=ata} (Sadd-Pad Ch.\,13), he also raises the issue that because \pali{atthiy\=a} and \pali{natthiy\=a} used in loc.\ can be found in the Abhidhammapi\d taka, they can decline into other cases as well (\pali{Iti atthiy\=anatthiy\=asadd\=ana\d m sattamyantabh\=ave siddheyeva tatiy\=acatutth\=ipa\~ncam\=acha\d t\d thiyantabh\=avopi siddhoyeva hoti}, \citealp[p.~300]{smith:sadd1}). However, you can argue that these two terms are used as a noun, so they decline to achieve their intended meaning. However, Aggava\d msa does not say they are noun.} I will never use this term, and continue using `indeclinables' for familiarity reason. Another point worth noting is all indeclinables can really be changed when joining (\pali{sandhi}) with other terms as you can see in Appendix \ref{chap:sandhi}. 

Does this sound a kind of nonsense to you? I suggest that we should not take this issue seriously. I think Aggava\d msa himself also sees this as a trivial matter. At the end, we just use particles uninflected. However, this discussion reminds us that in some case, when you read texts, you may encounter oddities.

\bigskip
Let us go back to the terms listed in the table at the beginning of the chapter. They are \pali{ti} (elided form of \pali{iti}), \pali{ca}, \pali{na}, \pali{v\=a}, and \pali{pana}.

\paragraph*{\fbox{Iti}} Throughout P\=ali scriptures \pali{iti} is used extensively. It is used mainly to denote direct speech, like we use quotation marks in English. So it is normally found with verbs expressing certain content, e.g.\ \pali{vadati} (say), \pali{pucchati} (ask), or \pali{cinteti} (think), for example, ``\pali{kasmi\d m gacchasi iti pucch\=ami}'' (I ask, ``where are you going?''). In most case, \pali{iti} will join (\pali{sandhi}) with the preceding word, so we normally put the sentence in this way: ``\pali{kasmi\d m gacchas\=iti pucch\=ami}.'' Redactors of the scriptures help us identify \pali{iti} by separating it like \pali{gacchas\=i'ti}, so we can detect it quite easily in modern text collections. And this explains why \pali{ti} is mostly found not \pali{iti}. There are many things to learn about direct speech (see Chapter \ref{chap:iti}). And you can learn word joining \pali{sandhi} in Appendix \ref{chap:sandhi}.

The remaining four nicely fit the traditional definition of particle (\pali{nip\=ata}). In Nep\=atikapada toward the end of N\=amaka\d n\d da (the 2nd chapter) of R\=upasiddhi there is an explanation: 

\begin{quote}
\pali{Samuccayavikappanapa\d tisedhap\=ura\d n\=adiattha\d m \\asatvav\=acaka\d m nep\=atika\d m pada\d m.}\footnote{Exactly the same wording is found in Sadd-Sut Ch.\,27 \citep[p.~886]{smith:sadd3}.} 
\end{quote}

This can be translated as: ``not denoting things (\pali{asatvav\=acaka\d m}), term denoting suchlike conjunction (\pali{samuccaya}), disjunction (\pali{vikappana}), negation (\pali{pa\d tisedha}), and filling (\pali{p\=ura\d na}) is particle (\pali{nep\=atika\d m pada\d m}).''\footnote{See also \citealp[pp.~121--2]{collins:grammar}.}

The last four particles in our list are the typical representatives of the four classes mentioned above.

\paragraph*{\fbox{Ca}} This is a conjuction particle meaning `and.' We can use this in various ways, for example, ``\pali{d\=arako \textbf{ca} d\=arik\=a \textbf{ca} k\=i\d lanti}'' (a boy and a girl play). This can also be put as ``\pali{d\=arako \textbf{ca} d\=arik\=a k\=i\d lanti}'' or ``\pali{d\=arako d\=arik\=a \textbf{ca} k\=i\d lanti}.'' Like English, when two subjects are connected with `and' the verb of the sentence is plural. Then we can say ``A boy and a girl play and laugh'' as follows:

\palisample{d\=arako ca d\=arik\=a k\=i\d lanti hasanti ca.}

\paragraph{\fbox{V\=a}} This is a disjunction particle meaning `or.' Like \pali{ca}, it can be used as ``\pali{d\=arako \textbf{v\=a} d\=arik\=a \textbf{v\=a} k\=i\d lati}'' (a boy or a girl plays) or ``\pali{d\=arako \textbf{v\=a} d\=arik\=a k\=i\d lati}'' or ``\pali{d\=arako d\=arik\=a \textbf{v\=a} k\=i\d lati}.'' The verb now has to be singular. In some context, \pali{v\=a} means inclusive or---both alternatives are included. So, sometimes it sounds like \pali{ca}. For example, ``\pali{puris\=a v\=a itth\=i v\=a maranti}'' means ``Men or women (all) die.''

\paragraph{\fbox{Na}} This is a negation particle meaning `not.' Normally, it is placed before the term to be negated. It can also be placed at the beginning to negate the whole sentence. For ``a boy not girl plays not laughs,'' we can say as follows:

\palisample{d\=arako na d\=arik\=a k\=i\d lati na hasati.}

\paragraph{\fbox{Pana}} This is a filler particle meaning nothing substantially. It is often used to connect or continue the story. It sounds like ``and, and now, further, moreover.'' You can say ``A boy and a girl play. And they also laugh'' as:

\palisample{d\=arako ca d\=arik\=a k\=i\d lanti. te pana hasanti ca.}

Try saying the sentence yourself with and without \pali{pana}. You will find that it sounds better with a filler. That is why apart from \pali{pana} P\=ali also has a lot of fillers, around two dozens. In the past these fillers might have particular functions like we use \emph{discourse markers} nowadays.

Another use of \pali{pana} is in contrasting. It means like `but' (\pali{ca} also has this use in some context), for example:

\begin{quote}
\pali{Sudassa\d m vajjama\~n\~nesa\d m, attano pana duddasa\d m}\footnote{Dhp\,18.252}\\
``Others' fault is easily seen, but one's own [fault] is hard to see.'' \\
\end{quote}

You can learn more about particles in Appendix \ref{chap:nipata}, also some of them in Chapter \ref{chap:ind-to}.

Now we will finish our task of this chapter. To say ``I and you do not go to school,'' we can put it like this:

\palisample{aha\d m ca tva\d m p\=a\d thas\=ala\d m na gacchatha.}

In the case of you might curious, as stated in Kacc\,409, R\=upa\,441, Sadd\,868, and Mogg\,1.22\footnote{See also Payo\,505 and Niru\,563.}, when multiple subjects do the same action, the verb agrees with the last one but in plural form. When you swap the subjects, you use different verb form. Hence, ``\pali{tva\d m aha\d m ca p\=a\d thas\=ala\d m na gacch\=ama.}'' It is quite counterintuitive because ``I and you'' has the sense of ``we.'' So, first person plural should be expected. You definitely can follow your intuition in your conversations, but be aware of this when you read texts.\footnote{See also a discussion of this issue in Chapter \ref{chap:vclass}, page \pageref{par:multiactors}.} We can use \pali{saddhi\d m} with ins.\ (see Chapter \ref{chap:ins}) to avoid this situation. Thus, we rephrase the sentence as:

\palisample{aha\d m tay\=a saddhi\d m p\=a\d thas\=ala\d m na gacch\=ami.}

Now the verb has to agree only with \pali{aha\d m}.

\section*{Exercise \ref{chap:ind-intro}}
Say these in P\=ali.
\begin{compactenum}
\item I ask that girl, ``What's your name?''
\item Our town has a factory and banks, but has no hospital and theater.
\item I do not find my phone, either a thief takes it or it is lost.
\item A teacher either goes to school with children by bus, or with a friend by car.
\item Either the cat or the dog breaks this bottle, not I and you or the children.
\end{compactenum}
