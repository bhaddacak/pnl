\chapter[There are 7.8 billion people]{There are 7.8 billion people in the world}\label{chap:num}

\phantomsection
\addcontentsline{toc}{section}{Introduction to Numerals}
\section*{Introduction to Numerals}

In traditional textbooks, I hardly find a satisfactory explanation on numerals. Even though Aggava\d msa wrote a sophisticated treatment of the topic in Sadd-Pad Ch.\,13, it is still not comprehensive enough.\footnote{In the chapter, Aggava\d msa also spends a lot of space to discuss \pali{atthi-natthi} in detail, irrelevantly to the topic.} Mostly, textbooks teach us how the terms decline, but much less on how to use them. We can say that the main purpose of the traditional textbooks is to read the P\=ali texts. We just learn to recognize numeric terms. After that, it is supposed to be easy and straightforward. However, when we learn to speak the language, I found that materials provided by the tradition is scanty, not enough to help us gain fully understanding of the subject. Here, I try my best to fulfill this gap.

In Chapter \ref{chap:pron-misc} we have learned that number 1--4 are used as pronouns. But in P\=ali the line between pronouns and adjectives is really thin, or nearly invisible in my view. So, it is not a big different when we use numbers as pronouns or pronominal adjectives.

To help you see a big picture of P\=ali numerals first, I list all useful numbers in Table \ref{tab:num}.

\bigskip
\begin{longtable}[c]{@{}>{\itshape\raggedright\arraybackslash}p{0.6\linewidth}>{\raggedright\arraybackslash}p{0.17\linewidth}@{}}
\caption{P\=ali numbers}\label{tab:num}\\
\toprule
\bfseries\upshape P\=ali& \bfseries\upshape Number\\ \midrule
\endfirsthead
\multicolumn{2}{c}{\tablename\ \thetable: P\=ali numbers (contd\ldots)}\\
\toprule
\bfseries\upshape P\=ali& \bfseries\upshape Number\\ \midrule
\endhead
\bottomrule
\ltblcontinuedbreak{2}
\endfoot
\bottomrule
\endlastfoot
%%
eka & 1 \\
dvi & 2 \\
ti & 3 \\
catu & 4 \\
pa\~nca & 5 \\
cha & 6 \\
satta & 7 \\
a\d t\d tha & 8 \\
nava & 9 \\
dasa & 10 \\
ek\=adasa, ek\=arasa & 11 \\
dv\=adasa, b\=arasa & 12 \\
terasa, te\d lasa & 13 \\
catuddasa, cuddasa, coddasa & 14 \\
pa\~ncadasa, pa\d n\d narasa, pannarasa & 15 \\
so\d lasa, sorasa & 16 \\
sattarasa, sattadasa & 17 \\
a\d t\d th\=arasa, a\d t\d th\=adasa & 18 \\
ek\=unav\=isati, \=unav\=isa & 19 \\
v\=isa, v\=isa\d m, v\=isati & 20 \\
ekav\=isati & 21 \\
dv\=av\=isati, b\=av\=isati & 22 \\
tev\=isati & 23 \\
catuv\=isati & 24 \\
pa\~ncav\=isati & 25 \\
chabb\=isati & 26 \\
sattav\=isati & 27 \\
a\d t\d thav\=isati & 28 \\
ek\=unati\d msa, \=unati\d msa & 29 \\
ti\d msa, ti\d msati, ti\d msa\d m & 30 \\
ekatti\d msa & 31 \\
dvatti\d msa, b\=atti\d msa & 32 \\
tetti\d msa & 33 \\
catutti\d msa & 34 \\
pa\~ncatti\d msa & 35 \\
chatti\d msa & 36 \\
sattatti\d msa & 37 \\
a\d t\d thatti\d msa & 38 \\
ek\=unacatt\=a\d l\=isa, \=unacatt\=a\d l\=isa & 39 \\
catt\=a\d l\=isa, catt\=al\=isa, catt\=ar\=isa, t\=al\=isa & 40 \\
ekacatt\=a\d l\=isa & 41 \\
dvecatt\=a\d l\=isa & 42 \\
tecatt\=a\d l\=isa & 43 \\
catucatt\=a\d l\=isa & 44 \\
pa\~ncacatt\=a\d l\=isa & 45 \\
chacatt\=a\d l\=isa & 46 \\
sattacatt\=a\d l\=isa & 47 \\
a\d t\d thacatt\=a\d l\=isa & 48 \\
ek\=unapa\~n\~n\=asa, \=unapa\~n\~n\=asa & 49 \\
pa\~n\~n\=asa, pa\d n\d n\=asa, pa\~n\~n\=asa\d m & 50 \\
sa\d t\d thi & 60 \\
sattati & 70 \\
as\=iti & 80 \\
catur\=as\=iti & 84 \\
navuti & 90 \\
ek\=unasata\d m & 99 \\
sata\d m\footnote{For the multification of 10, 100, and 1000, see Kacc\,393--4, R\=upa\,415--6, Sadd\,832--3.} & 100 \\
ek\=unadvisata\d m & 199 \\
dvisata\d m & 200 \\
tisata\d m & 300 \\
ek\=unanavasata\d m & 899 \\
ek\=unasahassa\d m & 999 \\
sahassa\d m & 1,000 \\
dvisahassa\d m & 2,000 \\
tisahassa\d m & 3,000 \\
dasasahassa\d m, nahuta\d m & 10,000 \\
satasahassa\d m, lakkha\d m & 100,000 \\
dasasatasahassa\d m & 1,000,000 \\
ko\d ti\footnote{For the huge numbers, see Kacc\,395, R\=upa\,417, Sadd\,833, Abhidh\=a 474--6.} & 10\textsuperscript{7} \\
pako\d ti & 10\textsuperscript{14} \\
ko\d tipako\d ti & 10\textsuperscript{21} \\
nahuta & 10\textsuperscript{28} \\
ninnahuta & 10\textsuperscript{35} \\
akkhobhin\=i & 10\textsuperscript{42} \\
bindu & 10\textsuperscript{49} \\
abbuda & 10\textsuperscript{56} \\
nirabbuda\footnote{There is a discrepancy here. In Sadd\,833, it is said that in the canon and commentaries \pali{nirabbuda} equals 20 times \pali{abbuda}, and this multiplication goes on towards the end of the list.} & 10\textsuperscript{63} \\
ahaha & 10\textsuperscript{70} \\
ababa & 10\textsuperscript{77} \\
a\d ta\d ta & 10\textsuperscript{84} \\
sogandhika & 10\textsuperscript{91} \\
uppala & 10\textsuperscript{98} \\
kumuda & 10\textsuperscript{105} \\
pu\d n\d dar\=ika & 10\textsuperscript{112} \\
paduma & 10\textsuperscript{119} \\
kath\=ana & 10\textsuperscript{126} \\
mah\=akath\=ana & 10\textsuperscript{133} \\
asa\.nkheyya & 10\textsuperscript{140} \\
\end{longtable}

As you have seen, the formation of number under 99, except the peculiar numbers ending with 9, is in reversed order comparing to English. P\=ali puts the least digit first. Several numbers under 40 have irregular combinations, so these are worth remembering. Numbers over 40 follow recognizable patterns. It is not necessary to list them all. Numbers ending with 9 have no specific name. For them, \pali{\=una} (less, minus) or \pali{ek\=una} (minus one) is used with the successive decade. For example, \pali{ek\=unav\=isa} literally means $20-1$, hence $19$.

Some forms of numbers undergo slight changes. For example, \pali{cha} becomes \pali{so} in \pali{so\d lasa}\footnote{Kacc\,376, R\=upa\,257, 806, Mogg\,3.101}; \pali{-ti} can be added to \pali{v\=isa} and \pali{ti\d msa}\footnote{Kacc\,378, R\=upa\,414, Sadd\,808}; \pali{da} in \pali{dasa} can become \pali{ra, \d la,} or \pali{la}\footnote{Kacc\,379, R\=upa\,258, Sadd\,809, Mogg\,3.104; Kacc\,381--2, R\=upa\,254, 259, Sadd\,812--3, Mogg\,3.103}; \pali{dvi} can become \pali{b\=a}\footnote{Kacc\,380, R\=upa\,255, Sadd\,810, Mogg\,3.98}; sometimes \pali{dvi} changes to \pali{du, di,} or \pali{do} in compounds, e.g.\ \pali{duratta\d m, diratta\d m} (2 nights), \pali{duvidho} (2 parts), \pali{digu} (2 oxen), \pali{doha\d lin\=i} (pregnant woman)\footnote{Sadd\,811, Mogg\,3.91--2}; ending vowels can become \pali{\=a}, e.g.\ \pali{dv\=adasa}, \pali{ek\=adasa}, \pali{a\d t\d th\=adasa}\footnote{Kacc\,383, R\=upa\,253, Sadd\,815, Mogg\,3.102, 3.94, 3.97}; \pali{ti} can become \pali{te}, e.g.\ \pali{terasa}\footnote{Mogg\,3.95--6}; \pali{pa\~nca} can change to \pali{panna} or \pali{pa\d n\d na}\footnote{Sadd\,814, Mogg\,3.99}; \pali{catu} can become \pali{cu}, \pali{co}, or \pali{ca}, e.g.\ \pali{cuddasa}, \pali{coddasa}, \pali{catt\=al\=isa}, \pali{cutt\=al\=isa}, \pali{cott\=al\=isa}, or can be deleted in \pali{t\=al\=isa}\footnote{Kacc\,390, R\=upa\,256, Sadd\,826--7, Mogg\,3.100}; \pali{catur\=as\=iti} can become \pali{cull\=as\=iti}\footnote{Sadd\,828}; \pali{dv\=asa\d t\d thi} can become \pali{dva\d t\d thi}\footnote{Sadd\,827};

\phantomsection
\addcontentsline{toc}{section}{Cardinal Numbers}
\section*{Cardinal Numbers}

After you know the numbers, now you can use them to count things. But we should talk about rules explained by textbooks first. As mentioned earlier, numbers 1--4 are pronouns. For the rest, 5--98 are adjectives, and 99 onwards are nouns. There is nothing special about number 99. It just has something to do with its ending. There is a difference between using numbers as a noun and using them as an adjective. This will be explained later.

Numbers 1--4 decline distinctively as shown on page \pageref{decl:one} onwards. Numbers 5--18\footnote{6--18 can also be used uninflected, \citealp[see][p.~71]{collins:grammar}.} decline in the same way in all genders as shown in Table \ref{tab:five}. Beyond 18, you have to consider the word's ending. For numbers between 19--98, if the term ends with \pali{i} (e.g.\ \pali{v\=isati, ti\d msati, navuti}), it decline as f.\ sg. If the term ends with \pali{a}\footnote{It seems that this ending can also be used bluntly uninflected, particularly in nom. See \pali{sattav\=isa} in the passage from Sadd-Pad on page \pageref{par:sabbanamani}.} (e.g.\ \pali{v\=isa, ti\d msa, pa\~n\~n\=asa}), its ending has to be changed to \pali{\=a}, then it declines as f.\ sg. If the terms ends with \pali{a\d m} (e.g.\ \pali{v\=isa\d m, ti\d msa\d m, pa\~n\~n\=asa\d m}), it decline unusually as f.\ sg.\ as shown in Table \ref{tab:twenty}. From 99 onwards, numbers ended with \pali{a\d m} (e.g.\ \pali{sata\d m, sahassa\d m}) decline as nt.\ both sg.\ and pl.\ as shown in Table \ref{tab:hundred}.

To summarize, 1 has both singular and plural forms (3 genders), 2--18 have only plural forms (3 genders, sort of), 19--98 use only feminine singular forms\footnote{Sadd\,825}, and 99 onwards use both singular and plural forms (one gender depending on term's ending). From \pali{ko\d ti} onwards, the terms decline as general nouns, f.\ for \pali{i} and \pali{\=i} endings, nt.\ for \pali{a} ending. It is a little confusing if you read this for the first time. In practice it is pretty easy. You will be familiar with these when you use them.

\begin{table}[!hbt]
\centering
\caption{Declension of number 5}
\label{tab:five}
\bigskip
\begin{tabular}{l>{\itshape}l} \toprule
\bfseries Case & \upshape\bfseries Plural \\
\midrule
1. nom. & pa\~nca \\
2. acc. & pa\~nca \\
3. ins. & pa\~ncahi \\
4. dat. & pa\~ncanna\d m \\
5. abl. & pa\~ncahi \\
6. gen. & pa\~ncanna\d m \\
7. loc. & pa\~ncasu \\
\bottomrule
\end{tabular}
\end{table}

\begin{table}[!hbt]
\centering
\caption{Declension of number 20}
\label{tab:twenty}
\bigskip
\begin{tabular}{l*{3}{>{\itshape}l}} \toprule
\bfseries Case & \multicolumn{3}{c}{\upshape\bfseries Singular} \\
\midrule
1. nom. & v\=isati & v\=is\=a & v\=isa\d m \\
2. acc. & v\=isati\d m & v\=isa\d m & v\=isa\d m \\
3. ins. & v\=isatiy\=a & v\=is\=aya & v\=is\=aya \\
4. dat. & v\=isatiy\=a & v\=is\=aya & v\=is\=aya \\
5. abl. & v\=isatiy\=a & v\=is\=aya & v\=is\=aya \\
6. gen. & v\=isatiy\=a & v\=is\=aya & v\=is\=aya \\
7. loc. & v\=isatiya\d m & v\=is\=aya\d m & v\=is\=aya\d m \\
\bottomrule
\end{tabular}
\end{table}

Now let us see some examples. To count things from 1 to 4, you have to know the gender of things you are counting, because these numbers can decline into three genders, except two/both. Number 1 has both singular and plural forms, and 2--4 has only plural forms. Why does 1 has plural form? If you can recall, we met this before in Chapter \ref{chap:pron-misc}, page \pageref{par:ekapl}. When you use 1 as a counter, it only takes singular forms. If you mean ``(a) certain'' or ``some (kind/kinds) of,'' it can also take plural forms. Here is an example: ``I have 4 brothers, 2 sisters. In brothers, 1 is my elder, 3 are my youngers. In sisters, they are all my youngers. I have no elder sister.''

\pagebreak\palisample{mama bh\=ataro catt\=aro santi, bhagin\=i dve. bh\=ataresu eko je\d t\d thabh\=at\=a, tayo ka\d ni\d t\d thabh\=ataro. bhaginiya\d m sabb\=a ka\d ni\d t\d thabhagin\=i. je\d t\d thabhagin\=i natthi.}

Be careful with irregular nouns. For the terms ended with \pali{bh\=ata} see their declension paradigm on page \pageref{decl:pitu}. See Chapter \ref{chap:loc} for an explanation on loc.\ used in ``In those\ldots'' or ``Among those\ldots'' If you can fluently deal with nouns' gender and number when declining words, you should not have any problem with this example.

Let us try a more challenging example: ``I buy 16 mangoes from a market. In 16 mangoes I give 12 (of them) to 6 childs. Each child get 2 mangoes. I get the remaining 4.''

I hint you some words: We use \palibf{ekeka}\footnote{See also page \pageref{par:ekeka}.} for `each' and use \palibf{sesa} for `remaining.' For other unknown words, please find in our vocabulary, Appendix \ref{chap:vocab}. Here we go:

\palisample{\=apa\d nasm\=a so\d lasa amb\=ani ki\d n\=ami. so\d lasasu channa\d m d\=arak\=ana\d m dv\=adasa demi. ekeko d\=arako dve amb\=ani labhati. aha\d m catt\=ari ses\=ani (amb\=ani) labh\=ami.}

Moving to the next numerical range, let us say this: ``In this room, there are 35 girls, 22 boys. I give 57 candies to all 57 (children).''

\palisample{imasmi\d m gabbhasmi\d m pa\~ncatti\d ms\=a d\=arik\=a(yo) santi, dv\=av\=isati d\=arak\=a. sabbesa\d m sattapa\~n\~n\=as\=aya sattapa\~n\~n\=asa\d m khajjak\=ani demi.}

From the above example, you can see that there are discrepancies in gender and number when we use numerals. It has a practical reason for this. We inevitably use, for instance, \pali{dv\=av\=isati} (f.\ sg.) with \pali{d\=arak\=a} (m.\ pl.), \pali{sabbesa\d m} (dat.\ pl.) with \pali{sattapa\~n\~n\=as\=aya} (dat.\ sg.), and \pali{sattapa\~n\~n\=asa\d m} (acc.\ sg.) with \pali{khajjak\=ani} (acc.\ pl.). Some rules are suspended here. Or, as you have seen above, exceptions have to be posted as rules. If you think of rules first, it will be a kind of headache. But if you just simply use them, it goes naturally. I would like to remind you again here that P\=ali grammatical rules come after its literature. Rules are orderly reconstruction from messy nature of the language.

\begin{table}[!hbt]
\centering
\caption{Declension of number 100}
\label{tab:hundred}
\bigskip
\begin{tabular}{l*{2}{>{\itshape}l}} \toprule
\bfseries Case & \upshape\bfseries Singular & \upshape\bfseries Plural  \\
\midrule
1. nom. & sata\d m & sat\=ani, sat\=a \\
2. acc. & sata\d m & sat\=ani, sate \\
3. ins. & satena & satehi, satebhi \\
4. dat. & satassa & sat\=ana\d m \\
5. abl. & sat\=a, satasm\=a, satamh\=a & satehi, satebhi \\
6. gen. & satassa & sat\=ana\d m \\
7. loc. & sate, satasmi\d m, satamhi & satesu \\
\bottomrule
\end{tabular}
\end{table}

Numbers beyond 98 can be used in two ways for all genders. Here are examples from Sadd-Pad Ch.\,13:

\begin{quote}
\pali{sata\d m bhikkh\=u. sata\d m itthiyo. sata\d m citt\=ani.}\\
or\\
\mbox{\pali{bhikkh\=una\d m sata\d m. itth\=ina\d m sata\d m. citt\=ana\d m sata\d m.}}\\[1.5mm]
``100 monks. 100 women. 100 minds.''
\end{quote}

In the former use, \pali{sata\d m} looks like an adjective, but the tradition maintains that numbers from 99 onwards are nouns. In English grammar, it can be seen as an \emph{apposition}.\footnote{``A relation between two phrases, especially noun phrases, in which the two phrases are simply juxtaposed. The second noun phrase refers to the same entity as the first one and merely adds extra information.'' \citep[p.~32]{brownmiller:dict}} However, Aggava\d msa explains this in a different way. In Sadd-Pad Ch.\,13, he distinguishes between number as subject (\pali{sa\.nkhy\=appadh\=ana} or \pali{visesaya}) and number as modifier (\pali{sa\.nkhyeyyappadh\=ana} or \pali{visesana}), and he confirms that numbers from \pali{v\=isa} to \pali{ko\d ti} perform both functions. That is to say, in the first use \pali{sata\d m} works as a modifier.

As mentioned above, the latter use treats \pali{sata\d m} as an independent noun, a subject. So, it has to relate to other noun by using genitive case. These can literally translated as ``a hundred of monks'' or ``a monks' hundred'' and so on.

Aggava\d msa also exemplifies with an interesting verse from the canon:

\begin{quote}
\pali{Sata\d m hatth\=i sata\d m ass\=a, sata\d m assatar\=irath\=a;}\\
\pali{Sata\d m ka\~n\~n\=asahass\=ani, \=amukkama\d niku\d n\d dal\=a;}\\
\pali{Ekassa padav\=itih\=arassa, kala\d m n\=agghanti so\d lasi\d m.}\footnote{Cv\,6.305; S1\,242 (SN\,10)}\\[1.5mm]
``100,000 elephants, 100,000 horses, 100,000 \\(she-)muled chariots;\footnote{I.\,B.\ Horner translated these as 100 elephants, 100 horses, and 100 chariots. See \citealp[p.~2197]{horner:discipline}.}\\
100,000 girls adorned with jeweled earrings;\footnote{\pali{\=amukkama\d niku\d n\d dal\=a} = \pali{\=amutta} + \pali{ma\d niku\d n\d dala}}\\
These are not worth the sixteenth part of one pace.\footnote{\pali{kal\=a} = a small part; \pali{n\=agghanti} = \pali{na} + \pali{agghanti}}''
\end{quote}

Aggava\d msa explains that \pali{sata\d m hatth\=i} functions as subject (\pali{visesaya}), but \pali{sahass\=ani} as modifier (\pali{visesana}). When distributing \pali{sahass\=ani} to each subject, we get ``\pali{sata\d m hatth\=i sahass\=ani}'', ``\pali{sata\d m ass\=a sahass\=ani}'', and so on, hence, 100,000 elephants, and so on. Another way to translate these is to use gen. Then we get ``\pali{hatth\=ina\d m satasahassa\d m}'' (hundred thousand of elephants), ``\pali{ass\=ana\d m satasahassa\d m}'' (hundred thousand of horses), ``\pali{assatar\=irath\=ana\d m satasahassa\d m}'' (hundred thousand of chariots), and ``\pali{\=amukkama\d niku\d n\d dal\=ana\d m ka\~n\~n\=ana\d m satasahassa\d m}'' (hundred thousand of ador\-ned girls).

The explanation so far is helpful to our understanding, but there is a trick. In Sadd-Pad Ch.\,13, Aggava\d msa uses split ``\pali{ka\~n\~n\=a sahass\=ani}'' rather than compound ``\pali{ka\~n\~n\=asahass\=ani}'' as we see in the canon. A single space changes everything! As a unit, \pali{ka\~n\~n\=asahass\=ani} is better seen as a subject with \pali{sata\d m} as modifier. This means ``100 thousand of girls.'' Moreover, \pali{sahassa\d m} should not be distributed to other subjects, because it unites with \pali{ka\~n\~n\=a} as a single word. Hence, I suggest that we should follow the translation of I.\,B.\ Horner, i.e.\ 100 elephants and so on.\footnote{It is likely that Aggava\d msa mistook the passage, or he intended to make it as such to make his point. The first line of the verse can also be found in Ja\,22:1357 which can be translated only to 100 elephants and so on.} Still, Aggava\d msa's explanation has its value. The lesson from this instance is significant. How do you remember space? No, the tradition remembered strings of words, not spaces. You might think it is not a big difference because it is just an allusion, but mistaking 100 for 100,000 or vice versa is quite a big miss. 

Let us move on. To tell that something has a particular amount of property, say, height of a mountain, Aggava\d msa gives us an example from a commentary:

\begin{quote}
\pali{Yojan\=ana\d m sat\=anucco, himav\=a pa\~nca pabbato;}\footnote{Sp1\,1}\\
``The Himalaya Mt.\ is 500 yojanas high.''\footnote{\pali{sat\=anucco} = \pali{sat\=ani} + \pali{ucca}; 1 yojana $\approx$ 7 miles}
\end{quote}

To make it simpler, we rearrange the sentence to ``\pali{himav\=a pabbato yojan\=ana\d m pa\~nca sat\=ani ucco (hoti)}'' (The Himalaya is high by 500 of yojanas). Aggava\d msa hints that \pali{pa\~nca sat\=ani} is in acc., so it works like an adverb (see Chapter \ref{chap:adv}).

To be complete on this issue, now you can tell how tall you are, but we have to know more on measurement units. I summarize the units of length used in P\=ali in Table \ref{tab:lenunits}.\footnote{Abhidh\=a\,195--7}

\begin{table}[!hbt]
\centering
\caption{Units of length}
\label{tab:lenunits}
\bigskip
\begin{tabular}{lll} \toprule
7 grains (\pali{dha\~n\~nam\=asa}) & = & 1 a\.ngula (nt.) (inch) \\
12 a\.ngulas & = & 1 vidatthi (f.) (span) \\
2 vidatthis & = & 1 ratana (nt.) (cubit) \\
7 ratanas & = & 1 ya\d t\d thi (f.) (stick) \\
20 ya\d t\d this & = & 1 usabha (nt.) (bull?) \\
80 usabhas & = & 1 g\=avuta (nt.) (league) \\
4 g\=avutas & = & 1 yojana (nt.) (yoke?) \\
\midrule
500 dhanus (nt.) (bows) & = & 1 kosa (m., nt.) \\
4 amba\d nas (nt.) & = & 1 kar\=isa (nt.) \\
28 hatthas (m.) & $\approx$ & 1 abbhantara (nt.) \\
\bottomrule
\end{tabular}
\end{table}

Dealing with measurement in P\=ali is a bit confusing, because different sources may give you different measures. For example, A.\,P.\ Buddhadatta gives us that 4 cubits equal 1 fathom (\pali{dhanu}), then 500 fathoms equal 1 league (\pali{g\=avuta} or \pali{kosa}).\footnote{\citealp[p.~30]{buddhadatta:aids}} It is problematic when we equate \pali{g\=avuta} with \pali{kosa}, which I think they come from different systems. From Ven.\, Buddhadatta's measurement, 1 league equals 2,000 cubits ($4 \times 500$), whereas from the table 1 league equals 11,200 cubits ($7 \times 20 \times 80$).

For just telling our height, let us make it simple by converting to our familiar units. One cubit is around 17--22 inches or 43--56 centimeters nowadays.\footnote{The American Heritage Dictionary, \url{https://www.ahdictionary.com/word/search.html?q=cubit}} If we take it at 18 inches, 2 cubits make 1 yard (36 inches or 3 feet). Or if we take it at 50 cm, 2 cubits make 1 meter. You can use either system. They both are close to the approximation. However, to make it more precise is difficult, for ancient inch and today inch are quite different.

Now if you are 6 feet tall, it will be easy. You are 4 cubits or \pali{ratana} tall. You can say this as follows:

\palisample{aha\d m catt\=ari ratan\=ani ucco homi.}

What if you are 150 cm tall? That is 3 cubits. So, we simply get ``\pali{aha\d m t\=i\d ni ratan\=ani ucco homi}.'' How about 175 cm? It is 3 cubits plus a half or 1 span (\pali{vidatthi}). You can say this as:

\palisample{eka\d m vidatthi\d m uttara\d m t\=i\d ni ratan\=ani ucco homi.}

We use \palibf{uttara} (over, higher) in this case (see more detail below). Or, alternatively, you can say ``I am 4 cubits minus 1 span tall'' by using \palibf{\=una} as follows:

\palisample{eka\d m vidatthi\d m \=una\d m catt\=ari ratan\=ani ucco homi.}

That is, I think, the best way we can deal with this situation. Try doing some math and making it easy to understand. It is not necessary to make it very precise. In that case, the best solution is to import modern units into P\=ali vocabulary, for example, using hybrid compound \pali{meter-m\=a\d na} for meter. Using some modifiers may be helpful, e.g.\ \pali{pam\=a\d nato/pam\=a\d nena} (approximately), \pali{bhiyyo} (exceedingly, more). For example, ``\pali{pam\=a\d nato bhiyyo t\=i\d ni ratan\=ani ucco homi}'' means ``Approximately I am more than 3 cubits tall.''

Here is a way to say ``I am 5 feet and 9 inches tall.''

\palisample{pa\~nca foot-m\=a\d n\=ani nava inch-m\=a\d n\=ani ca ucco homi.}

And here is for ``I am 178 cm tall.''

\palisample{a\d t\d thasattatayuttarasat\=ani centimeter-m\=a\d n\=ani ucco homi.}

How come the number? Please read on.

Now we will move to a more complicated matter, and I will focus mainly on using gen.\ in relating numeric terms. Saying round numbers in P\=ali is easy, such as \pali{n\=av\=ana\d m dvisata\d m} (200 of ships), \pali{ass\=ana\d m tisata\d m} (300 of horses). There is another way to render these numbers. You can split the numbers into two parts, put the nouns in between, and use plural form \pali{sat\=ani}\footnote{As far as I know, there is no rigid rule whether when we should use singular or plural form for \pali{sata} and \pali{sahassa}. We found both forms in the scriptures. From statistical data provided by program \textsc{P\=ali\,Platform} 3 using the CSTR collection, \pali{sata\d m} has 802 occurrences, whereas \pali{sat\=ani} has 262. You might think \pali{sata\d m} is used for numbers under 200. This is not the case, because \pali{dvisata\d m} has 5 occurrences, whereas \pali{dvisat\=ani} has only 2. To be complete, \pali{sahassa\d m} has 505 occurrences, \pali{sahass\=ani} 204, \pali{dvisahassa\d m} 3, and none for \pali{dvisahass\=ani}. So, in practice you can use either form. I just follow a suggestion from a textbook here. Moreover, when \pali{sata\d m} and \pali{sahassa\d m} are composed in sentences, they can take both singular or plural verbs. See Sadd-Pad Ch.\,13, ``\pali{Satamiti saddo}'' onwards.}. So, these are equivalent to the examples mentioned:

\palisample{dve n\=av\=ana\d m sat\=ani. t\=i\d ni ass\=ana\d m sat\=ani.}

Yet another rendition is to form a compound by dropping genitive ending of the noun and connecting it to the last part, as we have seen in ``\pali{ka\~n\~n\=asahass\=ani}'' above. So, you can say these also:

\palisample{dve n\=av\=asat\=ani. t\=i\d ni assasat\=ani.}

When the numbers are split, the two parts have to agree in case and number. We use \pali{t\=i\d ni} because of nt.\ \pali{sat\=ani}. It has nothing to do with the nouns. This form of rendition is a bit odd to English speakers. So, it need some practice to get familiar with.

Sometimes \palibf{matta} (measured as, or as much as) is added to form a compound with the number. This adds nothing to the meaning, but sometimes it suggests an approximation. Here are some examples:

\begin{quote}
\pali{bhagav\=a pa\~ncamatt\=ani mand\=amukhisat\=ani abhinimmini}\footnote{Mv\,1.49}\\
``The Blessed One miraculously created 500 coal-pans.''\\[1.5mm]
\pali{mahatiy\=a paribb\=ajakaparis\=aya saddhi\d m ti\d msamattehi paribb\=ajakasatehi}\footnote{D1\,406 (DN\,9)}\\
``[Po\d t\d thap\=ada] together with a great assembly of 3,000 wanderers \ldots''\\[1.5mm]
\pali{K\=iva d\=uro, mah\=ar\=aja, ito alasando hoti? Dvimatt\=ani, bhante, yojanasat\=ani.}\footnote{Mil\,2-3.4}\\
``How far, Your Majesty, is Alasanda island from here? 200 yojanas, Venerable.''\\
\end{quote}

Now we will combine numbers of the first range (1--98) with numbers ended with \pali{sata\d m}. Hence we can say any number under one thousand. The keyword used as a connector here is \palibf{uttara}. Does this sound familiar? If not, you should review Chapter \ref{chap:pron-misc} one more time. In that chapter we introduce \pali{uttara} as a pronoun meaning `northern' or `upper.' In that very sense, when we use with numbers, it functions as an adjective meaning `higher.' When you say `101,' you say something like  `100 higher by 1.' By the help of \emph{instrumental case}, thus you get this:

\palisample{ekena/ek\=aya uttara\d m sata\d m.\sampleor ekena/ek\=aya uttar\=ani sat\=ani}

You may go bluntly by using \pali{ca} (and) to combine the numbers like English, hence, ``sata\d m eko/ek\=a/eka\d m ca'' (100 and 1 = 101). This should be used with caution, because it will cause an unnecessary confusion. For example, ``dve sata\d m ca'' can means both 200 or 102. So, using \pali{ca} to combine the numbers is not recommended, except in spontaneous conversations and poetry.\footnote{There is a strange example from the canon: ``\pali{As\=iti dasa eko ca, indan\=am\=a mahabbal\=a}'' (80 + 10 + 1 (= 91) [sons] called \pali{Inda} [are] powerful). This is from \=A\d t\=an\=a\d tiyasutta, D3\,279 (DN\,32).}

When \pali{uttara} is used, 102 is \pali{dv\=ihi uttar\=ani sat\=ani}, 203 is \pali{t\=ihi uttar\=ani dvisat\=ani}, 998 is \pali{a\d t\d thanavutiy\=a uttar\=ani navasat\=ani}, and 999 is \pali{ek\=unasatehi uttar\=ani navasat\=ani}. Then we add a noun to the numbers, such as ``365 days.'' So, we get this:

\palisample{pa\~ncasa\d t\d thiy\=a (dinehi) uttar\=ani t\=i\d ni din\=ana\d m sat\=ani.}

A word-by-word translation of this can be: ``three hundred of days higher by sixty-five days.'' Be careful with cases used in this expression, gen.\ is used to relate noun to the hundred digit, and ins.\ is used to mark the excess remainder. By such a way, now you can say numbers up to 999. However, in practice we often pack the numbers into compounds by getting rid of terms' declensions. In Table \ref{tab:numover100} I list some numbers from 101--999, for you can see a quick picture.

\bigskip
\begin{longtable}[c]{@{}p{0.05\linewidth}%
	>{\itshape\raggedright\arraybackslash}p{0.45\linewidth}%
	>{\itshape\raggedright\arraybackslash}p{0.41\linewidth}@{}}
\caption{Numbers from 101--999}\label{tab:numover100}\\
\toprule
\bfseries\upshape N & \bfseries\upshape P\=ali & \bfseries\upshape Decomposition\\ \midrule
\endfirsthead
\multicolumn{3}{c}{\tablename\ \thetable: Numbers from 101--999 (contd\ldots)}\\
\toprule
\bfseries\upshape N & \bfseries\upshape P\=ali & \bfseries\upshape Decomposition\\ \midrule
\endhead
\bottomrule
\ltblcontinuedbreak{3}
\endfoot
\bottomrule
\endlastfoot
%
101 & ekuttarasata\d m & eka + uttara + sata\d m \\
102 & dvayuttarasata\d m\footnote{Normally, when the \pali{-i} ending joins with another term, it changes to \pali{ya} or \pali{aya}, as we see in the cases of \pali{dvi, ti,} etc. It is definitely alright if you prefer to use \pali{dvi-uttara, ti-uttara,} and so on.} & dvi + uttara + sata\d m \\
103 & tayuttarasata\d m & ti + uttara + sata\d m \\
104 & catuttarasata\d m & catu + uttara + sata\d m \\
105 & pa\~ncuttarasata\d m & pa\~nca + uttara + sata\d m \\
106 & chuttarasata\d m\footnote{Maybe \pali{cha-uttara} sounds a little better, as we find an instance of \pali{chauttarasaṭṭhiadhikasatehi} in the collections.} & cha + uttara + sata\d m \\
107 & sattuttarasata\d m & satta + uttara + sata\d m \\
108 & a\d t\d thuttarasata\d m & a\d t\d tha + uttara + sata\d m \\
109 & navuttarasata\d m & nava + uttara + sata\d m \\
110 & dasuttarasata\d m & dasa + uttara + sata\d m \\
111 & ek\=adasuttarasata\d m & ek\=adasa + uttara + sata\d m \\
201 & ekuttaradvisata\d m & eka + uttara + dvi + sata\d m \\
211 & ek\=adasuttaradvisata\d m & ek\=adasa + uttara + dvisata\d m \\
990 & navutayuttaranavasata\d m & navuti + uttara + navasata\d m \\
998 & \mbox{a\d t\d thanavutayuttaranavasata\d m} & a\d t\d thanavuti + uttara + navasata\d m \\
999 & ek\=unasatuttaranavasata\d m & ek\=unasata + uttara + navasata\d m \\
\end{longtable}

To understand what happens in the table, you need some knowledge of P\=ali word joining or \emph{Sandhi}. For a quick grasp, there are some intuitive rules you can observe here: (1) When a vowel meets a consonant, they can join unaltered. (2) When a vowel meets another vowel, if they are the same and short, the outcome can be a long vowel of that sound. If not, one of them has to be dropped, or one of them is transformed before the drop. A worth noting case above is when \pali{i} meets another vowel. According to certain phonetic adaptation, \pali{i} is changed to \pali{aya} (its semivowel equivalent).\footnote{Under the same situation, \pali{u} is changed to \pali{ava}. See also the end of Chapter \ref{chap:nuts}.} Then the last \pali{a} is dropped, hence we get \pali{dvayuttara} from \pali{dvi + uttara}. If you are more curious, learn more about Sandhi in Appendix \ref{chap:sandhi}. If you are not baffled by now, you should not have any problem with numbers under 1,000.

When we use these numbers with nouns, we have two options. First, the bunch of number is used as a compound unit. For example, ``One year is 365 (of) days'' is:

\palisample{eka\d m sa\d mvacchara\d m din\=ana\d m pa\~ncasa\d t\d th\textbf{uttara}tisata\d m hoti.}

When the bunch of number is long, it is a little of a mouthful. As the second option, you can split the number into three parts, so we get this instead:

\palisample{eka\d m sa\d mvacchara\d m pa\~ncasa\d t\d thuttar\=ani t\=i\d ni din\=ana\d m sat\=ani hoti.\sampleor \ldots t\=i\d ni dinasat\=ani hoti.\sampleor \ldots pa\~ncasa\d t\d thi\textbf{din}uttar\=ani\footnote{\pali{pa\~ncasa\d t\d thi + dina + uttara}} t\=i\d ni dinasat\=ani hoti.}

Now, if you are ready, we will move to thousands. When we add a number under 99 to a thousand, you can follow the method described above, for example, \pali{ekuttarasahassa\d m} (1,001), \pali{dvayuttarasahassa\d m} (1,002), \pali{a\d t\d thanavutayutarasahassa\d m} (1,098). When a number is accompanied with a noun, it follow the same pattern. For example, you can say ``2021 (of) years'' as:

\palisample{sa\d mvacchar\=ana\d m ekav\=isuttaradvisahassa\d m. \sampleor ekav\=isuttar\=ani dve sa\d mvacchar\=ana\d m sahass\=ani. \sampleor ekav\=isuttar\=ani dve sa\d mvaccharasahass\=ani. \sampleor ekav\=isa\textbf{sa\d mvacchar}uttar\=ani dve sa\d mvaccharasahass\=ani.}

When a digit of hundred is added to the number, a new connector is needed---\palibf{adhika} (exceeding, superior). We use \pali{uttara} to mark numbers below 99, and use \pali{adhika} to mark the digit of hundred. We always put the least digit first. Hence, ``4,321 people'' can be rendered bluntly as:

\palisample{jan\=ana\d m ekav\=is\textbf{uttara}tisat\textbf{\=adhika}catusahassa\d m.}

That is a mouthful. Then we split the bunch of number as follows:

\palisample{ekav\=isuttar\=ani tisat\=adhik\=ani jan\=ana\d m catusahass\=ani.\sampleor ekav\=isa\textbf{jan}uttar\=ani \ldots}

We can also isolate \pali{adhika} from the compounds and restore the numbers' declension. You have to keep in mind that the numbers related to \pali{adhika} take \emph{instrumental case} in the sense of ``exceeding by.'' Thus, we get this:

\palisample{ekav\=isuttar\=ani ti\textbf{satehi} adhik\=ani jan\=ana\d m catusahass\=ani.}

We can split this furthermore by isolating \pali{uttara} and breaking down the hundred and thousand digits. Practically, the noun is usually inserted before \pali{uttara}, e.g. \pali{ekav\=isajanuttar\=ani}. Therefore, the final split looks like this:

\palisample{ekav\=is\=aya janehi uttar\=ani t\=i\d ni jan\=ana\d m satehi adhik\=ani catt\=ari jan\=ana\d m sahass\=ani.}

We can translate this word-by-word as: ``four thousands of people exceeding by three hundreds of people higher by twenty-one people.'' If this translation makes sense to you, it means you understand what is going on here. If not, please try carefully reviewing the content again. It takes time to digest the complication.

Now you can tell the year. For instance, the Buddhist year 2564 can be rendered separately as:

\palisample{catusa\d t\d thiy\=a sa\d mvaccharehi uttar\=ani pa\~ncahi sa\d mvacchar\=ana\d m satehi adhik\=ani dve sa\d mvacchar\=ana\d m sahass\=ani.}

Or, if you like compound form:

\palisample{catusa\d t\d thisa\d mvaccharuttarapa\~ncasat\=adhik\=ani dvesa\d mvaccharasahass\=ani.}

Formally, before a monk give a dhamma talk, he tells the year in this way: ``\pali{it\=ani} (now) \pali{catusa\d t\d thi\ldots sahass\=ani} (2564 years) \pali{atikkant\=ani} (went beyond).''

Numbers beyond 9,999 will be easy if you stick to compound form. You just separate the hundred part and bunch the rest together. For example, ``12,345 people'' can be said as ``12,000 people exceeding by 345 people'', hence:

\palisample{pa\~ncacatt\=a\d l\=isajanuttaratisat\=adhik\=ani dv\=adasajanasahass\=ani.}

And ``123,456 people'' can be as:

\palisample{chapa\~n\~n\=asajanuttaracatusat\=adhik\=ani tev\=isatayuttarasatajanasahass\=ani.}

Finally, ``1,234,567 people'' can be as follows:

\palisample{sattasa\d t\d thijanuttarapa\~ncasat\=adhik\=ani catuti\d msuttaradv\=adasasatajanasahass\=ani.}

Beyond this, if it is not a round number, it is quite very confusing. When the last compound is bigger, it is difficult to handle. Perhaps, it is viable to break the compound apart resulting in a lot of individual words. That does not seem to be the good solution either. You may play around with this to get some familiarity. This shows that P\=ali is not suitable for big numbers with high precision. It is not a language for mathematicians, so to speak. However, P\=ali does quite easily with round big numbers. For example, ``a billion (1,000 millions) of people'' can be simply put as:

\palisample{jan\=ana\d m satako\d ti.\sampleor janasatako\d ti.}

Now we can finish our heading sentence, ``There are 7.8 billion people in the world.'' We have to make a conversion from 7.8 billion to 780 \pali{ko\d ti} first. Then we get this:

\palisample{loke jan\=ana\d m as\=itayuttarasattasatako\d ti.}

Other huge numbers can be treated in the same way. Be careful with \palibf{nahuta}.\footnote{In Niru\,151, a passage shows that, ``\pali{Sahassa\d m k\=asi n\=ama, dasasahassa\d m nahuta\d m n\=ama, satasahassa\d m lakkha\d m n\=ama}'' (1,000 [is] called \pali{k\=asi}, 10,000 [is] called \pali{nahuta}, 100,000 [is] called \pali{lakkha}). See also Sadd\,833.} It can mean both 10,000 and 10\textsuperscript{28}. In very rare case we will use the latter huge figure. Here is an example from a commentary:

\begin{quote}
\pali{Duve satasahass\=ani, catt\=ari nahut\=ani ca;} \\
\pali{Ettaka\d m bahalattena, sa\.nkh\=at\=aya\d m vasundhar\=a.}\footnote{Sp1\,1} \\[1.5mm]
``200,000 and 40,000 [yojanas], \\
this much by thickness calculated, [is] the earth.''
\end{quote}

Aggava\d msa explains that \pali{duve} modifies \pali{satasahass\=ani}, hence 200,000, and \pali{catt\=ari} modifies \pali{nahut\=ani}, hence 40,000. With \pali{ca} the combination yields 240,000 (\pali{dvisatasahassa\d m catunahuta\d m}).

Now I will show you some minor interesting uses of numbers. You can say `many \ldots' by using \palibf{aneka} (not one, various) or \palibf{pahu} (many), for example, ``\pali{Gha\d t\=anekasahass\=ani, kumbh\=ina\~nca sat\=a bah\=u;}''\footnote{Bv\,2:169} (many thousands pots, many hundreds water pots).

You can use \palibf{paro} for `more than,' for example ``\pali{Paropa\~n\~n\=asa n\=atikiy\=a paric\=arak\=a abbhat\=it\=a k\=ala\.nkat\=a}''\footnote{D2\,273 (DN\,18)} (More than 50 of the villagers of N\=atika, once benefactors [of the religion], who had died in the past).

You can approximate a number by giving its range. For example, you can say ``There are a few (2--3) dogs'' as ``\pali{dvetayo sunakh\=a santi}.'' Here is an example from the canon: ``\pali{dasav\=isasahass\=ana\d m}''\footnote{Bv\,27:8} (10,000--20,000 of [people]).

Using \palibf{katipaya} (a few, some, several) can yield a similar result. For example, ``\pali{katipay\=a sunakh\=a santi}'' means ``There are some dogs.'' I summarize the declension of \pali{katipaya} in Table \ref{tab:katipaya}.\footnote{Sadd-Pad Ch.\,11} It is always used in plural.

\begin{table}[!hbt]
\centering
\caption{Declension of \pali{katipaya}}
\label{tab:katipaya}
\bigskip
\begin{tabular}{@{}l*{3}{>{\itshape}l}@{}} \toprule
\bfseries Case & \bfseries\upshape m. pl. & \bfseries\upshape f. pl. & \bfseries\upshape nt. pl. \\
\midrule
1. nom. & katipay\=a & katipay\=ayo & katipay\=ani \\
2. acc. & katipaye & katipay\=ayo & katipay\=ani, katipaye \\
3. ins. & katipaye(b)hi & katipay\=a(b)hi & katipaye(b)hi \\
4. dat. & katipay\=ana\d m & katipay\=ana\d m & katipay\=ana\d m \\
5. abl. & katipaye(b)hi & katipay\=a(b)hi & katipaye(b)hi \\
6. gen. & katipay\=ana\d m & katipay\=ana\d m & katipay\=ana\d m \\
7. loc. & katipayesu & katipay\=asu & katipayesu \\
\bottomrule
\end{tabular}
\end{table}

It will not be complete if we do not talk about how to ask for numbers. A common keyword used here is \palibf{kati} (how many?). This is used as an adjective uniformly in three genders as shown in Table \ref{tab:kati}.\footnote{Sadd-Pad Ch.\,11; R\=upa\,259; Mogg\,2.168; Niru\,237; in Payo\,234 also \pali{kat\=i(b)hi}; in Mogg\,2.48, Niru\,238 also \pali{katinna\d m}} It is also used only in plural form.

\begin{table}[!hbt]
\centering
\caption{Declension of \pali{kati}}
\label{tab:kati}
\bigskip
\begin{tabular}{@{}l>{\itshape}l@{}} \toprule
\bfseries Case & \bfseries\upshape Plural \\ 
\midrule
1. nom. & kati \\
2. acc. & kati \\
3. ins. & kati(b)hi \\
4. dat. & katina\d m \\
5. abl. & kati(b)hi \\
6. gen. & katina\d m \\ 
7. loc. & katisu \\
\bottomrule
\end{tabular}
\end{table}

To ask how many people in the world, we go simply like this:

\palisample{loke kati jan\=a honti.}

Here is an example from the canon:

\begin{quote}
\pali{Kati j\=agarata\d m sutt\=a, kati suttesu j\=agar\=a;} \\
\pali{Katibhi rajam\=adeti, katibhi parisujjhati.}\footnote{S1\,6 (SN\,1); In Sadd-Pad Ch.\,11 \pali{rajam\=aneti} is found.} \\[1.5mm]
``How many are asleep when [others] are awake? \\
How many are awake when [others] sleep? \\
By how many does one gather dust? \\
By how many is one purified''\footnote{\citealp[pp.~91--2]{bodhi:connected}} \\
\end{quote}

You may also find \pali{kati} in compound forms that can be used conveniently. For example, \pali{kativassa} (how old?) can be used for age inquiry, such as ``\pali{kativasso/kativass\=a'si}'' (how old are you?); \pali{katividha} (how many kinds?) such as ``\pali{Katividho sam\=adhi?}''\footnote{Vism\,3.38} (How many kinds of concentration?). It can be indeclinable such as \pali{katikhattu\d m} (how many times), for example, ``\pali{katikhattu\d m imasmi\d m \=agacchasi}'' (How many times you come here?).

There is a useful \pali{paccaya} (suffix) added to some pronouns to make them number-related. It is \pali{ttaka} for m.\ and nt.\ or \pali{ttika} for f. I summarize this group of words in Table \ref{tab:ttaka}.\footnote{Sadd-Pad Ch.\,12, from \pali{Apica ya ta ki\d m etaiccetehi} onwards.}

\begin{table}[!hbt]
\centering
\caption{Terms with \pali{ttaka/ttika}}
\label{tab:ttaka}
\bigskip
\begin{tabular}{@{}*{2}{>{\itshape}l}l@{}} \toprule
\bfseries\upshape m. \& nt. & \bfseries\upshape f. & \bfseries Meaning \\ 
\midrule
kittaka & kittik\=a & how many?, how much?, how large? \\
yattaka & yattik\=a & which amount/size \\
tattaka & tattik\=a & that amount/size \\
ettaka & ettik\=a & this amount/size \\
\bottomrule
\end{tabular}
\end{table}

These terms when composed as such are no longer pronoun. They decline as normal nouns. To ask how many people there are in the world, you can also say this instead:

\palisample{loke kittak\=a jan\=a honti.}

You can use \pali{ya-ta} structure to say ``Write it down how many people in the world'' as follows:

\palisample{yattak\=a jan\=a loke, tattika\d m ga\d nana\d m likh\=ahi.}

The sentence above have to be rephrased first as ``Which amount of people in the world, write down that number.'' This is an imperative sentence. And the following is for ``I give to this amount of people.''

\palisample{aha\d m ettak\=ana\d m jan\=ana\d m demi.}

Another way to ask `how much' or `how long' or `how far' or `how \ldots' is to use particle \palibf{k\=iva}\footnote{See page \pageref{nip:kiiva}.} with a suitable adjective (or adverb). For example, you can ask ``How long have you lived here?'' as follows:

\palisample{k\=iva cira\d m tva\d m imasmi\d m vasi.}

Here is for ``How far is your school?''

\palisample{k\=iva d\=ur\=a tava p\=a\d thas\=al\=a hoti.}

Here is for ``How big is your house?''

\palisample{k\=iva mahanta\d m tava geha\d m hoti.}

And here is for ``How many books do you have?''

\palisample{k\=iva bahuk\=a(ni) tava potthak\=a(ni) santi.}

\clearpage
\phantomsection
\addcontentsline{toc}{section}{Ordinal Numbers}
\section*{Ordinal Numbers}

We use cardinal numbers in counting and we use ordinal numbers to tell the position in a series, such as the first (thing), the second (thing), and so on. All ordinal numbers are used as adjectives, so they can be of three genders. There are five endings that mark ordinal function, i.e.\ \pali{tiya, tha, \d tha, ma,} and \pali{\=i}. For more detail of these, see Appendix \ref{chap:taddhita}, page \pageref{par:sankhyataddhita}. I list some ordinal numbers in Table \ref{tab:ornum}.

\bigskip
\begin{longtable}[c]{@{}%
	>{\itshape\raggedright\arraybackslash}p{0.35\linewidth}%
	>{\itshape\raggedright\arraybackslash}p{0.33\linewidth}%
	>{\raggedright\arraybackslash}p{0.12\linewidth}@{}}
\caption{P\=ali ordinal numbers}\label{tab:ornum}\\
\toprule
\bfseries\upshape m.\ \& nt.& \bfseries\upshape f.& \bfseries\upshape Order\\ \midrule
\endfirsthead
\multicolumn{3}{c}{\tablename\ \thetable: P\=ali ordinal numbers (contd\ldots)}\\
\toprule
\bfseries\upshape m.\ \& nt.& \bfseries\upshape f.& \bfseries\upshape Order\\ \midrule
\endhead
\bottomrule
\ltblcontinuedbreak{3}
\endfoot
\bottomrule
\endlastfoot
%
pa\d thama & pa\d tham\=a & 1st \\
dutiya & dutiy\=a & 2nd \\
tatiya & tatiy\=a & 3rd \\
catuttha & catutth\=a, catutth\=i & 4th \\
pa\~ncama & pa\~ncam\=a, pa\~ncam\=i & 5th \\
cha\d t\d tha(ma) & cha\d t\d th\=a, cha\d t\d th\=i & 6th \\
sattama & sattam\=a, sattam\=i & 7th \\
a\d t\d thama & a\d t\d tham\=a, a\d t\d tham\=i & 8th \\
navama & navam\=a, navam\=i & 9th \\
dasama & dasam\=a, dasam\=i & 10th \\
ek\=adasama & ek\=adas\=i & 11th \\
\mbox{dv\=adasama, b\=arasama} & dv\=adas\=i, b\=aras\=i & 12th \\
terasama & teras\=i & 13th \\
catuddasama & \mbox{catuddas\=i, c\=atuddas\=i} & 14th \\
pa\d n\d narasama & pa\d n\d naras\=i & 15th \\
so\d lasama & so\d las\=i & 16th \\
sattarasama & sattaras\=i & 17th \\
a\d t\d th\=arasama & a\d t\d th\=aras\=i & 18th \\
ek\=unav\=isatima & ek\=unav\=isatim\=a &  19th \\
v\=isatima & v\=isatim\=a & 20th \\
ti\d msatima & ti\d msatim\=a & 30th \\
catt\=a\d l\=isatima & catt\=a\d l\=isatim\=a & 40th \\
pa\~n\~nasatima & pa\~n\~nasatim\=a & 50th \\
sa\d t\d thima & sa\d t\d thim\=a & 60th \\
sattatima & sattatim\=a & 70th \\
as\=itima & as\=itim\=a & 80th \\
navutima & navutim\=a & 90th \\
satama & satam\=a & 100th \\
sahassama & sahassam\=a & 1000th \\
\end{longtable}

Please look closely to f.\ 11th--18th, they take a slightly different pattern. After that the numbers follows a predictable pattern. Using these numbers are straightforward like other adjectives. You just take care of the gender properly. For example, ``My first son is 20 years old'' can be said as:

\palisample{mama pa\d thamo putto v\=isativasso hoti.}

And this is for ``Tonight is the fifteenth (night) of the month.''

\palisample{aya\d m ratti m\=asassa pa\~n\~naras\=i hoti.}

Asking for ordinal number, we use \palibf{katima} (m., nt.) and \palibf{katim\=i} (f.). For example, to the answer above we ask ``\pali{katim\=i, bhante, pakkhassa}''\footnote{Mv\,2.156} (Sir, of what fortnight [is tonight]?).

Another use of ordinal numbers which is a bit challenging is to use with \palibf{a\d d\d dha} (half). Like English, we can say ``a half of\ldots'' by using \pali{a\d d\d dha}. For example, 50 is \pali{a\d d\d dhasata\d m}, 500 is \pali{a\d d\d dhasahassa\d m}, and 5000 is \pali{a\d d\d dhadasasahassa\d m}. The terms are compounds. When they are broken down, \pali{a\d d\d dha} takes ins., e.g.\ \pali{a\d d\d dhena}, in the sense of `by a half.' But the numbers have to be modified by ordinals. That is to say, 50 is literally (and confusingly) ``the first hundred by a half,'' 150 is ``the second hundred by a half,'' 250 is ``the third hundred by a half,'' and so on. I summarize these in Table \ref{tab:half}.\footnote{For 150, 250, and 350, see Abhidh\=a 477--8.}

\begin{table}[!hbt]
\centering
\caption{The use of \pali{a\d d\d dha}}
\label{tab:half}
\bigskip
\begin{tabular}{l>{\itshape}l>{\itshape}l} \toprule
\bfseries Num & \upshape\bfseries Analyzed form & \upshape\bfseries Compound \\
\midrule
50 & a\d d\d dhena pa\d thama\d m sata\d m & a\d d\d dhasata\d m \\
150 & a\d d\d dhena dutiya\d m sata\d m & \texthl{diya\d d\d dha}sata\d m \\
250 & a\d d\d dhena tatiya\d m sata\d m & \texthl{a\d d\d dhateyya}sata\d m \\
350 & a\d d\d dhena catuttha\d m sata\d m & \texthl{a\d d\d dhu\d d\d dha}sata\d m \\
450 & a\d d\d dhena pa\~ncama\d m sata\d m & a\d d\d dhapa\~ncamasata\d m \\
\bottomrule
\end{tabular}
\end{table}

The rows with a color-highlighted part is irregular, so they should be remembered.\footnote{The formula is described in Kacc\,387, R\=upa\,411, Sadd\,819, and Mogg\,3.105--6. In Mogg\,3.106, one and a half can also be \pali{diva\d d\d dha}.} Numbers greater than those in the table follow the regular pattern of 450. Numbers in the range of thousands are rendered in the same way, e.g.\ 1500 is \pali{diya\d d\d dhasahassa\d m}. When the numbers are used with a noun, they go like this: for example, ``150 people'' is:

\palisample{a\d d\d dhena dutiya\d m jan\=ana\d m sata\d m.\sampleor[then]diya\d d\d dhajanasata\d m.}

And ``3,500 stars'' is:

\palisample{a\d d\d dhena catuttha\d m t\=arak\=ana\d m sahassa\d m.\sampleor[then]a\d d\d dhu\d d\d dhat\=arak\=asahassa\d m.}

From the compounds, we can split the numbers into two parts. The \pali{a\d d\d dha} part is used as adjectives, for it is formed by ordinals, thus its case has to be conform with the other. Hence, the above examples can be as follows:

\palisample{diya\d d\d dh\=ani janasat\=ani.\sampleor[and]a\d d\d dhu\d d\d dh\=ani t\=arak\=asahass\=ani.}

\section*{Exercise \ref{chap:num}}
Say these in P\=ali.
\begin{compactenum}
\item How many people are COVID-infected so far?
\item By 17th February 2021, it is 110,035,725.\footnote{data from \url{https://www.worldometers.info/coronavirus/}}
\item What are the most infected countries?
\item The first is America, around 28 millions, the second India, 11 millions, and the third is Brazil, 10 millions.
\item How about China?
\item It has 89,795 so far, 84th in the list.
\item What is the death rate now?
\item Around 2 percents. It is a dreadful disease indeed.
\item How long we will be in this pandemic state.
\item Since we have vaccination now, perhaps it may go on a few years.
\item Maybe this is an apocalypse, revenge of the nature.
\item How often have you watched movies recently?
\item Around a dozen this week.
\item Maybe that is too much.
\end{compactenum}
