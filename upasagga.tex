\chapter{\headhl{Upasagga} (Prefixes)}\label{chap:upasagga}

\pali{Upasagga} is a technical term in P\=ali. It is a word class, often bundled with \pali{nip\=ata} (particles) and called \pali{abyaya} or \pali{avyaya} (indeclinables) as a whole group. Unlike \pali{nip\=ata} that can stand alone as an independent unit, \pali{upasagga} normally has to be appended with other part, normally a verb or noun (adjective included), to modify the term's meaning. So, we can call an \pali{upasagga} roughly a prefix. However, it is not a prefix in general, because there are only twenty of them, namely \pali{\=a, u, ati, pati, pa, pari, ava, par\=a, adhi, abhi, anu, upa, apa, api, sa\d m, vi, ni, n\=i, su,} and \pali{du}.\footnote{Sadd-Sut Ch.\,27, \citealp[p.~880]{smith:sadd3}. R\=upa between 281 and 282 has the same list but different order. In Mogg\,5.131, after Payo\,253, and Niru\,288, \pali{n\=i} is not found but \pali{o}, still twenty altogether. In Niru\,288 there is an account that Kacc\=ayana sees \pali{o} as another form of \pali{ava}, so he does not include \pali{o} in the list. Whereas Moggall\=ana sees \pali{n\=i} as just a long form of \pali{ni}, so he does not include \pali{n\=i} in the list. As we shall see below, both sides have a good reason. Maybe 19 \pali{upasagga}s are more sensible. From my view, Moggall\=ana's reason is more convincing, because we can find terms with \pali{ava} and \pali{o} connecting to the same base but having different meaning. In official Thai P\=ali textbooks, \pali{n\=i} is not regarded as \pali{upasagga} by its own right, in line with Moggall\=ana's view, but \pali{ni} is split into two items. The first \pali{ni} means `down,' the second `out.' Thus twenty \pali{upasagga} is maintained. However, to make my approach unified, in this present book we will mostly follow Kacc\=ayana-Saddan\=ita school.}

It is worth reading the summary Aggava\d msa wrote at the end of the \pali{upasagga} part. So, I quote it in full.

\begin{quote}
\pali{Eva\d m v\=isati uppasagg\=a anekatth\=a hutv\=a n\=am\=akhy\=atavisesak\=arak\=a bhavanti. Upecca n\=ama\~nca \=akhy\=ata\~nca sajanti lagganti tesa\d m attha\d m visesent\=i'ti upasagg\=a.}\footnote{Sadd-Sut Ch.\,27, \citealp[p.~886]{smith:sadd3}} \\[1.5mm]
``There are twenty \pali{upasagga}s, with various meanings, which specify (the meaning of) nouns and verbs. Applying to nouns and verbs they approach and adhere to them, specifying their meaning, thus they are called \pali{upasagga}.''\footnote{\citealp[p.~125]{collins:grammar}}
\end{quote}

Some words seem to be used in a similar way, but not counted as \pali{upasagga}. For instance, new students often mistake \pali{mah\=a} (big) as \pali{upasagga}, as we find in \pali{mah\=ajano} (the public, masses of people, or a big person literally). This word is a compound which \pali{mah\=a} is the elided form of \pali{mahanto} (big). Another one is \pali{`a'} in \pali{adhamma} (false doctrine). This is also not \pali{upasagga}. It is negative particle \pali{a}. So, it is helpful to keep in mind all twenty \pali{upasagga}s. If you find something functions alike but not in the list, suspect it as an independent term derived from other form.

I reorder \pali{upasagga}s alphabetically following Steven Collins and list all of them in the table below.\footnote{adapted from \citealp[p.~125]{collins:grammar}} The meanings given in the table are just a rough picture to help you make a quick grab. Each \pali{upasagga} has several strands of meaning. It is better to go into examples of them.

\bigskip
\begin{longtable}[c]{%
	>{\itshape\raggedright\arraybackslash}p{0.18\linewidth}%
	>{\raggedright\arraybackslash}p{0.48\linewidth}%
	>{\raggedleft\arraybackslash}p{0.1\linewidth}}
\caption*{List of 20 \pali{Upasagga}s}\\
\toprule
\bfseries Upasagga & \bfseries\upshape Meaning & \bfseries\upshape Page \\ \midrule
\endfirsthead
\toprule
\bfseries Upasagga & \bfseries\upshape Meaning & \bfseries\upshape Page \\ \midrule
\endhead
\bottomrule
\ltblcontinuedbreak{3}
\endfoot
\bottomrule
\endlastfoot
%
ati & \mbox{beyond, too much, very much} & \pageref{upasagga:ati} \\
adhi & towards, up to, over, above & \pageref{upasagga:adhi} \\
anu & following, after & \pageref{upasagga:anu} \\
apa & away from & \pageref{upasagga:apa} \\
api, pi & on, over & \pageref{upasagga:api} \\
abhi & towards, over & \pageref{upasagga:abhi} \\
ava, o & down, away & \pageref{upasagga:ava} \\
\=a & near to, away & \pageref{upasagga:aa} \\
u, ud & up, out of, away from & \pageref{upasagga:u} \\
upa & towards, be subordinate to & \pageref{upasagga:upa} \\
du, dur & bad, wrong & \pageref{upasagga:du} \\
ni & down, out & \pageref{upasagga:ni} \\
n\=i & away, out & \pageref{upasagga:nii} \\
pa & towards, onward & \pageref{upasagga:pa} \\
pati, pa\d ti & back to, opposite & \pageref{upasagga:pati} \\
par\=a & on, over & \pageref{upasagga:paraa} \\
pari & round, about, complete & \pageref{upasagga:pari} \\
vi & apart, separate & \pageref{upasagga:vi} \\
sa\d m & together & \pageref{upasagga:sadm} \\
su & well, right, very & \pageref{upasagga:su} \\
\end{longtable}

The best way to learn how all these work is to see a lot of examples. So I show several of them below for each item. Sometimes the meaning of the terms does not go straightforwardly, so you have to add some imagination or think it figuratively. In traditional approach, these are explained by their nuances of meaning. I skip that meaning classification because I found some of them out of place and I do not want to rationalize them. It is better to exercise your mental creativity by extending the main theme to the possible meanings. This somehow brings a lot of fun, like, say, Tarot reading. To know them statistically, I mark instances with an asterisk (*) showing that the term is the most frequent instance found, among its group, in the collection. In Appendix \ref{chap:samasa}, page \pageref{sec:abyayi}, there is a type of compound related to \pali{upasagga}. Please see there for more information.

\section*{\fbox{\pali{Ati}}}\label{upasagga:ati}
\begin{compactitem}
\item \pali{ativiya}* (\pali{ati + iva}) = (ind.) excessively, very much
\item \pali{atiruccati} (\pali{ati + ruca}) = (v.) to outshine
\item \pali{at\=ito} (\pali{ati + i + ta}) = (n. m.) the past, (time) gone beyond
\item \pali{accanta}\footnote{According to Sandi rules, \pali{ti} can become \pali{cc}, Kacc\,19; R\=upa\,22; Sadd\,46; Mogg\,1.30, 1.48--9.} (\pali{ati + anta}) = (adj.) extreme
\item \pali{atikusalo} (\pali{ati + kusala}) = (adj.) very skillful, very clever
\item \pali{atikkodho} (\pali{ati + kodha}) = (n. m.) intense anger
\item \pali{ativuddhi} (\pali{ati + vuddhi}) = (n. f.) great prosperity
\end{compactitem}

\section*{\fbox{\pali{Adhi}}}\label{upasagga:adhi}
\begin{compactitem}
\item \pali{adhipp\=ayo}* (\pali{adhi + p\=aya}) = (n. m.) intention
\item \pali{adhis\=ila\d m} (\pali{adhi + s\=ila}) = (n. nt.) higher morality
\item \pali{adhipati} (\pali{adhi + pati}) = (n. m.) head leader
\item \pali{adhiseti} (\pali{adhi + si}) = (v.) to lie on 
\item \pali{adhirohati} (\pali{adhi + ruha}) = (v.) to ascend, to climb
\item \pali{adhirohan\=i} (\pali{adhi + ruha}) = (n. f.) a ladder
\item \pali{adhibhavati} (\pali{adhi + bh\=u}) = (v.) to overpower
\item \pali{adhi\d th\=ana\d m} (\pali{adhi + \d th\=a}) = (n. nt.) resolution, determination
\item \pali{adhimokkho} (\pali{adhi + muca}) = (n. m.) decision, determination 
\item \pali{adhigacchati} (\pali{adhi + gamu}) = (v.) to attain 
\end{compactitem}

\section*{\fbox{\pali{Anu}}}\label{upasagga:anu}
\begin{compactitem}
\item \pali{anuj\=an\=ati}* (\pali{anu + \~n\=a}) = (v.) to allow, to give permission
\item \pali{anugacchati} (\pali{anu + gamu}) = (v.) to follow 
\item \pali{anusayo} (\pali{anu + si}) = (n. m.) a dormant disposition 
\item \pali{anuratha\d m} (\pali{anu + ratha}) = (n. nt.) rear part of a car\footnote{Why is it not a following car? It can be if you use as m., hence \pali{anuratho}. Used as nt., this should be a part of a car.}
\item \pali{anuratto} (\pali{anu + ranja + ta}) = (adj. p.p.) attached to, fond of
\item \pali{anur\=upa\d m} (\pali{anu + r\=upa}) = (adj.) suitable
\item \pali{anva\d d\d dham\=asa\d m}\footnote{This instance and the followings come from Niru\,288.} (anu + a\d d\d dha + m\=asa) = (adv.) every fortnight
\item \pali{anusa\d mvacchara\d m} (anu + sa\d mvacchara) = (adv.) every year
\item \pali{anubuddho} (anu + buddha) = (n. m.) a lesser Buddha, one enlightened after the Buddha
\item \pali{anuthero} (anu + thera) = (n. m.) one who comes next to the elder (PTSD)
\end{compactitem}

There are some peculiar uses of \pali{anu} that I leave out, but some are worth mentioning anyway. \pali{Anu}, together with a few others\footnote{In R\=upa\,288, Sadd\,584, \pali{pati} and \pali{pari} are added. In Mogg\,2.8, Niru\,298, \pali{abhi} is mentioned.}, is called \pali{kammappavacan\=iya} (calling for object?).\footnote{Sadd\,582} When this kind of thing happens, it has accusative form.\footnote{Kacc\,299, R\=upa\,288, Sadd\,586, Mogg\,2.8, Niru\,298} And this has six possible meanings, namely \pali{lakkha\d na} (mark, sign), \pali{saha} (with), \pali{h\=ina} (inferior), \pali{itthambh\=utakkh\=ana} (certain special characteristic), \pali{bh\=aga} (part), and \pali{vicch\=a} (repetition).

As a sign, \pali{anu} works in this way, for example, \pali{pabbajitamanupabbaji\d msu}\footnote{Kacc\,299} or \pali{pabbajitamanu pabbaji\d msu} (\pali{pabbajita\d m + anu + pabbaji\d msu}).\footnote{R\=upa\,288, Sadd\,583} I have to be precise to show you something. This means when the Bodhisatta went forth (\pali{pabbajita\d m}), it is taken as a sign by which people went forth likewise (\pali{pabbaji\d msu}). I try searching this instance in the canon then I find this ``\pali{\ldots mah\=agovinda\d m br\=ahma\d na\d m ag\=arasm\=a anag\=ariya\d m pabbajita\d m \textbf{anu}pabbaji\d msu}.''\footnote{D2\,326 (DN\,19)} I translate this as ``(many people) went forth following Brahman Mah\=agovinda who has gone forth from the lay life to homelessness.'' Where is the sign? There is no such a thing when you treat \pali{anu} as a part of the following term. When you (mis)take it as an independent term or a part of the former term, you have to find some reason of that. Thus peculiar accounts come along.

Another example of sign is ``\pali{rukkha\d m anu vijjotate vijju}''\footnote{R\=upa between 281 and 282. In R\=upa\,288 \pali{rukkha\d m pati \ldots} and \pali{rukkha\d m pari \ldots} are also shown. In Sadd\,584, there are \pali{rukkha\d m pati/pari/anu cando} (The moon shines on a tree). In Mogg\,2.8, Niru\,298, it is \pali{rukkhamabhi vijjotate vijju}.} (Lightning strikes a tree). The explanation goes as you expect that \pali{anu} marks the tree as a sign that lightning can see and hit. I found no instance of this, fortunately.

In the sense of \pali{saha}, here is an example, \pali{nadimanvavasit\=a sen\=a}\footnote{This is from R\=upa between 281 and 282. In R\=upa\,288 \pali{nadimanvavasit\=a b\=ar\=a\d nas\=i} (nearby-rivered Benares) is also exemplified. However, in Sadd-Sut Ch.\,27, it is \pali{nadi\d m anav\=avasit\=a sen\=a} \citep[p.~883]{smith:sadd3}.} (an army that is nearby a river). This can be broken down to \pali{nadi + anu + ava + si + ta + sen\=a}. This instance is a bit complicated to analyze. Let me try unraveling this. \pali{Avasita} is probably a past participle form of \pali{avasseti} or \pali{avassayati} (to lean against, lie down on). If you take \pali{anu} as a part of this verb, it makes a perfect sense. Hence, the army is leaning against and following a river. However, the tradition explains that \pali{anu} marks instrumental case to \pali{nadi}. Therefore, the army is leaning against \emph{with} a river. That sounds reasonable, but how and why \pali{upasagga} can do \pali{vibhatti} job is still a mystery to me.

I found an example, perhaps, from the oldest layer of the canon, but with \pali{pati}: ``\pali{nadi\d m nera\~njara\d m pati}''\footnote{Snp\,3.427; Thig\,13.307, 310} (nearby Nera\~njar\=a river). Another similar one is ``\pali{mig\=a nera\~njara\d m pati}''\footnote{Ja\,15:39} ([We were born as] deers nearby the Nera\~njar\=a). As the explanation goes, we can use \pali{anu} instead of \pali{pati} here. This unusual use of \pali{upasagga} happens only in old poetic works, I think. 

Here is an example of inferiority, \pali{anu s\=ariputta\d m pa\~n\~nav\=a} (one who is less wise than Ven.\,S\=ariputta).\footnote{R\=upa\,288, Sadd\,583} This makes some sense when we take \pali{anu} as `lesser' like \pali{anubuddho} above. Then \pali{anus\=ari\-putta} means ``minor S\=ariputta'' or ``little S\=ariputta'' or ``the second S\=ariputta.'' So, \pali{anus\=ariputta\d m pa\~n\~nav\=a} means one who wise as a lesser S\=ariputta. This sounds a bit positive. Why does \pali{anu} stands apart in the example? That looks odd. I have no idea. Only instance found in the canon that \pali{anu} stands alone is in a verse: ``\pali{S\=im\=a mahat\=i nadiy\=a, anu dve khuddak\=ani ca}.''\footnote{Mv\,2.183} This possibly means a boundary in the river, a big one, and two small ones successively.\footnote{See \citealp[p.~1581]{horner:discipline}.} So, it is better to treat \pali{anu} here as an particle.

I will stop explaining \pali{anu} here, because I have problems with the rest of meanings listed above. I do not want to pretend that I understand them. Here is the lesson from my observation. It seems that the explanations alien to P\=ali literature as a whole come from Sanskrit grammar of P\=a\d nini which P\=ali grammarians resorted to. As a result, we have only half-baked explanations and outlandish examples. Moggall\=ana might see this and did not explain \pali{anu} as we find in Padar\=upasiddhi and Saddan\=iti. But in Pay\=agasiddhi (after Payo\,253), the same set of meanings and examples are found. Le\d d\=i Say\=a\d do discards these altogether in Niruttid\=ipan\=i (after Niru\,288) and presents more familiar meanings and examples. From my quite a long discussion of \pali{anu} here, one might not gain much knowledge on how to use it. But I hope that it can shed some light to the characteristics of P\=ali grammatical textbooks and the language itself. My suggestion is that you should follow simple examples that are understandable in canonical context. Do not ever copy bizarre examples. And do not haste grasping everything you are told, even by renowned works.

\section*{\fbox{\pali{Apa}}}\label{upasagga:apa}
\begin{compactitem}
\item \pali{apaneti}* (\pali{apa + ni}) = (v.) to lead away, to remove
\item \pali{apagacchati} (\pali{apa + gamu}) = (v.) to go away, to disappear 
\item \pali{apagabbho}\footnote{There is an interesting instance of this in the Vinaya: \pali{Apagabbho bhava\d m gotamo} (Buv1\,10). At first it is used to reproach the Buddha, like ``How bad conception are you!'' I.\,B.\ Horner uses `withdrawn' here \citep[p.~88]{horner:discipline}. Then the Buddha twists the meaning to ``there is no further rebirth for me.''} (\pali{apa + gabbha}) = (adj.) going away from the womb, not destined to another rebirth (PTSD)
\item \pali{aparajjhati} (\pali{apa + r\=adha + ya}) = (v.) to offend against, to go wrong\footnote{\pali{R\=adheti} mean `to succeed' (see PTSD). When \pali{apa} is added, it means like ``to go away from success.''} 
\item \pali{apas\=al\=aya \=ayanti v\=a\d nij\=a}\footnote{Kacc\,272, R\=upa\,309, Sadd-Sut Ch.\,27, Mogg\,2.27, Payo\,284, Niru\,313} = Avoiding entering the hall, Merchants come.\footnote{Despite strange looking, this is understandable. With ins., \pali{apas\=al\=aya} functions like an adverb. \pali{\=Ayanti} is pl.\ of \pali{\=ayati} (\pali{\=a + y\=ati}) means `to come.'}
\end{compactitem}

When the sense of `away' is clear, verbs with \pali{apa} normally take ablative object (\dots away from \dots).

\section*{\fbox{\pali{Api}}}\label{upasagga:api}
\begin{compactitem}
\item \pali{pidahati}* (\pali{api + dh\=a}) = (v.) to cover, to close, to conceal
\item \pali{apidahati} (\pali{api + dh\=a}) = (v.) to cover up, to obstruct
\item \pali{apidh\=ana\d m} (\pali{api + dh\=a}) = (n. nt.) a cover, a lid 
\item \pali{api\d lahati} (\pali{api + naha}) = (v.) to bind on, to put on, to adorn (also \pali{api\d landhati}, but more often \pali{pilandhati})
\item \pali{apil\=apa\d na\d m} (\pali{api + l\=apana}) = (n. nt.) counting up, repetition
\end{compactitem}

I have problems with examples illustrated in the traditional textbooks on this \pali{upasagga}. Most of them, of not all, treat \pali{api} as a particle, meaning `even' or `yet.' So, they look very confusing whether it is a prefix or not. Hence, I left out all of them and propose a more sensible instances found in PTSD.

\section*{\fbox{\pali{Abhi}}}\label{upasagga:abhi}
\begin{compactitem}
\item \pali{abhiv\=adeti}* (\pali{abhi + vad\=i}) = (v.) to bow down, to salute 
\item \pali{abhij\=an\=ati} (\pali{abhi + \~n\=a}) = (v.) to know by experience, to know fully or thoroughly
\item \pali{abhimukho} (\pali{abhi + mukha}) = (adj.) facing, turned towards, face-to-face with 
\item \pali{abhikkamati} (\pali{abhi + kamu}) = (v.) to proceed, to step forwards
\item \pali{abhidhammo} (\pali{abhi + dhamma}) = (n. m.) special doctrine, the Abhidhamma 
\item \pali{abhivassati} (\pali{abhi + vassa}) = (v.) to rain heavily
\item \pali{abhiruhati} (\pali{abhi + ruha}) = (v.) to ascend, to go up 
\item \pali{abhij\=ato} (\pali{abhi + j\=ata}) = (adj.) well-born, of noble birth 
\item \pali{abhir\=upa} (\pali{abhi + r\=upa}) = (adj.) handsome, beautiful, lovely
\end{compactitem}

\section*{\fbox{\pali{Ava}}}\label{upasagga:ava}
\begin{compactitem}
\item \pali{avaseso}* (\pali{ava + sesa}) = (adj.) remaining
\item \pali{avakkhipati} (\pali{ava + khipa}) = (v.) to throw down, to drop 
\item \pali{avakkhittacakkhu} (\pali{ava + khipa + ta + cakkhu}) = (adj.) having cast-down eyes
\item \pali{omu\~ncati} (\pali{o + muca}) = (v.) to take off, to undress, to unfasten 
\item \pali{omukkaup\=ahano} (\pali{o + muca + ta + up\=ahana}) = (adj.) having shoes taken off 
\item \pali{avakokila\d m vana\d m} (\pali{ava + kokila + vana}) = a cuckoo-departed forest
\item \pali{avaj\=an\=ati} (\pali{ava + \~n\=a}) = (v.) to despise, to scorn
\item \pali{avama\~n\~nati} (\pali{ava + mana}) = (v.) to slight, to disregard, to despise
\item \pali{avagacchati} (\pali{ava + gamu}) = (v.) to understand, to attain 
\item \pali{vod\=ana\d m} (\pali{vi + ava + d\=a}) = (n. nt.) purity 
\item \pali{avadh\=ara\d na\d m} (\pali{ava + dhara}) = (n. nt.) affirmation, emphasis, selection 
\item \pali{avak\=aso} (\pali{ava + k\=asa}) = (n. m.) an opportunity, a chance, a space (also \pali{ok\=aso})
\item \pali{avaharati} (\pali{ava + hara}) = (v.) to steal, to take away
\item \pali{ocarati} (\pali{o + cara}) = (v.) to be after something, to go into, to search
\item \pali{ocarako} (\pali{o + cara}) = (n. m.) an informant scout, a spy, an investigator
\item \pali{avarundhati} (\pali{ava + rudhi}) = (v.) to put under restraint, to put into one's harem as subsidiary wife
\item \pali{orodho} (\pali{o + rudhi}) = (n. m.) a harem, a confinement, a concubine
\end{compactitem}

\section*{\fbox{\pali{\=A}}}\label{upasagga:aa}
\begin{compactitem}
\item \pali{\=apatti}* (\pali{\=a + pada}) = (n. f.) an ecclesiastical offense
\item \pali{\=agacchati} (\pali{\=a + gamu}) = (v.) to come
\item \pali{\=arohati} (\pali{\=a + ruha}) = (v.) to ascend, to climb
\item \pali{\=apajjati} (\pali{\=a + pada}) = (v.) to get into, to meet with, to undergo 
\item \pali{\=aka\.nkhati} (\pali{\=a + kakhi}) = (v.) to desire, to wish for 
\item \pali{\=ali\.ngati} (\pali{\=a + lagi}) = (v.) to embrace, to enfold
\item \pali{\=arabhati} (\pali{\=a + rabha}) = (v.) to begin 
\item \pali{\=ad\=ati} (\pali{\=a + d\=a}) = (v.) to take up, to accept, to grasp, to seize (also \pali{\=adiyati} in the same meaning)
\item \pali{\=alambati} (\pali{\=a + labi}) = (v.) to hang on to, to take hold of, to fasten to
\item \pali{\=avasati} (\pali{\=a + vasa}) = (v.) to live in, to inhabit, to reside
\item \pali{\=as\=idati} (\pali{\=a + sada}) = (v.) to come together (to sit by), to come or go near, to approach (PTSD) 
\item \pali{\=asanno} (\pali{\=a + sada}) = (adj.) near (p.p.\ of \pali{\=as\=idati})
\item \pali{\=amanteti} (\pali{\=a + manta}) = (v.) to call, to address, to invite
\item \pali{\=apabbat\=a khetta\d m} (\pali{\=a + pabbata + khetta}) = a field stretching to a mountain\footnote{This is a stock example found in all textbooks. I am curious at first why these two terms do not take the same case as we treat \pali{\=apabbata} as an adjective. We can also see that \pali{\=apabbat\=a} is in abl. So, it can be translated literally as ``a field stretching from a mountain.'' With \pali{y\=ava}, abl.\ can also mean `up to' (see page \pageref{par:abltill}). In Niru\,341, the formula states that when \pali{pari, apa, \=a, bahi, tiro, pure, } or \pali{pacch\=a} is compounded with a noun, the term can be in ablative case. However, I think it still makes sense to say \pali{\=apabbata\d m khetta\d m}. I find that \pali{\=apabbatassa khettassa} is used in Niru\,490.}
\item \pali{\=akum\=ara\d m yaso kacc\=ayanassa} = Ven.\,Kacc\=ayana's fame spreading to children\footnote{This instance looks strange to me. The function of \pali{\=akum\=ara\d m} is unclear. If it takes acc., it can be an adverb. If we put \pali{hoti} in the sentence, it looks clearer. In R\=upa\,336, there is an analytic sentence read ``\pali{\=akum\=arehi yaso kacc\=ayanassa \=akum\=ara\d m}'' (\pali{\=akum\=ara\d m} is Ven.\,Kacc\=ayana's fame spreading to children). In Sadd\,696, ``\pali{\=a kom\=ar\=a yaso kacc\=ayanassa \=akom\=ara\d m}'' is found instead. It seems that ablative case is used here. In Niru\,341, all these variations are mentioned.}
\end{compactitem}

As you may realize, sometimes \pali{\=a} adds nothing to the meanings, even though the textbooks have explanations for that anyway. It works much like a filler sometimes. When words do not come up, you say `aa\dots' or `err\dots' to fill the gap. I think, perhaps, that is how it comes. If my speculation sounds silly, just ignore it.

\section*{\fbox{\pali{U}}}\label{upasagga:u}
\begin{compactitem}
\item \pali{uppajjati}* (\pali{u + pada}) = (v.) to come out, to be born, to arise, to be produced
\item \pali{uggacchati} (\pali{u + gamu}) = (v.) to rise, to go up\footnote{For example, \pali{aru\d no uggacchati} means ``The dawn/sun is rising.''}
\item \pali{u\d t\d thahati} (\pali{u + \d th\=a}) = (v.) to rise, to stand up, to get up\footnote{For example, \pali{\=asan\=a u\d thito} means ``got up from the seat.''} (also \pali{u\d th\=ati})
\item \pali{ugga\d nh\=ati} (\pali{u + gaha}) = (v.) to take up, to acquire, to learn
\item \pali{ukkhipati} (\pali{u + khipa}) = (v.) to hold up, to take up 
\item \pali{ubbhavo} (\pali{u + bh\=u}) = (n. m.) birth, origination, production 
\item \pali{ussahati} (\pali{u + saha}) = (v.) to be able, to be fit for, to venture, to strive\footnote{For example, \pali{ussahati gantu\d m} means ``be able to go.''}
\item \pali{uddisati} (\pali{u + disa}) = (v.) to propose, to point out, to appoint, to specify
\item \pali{udikkhati} (\pali{ud + ikkha}) = (v.) to look at, to survey, to perceive 
\end{compactitem}

\section*{\fbox{\pali{Upa}}}\label{upasagga:upa}
\begin{compactitem}
\item \pali{upasa\.nkamati}* (\pali{upa + sa\d m + kamu}) = (v.) to go up to, to approach
\item \pali{upagacchati} (\pali{upa + gamu}) = (v.) to approach, to undergo, to undertake
\item \pali{upanis\=idati} (\pali{upa + ni + sada}) = (v.) to sit close to
\item \pali{upakaroti} (\pali{upa + kara}) = (v.) to help, to support, to serve 
\item \pali{upa\d t\d thahati} (\pali{upa + \d th\=a}) = (v.) to stand near, to wait on, to attend on, to look after, to nurse (also \pali{upa\d t\d th\=ati})
\item \pali{upanagara\d m} (\pali{upa + nagara}) = (n. nt.) a suburb
\item \pali{upapajjati} (\pali{upa + pada}) = (v.) to get to, to be reborn in, to originate, to rise\footnote{See also the entry in PTSD, comparing to \pali{uppajjati}.}
\item \pali{upekkhati} (\pali{upa + ikkha}) = (v.) to look on, to be disinterested
\item \pali{upam\=ana\d m} (\pali{upa + m\=a}) = (n. nt.) a simile, a parable, a comparison 
\item \pali{upasampajjati} (\pali{upa + sa\d m + pada}) = (v.) to attain, to enter on, to become fully ordained
\item \pali{upavasati} (\pali{upa + vasa}) = (v.) to observe the fast day
\item \pali{upavadati} (\pali{upa + vada}) = (v.) to tell (secretly) against, to tell tales, to insult, to blame
\item \pali{upasagga} (\pali{upa + sajja}) = (n. m.) a danger, a trouble; the \pali{upasagga}s 
\item \pali{up\=adiyati} (\pali{upa +\=a + d\=a}) = (v.) to take hold of, to grasp, to cling to 
\item \pali{up\=ay\=aso} (\pali{upa + \=ay\=asa}) = (n. m.) a trouble, a turbulence, a tribulation, an unrest, a grief
\item \pali{upanissayati} (\pali{upa + ni + si}) = (v.) to depend on, to rely on
\item \pali{upar\=aj\=a} (\pali{upa + r\=aja}) = (n. m.) a secondary king, a deputy king 
\end{compactitem}

\section*{\fbox{\pali{Du}}}\label{upasagga:du}
\begin{compactitem}
\item \pali{dukkha\d m}* (\pali{du + kha}) = (n. nt.) suffering
\item \pali{duggandha} (\pali{du + gandha}) = (adj.) having a bad smell 
\item \pali{dubbhikkha\d m} (\pali{du + bhikkha}) = (n. nt.) a famine, scarcity of food 
\item \pali{dukkata\d m} (\pali{du + kara + ta}) = (n. nt.) a wrong action
\item \pali{dukkaro} (\pali{du + kara}) = (adj.) dificult to do 
\item \pali{dusassa\d m} (\pali{du + sassa}) = (adj.) having bad crops 
\item \pali{dubba\~n\~no} (\pali{du + va\~n\~na}) = (adj.) of bad color, discolored, ugly
\item \pali{dummukho} (\pali{du + mukha}) = (adj.) having a sad face
\item \pali{durutta\d m} (\pali{dur + utta}) = (n. nt.) bad speech 
\item \pali{duppa\~n\~no} (\pali{du + pa\~n\~n\=a}) = (adj.) foolish
\end{compactitem}
In most cases, when the meaning allows, you can replace \pali{du} with \pali{su} to make the term positively opposite. See \pali{su} below.

\section*{\fbox{\pali{Ni}}}\label{upasagga:ni}
\begin{compactitem}
\item \pali{ni\d t\d thito}* (\pali{ni + \d th\=a + ta}) = (p.p.) was finished, was completed
\item \pali{nisseso} (\pali{ni + sesa}) = (adj.) whole, entire, no remainder (\pali{sesa\d m} = remainder)
\item \pali{nirutti} (\pali{ni + vaca}) = (n. f.) a language, philology\footnote{In R\=upa after 281, it is explained as \pali{nissese nirutti}. This can be rendered as ``\pali{nirutti} is in the meaning of entirety (of utterances).''}
\item \pali{nigacchati} (\pali{ni + gamu}) = (v.) to go down to, to undergo, to enter
\item \pali{niggacchati} (\pali{ni + gamu}) = (v.) to go out, to disappear, to proceed from
\item \pali{nikkileso} (\pali{ni + kilesa}) = (adj.) free from depravity, unstained
\item \pali{niddh\=ara\d na\d m} (\pali{ni + dhara}) = (n. nt.) withdrawal\footnote{The explanation found in R\=upa after 281 is \pali{n\=ihara\d ne niddh\=ara\d na\d m} (In taking out is \pali{niddh\=ara\d na\d m}).}
\item \pali{nivasati} (\pali{ni + vasa}) = (v.) to live, to dwell, to inhabit
\item \pali{nikhanati} (\pali{ni + khanu}) = (v.) to dig into, to bury 
\item \pali{nimmakkhiko} (\pali{ni + makkhika}) = (adj.) free from flies 
\item \pali{niv\=areti} (\pali{ni + vara}) = (v.) to prevent, to keep back, to forbid, to obstruct
\item \pali{nibbano} (\pali{ni + vana}) = (adj.) free from craving (without forest, woodless)
\item \pali{nikkhamati} (\pali{ni + kamu}) = (v.) to go forth from, to come out of (+ abl.)
\item \pali{nimmin\=ati} (\pali{ni + m\=a}) = (v.) to measure out, to fashion, to build, to make by miracle 
\item \pali{nicchayo} (\pali{ni + ci}) = (n. m.) resolution, determination, discrimination 
\item \pali{niddeso} (\pali{ni + dis\=i}) = (n. m.) description, analytic explanation 
\item \pali{nidassana\d m} (\pali{ni + dassana}) = (n. nt.) an example, evidence, comparison
\item \pali{nis\=ameti} (\pali{ni + samu}) = (v.) to attend to, to listen, to observe, to be careful of
\item \pali{ni\d t\d th\=ati} (\pali{ni + \d th\=a}) = (v.) to be at and end, to be finished (often found in p.p.\ \pali{ni\d t\d thita})
\item \pali{nipu\d no} (\pali{ni + pu\d na}) = (adj.) clever, skillful
\item \pali{nir\=ah\=aro} (\pali{ni + \=ah\=ara}) = (adj.) foodless, fasting
\item \pali{nirupamo} (\pali{ni + upama}) = (adj.) incomparable
\end{compactitem}
There is an observation from Thai tradition worth mentioning here. Generally speaking, \pali{ni} has two shades of meaning: (1) down/in and (2) out/free from. When composed with other terms, these two nuances behave differently. When it means `down' or `in,' it connects directly to the base without doubling the first consonant, e.g.\ \pali{nikhanati, nigacchati,} and \pali{nivasati}. When it means `out' or `free from,' double consonants are often seen, e.g.\ \pali{niggacchati, nikkhamati,} and \pali{nisseso}. But if the base has the first character of \pali{avagga} (\pali{ya, ra, la, va, sa, ha, \d l}), \pali{ni} becomes \pali{n\=i} (see below). That is a good reason to regard \pali{n\=i} as lengthened \pali{ni}, not an another \pali{upasagga}. Furthermore, when this second sense connects to a term started with a vowel, it becomes \pali{nir}, e.g.\ \pali{nir\=ah\=aro,} and \pali{nirupamo}.

\section*{\fbox{\pali{N\=i}}}\label{upasagga:nii}
\begin{compactitem}
\item \pali{n\=iharati}* (\pali{n\=i + hara}) = (v.) to take out, to throw out, to drive out
\item \pali{n\=ivara\d na\d m} (\pali{n\=i + vara}) = (n. nt.) an obstacle, a hindrance
\item \pali{n\=iraso} (\pali{n\=i + rasa}) = (adj.) tasteless, sapless, dried up, withered
\end{compactitem}

\section*{\fbox{\pali{Pa}}}\label{upasagga:pa}
\begin{compactitem}
\item \pali{pa\d t\d th\=aya}* (\pali{pa + \d th\=a + tv\=a}) = (ind.) beginning with, henceforth, from the time of
\item \pali{pakkamati} (\pali{pa + kamu}) = (v.) to step forwards, to go away
\item \pali{pakkosati} (\pali{pa + kusa}) = (v.) to call, to summon
\item \pali{pakaroti} (\pali{pa + kara}) = (v.) to effect, to perform, to prepare
\item \pali{pa\~n\~n\=a} (\pali{pa + \~n\=a}) = (n. f.) wisdom, knowledge, insight\footnote{In R\=upa after 281, the explanation of this is very broad: \pali{pak\=are pa\~n\~n\=a}. Since \pali{pak\=ara} means ``mode, method, manner, way,'' \pali{pa\~n\~n\=a} may mean to know various things in general.}
\item \pali{pa\d n\=ito} (\pali{pa + n\=i + ta}) = (adj.) brought out, raised, exalted, excellent
\item \pali{pabh\=u} (\pali{pa + bh\=u}) = (n. m.) master, ruler, owner
\item \pali{pakkhippati} (\pali{pa + khipa}) = (v.) to put in, to enclose in, to throw into
\item \pali{pass\=aso} (\pali{pa + s\=asa}) = (n. m.) inhaled breath
\item \pali{pavasati} (\pali{pa + vasa}) = (v.) to dwell abroad, to be away from home
\item \pali{p\=acariyo} (\pali{pa + \=acariya}) = (n. m.) a teacher of teacher
\item \pali{paputto} (\pali{pa + putta}) = (n. m.) a grandson
\item \pali{panatt\=a} (\pali{pa + nattu}) = (n. m.) a great grandson
\item \pali{pabhavati} (\pali{pa + bh\=u}) = (v.) to flow down, to originate\footnote{An example given by R\=upa is ``\pali{himavat\=a ga\.ng\=a pabhavati}'' (The Ganges originates from the Himalaya).}
\item \pali{pah\=uto} (\pali{pa + h\=u}) = (adj.) sufficient, abundant
\item \pali{pas\=idati} (\pali{pa + sada}) = (v.) to become bright, to brighten up, to be purified (p.p.\ \pali{pasanna})
\item \pali{pasannamudaka\d m} (\pali{pa + sada + udaka}) = (n. nt.) clear water
\item \pali{pa\d nidahati} (\pali{pa + ni + dh\=a}) = (v.) to put forth, to direct, to intend, to aspire to, to long for
\item \pali{pa\d t\d thahati} (\pali{pa + \d th\=a}) = (v.) to put down, to set down, to provide (often found in abs.\ \pali{pa\d t\d th\=aya})
\end{compactitem}

\section*{\fbox{\pali{Pati}}}\label{upasagga:pati}
\begin{compactitem}
\item \pali{paccayo}* (\pali{pati + i}) = (n. m.) cause, motive, requisite
\item \pali{pa\d tikkamati} (\pali{pati + kamu}) = (v.) to step backwards, to return (opposite of \pali{abhikkamati})
\item \pali{pa\d tigacchati} (\pali{pati + gamu}) = (v.) to give up, to leave behind
\item \pali{pa\d tikaroti} (\pali{pati + kara}) = (v.) to redress, to repair, to act against
\item \pali{pa\d tinissajjati} (\pali{pati + ni + saja}) = (v.) to give up, to renounce, to forsake
\item \pali{pa\d tinivattati} (\pali{pati + ni + vatu}) = (v.) to turn back again
\item \pali{pa\d tidad\=ati} (\pali{pati + d\=a}) = (v.) to give back, to restore
\item \pali{pa\d tisedhati} (\pali{pati + sedha}) = (v.) to refuse, to prevent, to prohibit (also \pali{pa\d tisedheti})
\item \pali{pa\d tir\=upo} (\pali{pati + r\=upa}) = (adj.) fit, proper, suitable
\item \pali{pa\d tir\=upako} (\pali{pati + r\=upaka}) = (adj.) like, resembling, disguised as, in the appearance of
\item \pali{pa\d puggalo} (\pali{pati + puggala}) = (n. m.) a person equal to another, a compeer, a match
\item \pali{pa\d tigga\d nh\=ati} (\pali{pati + gaha}) = (v.) to accept, to receive (also pa\d tiga\d nh\=ati)\footnote{In R\=upa, it is \pali{patigga\d nh\=ati}. This form is found only in that book.}
\item \pali{pa\d tivijjhati} (\pali{pati + vidha}) = (v.) to pierce through, to penetrate, to comprehend, to master
\item \pali{pa\d tipajjati} (\pali{pati + pada}) = (v.) to enter upon, to go along, to follow out (a way or path)
\item \pali{paccakkho} (\pali{pati + akkha}) = (adj.) before the eye, perceptible to the senses, evident
\item \pali{pa\d tisota\d m} (\pali{pati + sota}) = (adv.) against the stream
\item \pali{pa\d tisandhi} (\pali{pati + sandhi}) = (n. f.) reunion, reincarnation, conception
\end{compactitem}

\section*{\fbox{\pali{Par\=a}}}\label{upasagga:paraa}
\begin{compactitem}
\item \pali{par\=amasati}* (\pali{par\=a + \=a + masa}) = (v.) to touch, to hold on to, to deal with
\item \pali{par\=abhavo} (\pali{par\=a + bhava}) = (n. m.) defeat, destruction, ruin, disgrace
\item \pali{par\=ajeti} (\pali{par\=a + ji}) = (v.) to defeat, to conquer
\item \pali{par\=ajito} (\pali{par\=a + ji + ta}) = (adj.) defeated, having suffered a loss
\item \pali{par\=ayana\d m} (\pali{par\=a + aya}) = (n. nt.) the final end, support, rest (also \pali{par\=aya\d na\d m})
\item \pali{parakkamati} (\pali{par\=a + kamu}) = (v.) to exert, to show courage
\item \pali{par\=amasana\d m} (\pali{par\=a + \=a + masa}) = (n. nt.) touching, handling, contagion
\end{compactitem}

\section*{\fbox{\pali{Pari}}}\label{upasagga:pari}
\begin{compactitem}
\item \pali{parij\=an\=ati}* (\pali{pari + \~n\=a}) = (v.) to know accurately, to comprehend, to recognize
\item \pali{pariv\=areti} (\pali{pari + vara}) = (v.) to cover, to encompass, to surround (p.p.\ \pali{parivuta})
\item \pali{pariharati} (\pali{pari + hara}) = (v.) to keep up, to protect, to carry about, to avoid
\item \pali{parissajati} (\pali{pari + saja}) = (v.) to embrace, to enfold
\item \pali{paricarati} (\pali{pari + cara}) = (v.) to go about, to look after, to worship
\item \pali{parivisati} (\pali{pari + visa}) = (v.) to serve with food, to wait upon when food is taken
\item \pali{paribhavati} (\pali{pari + bh\=u}) = (v.) to treat with contempt, to despise, to abuse
\item \pali{paribh\=asati} (\pali{pari + bhasa}) = (v.) to abuse, to scold, to defame
\end{compactitem}

\section*{\fbox{\pali{Vi}}}\label{upasagga:vi}
\begin{compactitem}
\item \pali{viharati}* (\pali{vi + hara}) = (v.) to stay, to abide, to dwell
\item \pali{vij\=an\=ati} (\pali{vi + \~n\=a}) = (v.) to have discriminative knowledge, to recognize, to perceive, to understand, to know
\item \pali{vipassati} (\pali{vi + disa}) = (v.) to see clearly, to have intuition, to obtain spiritual insight
\item \pali{vimuccati} (\pali{vi + muca}) = (v.) to be release, to be free, to be emancipated
\item \pali{visissati} (\pali{vi + sisa}) = (v.) to differ, to be distinguished (often found as p.p.\ \pali{visi\d t\d tho})
\item \pali{vimati} (\pali{vi + mati}) = (n. f.) doubt, perplexity
\item \pali{vicinteti} (\pali{vi + cinta}) = (v.) to think, to consider
\item \pali{vicin\=ati} (\pali{vi + ci}) = (v.) to investigate, to examine, to discriminate
\item \pali{vivadati} (\pali{vi + vada}) = (v.) to dispute, to quarrel
\item \pali{vigacchati} (\pali{vi + gamu}) = (v.) to depart, to disappear, to decrease
\item \pali{vimalo} (\pali{vi + mala}) = (adj.) without stains, spotless, clean
\item \pali{viyogo} (\pali{vi + yoga}) = (n. m.) separation
\item \pali{vir\=upo} (\pali{vi + r\=upa}) = (adj.) deformed, ugly
\item \pali{vippa\d tis\=aro} (\pali{vi + pati + sara}) = (n. m.) bad conscience, remorse, regret 
\end{compactitem}

\section*{\fbox{\pali{Sa\d m}}}\label{upasagga:sadm}
\begin{compactitem}
\item \pali{sandh\=aya}* (\pali{sa\d m + daha + tv\=a}) = (ind.) with reference to, concerning
\item \pali{sandhi} (\pali{sa\d m + dh\=a}) = (n. f.) union, junction, connection
\item \pali{sam\=adhi} (\pali{sa\d m + \=a + dh\=a}) = (n. m.) meditation, concentration
\item \pali{sampayojeti} (\pali{sa\d m + pa + yuja}) = (v.) to associate with, to quarrel (often found in p.p.\ \pali{sampayutta})
\item \pali{sa\d mkilissati} (\pali{sa\d m + kilisa}) = (v.) to become soiled, to become impure
\item \pali{samullapati} (\pali{sa\d m + u + lapa}) = (v.) to talk, to converse
\item \pali{sa\.ngacchati} (\pali{sa\d m + gamu}) = (v.) to come together, to meet with
\item \pali{sa\.nkhipati} (\pali{sa\d m + khipa}) = (v.) to collect, to withdraw, to concentrate, to abridge, to shorten
\item \pali{sa\.nga\d nh\=ati} (\pali{sa\d m + gaha}) = (v.) to comprise, to collect, to compile, to sympathize with
\item \pali{sa\.nkirati} (\pali{sa\d m + kira}) = (v.) to mix together (often found in p.p.\ \pali{sa\d mki\~n\~na})
\item \pali{sam\=aso} (\pali{sa\d m + asu}) = (n. m.) compound, combination, an abridgement
\item \pali{sambhogo} (\pali{sa\d m + bhoga}) = (n. m.) eating or living together with
\item \pali{s\=arajjati} (\pali{sa\d m + ranja}) = (v.) to be pleased with, to be attached to
\item \pali{sa\d mvasati} (\pali{sa\d m + vasa}) = (v.) to associate, to live together, to cohabit
\item \pali{sa\d mv\=aso} (\pali{sa\d m + vasa}) = (n. m.) co-residence, intimacy, sexual intercourse
\item \pali{sambhavo} (\pali{sa\d m + bhava}) = (n. m.) origin, birth, production, semen virile
\item \pali{sammukho} (\pali{sa\d m + mukha}) = (adj.) face to face with, in presence
\item \pali{sa\d mvarati} (\pali{sa\d m + vara}) = (v.) to restrain, to hold (p.p.\ \pali{sa\d mvuta})
\item \pali{sandh\=avati} (\pali{sa\d m + dh\=avu}) = (v.) to run through, to transmigrate
\item \pali{sampajjati} (\pali{sa\d m + pada}) = (v.) to succeed, to prosper, to happen, to become (p.p.\ sampa\~n\~na)
\item \pali{sandahati} (\pali{sa\d m + daha}) = (v.) to put together, to connect, to fit, to arrange (often found in abs.\ \pali{sandh\=aya})
\end{compactitem}
As explained in PTSD, \pali{sa\d m} can be shortened to \pali{sa} in compounds meaning like `with \ldots,' for example \pali{sadevaka} (with gods), \pali{sadhammika} (having common faith). However, in the traditional account, this \pali{sa} is a contracted form of \pali{saha}. So, the outcome is of \pali{Sahapubbapadapahubb\=ihi} compound (see page \pageref{par:sahapubba}).

\section*{\fbox{\pali{Su}}}\label{upasagga:su}
\begin{compactitem}
\item \pali{sukha\d m}* (\pali{su + kha}) = (n. nt.) happiness, comfort
\item \pali{sugandha} (\pali{su + gandha}) = (adj.) fragrant
\item \pali{sugato} (\pali{su + gamu + ta}) = (adj.) well gone\footnote{\pali{su\d t\d thu gato sugato, samm\=a gatotipi sugato} (R\=upa after 281)}, faring well, happy (m.\ the Buddha)
\item \pali{subhikkho} (\pali{su + bhikkha}) = (adj.) having plenty of food
\item \pali{sukaro} (\pali{su + kara}) = (adj.) easy to do 
\item \pali{sumano} (\pali{su + mana}) = (adj.) glad
\end{compactitem}

\section*{Some common verbs with prefixes}

To see a clearer picture how these \pali{upasagga}s work in action, I list some common verbs and their compositions in the tables below. Some terms have been introduced already above, but some are newly added.

\bigskip
\begin{longtable}[c]{@{}%
	>{\itshape\raggedright\arraybackslash}p{0.24\linewidth}%
	>{\itshape\centering\arraybackslash}p{0.19\linewidth}%
	>{\raggedright\arraybackslash}p{0.46\linewidth}@{}}
\toprule
\bfseries\upshape Verb & \bfseries\upshape Prefix & \bfseries\upshape Meaning \\ \midrule
\endfirsthead
\toprule
\bfseries\upshape Verb & \bfseries\upshape Prefix & \bfseries\upshape Meaning \\ \midrule
\endhead
\bottomrule
\ltblcontinuedbreak{3}
\endfoot
\bottomrule
\endlastfoot
%
\bfseries bhavati\footnote{The list of \pali{bhavati} mainly comes from Sadd-Pad Ch.\,1. Most forms can be used \pali{bhoti} instead.} & & to be, to exist \\
ubbhavati & u & to be born, to arise (= \pali{uppajjati}) \\
samubbhavati & sa\d m + u & to happen, to become (= \pali{sampajjati}) \\
pabhavati & pa & to originate (= \pali{sambhavati}) \\
par\=abhavati & par\=a & to decline \\
sambhavati & sa\d m & to arise, to be produced \\
vibhavati & vi & to cease to exist (= \pali{ucchijjati, vinassati, vipajjati}) \\
p\=atubhavati & p\=atu\footnote{This is an indeclinable meaning `in front, visible, manifest.'} & to appear, to become manifest (= \pali{pak\=asati, dissati}) \\
paribhavati & pari & to treat with contempt, to despise, to abuse \\
abhibhavati & abhi & to overcome, to conquer \\
adhibhavati & adhi & to overpower \\
atibhavati & ati & to excel, to overcome \\
anubhavati & anu & to undergo, to eat \\
\mbox{samanubhavati} & sa\d m + anu & to undergo well \\
\mbox{abhisambhavati} & abhi + sa\d m & to be able to, to attain\footnote{In Sadd-Pad Ch.\,1, this means `to overpower/crush others' (\pali{para\d m ajjhottharati maddati}).} \\
\midrule
\bfseries kamati & & to go, to enter into \\
abhikkamati & abhi & to proceed \\
akkamati & a & to step upon, to subjugate \\
anuca\.nkamati & anu & to follow one who is walking back and forth \\
anukkamati & anu & to follow \\
apakkamati & apa & to depart, to go away \\
atikkamati & ati & to go beyond, to overcome, to surpass \\
avakkamati & ava & to enter, to overwhelm \\
okkamati & o & to enter, to fall into \\
pakkamati & pa & to step forward, to go away \\
parakkamati & par\=a & to exert, to show courage \\
pa\d tikkamati & pati & to step backward, to go back \\
samatikkamati & sa\d m + ati & to pass over, to transcend, to remove \\
sa\.nkamati & sa\d m & to pass over to, to shift, to transmigrate \\
upakkamati & upa & to strive, to undertake, to begin, to attack \\
upasa\.nkamati & upa + sa\d m & to appraoch \\
vikkamati & vi & to exert oneself, to step forward \\
vokkamati & vi + u & to turn aside, to deviate from \\
v\=itikkamati & vi + ati & to transgress, to go beyond \\
\midrule
\bfseries gacchati & & to go, to move, to walk \\
\mbox{ajjhupagacchati} & adhi + upa & to arrive, to reach, to consent \\
atigacchati & ati & to overcome, to surpass, to surmount \\
adhigacchati & adhi & to attain, to obtain, to understand \\
anugacchati & anu & to follow, to go after \\
\mbox{anuparigacchati} & anu + pari & to go round about \\
apagacchati & apa & to go away, to turn aside \\
\mbox{abbhuggacchati} & abhi + u & to rise up, to be diffused \\
avagacchati & ava & to attain, to obtain, to understand \\
\=agacchati & \=a & to come, to approach \\
uggacchati & u & to rise, to go up \\
upagacchati & upa & to approach, to undergo, to undertake \\
ogacchati & o & to go down, to sink down \\
nigacchati & ni & to undergo, to come to \\
niggacchati & ni & to go out, to proceed from \\
pacc\=agacchati & pati + \=a & to return, to come back, to withdraw \\
\mbox{paccuggacchati} & pati + u & to go out to meet \\
vigacchati & vi & to depart, to disappear, to go away \\
vyapagacchati & vi + apa & to depart \\
sa\.ngacchati & sa\d m + & to meet with, to come together \\
samadhigacchati & sa\d m + ahi & to attain, to understand clearly \\
sam\=agacchati & sa\d m + \=a & to meet together, to assemble \\
samuggacchati & sa\d m + u & to arise, to come to existence \\
\mbox{samupagacchati} & sa\d m + upa & to approach \\
\midrule
\bfseries j\=an\=ati & & to know, to find out \\
anuj\=an\=ati & anu & to allow, to give permission \\
abhij\=an\=ati & abhi & to know fully, to know by experience \\
avaj\=an\=ati & ava & to despise \\
\=aj\=an\=ati & \=a & to know, to understand \\
upaj\=an\=ati & upa & to learn, to know \\
pa\d tij\=an\=ati & pati & to acknowledge, to promise, to consent \\
pa\d tivij\=an\=ati & pati + vi & to recognize, to know \\
paj\=an\=ati & pa & to know clearly \\
parij\=an\=ati & pari & to know accurately, to comprehend \\
vij\=an\=ati & vi & to know, to understand, to perceive, to have discriminative knowledge \\
sa\~nj\=an\=ati & sa\d m & to recognize, to be aware of \\
samanuj\=an\=ati & sa\d m + anu & to approve \\
samabhij\=an\=ati & sa\d m + abhi & to recollect, to know \\
sampaj\=an\=ati & sa\d m + pa & to know \\
\midrule
\bfseries karoti & & to do, to act, to make \\
anukaroti & anu & to imitate \\
apakaroti & apa & to throw away, to hurt, to offend \\
avakaroti & ava & to put down, to despise, to throw away \\
av\=akaroti & ava + \=a & to revoke, to undo, to give back, to restore \\
nikaroti & ni & to bring down, to humiliate, to deceive \\
nira\.nkaroti & ni + \=a & to repudiate, to disregard \\
nir\=akaroti & ni + \=a & to repudiate, to disregard \\
pakaroti & pa & to effect, to perform, to prepare \\
pa\d tikaroti & pati & to redress, to expiate, to act against \\
parikaroti & pari & to surround, to serve, to wait upon \\
sakkaroti & sa\d m & to honor, to treat with respect \\
upakaroti & upa & to help, to support, to serve \\
vikaroti & vi & to undo, to alter \\
vippakaroti & vi + pa & to treat, to abuse \\
vy\=akaroti & vi + \=a & to explain, to declare, to answer \\
\midrule
\bfseries pajjati\footnote{This term is only found with a certain prefix, not an independent verb.} & & to go \\
ajjh\=apajjati & adhi + \=a & to commit an offend, to incur \\
anupajjati & anu & to follow, to accompany \\
abhinipajjati & abhi + ni & to lie down on \\
abhinippajjati & abhi + ni & to be produced, to accrue \\
\=apajjati & \=a & to get into, to undergo, to meet with \\
upanipajjati & upa + ni & to lie down close to \\
upapajjati & upa & to be reborn in, to rise \\
upasampajjati & upa + sa\d m & to attain, to enter on, to be fully ordained \\
uppajjati & u & to be born, to arise \\
nipajjati & ni & to lie down, to sleep \\
nippajjati & ni & to be produced, to be accomplished \\
pa\d tipajjati & pati & to enter upon a path, to go along, to follow a method \\
pariy\=apajjati & pari + \=a & to be finished \\
vipajjati & vi & to fail, to go wrong, to perish \\
vippa\d tipajjati & vi + pati & to err, to fail, to commit sins \\
vy\=apajjati & vi + \=a & to fail, to be troubled, to be vexed \\
sampajjati & sa\d m & to succeed, to prosper, to happen, to become \\
samuppajjati & sa\d m + u & to arise, to be produced \\
sam\=apajati & sa\d m + \=a & to enter upon, to engage in \\
\midrule
\bfseries ga\d nh\=ati & & to take, to hold of, to seize \\
adhiga\d nh\=ati & adhi & to supass, to excel \\
atinigga\d nh\=ati & ati + ni & to rebuke too much \\
anugga\d nh\=ati & anu & to help, to have pity on \\
abhiga\d nh\=ati & abhi & to surpass, to possess, to overpower \\
abhinigga\d nh\=ati & abhi + ni & to hold back, to restain, to prevent \\
ugga\d nh\=ati & u & to learn, to acquire, to take up \\
nigga\d nh\=ati & ni & to rebuke, to censure, to restrain \\
pagga\d nh\=ati & pa & to hold up, to take up, to support \\
pa\d tigga\d nh\=ati & pati & to take, to receive, to accept \\
pariga\d nh\=ati & pari & to explore, to examine, to search \\
sa\.nga\d nh\=ati & sa\d m & to treat kindly, to compile, to collect \\
sanniga\d nh\=ati & sa\d m + ni & to restrain \\
\mbox{samadhiga\d nh\=ati} & sa\d m + adhi & to reach, to get, to exceed, to surpass \\
\mbox{samatigga\d nh\=ati} & sa\d m + ati & to strecth over, to rise above, to reach beyond \\
samugga\d nh\=ati & sa\d m + u & to learn well, to seize, to embrace \\
\mbox{sampagga\d nh\=ati} & sa\d m + pa & to exert, to strain, to favor, to befriend \\
\end{longtable}
