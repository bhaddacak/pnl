\chapter{(There is) a \headhl{big} book}\label{chap:adj}

\phantomsection
\addcontentsline{toc}{section}{Introduction to Adjective}
\section*{Introduction to Adjective}

In this chapter we will add a modifier, an adjective, to nouns. Adjective, called \pali{gu\d nan\=ama} by its word group and \pali{visesana} by its function, modifies a noun to make it more specific or to express its quality. In P\=ali adjectives have no gender.\footnote{It can be seen as having all three genders, if you will.} They take gender and number from the noun they modify. In traditional textbooks, adjectives are not a big deal. I mean I cannot find a dedicated section for the topic from such textbooks. I think traditional grammarians see adjectives in a different way, unlike modern grammarians who classify adjectives as one separate category. To the tradition, adjectives are more or less nouns with three gender forms.\footnote{A key difference between a noun and an adjective is that when the meaning allows an adjective can become an adverb by assuming accusative case (see Chapter \ref{chap:adv}), whereas a noun cannot. Another difference is an adjective can have comparative and superlative forms (see Chapter \ref{chap:adjcomp}).} As a result, adjectives can be translated as a thing that has certain quality, for instance, `big' can mean ``a thing that has a quality of `bigness'.'' So, ``a big book'' can mean ``a book (is) a thing that has a quality of `bigness'.'' And the way to associate `big' with `book' is to make them the same case. By this reason sometimes, if not often, we see an adjective in P\=ali stands alone without a noun if the modified noun is understood.

Here is a general guideline when we use an adjective. Check a dictionary to find out the word's ending. If it has \pali{a} ending, take it as m.\ and nt., and change the ending to \pali{\=a} to use it as f. If it has \pali{\=a} ending, take it as f., and change the ending to \pali{a} to use it as m.\ and nt. If it has \pali{\=i} or \pali{\=u} ending, shorten the ending to \pali{i} or \pali{u} for taking it as nt. Other endings not mentioned above already have their corresponding genders. The summary of the guideline is shown in Table \ref{tab:adjguide}.

\begin{table}[!hbt]
\centering
\caption{Adjective selection guide}
\label{tab:adjguide}
\bigskip
\begin{tabular}{c*{6}{>{\itshape}l}} \toprule
\multirow{2}{*}{\bfseries Gender} & \multicolumn{6}{c}{\bfseries Endings} \\ \cmidrule(l){2-7}
& a & \=a & i & \=i & u & \=u \\ \midrule
m. & a & \replacewith{\=a}{a} & i & \=i & u & \=u \\
f. & \replacewith{a}{\=a} & \=a & i & \=i & u & \=u \\
nt. & a & \replacewith{\=a}{a} & i & \replacewith{\=i}{i} & u & \replacewith{\=u}{u} \\
\bottomrule
\end{tabular}
\end{table}

Let us do our task, to say ``(There is) a big book.'' First we have to find an adjective that means `big.' The most common word of this is \pali{mahanta}.\footnote{This term has its f.\ form as \pali{mahat\=i} or \pali{mahant\=a}. It is more often to be found in compounds as \pali{mah\=a-}.} But this is not the right word for this context, because \pali{mahanta} has a connotation of `great', `fabulous' and `wealthy.' The most appropriate word for our purpose is \pali{th\=ula} which means `thick', `fat' or `massive.' We have to use this word as m.\ or nt.\ corresponding to \pali{potthaka}. After consulting the guideline above (nothing to do in this case), then changing it to nominative case, we get this P\=ali sentence:

\palisample{th\=ulo potthako.\sampleor th\=ula\d m potthaka\d m.}

The order of words can be reversed. So, ``\pali{potthako th\=ulo}'' is also valid. And here is for ``(There are) big books'':

\palisample{th\=ul\=a potthak\=a.\sampleor th\=ul\=ani potthak\=ani.}

Now you can say ``(There is) a fat girl.''

\palisample{th\=ul\=a ka\~n\~n\=a.}

And here is for ``(There are) fat girls.''

\palisample{th\=ul\=a ka\~n\~n\=a.\sampleor th\=ul\=ayo ka\~n\~n\=ayo.}

As we have seen, an important rule about adjective we have to remember is \emph{adjectives must agree with the noun they modify in case, gender, and number}.

There are a number of adjectives, mostly ended with \pali{-antu}, that have irregular forms of declension. These words can also be used as nouns with three genders. For example, \pali{dhanavantu}, meaning `rich' or `rich person,' can decline in three ways following the paradigm of \pali{gu\d navantu}: m.\ \pali{dhanavantu} (see page \pageref{decl:gunavm}), nt.\ \pali{dhanavantu} (see page \pageref{decl:gunavnt}), and f.\ \pali{dhanavat\=i} or \pali{dhanavant\=i} (see page \pageref{decl:gunavf}).

Hence, to say ``(There is) a rich man'' you can put it as:

\palisample{dhanav\=a puriso.\sampleor[or just]dhanav\=a.}

``(There are) rich men.''

\palisample{dhanavanto puris\=a.\sampleor dhanavant\=a puris\=a.\sampleor[or just]dhanavanto. \textup{\normalsize or} dhanavant\=a.}

``(There is) a rich girl.''

\palisample{dhanavat\=i ka\~n\~n\=a.\sampleor dhanavant\=i ka\~n\~n\=a.\sampleor[or just]dhanavat\=i. \textup{\normalsize or} dhanavant\=i.}

In fact if you can remember the regular declension of f.\ \pali{\=i} ending, you do not need to remember this f.\ rule. It goes the normal way. Here is for ``(There are) rich girls.''

\palisample{dhanava(n)t\=i ka\~n\~n\=a.\sampleor dhanava(n)tiyo ka\~n\~n\=ayo.\sampleor[or just]dhanava(n)t\=i. \textup{\normalsize or} dhanava(n)tiyo.}

``(There is) a rich family.''

\palisample{dhanava\d m kula\d m.}

``(There are) rich families.''

\palisample{dhanavanti kul\=ani.\sampleor dhanavant\=ani kul\=ani.}

In certain situations, there can be a gender conflict when an adjective is used to modify different nouns with various genders, for example, ``(There are) a good-looking boy, a good-looking girl, and a good-looking book.'' If you want to use only one \pali{sundara} as `good-looking,' you can put it in this way:

\palisample{d\=arako, d\=arik\=a, potthaka\d m (v\=a) sundara\d m.}

Particle \pali{v\=a} here means `or/and,' but ignore this for now because its own lesson is in Chapter \ref{chap:ind-intro}. A.\,K.\,Warder says that ``Where the genders conflict, the masculine takes precedence over the feminine, the neuter over both.''\footnote{\citealp[p.~61]{warder:intro}} That is why we use the adjective in nt.\ form. So, if we say ``There are a good-looking boy, a good-looking girl,'' it should be as follows:

\palisample{d\=arako, d\=arik\=a (v\=a) sundaro.}

Do not forget to do our exercise below.

\section*{Exercise \ref{chap:adj}}
Say these in P\=ali using adjectives in Appendix \ref{vocab:adj} and nouns in Appendix \ref{vocab:noun}.
\begin{compactenum}
\item (There is) a difficult language.
\item (There is) a young elephant.
\item (There are) many geckoes.
\item (There are) beautiful women.
\item (There are) shining eyes.
\item (There is) a thin, fearful dog.
\item (There are) big, heavy stones.
\item (There is) a wise, kind teacher.
\item (There are) beautiful red flowers.
\item (There is) a fast long train.
\end{compactenum}
