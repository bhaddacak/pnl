\markboth{}{Abbreviations}
\clearpage
\phantomsection
\addcontentsline{toc}{chapter}{Abbreviations}
\setcounter{footnote}{0}
\chapter*{Abbreviations}

\section*{References to Literary Works}

The main part of the P\a{=}ali literature cited here is collected in the corpus of \emph{Cha\a{d}t\a{d}tha Sa\.ng\a{=}ayana Tipi\a{d}taka Restructured} prepared by the author.\footnote{See its website at \url{bhaddacak.github.io/cst-kit}. This collection is in turn based on the old CSCD published by Vipassana Research Institute (\url{tipitaka.org}), but it has gone through a lot of improvements.} Referencing method to this collection is simple, thanks to the clear structure of the collection. We use the abbreviation of text's names followed by paragraph numbers. For example, \texttt{M2\,345} means the paragraph numbered 345 of Majjhimanik\a{=}aya's Majjhimapa\d n\d n\a{=}asa (the second fifty).\footnote{For D\a{=}ighanik\a{=}aya and Majjhimanik\a{=}aya, the sutta numbers may also be given, marked by \texttt{DN} and \texttt{MN}, for example, \texttt{D2\,375 (DN\,22)}. Likewise for Sa\d myuttanik\a{=}aya, the sa\d myutta numbers may also be given in the form of \texttt{SN}, e.g.\ \texttt{S1\,224 (SN\,9)}.}

In some cases, chapter numbers are also given in decimal form. For example, \texttt{Vism\,14.428} means the paragraph numbered 428 of Visuddhimagga's chapter 14.\footnote{The chapter numbers are redundant because now all fragments are combined into books. In \textsc{P\a{=}ali\,Platform} 3, the user can go to the paragraph directly. This companion program is now indispensable to our study. For some more information on text referencing in this corpus, see \url{bhaddacak.github.io/cstpage}.} In the same way, khandaka numbers in Mah\a{=}avagga and C\a{=}u\d lavagga, despite redundant, are also given, for example, \texttt{Mv\,2.142}, \texttt{Cv\,5.285}.\footnote{We normally do not use khandaka numbers when we mention the text by \texttt{Mv} or \texttt{Cv}. However, the text collection of \emph{SuttaCentral} (\url{suttacentral.net}) combines both into Khandakas, 22 totally. This can ease the lookup across corpora. Remember that khandaka 1 in \texttt{Cv} is equivalent to khandaka 11 in the SuttaCentral collection.} Also, the texts that have vagga numbers use the same form, for example, \texttt{Dhp\,8.106}. In some texts, such as J\a{=}ataka, Apad\a{=}ana, Buddhava\d msa and several more, we can also see section numbers with a colon, for example, \texttt{Ja\,22:1880}, \texttt{Ap1\,1:511}, and \texttt{Bv\,27:22}.

For grammatical textbooks, we have all essential books in the collection \emph{``Traditional P\a{=}ali Grammar Books''}. Most of the books here have sutta or stanza numbers to refer to. For Payogasiddhi, the new running numbers are given to make it less confusing.

Aggava\d msa's Saddan\a{=}iti is a little complicated to deal with. There are all 28 chapters in the treatise. The first part, Padam\a{=}al\a{=}a has 14 chapters, the second part, Dh\a{=}atum\a{=}al\a{=}a 5, and the third, Suttam\a{=}al\a{=}a 9. Padam\a{=}al\a{=}a has no number to refer to, even heading numbers. In this case, only chapter numbers are given and P\a{=}ali passages will be fully quoted, long enough to be a distinct searching query. In some cases, the publication's page numbers (\citealp{smith:sadd1}) are also given.

For Dh\a{=}atum\a{=}al\a{=}a, fortunately we have definition numbers now (only this collection does the text have these numbers). That can make root referencing in this text much easier. In Suttam\a{=}al\a{=}a the first 7 chapters have sutta numbers to refer to. The last two have none. To cite suttas in Suttam\a{=}al\a{=}a, only \texttt{Sadd} followed by sutta numbers will be used. Otherwise, \texttt{Sadd-Sut} followed by a chapter number, and sometimes with the publication's page numbers (\citealp{smith:sadd3}), will be used.\footnote{Three parts of Saddan\a{=}iti P\a{=}ali are available online, see their entry in the bibliography. Discrepancies can be detected though, but differences are not significant.} For more detail on grammatical textbooks, see Appendix \ref{chap:textbook}.

Here are abbreviations of P\a{=}ali works used in this book.

\bigskip
\begin{longtable}[c]{@{}>{\raggedright\arraybackslash}p{0.17\linewidth}>{\raggedright\arraybackslash}p{0.78\linewidth}@{}}
\toprule
\bfseries\upshape \mbox{Abbrev.} & \bfseries\upshape Description \\ \midrule
\endfirsthead
\toprule
\bfseries\upshape \mbox{Abbrev.} & \bfseries\upshape Description \\ \midrule
\endhead
\bottomrule
\ltblcontinuedbreak{2}
\endfoot
\bottomrule
\endlastfoot
%%
A2 & Dukanip\a{=}ata, A\a{.}nguttaranik\a{=}aya, Suttapi\a{d}taka \\
A3 & Tikanip\a{=}ata, A\a{.}nguttaranik\a{=}aya, Suttapi\a{d}taka \\
A4 & Catukkanip\a{=}ata, A\a{.}nguttaranik\a{=}aya, Suttapi\a{d}taka \\
A5 & Pa\a~ncakanip\a{=}ata, A\a{.}nguttaranik\a{=}aya, Suttapi\a{d}taka \\
A6 & Chakkanip\a{=}ata, A\a{.}nguttaranik\a{=}aya, Suttapi\a{d}taka \\
A7 & Sattakanip\a{=}ata, A\a{.}nguttaranik\a{=}aya, Suttapi\a{d}taka \\
A8 & A\a{d}t\a{d}thakanip\a{=}ata, A\a{.}nguttaranik\a{=}aya, Suttapi\a{d}taka \\
A10 & Dasakanip\a{=}ata, A\a{.}nguttaranik\a{=}aya, Suttapi\a{d}taka \\
Abhidh\a{=}a & Abhidh\a{=}anappad\a{=}ipik\a{=}ap\a{=}a\a{d}tha \\
Ap1 & Ther\a{=}apad\a{=}ana, Khuddakanik\a{=}aya, Suttapi\a{d}taka \\
Ap2 & Ther\a{=}iapad\a{=}ana, Khuddakanik\a{=}aya, Suttapi\a{d}taka \\
Buv & Bhikkhuvibha\a{.}nga (Mah\a{=}avibha\a{.}nga), Vinayapi\a{d}taka \\
Bv & Buddhava\a{d}msa, Khuddakanik\a{=}aya, Suttapi\a{d}taka \\
Cp & Cariy\a{=}api\a{d}taka, Khuddakanik\a{=}aya, Suttapi\a{d}taka \\
Cv & C\a{=}u\a{d}lavagga, Vinayapi\a{d}taka \\
D1 & S\a{=}ilakkhandhavagga, D\a{=}ighanik\a{=}aya, Suttapi\a{d}taka \\
D2 & Mah\a{=}avagga, D\a{=}ighanik\a{=}aya, Suttapi\a{d}taka \\
D3 & P\a{=}athikavagga, D\a{=}ighanik\a{=}aya, Suttapi\a{d}taka \\
Dhp & Dhammapada, Khuddakanik\a{=}aya, Suttapi\a{d}taka \\
Dhp-a & Dhammapada-a\a{d}t\a{d}thakath\a{=}a \\
Dhs & Dhammasa\a{.}nga\a{d}n\a{=}i, Abhidhammapi\a{d}taka \\
DN & D\a{=}ighanik\a{=}aya (with sutta no.) \\
It & Itivuttaka, Khuddakanik\a{=}aya, Suttapi\a{d}taka \\
Ja & J\a{=}ataka, Khuddakanik\a{=}aya, Suttapi\a{d}taka \\
Kacc & Kacc\a{=}ayanaby\a{=}akara\a{d}na\a{d}m \\
Khp & \mbox{Khuddakap\a{=}a\a{d}tha, Khuddakanik\a{=}aya, Suttapi\a{d}taka} \\
Kv & Kath\a{=}avatthu, Abhidhammapi\a{d}taka \\
M1 & M\a{=}ulapa\a{d}n\a{d}n\a{=}asa, Majjhimanik\a{=}aya, Suttapi\a{d}taka \\
M2 & Majjhimapa\a{d}n\a{d}n\a{=}asa, Majjhimanik\a{=}aya, Suttapi\a{d}taka \\
M3 & Uparipa\a{d}n\a{d}n\a{=}asa, Majjhimanik\a{=}aya, Suttapi\a{d}taka \\
Mil & Milindapa\a~nh\a{=}a, Khuddakanik\a{=}aya, Suttapi\a{d}taka \\
MN & Majjhimanik\a{=}aya (with sutta no.) \\
MN-a & Majjhimanik\a{=}aya-a\a{d}t\a{d}thakath\a{=}a (with sutta no.) \\
Mnp & Manorathapur\a{=}a\a{d}n\a{=}i, A\a{.}nguttaranik\a{=}aya-a\a{d}t\a{d}thakath\a{=}a \\
Mogg & Moggall\a{=}anaby\a{=}akara\a{d}na\a{d}m \\
Mv & Mah\a{=}avagga, Vinayapi\a{d}taka \\
Mv-a & Mah\a{=}avagga-a\a{d}t\a{d}thakath\a{=}a (= Sp4) \\
Nidd1 & Mah\a{=}aniddesa, Khuddakanik\a{=}aya, Suttapi\a{d}taka \\
Niru & Le\a{d}d\a{=}i Say\a{=}a\a{d}do's Niruttid\a{=}ipan\a{=}i \\
Payo & Payogasiddhip\a{=}a\a{d}tha \\
Pps & Papa\a~ncas\a{=}udan\a{=}i, Majjhimanik\a{=}aya-a\a{d}t\a{d}thakath\a{=}a \\
PTR & P\a{=}ali Text Reading: A Handbook\footnote{\url{bhaddacak.github.io/ptr}} \\
PTSD & \mbox{The Pali Text Society's Pali-English Dictionary}\footnote{\citealp{rhys:ptsd}} \\
Pv & Petavatthu, Khuddakanik\a{=}aya, Suttapi\a{d}taka \\
Pvr & Pariv\a{=}ara, Vinayapi\a{d}taka \\
R\a{=}upa & Padar\a{=}upasiddhi \\
S1 & Sag\a{=}ath\a{=}avagga (SN\,1--11), Sa\a{d}myuttanik\a{=}aya, Sut. \\
S2 & Nid\a{=}anavagga (SN\,12--21), Sa\a{d}myuttanik\a{=}aya, Sut. \\
S3 & Khandhavagga (SN\,22--34), Sa\a{d}myuttanik\a{=}aya, Sut. \\
S4 & \mbox{Sa\a{d}l\a{=}ayatanavagga (SN\,35--44), Sa\a{d}myuttanik\a{=}aya, Sut.} \\
S5 & Mah\a{=}avagga (SN\,45--56), Sa\a{d}myuttanik\a{=}aya, Sut. \\
Sadd & \mbox{Saddan\a{=}itipakara\a{d}na\a{d}m, Suttam\a{=}al\a{=}a (with sutta no.)} \\
Sadd-Dh\a{=}a & Saddan\a{=}itipakara\a{d}na\a{d}m, Dh\a{=}atum\a{=}al\a{=}a \\
Sadd-Pad & Saddan\a{=}itipakara\a{d}na\a{d}m, Padam\a{=}al\a{=}a \\
Sadd-Sut & Saddan\a{=}itipakara\a{d}na\a{d}m, Suttam\a{=}al\a{=}a \\
SN & Sa\a{d}myuttanik\a{=}aya (with sa\a{d}myutta no.) \\
SN-a & Sa\a{d}myuttanik\a{=}aya-a\a{d}t\a{d}thakath\a{=}a (with sa\a{d}myutta no.) \\
Snp & Suttanip\a{=}ata, Khuddakanik\a{=}aya, Suttapi\a{d}taka \\
Sp & Samantap\a{=}as\a{=}adik\a{=}a, Vinaya-a\a{d}t\a{d}thakath\a{=}a \\
Thag & Therag\a{=}ath\a{=}a, Khuddakanik\a{=}aya, Suttapi\a{d}taka \\
Thig & Ther\a{=}ig\a{=}ath\a{=}a, Khuddakanik\a{=}aya, Suttapi\a{d}taka \\
Ud & Ud\a{=}ana, Khuddakanik\a{=}aya, Suttapi\a{d}taka \\
Vism & Visuddhimagga \\
Vv & Vim\a{=}anavatthu, Khuddakanik\a{=}aya, Suttapi\a{d}taka \\
Ym & Yamaka, Abhidhammapi\a{d}taka \\
\end{longtable}

\newpage
\section*{Grammatical Terms}
Here are grammatical terms abbreviated and used in this book.

\bigskip
\begin{longtable}[c]{@{}>{\raggedright\arraybackslash}p{0.17\linewidth}>{\raggedright\arraybackslash}p{0.78\linewidth}@{}}
\toprule
\bfseries\upshape \mbox{Abbrev.} & \bfseries\upshape Description \\ \midrule
\endfirsthead
\toprule
\bfseries\upshape \mbox{Abbrev.} & \bfseries\upshape Description \\ \midrule
\endhead
\bottomrule
\ltblcontinuedbreak{2}
\endfoot
\bottomrule
\endlastfoot
%%
abl. & Ablative case (Pa\a~ncam\a{=}i) \\
abs. & Absolutive \\
acc. & Accusative case (Dutiy\a{=}a) \\
adj. & Adjective (Gu\a{d}nan\a{=}ama) \\
adv. & Adverb \\
aor. & Aorist tense (Ajjatan\a{=}i) \\
cond. & Conditional mood (K\a{=}al\a{=}atipatti) \\
dat. & Dative case (Catu\a{d}t\a{d}th\a{=}i) \\
dict. & Dictionary form \\
f. & Feminine gender (Itth\a{=}ili\a{.}nga) \\
fut. & Future tense (Bhavissanti) \\
g. & gender (Li\.nga) \\
gen. & Genitive case (Cha\a{d}t\a{d}th\a{=}i) \\
imp. & Imperative mood (Pa\a~ncam\a{=}i) \\
imperf. & Imperfect tense (Hiyyattan\a{=}i) \\
ind. & Indeclinable (Avy\a{=}aya) \\
ins. & Instrumental case (Tatiy\a{=}a) \\
loc. & Locative case (Sattam\a{=}i) \\
m. & Masculine gender (Pulli\a{.}nga) \\
n. & Noun (N\a{=}ama) \\
nom. & Nominative case (Pa\a{d}tham\a{=}a) \\
nt. & Neuter gender (Napu\a{d}msakali\a{.}nga) \\
num. & Number (Vacana) \\
opt. & Optative mood (Sattam\a{=}i) \\
p.p. & Past Participle \\
perf. & Perfect tense (Parokkh\a{=}a) \\
pl. & Plural (Bahuvacana) \\
pr.p. & Present Participle \\
pres. & Present tense (Vattam\a{=}an\a{=}a) \\
pron. & Pronoun (Sabban\a{=}ama) \\
sg. & Singular (Ekavacana) \\
v. & Verb (\a{=}Akhay\a{=}ata) \\
v.i. & Intransitive verb \\
v.t. & Transitive verb \\
voc. & Vocative case (\a{=}Alapana) \\
\end{longtable}
