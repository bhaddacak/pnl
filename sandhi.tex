\chapter{\headhl{Sandhi} (Word Joining)}\label{chap:sandhi}

Whereas most traditional textbooks that put Sandhi to the first chapter after the sound system is introduced, meaning that it should be learned at the very beginning, I mention this topic very late in our lessons. The main reason is that Sandhi is best learned by seeing it (a lot). However, for new students who have not yet seen it a lot enough, it is somewhat baffling and sometimes frustrating when they find that some simple terms are not in a dictionary where they really should be.

What is Sandhi then? It is roughly about combining words, but totally different from compounds (see Appendix \ref{chap:samasa}). The main purpose of word combination in compounds is about grammatical contraction. Whereas Sandhi has something to with sound or phonetic level, not meaning or grammatical functions. When two words, or alphabets at lower level, are juxtaposed, they can be welded or joined together as a single sound unit. There are many rules enumerated by textbooks. These rules came mostly from observations. They are not prescriptive. That means you can choose whether to obey the rules or not, or you can use them as long as you see suitable. The main benefit of learning Sandhi is ability to recognize terms when we read texts, because Sandhi is used extensively throughout the texts. When you use it to compose a sentence, mostly in conversations, it can save your time and energy by blending some words together. It has stylistic side as well, like when you say ``gonna'' or ``gimme'' in English. Sandhi is also an indispensable tool in composing verses for making terms fit the meter.

I will not talk about Sandhi as the tradition does, because the traditional way is overwhelming with rules. Some rules are established for only a single instance found. I see little use of such rules. I will teach you by examples first and (some) rules later. That is the fastest way to learn.

Before we go to the list, there are some terminology concerning Sandhi that we have to know, in case you go digging further in the textbooks. I inevitably follow the tradition here. Sandhi can be divided roughly into 3 types: \pali{sarasandhi} (joining vowels), \pali{bya\~njanasandhi} (joining consonants), and \pali{niggahitasandhi} (joining \pali{\d m}) called \pali{vomissasandhi} by Sadd. The last two are somehow misnomer, because all P\=ali words end with a vowel, if not \pali{\d m}, and no single word starts with \pali{\d m}. So, \pali{bya\~njanasandhi} precisely means joining the vowel of the first term to the consonant of the second. And \pali{niggahitasandhi} precisely means joining \pali{\d m} with anything, except \pali{\d m} itself.

Moreover following Sadd, Sandhi can be divided further to \pali{padasandhi} and \pali{va\~n\~nasandhi}. The former is the combination between terms, e.g.\ \pali{tatra} + \pali{aya\d m} = \pali{tatr\=aya\d m}. The latter is between letters, e.g.\ \pchangeto{khattiy\=a}{khaty\=a}. We will see more of these in due course.

The main approach in traditional textbooks is to learn tools for making Sandhi, \pali{sandhikiriyopakara\d na}.\footnote{Sadd\,24} The most used \emph{elision} (\pali{lopa})\footnote{from Kacc\,12, R\=upa\,13, Sadd\,30, and Mogg\,1.26 onwards} is one of them, for instance. I will not tell you all these tools. You just see what happens and remember the pattern. That is the way I learn them without knowing what I see are called.

From phonetic point of view, there are relations between \pali{i} and \pali{e} and \pali{y}, and between \pali{u} and \pali{o} and \pali{v}. So, these vowels and consonants can be changed to one another (see the end of Chapter \ref{chap:nuts}). Many other transformations can also occur, even non-transformative connection (\pali{pakatisandhi}). I will make remarks in the table only for some noteworthy points. The table below has a good coverage, but I do not include all of instances formulated by the textbooks.

A simple guide to learn the table is to go through the items one by one, and try figuring out why they are so. You may find some recognizable patterns. That is good, but do not take them seriously. There is no rigid rule of Sandhi. It is mostly about optional operation upon words. Sometimes they go likewise but sometimes they do not. The best way to learn is to be familiar with unusual terms as many as possible, particularly terms that are composed with the common ones, e.g.\ \pali{iti, iva, eva, so, ta\d m, aha\d m,} etc.

\bigskip
\begin{longtable}[c]{@{}>{\itshape}p{0.44\linewidth}%
	>{\itshape}p{0.5\linewidth}@{}}
\toprule
\bfseries\upshape Specimen & \bfseries\upshape Former form \\ \midrule
\endfirsthead
\toprule
\bfseries\upshape Specimen & \bfseries\upshape Former form  \\ \midrule
\endhead
\bottomrule
\ltblcontinuedbreak{2}
\endfoot
\bottomrule
\endlastfoot
%
yassindriy\=ani & yassa + indriy\=ani \\
sadhindriya\d m & sadh\=a + indriya\d m \\
no heta\d m, noheta\d m & no hi + eta\d m \\
bhikkhunov\=ado & bhikkhun\=i + ov\=ado \\
samet\=ayasm\=a & sametu + \=ayasm\=a \\
abhibh\=ayana\d m & abhibh\=u + \=ayatana\d m \\
putt\=a matthi & putt\=a me + atthi \\
asantettha & asanto + ettha \\
nasi & na + asi \\
ajjuposatho & ajja + uposatho \\
eken\=un\=ani & ekena + \=un\=ani \\
yassete & yassa + ete \\
sotuk\=amattha & sotuk\=am\=a + attha \\
m\=avuso & m\=a + avuso \\
sabb\=itiyo & sabb\=a + \=itiyo \\
n\=agan\=as\=ur\=u & n\=agan\=as\=a + \=ur\=u \\
labhantatthe & labhanti + atthe \\
uddh\=umiyo & uddhi + \=umiyo \\
aggobh\=aso & aggi + obh\=aso \\
itth\=aya\d m & itth\=i + aya\d m \\
r\=ajin\=ur\=u & r\=ajin\=i + \=ur\=u \\
ucchagga\d m & ucchu + agga\d m \\
\=anenteta\d m & \=anentu + eta\d m \\
m\=atupa\d t\d th\=ana\d m & m\=atu + upa\d t\d th\=ana\d m \\
vijjobh\=aso & vijju + obh\=aso \\
jambissaro & jamb\=u + issaro \\
jambont\=a & jamb\=u + ont\=a \\
matthi & me + atthi \\
meta\d m & me + eta\d m \\
mok\=aso & me + ok\=aso \\
es\=avuso & eso + \=avuso \\
satt\=upalabbhati & satto + upalabbhati \\
kutettha & kuto + ettha \\
sopi & so + api \\
s\=ava & s\=a + iva \\
papa\d m & pa + \=apa\d m \\
pad\=atave, p\=ad\=atave & pa + \=ad\=atave \\
iti\footnote{Sadd\,33} & i + iti \\
bandhusseva & bandhussa + iva \\
nopeti & na + upeti \\
v\=amor\=u & v\=ama + ur\=u \\
v\=aterita\d m & v\=ata + \=irita\d m \\
ateva\~n\~nehi & ati + iva + a\~n\~nehi \\
vodaka & vi + udaka \\
tasseda\d m & tassa + ida\d m \\
lat\=ava\footnote{Sadd\,38}, lateva\footnote{Sadd\,40; Mogg\,1.28} & lat\=a + iva \\
patin\=ava, patineva & patin\=a + iva \\
ceti & ca + iti \\
gu\d neneti & gu\d nena + iti \\
sa\~n\~n\=ati & sa\~n\~n\=a + iti \\
r\=aj\=ati & r\=aj\=a + iti \\
c\=ubhaya\d m & ca + ubhaya\d m \\
saddh\=idha & saddh\=a + idha \\
buddh\=anussati & buddha + anussati \\
tatr\=aya\d m & tatra + aya\d m \\
lokuttara\d m & loka + uttara\d m \\
n\=ayyo & na + ayyo \\
n\=a\~n\~nama\~n\~nassa & na + a\~n\~nama\~n\~nassa \\
n\=agghanti & na + agghanti \\
n\=assudha & na + assudha \\
n\=assa & na + assa \\
m\=ayyo & m\=a + ayyo \\
m\=assu & m\=a + assu \\
tad\=assu & tad\=a + assu \\
kad\=assu & kad\=a + assu \\
v\=assa & v\=a + assa \\
tasm\=assa & tasm\=a + assa \\
tatr\=assa & tatra + assa \\
ta\d nh\=assa & ta\d nh\=a + assa \\
katv\=atra & katv\=a + atra \\
s\=anutev\=asiko & sa + anutev\=asiko \\
s\=attha\d m & sa + attha\d m \\
s\=atthik\=a & sa + atthik\=a \\
sattho & sa + attho \\
s\=adh\=uti & s\=adhu + iti \\
munelayo & muni + \=alayo \\
rathesabho & rath\=i + usabho \\
sotth\=i & su + itth\=i \\
ty\=aha\d m & te + aha\d m \\
ty\=assa & te + assa \\
my\=aya\d m & me + aya\d m \\
yassa & ye + assa \\
yassu & ye + assu \\
y\=abhivadanti & ye + abhivadanti \\
y\=avatakvassa & y\=avatako + assa \\
khvassa & kho + assa \\
cakkhv\=ap\=atham\=agacchati & cakkhu + \=ap\=atha\d m + \=agacchati \\
p\=atv\=ak\=asi & p\=atu + ak\=asi \\
yatv\=adhikara\d na\d m & yato + adhikara\d na\d m \\
vatthvettha & vatthu + ettha \\
dv\=ak\=are & du + \=ak\=are \\
anuv\=agantv\=ana & anu + \=agantv\=ana \\
yv\=aya\d m & yo + aya\d m \\
sv\=assa & so + assa \\
sv\=agata\d m & su + \=agata\d m \\
bahv\=ab\=adho & bahu + \=ab\=adho \\
hetuttho, hetuattho & hetu + attho \\
dh\=atuttho & dh\=atu + attho \\
hetindriy\=ani & hetu + indriy\=ani \\
khandhadh\=at\=ayatan\=ani & khandhadh\=atu + \=ayatan\=ani \\
iccassa & iti\footnote{\pchangeto{ti}{cc}; Kacc\,19; R\=upa\,22; Sadd\,46. But Mogg\,1.30, 1.48, and 1.49 explain that there is a phonetic operation in process making, \pali{ti} $\rightarrow$ \pali{tya} $\rightarrow$ \pali{cya} $\rightarrow$ {cca}. Hence, \pali{iti + assa} becomes \pali{ityassa}, then becomes \pali{iccassa}.} + assa \\
icceta\d m & iti + eta\d m \\
accanta\d m & ati + anta\d m \\
pacc\=aharati & pati + \=aharati \\
paccuttaritv\=a & pati + uttaritv\=a \\
at\=isiga\d no & ati + isiga\d no \\
at\=irita\d m & ati + \=irita\d m \\
at\=ito & ati + ito \\
pat\=ito & pati + ito \\
it\=iti & iti + iti \\
it\=ida\d m & iti + ida\d m \\
pa\d n\d dit\=atyamha & pa\d n\d dit\=a + iti + amha \\
itveva & iti\footnote{\pchangeto{ti}{tv}; Sadd\,49, Mogg\,1.36} + eva; \\
vilapatveva & vilapati + eva \\
isigilitveva & isigiliti + eva \\
ekamid\=aha\d m & eka\d m\footnote{\pchangeto{\d m}{m}} + idha\footnote{\pchangeto{dha}{da}; Kacc\,20, R\=upa\,27, Sadd\,50} + aha\d m \\
idheva & idha + eva \\
evamidhekacco & eva\d m + idha + ekacco \\
pa\d tisanth\=aravutyassa & pa\d tisanth\=aravutti + assa \\
vity\=anubh\=uyate & vitti + anubh\=uyate \\
by\=ak\=asi &  vi\footnote{\pchangeto{vi}{bya}} + \=a + ak\=aki \\
bya\~njana\d m & vi + a\~njana\d m \\
by\=akato & vi + \=akato \\
d\=asy\=aha\d m & d\=as\=i + aha\d m \\
abbhud\=irita\d m & abhi\footnote{\pchangeto{abhi}{abbha}; Kacc\,44, R\=upa 24, Sadd\,57} + ud\=irita\d m \\
abbhuggacchati & abhi + uggacchati \\
ajjh\=agam\=a & adhi\footnote{\pchangeto{adhi}{ajjha}; Kacc\,45, r\=upa 25, Sadd\,58} + \=agam\=a \\
ajjh\=aharati & adhi + \=aharati \\
ajjhok\=ase & adhi + ok\=ase \\
abhicchita\d m & abhi + icchita\d m \\
adh\=irita\d m, abbh\=irita\d m & adhi + \=irita\d m \\
ajjhi\d nmutto & adhi + i\d namutto \\
yathariva & yath\=a + eva\footnote{\pchangeto{eva}{riva}; Kacc\,22, R\=upa\,28, Sadd\,52} \\
tathariva & tath\=a + eva \\
manu\~n\~na\d m & mano\footnote{\pchangeto{o}{u}; Sadd\,55} + a\~n\~na\d m \\
gavassa\d m & go\footnote{\pchangeto{o}{ava}; Mogg\,1.32} + assa\d m \\
idhappam\=ado\footnote{A consonant is duplicated ; Kacc\,28, R\=upa\,40, Sadd\,67} & idha + pam\=ado \\
c\=atuddas\=i & c\=atu + das\=i \\
pa\~ncaddas\=i & pa\~nca + das\=i \\
abhikkantataro & abhi + kantataro \\
cajjh\=anapphalo\footnote{The consonant's voiced or voiceless pair is added; Kacc\,29, R\=upa\,42, Sadd\,68, Mogg\,1.35; e.g.\ \pchangeto{kh}{kkh}, \pchangeto{gh}{ggh}, \pchangeto{ch}{cch}, \pchangeto{jh}{jjh}, and so on} & ca + jh\=anapphalo \\
yatra\d t\d thita\d m & yatra + \d thita\d m \\
viddha\d mseti & vi + dha\d mseti \\
vibbhamati & vi + bhamati \\
nigghoso & ni + ghoso \\
akkhanti & a + khanti \\
pa\d taggi & pati\footnote{\pchangeto{pati}{pa\d ti}; Kacc\,48, R\=upa\,43, Sadd\,137} + aggi \\
pa\d tiha\~n\~nati & pati + ha\~n\~nati \\
puthujjano & putha\footnote{\pchangeto{putha}{puthu}; Kacc\,49, R\=upa\,44, Sadd\,129} + jano \\
puthubh\=uta\d m & putha + bh\=uta\d m \\
onaddh\=a & ava\footnote{\pchangeto{ava}{o}; Kacc\,50, R\=upa\,45, Sadd\,126} + naddh\=a \\
ovadati & ava + vadati \\
os\=ana\d m & ava + s\=ana\d m \\
avekkhati & ava + ikkhati \\
s\=ahu & s\=adhu\footnote{\pchangeto{dha}{ha}; Sadd\,72. In Sadd\,72--133, Aggava\d msa shows that some characters can be changed to another, like this one. They are too many to list here. It is a kind of redundancy, for we mostly find the terms in a dictionary. However, I list some here because they look interesting in certain way.} \\
jaccandho & j\=ati\footnote{(\pali{ti}) \pchangeto{tya}{cca}, (\pali{di}) \pchangeto{dya}{jja}; Sadd\,104} + andho \\
yajjeva\d m & yadi + eva\d m \\
agy\=ag\=ara\d m\footnote{Triple consonant can be reduced; Sadd\,120} & aggi + \=ag\=ara\d m \\
guyha\footnote{Consonant can be interchanged; Mogg\,1.50, Sadd\,154} & guhya \\
bavuh\=ab\=adho & bahuv\=ab\=adho \\
kayira & kariya \\
makas\=a & masak\=a \\
ayir\=a & ariy\=a \\
yathayida\d m\footnote{\pali{ya} is added. Consonants able to be added in this way are \pali{ya, va, ma, da, na, ta, ra, la, \d la, ha,} and \pali{ga}. Kacc\,35, R\=upa\,34, Sadd\,56, Mogg\,1.45--6. See also \emph{junction consonants} in \citealp[p.~255]{warder:intro}.} & yath\=a + ida\d m \\
tiva\.ntika\d m & ti + a\.ntika\d m \\
lahumessati & lahu + essati \\
sama\d namacalo & sama\d na + acalo \\
sammadeva & samm\=a + eva \\
aggadattha\d m & agga + attha\d m \\
ajjadagge & ajja + agge \\
ito n\=ayati & ito \=ayati \\
yasm\=atiha & yasm\=a + iha \\
sabbhireva & sabbhi + eva \\
\=araggeriva & \=aragge + iva \\
cha\d labhi\~n\~n\=a & cha + abhi\~n\~n\=a \\
sa\d l\=ayatana\d m & cha + \=ayatana\d m \\
suhuju & su + uju \\
suhu\d t\d thita\d m & su + u\d t\d thita\d m \\
puthageva\footnote{Kacc\,42, R\=upa\,32, Sadd\,53} & putha + eva \\
puthagaya\d m & putha + aya\d m \\
pageva\footnote{Kacc\,43, R\=upa\,33, Sadd\,54} & p\=a + eva \\
parosahassa\d m\footnote{\pali{o} is added; Kacc\,36, R\=upa\,47, Sadd\,130} & para + sahassa\d m \\
saradosata\d m &  sarada + sata\d m \\
cakkhu\d m udap\=adi\footnote{\pali{\d m} is added (sometimes also changed to nasal consonants); Kacc\,37, R\=upa\,57, Sadd\,146, Mogg\,1.38} & cakkhu udap\=adi \\
ava\d msiro & avasiro \\
y\=ava\~ncidha & y\=ava + ca + idha \\
a\d nu\d mth\=ul\=ani & a\d nu + th\=ul\=ani \\
pubba\.ngam\=a & pubba + gam\=a \\
d\=ipa\.nkaro & d\=ipa\d m\footnote{\pchangeto{\d m}{nasal consonants}; Kacc\,31, R\=upa\,49, Sadd\,138, Mogg\,1.41} + karo \\
dhamma\~ncare & dhamma\d m + care \\
sa\d n\d thiti & sa\d m + \d thiti \\
tanniccuta\d m & ta\d m + niccuta\d m \\
sa\d mghasammato & sa\d mgha + sa\d m + mato \\
pulli\.nga\d m & pu\d m\footnote{\pchangeto{\d m}{l}; Sadd\,139} + li\.nga\d m \\
sallakkhan\=a & sa\d m + lakkhan\=a \\
asall\=ina\d m & asa\d m + l\=ina\d m \\
pa\d tisall\=ino & pa\d tisa\d m + l\=ino \\
paccatta\~n\~neva & paccatta\d m\footnote{\pchangeto{\d m}{\~n}; Kacc\,32, R\=upa\,50, Sadd\,140, Mogg\,1.42} + eva \\
ta\~n\~neva & ta\d m + eva \\
eva\~nhi & eva\d m + hi \\
ta\~nhi & ta\d m + hi \\
sa\~nhito & sa\d m + hito \\
sa\~n\~nogo, sa\d myogo\footnote{Kacc\,33, R\=upa\,51, Sadd\,141, Mogg\,1.43} & sa\d m + yogo \\
sa\~n\~nutta\d m, sa\d myutta\d m & sa\d m + yutta\d m \\
\mbox{sa\~n\~nyojana\d m, sa\d myojana\d m} & sa\d m + yojana\d m \\
tamaha\d m & ta\d m\footnote{\pchangeto{\d m}{m, d}; Kacc\,34, R\=upa\,52, Sadd\,142--5, Mogg\,1.44} + aha\d m \\
etadavoca & eta\d m + avoca \\
yadabravi & ya\d m + abravi \\
tadev\=aramma\d na\d m & ta\d m + eva + \=aramma\d na\d m \\
yam\=ahu & ya\d m + \=ahu \\
tamattha\d m & ta\d m + attha\d m \\
etamattha\d m & eta\d m + attha\d m \\
yadantara\d m & ya\d m + anantara\d m \\
tadantara\d m & ta\d m + anantara\d m \\
etadattho & eta\d m + attho \\
evameta\d m & eva\d m + eta\d m \\
ahameva & aha\d m + eva \\
tvameva & tva\d m + eva \\
tayida\d m & ta\d m\footnote{\pchangeto{\d m}{y}; Mogg\,1.44} + ida\d m \\
tadate\footnote{Sadd\,131} & ta\d m + te \\
etadaki\~nci & eta\d m + ki\~nci \\
t\=as\=aha\d m & t\=asa\d m\footnote{\remove{\d m}; Kacc\,38--9, R\=upa\,53--4, Sadd\,147, Mogg\,1.39} + aha\d m \\
vid\=unagga\d m & vid\=una\d m + agga\d m \\
sabbadass\=av\=i & sabba\d m + dass\=av\=i \\
ariyasacc\=ana dassana\d m & ariyasacc\=ana\d m dassana\d m \\
eta\d m buddh\=ana s\=asana\d m & eta\d m buddh\=ana\d m s\=asana\d m \\
abhinandunti & abhinandu\d m + iti \\
uttatta\d mva & uttatta\d m + iva \\
yath\=ab\=ija\d mva & yath\=ab\=ija\d m + iva \\
idampi & ida\d m + api \\
kind\=ani & ki\d m + id\=ani \\
tva\d msi & tva\d m + asi \\
sadisa\d mva & sadisa\d m + eva \\
eva\d msa & eva\d m + assa \\
puppha\d ms\=a & puppha\d m + ass\=a \\
tadamin\=a & ta\d m + imin\=a \\
evuma\d m\footnote{Sadd\,150} & eva\d m + ima\d m \\
kaha\d m, keha\d m\footnote{Sadd\,151} & ka\d m + aha\d m \\
sak\=ad\=ag\=am\=i\footnote{This item and the rest towards the end are from Mogg\,1.47. I list these to show that how wild Sandhi can go. The original form of some terms are close to what we call \emph{analytic form}. That is to say, they are better to see these as compounds rather than terms produced by Sandhi process. There are also many outlandish transformations described in the textbooks I left out. Do not take these seriously. It is unlikely that you will make your own words like these. And it is likely that you can find the words in a dictionary by their own right if they are really used somewhere.} & saki\d m + \=ag\=am\=i \\
sa\d mvid\=avah\=aro & sa\d mvidh\=aya + avah\=aro \\
val\=ahako & v\=arino + v\=ahako \\
j\=imuto & j\=ivanassa + muto \\
sus\=ana\d m & chavassa + sayana\d m \\
udukkhala\d m & uddha\d m + khamassa \\
may\=uro & mahiya\d m + ravat\=iti \\
\end{longtable}



