\chapter{\headhl{Sam\=asa} (Compounds)}\label{chap:samasa}

As the time I was preparing materials on compounds\footnote{A compound word is ``A word consisting of two or more stems which may themselves be words, as in \emph{arm+chair}, or parts of words, as in \emph{retro+spect}'' \citep[p.~93]{brownmiller:dict}.}, I thought whether we really have to know these things concerning the present book. I have never taken explanations on compounds seriously as the tradition expects. In English we hardly have a theory why or how words bunch together as a unit. They are just so. We happily find them in a dictionary and create some new ones with no difficulty. Nevertheless, in P\=ali, compounds, or Sam\=asa in traditional terminology, are quite a big deal. Every textbook has a dedicated chapter for them. It is good to know, but practically it is not essential. So, I put these materials in the appendices. For those who are curious how words put together, you can go through this chapter optionally. And for those who want to go deeper in P\=ali studies, it is still important to know all of this.

Unlike Sandhi that has things to do with sound, Sam\=asa is combination of meaning, mostly from nouns\footnote{Kacc\,316, R\=upa\,331}, but prefixes and particles can also be a component\footnote{Sadd\,675}. It is very helpful in inflectional languages like P\=ali, because you can put several words with the same declension into a single unit.\footnote{Mogg\,3.1, Niru\,331} As you will learn in due course, different cases can also be put together. Most of the time, when inflected words are put together, the redundant endings are left out. For example, \pali{sama\d nabr\=ahma\d n\=a} (ascetics and brahmans) comes from \pali{sama\d n\=a ca br\=ahma\d n\=a}. Don't be tempted to think this is a simple word joining (Sandhi) with a vowel elided. You have to put \pali{ca} (and) in the meaning of the whole unit, whereas Sandhi has nothing to do with \pali{ca} if it is not present. Where does this \pali{ca} come from? That is a good question. Well, put it this way, when no one tells you what is hidden in the compound, you have to guess it yourselves. The whole job of learning Sam\=asa in P\=ali is to determine this hidden things and postulate an \emph{analytic sentence} of the term. We will learn this as well. An example of analytic sentence of the above example is ``\pali{sama\d n\=a ca br\=ahma\d n\=a sama\d nabr\=ahma\d n\=a hoti}'' (Ascetics and brahmans is ascetic-brahmans). It is pretty easy and straight forward in this example, but many are more difficult than this.

When compounds are composed from different cases, sometimes the declensions are retained. For example, \pali{d\=urenid\=ana\d m} (a long past story) comes from \pali{d\=ure} (loc.) + \pali{nid\=ana\d m} (nom.). Sometimes meaning of the unit is somehow related to its sources, as we have seen in previous examples. Sometimes it is not. For example \pali{urasilomo} (hair on chest) means someone who has hair on his chest. This is a completely new word with a new reference. If you mistake it as a Sandhi, you are doomed.

Before various kinds of compounds are elaborated, general principles should be addressed first. As we have a glimpse above, the \pali{vibhatti} (ending) of nouns according to their cases sometimes is elided.\footnote{Kacc\,317, R\=upa\,332, Sadd\,676} When the ending is deleted, their original form is restored.\footnote{Kacc\,138, R\=upa\,333, Sadd\,693} For example, \pali{ra\~n\~no putto} (king's son) becomes \pali{r\=ajaputto} (gen.\ ending is elided). Sometimes the \pali{vibhatti} is retained\footnote{Sadd\,686}, for example, \pali{manasik\=aro} (action in mind = consideration).
Other additional parts of verbs, compounds, and derivatives may also be elided.\footnote{Sadd\,677} I leave out other trivial principles discribed in the textbooks, for you can capture the big picture easily. 

To put it in my terms, the meta-rule of compounds is there are no rules at all. When you create some yourselves, do it in a proper way; in an intelligible way, I mean. When you read texts and find some of them, try breaking down the components. If everything is clear, it is fine; if not, just do some guesswork. Sometimes it is easy to crack the code, sometimes it is not. That is the real nature of compounds in P\=ali. Every students, even experts, have to deal with them in this way. It can be more manageable if we are familiar with typical kinds of compounds. There are six kinds of compounds described in the textbooks. Names of them are varied according to schools. I summarize these in Table \ref{tab:samasa}. For our concern, we will follow Kacc\=ayana and Saddan\=iti schools.

\begin{table}[!hbt]
\centering
\caption{Types of P\=ali compounds}
\label{tab:samasa}
\bigskip
\begin{tabular}{l*{2}{>{\itshape}l}r} \toprule
& \bfseries\upshape Kacc, Sadd & \bfseries\upshape Mogg & \bfseries Page \\ 
\midrule
1. & Abyay\=ibh\=ava & Asa\.nkhayattha & \pageref{sec:abyayi} \\
2. & Kammadh\=araya & Visesana & \pageref{sec:kamma} \\
3. & Digu & Visesana & \pageref{sec:digu} \\
4. & Tappurisa & Am\=adi & \pageref{sec:tappu} \\
5. & Bahubb\=ihi & A\~n\~nattha & \pageref{sec:bahub} \\
6. & Dvanda & Cattha & \pageref{sec:dvan} \\
\bottomrule
\end{tabular}
\end{table}

\section*{Analytic sentence of compounds}

Before we go into each type, it is better to talk about analytic sentence of a compound a little bit. The analytic sentence decomposes the compound and clarifies what it really means. There is no exact principle about this. By traditional way of learning, students are encouraged to postulate it when they meet a compound. If you are the one who create that compound, the analytic sentence is your declaration of it, or better the instructional manual of it. If the compound is the established one, the analytic sentence is the explanation of it. It is true that different persons and contexts can generate different analytic sentences. There is no single right explanation. Some may be better than others. 

Here is a practical example of analytic sentence of \pali{mah\=amaggo} (highway). You can simply write the sentence as a mathematic equation, such as \pali{mahanto} + \pali{maggo} = \pali{mah\=amaggo} (big + way = highway). This is not fashionable in traditional schools, but sometimes it make better understanding for modern minds. To make it traditional style, you have to put this in sentence structure using verb `to be.' Then we get this:

\palisample{mahanto maggo mah\=amaggo (hoti).}

As we have learned that verb `to be' in P\=ali is mostly negligible, so it is normally left out. That is a short form. To be more sophisticated, the tradition uses a full form of the analytic sentence as follows:

\palisample{mahanto ca so maggo c\=ati mah\=amaggo.}

This can be rendered as ``That way and big (way) also, thus highway.'' We add \pali{so} to specify the object. We have two \pali{ca}s to connect the meaning. And we add \pali{iti} (\pali{c\=ati} = \pali{ca + iti}) to mark the end term (think it as an equal sign). That the way the tradition does it, a little nitpicky. If you go through traditional textbooks, you will meet this a lot. This form is only for \pali{Kammadh\=arayasam\=asa}. Other types of compound use different structures of analytic sentence. I will not go to explain all of those. You have to observe by yourselves. New students, however, can ignore them altogether, except ones explained in detail.

The word \pali{mah\=a} is a good place to start, because it is used so extensively that it becomes an independent word.\footnote{In Sadd-Pad Ch.\,7, Aggava\d msa shows that \pali{mah\=a} is nom.\ of \pali{mahanta}. So, when we use it in compounds, we use its nominative form. See also Kacc\,330, R\=upa\,340, Sadd\,710--2. Sometimes it becomes \pali{maha} (Sadd\,713), e.g.\ \pali{mahapphala\d m} (fruitful).} It is very handy to use. For example, there is no `bus' in P\=ali scriptures. Now we have to say it, then we create it simply as \pali{mah\=aratho} (a big car). Here is its analytic sentence: ``\pali{mahanto ratho mah\=aratho}.'' If you come up with a better idea, you can propose your word with its manual. For instance, I think that a bus has many windows, then I call it \pali{bahuv\=atap\=anaratho} (a multi-windowed car). And this is its analytic sentence: ``\pali{yassa rathassa bahuk\=a v\=atap\=an\=a santi, so bahuv\=atap\=anaratho hoti}'' (Which car has many windows, that car is `a bus'). That makes sense but it is a mouthful to say. So, no one will ever use my word because it is too difficult to say.

Now you see how important analytic sentence of compounds is, in the traditional point of view. You are encouraged to do likewise. There are some technical terms concerning this matter we have to know. When a compound is broken down into two parts, the first part is called \pali{pubbapada} (the former term), e.g.\ \pali{mahanto} in the above example, and the second \pali{uttarapada} (the latter term), e.g.\ \pali{ratho} above. We will meet these in due course.

\section*{1. \pali{Abyay\=ibh\=avasam\=asa}}\label{sec:abyayi}

Compounds of \pali{Abyay\=ibh\=ava} are those which have \pali{upasagga} (prefixes) or \pali{nip\=ata} (particles) as the first part (\pali{pubbapada}).\footnote{Kacc\,319, R\=upa\,330, Sadd\,695--6} This kind of compounds ends up as neuter (nt.) nouns\footnote{Kacc\,320, R\=upa\,335, Sadd\,698, Mogg\,3.9, Niru\,334} or adjectives.

Here are examples of compounds with \pali{upasagga} as the first part. I also show the analytic part of each instance. All examples come from Sadd\,696.\footnote{\citealp[pp.~746--50]{smith:sadd3}}

$\bullet$ \pali{upa} in the sense of `vicinity' (\pali{sam\=ipa})\par
\palibf{upanagara\d m} (\pali{nagarassa sam\=ipa\d m})\par \hspace{3mm} = a suburb, outskirt of a city\par
\palibf{upaga\.nga\d m} (\pali{ga\.ng\=aya sam\=ipa\d m})\par \hspace{3mm} = neighboring area of the Ganges\par
\palibf{upavadhu} (\pali{vadhuya sam\=ipa\d m})\par \hspace{3mm} = an area near a girl\par
\palibf{upagu} (\pali{gunna\d m sam\=ipa\d m})\par \hspace{3mm} = an area near cattle

$\bullet$ \pali{ni} in the sense of `non-existence' (\pali{abh\=ava})\par
\palibf{niddaratha\d m}\footnote{The full analytic sentence given by Sadd is ``\pali{natthi daratho ettass\=ati niddaratho, puriso}'' (No anxiety for that person, thus anxiety-free)}(\pali{darathassa abh\=avo})\par \hspace{3mm} = absence of anxiety\par
\palibf{nimmakasa\d m}\footnote{The full analytic sentence given by Sadd is ``\pali{natthi makas\=a etth\=ati nimmakasa\d m, \d th\=ana\d m}'' (No mosquitos in that place, thus mosquito-free)} (\pali{makas\=ana\d m abh\=avo})\par \hspace{3mm} = absence of mosquito

$\bullet$ \pali{anu} in the sense of `going after' (\pali{pacch\=a})\par
\palibf{anuratha\d m} (\pali{rathassa pacch\=a})\par \hspace{3mm} = the rear part of a car\par
\palibf{anuv\=ata\d m} (\pali{v\=atassa pacch\=a})\par \hspace{3mm} = the aftermath of wind

$\bullet$ \pali{anu} in the sense of `suitableness' (\pali{yogga\d m})\par
\palibf{anur\=upa\d m} (\pali{r\=upassa yogga\d m})\par \hspace{3mm} = suitableness of form (mostly used as adj.\ suitable)

$\bullet$ \pali{pati, anu} in the sense of `distributed individuality' (\pali{vicch\=a}\footnote{This technical term means repetition to make individual distribution. Aggava\d msa shows two lines of account concerning these instances. The first is from grammarians (\pali{akkharacintaka}) who give the analytic parts as ``\pali{att\=ana\d m att\=ana\d m pati paccatta\d m}'' and ``\pali{addham\=asa\d m addham\=asa\d m anu anvaddham\=asa\d m}.'' The second is from commentators (\pali{a\d t\d thakath\=acariya}) who give those shown above. For more information about repetition, see Chapter \ref{chap:adv}, page \pageref{sec:repetition}.})\par
\palibf{paccatta\d m} (\pali{pati pati att\=ana\d m})\par \hspace{3mm} = individuality of self (often used as adv.\ meaning `individually' or `separately')\par
\palibf{anvaddham\=asa\d m} (\pali{anu anu addham\=asa\d m})\par \hspace{3mm} = every fortnight\par
\palibf{anughara\d m} (\pali{anu anu ghara\d m})\par \hspace{3mm} = every individual household

$\bullet$ \pali{anu} in the sense of `succession' (\pali{anupubbi})\par
\palibf{anuje\d t\d tha\d m} (\pali{anuje\d t\d th\=ana\d m anupubbo})\par \hspace{3mm} = order of brotherhood

$\bullet$ \pali{pa\d ti} in the sense of `counteraction' (\pali{anuloma\d m})\par
\palibf{pa\d tisota\d m} (\pali{sotassa pa\d tiloma\d m})\par \hspace{3mm} = counteraction of stream (against the steam)

$\bullet$ \pali{adhi} in the sense of `causal contribution' (\pali{adhikacca pavatta\d m})\par
\palibf{ajjhatta\d m} [\pali{adhi + atta}] (\pali{att\=ana\d m adhikacca pavatta\d m})\par \hspace{3mm} = that which is personal, subjective; that which arises from within\footnote{This technical term has a lot to do with the Buddhist doctrine. Aggava\d msa adds that it is the eye, which is an internal sense-base, for example (\pali{cakkh\=adi}).}\par
\palibf{adhicitta\d m} (\pali{cittamadhikacca pavatta\d m dhammaj\=ata\d m})\par \hspace{3mm} = a nature which is contributed by the mind\footnote{This term is purely technical. It is never translated literally. It particularly means meditation or concentration, maintained by Aggava\d msa. Analyzed another way, \pali{adhicita\d m} can be of \pali{kammadh\=araya}, i.e.\ \pali{adhika\d m citta\d m adhicitta\d m}. This makes \pali{adhicitta\d m} means `superior mind' which again denotes meditation.}\par
\palibf{adhitthi} [\pali{adhi + itth\=i}] (\pali{itth\=isu eka\d m adhikacca kath\=a pavattati, s\=a kath\=a adhitthi})\par \hspace{3mm} = a conversation to one woman among many others.

$\bullet$ \pali{\=a} in the sense of `setting limit' (\pali{mariy\=ad\=abhividhi})\par
\palibf{\=ap\=a\d nako\d tiya\d m}\footnote{In a dictionary, we find \pali{\=ap\=a\d nako\d tika}.} (\pali{\=a p\=a\d nako\d tiy\=a})\par \hspace{3mm} = limited with the end of life\par
\palibf{\=akom\=ara\d m} (\pali{\=a kom\=ar\=a yaso kacc\=ayanassa})\par \hspace{3mm} = spreading to children (Ven.\,Kacc\=ayana's fame)

$\bullet$ \pali{su} in the sense of `prosperity' (\pali{samiddhi})\par
\palibf{subhikkha\d m} (\pali{bhikkh\=ana\d m samiddhi})\par \hspace{3mm} = prosperity of food

$\bullet$ \pali{du} in the sense of `scarcity' (\pali{asamiddhi})\par
\palibf{dubbhikkha\d m} (\pali{bhikkh\=ana\d m asamiddhi})\par \hspace{3mm} = scarcity of food

\bigskip
Here are examples of compounds with particles (\pali{nip\=ata}) as the first part.

$\bullet$ \pali{yath\=a} in the sense of `succession' (\pali{pa\d tip\=a\d ti})\par
\palibf{yath\=avu\d d\d dha\d m} (\pali{vu\d d\d dh\=ana\d m pa\d tip\=a\d ti})\par \hspace{3mm} = succession by seniority\par
\palibf{yath\=abhir\=upa\d m} (\pali{abhir\=up\=ana\d m pa\d tip\=a\d ti})\par \hspace{3mm} = succession by handsomeness

$\bullet$ \pali{yath\=a} in the sense of `repetition' (\pali{vicch\=a})\par
\palibf{yath\=avu\d d\d dha\d m} (\pali{ye ye vu\d d\d dh\=a})\par \hspace{3mm} = the elderly\footnote{Other some teachers (\pali{keci}) say that it can be distributed to each individual as we found elsewhere. So, it can mean the elders individually. Likewise, \pali{yath\=abhir\=upa\d m} can mean handsome ones individually (\pali{ye ye abhir\=up\=a}).}\par

$\bullet$ \pali{yath\=a} in the sense of ``not exceeding the boundary of term's meaning'' (\pali{padatth\=anatikkama})\par
\palibf{yath\=akkama\d m} (\pali{kama\d m anatikkamma pavattana\d m})\par \hspace{3mm} = in succession (not out of order)\par
\palibf{yath\=asatti} (\pali{satti\d m anatikkamma pavattana\d m})\par \hspace{3mm} = within one's own ability\par
\palibf{yath\=abala\d m} (\pali{bala\d m anatikkamma pavattana\d m})\par \hspace{3mm} = within one's own strength

$\bullet$ \pali{y\=ava} in the sense of ``demarcation'' (\pali{pariccheda})\par
\palibf{y\=avaj\=iva\d m} (\pali{j\=ivassa yattako paricchedo})\par \hspace{3mm} = for the length of one's life\par
\palibf{y\=avat\=ayuka\d m} (\pali{\=ayussa yattako paricchedo})\par \hspace{3mm} = for the length of one's life

$\bullet$ in other senses (\pali{parabh\=aga})\par
\palibf{tiropabbata\d m} (\pali{pabbatassa tiro})\par \hspace{3mm} = outside of the mountain\par
\palibf{antop\=as\=ada\d m} (\pali{p\=as\=adassa anto})\par \hspace{3mm} = inside of the castle\par
\palibf{bahinagara\d m} (\pali{nagarato bahi})\par \hspace{3mm} = outside of the city\par
\palibf{uparip\=as\=ada\d m} (\pali{p\=as\=adassa upari})\par \hspace{3mm} = inside of the castle\par
\palibf{he\d t\d th\=ama\~nca\d m} (\pali{ma\~ncassa he\d t\d th\=a})\par \hspace{3mm} = underneath of the bed\par
\palibf{purebhatta\d m} (\pali{bhattassa pure})\par \hspace{3mm} = before the food time\par
\palibf{pacch\=abhatta\d m} (\pali{bhattassa pacch\=a})\par \hspace{3mm} = after the food time\par

$\bullet$ \pali{sa} in the sense of ``all'' (\pali{s\=akalla})\par
\palibf{samakkhika\d m} (\pali{makkhik\=aya saha})\par \hspace{3mm} = eating all even a fly\footnote{Aggava\d msa gives us an additional account: ``\pali{tattha samakkhika\d m ajjhoharati, na ki\~nci parivajjet\=iti attho}'' (That term means ``[one] swallows even a fly, not leave anything out''). Likewise, \pali{sati\d na\d m} means ``eating all even grass.''}\par

Aggava\d msa adds an account that terms not formed by \pali{upasagga} or \pali{nip\=ata} but look similar count as \pali{Abyay\=ibh\=ava} as well, for example, \pali{ti\d t\d thagu} [\pali{\d th\=a + go}] (cattle stand), \pali{vahagu} [{\pali{vaha + go}}] (time or place to let cattle graze), and \pali{khaleyava\d m} [\pali{khala + yava}] (time when barley in the threshing ground).\footnote{Sadd\,697} These look like indeclinables because of their use of verb stem form as the first part. This is somewhat unusual.

\section*{2. \pali{Kammadh\=arayasam\=asa}}\label{sec:kamma}

Perhaps the most used, \pali{Kammadh\=araya} compound or \pali{Visesanasa\-m\=asa} is composed of two terms that have the same case (\pali{tuly\=adhikara\d na}).\footnote{Kacc\,324, R\=upa\,339, Sadd\,702} Put it another way, one or both terms functions as a modifier which agrees in case.\footnote{Mogg\,3.11} Aggava\d msa classifies nine types of constituent parts of this compound. We will follow this enumeration.

\paragraph*{(1) \pali{Visesanapubbapada}} (modifier as the first part)\par
Examples: \pali{mah\=apuriso}\footnote{\pali{mahanto ca so puriso c\=ati mah\=apuriso.}} (a great person), \pali{ka\d nhasappo} (a black snake), \pali{n\=iluppala\d m} (a blue waterlily), \pali{lohitacandana\d m} (a red sandalwood), \pali{khattiyaka\~n\~n\=a} (a girl of the warrior caste).

\paragraph*{(2) \pali{Visesanuttarapada}} (modifier as the second part)\par
Examples: \pali{s\=ariputtathero}\footnote{\pali{s\=ariputto ca so thero c\=ati s\=ariputtathero.}} (elder S\=ariputta), \pali{buddhaghos\=acariyo} (master Buddhaghosa), \pali{mahosadhapa\d n\d ditto} (wise man Mahosadha), \pali{sattaviseso} (a kind of being).

\paragraph*{(3) \pali{Visesanobhayapada}} (both modifiers)\par
Examples: \pali{gil\=anavu\d t\d thito}\footnote{\pali{gil\=ano ca so vu\d t\d thito c\=ati gil\=anavu\d t\d thito.}} (sick and getting well), \pali{sittasamma\d t\d tha\d m} (sprinkled and swept), \pali{andhabadhiro} (blind and deaf) \pali{kha\~njakhujjo} (lame and humpbacked).

\paragraph*{(4) \pali{Upam\=anuttarapada}} (simile as the second part)\par
Examples: \pali{buddhas\=iho}\footnote{\pali{s\=iho viya s\=iho, buddho ca so s\=iho c\=ati buddhas\=iho.}} (the lion-like Buddha), \pali{\~n\=a\d nacakkhu} (eye-like insight), \pali{pa\~n\~n\=ap\=as\=ado} (castle-like wisdom).

However, there is a good chance you will meet or compose the simile as the first part, for example, \pali{sa\.nkhapa\d n\d dara\d m} (white like a conch), \pali{k\=akas\=uro} (bold as a crow), \pali{dibbacakkhu} (divine-like eyes). These words by no means have rigid meaning. You have to know what you are doing. For example, \pali{k\=akas\=uro} somehow can mean `a bold crow' which becomes another kind of compound. When you use such a term, it is better to accompany it with a manual or an analytic sentence.

\paragraph*{(5) \pali{Sambh\=avan\=apubbapada}} (\pali{sambh\=avana} as the first part)\par
Examples: \pali{dhammabuddhi}\footnote{\pali{dhammoti buddhi dhammabuddhi.}} (knowledge of the Dhamma), \pali{dhammasa\~n\~n\=a} (recognition of the Dhamma), \pali{sama\d nasa\~n\~n\=a} (recognition of ascetic status), \pali{sattasa\~n\~n\=a} (recognition of being status), \pali{bhikkhupa\d ti\~n\~n\=a} (vow of monkhood), \pali{khattiyam\=ano} (conceit in warrior status).

This kind of compound is a bit difficult to understand, and to explain as well. This compound often a noun denoting mental state as the main/second term, such as \pali{buddhi} (knowledge), \pali{sa\~n\~n\=a} (recognition), \pali{pa\d ti\~n\~n\=a} (vow), or \pali{m\=ana} (pride/conceit). To see it clearer, we have to know its analytic sentence, for example:

\begin{quote}
\pali{sama\d no (aha\d m homi) iti sa\~n\~n\=a sama\d nasa\~n\~n\=a (hoti).}
``Recognition that I am an ascetic is ascetic-recognition.''
\end{quote}

This analytic sentence is for \pali{sama\d nasa\~n\~n\=a}. I insert \pali{aha\d m homi} here to make it clearer. It is normally not present and can be replaced with other pronoun. For \pali{dhammabuddhi}, it can be ``\pali{dhammo (aya\d m hoti) iti buddhi}'' (knowledge that this is the Dhamma). So, \pali{iti} plays a significant role here. It marks the target of the mental state, and it is an intrinsic component of this compound. To say whether a compound is of this type or not, you have to form an analytic sentence as such, and see how agreeable with \pali{iti} it makes. Can \pali{sama\d nasa\~n\~n\=a} mean ``recognition of an ascetic''? Of course, it can. In that case, you use another structure of analytic sentence.

\paragraph*{(6) \pali{Avadh\=ara\d napubbapada}}\label{par:samasa-avadh} (\pali{avadh\=ara\d na} as the first part)\par
Examples: \pali{buddhavaro}\footnote{\pali{buddho eva varo buddhavaro.}} (only the Buddha the excellence), \pali{pa\~n\~n\=apajjoto} (only wisdom the brightness), \pali{sadh\=adhana\d m}\footnote{\pali{dhana\d m viy\=ati dhana\d m, saddh\=a eva dhana\d m sadh\=adhana\d m.}} (only faith like wealth),  \pali{s\=iladhana\d m} (only morality like wealth).

This compound looks like simile but it does more than that. Literally, \pali{avadh\=ara\d na} means `emphasis' or `selection.' This compound singles out an object as the only one of its class, hence preventing other object to have the equal quality. So, \pali{eva} (only, this very thing) is the crucial part of the analytic sentence.

\paragraph*{(7) \pali{Nanip\=atapubbapada}} (particle \pali{na} as the first part)\par
Examples: \pali{abr\=ahma\d no}\footnote{\pali{na br\=ahma\d no abr\=ahma\d no.}} (a non-brahman), \pali{amanusso} (a non-human), \pali{amitto} (a non-friend), \pali{akusal\=a dhamm\=a} (unskillful nature), \pali{anasso} (a non-hourse).

As you have seen, \pali{na} is changed to \pali{a} when composed.\footnote{Kacc\,333, R\=upa\,344, Sadd\,717, Mogg\,3.74} But when the noun begins with a vowel, it becomes \pali{an}, e.g.\ \pali{anasso = na + asso}.\footnote{Kacc\,334, R\=upa\,345, Sadd\,718, Mogg\,3.75} It seems to me that grammarians are in two minds concerning this negation. On one hand, they present particle \pali{a} (see page \pageref{nip:a}) to do this very job. On the other hand, they explain that it is in fact \pali{na} (see page \pageref{nip:na}) that changes itself to \pali{a}.

\paragraph*{(8) \pali{Kupubbapada}} (\pali{ku} as the first part)\par
Examples: \pali{kudi\d t\d thi}\footnote{\pali{kucchit\=a di\d t\d thi kudi\d t\d thi.}} (contemptible view), \pali{k\=apuriso} (an evil person), \pali{kadanna\d m} (spoiled rice, bad food), \pali{k\=alava\d na\d m}\footnote{\pali{appaka\d m lava\d na\d m k\=alava\d na\d m.}} (little salt).

There are rules explained by textbooks of this.\footnote{Kacc\,335--6, R\=upa\,346--7, Sadd\,719--21, Mogg\,3.107--8} They explain nothing but just give us a variation of forms and meaning.

\paragraph*{(9) \pali{P\=adipubbapada}} (\pali{upasagga} suchlike \pali{pa} as the first part)\par
Examples: \pali{p\=avacana\d m}\footnote{\pali{padh\=ana\d m vacana\d m p\=avacana\d m.}} (main term), \pali{sam\=adh\=ana\d m}\footnote{\pali{sama\d m samm\=a v\=a \=adh\=ana\d m sam\=adh\=ana\d m.}} (placing oneself evenly or well), \pali{vimati}\footnote{\pali{vividh\=a mati vimati.}} (various ideas), \pali{vikappo}\footnote{\pali{vividho visi\d t\d tho v\=a kappo vikappo.}} (various or extraordinary thought), \pali{abhidhammo}\footnote{\pali{atireko adhiko v\=a dhammo abhidhammo.}} (many or superior Dhamma).

The last two types are called \pali{niccasam\=asa} (permanent compound). I think this means they are not composed \textit{ad hoc} or on the fly. They were introduced to the word pool long time ago, and we use them with the meaning provided. You cannot guess what they are intended to mean in the first place. You have to follow the existing manuals. It is better to treat these as individual words by their own, but they are compounds anyway. You may compose your own words likewise, of course, but do not think others will understand your thought. Do not forget to provide your analytic sentences, otherwise you will cause a lot of trouble to the coming generations, as we have undergone nowadays due to the canon.

\section*{3. \pali{Digusam\=asa}}\label{sec:digu}

This compound in fact is a subtype of \pali{Kammadh\=araya}. When the first part is a modifier and it is a number, the compound is called \pali{Digu} (two cows).\footnote{Kacc\,325, R\=upa\,348, Sadd\,703, Mogg\,3.21} Most of these compounds are singular neuter.\footnote{Kacc\,321, R\=upa\,349, Sadd\,699} However, Aggava\d msa tells us that there are also those that are not neuter. So, he gives us two subtypes of this compound.\footnote{\pali{sam\=ah\=ara} and \pali{asam\=ah\=ara}} First, singular neuter \pali{Digu} is exemplified by \pali{catusacca\d m}\footnote{\pali{catt\=ari sacc\=ani sam\=aha\d t\=ani catusacca\d m.}} (the four truths), \pali{dvipada\d m} (a biped), \pali{timala\d m} (three stains), \pali{tida\d n\d da\d m} (three sticks), \pali{tiphala\d m} (three fruits), \pali{catuddisa\d m} (four directions), \pali{pa\~ncindriya\d m} (five faculties), \pali{pa\~ncagava\d m} (five cows).

Second, the rest of those are, for example, \pali{ekapuggalo} (one person), \pali{tibhav\=a} (three spheres of existence), \pali{catuddis\=a} (four directions), \pali{dasasahassacakkav\=a\d l\=ani} (10,000 solar systems).

\section*{4. \pali{Tappurisasam\=asa}}\label{sec:tappu}

As we have seen previously in \pali{Kammadh\=araya} and \pali{Digu}, both constituent parts of compound have the same case. In this type of compound, the first part, as a modifier, has a different case apart from the main/second part.\footnote{Kacc\,327, R\=upa\,351, Sadd\,704, Mogg\,3.10} That means we have six subtypes here. Ending of cases of the modifier part is only seen in the analytic sentence of the term. It is implied in the compound. In some cases you can guess from the compounds with ease, but some are more difficult.

\paragraph*{(1) \pali{Dutiy\=atappurisa}} (accusative modifier)\par
Example: \pali{bh\=umigato}\footnote{\pali{bh\=umi\d m gato bh\=umigato.}} (one who went to the ground/earth), \pali{ara\~n\~nagato} (one who went to the woods), \pali{sot\=apanno}\footnote{\pali{sota\d m \=apanno sot\=apanno.}} (one who entered the steam), \pali{maggappa\d tipanno} (one who followed the path), \pali{sabbarattisobha\d no} (one who is beautiful throughout the night), \pali{kammak\=aro}\footnote{\pali{kamma\d m karot\=iti kammak\=aro.}} (one who works, a worker).

\paragraph*{(2) \pali{Tatiy\=atappurisa}} (instrumental modifier)\par
Examples: \pali{issarakata\d m}\footnote{\pali{issarena kata\d m (kamma\d m) issarakata\d m.}} ([an action] done by the lord), \pali{salla\-viddho}\footnote{\pali{sallena viddho (puriso) sallaviddho.}} ([a person] pierced by an arrow), \pali{kh\=irodano}\footnote{\pali{kh\=irena sa\d msa\d t\d tho odano kh\=irodano.}} (rice mixed with milk), \pali{assaratho}\footnote{\pali{assena yutto ratho assaratho.}} (a carriage yoked with a horse).

\paragraph*{(3) \pali{Catutth\=itappurisa}} (dative modifier)\par
Examples: \pali{ka\d thinadussa\d m}\footnote{\pali{ka\d thinassa dussa\d m ka\d thinadussa\d m.}} (cloth for Kathina), \pali{\=agantukabhatta\d m}\footnote{\pali{\=agantukassa bhatta\d m \=agantukabhatta\d m}} (food for guest).

\paragraph*{(4) \pali{Pa\~ncam\=itappurisa}} (ablative modifier)\par
Examples: \pali{methun\=apeto}\footnote{\pali{methun\=a apeto methun\=apeto.}} (one who went away from sexual intercourse), \pali{corabhaya\d m}\footnote{\pali{cor\=a uppanno bhaya\d m corabhaya\d m.}} (danger from theft).

\paragraph*{(5) \pali{Cha\d t\d th\=itappurisa}} (genitive modifier)\par
Example: \pali{r\=ajaputto}\footnote{\pali{r\~n\~no putto r\=ajaputto.}} (a king's son), \pali{dha\~n\~nar\=asi}\footnote{\pali{dha\~n\~n\=ana\d m r\=asi dha\~n\~nar\=asi.}} (a heap of grains), \pali{k\=ayalahut\=a} (lightness of the body).

\paragraph*{(6) \pali{Sattam\=itappurisa}} (locative modifier)\par
Example: \pali{r\=upasa\~n\~n\=a}\footnote{\pali{r\=upe sa\~n\~n\=a r\=upasa\~n\~n\=a.}} (recognition in/of form), \pali{sa\d ms\=ara\-dukkha\d m} (suffering in circulation of rebirth), \pali{vanapuppha\d m} (a flower in a forest).

\bigskip
You might think why nominative case is left out. It seems that the tradition has already thought of that. If we include nom.\ to this compound, both \pali{Kammadh\=araya} and \pali{Digu} can also be called \pali{Tappurisa}.\footnote{Kacc\,326, R\=upa\,341, Sadd\,707} In some case, the two parts of compound switch their role, so the first becomes the main part. This is also called \pali{Tappurisa}\footnote{Sadd\,706}, for example, \pali{pubbak\=ayo}\footnote{\pali{pubba\d m k\=ayassa pubbak\=ayo.}} (the front part of the body), \pali{a\d d\d dhapipphal\=i} (a half of a long pepper).

In Padar\=upasiddhi, other two subtypes are added, namely \pali{Am\=a\-diparatappuriso} and \pali{Alopatappuriso}.\footnote{in R\=upa\,351} I find the former incomprehensible, so I skip it, perhaps like Aggava\d msa who also skips this. The later is more understandable. \pali{Alopatappuriso} is the compound which the ending of the first part is not removed. So, we can see the case ending, or a trace of it, in this compound, for example, \pali{pabha\.nkaro}\footnote{\pali{pabha\d m karot\=iti pabha\.nkaro.}} (one who do the light, the sun), \pali{attanopada\d m} (a term for one's self), \pali{manasik\=aro} (doing in mind, consideration). In Saddan\=iti, there is no separate type of this compound, but the essence is described in Sadd\,686.

\section*{5. \pali{Bahubb\=ihisam\=asa}}\label{sec:bahub}

As we go so far, we have seen that of components of compounds, one part is modifier an another is the main element. Differently, \pali{Bahubb\=ihi} has no main part of its own, so it need another term to be modified.\footnote{Kacc\,328, R\=upa\,352, Sadd\,708, Mogg\,3.17} That is to say, the whole part of this compound functions as an adjective. There are nine main types of \pali{Bahubb\=ihi} described in Sadd.

\paragraph*{(1) \pali{Dvipadabahubb\=ihi}} This compound is formed by two terms. There are six subtypes of this.

\subparagraph*{(i) \pali{Dutiy\=abahubb\=ihi}} An accusative external term is used as the main noun in the analytic sentence, for example, \pali{\=agatasama\d no sa\d mgh\=ar\=amo}\footnote{\pali{\=agat\=a sama\d n\=a ima\d m sa\d mgh\=ar\=ama\d m soya\d m \=agatasama\d no, sa\d mgh\=ar\=amo.} In this sentence, \pali{sa\d mgh\=ar\=ama\d m} (to monastery) is the external accusative noun. This can be rendered as ``Ascetics went to this monastery, that (monastery) is visited by ascetics.''} (a monastery visited/come by ascetics), \pali{\=agatasam\-a\d n\=a s\=avatthi} (S\=avatthi visited by ascetics), \pali{\=agatasama\d na\d m jetavana\d m} (Jetavana visited by ascetics).

\subparagraph*{(ii) \pali{Tatiy\=abahubb\=ihi}} The external main noun takes instrumental case, for example, \pali{jitindriyo sama\d no}\footnote{\pali{jit\=ani indriy\=ani yena sama\d nena soya\d m jitindriyo, sama\d no.}} (an ascetic whose faculties are won).

\subparagraph*{(iii) \pali{Catutth\=ibahubb\=ihi}} The main noun takes dative case, for example, \pali{dinnasu\.nko r\=aj\=a}\footnote{\pali{dinno su\.nko yassa ra\~n\~no soya\d m dinnasu\.nko, r\=aj\=a.}} (a king who received tax given).

\subparagraph*{(iv) \pali{Pa\~ncam\=ibahubb\=ihi}} This has ablative main noun, for example, \pali{niggatajano g\=amo}\footnote{\pali{niggat\=a jan\=a yasm\=a g\=am\=a soya\d m niggatajano, g\=amo.}} (a village from where people went away).

\subparagraph*{(v) \pali{Cha\d t\d th\=ibahubb\=ihi}} This has genitive noun, for example, \pali{chinnahattho puriso}\footnote{\pali{chinno hattho yassa purisassa soya\d m chinnahattho, puriso.}} (a man whose hand is cut).

\subparagraph*{(vi) \pali{Sattam\=ibahubb\=ihi}} This has locative noun, for example, \pali{sampannasasso janapado}\footnote{\pali{sampan\=ani sass\=ani yasmi\d m janapade soya\d m sampannasasso, janapado}} (a province in where crop flourished).

\paragraph*{(2) \pali{Bhinn\=adhikara\d nabahubb\=ihi}} This compound combines various cases together, for example, \pali{ekarattiv\=aso}\footnote{\pali{ekaratti\d m v\=aso ass\=ati ekarattiv\=aso.} To unpack this more, \pali{ass\=ati} is \pali{assa (purisassa)} + \pali{iti}. The whole means ``Living throughout one night of this (person) is thus called `living one night'.''} (living one night), \pali{chattap\=a\d ni}\footnote{\pali{chatta\d m p\=a\d nimhi ass\=ati chattap\=a\d ni.} This literally means having an umbrella in hand.} (holding an umbrella).

\paragraph*{(3) \pali{Tipadabahubb\=ihi}} This compound is formed by three components, for example, \pali{parakkam\=adhigatasampad\=a}\footnote{\pali{parakkamena adhigat\=a sampad\=a yehi te bhavanti parakkam\=adhigatasampad\=a, mah\=apuris\=a.}} ([a great person] who got results obtained by effort), \pali{o\d nitapattap\=a\d ni} (having hand out of the bowl).

\paragraph*{(4) \pali{Nanip\=atapubbapadabahubb\=ihi}} This compound has \pali{na} as the first part, for example, \pali{asamo}\footnote{\pali{natthi etassa samoti asamo, bhagav\=a.}} (unequalled), \pali{avu\d t\d thiko}\footnote{\pali{na vijjate vu\d t\d thi etth\=ati avu\d t\d thiko, janapado.}} (rainless).

\paragraph*{(5) \pali{Sahapubbapadabahubb\=ihi}}\label{par:sahapubba} This compound has \pali{saha} as the first part, for example, \pali{sahetuko} or \pali{sahetu}\footnote{\pali{saha hetun\=a yo vattati so sahetuko, suhetu v\=a.}} (accompanied with cause). This can be found in a well-known passage from chanting books: ``\pali{So ima\d m loka\d m sadevaka\d m sam\=araka\d m sabrahmaka\d m sassama\d mabr\=ahma\d ni\d m paja\d m sadevamanussa\d m saya\d m abhi\~n\~n\=a sacchikatv\=a pavedeti.}''\footnote{Buv1\,1. I.\,B.\,Horner renders this as ``Having realised with his own direct knowledge this world with its gods, its lords of death and its supreme beings, this population with its recluses and brahmins, its gods and humans, he makes it known to others'' \citep[pp.~84--5]{horner:discipline}.} In this passage, \pali{ima\d m loka\d m sadevaka\d m} means ``(to) this world together with gods (and so on).''

\paragraph*{(6) \pali{Upam\=anapubbapadabahubb\=ihi}} This compound has simile as the first part, for example, \pali{sa\.nkhapa\d n\d dara\d m}\footnote{\pali{sa\.nkho viya pa\d n\d dara\d m ya\d m vattha\d m ta\d m sa\.nkhapa\d n\d dara\d m, vattha\d m.}} ([cloth] white like a conch), \pali{suva\d n\d nava\d n\d no}\footnote{\pali{suva\d n\d nassa viya va\d n\d no yassa soya\d m suva\d n\d nava\d n\d no.}} (having bright complexion like gold).

\paragraph*{(7) \pali{Sa\.nkhyobhayapadabahubb\=ihi}} This compound has numbers as its components, for example, \pali{dvittipatt\=a}\footnote{\pali{dve v\=a tayo v\=a patt\=a dvittipatt\=a}} (2 or 3 bowls), \pali{chappa\~ncav\=ac\=a} (5--6 words). The external element added in the analytic sentence of this is not a noun but \pali{v\=a}, so it is also counted as \pali{Bahubb\=ihi}, maintained by Aggava\d msa.

\paragraph*{(8) \pali{Disantr\=a\d latthabahubb\=ihi}} This compound describes in-bet\-ween directions, for example, \pali{pubbadakkhi\d n\=a}\footnote{\pali{pubbass\=a ca dakkhi\d nass\=a ca dis\=aya yadantr\=a\d la\d m s\=aya\d m pubbadakkhi\d n\=a, vidis\=a.}} (south-east), \pali{pubbuttar\=a} (north-east), \pali{aparadakkhi\d n\=a} (south-west).

\paragraph*{(9) \pali{Byatih\=aralakkha\d nabahubb\=ihi}} This compound expresses a conflict or dispute\footnote{See also Mogg\,3.18.}, for example, \pali{kes\=akes\=i}\footnote{\pali{kesesu ca kesesu ca gahetv\=a ida\d m yutta\d m pavattat\=iti kes\=akes\=i.}} ([a fight] by grabbing each other's hair), \pali{da\d n\d d\=ada\d n\d d\=i} ([a fight] by hitting each other with a stick).

\section*{6. \pali{Dvandasam\=asa}}\label{sec:dvan}

Other kinds of compound as we have seen have at least one part that functions as modifier. This last type of compound has none. It is a combination of nouns with the same case.\footnote{Kacc\,329, R\=upa\,357, Sadd\,709, Mogg\,3.19} This compound has three subtypes.

\paragraph*{(1) Singular neuter} When nouns of parts of the body, music related, professions, military related, minor animals, opposite pairs, things able to fit together, etc., are combined in a compound, the result is singular neuter.\footnote{Kacc\,322, R\=upa\,359, Sadd\,700} 

Here are some examples: \pali{cakkhusota\d m}\footnote{\pali{cakkhu ca sota\~nca cakkhusota\d m}} (eyes and ears), \pali{chavima\d msalohita\d m} (skin, flesh and blood), \pali{sa\.nkhapa\d nava\d m} (conch and small drum), \pali{g\=itav\=adita\d m} (singing and playing instruments), \pali{yuggana\.ngala\d m} (yoke and plough), \pali{asicamma\d m} (sword and shield), \pali{hatthiassa\d m} (elephant and horse [in an army]), \pali{\d da\d msamakasa\d m} (gadfly amd mosquito), \pali{ahinakula\d m} (snake and mongoose), \pali{vi\d l\=aramusika\d m} (cat and mouse), \pali{samathavipassana\d m} (concentration and insight), \pali{vijj\=acara\d na\d m} \\(knowledge and conduct), \pali{d\=asid\=asa\d m} (male and female slaves), \pali{itthipuma\d m} (female and male), \pali{pattac\=ivara\d m} (bowl and robe), \pali{tikacatukka\d m} (threefold and fourfold [group]), \pali{d\=ighamajjhima\d m} (long and middle [something]), \pali{venarathak\=ara\d m} (weaver and mechanic).

\paragraph*{(2) Singular neuter or as the last part} This compound may be of sig.\ nt.\ or of the gender of the last part. This includes elements of tree, grass, quadruped, wealth, crop, grain, provincial area, etc.\footnote{Kacc\,323, R\=upa\,360, Sadd\,701} 

Here are some examples: \pali{assatthakapittha\d m/assatthakapitth\=a}\footnote{\pali{assattho ca kapittho ca assatthakapittha\d m assatthakapitth\=a v\=a.}} (bo tree and wood-apple tree), \pali{us\=irab\=ira\d na\d m/us\=irab\=ira\d n\=a} (Us\=ira and B\=ira\d na grass), \pali{aje\d laka\d m/aje\d lak\=a} (goat and ram), \pali{hira\~n\~nasuva\d n\d na\d m/hira\~n\~nasuva\d n\d n\=a} (silver and gold), \pali{s\=aliyava\d m/s\=ali\-yav\=a} (rice and barley), \pali{k\=asikosala\d m/k\=asikosal\=a} (K\=as\=i and Kosala), \pali{h\=inapa\d n\=ita\d m/h\=inapa\d n\=it\=a} (coarse and fine), \pali{ka\d nhasukka\d m/ka\d nhasukk\=a} (black and white).

\paragraph*{(3) Plural} This compound always ends up with a plural noun, for example, \pali{candimas\=uriy\=a} (the moon and the sun), \pali{sama\d nabr\=ahma\d n\=a} (ascetic and brahman), \pali{s\=ariputtamoggall\=an\=a}\footnote{In Sadd\,821--2, this bunch can be shortened to just \pali{s\=ariputt\=a}. In the same way, \pali{m\=at\=apitaro} can be just \pali{pitaro} (mother and father).} (Ven.\,S\=arip\-utta and Ven.\,Moggall\=ana), \pali{br\=ahma\d nagahapatik\=a} (brahman and householder).

\section*{Minor matters}

There are some things I want to highlight here for new students. You may have noticed that when the ending of the first part is a long vowel, it is normally shortened, for example, \pali{hatth\=i} + \pali{assa} = \pali{hatthiassa}. Moreover, when the final term is nt., like \pali{Abyay\=ibh\=ava}, the final vowel is always short\footnote{Kacc\,342, R\=upa\,337, Sadd\,734, Mogg\,3.23}, e.g.\ \pali{upa} + \pali{vadh\=u} = \pali{upavadhu}. Finally, it is not necessary to understand everything. Even grammarians cannot explain some point intelligibly. They just say ``Here they are, so take it.'' For example, do not ask further why \pali{upa} + \pali{go} becomes \pali{upagu}.\footnote{Mogg\,3.25, Sadd\,722} You just take it as such.

When you read P\=ali texts, you will find that compounds are used extensively. Sometimes they come out spontaneously. That is the real use of them. You can save your time and energy from composing very complex sentences by using compounds. For example, we can say ``Those who do not go to school have no friend'' succinctly as follows:

\palisample{ap\=a\d thas\=alagat\=ana\d m jan\=ana\d m mitt\=a natthi.}

Sometimes, particularly in postcanonical texts, compounds can be very complex, for example:

\begin{quote}
\pali{p\=inaga\d n\d davadanathan\=urujaghan\=a}\footnote{in Sadd\,708}
\end{quote}

This \pali{Bahubb\=ihi} compound can be broken down to \pali{p\=ina} (sexy)\footnote{In PTSD, \pali{p\=ina} means `fat, swollen.' In modern context, `sexy' is a close word.} + \pali{ga\d n\d da}\footnote{Abhidh\=a 262} (cheek) + \pali{vadana} (face) + \pali{thana} (breast) + \pali{\=uru} (thigh) + \pali{jaghana} (buttocks). So, the whole unit means ``having sexy cheek, face, breast, thigh, and buttocks.'' This adjective is normally used with f.\ nouns.

You can see that the challenging task when you encounter complex compounds is to break down the components. If you know many of basic words, it will be easy, or not too difficult. The knowledge of word joining (Sandhi) is also crucial here. That can help you determine which point should be cut. Here is the longest compound of all.

\begin{quote}
\pali{%
avippav\=asasammutisanthatasammutibhattuddesakasen\=asa-
nagg\=ah\=apakabha\d n\d d\=ag\=arikac\=ivarappa\d tigg\=ahakay\=agubh\=ajaka\-phalabh\=ajakakhajjabh\=ajakaappamattakavissajjakas\=a\d tiyagg\-\=ahapakapattagg\=ah\=apaka\=ar\=amikapesakas\=ama\d nerapesakasam\-mut\=iti%
}
\end{quote}

The instance comes from the subcommentary (\d T\=ik\=a) of Bhik\-khu P\=atimokkha (Dvem\=atik\=a, P\=acittiyaka\d n\d do). Will you take the challenge to decompose this? You may try it for fun, but in practice I suggest that you should never make thing like this. It is horrible.

\section*{Concluding remarks}

After all these types of compound are explained in the textbooks, then there come rules of how to connect parts together and what gender of the result should be. These are quite numerous, so I skip them. It is better not to read the instruction as rules, but an exploration of the possibility of compounds. Almost everything can happen, rendering there is virtually no rule at all. For example, when a f.\ noun is composed, the whole result can be m.\footnote{Kacc\,331, R\=upa\,353, Sadd\,715, Mogg\,3.67}, or nt.\footnote{Sadd\,714} There are also several minor rules. Some of them are very specific to particular words. I suggest that do not bother much with these rules unless you have to do a master thesis out of them.

There are some big things to keep in mind, though. First, compound is all about nouns and adjectives. We hardly see pronouns in composition here. And it never produces any verb, even though prefixes is used likewise. Verb formation undergoes another process. Second, the final gender of the compound depends on several factors. So, you should be alerted when you read texts, and just take it easy when you make your own words. Do it properly and reasonably. No one can say you are wrong if you have a reason for it, even if your use is not found in any traditional text.\footnote{You can even go against the texts if you have a better reason. That is my position.} And third, the more you see it the more you master it. It may be awkward at first when you encounter an unexpected, bizarre compound. Do not worry about this. Everyone has this moment. You just go on reading and be familiar with the archaic mind. More outlandish things are still waiting in the texts. No one understands everything clearly. The more you see the more you have a chance to make a probable guess.
