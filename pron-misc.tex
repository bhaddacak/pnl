\chapter{\headhl{All} I have are \headhl{four} books}\label{chap:pron-misc}

We have learned about important pronouns in several previous chapters. Now we will address the rest of them. Aggava\d msa gives us a list of 27 pronouns (\pali{sattav\=isa sabban\=am\=ani}). I put them verbatim here:

\begin{quote}\label{par:sabbanamani}
\pali{Sabban\=am\=ani n\=ama---sabba katara katama ubhaya itara a\~n\~na a\~n\~natara a\~n\~natama pubba para apara dakkhi\d na uttara adhara ya ta eta ima amu ki\d m eka ubha dvi ti catu tumha amha---iccet\=ani sattav\=isa.}\footnote{Sadd-Pad Ch.\,12; \citealp[p.~266]{smith:sadd1}. Called pronouns, \pali{sabba\ldots amha}, [\pali{iccet\=ani} (\pali{iti + et\=ani})] thus these (are) twenty-seven.}
\end{quote}

\phantomsection
\addcontentsline{toc}{section}{Miscellaneous Pronouns}
\section*{Miscellaneous Pronouns}

Among pronouns in the list we have already learned eight of them, namely \pali{ya, ta, eta, ima, amu, ki\d m, tumha,} and \pali{amha}. The rest of them are shown in Table \ref{tab:miscpron} with their corresponding declensional paradigm. To be complete, I also include an indefinite pronoun \pali{ki\d m+ci} (\pali{ka+ci}).

\bigskip
\begin{longtable}[c]{@{}%
	>{\itshape\raggedright\arraybackslash}p{0.18\linewidth}%
	>{\raggedright\arraybackslash}p{0.4\linewidth}%
	>{\raggedright\arraybackslash}p{0.3\linewidth}%
	@{}}
\caption{Miscellaneous pronouns}\label{tab:miscpron}\\
\toprule
\bfseries\upshape Pronoun& \bfseries\upshape Meaning& \bfseries Paradigm\\ \midrule%
\endfirsthead
\multicolumn{3}{c}{\tablename\ \thetable: Miscellaneous pronouns (contd\ldots)}\\
\toprule
\bfseries\upshape Pronoun& \bfseries\upshape Meaning& \bfseries Paradigm\\ \midrule%
\endhead
\bottomrule
\ltblcontinuedbreak{3}
\endfoot
\bottomrule
\endlastfoot
%
sabba & all, every, whole & \rdelim{\}}{7}{0.3\linewidth}[\pali{sabba}, page \pageref{decl:sabba}] \\
katara & \mbox{which one? (among a few)} & \\ 
katama & \mbox{which one? (among many)} & \\ 
ubhaya & both & \\
itara & the other & \\
a\~n\~na & other, another, else & \\
a\~n\~natara & one of a certain number & \\
a\~n\~natama & one out of many & \pali{sabba}, page \pageref{decl:sabba} \\
pubba & the former & \rdelim{\}}{6}{0.3\linewidth}[\pali{pubba}, page \pageref{decl:pubba}] \\
para & \mbox{other, another, the latter} & \\
apara & other, another & \\
dakkhi\d na & southern, right & \\
uttara & northern, the higher & \\
adhara & the lower & \\
eka & one & page \pageref{decl:one} \\
ubha & both & page \pageref{decl:two} \\
dvi & two & page \pageref{decl:two} \\
ti & three & page \pageref{decl:three} \\
catu & four & page \pageref{decl:four} \\
ki\d m+ci & some one, whoever & page \pageref{decl:koci} \\
\end{longtable}

When we talk about pronouns here, we include that they can function as pronominal adjectives at anytime. And in P\=ali, a noun modified by an adjective can be omitted if the context makes clear what it refers to. This means the difference between pronouns and adjectives is not a big deal in P\=ali. That is why the both are subsumed under \pali{n\=ama} (`name' = noun) category. For a clearer picture, let us see some examples.

I start with ``I give candies to children.''

\palisample{d\=arak\=ana\d m khajjak\=ani demi.\footnote{You can also use \pali{kha\d n\d da} (m.) for candy.}}

Then we pepper the sentence with \palibf{sabba} (all): ``I give all candies to all children.''

\palisample{sabbesa\d m d\=arak\=ana\d m sabb\=ani khajjak\=ani demi.}

The both \pali{sabba}s function as pronominal adjectives, because they are accompanied with a noun. If they act as pronouns (or, in other words, as adjectives with the noun left out), it will be:

\palisample{sabbesa\d m sabb\=ani demi.}

This sentence says nothing, if it stands alone. But if it is a part of a larger story that `children' and `candies' are mentioned before, it make some sense. That is the good part of gender differentiation. You can derive the references of pronouns by looking at their gender. Number is another helpful clue to determine what refers to what, but in this case number does not help.

Let us play around further. How about ``I give some candies to some children''? Don't hurry for this. Thinking it over, you will realize that `some' is a tricky word. It can mean (1) an unspecified amount or number, `not all' or `not many' or `a certain number of'; or (2) an unknown or unspecified person or thing, someone or something. In English we use the same word in both senses, but in P\=ali we have to be more cautious, because we have words for each meaning. In the first sense, we use \palibf{ekacca} (adj.)\footnote{also \pali{ekatiya} and \pali{ekacciya}}, whereas \pali{ki\d m+ci} is used in the second sense.

Therefore, if you want to say ``I give a certain number of candies to a certain number of children.'' It should go like this:

\palisample{ekacc\=ana\d m d\=arak\=ana\d m ekacc\=ani khajjak\=ani demi.}

On the other hand, if you want to say ``I give a certain kind of candies to certain children.'' It goes like this:

\palisample{kesa\~nci d\=arak\=ana\d m k\=anici khajjak\=ani demi.}

We can also use \palibf{eka} (pl.)\label{par:ekapl} in this sense, meaning ``(certain) ones of.'' So, we get this instead:

\palisample{ekesa\d m d\=arak\=ana\d m ek\=ani khajjak\=ani demi.}

Now let us say this: ``I give some candies to some child.'' The context makes clear that the first `some' tells us about number and the second tells us that the individual (suppose it is a boy) is unspecified by or unknown to the speaker. It goes simply as:

\palisample{kassaci d\=arakassa ekacc\=ani khajjak\=ani demi.}

For `to some child,' you can use \palibf{eka} (sg.) or \palibf{a\~n\~natara}\footnote{In this sense, \pali{a\~n\~natara} is often used as indefinite article `a,' see PTSD in the entry, \citealp[see also][p.~46]{cone:dict1}.} (one of a certain number) instead, like `to a child' or `to one child' in English. So, we can also say this:

\palisample{ekassa d\=arakassa ekacc\=ani khajjak\=ani demi.\sampleor a\~n\~natarassa d\=arakassa ekacc\=ani khajjak\=ani demi.}

Now I will make the sentence more vague by dropping `child' and use `someone' instead. Hence, ``I give some candies to someone.'' In this, \pali{eka} or \palibf{ki\d m+ci} can be use as pronoun.

\palisample{ekassa ekacc\=ani khajjak\=ani demi.\sampleor kassaci ekacc\=ani khajjak\=ani demi.}

If you say ``\pali{ekassa ekacc\=ani demi},'' you mean ``I give a certain number of a thing to someone.'' If you want to say ``I give something to someone,'' you should say this:

\palisample{ekassa ki\~nci demi.\sampleor{kassaci ki\~nci demi.}}

The two sentences above are not completely the same. There are a nuance, or a difference, when we say ``to someone'' and ``to anyone'' and ``to whoever.'' In P\=ali, \pali{ekassa} is close to ``to someone,'' whereas \pali{kassaci} is closer to ``to anyone'' and ``to whoever.'' Another term close to the former sense is \pali{a\~n\~natarassa} (see above), and \pali{a\~n\~natamassa} is close to the latter.

By its meaning, \pali{ki\d m+ci} is often used in questioning and negation. For example, to ask ``Do you have any candy?'' you can say as follows:

\palisample{atthi nu tava ki\~nci khajjaka\d m.}

Do not worry about \pali{nu} for now. We will learn more about quetioning in Chapter \ref{chap:ques}. And this is for ``I do not have any candy'':

\palisample{mama ki\~nci khajjaka\d m natthi.}

Do you remember \pali{ya-ta} pair in correlative sentences we have met in Chapter \ref{chap:yata}? This can be used with \pali{ki\d m+ci} to mean `whoever' or `whatever' or `whichever.' For example, you can say ``Whatever candies I have, I give them (all) to children'' as:

\palisample{mama y\=ani k\=anici khajjak\=ani santi, (sabb\=ani) t\=ani d\=arak\=ana\d m demi.\footnote{See the declension of \pali{ya ki\d m+ci} on page \pageref{decl:yokoci}.}}

Here is an example from the canon:

\palisample{ye keci kusal\=a dhamm\=a, sabbe te kusalam\=ul\=a.\footnote{Ym\,1:1}}

I render it by myself bluntly as ``Whatever (are) virtuous natures, they all (are) virtue-rooted.'' In fact, the text posts this as a question, but that is beside the point here. Another famous passage from the canon is this:

\palisample{ya\d m ki\~nci samudayadhamma\d m sabba\d m ta\d m nirodhadhamma\d m.\footnote{Mv\,1.16}}

This explains how the foremost disciple of the Buddha understood the Dhamma: ``Whatever (is of) rising nature, it all (is of) ceasing nature.'' You may come across translations of this passage many times. They possibly have various renditions that baffle you what the passage really means. Once you know it in P\=ali, you can say with confidence what it really means. This does not mean you will understand it clearly. You just know how clearly or vaguely or ambiguously the text is. Hence you know the meaning boundary of the text. If you rely heavily on others' translation, you are at risk of misunderstanding due to an extrapolation. So, it is always illuminating when you go back to the P\=ali version. You have to see it by yourself whether it is crystal clear or nebulously cryptic when certain translation is obtained. We are often overconfident in a selective translation from unclear sources. Now let us turn back to the lesson.

It is a little confusing when \pali{eka} is used because it carries multiple meaning. When using this to mean `single' or `alone' or `unaccompanied' (\pali{asah\=aya}), you can optionally use \palibf{ekaka} instead. It declines as adjectives, and can be sg.\ and pl. Here is a good example:

\begin{quote}
\pali{Catt\=aro ekak\=a siyu\d m}\footnote{Sadd-Pad Ch.\,12. There is an explanation in Niru\,635 showing that \pali{siy\=a} and \pali{siyu\d m} can function as a particle, meaning \pali{ekacco} (some) or \pali{kinnu} (how) or \pali{bhavanti} (be). In this instance, it stands for verb `to be.'}\\
``There are four single-itemed [\pali{dhamma}s].''\footnote{In the same manner, you can use \pali{duka, tika, catukka,} and so on to mean `twofold', `threefold', etc.}\\
\end{quote}

In the sentence above, it can be unclear if you use \pali{eka}. 

Also, \pali{ekaka} can mean `each.' For example, instead of saying ``\pali{ayampi gahapati ekova \=agato, ayampi ekova \=agato}'' (This householder came alone, yet this [also] came alone), you can say ``\pali{ime gahapatayo ekak\=a \=agat\=a}''\footnote{Sadd-Pad Ch.\,12} (These householders came alone). This means each of them came individually. 

For these lonely people, P\=ali has a word for them. It is \palibf{ek\=ak\=i}. This can be in three genders, but shortened \pali{ek\=aki} for nt. So, it makes sense to say ``\pali{ime gahapat\=i ek\=ak\=i honti}'' (These householders are lone comers).

Another term can be used to mean `each' is \palibf{ekeka}.\label{par:ekeka} Here are some examples:

\begin{quote}
\pali{Ekeka\d m me, bhonto, patta\d m dadantu}\footnote{Buv1\,345}\\
``Give me, birds, a feather [of yours] each.''\\[1.5mm]
\pali{ekeka\d m p\=uva\d m dento a\~n\~natariss\=a paribb\=ajik\=aya eka\d m ma\~n\~nam\=ano dve p\=uve ad\=asi.}\footnote{Buv2\,269. In this instance, \pali{dento} and \pali{ma\~n\~nam\=ano} are present participle. We will learn this verb form in Chapter \ref{chap:prp}.}\\
``[While] giving each cake, [\=Ananda] gave two cakes to a [female] wanderer, [by] thinking it is one.''\\
\end{quote}

Yet another way to say `each' or `individually' is to use repetition. We will learn this in Chapter \ref{chap:adv}.

Like \pali{eka} (one), other numbers (2--4), including `both' (\palibf{ubha, ubhaya}), are used in the same way as pronouns. We will learn P\=ali numerals in detail in Chapter \ref{chap:num}. Here we focus only on 1--4, for they are, unlike other numbers, pronouns which can decline into three genders (except 2 has only one form for all genders). Here is an example for saying ``I have two candies. I give (these) both to two (children).''

\palisample{mama dve khajjak\=ani santi. dvinna\d m (d\=arak\=ana\d m) (t\=ani) ubhe/ubhaye demi.}

To remind you, in the above sentence we use `two/both' in three cases: nom., acc., and dat. You should not be confused by now. If everything is clear, using other numbers should be easy as this. So, let us move to other pronouns.

As you may guess, \palibf{katara} and \palibf{katama} are used for questioning. The sign of \pali{ka} (\pali{ki\d m}) is obvious. These two mean ``which one?'' If it is drawn from a few things, \pali{katara} is normally used, otherwise \pali{katama} is used. But sometimes both are used interchangeably. If you ask me that ``You have two candies. Which one do you give to that child?'' You can say this:

\palisample{tava dve khajjak\=ani santi. tassa d\=arakassa katara\d m desi.}

If you precisely ask ``Which one do you give to which (child)?,'' you can say this:

\palisample{katarassa (d\=arakassa) katara\d m desi.}

Using \pali{katama} goes in the same way with a nuance. For example, when you ask me ``katamasmi\d m magge geha\d m gacchasi?'' You mean ``in which path'' (among many) I go home, or you mean generally ``how do I go home?.'' Instead, if you ask me ``katarasmi\d m magge geha\d m gacchasi?,' you ask me when we meet a fork on the path and you wonder which way leads to my home.

These two question words can also be used simply to ask for `what?', for example, ``\pali{samuddo katamo aya\d m}''\footnote{Ja\,11:108; In Sadd-Pad Ch.\,12 \pali{kataro} is used.} (What is this ocean?); or to ask for numbers like \pali{kati} (see Chapter \ref{chap:num}), for example, ``\pali{Katame dhamm\=a kusal\=a?}''\footnote{Dhs\,3:1} (How many virtuous natures are there?), or ``\pali{Katamo tasmi\d m samaye phasso hoti?}''\footnote{Dhs\,3:2} (How many/What [kinds of] contact [are] in that time?).

Let us move on by saying ``I have two candies. I give one to you. I give the other to a child.'' We can use \palibf{itara} or \palibf{a\~n\~na} (or \pali{para}, see below) in the last sentence.

\palisample{mama dve khajjak\=ani santi. tava eka\d m demi. d\=arakassa itara\d m/a\~n\~na\d m demi.}

Noted by Aggava\d msa, \pali{itara, a\~n\~na, a\~n\~natara,} and \pali{a\~n\~natama} have peculiar forms as found in the canon: ``\pali{a\~n\~nataro bhikkhu \textbf{a\~n\~natariss\=a} itthiy\=a pa\d tibaddhacitto hoti}''\footnote{Buv1\,73} (a monk is bound in love with a woman). Upon this instance, Aggava\d msa suggests that these following forms should be added to the declension of these terms, only for f.\ sg.\footnote{Sadd-Pad Ch.\,12}

\begin{quote}
{[ins., dat., abl., gen.]} \\
\pali{itariss\=a, itar\=aya} \\
\pali{a\~n\~niss\=a, a\~n\~n\=aya} \\
\pali{a\~n\~natariss\=a, a\~n\~natar\=aya} \\
\pali{a\~n\~natamiss\=a, a\~n\~natam\=aya} \\
{[loc.]} \\
\pali{itariss\=a, itarissa\d m, itar\=aya, itar\=aya\d m} \\
\pali{a\~n\~niss\=a, a\~n\~nissa\d m, a\~n\~n\=aya, a\~n\~n\=aya\d m} \\
\pali{a\~n\~natariss\=a, a\~n\~natarissa\d m, a\~n\~natar\=aya, a\~n\~natar\=aya\d m} \\
\pali{a\~n\~natamiss\=a, a\~n\~natamissa\d m, a\~n\~natam\=aya, a\~n\~natam\=aya\d m}
\end{quote}

%%
We have talked about \palibf{a\~n\~natara} and \palibf{a\~n\~natama} briefly above in one sense of the terms. Here we will look into the main use of these. You may guess that these two terms have something to do with \pali{a\~n\~na}. They are \pali{a\~n\~na} in comparative and superlative degree respectively (see Chapter \ref{chap:adjcomp}).

Literally, \pali{a\~n\~natara} means ``further other,'' whereas \pali{a\~n\~natama} means ``the furthest other'' which means like ``yet further other.'' Let us see an example. When I want to say ``I have candies. I give one to a child. I give other (one) to other (child). I give further other (one) to further other (child). I give yet further other (one) to yet further other (child).'' I go like this:

\palisample{mama khajjak\=ani santi. (ekassa) d\=arakassa eka\d m demi. a\~n\~nassa a\~n\~na\d m demi. a\~n\~natarassa a\~n\~natara\d m demi. a\~n\~natamassa a\~n\~natama\d m demi.}

In a similar sense, \palibf{para} and \palibf{apara} can be used instead of \pali{a\~n\~na} and \pali{a\~n\~natara} respectively. So, you can say ``I have three candies. I give one to a child. I give other one to other child. I give yet other one to yet other child'' as follows:

\palisample{mama t\=i\d ni khajjak\=ani santi. (ekassa) d\=arakassa eka\d m demi. parassa para\d m demi. aparassa apara\d m demi.}

As you may see, P\=ali language has an elegant way to say things that look ugly in English.

When \pali{para} appears with \palibf{pubba}, it can mean `latter' whereas \pali{pubba} means `former.' Consider this example: ``I have candies. I give them to two childs. One is fat, the other is thin. I give one (candy) to the former. The latter I give two.'' Here we go:

\palisample{mama khajjak\=ani santi. dvinna\d m d\=arak\=ana\d m t\=ani demi. eko th\=ulo, a\~n\~no kiso. pubbassa eka\d m (khajjaka\d m) demi. parassa dve demi.}

We have the last three pronouns in the list provided by Aggava\d msa that are not yet mentioned: \palibf{dakkhi\d na}, \palibf{uttara}, and \palibf{adhara}. These three are about location.\footnote{In Sadd-Pad Ch.\,12, Aggava\d msa explains that when \pali{pubba, para, apara, dakkhi\d na,} and \pali{uttara} are used as m.\ they refer to time and location, when used as f.\ they refer to direction, and when used as nt.\ they refer to location (\pali{Tath\=a hi pubba par\=a para dakkhi\d nuttarasadd\=a pulli\.ngatte yath\=araha\d m k\=alades\=adivacan\=a \ldots}). This means, I think, when we use such terms as a noun, e.g.\ \pali{pubb\=a} (the east), \pali{par\=a} (the west), \pali{dakkhi\d n\=a} (the south), and \pali{uttar\=a} (the north).} There are two opposite pairs here: \pali{dakkhi\d na--uttara} is southern-northern relation; \pali{uttara--adhara} is upper-lower relation. When you say ``I go to the north (of the city). You go to the south,'' you put it this way:

\palisample{(nagarassa) uttara\d m gacch\=ami. dakkhi\d na\d m gacchasi.}

When you want to say ``The head is the upper part (of the body). The feet is the lower,'' you use another pair:

\palisample{s\=isa\d m (k\=ayassa) uttara\d m (a\.nga\d m) (hoti). p\=ad\=a adhar\=a (honti).}

How about left-right relation? Well, as you may realize that pronouns and adjectives in P\=ali are more or less the same kind of words, under the same rubric \pali{n\=ama}, hence, to make an exhaustive list of pronouns is impossible, for it will include all adjectives as well. We follow Aggava\d msa's list because it is a good point to start.

To the point of left-right relation, in P\=ali there is \pali{v\=ama} meaning `left' in contrast with \pali{dakkhi\d na} `right.' Now you can tell a direction in a simple way. For example, let us try this: ``You go to the south of the town. At the crossroad, you go to the right, go to the left, go to the right (again). At the end (it) is a hospital.'' Here we go:

\palisample{nagarassa dakkhina\d m gacchasi. maggasandhiya\d m dakkhi\d ne gacchasi, v\=ame gacchasi, (puna) dakkhi\d ne gacchasi. os\=ane (s\=a) \=arogyas\=al\=a hoti.}

Since `southern' and `right' use the same word, we have to be clear. I use acc.\ in the former sense to denote a crude direction. In the latter sense, I use loc.\ instead to stress the proximity. So, saying ``go into the right'' makes a clearer picture than just ``go to the right.'' However, in Chapter \ref{chap:ind-to} we will learn that \pali{dakkhi\d nato} and \pali{v\=amato} are more suitable in such a situation.

Now it is the time to tackle our heading sentence, ``All I have are four books.'' Here is its P\=ali:

\palisample{mama sabb\=ani catt\=ari potthak\=ani santi.\footnote{A more stylistic rendition of this is in \pali{ya-ta} structure: \pali{ya\d m mama sabba\d m atthi, ta\d m cattāri potthakāni honti.} This is suggested by Antonio Costanzo. However, I see the sentence a little advanced. To understand this you have to read PTR, Chapter\,21 first.}\sampleor[or, m.]mama sabbe catt\=aro potthak\=a santi.}

If we add the sentence to ``All I have are four books. I keep three, and I give you the others,'' we get this:

\palisample{\ldots, t\=i\d ni dh\=aremi, tuyha\d m a\~n\~n\=ani demi.\sampleor[or, m.]\ldots, tayo dh\=aremi, tuyha\d m a\~n\~ne demi.}

And let us try this sentence, ``Of my three, one is lost, other two is found.''

\palisample{mama ti\d n\d na\d m, eka\d m nassati, a\~n\~n\=ani dve vijjanti.\sampleor[or, m.]mama ti\d n\d na\d m, eko nassati, a\~n\~ne dve vijjanti.} 

You can use loc.\ instead of gen.\ like ``\pali{mama t\=isu} \ldots'' in the sense of ``in my three'' or ``among my three.'' If you cannot recall this usage, please review Chapter \ref{chap:gen} and \ref{chap:loc}. Like verb to-be, \pali{vijjati}\footnote{Sadd-Dh\=a\,905, \pali{vida satt\=aya\d m}} means `to exist,' but it is more appropriate to be used in the sense of ``to be found'' or ``to be present.''

Aggava\d msa also reminds us that some pronouns look like noun\footnote{Sadd-Pad Ch.\,12, from \pali{A\~n\~nasaddo pubbasaddo, dakkhi\d no cuttaro paro} onwards}, for example, \pali{a\~n\~na} can be a noun in three genders, i.e.\ (nom.\ sg.) \pali{a\~n\~no} (m.), \pali{a\~n\~n\=a} (f.), and \pali{a\~n\~na\d m} (nt.), which mean one who is ignorant. These three decline as normal noun according to their gender. There are only two points to tell whether it is used as a noun: in nom.\ pl.\ and dat./gen.\ pl. For example, ``\pali{a\~n\~ne jan\=a}'' means ``other people,'' whereas ``\pali{a\~n\~n\=a jan\=a}'' means ``ignorant people''; and ``\pali{a\~n\~nesa\d m jan\=ana\d m}'' means ``for/of other people,'' whereas ``\pali{a\~n\~n\=ana\d m jan\=ana\d m}'' means ``for/of ignorant people.'' These two points mark a line between pronouns and other nominal forms including adjectives. The best clue to tell that whether a term is pronoun or not is dat./gen.\ pl.\ \pali{-sa\d m} or \pali{-s\=ana\d m} form, because nom.\ pl.\ is hard to tell sometimes.

In the same manner, \pali{pubbo} (m.), \pali{pubb\=a} (f.), and \pali{pubba\d m} (nt.) can mean `chief' or `main' (used as an adjective). As mentioned earlier, \pali{pubb\=a} (f.) also means `the east' (\pali{pubbadis\=a}). And \pali{pubbo} (m.) means `pus.' Also in the same vein, \pali{uttara} and \pali{para} can mean `excellent (one)'; \pali{dakkhi\d na} can mean `skilled or well-trained (one).' As mentioned earlier, \pali{par\=a} (f.) means `the west' (\pali{paradis\=a}), \pali{uttar\=a} (f.) means `the north' (\pali{uttaradis\=a}), and \pali{dakkhi\d n\=a} (f.) means `the south' (\pali{dhakki\d nadis\=a}). All these remind us to polysemous nature of words. So, we should handle them with great care.

\section*{Exercise \ref{chap:pron-misc}}
Say these in P\=ali. Do not go too literal. Consult a dictionary, if necessary. And keep it simple.
\begin{compactenum}
\item Do you surely know the way to the theater?
\item Yes, sort of. I have been there one time. What does the GPS say now?
\item The GPS says we have to turn right at the crossroad ahead.
\item I think it tells a wrong way. That street heads to the southern suburb. We have to go downtown, haven't we?
\item That's right. But from that there is another turn leading us to the downtown. We should follow the GPS, because computer is never wrong.
\item Okay, that's all we have. We have two ways ahead. Which way should we go?
\item It says we should go left now.
\item There must be something wrong. That way leads to the north. It is the way to our college, I remember. What destination did you set in the GPS?
\item Let me see. Sorry! It leads us to the college indeed.
\item Computer is never wrong, but humans are.
\item Sorry!
\end{compactenum}
