\chapter{You \headhl{should} go to school}\label{chap:opt}

\phantomsection
\addcontentsline{toc}{section}{Optative Mood}
\section*{Optative Mood}

Optative mood, called \pali{Sattam\=i} (seventh) by the tradition, is very much like imperative, but sounds less pressing. In some context they are even used interchangeably. It is mainly used in giving permission, supposing, and instructing. The conjugation of the optative in shown in Table \ref{tab:conjopt}. Stem forms used in this conjugation are without ending vowel, for we already have \pali{e} in the formula.

\begin{table}[!hbt]
\centering
\caption{Endings of optative conjugation}
\label{tab:conjopt}
\bigskip
\begin{tabular}{l*{2}{>{\itshape}l}} \toprule
\bfseries Person & \bfseries\upshape Singular & \bfseries\upshape Plural \\ \midrule
3rd & eyya, e & eyyu\d m, u\d m \\
2nd & eyy\=asi, e & eyy\=atha \\
1st & eyy\=ami, eyya\d m, e & eyy\=ama, emu \\
\bottomrule
\end{tabular}
\end{table}

There are some variations from general formula of this conjugation. The singular forms of all persons can be shortened to just \pali{e} (Sadd\,1088, Mogg\,6.75, Niru\,581). From \pali{attanopada}, \pali{eyya\d m} is often used instead of \pali{parassapada}'s \pali{eyy\=ami}.\footnote{From \textsc{P\=ali\,Platform} 3, \pali{kareyya\d m} has 107 occurrences in the whole collection of CSTR comparing to 4 of \pali{kareyy\=ami}. For more detail about \pali{attanopada} (middle voice), see Chapter \ref{chap:pass}.} Sometimes, \pali{eyy\=ama} is changed to \pali{emu} (Sadd\,1070, Mogg\,6.78, Niru\,583), e.g.\ \pali{vih\=aremu, j\=anemu}. In certain roots, it becomes \pali{omu} (Sadd\,1071), e.g.\ \pali{tanomu}. Mogg adds that \pali{eyy\=ama} sometimes becomes \pali{eyy\=amu}, e.g.\ \pali{bhaveyy\=amu}. And \pali{eyyu\d m} sometimes is contracted to \pali{u\d m} (Mogg\,6.47, Niru\,582), e.g.\ \pali{gacchu\d m}.

These sound a little overwhelming with detail, and some form is indistinct, such as \pali{e}. But it is good to know in advance that what is waiting for you in the texts. When you use these by yourselves, just use common forms. Table \ref{tab:conjsiya} shows irregular forms of verb \pali{atthi} (to be).\footnote{R\=upa\,500} Another verb that has odd optative forms is \pali{karoti} (to do). I show this irregularity in Table \ref{tab:conjkayira}.\footnote{R\=upa\,522} However, its normal forms, such as \pali{kareyya, kare,} etc., are still widely used.

\begin{table}[!hbt]
\centering
\caption{Optative conjugation of \pali{atthi}}
\label{tab:conjsiya}
\bigskip
\begin{tabular}{l*{2}{>{\itshape}l}} \toprule
\bfseries Person & \bfseries\upshape Singular & \bfseries\upshape Plural \\ \midrule
3rd & siy\=a, assa & siyu\d m, assu \\
2nd & assa & assatha \\
1st & assa\d m & ass\=ama \\
\bottomrule
\end{tabular}
\end{table}

\begin{table}[!hbt]
\centering
\caption{Alternative optative conjugation of \pali{karoti}}
\label{tab:conjkayira}
\bigskip
\begin{tabular}{l*{2}{>{\itshape}l}} \toprule
\bfseries Person & \bfseries\upshape Singular & \bfseries\upshape Plural \\ \midrule
3rd & kayir\=a & kayiru\d m \\
2nd & kayir\=asi & kayir\=atha \\
1st & kayir\=ami & kayir\=ama \\
\bottomrule
\end{tabular}
\end{table}

Like the imperative, the best explanation for the usages of this mood is from Aggava\d msa.

\pagebreak
\begin{quote}
Sadd\,881: \pali{Anumatiparikappavidhinimantan\=at\=isu sattam\=i.}\footnote{\citealp[pp.~815]{smith:sadd3}}\\
``[Used] in permission, supposition, advising, inviting, etc., [these are] \pali{sattam\=i}.''
\end{quote}

The first two uses are new, and the rest from \emph{advising} are the same as the imperative. So, for the optative we have eight senses in total (plus one from my addition, see below).

\paragraph*{(1) \pali{Anumatiya\d m} (in permission)} Suppose you are a teacher who are telling the children that they can go home. You say this, ``\pali{geha\d m gaccheyy\=atha, kum\=ar\=a}.''

\paragraph*{(2) \pali{Parikappe} (in supposition)} For example, ``\pali{geha\d m gacche/gaccheyya}'' means ``He/She might be going home'' or ``He/She goes home, I suppose.''

\paragraph*{(3) \pali{Vidhi\d mhi} (in advising)} Instead of using imperative, you can also say this, ``\pali{ara\~n\~na\d m gacche, tasmi\d m rama\d n\=iya\d m}'' ([You] should go to the forest, [it is] pleasurable in that). This sounds softer than imperative. This use corresponds to the heading task of this chapter. So, we can say ``You should go to school'' likewise as follows:

\palisample{(tva\d m) p\=a\d thas\=ala\d m gaccheyy\=asi/gacche.\sampleor (tumhe) p\=a\d thas\=ala\d m gaccheyy\=atha.}

\paragraph*{(4) \pali{Nimantane} (in inviting)} When you invite someone to have food at your house, you can say this, ``\pali{gehasmi\d m me bhatta\d m bhu\~njeyy\=asi}'' (Would you have food at my house?).

\paragraph*{(5) \pali{\=Amantane} (in calling)} To call someone, you can say this, ``\pali{idha nis\=ide}'' (Would you [come and] sit here?).

\paragraph*{(6) \pali{Ajjhi\d t\d the} (in requesting)} To ask someone direction, you say this, ``\pali{magga\d m \=aroceyy\=asi, bho}'' (Would you tell me the way, sir?).

\paragraph*{(7) \pali{Sampucchane} (in questioning or reflecting)} If you use optative in this sentence in stead of imperative, ``\pali{maccha\d m bhu\~njeyy\=ami ud\=ahu haritak\=ani}'' It means ``Should I eat fish or vegetables?,'' which sounds a bit softer.

\paragraph*{(8) \pali{Patthan\=aya\d m} (in aspiring or hoping)} For example, ``\pali{puna tva\d m na passeyya\d m}'' means ``[I hope] not to see you again.''

From my reading, let me add the last one which I feel it should be in the list.

\paragraph*{(9) \pali{Upal\=apane} (in persuading)} It can be used to convince someone to do something, for example, ``\pali{nagara\d m maya\d m gaccheyy\=ama}'' (Let's go to town).

\bigskip
As optative mood is used in supposition, it is normally accompanied with conditional particles, such as \pali{ce} or \pali{sace} (if). We will learn more on conditionals in Chapter \ref{chap:cond}.

Another use of the optative frequently found is in an idiom of ``it is (not) possible'' or ``it is (not) the case.'' There are two ways to do this: (1) with \pali{siy\=a}\footnote{\citealp[p.~387]{perniola:grammar}} and (2) with \pali{(na) \d th\=ana\d m vijjati}\footnote{\citealp[p.~63, 73, 88, 333]{warder:intro}}. Here are some examples:

\begin{quote}
\pali{siy\=a nu kho a\~n\~no maggo bodh\=aya}\footnote{M2\,335 (MN\,85). For \pali{nu}, a question marker, see Chapter \ref{chap:ques}.}\\
``Would there be another way for enlightenment?''\\[1.5mm]
\pali{siy\=a nu kho, bhante, bhagavat\=a a\~n\~nadeva ki\~nci sandh\=aya bh\=asita\d m, ta\~nca jano a\~n\~nath\=api pacc\=agaccheyya}\footnote{M2\,378 (MN\,90). Here, \pali{bh\=asita\d m} is past participle in passive structure (see Chapter \ref{chap:pp} and \ref{chap:pass}).}\\
``Is it the case, sir, that something having been said by the Buddha with one sense, but people would take it by another sense?''\\[1.5mm]
\pali{\d Th\=ana\d m kho paneta\d m, kassapa, vijjati, ya\d m vi\~n\~n\=u samanuyu\~njant\=a samanug\=ahant\=a samanubh\=asant\=a eva\d m vadeyyu\d m}\footnote{D1\,386 (DN\,8). For present participles, see Chapter \ref{chap:prp}.}\\
``It is possible, Kassapa, that wise persons, cross-questioning, asking, discussing, would say as follows \ldots''\\[1.5mm]
\pali{\d Th\=ana\d m kho paneta\d m, \=avuso, vijjati ya\d m idhekaccassa bhikkhuno eva\d m icch\=a uppajjeyya}\footnote{M1\,60 (MN\,5)}\\
``It is possible, Venerable, that the following desire would arise to some monk in this [religion], \ldots''\\[1.5mm]
\pali{Yo hi koci, bhikkhave, sama\d no v\=a br\=ahma\d no v\=a eva\d m vadeyya \ldots neta\d m \d th\=ana\d m vijjati.}\footnote{M3\,23 (MN\,102)}\\
``Whoever, monks, ascetic or brahman, would say thus \ldots, that is not possible.''\\[1.5mm]
\pali{A\d t\d th\=ana\d m kho eta\d m, tapassi, anavak\=aso ya\d m up\=ali gahapati sama\d nassa gotamassa s\=avakatta\d m upagaccheyya. \d Th\=ana\~nca kho eta\d m vijjati ya\d m sama\d no gotamo up\=alissa gahapatissa s\=avakatta\d m upagaccheyya.}\footnote{M2\,60 (MN\,56)}\\
``It is impossible, Tapass\=i, not a chance, that householder Up\=al\=i would be a listener of ascetic Gotama. It is possible that ascetic Gotama would be a listener of householder Up\=al\=i.''\\[1.5mm]
\end{quote}

You can see \pali{ya-ta} structure is also used in these instances. In negative sense, \pali{anavak\=aso} (not a chance) can be added to stress the unlikeliness. Sometimes present tense is used instead of optative mood. This may show a stronger confidence of the claim, not just a speculation, for example:

\begin{quote}
\pali{\d Th\=ana\d m kho paneta\d m, bhikkhave, vijjati, ya\d m a\~n\~nataro satto tamh\=a k\=ay\=a cavitv\=a itthatta\d m \=agacchati.}\footnote{D1\,44 (DN\,1). For the absolutive, verbs in \pali{tv\=a} form, see Chapter \ref{chap:pp}.}\\
``It is possible, monks, that other being, having moved from that body, comes into this present state.''\\[1.5mm]
\end{quote}

Apart from using with the optative, \pali{(na) \d th\=ana\d m vijjati} can be used with \pali{iti} clauses or direct speech (see Chapter \ref{chap:iti}), for example:

\begin{quote}
\pali{So vata, cunda, attan\=a palipapalipanno para\d m palipapalipanna\d m uddharissat\=i'ti neta\d m \d th\=ana\d m vijjati.}\footnote{M1\,87 (MN\,8)}\\
``It is not possible, Cunda, thus `that person who has sunk into a marsh will pull out one who [also] has sunk into a marsh'\,''\\[1.5mm]
\end{quote}

Optative mood can also be found in comparison, particularly in similes, often with \pali{seyyath\=api} and \pali{evameva}.\footnote{See also page \pageref{sec:nip-comparing}.} Here is an example:

\begin{quote}
\pali{Seyyath\=api, bhikkhave, puriso sakamh\=a g\=am\=a a\~n\~na\d m g\=ama\d m gaccheyya, tamh\=api g\=am\=a a\~n\~na\d m g\=ama\d m gaccheyya, so tamh\=a g\=am\=a saka\d myeva g\=ama\d m pacc\=agaccheyya. \ldots\ Evameva kho, bhikkhave, bhikkhu anekavihita\d m pubbeniv\=asa\d m anussarati.}\footnote{M1\,431 (MN\,39)}\\
``Just as a man, monks, might go from his own village to another village, [then he] might go from that village to yet another village, [then] he might return to his own village from that village. \ldots\ In the same way, monks, a monk remembers many lives in the past.''\\[1.5mm]
\end{quote}

\section*{Exercise \ref{chap:opt}}
Say these in P\=ali. Try to think in P\=ali. Do not take the English sentences seriously (literally).
\begin{compactenum}
\item Would you go to the party at Liza's house tonight?
\item What kind of party?
\item Birthday party, I suppose.
\item I should not go because I am not familiar with her.
\item To be familiar with her, you should meet her again and again. So, you should go with me.
\item Should I take a present with me?
\item That is a birthday party is all about, I think.
\end{compactenum}
