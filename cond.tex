\chapter{\headhl{If} you go to school, you will be wise}\label{chap:cond}

In this chapter we will learn to compose conditional sentences. Before we do it in P\=ali, let us review English grammar a little bit. Conditional sentences are about imagination or supposition, some are possible, some are not. We normally use `\emph{if}' as a conditional marker. The structure of \emph{if}-sentences basically goes in three ways: present form, past form, and perfect form. I summarize the structure in Table \ref{tab:if}, classifying by type.\footnote{according to \citealp[\S257]{eastwood:guide}} This does not mean that P\=ali conditional sentences will correspond to this structure. I see only some similarity. However, it is a good place to start with.

\begin{table}[!hbt]
\centering
\caption{Structure of English conditional sentences}
\label{tab:if}
\bigskip\footnotesize
\begin{tabular}{@{}>{\raggedright\arraybackslash}p{0.05\linewidth}%
	>{\raggedright\arraybackslash}p{0.6\linewidth}@{}} \toprule
Type & Conditional Sentence \\ \midrule
0 & \bfseries \fbox{\emph{If} + present},\ \fbox{present} \\[1mm]
1 & \bfseries \fbox{\emph{If} + present},\ \fbox{\emph{will} + infinitive} \\[1mm]
& $\bullet$ Uncertain situations \\
& $\bullet$ Possible conditions \\[2mm]
2 & \bfseries \fbox{\emph{If} + past},\ \fbox{\emph{would} + infinitive} \\[1mm]
& $\bullet$ Unreal situations \\[2mm]
3 & \bfseries \fbox{\emph{If} + past perfect},\ \fbox{\emph{would have} + past participle} \\[1mm]
& $\bullet$ Unreal past situations \\
\bottomrule
\end{tabular}
\end{table}

In Table \ref{tab:indif}, I list some of particles used to mark a condition or supposition. Some of them may have other meaning in other context. In this matter, all of them can be translated simply as `if,' or `if not' for the bottom part. Among all these, \pali{sace} seems to be the most common use and have a distinct function. 

\begin{table}[!hbt]
\centering
\caption{Some conditional particles}
\label{tab:indif}
\bigskip
\begin{tabular}{*{3}{>{\itshape}l}} \toprule
\multicolumn{3}{c}{\bfseries\upshape if} \\
ce & sace & yadi \\
atha & appeva & appeva n\=ama \\
\midrule
\multicolumn{3}{c}{\bfseries\upshape if not, unless} \\
noce & no ce & yadi na \\
\bottomrule
\end{tabular}
\end{table}

Uncertain situations are about present or future events. When we surmise about an uncertain event which we do not know exactly whether it happens or will happen or not, we normally use present or future tense. For example, you can say ``If it rains, she does/will not come'' as follows:

\palisample{sace vassati, s\=a na \=agacchati.\sampleor sace vassati, s\=a na \=agamissati.}

As we have seen in Chapter \ref{chap:opt}, optative mood is also common to use in this meaning. So, it is equivalent, or, perhaps, better, to say ``If it rains, she might not come.''

\palisample{sace vassati, s\=a na \=agaccheyya.}

In P\=ali, I think, it is not a big difference whether we use present or future tense or optative mood in the subordinate clause.\footnote{A.\,K.\,Warder explains that if it is a pure hypothesis, ``the verbs in both relative and main clauses will be in the optative'' \citep[p.~295; see also p.~333]{warder:intro}. See also \citealp[p.398]{perniola:grammar}.} Therefore, it is equivalent to say this as well:

\palisample{sace vassissati, s\=a na \=agaccheyya.\sampleor sace vasseyya, s\=a na \=agaccheyya.}

Possible condition is very close to uncertain situation, but it is not just a guess. It is an assertion of certain causality. When I say ``If it rains, I do/will not come,'' I do not make an assumption but assert some condition. You can replace `if' with `when' in this case. So, it is exactly the same to say ``When it rains, I do/will not come.'' In P\=ali, it goes likewise, and, I think, imperative mood can also be used here. So, we get this:

\palisample{sace vassati, aha\d m na \=agacch\=ami\footnote{For it takes the same form, this can also be interpreted as imperative mood. In this sense, I assert my hope or aspiration upon a condition (see Chapter \ref{chap:imp}).}.\sampleor sace vassati, aha\d m na \=agamiss\=ami. \sampleor[or, softer, ``I may not come'']sace vassati, aha\d m na \=agaccheyya\d m.}

For better understanding, let us see some examples from the canon.\footnote{These are suggested by Warder \citep[pp.~294--5]{warder:intro}. For some more, see that work.} Some of these may be difficult for you right now. Do not worry about that.

\begin{quote}
\pali{Sace te agaru bh\=asassu}\footnote{D2\,367 (DN\,21)}\\
``If [it is] not troublesome to you, say [it].''\\[1.5mm]
\pali{Sace tva\d m, \=ananda, tath\=agata\d m y\=aceyy\=asi, dveva te v\=ac\=a tath\=agato pa\d tikkhipeyya, atha tatiyaka\d m adhiv\=aseyya.}\footnote{D2\,181 (DN\,16)}\\
``If you, \=Ananda, asked the Buddha, he might refuse your second request, then [he] would accept your third try.''\\[1.5mm]
\pali{Sace ag\=ara\d m ajjh\=avasati, r\=aj\=a hoti}\footnote{D3\,199 (DN\,30)}\\
``If [he] lives in household life, [he will] become a king.''\\[1.5mm]
\pali{ito cepi so bhava\d m gotamo yojanasate viharati, alameva saddhena kulaputtena dassan\=aya upasa\.nkamitu\d m api pu\d tosena}\footnote{D3\,37 (DN\,24)}\\
``Even if Ven.\,Gotama lives 100 Yojanas from here, it is suitable to approach for seeing [him] by a faithful fellow, even with provision [for going].''\\[1.5mm]
\pali{Ta\d m ki\d m ma\~n\~nasi, mah\=ar\=aja, yadi eva\d m sante hoti v\=a sandi\d t\d thika\d m s\=ama\~n\~naphala\d m no v\=a}\footnote{D1\,185 (DN\,2)}\\
``What do you think, Your Majesty, whether, [if] being so, there is visible fruit of religious life or not?''\\[1.5mm]
\end{quote}

As we go so far, it is enough to finish our task in this chapter as posted in the title, ``If you go to school, you will be wise.'' This sentence is a possible condition, so we ge this:

\palisample{sace p\=a\d thas\=ala\d m gacchasi, pa\~n\~nav\=a bhavissasi.\sampleor[or, in plural]sace p\=a\d thas\=ala\d m gacchatha, pa\~n\~navanto bhavissatha.}

We can use the imperative or optative instead in the main clause. Perhaps, this is more common to use:

\palisample{sace p\=a\d thas\=ala\d m gacchasi, pa\~n\~nav\=a bhav\=ahi/bhava.\sampleor[or, imp. pl.]\ldots, pa\~n\~navanto bhavatha.\sampleor[or, opt. sg.]\ldots, pa\~n\~nav\=a bhaveyy\=asi/bhave.\sampleor[or, opt. pl.]\ldots, pa\~n\~navanto bhaveyy\=atha.}

What if we use these tenses and moods to talk about unreal situations? For example, I have a fantasy that ``\pali{aha\d m ce pakkhino bhav\=ami, tava geha\d m uppatiss\=ami}'' (If I am a bird, I will fly to your house), or in English, ``If I were a bird, I would fly to your house.'' I think it is valid to say so without using past structure. However, past tense can be used in conditional sentences, like English, to refer to conditions that happened in the past. To say whether it is a real event or not, I think, it is in the content itself. However, P\=ali has another structure to help us deal with unreal past situations. That is the topic of the following section. 

In sum for now, for type-0, 1, and 2 conditions, we can use present and future tenses, and imperative and optative moods. For type-3 condition, we use \emph{conditional mood}.

\phantomsection
\addcontentsline{toc}{section}{Conditional Mood}
\section*{Conditional Mood}

In P\=ali when we talk about events that do not really happen, we normally use \emph{conditional mood} (K\=al\=atipatti). It is somehow like future tense plus past tense, as you can see its endings in the Table \ref{tab:conjcond}.

\begin{table}[!hbt]
\centering
\caption{Endings of conditional mood conjugation}
\label{tab:conjcond}
\bigskip
\begin{tabular}{l*{2}{>{\itshape}l}} \toprule
\bfseries Person & \bfseries\upshape Singular & \bfseries\upshape Plural \\ \midrule
3rd & ss\=a & ssa\d msu \\
2nd & sse & ssatha \\
1st & ssa\d m & ss\=amh\=a \\
\bottomrule
\end{tabular}
\end{table}

Like past tense, prefix \pali{a} (augment) is commonly used in this mood. I show typical renditions of verb \pali{gacchati} in conditional mood in Table \ref{tab:condgacch} for you can get some picture.

\begin{table}[!hbt]
\centering
\caption{Conditional mood conjugation of \pali{gacchati}}
\label{tab:condgacch}
\bigskip
\begin{tabular}{l*{2}{>{\itshape}l}} \toprule
\bfseries Person & \bfseries\upshape Singular & \bfseries\upshape Plural \\ \midrule
3rd & agacchiss\=a & agacchissa\d msu \\
2nd & agacchisse & agacchissatha \\
1st & agacchissa\d m & agacchiss\=amh\=a \\
\midrule
3rd & agamiss\=a & agamissa\d msu \\
2nd & agamisse & agamissatha \\
1st & agamissa\d m & agamiss\=amh\=a \\
\bottomrule
\end{tabular}
\end{table}

Conditional mood can refer to past events\footnote{Kacc\,422, R\=upa\,475} that had never occurred, but being used as speculations. This is like type-3 condition in English. Here is an example from Kacc: ``\pali{So ce ta\d m y\=ana\d m alabhiss\=a, agacchiss\=a}'' (If he had got that vehicle, he would have gone). In reality, he does not go, because he did not get the vehicle.

It can also refer to future events\footnote{Sadd\,895} which sounds close to type-1 condition, and to some extent type-2 condition. Here is an example from the canon:

\begin{quote}
\pali{Cirampi bhakkho abhavissa, sace na vivademase;}\footnote{Ja\,7:34. According to Sadd\,1041 and Mogg\,6.33, long vowel endings may be shortened. So, we get \pali{abhavissa} rather than \pali{abhaviss\=a}. Unusual, maybe very old, \pali{vivademase} has only this one occurrence in the whole collection. It possibly takes imp.\ in 1st-person pl.\ attanopada.}\\
``[Our] food will last long, if [we] do not dispute.''
\end{quote}

I suppose that this can also be used with my fantasy as a bird. So, we can say ``\pali{aha\d m ce pakkhino abhavissa\d m, tava geha\d m uppatissa\d m}.'' Aggava\d msa's observation makes the function of this mood less distinct. I suggest that we should use P\=ali conditional mood only for type-3 conditions. But be aware when you read texts. You may encounter a future condition as Aggava\d msa reminds us.

As Vito Perniola points out, optative mood alone, or with conditional mood, can be used in type-3 condition.\footnote{\citealp[p.398]{perniola:grammar}} Here are some examples:

\begin{quote}
\pali{Sace hi so, bhikkhave, bhikkhu im\=ani catt\=ari ahir\=ajakul\=ani mettena cittena phareyya, na hi so, bhikkhave, bhikkhu ahin\=a da\d t\d tho k\=ala\.nkareyya.}\footnote{Cv\,5.251; A4\,67}\\
``Monks, if that monk had extended his loving kindness to these four families of serpent king, that bitten monk would not have died.''\\[1.5mm]
\pali{Sace tva\d m, \=ananda, tath\=agata\d m y\=aceyy\=asi, dveva te v\=ac\=a tath\=agato pa\d tikkhipeyya, atha tatiyaka\d m adhiv\=aseyya.}\footnote{D2\,181 (DN\,16)}\\
``If you, \=Ananda, had asked the Buddha, he might have refused your second request, then [he] would have accepted your third try.''\footnote{We have already met this instance above. I repeat it here with slightly different translation. Warder sees this as a pure hypothesis. But Perniola sees it as an unverified condition. They are different ways in seeing the same thing.}\\[1.5mm]
\pali{No ceta\d m, bhikkhave, b\=alo duccintitacint\=i ca abhavissa dubbh\=asitabh\=as\=i ca dukka\d takammak\=ar\=i ca kena na\d m pa\d n\d dit\=a j\=aneyyu\d m}\footnote{M3\,246 (MN\,129)}\\
``Monks, if a fool were not an evil-thinker, evil-speaker, and evil-doer, how would the wise know him thus \ldots?''\\[1.5mm]
\end{quote}

\section*{Exercise \ref{chap:cond}}
Say these in P\=ali (as much as you can, before you peek at the solution).
\begin{compactenum}
\item May we have a talk, sir, if you have time?
\item Yes, if it is not too long. I have a teaching in half an hour.
\item What's wrong with my article? Why did you give me a `D'?
\item If you had listened to me carefully in the class, you would have know that [I expected] `democracy' not `people's dead government.'
\item Isn't it `people's dead government'?
\item Not that. Why didn't you ask your friends?
\item I suppose we understand the same thing. Can I fix this, if you allow?
\item If you want, rewrite it again with `democracy' and give me by tomorrow.
\item Thank you, sir.
\end{compactenum}
