\chapter{\headhl{Go} to school, boys}\label{chap:imp}

\phantomsection
\addcontentsline{toc}{section}{Imperative Mood}
\section*{Imperative Mood}

Now we will talk about moods, starting with the \emph{imperative}, another easy verb form to deal with. Conjugation of the imperative is similar to the present tense, just change \pali{ti} to \pali{tu} and \pali{si} to \pali{hi}. I summarize the conjugation in Table \ref{tab:conjimp}. The main use of this mood is to order, implore, and wish. The tradition calls this mood \pali{Pa\~ncam\=i} (fifth). ``Of what?,'' you may ask. I have to admit that I do not clearly understand the explanation of this. It has something to do with certain order of time.\footnote{If you are curious, try reading verses in Sadd-Pad Ch.\,3 from ``\pali{Chadh\=a id\=ani k\=al\=ana\d m, sa\.ngaho n\=ama niyyate}'' onwards. Even I have a full translation of this, I grasp nothing. The order clearly comes from Sanskrit grammar. Looking at \citealp[p.~14]{collins:grammar}, you may get some idea.}

\begin{table}[!hbt]
\centering
\caption{Endings of imperative conjugation}
\label{tab:conjimp}
\bigskip
\begin{tabular}{l*{2}{>{\itshape}l}} \toprule
\bfseries Person & \bfseries\upshape Singular & \bfseries\upshape Plural \\ \midrule
3rd & tu & antu \\
2nd & hi, a & tha \\
1st & mi & ma \\
\bottomrule
\end{tabular}
\end{table}

There are additional rules concerning \pali{hi} ending. First, if the stems end with \pali{a}, it has to be lengthened\footnote{Kacc\,478, R\=upa\,438, Sadd\,959, Mogg\,6.57, Niru\,567.}, for example, \pali{bhav\=ahi, gacch\=ahi}. And second, the \pali{hi} itself can be omitted after \pali{a}-ending stems\footnote{Kacc\,479, R\=upa\,452, Sadd\,960, Mogg\,6.48, Niru\,576}, e.g.\ \pali{gacch\=ahi}$\rightarrow$\pali{gaccha}, \pali{gam\=ahi}$\rightarrow$\pali{gama}, but \pali{hohi, karohi, dehi, br\=uhi}. For the irregular \pali{atthi} (to be), I show its imperative forms in Table \ref{tab:conjatthu}.\footnote{R\=upa\,500}

\begin{table}[!hbt]
\centering
\caption{Imperative conjugation of \pali{atthi}}
\label{tab:conjatthu}
\bigskip
\begin{tabular}{l*{2}{>{\itshape}l}} \toprule
\bfseries Person & \bfseries\upshape Singular & \bfseries\upshape Plural \\ \midrule
3rd & atthu & santu \\
2nd & ahi & attha \\
1st & asmi & asma \\
\bottomrule
\end{tabular}
\end{table}

Following Aggava\d msa, imperative mood can be used in 11 senses:

\begin{quote}
Sadd\,880: \pali{\=A\d naty\=asi\d t\d thakkosasapathay\=acanavidhi-\\nimantan\=amantan\=ajjhi\d t\d thasampucchanapatthan\=asu \\pa\~ncam\=i.}\footnote{\citealp[pp.~813--4]{smith:sadd3}}\\
``[Used] in commanding, wishing, cursing, swearing, begging, advising, inviting, calling, requesting, questioning, [and] aspiring, [these are] \pali{pa\~ncam\=i}.''
\end{quote}

\paragraph*{(1) \pali{\=A\d nattiya\d m} (in commanding)} In English we do this simply by putting verbs at the beginning of the sentence, for example ``Go home.'' The subject `you' is left out, because commanding happens in conversation, so the interlocutor is implied. In P\=ali it goes like this, ``\pali{geha\d m gaccha/gacch\=ahi}.'' However, in P\=ali the subject can also be third person, for example ``\pali{geha\d m gacchatu}.'' In this case, the command is targeted at somebody mentioned. It somehow sounds like ``He/She is to go home'' or ``He/She shall go home'' or ``Let he/she go home.''

By this sense, we can accomplish our task in the title of this chapter, ``Go to school, boys'' as follows:

\palisample{p\=a\d thas\=ala\d m gacchatha, kum\=ar\=a.}

To stress the command, imperative verbs are often put at the beginning position.\footnote{\citealp[p.~35]{warder:intro}} Therefore, the sentence sounds more compelling, when it is said in this way:

\palisample{gacchatha, kum\=ar\=a, p\=a\d thas\=ala\d m.}

In this sentence the speaker talks to some kids. So, \pali{kum\=ar\=a} is used for addressing the interlocutor (see Chapter \ref{chap:vockim}). It is not the subject of the sentence which is the omitted `you' (pl.). What if `boys' is the subject? It looks unusual in English but comprehensible in P\=ali. In this sense, the command targets to the mentioned `boys.' Hence, we get this instead:

\palisample{gacchantu kum\=ar\=a p\=a\d thas\=ala\d m.}

This means ``Let boys go to school.''

\paragraph*{(2) \pali{\=Asi\d t\d the} (in wishing, for others)} Unlike English, in P\=ali commanding and wishing use the same structure. The difference can be discerned only by the context. So, the examples above can also mean ``I wish you, kids, to go to school'' and ``I wish kids to go to school'' respectively. The common use of this is for blessing, for example, ``\pali{arog\=a sukhit\=a hotha}'' (May you be healthy [and] happy), ``\pali{d\=igh\=ayuko hotu aya\d m kum\=aro}'' (Long live this boy).

\paragraph*{(3) \pali{Akkose} (in cursing)} Like wishing but in a bad way, you can curse others by using these verb forms. For example, you can say ``(I damn you to) burn in hell'' as ``\pali{narake daha/dah\=ahi},'' or ``(I damn they to) be penniless'' as ``\pali{da\d lidd\=a bhavantu}.''

\paragraph*{(4) \pali{Sapathe} (in swearing)} This sounds like bad wishing or curse, but not so seriously. It may come out of upset or annoyance, and sometimes in obscene language. Here is an example from the canon: ``\pali{Ekik\=a sayane setu, y\=a te ambe av\=ahari}''\footnote{Ja\,4:176. In a dictionary you can find \pali{avaharati} (steal), and \pali{avahari/av\=ahari} is its aor.} (Lie in bed alone, who stole those mangoes). The swearer might wish the stealer, a female, as shown by \pali{y\=a}, cannot find any husband.

\paragraph*{(5) \pali{Y\=acane} (in begging)} This is straightforward. For example, ``\pali{dhana\d m me dehi}'' means ``Give me wealth/money.'' The context can tell if this is a request or an order.

\paragraph*{(6) \pali{Vidhi\d mhi} (in advising)} You can use this in telling direction, for example, ``\pali{v\=ame gaccha, tato dakkhi\d ne gaccha}'' ([You] go left, then go right). It is also common in giving an instruction, for instance, ``\pali{ara\~n\~ne gaccha, tasmi\d m rama\d n\=iya\d m}'' ([You] go into the forest, [it is] pleasurable in that).

\paragraph*{(7) \pali{Nimantane} (in inviting)} When someone invite the Buddha to have a meal at his or her house, the asking goes like this: ``\pali{adhiv\=asetu me, bhante, bhagav\=a sv\=atan\=aya bhatta\d m}''\footnote{Mv\,6.280} ([Please] accept my food, sir, the Blessed one, for tomorrow). As you can see, sometimes 3rd person verb (\pali{-tu}) is used instead of 2nd person (\pali{-hi}). It sounds softer and more polite (see below).

\paragraph*{(8) \pali{\=Amantane} (in calling)} This is used when you beckon someone, for example, ``\pali{\=agaccha d\=araka}'' (Come here, boy). It can be in terms of inviting and addressing, for example, ``\pali{ettha nis\=idatha}'' (Please take a seat here).

\paragraph*{(9) \pali{Ajjhi\d t\d the} (in requesting)} In the scripture, when people request the Buddha to talk Dhamma, they say this: ``\pali{desetu, bhante, bhagav\=a dhamma\d m}'' ([Please] expound the Dhamma, sir, the Blessed one).

\paragraph*{(10) \pali{Sampucchane} (in questioning)} When a kid asks his or her parent that ``Do I have to go to school?,'' he or she can say this: ``\pali{gacch\=ami nu p\=a\d thas\=ala\d m}.'' Even this verb form looks the same as present tense, but the context tells us that some obligation is in concern. It is not simply the question of ``Do I go to school?'' You may use this for a reflection to make a decision, like ``\pali{maccha\d m bhu\~nj\=ami ud\=ahu haritak\=ani}'' (Shall I eat fish or vegetables?). For more detail about questioning, see Chapter \ref{chap:ques}.

\paragraph*{(11) \pali{Patthan\=aya\d m} (in aspiring)} The mood can also be used to make certain aspiration or hope for yourselves, for example, ``\pali{m\=agadhiko iva p\=aliy\=a bh\=as\=ami}'' (May I speak P\=ali like a Magadhian).

\bigskip
There is a custom concerning social hierarchy worth noting here. When subordinates talk to superiors using imperative mood, to make the request sound polite we normally use verbs in plural form. So, when you invite a teacher to your house, it is customary to say this even only one person is listening:

\palisample{geha\d m me \=agacchatha, \=acariya}

Another way to make the request courteous and polite, verbs in 3rd person are used instead.\footnote{\citealp[p.~350]{perniola:grammar}} Here are some examples from the canon:

\begin{quote}
\pali{etu kho, bhante, bhagav\=a}\footnote{D3\,55 (DN\,25)}\\
``[Please] come, sir, the Blessed One.''\\[1.5mm]
\pali{appasadd\=a bhonto hontu}\footnote{D3\,51 (DN\,25)}\\
``[Please] be quiet, Venerables.''\\[1.5mm]
\pali{putto te, deva, j\=ato, ta\d m devo passatu}\footnote{D2\,33 (DN\,14)}\\
``Your son has been born, Your Majesty, may [you] the king see him.''\\[1.5mm]
\end{quote}

Negation of command is prohibition. In a simple way, we can negate imp.\ with \pali{na}, such as ``\pali{na gacchatu}'' (Don't let him/her/it go). However, this is not a good solution, because the imperative share several forms of present tense. It can be indistinguishable from simple negative statement. P\=ali has another particle dedicated to this purpose---\pali{m\=a}. So, it is better to say ``\pali{m\=a gacchatu}'' instead. Yet, as the tradition notes, prohibition often expresses in past tenses.\footnote{Kacc\,420, R\=upa\,471, Sadd\,888, Mogg\,6.13} Here are examples from the canon:

\begin{quote}
\pali{kha\d no vo m\=a uppaccag\=a}\footnote{Dhp\,22.315} \\
``Don't let the moment passed.'' \\[2mm]
\pali{m\=a vo ruccittha gamana\d m}\footnote{Ja\,22:1891} \\
``Don't be delighted in going'' \\[2mm]
\pali{m\=akattha p\=apaka\d m kamma\d m}\footnote{Ud\,5.44} \\
``Don't do evil action''
\end{quote}

Aggava\d msa describes that in the canon using \pali{m\=a} in imperative is rare but it is more found in the commentaries.\footnote{Sadd\,889} It is common in short prohibition, e.g.\ \pali{m\=a vada/vad\=ahi} (Don't say), \pali{m\=a gaccha/gacch\=ahi} (Don't go), \pali{m\=a bhu\~njassu} (Don't eat), and \pali{m\=a hotu} (Don't be). In present and perfect tense, it even rarer\footnote{Sadd\,890} but some instances can be found, e.g.\ ``\pali{m\=a kisittho may\=a vin\=a}''\footnote{j\=a 22.1713} (Without me, don't be exhausted), and ``\pali{m\=a deva paridevesi}''\footnote{Ja\,22:1857} (Lord sir, don't lament).

To conclude, in fact there is no rule to forbid using \pali{m\=a} in a particular manner. Observations from grammarians tell us that it is fashionable in a specific structure. That is good to know. When you use \pali{m\=a} in prohibition, however, I suggest you to feel free. If it sounds sensible, you can use it in any manner.

\section*{Exercise \ref{chap:imp}}
Say these in P\=ali.
\begin{compactenum}
\item Please tell me the way to the library.
\item From here, [you] go this way to the second crossroad and turn right.
\item I see.
\item From there, you will see a red building. Go beyond that building. The library stands on the left.
\item Please tell me when the library closes.
\item 5 p.m. Therefore you have to hurry.
\item I hope I reach there before that.
\item Don't walk. Run.
\end{compactenum}

