\chapter{Cases Summarized}\label{chap:cases}

This chapter is, in a way, the wrap-up of the heart of P\=ali grammar. If we do not understand this, we cannot understand P\=ali at all. With cases, components of the language can be put together in a meaningful way. So, in this chapter what we have learned from the beginning will be summarized, and what have never been mentioned will be introduced. To this point, the readers are supposed to be familiar with the language to some extent. So, I will not hesitate to use jargon and go deeper as far as the tradition leads us. You will exercise your mental muscle a lot here. Usually I am not fond of using jargon, but in this situation I find it is really inevitable and it makes many things easier. It is better than creating vague English terms to mislead you in the end. To make this less intimidating, you will find some of technical terms turning into English anyway. So, be patient and stay with me until you are familiar with them all.

\phantomsection
\addcontentsline{toc}{section}{Introduction to \pali{K\=araka}}
\section*{Introduction to \pali{K\=araka}}

The technical term that is used to call this matter in general is \pali{k\=araka}. The term literally means `doer.' In specific sense, it means cases we use in sentences. It can also mean more or less `syntax' as we use in English.\footnote{Steven Collins translates \pali{k\=araka} (or perhaps by rendering \pali{kiriy\=animitta\d m}) as `factor of action' \citep[p.~42]{collins:grammar}. I find this of little help.} To the tradition, grammatically \pali{k\=araka} means `cause' or `sign' of verbs (\pali{kiriy\=animitta\d m k\=araka\d m}\footnote{Sadd\,547}). That definition does not really help much. At the end of the mentioned formula, a more detailed description goes ``\pali{kriy\=abhisambandhalakkha\d na\d m k\=araka\d m}'' (\pali{k\=araka} has the characteristic of verbal relation). That sounds a little better. This description reminds us to cases we use in sentences. When nominative case is used, it relates to the verb as a subject. Likewise when accusative case is used, it relates to the verb as an object, and so on.

Corresponding to cases, there are six kinds of \pali{k\=araka}: \pali{kattuk\=araka} (comparable to nom.), \pali{kammak\=araka} (comparable to acc.), \pali{kara\d nak\=araka} (comparable to ins.), \pali{sampad\=anak\=araka} (comparable to dat.), \pali{apad\=anak\=araka} (comparable to abl.), and \pali{ok\=asak\-\=araka} (comparable to loc.). In Mogg, the last two are called \pali{ava\-dhik\=araka} and \pali{\=adh\=arak\=araka} respectively. Where is genitive case then? If this question pops up in your mind, you probably do not understand gen.\ well enough. I will not tell you right now. You should think carefully about it, or just read on; the answer waits ahead. However, we can occasionally see terms in gen.\ form take the position of other cases, particularly acc.

As you have seen, `\pali{k\=araka}' itself has a wide range of meaning and use. It is really difficult to find an English equivalent, so I will not translate the term and use \pali{k\=araka} throughout this chapter.

Before we go further, it is better to clarify some grammatical terms being used in due course. Like English in general, when we talk about \emph{sentence} it means ``a complete unit of word combination conveying certain idea'' (my definition). To be complete, at least a \emph{verb} has to be present, for example, ``Go!'' Other components of a sentence are \emph{subject}, the actor of the verb, and \emph{object}, the object that the verb does onto. For example, in ``I kick a ball'' `I' is subject and `a ball' is object. These are the basic terms.

Subject and object are basically nouns or noun phrases. Nouns taking subject role do the verb. That is simple. But object role can be divided into \emph{direct} object marked by accusative case, and \emph{oblique}\footnote{``Any case affix other than nominative or accusative'' \citep[p.~318]{brownmiller:dict}. Vocative case is also not oblique.} object marked by other cases. When a noun is marked by dative case, we call it an \emph{indirect} object.

Precisely, verbs can be structured into to three types of perspective: active, passive, and middle voice. Active voice is straightforward: Subject does something to object. Technically we call subject \emph{agent}\footnote{``Prototypical agents are human beings acting of their own volition, using their own energy, producing an effect on something or creating something'' \citep[p.~387]{brownmiller:dict}.}, and call direct object \emph{patient}\footnote{``Prototypical patients are animate or inanimate, do not exercise their own volition or produce an effect but undergo an action or process'' \citep[p.~387]{brownmiller:dict}.}. In my example ``I kick a ball,'' `I' takes agent role and `a ball' takes patient role. Subject in active structure is both \emph{grammatical} subject and \emph{logical} subject.

In passive voice, on the other hand, patient in a sentence turns to be subject, whereas agent turns to be oblique. In P\=ali this agent is marked by instrumental case. In passive structure, the grammatical subject\footnote{Some may call this \emph{theme} but I will not use this term.} is the patient and not equal to the logical subject which is the oblique one. My example is ``A ball is kicked by me.'' In this sentence, `a ball,' the patient is the subject, and `by me' denotes the agent. In passive structure, the verb takes a different form. In English it is auxiliary `be' plus a past participle. In P\=ali, for \pali{\=akhy\=ata} a different verb formation is used, and for verbal \pali{kita} some \pali{paccaya}s is used only for active voice, some only for passive voice, some for both. Learn more about passive voice, see Chapter \ref{chap:pass}.

Middle voice goes in between. Here is an example, ``This sweater washes well.'' As you have seen, it looks unusual because the sweater is not supposed to wash itself. However, the sentence has nothing to do neither with agent nor patient. It denotes the participant that controls the situation.\footnote{\citealp[p.~467]{brownmiller:dict}} In P\=ali, we can say that middle voice exists only in form (\pali{attanopada}), and has no substantial use.

\phantomsection
\addcontentsline{toc}{subsection}{\pali{Kattuk\=araka}}
\subsection*{1.\,\pali{Kattuk\=araka}}

A person (or thing) that does the action is called \pali{kattuk\=araka}.\footnote{Kacc\,281, R\=upa\,294, Sadd\,548} This is equivalent to `subject' in English grammar. There are three kinds of it:

\paragraph*{(1) Direct subject (\pali{Suddhakatt\=a})} This is the most ordinary way when we think of subject. It is the actor of the verb, for example, ``\textbf{I} go'' (\pali{\textbf{aha\d m} gacch\=ami}), ``\textbf{A cook} cooks food'' (\pali{\textbf{s\=udo} bhatta\d m pacati}), ``\textbf{A child} is born'' (\pali{\textbf{putto} j\=ayati}). Things of imagination can be the subject as well\footnote{Sadd\,549}, for example, ``\textbf{A horn} of a rabbit stands'' (sasa\textbf{vis\=a\d na\d m} ti\d t\d thati), ``\textbf{A son} of a sterile woman runs'' (va\~njh\=a\textbf{putto} dh\=avati). As you have seen, subject can be a part of compounds, so you have to know how to break the chunk down, see Appendix \ref{chap:samasa} for more information.

\paragraph*{(2) Causative subject (\pali{Hetukatt\=a})} This is not the direct actor of the verb, but one who causes the real actor does the action. It is the man who gets the (other) man to do the action (``\pali{\textbf{puriso} purisa\d m kamma\d m k\=areti}''), or the man who gets the (other) man to rise from the seat (``\pali{\textbf{puriso} purisa\d m \=asan\=a u\d t\d th\=apeti}'').\footnote{Kacc\,282, R\=upa\,295, Sadd\,550. That is the sense explained by the tradition.} The verb used in this case takes a different form, as you may see. To learn more about causative structure, see Chapter \ref{chap:caus}.

\paragraph*{(3) Objective subject (\pali{Kammakatt\=a})} This is a bit baffled to English speakers. It is the object of the verb that does the action by its own terms. Here is an example, ``\pali{\textbf{odano} paciyati}'' (Rice cooks). This is what we call middle voice explained above. In this case the state of being cooked happens to the rice\footnote{The example given by the textbooks is actually ``\pali{sayameva paciyati odano}'' (Rice is cooked by itself). The reason given is that it is so easy to do by itself, \pali{kammabh\=utopi sukaratt\=a sayameva sijjhanto viya hoti} (Sadd\,548).}, or it is cooked in a miraculous way. It is more sensible, I think, to see this as passive voice, thus it should be read ``Rice is cooked.'' We use a different form of verb to mark this structure. Another example is more familiar to us, ``\pali{s\=udena \textbf{odano} paciyati}'' (Rice is cooked by a chef). This is in passive structure with patient as subject. This sentence shows the agent by marking it with instrumental case.

\bigskip
According to its role, \pali{katt\=a} can be seen as the agent of the action. As such, it can be divided further into two categories: subject agent and non-subject agent.\footnote{The words chosen here is awkward, but they are closest to my understanding. Steven Collins would call these `expressed' and `unexpressed' agent \citep[p.~143]{collins:grammar}.}

\paragraph*{(1) Subject agent (\pali{Abhihitakatt\=a})} This agent agrees with the verb of sentences which occupies the subject position, and it is marked by nominative case, for example, ``\pali{\textbf{puriso} magga\d m gacchati}'' (\textbf{A man} goes the path).

\paragraph*{(2) Non-subject agent (\pali{Anabhihitakatt\=a})} This agent is not put in the subject position. It is the agent of the patient in passive structure. It takes instrumental case when composed in sentences, for example, \pali{s\=udena} in ``\pali{s\=udena odano paciyati}'' above. Another example using \pali{kita} verb is ``\pali{\textbf{buddhena} jito m\=aro}'' (The Evil One was won by \textbf{the Buddha}).

\phantomsection
\addcontentsline{toc}{subsection}{\pali{Kammak\=araka}}
\subsection*{2.\,\pali{Kammak\=araka}}

In Kacc, an explanation goes succinctly as ``\pali{ya\d m karoti ta\d m kamma\d m}''\footnote{Kacc\,280, R\=upa\,285. In Sadd\,551 ``or what one sees'' is added.} (What one does, it is \pali{kamma}). This is what we call `direct object' in English grammar. In P\=ali, accusative case is the sign of this object. However, in P\=ali \pali{kamma} has a wider denotation. It can be things created (\pali{nipphattan\=iyakamma}), e.g.\ ``\pali{\textbf{chatta\d m} karoti}'' ([One] makes an umbrella). It can be things transformed (\pali{vikara\d n\=iyakamma}), e.g.\ ``\pali{\textbf{ka\d t\d thama\.ng\=ara\d m} karoti}'' ([One] transforms a piece of wood into charcoal), \pali{\textbf{suva\d n\d na\d m key\=ura\d m karoti}} ([One] transforms gold into an arm-bracelet), \pali{\textbf{v\=ihiyo} lun\=ati} ([One] reaps [transforms the plants into] paddy). It is worth noting here that verb \pali{karoti} can take two objects in the manner that certain transformation happens between the two. It is like we say, for example, ``I make wood charcoal'' in English. And, \pali{kamma} can be of other verbs as well (\pali{p\=apan\=iyakamma}), e.g.\ \pali{\textbf{nivesana\d m} pavisati} ([One] enters the house), \pali{\textbf{r\=upa\d m} passati} ([One] sees an image), \pali{\textbf{dhamma\d m} su\d n\=ati} ([One] listens to the Dhamma), \pali{\textbf{pa\d n\d dite} payirup\=asati} ([One] associates with wise men).

In Sadd\,551, other classification of \pali{kamma} can be seen. It can be of one's desire (\pali{icchitakamma}), e.g.\ \pali{\textbf{bhatta\d m} bhu\~njati} ([One] eats food), or the negative of that (\pali{anicchitakamma}), e.g.\ \pali{\textbf{visa\d m} gilati} ([One] swallows a poison). And it can be `said' (\pali{kathitakamma}) or `unsaid' (\pali{akathitakamma}), for example, \pali{aja\d m g\=ama\d m nayati} ([he] leads a goat to the village). In the example, \pali{aja\d m} is `said,' this is the direct object; and \pali{g\=ama\d m} is `unsaid,' in English terms this is an oblique object or the object of proposition `to.' In P\=ali it can be problematic with verbs that take two objects or more at the same time like this one. The context can help the translation: you can lead a goat to the village but you cannot lead the village to the goat. If we change the sentence to ``\pali{aja\d m d\=araka\d m nayati}'' it will cause a headache, because this can mean you lead a goat and a child to somewhere, or you lead a goat to a child, or you lead a child to a goat. P\=ali has quite a lot of ambiguity pitfalls, so to speak.

If \pali{kamma} is in patient role, we can classify it into two types: subject patient, and non-subject patient.

\paragraph*{(1) Subject patient (\pali{Abhihitakamma})} This is the subject of a passive sentence that takes nominative case. For example, it is \pali{odano} in ``\pali{s\=udena \textbf{odano} paciyati}.''

\paragraph*{(2) Non-subject patient (\pali{Anabhihitakamma})} This is the direct object of an active sentence that takes accusative case. For example, it is \pali{chatta\d m} in ``\pali{\textbf{chatta\d m} karoti}.''

\phantomsection
\addcontentsline{toc}{subsection}{\pali{Kara\d nak\=araka}}
\subsection*{3.\,\pali{Kara\d nak\=araka}}

This is the instrument one uses to do an action, or one uses to see things.\footnote{Kacc\,279, R\=upa\,292, Sadd\,552} There are two kinds of instrument: inside and outside the body. As you may guess, the instrument used is marked by instrumental case.

\paragraph*{(1) Internal instrument (\pali{Ajjhattikakara\d na})} Here are examples:\par
- \pali{\textbf{cakkhun\=a} r\=upa\d m passati} ([One] sees an image with an eye.)\par
- \pali{\textbf{sotena} sadda\d m su\d n\=ati} ([One] hears with an ear.)\par
- \pali{\textbf{manas\=a} dhamma\d m vij\=an\=ati} ([One] knows the Dhamma with the mind.)\par

\paragraph*{(2) External instrument (\pali{B\=ahirakara\d na})} Examples are:\par
- \pali{\textbf{dattena} v\=ihi\d m lun\=ati} ([One] reaps paddy with a sickle.)\par
- \pali{\textbf{pharasun\=a} rukkha\d m chindati} ([One] cuts a tree with a hatchet.)\par

\phantomsection
\addcontentsline{toc}{subsection}{\pali{Sampad\=anak\=araka}}
\subsection*{4.\,\pali{Sampad\=anak\=araka}}

In English terms, this can be seen as indirect object, the recipient of a giving. But in P\=ali there are more things than that to be concerned. In Kacc, the explanation goes like this: ``For whom an intended gift goes, a satisfaction goes, or a holding goes, it is \pali{sampad\=ana}.''\footnote{\pali{yassa d\=atuk\=amo rocate dh\=arayate v\=a ta\d m sampad\=ana\d m} (Kacc\,276, R\=upa\,302). In Sadd\,553, holding is not mentioned.} When composed in a sentence, this is normally marked by dative case.

Here are basic examples:\par
- \pali{\textbf{rukkhassa} jala\d m dad\=ati} ([One] gives water to a tree.)\par
- \pali{\textbf{y\=acak\=ana\d m} bhojana\d m dad\=ati} ([One] gives food to beggars.)\par
- \pali{\textbf{sama\d nassa} rocate sacca\d m} (Truth satisfies an ascetic.)\par
- \pali{\textbf{devadattassa} suva\d n\d nacchatta\d m dh\=arayate ya\~n\~nadatto} \\(Ya\~n\~nadatta holds a golden parasol for Devadatta.)\par

Moreover, in Kacc\,277, R\=upa\,303, and Sadd\,554, \pali{sampad\=ana} can relate to other roots or terms and sometimes has loc.\ and ins.\ sense. To English speakers it sounds much like an object of a verb or preposition, but in dative form.

\paragraph*{Relation to \pali{sil\=agha}} For example:\par
- \pali{\textbf{buddhassa} sil\=aghate} ([One] praises the Buddha.)\par

\paragraph*{Relation to \pali{hanu}} For example:\par
- \pali{\textbf{ra\~n\~no} hanute} ([One] deceives\footnote{This means hiding something by not talking about it. \pali{ettha ca hanuteti apanayati, apalapati all\=apasall\=apa\d m na karot\=iti attho} (Sadd\,554).} the king.)\par

\paragraph*{Relation to \pali{\d th\=a}}  For example:\par
- \pali{upati\d t\d theyya \textbf{sakyaputt\=ana\d m} va\d d\d dhak\=i} (The carpenter \\should look after the S\=akya's sons.)\par

\paragraph*{Relation to \pali{sapa}} For example:\par
- \pali{\textbf{mayha\d m} sapate} ([He] swears to me)\par
- \pali{sapathampi \textbf{te} samma aha\d m karomi}\footnote{Ja\,21:407} (Dear friend, I even do the swear to you.)\par

\paragraph*{Relation to \pali{dh\=ara}\footnote{In Sadd-Dh\=a, this should be \pali{dhara}.}} This is in the sense of obligation to pay back, for example:\par
- \pali{suva\d n\d na\d m \textbf{te} dh\=arayate} ([He] owes gold to you.)\par
- \pali{tassa \textbf{ra\~n\~no} maya\d m n\=aga\d m dh\=aray\=ama} (We owe an elephant to that king.)\par

\paragraph*{Relation to \pali{piha}} For example:\par
- \pali{\textbf{buddhassa} a\~n\~natitthiy\=a pihayanti} (Other adherents compliment the Buddha).

\paragraph*{Relation to \pali{kudha}} For example:\par
- \pali{kujjhati \textbf{devadattassa}} ([He] gets angry with Devadatta.)\par
- \pali{\textbf{Tassa} kujjha mah\=av\=ira}\footnote{Ja\,4:49} (Be angry with him, Mah\=av\=ira.)\par

\paragraph*{Relation to \pali{duha}} This is in the sense of destruction, for example:\par
- \pali{duhayati \textbf{dis\=ana\d m} megho} (The cloud ruins directions = There are clouds in all directions = The sky is full of clouds.)\par
- \pali{yo \textbf{mitt\=ana\d m} na dubbhati}\footnote{e.g.\ Ja\,22:12} ([The one] who does not do harm to friends.)\par
								 
\paragraph*{Relation to \pali{issa}} For example:\par
- \pali{titthiy\=a issayanti \textbf{sama\d n\=ana\d m}} (Other adherents envy for monks.)\par
- \pali{dev\=a na issanti \textbf{purisaparakkamassa}}\footnote{Ja\,4:4} (Gods do not envy for human's endeavor.)\par
								 
\paragraph*{Relation to \pali{us\=uya}} For example:\par
- \pali{dujjan\=a \textbf{gu\d navant\=ana\d m} us\=uyanti}\footnote{In Sadd\,554 it is \pali{ussuyyanti}. This verb can take acc.\ object as well, but it becomes \pali{kammak\=araka}.} (Bad people envy virtuous ones.)\par
								 
\paragraph*{Relation to \pali{r\=adha}} For example:\par
- \pali{\=ar\=adhoha\d m \textbf{ra\~n\~no}}\footnote{This can be \pali{kammak\=araka} by taking acc.\ object, thus \pali{r\=aj\=ana\d m}.} (I am a pleasing one for the king = I please the king.)\par
								 
\paragraph*{Relation to \pali{ikkha}} For example:\par
- \pali{\textbf{\=ayasmato up\=alissa} upasampad\=apekkho upatisso}\footnote{This also can be \pali{kammkak\=araka} by taking acc.\ object, thus \pali{up\=ali\d m}.} (Upatissa who is a candidate of the ordination for/of Ven.\,Up\=al\=i.)\footnote{We can also see this as genitive case that relates \pali{apekkha} to \pali{up\=al\=i}, thus a candidate of Up\=al\=i (\pali{up\=alissa apekkho}). This sounds better because there is no verb to relate in this sentence.}\par
								 
\paragraph*{Relation to \pali{su}} For example:\par
- \pali{``Eva\d m, bhante''ti kho s\=ariputtamoggall\=an\=a \textbf{bhagavato} paccassosu\d m}\footnote{Cv\,1.23. In this example, \pali{paccassosu\d m} is an aorist form of \pali{pa\d tissu\d n\=ati (pati + su)}.} (``Yes, sir,'' Ven.\,S\=ariputta and Moggall\=ana agreed with the Buddha.)\par
								 
\paragraph*{Relation to \pali{ge}} For example:\par
- \pali{bhikkhu jana\d m dhamma\d m s\=aveti, \textbf{tassa bhikkhuno} jano anug\-i\d n\=ati} (A monk has a person listen to the Dhamma, the person utters after that monk.)\par
								 
\paragraph*{Relation to `tell' or `show'} For example:\par
- \pali{\=arocay\=ami \textbf{vo} bhikkhave} (Monks, I will tell you [something].)\par
- \pali{pa\d tiveday\=ami \textbf{vo} bhikkhave} (Monks, I will show you [something].)\par
								 
\paragraph*{Relation to `benefit'} For example:\par
- \pali{\textbf{buddhassa atth\=aya} j\=ivita\d m pariccaj\=ami} (I give up the life for the benefit of the Buddha.)\par
								 
\paragraph*{Relation to \pali{-tu\d m}} This means terms in dative case can substitute the infinitive (verbs in \pali{tu\d m} form). For example:\par
- \pali{\textbf{lok\=anukamp\=aya}}\footnote{= \pali{loka\d m anukampitu\d m.}} (for compassion to the world.)\par
- \pali{\textbf{bhikkh\=una\d m ph\=asuvih\=ar\=aya}}\footnote{= \pali{ph\=asuviharitu\d m}} (for well-being of monks.)\par
								 
\paragraph*{Relation to \pali{ala\d m}} As an indeclinable \pali{ala\d m} has two senses: `suitable for' and `enough!' or `stop!' In the former sense, here are examples:\par
- \pali{ala\d m \textbf{me} rajja\d m} (The kingship [is] suitable for me.)\par
- \pali{ala\d m bhikkhu \textbf{pattassa}} (The monk [is] suitable for the bowl.)\par
- \pali{ala\d m mallo \textbf{mallassa}} ([This] wrestler [is] suitable for [that] wrestler.)\par
In the latter sense, examples are:\par
- \pali{ala\d m \textbf{me} hira\~n\~nasuva\d n\d nena} (Stop! for me, with silver and gold [I have had enough of silver and gold].)\par
- \pali{ala\d m \textbf{te} idha v\=asena} (That's enough for you to live here.)\par
								 
\paragraph*{Relation to \pali{mana}} This means `think' but in a disrespectful way by comparing with things, for example:\par
- \pali{\textbf{ka\d t\d thassa} tuva\d m ma\~n\~ne} (I think you are a piece of wood.)\par
If positive meaning is intended, acc.\ is used, e.g.\ \pali{suva\d n\d na\d m ta\d m ma\~n\~ne} (I think you are gold). Also if living being is compared with disrespect, acc.\ is used, e.g.\ \pali{gadrabha\d m tuva\d m ma\~n\~ne} (I think you are a donkey).

\paragraph*{Relation to \pali{gamu}} For example:\par
- \pali{\textbf{g\=amassa} p\=adena gato} ([One] went to the village by foot)\par
- \pali{appo \textbf{sagg\=aya} gacchati}\footnote{Dhp\,13.174} (A small number [of people] go to heaven.)\par
Normally we use acc.\ to mark the destination of going, e.g.\ \pali{g\=ama\d m p\=adena gato}. In that case it becomes \pali{kammak\=araka} instead.

\paragraph*{Relation to `wish'} For example:\par
- \pali{\textbf{\=ayasmato} d\=igh\=ayu hotu} (Long live venerable.)\par
- \pali{bhadda\d m \textbf{bhavato} hotu} (May you be lucky.)\par

\paragraph*{Relation to \pali{sammati/sammuti}} For example:\par
- \pali{s\=adhu sammuti me tassa bhagavato \textbf{dassan\=aya}} (Letting me see that Buddha is good for me = Please let me see the Buddha.)\par

\paragraph*{Relation to \pali{bhiyya}} For example:\par
- \pali{bhiyyoso \textbf{matt\=aya}}\footnote{It is said in Sadd\,554 that this dative form has ablative sense.} (More than [one can] measure.)\par

\paragraph*{In locative sense} For example:\par
- \pali{\textbf{tuyha\~ncassa} \=avikaromi} (I will reveal in your [place].)\par
- \pali{\textbf{tassa} me sakko p\=aturahosi} (The king of the gods appears in that [place] of mine.)\par

\paragraph*{In instrumental sense} For example:\par
- \pali{asakkat\=a casma \textbf{dhana\~njay\=aya}}\footnote{Ja\,4:113} (We were shown a lack of respect by King Dhana\~njaya = King Dhana\~njaya humiliated us.)\par

\paragraph*{In other various uses} Such as:\par
- \pali{upama\d m \textbf{te} kariss\=ami} (I will do/show a simile to you.)\par
- \pali{dhamma\d m \textbf{vo} desess\=ami} (I will preach the Dhamma to you.)\par
- \pali{\textbf{tassa} ph\=asu hoti} (May well-being happen to him.)\par
- \pali{\textbf{etassa} pahi\d neyya} (Send to that [person].)\par
- \pali{kappati \textbf{sama\d n\=ana\d m} \=ayogo} (Effort is suitable for monks.)\par
- \pali{\textbf{amh\=aka\d m} ma\d nin\=a attho} (Benefit with the jewel is for me.)\par
- \pali{seyyo \textbf{me} attho} (The greater benefit is for me.)\par
- \pali{bah\=upak\=ar\=a, bhante, mah\=apaj\=apati gotam\=i \textbf{bhagavato}}\footnote{Cv\,10.402} (A lot of support, sir, Mahapaj\=apati Gotam\=i [gave] to the Blessed One.)\par

\phantomsection
\addcontentsline{toc}{subsection}{\pali{Apad\=anak\=araka}}
\subsection*{5.\,\pali{Apad\=anak\=araka}}

The tradition explains this as: ``From where one goes away, from whom or which one fears, from whom or where one learns, that is \pali{apad\=ana}.''\footnote{Kacc\,271, R\=upa\,88, 308, Sadd\,555--6} This normally corresponds with ablative case. Here are some examples:\par
- \pali{\textbf{g\=am\=a} \=apenti munayo} (From the village, go away sages.)\par
- \pali{\textbf{nagar\=a} niggato r\=aj\=a} (From the city, went out the king.)\par
- \pali{\textbf{s\=avatthito} \=agacchati} (From S\=avatth\=i, [one] comes.)\par
- \pali{\textbf{cor\=a} bhaya\d m j\=ayate} (From thiefs, fear arises.)\par
- \pali{\textbf{k\=amato} j\=ayate bhaya\d m}\footnote{Dhp\,16.215} (From pleasure, arises fear.)\par
- \pali{\textbf{ta\d nh\=aya} j\=ayati soko}\footnote{Dhp\,16.216} (From craving, arises grief.)\par
- \pali{\textbf{\=acariyupajjh\=ayehi} sikkha\d m ga\d nh\=ati sisso} (From teacher and preceptor, a student learns the discipline.)\par
- \pali{\textbf{kus\=ulato} pacati}\footnote{Sadd\,557} (From the granary, [one] cooks.)\par
- \pali{\textbf{val\=ahak\=a} vijjotati cando}\footnote{Sadd\,557} (From clouds, shines the moon.)\par
- \pali{m\=athur\=a \textbf{p\=a\d taliputtakehi} abhir\=up\=a} (People of Mathura are beautiful than those of P\=a\d taliputta.)\par
- \pali{dh\=avat\=a \textbf{hatthimh\=a} patito a\.nkusadh\=ar\=i} (From a running elephant, falls the mahout.)\par
- \pali{\textbf{pabbat\=a} otaranti vanacar\=a} (From the mountain, descends foresters.)\par

\bigskip
In addition, \pali{apad\=ana} can relate to a number of roots and terms.

\paragraph*{Relation to \pali{ji, bh\=u}\footnote{Sadd\,558}} For example:\par
- \pali{\textbf{buddhasm\=a} par\=ajenti a\~n\~natitthiy\=a} (From the Buddha, other adherents are defeated.)\par
- \pali{\textbf{himavat\=a} pabhavanti pa\~nca mah\=anadiyo} (From the Himalaya, originate the great five rivers.)\par

\paragraph*{Relation to \pali{a\~n\~na, para}\footnote{Sadd\,559}} For example:\par
- \pali{\textbf{tato kammato} a\~n\~na\d m kamma\d m} (other action [apart] from that action)\par
- \pali{N\=a\~n\~natra \textbf{dukkh\=a} sambhoti, n\=a\~n\~na\d m dukkh\=a nirujjhati}\footnote{S1\,171 (SN\,5)} (No other from suffering arises, no other from suffering ceases.)\par
- \pali{\textbf{tato} para\d m} (other than that)\par

\paragraph*{Relation to \pali{apa, pari}\footnote{Sadd\,560, Mogg\,2.27}} This has the sense of `avoiding' or `apart from,' for example:\par
- \pali{apa \textbf{s\=al\=aya} \=ayanti v\=a\d nij\=a} (Avoiding from the hall, come merchants.)\par
- \pali{pari \textbf{pabbat\=a} devo vassati} (Apart from the mountain area, the rain falls.)\par

\paragraph*{Relation to \pali{u, pari (upari)}\footnote{Sadd\,561}} This has the sense of `all over,' for example:\par
- \pali{upari \textbf{pabbat\=a} devo vassati} (All over the mountain area, the rain falls.)\par

\paragraph*{Relation to \pali{\=a, y\=ava}\footnote{Sadd\,562}} This has the sense of `spreading,' for example:\par
- \pali{\=a \textbf{pabbat\=a} khetta\d m ti\d t\d thati} (To/from the mountain, the field occupies.)\par
- \pali{\=a \textbf{nagar\=a} khadiravana\d m ti\d t\d thati} (To/from the city, Acacia forest occupies.)\par
- \pali{y\=ava \textbf{brahmalok\=a} saddo abbhuggacchi}\footnote{Mv\,1.17} (Up to the Brahma world, the sound rises.)\par
- \pali{y\=ava \textbf{brahmalok\=a} ekakol\=ahala\d m j\=ata\d m} (Up to the Brahma world, the same uproar arose.)\par

\paragraph*{Relation to \pali{pati}\footnote{Sadd\,563, Mogg\,2.28}} This has the sense of `substitution,' for example:\par
- \pali{\textbf{buddhasm\=a} pati s\=ariputto dhammadesan\=aya \=alapati tem\=asa\d m} (Substituting for the Buddha, Ven.\,S\=ariputta calls [monks] for teaching the Dhamma in three months.)\par
- \pali{ghatamassa \textbf{telasm\=a} pati dad\=ati} ([One] gives ghee to him instead of oil.)\par
- \pali{kanakamassa \textbf{hira\~n\~nasm\=a} pati dad\=ati} ([One] gives gold to him instead of silver.)\par

\paragraph*{Relation to \pali{visu\d m, putha}\footnote{Sadd\,564, Mogg\,2.31}} For example:\par
- \pali{\textbf{tehi} visu\d m} (apart from them)\par
- \pali{\textbf{tato} visu\d m} (apart from that [group])\par
- \pali{\textbf{ariyehi} puthagev\=aya\d m jano} (This person [is] different from noble ones.)\par
- \pali{puthageva \textbf{janasm\=a}}\footnote{In Mogg\,2.31 using ins.\ is equivalent, thus \pali{puthageva janena}.} (only different from person)\par
- \pali{\textbf{janasm\=a} n\=an\=a}\footnote{In Mogg\,2.31 this can also be \pali{janena n\=an\=a}.} (different from person)\par

\paragraph*{Relation to \pali{a\~n\~natra}\footnote{Sadd\,565, Mogg\,2.30}} This can be in both abl.\ and ins., for example:\par
- \pali{n\=a\~n\~natra \textbf{sabbanissagg\=a}, sotthi\d m pass\=ami p\=a\d nina\d m}\footnote{S1\,98 (SN\,2)} \\(Other than giving up all [unwholesomeness], I see no well-being of the living.)\par
- \pali{a\~n\~natra \textbf{buddhupp\=ad\=a} lokassa sacc\=abhisamayo natthi} \\(Other than the arising of the Buddha, there is no occasion of [knowing] the truth of the world.)\par
- \pali{tadantara\d m ko j\=aneyya a\~n\~natra \textbf{tath\=agatena}}\footnote{A6\,44} (For that matter, who should know apart from the Enlightened One?)\par

\paragraph*{Relation to \pali{rite, vin\=a}\footnote{Sadd\,566, Mogg\,2.29--30}} This can be in abl., ins., and acc. for example:\par
- \pali{rite \textbf{saddhamm\=a} kuto sukha\d m bhavati}\footnote{This can also be \pali{rite saddhammena \ldots} or \pali{rite saddhamma\d m \ldots}} (Without the true doctrine, from where happiness exists.)\par
- \pali{vin\=a \textbf{saddhamm\=a} nattha\~n\~no koci n\=atho loke vijjati}\footnote{This can also be \pali{vin\=a saddhammena \ldots} or \pali{vin\=a saddhamma\d m \ldots}} (Without the true doctrine, any other protector in the world does not exist.)\par

\paragraph*{Relation to `beginning from' or `since' (\pali{pabhuti})\footnote{Sadd\,567}} For example:\par
- \pali{\textbf{yato}ha\d m, bhagini, ariy\=aya j\=atiy\=a j\=ato}\footnote{M2\,351 (MN\,86)} (Sister, since when I was born with the noble birth)\par
- \pali{\textbf{yato} sar\=ami att\=ana\d m} (Since when I remember myself)\par
- \pali{\textbf{yato} pabhuti} (since when)\par
- \pali{\textbf{yato} pa\d t\d th\=aya} (since when)\par
- \pali{\textbf{ito} pa\d t\d th\=aya} (since this [time])\par
- \pali{\textbf{ajjato} pa\d t\d th\=aya} (since today)\par

\paragraph*{Relation to `duration' and `distance'\footnote{Sadd\,568}} For example:\par
- \pali{\textbf{ito pakkhasm\=a} vijjhati miga\d m luddako} (From this fortnight, the hunter will shoot a deer.)\par
- \pali{\textbf{ito kos\=a} vijjhati ku\~njara\d m} (From this kosa [$\approx$500 bows of distance], [the hunter] shoots an elephant.)\par
- \pali{\textbf{ito m\=asasm\=a} bhu\~njati bhojana\d m} (From this month, he/she will eat food.)\par

\paragraph*{Relation to `protection'\footnote{Kacc\,237, R\=upa\,310, Sadd\,569}} For example:\par
- \pali{k\=ake rakkhanti \textbf{ta\d n\d dul\=a}} (They prevent crows from rice-grain.)\par
- \pali{\textbf{yav\=a} pa\d tisedhenti g\=avo} (They prevent cows from barley.)\par
- \pali{\textbf{n\=an\=arogato} v\=a \textbf{n\=an\=aupaddavato} v\=a \=arakkha\d m ga\d nhantu} (Take the protection from various diseases or various dangers.)\par
- \pali{mantino mantena \textbf{d\=arakehi} pis\=ace rakkhanti} (Enchanters prevent demons from children with a spell.)\par
- \pali{\textbf{p\=ap\=a} citta\d m niv\=araye}\footnote{Dhp\,9.116} ([One] should protect the mind from evils.)\par

\paragraph*{Relation to `disappearing'\footnote{Kacc\,274, R\=upa\,311, Sadd\,570}} For example:\par
- \pali{\textbf{upajjh\=ay\=a} antaradh\=ayati sisso} (From the preceptor, disappears a student.)\par
- \pali{\textbf{m\=atar\=a} ca \textbf{pitar\=a} ca antaradh\=ayati putto} (From mother and father, disappears a child.)\par
- \pali{\textbf{jetavane} antarahito}\footnote{In this instance and the following, loc.\ is used. In Sadd\,570 it is explained that when things or persons disappear due to danger, abl.\ is used. If the disappearance is caused by miracle, loc.\ is used instead.} ([One] disappeared in the Jetavana.)\par
- \pali{yakkho \textbf{tatthe}vantaradh\=ayati} (The demon disappears at that place.)\par

\bigskip
There are other miscellaneous concerns mentioned in Kacc\,275, R\=upa\,312, Sadd\,571.
\paragraph*{Relation to `remoteness' (\pali{d\=ura})} For example:\par
- \pali{k\=ivad\=uro \textbf{ito} na\d lak\=arag\=amo} (How far from here [is] the village of basket-makers?)\par
- \pali{\=arak\=a te moghapuris\=a \textbf{imasm\=a dhammavinay\=a}}\footnote{It is said that acc.\ and ins.\ can also be used, thus \pali{\=arak\=a \ldots ima\d m dhammavinaya\d m}, \pali{\=arak\=a \ldots anena dhammavinayena}.} (Those useless men [are] far away from this teaching and discipline.)\par
- \pali{\textbf{tato} have d\=uratara\d m vadanti}\footnote{Ja\,21:414} ([They] say [it is] farther than that.)\par
- \pali{\textbf{g\=amato} n\=atid\=ure} (in [the place] not too far from the village)\par
- \pali{\textbf{d\=ur\=a g\=am\=a} \=agato}\footnote{Also acc.\ and ins.\ can be used, hence \pali{d\=ura\d m g\=ama\d m \=agato}, \pali{d\=urena g\=amena \=agato}.} ([One] came from a distant village.)\par

\paragraph*{Relation to `closeness' (\pali{antika})} For example:\par
- \pali{antika\d m/\=asanna\d m/sam\=ipa\d m \textbf{g\=am\=a}}\footnote{In this sense, acc., ins., and gen.\ can also be used, thus \ldots \pali{g\=ama\d m} or \pali{g\=amena} or \pali{g\=amassa}.} (a near [place] from the village)\par
- \pali{sam\=ipa\d m \textbf{saddhamm\=a}} (closeness from the true teaching)\par
- \pali{\textbf{nibb\=anasseva} santike}\footnote{Dhp\,2.32. In this instance gen.\ is used.} (in closeness of nirvana)\par

\paragraph*{Relation to `measurement of distance'} For example:\par
- \pali{\textbf{ito mathur\=aya} cat\=usu yojanesu sa\.nkassa\d m n\=ama nagara\d m atthi} (There is a city called Sa\.nkassa 4 yojanas from this Mathur\=a.)\par
- \pali{\textbf{r\=ajagahato} pa\~ncacatt\=a\d l\=isayojanamatthake s\=avatthi} (S\=avatth\=i resides in 45 yojanas from R\=ajagaha.)\par

\paragraph*{Relation to `measurement of time'} For example:\par
- \pali{\textbf{Ito} so, bhikkhave, ekanavutikappe ya\d m vipass\=i bhagav\=a araha\d m samm\=asambuddho loke udap\=adi}\footnote{D2\,4 (DN\,14)} (Ninety-one eons from this one, monks, that Vipass\=i Buddha, an arhant, a perfectly Enlightned One, arose in the world.)\par
- \pali{\textbf{Ito} ti\d n\d na\d m m\=as\=ana\d m accayena tath\=agato parinibb\=ayissati}\footnote{D3\,168 (DN\,16)} (From now by a lapse of 3 months the Enlightened One will attain the final release.)\par

\paragraph*{Relation to `deleted absolutives'} For example:\par
- \pali{\textbf{p\=as\=ad\=a} sa\.nkameyya}\footnote{The deleted terms are \pali{p\=as\=ada\d m abhiruhitv\=a}. So, the meaning is ``Having ascended the mansion, one should get out of it.''} ([One] should get out from the mansion.)\par
- \pali{\textbf{\=asan\=a} vu\d t\d thaheyya}\footnote{The deleted terms are \pali{\=asane nis\=iditv\=a}. So, the meaning is ``Having sat down in the seat, one should arise from it.''} ([One] should arise from the seat.)\par

\paragraph*{Relation to `directions'} For example:\par
- \pali{\textbf{ito} s\=a purim\=a dis\=a}\footnote{D3\,278 (DN\,32)} (From this [point], that direction [is] the east.)\par
- \pali{\textbf{puratthimato dakkhi\d nato pacchimato uttarato} agg\=i pajjalanti} (From the east, south, west, north the fire blazes up.)\par
- \pali{uddha\d m \textbf{p\=adatal\=a}} (upwards from the sole)\par
- \pali{adho \textbf{kesamatthak\=a}} (downwards from the hair)\par

\paragraph*{Relation to `classification'} For example:\par
- \pali{\textbf{yato} pa\d n\=itataro v\=a visi\d t\d thataro v\=a natthi} (There is no [teaching] more exalted or more excellent than which [of the Buddha].)\par
- \pali{m\=athur\=a \textbf{p\=a\d taliputtakehi} abhir\=up\=a} (People of Mathura are beautiful than those of P\=a\d taliputta.)\par
- \pali{attadanto \textbf{tato} vara\d m}\footnote{Dhp\,23.322} (A self-restrained person [is] more excellent than that [well-trained horses and elephants].)\par

\paragraph*{Relation to `abstinence'} For example:\par
- \pali{\textbf{p\=a\d n\=atip\=at\=a} verama\d n\=i} (abstinence from taking lives)\par
- \pali{\textbf{micch\=a\=aj\=iv\=a} \=arati virati pa\d tivirati verama\d n\=i}\footnote{M3\,140 (MN\,117)} (abstinence from wrong livelihood)\par

\paragraph*{Relation to `cleanness'} For example:\par
- \pali{so\d nada\d n\d do \textbf{ubhato} suj\=ato \textbf{m\=atito} ca \textbf{pitito} ca, sa\d msuddhagaha\d niko}\footnote{D1\,303 (DN\,4)} (So\d nada\d n\d da [was] well born, of pure descent, from both mother's and father's side.)\par

\paragraph*{Relation to `liberating'} For example:\par
- \pali{na te muccanti \textbf{maccun\=a}} (They are not liberated from death)\par
- \pali{mokkhanti \textbf{m\=arabandhan\=a}}\footnote{Dhp\,3.37} ([They] are set free from the imprisonment of death.)\par

\paragraph*{Relation to `cause'} For example:\par
- \pali{\textbf{kasm\=a} nu tumha\d m dahar\=a na miyyare}\footnote{Ja\,10:92} (Why don't your young people die?)\par
- \pali{\textbf{kasm\=a} idheva mara\d na\d m bhavissati} (Why does death exist only here?)\par

\paragraph*{Relation to `seclusion'} For example:\par
- \pali{vivitto \textbf{p\=apak\=a dhamm\=a}} ([One] secluded from evil nature)\par
- \pali{vivicca \textbf{akusalehi dhammehi}}\footnote{D1\,467 (DN\,10)} (having secluded from unwholesome natures)\par

\paragraph*{Relation to `measurement/approximation'} For example:\par
- \pali{\textbf{\=ay\=amato ca vitth\=arato ca} yojana\d m candabh\=ag\=aya \\pam\=a\d na\d m}\footnote{Using ins.\ is also valid here, thus \pali{\=ay\=amena, vitth\=arena}.} (From/by length and breath, the river Candabh\=aga [is] 1 yojana.)\par
- \pali{\textbf{parikkhepato} navasatayojanaparim\=a\d no majjhimadeso} (The middle country is 900 yojanas from/by circumference.)\par

\paragraph*{Relation to `former (time)' (\pali{pubba})} For example:\par
- \pali{Pubbeva me, bhikkhave, \textbf{sambodh\=a}}\footnote{A3\,104} (Monks, in the former time from my enlightenment)\par

\paragraph*{Relation to `binding'} For example:\par
- \pali{\textbf{satasm\=a} bandho naro ra\~n\~n\=a}\footnote{It is logical to use ins.\ also, thus \pali{satena}.} (A person was bound from [debt of] 100 by the king.)\par

\paragraph*{Relation to `characteristic identification'} For example:\par
- \pali{\textbf{pa\~n\~n\=aya} vimuttimano} (released mind [is] from wisdom)\par
- \pali{\textbf{S\=ilato} na\d m pasa\d msanti}\footnote{A4\,6} ([They] praise him from moral [because of his moral].)\par

\paragraph*{Relation to `questioning'} For example:\par
- \pali{\textbf{kuto}si tva\d m} (Where are you from?)\par

\paragraph*{Relation to `little, difficult'} For example:\par
- \pali{\textbf{thok\=a} muccanti}\footnote{It is reasonable that ins.\ should be used instead, thus \pali{thokena}, and \pali{appamattakena, kicchena} in the following.} ([They] are a little free.)\par
- \pali{\textbf{appamattak\=a} muccanti} ([They] are a kind of little free.)\par
- \pali{\textbf{Kicch\=a} laddho piyo putto}\footnote{Ja\,22:353} (A beloved son was obtained [by him] from difficulty.)\par

\phantomsection
\addcontentsline{toc}{subsection}{\pali{Ok\=asak\=araka}}
\subsection*{6.\,\pali{Ok\=asak\=araka}}

On where the action stands (\pali{\=adh\=ara}), it is \pali{ok\=asa}.\footnote{Kacc\,278, R\=upa\,320, Sadd\,572} This \pali{k\=araka} is normally in loc.\ form. Here are some examples:\par
- \pali{\textbf{jalesu} kh\=ira\d m ti\d t\d thati} (Milk stays in the water.)\par
- \pali{\textbf{tilesu} tela\d m ti\d t\d thati} (Oil resides in the sesame seeds.)\par
- \pali{\textbf{\=asane} nisinno sa\.ngho} (The group sat on the seat.)\par
- \pali{\textbf{th\=aliya\d m} odana\d m pacati} ([One] cooks rice in a pot.)\par
- \pali{\textbf{gha\d tesu} udaka\d m atthi} (There is water in water-pots.)\par
- \pali{\textbf{bh\=um\=isu} manuss\=a caranti} (Human beings travel on the ground.)\par
- \pali{\textbf{\=ak\=ase} saku\d n\=a pakkandanti} (Birds fly in the air.)\par
- \pali{\textbf{ga\.ng\=aya} ghoso ti\d t\d thati} (Ghosa stands nearby the Ganges.)\par
- \pali{\textbf{s\=avatthiya\d m} viharati jetavane} ([The Blessed One] lives in the Jetavana nearby S\=avatth\=i.)\par
- \pali{S\=a devat\=a antarahit\=a, \textbf{pabbate gandham\=adane}}\footnote{Ja\,22:334. In Sadd\,573, it is stressed that miraculous disappearance is marked by loc.\ not abl.} (That deity disappear at mount Gandham\=adana.)\par

\bigskip
Let me wrap up the part of \pali{k\=araka} here. As you have seen, some of them have a straitforward and limited use, some have a variety of denotation. Several of them can be used interchangeably. That might be the hard part, or easy part depending on your application. One important thing to keep in mind here is \pali{k\=araka} is all about the relation to verbs in sentences. That is the main reason why genitive case and vocative case are not \pali{k\=araka}.\footnote{Sadd\,574, 576} Genitive case marks the relation between nouns; and vocative case is used only for addressing, no relation whatsoever.

However, the tradition seems to be inconsistent on this point. My question is whether we can use ``\pali{nagar\=a puriso}'' to mean ``a man from the city.'' It sounds logical to do so, albeit it shows a relation between two nouns not noun and verb. If this is usable, then is \pali{nagar\=a} \pali{Apad\=anak\=araka}? Perhaps, in traditional point of view ``\pali{nagar\=agato puriso}'' (a man who came from the city) sounds better grammatically. However, it is not hard to find a similar example from the textbooks. For instance in Kacc\,275, R\=upa\,372, Sadd\,571 we find these: ``\pali{\=asanna\d m g\=am\=a}'' (a neighborhood nearby the village), ``\pali{uddha\d m p\=adatal\=a}'' (upwards from the sole), ``\pali{p\=a\d n\=atip\=at\=a verama\d n\=i}'' (abstinence from taking lives), and ``\pali{kasm\=a hetun\=a}'' (from/by what cause?). To be consistent, we should answer `No' to my question. Therefore, all these instances are not \pali{Apad\=anak\=araka} because there is no relation to any verb. But, the tradition put these as examples of the \pali{k\=araka}. Then the next question comes: ``Why can't genitive relation be treated in the same way?''\footnote{Some teachers try to introduce \pali{Sambandhak\=araka} in order to fill the gap, but it seems unacceptable to P\=ali scholars. See a discussion in \citealp[pp.~304--5]{supaphan:pali}.}

Strictly speaking \pali{k\=araka} and cases are not the same thing, but closely related. What we call cases in P\=ali is called nominal \pali{vibhatti} that we use in declension. There are seven or eight of them, as we have learned from the start of our course. But \pali{k\=araka} has six kinds, as you have read so far to this point. You may notice that one \pali{k\=araka} can be marked with unrelated cases. The important ones are \pali{kattuk\=araka} that can be in nom.\ (active) and ins.\ (passive), and \pali{kammak\=araka} that can be in acc.\ (active) and nom.\ (passive). It is also a good chance, when you read texts, that you can see \pali{kammak\=araka} in gen.\ form. So, the two areas, even if they have a big overlap, are not the same. We can say roughly that \pali{k\=araka} is function-oriented, whereas \pali{vibhatti} is form-oriented. The two are different sides of the same thing.

Is that a kind of big redundancy? Is it better to merge them together and explain as the same topic, so it will be easier to follow? You can think of that matter if you want to be a progressive P\=ali scholar. I think the system laid down by the tradition is not so airtight or effective that loopholes can not be found. There are many things to do, if you wish, in the field of P\=ali studies. In the following part, we will deal with nominal \pali{vibhatti}.

\clearpage
\phantomsection
\addcontentsline{toc}{section}{Nominal \pali{Vibhatti} Usage}
\section*{Nominal \pali{Vibhatti} Usage}

In this part, what we have learned from the beginning concerning cases, the nominal \pali{vibhatti}, will be summarized here. Unlike the easy way we approached the topic previously, now we shall see how the tradition handles this matter. What we have not yet learned before will be addressed here all. The first thing to be kept in mind is that `cases' is English grammatical term, whereas `\pali{vibhatti}' is P\=ali grammatical term. They are not really the same thing, strictly speaking. There are eight cases as we have learned so far, but only seven nominal \pali{vibhatti}s. The missing one is vocative case that uses the same \pali{vibhatti} as nom.

\phantomsection
\addcontentsline{toc}{subsection}{The First \pali{Vibhatti} (Nom.\ \& Voc.)}
\subsection*{The First \pali{Vibhatti} (Nom.\ \& Voc.)}

When the first (\pali{pa\d tham\=a}) \pali{vibhatti} is used, nominative case is expected in most cases, and vocative case in lesser extent. Nom.\ is mainly used to mark the subject of sentences: agent in active form and patient in passive form. Voc.\ is used for addressing. There are other concerns with this \pali{vibhatti} enumerated by the tradition as follows:

\paragraph*{(1) \pali{Li\.ngattha}} (Kacc\,284, R\=upa\,283, Sadd\,577, Mogg\,2.37)\par
Generally speaking, \pali{li\.nga} is nouns in their raw form or before declension, e.g.\ \pali{purisa}.\footnote{Moreover, \pali{upasagga} (prefixes) and \pali{nip\=ata} (particles) are also \pali{li\.nga} (Sadd\,197). In traditional point of view, they are also marked by cases but the marking is deleted. I discussed this matter in Chapter \ref{chap:ind-intro}.} In such a form, nouns do not have any meaning, just certain potential. Once the nouns are marked by nominative case, they exist as meaningful terms. The shortest sentence in P\=ali can be in one word, the noun itself, for example, ``\pali{puriso}'' ([There is] a man). We call this kind of sentence, which the verb is omitted, \pali{li\.ngattha}.\footnote{Steven Collins mentions `Hanging nominative' as one function of nom.\ \citep[p.~20]{collins:grammar}. I do not know whether this can mean the same thing.} In that manner, nominative case makes nouns become visible as a subject, so to speak.

\paragraph*{(2) Vocatives (\pali{\=alapana})} (Kacc\,285, R\=upa\,70, Sadd\,578, Mogg\,2.38)\par
As you see in Appendix \ref{chap:decl}, \pali{vibhatti} used to form nom.\ and voc.\ terms is the same, i.e.\ \pali{si, yo}, even though both cases are rendered differently sometimes. From traditional point of view, there is no vocative \pali{vibhatti} to talk about. There is just the first (\pali{pa\d tham\=a}) \pali{vibhatti} that has nominative sense or vocative sense. Here are examples: ``\pali{bho purisa}'' (Sir [man]), ``\pali{bhavanto puris\=a}'' (Sirs [men]), ``\pali{bho r\=aja}'' (Sir king), ``\pali{bhanvanto r\=aj\=ano}'' (Sirs kings), ``\pali{bhoti ka\~n\~ne}'' (Madam [girl]), ``\pali{bhotiyo ka\~n\~n\=ayo}'' (Mesdames [girls]), ``\pali{he sakkhe}'' (Hey friend), ``\pali{he sakkhino}'' (Hey friends).

\paragraph*{(3) Causes} (Sadd\,579)\par Among other cases as we shall see below, nom.\ can mark causes of the action in some cases, for example:\par
- \pali{Na \textbf{attahet\=u} alika\d m bha\d neti}\footnote{Ja\,17:76} (Not because of oneself does one [should] tell a lie.)\par
This can be used with \pali{ki\d m, ya,} and \pali{ta} (Sadd\,649), for example:\par
- \pali{\textbf{ki\d m k\=ara\d na\d m} bhagavanta\d m nind\=ama} (Why do we insult the Blessed One?)\par
- \pali{\textbf{Ki\d m} nu \textbf{j\=ati\d m} na rocesi}\footnote{S1\,167 (SN\,5)} (Why don't you like birth?)\par
- \pali{\textbf{Yañ}ca putte na pass\=ami}\footnote{Ja\,22:2223} (Which reason I do not see the children)\par
- \pali{\textbf{Ta\d m ta\d m} gotama pucch\=ami}\footnote{S1\,192 (SN\,7)} (Gotama, I ask [for] that, that reason.)\par

\paragraph*{(4) Instruments} (Sadd\,660)\par 
In rare case nom.\ form can have ins.\ meaning, for example:
- \pali{\textbf{ajjh\=asaya\d m \=adibrahmacariya\d m}}\footnote{D3\,54 (DN\,25)} (by disposition which is the base of religious life)\par

\phantomsection
\addcontentsline{toc}{subsection}{The Second \pali{Vibhatti} (Acc.)}
\subsection*{The Second \pali{Vibhatti} (Acc.)}

When the second (\pali{dutiy\=a}) \pali{vibhatti} is used, accusative case is mostly expected. It mainly marks the direct object in sentences. All functions of this case described by textbooks are shown as follows:

\paragraph*{(1) Direct objects} (Kacc\,297, R\=upa\,284, Sadd\,580, Mogg\,2.2)\par 
- \pali{\textbf{g\=ava\d m} hanati} ([One] kills a cow.)\par
- \pali{\textbf{v\=ihayo} lun\=ati} ([One] reaps paddy.)\par
- \pali{\textbf{sattha\d m} karoti} ([One] make a weapon.)\par
- \pali{\textbf{gha\d ta\d m} karoti} ([One] make a pot.)\par
- \pali{\textbf{dhamma\d m} su\d n\=ati} ([One] listens to the Dhamma.)\par
- \pali{\textbf{buddha\d m} p\=ujeti} ([One] honors the Buddha.)\par
- \pali{\textbf{v\=aca\d m} bh\=asati} ([One] says a speech.)\par
- \pali{\textbf{ta\d n\d dula\d m} pacati} ([One] cooks rice.)\par
- \pali{\textbf{cora\d m} gh\=ateti} ([One] kills a thief.)\par

\paragraph*{(2) Continuity of time and space} (Kacc\,298, R\=upa\,287, Sadd\,581, Mogg\,2.3)\par 
- \pali{\textbf{satt\=aha\d m} gavap\=ana\d m} ([There is] cow milk during seven days.)\par
- \pali{\textbf{m\=asa\d m} ma\d msodana\d m bhu\~njati} ([One] eats boiled-rice with meat during a month.)\par
- \pali{\textbf{sarada\d m} rama\d n\=iy\=a nad\=i} (The river [is] charming during the autumn.)\par
- \pali{\textbf{m\=asa\d m} sajjh\=ayati} ([One] rehearses throughout one month.)\par
- \pali{\textbf{tayo m\=ase} abhidhamma\d m deseti} ([The Buddha] preachs the Abhidhamma throughout three months.)\par
- \pali{\textbf{yojana\d m} vanar\=aji} ([There is] a line of forest throughout one yojana.)\par
- \pali{\textbf{yojana\d m} d\=igho pabbato} ([There is] a mountain one yojana high.)\par
- \pali{\textbf{kosa\d m} sajjh\=ayati} ([One] recites during [a going of] one kosa long.)\par
- \pali{\textbf{kosa\d m} ku\d til\=a nad\=i} ([There is] a river crooked throughout a kosa long.)\par
If there is no continuity, locative case is used instead, for example:\par
- \pali{\textbf{sa\d mvacchare} bhojana\d m bhu\~njati} ([One] eats food in one year.)\par
- \pali{\textbf{m\=ase m\=ase} bhu\~njati}\footnote{This is a way to say `every' or `each' in P\=ali. You just repeat the word.} ([One] eats in every month.)\par
- \pali{\textbf{yojane yojane} vih\=ara\d m pati\d t\d th\=apeti} ([One] has a temple built in every yojana.)\par

\paragraph*{(3) With some prefixes and particles} (Kacc\,299, R\=upa\,288, Sadd\,582--5, Mogg\,2.7--13)\par
Technically, this is called \pali{kammappavacan\=iya}. For some discussion, see Appendix \ref{chap:upasagga} on \pali{anu}, page \pageref{upasagga:anu}.\par
- \pali{\textbf{pabbajita\d m} anupabbaji\d msu} ([People] went forth after the one who having gone forth.)\par
- \pali{\textbf{rukkham}anu vijjotate vijju} (Lightning flashes over a tree.)\par
- \pali{\textbf{nadim}anvavasit\=a b\=ar\=a\d nas\=i} (nearby-rivered Benares)\par
- \pali{\textbf{nadi\d m nera\~njara\d m} pati}\footnote{Snp\,3.427; Thig\,13.307, 310} (nearby Nera\~njar\=a river)\par
- \pali{\textbf{pabbatam}anu sen\=a ti\d t\d thati} (An army is located along the mountain.)\par
- \pali{anu \textbf{s\=ariputta\d m} pa\~n\~nav\=a}\footnote{In Mogg\,2.13, \pali{upa} can be used instead of \pali{anu}.} (a wise one inferior to Ven.\,S\=ar\=iputta)\par
- \pali{\textbf{s\=uriyuggamana\d m} pati; Dibb\=a bhakkh\=a p\=atubhaveyyu\d m}\footnote{Ja\,13:98} (The divine food appears with the rise of the sun.)\par
- \pali{\textbf{rukkha\d m} pati vijjotate cando}\footnote{Like \pali{pati}, \pali{anu}, \pali{pari}, and \pali{abhi} can be used in the same way. This is true in the following too.} (The moon shines over a tree.)\par
- \pali{s\=adhu devadatto \textbf{m\=atara\d m} pati} (Devadatta is good to mother.)\par
- \pali{\textbf{yad}ettha ma\d m pati siy\=a, \textbf{ta\d m} d\=iyatu} (Which is mine, you should give that to me.)\par
- \pali{\textbf{rukkha\d m rukkha\d m} pati vijjotate cando} (The moon shines over every tree.)\par
- \pali{Dhi br\=ahma\d nassa \textbf{hant\=ara\d m}}\footnote{Dhp\,26.389} (That's wrong!, killer of a brahman.)\par
- \pali{Dhiratthu\textbf{ma\d m \=atura\d m p\=utik\=aya\d m}}\footnote{Ja\,3:129} (Disgusting me!, the sick, rotten body.)\par
- \pali{antar\=a ca \textbf{r\=ajagaha\d m} antar\=a ca \textbf{ves\=ali\d m}}\footnote{Mv\,8.346} (between R\=ajagaha and Ves\=al\=i)\par

\paragraph*{(4) With some roots} (Kacc\,300, R\=upa\,286, Sadd\,587, Mogg\,2.4--5)\par
Roots involved here can be used in causatives. Sometimes ins.\ can also be used, thus ``\pali{puriso purisena g\=ama\d m gamayati}.''\par
- \pali{puriso \textbf{purisa\d m} g\=ama\d m gamayati} (A man has [another] man go to the village.)\par
- \pali{puriso \textbf{purisa\d m} dhamma\d m bodhayati} (A man has [another] man know the Dhamma.)\par
- \pali{puriso \textbf{purisa\d m} bhojana\d m bhojayati} (A man has [another] man eat food.)\par
- \pali{puriso \textbf{purisa\d m} dhamma\d m p\=a\d thayati} (A man has [another] man recite the Dhamma.)\par
- \pali{puriso \textbf{purisa\d m} bh\=ara\d m h\=arayati} (A man has [another] man carry a load.)\par
- \pali{puriso \textbf{purisa\d m} kamma\d m k\=arayati} (A man has [another] man do work.)\par
- \pali{puriso \textbf{purisa\d m} say\=apayati} (A man has [another] man sleep.)\par
In Mogg\,2.6 exceptions are mentioned as follows:\par
- \pali{kh\=adayati \textbf{devadattena}} ([One] has Devadatta eat.)\par
- \pali{\=adayati \textbf{devadattena}} ([One] has Devadatta seize.)\par
- \pali{avh\=apayati \textbf{devadattena}} ([One] has Devadatta call.)\par
- \pali{sadd\=ayayati \textbf{devadattena}} ([One] has Devadatta utter.)\par
- \pali{kandayati \textbf{devadattena}} ([One] has Devadatta cry.)\par
- \pali{n\=ayayati \textbf{devadattena}} ([One] has Devadatta lead.)\par

\paragraph*{(5) In genitive sense} (Kacc\,306, R\=upa\,289, Sadd\,588)\par
This involves some terms, i.e.\ \pali{antar\=a, abhito, parito, pati,} and \pali{pa\d tibh\=ati}.\par
- \pali{eka\d m samaya\d m bhagav\=a antar\=a ca \textbf{r\=ajagaha\d m} antar\=a ca \textbf{n\=a\d landa\d m} addh\=anamaggappa\d tipanno hoti}\footnote{D1\,1 (DN\,1)} (In one occasion, the Blessed One was going along the road between R\=ajagaha and N\=a\d land\=a.)\par
- \pali{abhito \textbf{g\=ama\d m} vasati} ([One] lives nearby the village.)\par
- \pali{parito \textbf{g\=ama\d m} vasati} ([One] lives around the village.)\par
- \pali{\textbf{nadi\d m nera\~njara\d m} pati} (nearby river Nera\~njar\=a)\par
- \pali{Apissu\textbf{ma\d m}, aggivessana, tisso upam\=a pa\d tibha\d msu}\footnote{M1\,374 (MN36)} (Aggivessana, three similes came into my mind.)\par

\paragraph*{(6) In instrumental and locative sense} (Kacc\,307, R\=upa\,290, Sadd\,589)\par
- \pali{Sace \textbf{ma\d m} sama\d no gotamo \=alapissati}\footnote{S1\,201 (SN\,7)} (If ascetic Gotama talks with me.)\par
- \pali{tva\~nca \textbf{ma\d m} n\=abhibh\=asasi}\footnote{Ja\,22:2223} (Also you do not talk with me.)\par
- \pali{vin\=a \textbf{saddhamma\d m} kuto sukha\d m} (without the true teaching, whence happiness?)\par
- \pali{\textbf{pubba\d nhasamaya\d m} niv\=asetv\=a}\footnote{Mv\,6.271} (having dressed oneself in the morning)\par
- \pali{\textbf{eka\d m samaya\d m} bhagav\=a}\footnote{D1\,1 (DN\,1)} (in one occasion, the Blessed One)\par
- \pali{\textbf{ima\d m}, bhikkhave, \textbf{ratti\d m} catt\=aro mah\=ar\=aj\=a}\footnote{D3\,285 (DN\,32)} (in this night, monks, the four kings)\par
- \pali{\textbf{Purima\~n}ca disa\d m r\=aj\=a, dhatara\d t\d tho pas\=asati}\footnote{D2\,336 (DN\,20)} (In the east king Dhatara\d t\d tha rules.)\par
- \pali{\textbf{g\=ama\d m} upavasati} ([One] lives in a village.)\par
- \pali{\textbf{g\=ama\d m} anuvasati} ([One] lives in a village.)\par
- \pali{\textbf{vih\=ara\d m} adhivasati} ([One] lives in a temple.)\par
- \pali{\textbf{g\=ama\d m} \=avasati} ([One] lives in a village.)\par
- \pali{\textbf{\=agara\d m} ajjh\=avasati}\footnote{D1\,258 (DN\,3)} ([One] lives in a house.)\par
- \pali{\textbf{pa\d thavi\d m} adhisessati}\footnote{Dhp\,3.41} ([One] lies on the ground.)\par
- \pali{\textbf{g\=ama\d m} adhiti\d t\d thati} ([One] stands in a village.)\par
- \pali{\textbf{nadi\d m} pivati} ([One] drinks in a river.)\par
- \pali{\textbf{g\=ama\d m} carati} ([One] travels in a village.)\par

\paragraph*{(7) As adverbials} (Sadd\,590)\par
- \pali{\textbf{visama\d m} candimas\=uriy\=a parivattanti}\footnote{A4\,70} (The moon and the sun revolve unevenly.)\par
- \pali{\textbf{ekamanta\d m} a\d t\d th\=asi}\footnote{e.g.\ Mv\,10.457} ([One] stood on one side [= properly].)\par
- \pali{Ta\d m su\d n\=ahi, \textbf{s\=adhuka\d m} manasi karohi, bh\=asiss\=ami}\footnote{M1\,367 (MN\,36)} (Listen to that, keep in mind thoroughly, I will say.)\par

\paragraph*{(8) As absolute construction} There are accusative phrases that have no grammatical relation to other part of the sentences. You may see this as adverbial phrase embedded in sentences. Here are some examples:\footnote{Some are suggested in \citealp[p.~315]{perniola:grammar}.}
\begin{quote}
\pali{Ar\=up\=i ca hi te, po\d t\d thap\=ada, att\=a abhavissa sa\~n\~n\=amayo, \textbf{eva\d m santam}pi kho te, po\d t\d thap\=ada, a\~n\~n\=ava sa\~n\~n\=a bhavissati a\~n\~no att\=a.}\footnote{D1\,419 (DN\,9)}\\
``Po\d t\d thap\=ada, the self [you are talking about] were formless, created by perception. [If] this is the case, Po\d t\d thap\=ada, perception will be a thing other than the self.''\\[1.5mm]
\pali{\textbf{Santa\d myeva pana para\d m loka\d m} `natthi paro loko' tissa di\d t\d thi hoti; s\=assa hoti micch\=adi\d t\d thi.}\footnote{M2\,95 (MN\,60)}\\
``[As a matter of fact that] another world exits, he has a view thus `there is no another world.' [Therefore] his [view] is a wrong view.''\\[1.5mm]
\pali{Atha kho br\=ahma\d no pokkharas\=ati \textbf{bhagavanta\d m bhutt\=avi\d m on\=itapattap\=a\d ni\d m} a\~n\~natara\d m n\=ica\d m \=asana\d m gahetv\=a ekamanta\d m nis\=idi.}\footnote{D1\,297 (DN\,3)}\\
``Then Brahman Pokkharas\=ati, [when] the Blessed One who has finished the food and put the hand out of the bowl, having taken another lower seat, sat down on one side.''\\[1.5mm]
\end{quote}

\phantomsection
\addcontentsline{toc}{subsection}{The Third \pali{Vibhatti} (Ins.)}
\subsection*{The Third \pali{Vibhatti} (Ins.)}

This \pali{vibhatti} mainly corresponds to instruments used in the action, thus instrumental case. Also it is an important component of passive structure. It can do other things too.

\paragraph*{(1) As instruments} (Kacc\,286, R\=upa\,291, Sadd\,591, Mogg\,2.16)\par
- \pali{\textbf{aggin\=a} ku\d ti\d m jh\=apeti} ([One] burns a hut with fire.)\par
- \pali{\textbf{k\=ayena} kamma\d m karoti} ([One] does work with the body.)\par

\paragraph*{(2) As the agent in passive structure} (Kacc\,288, R\=upa\,293, Sadd\,594, Mogg\,2.16)\par
- \pali{\textbf{bhagavat\=a} dhammo desiyati}\footnote{The active form of this sentence is ``\pali{bhagav\=a dhamma\d m deseti}'' (The Blessed One preaches the Dhamma).} (By the Blessed One, the Dhamma is preached.)\par
- \pali{\textbf{ahin\=a} da\d t\d tho naro}\footnote{The active form can be ``\pali{ahi nara\d m da\d msi}'' (A snake bit a person).} (By a snake, a person was bitten.)\par

\paragraph*{(3) With \pali{saha}, etc.} (Kacc\,287, R\=upa\,296, Sadd\,592, Mogg\,2.17)\par
- \pali{\textbf{puttena} saha gato} (Together with a son, [one] went.)\par
- \pali{\textbf{puttena} saddhi\d m \=agato} (Together with a son, [one] came.)\par
- \pali{sa\.ngho saha v\=a \textbf{gaggena} vin\=a v\=a \textbf{gaggena} uposatha\d m kareyya}\footnote{Mv\,2.167} (The Sangha, with or without monk Gagga, should do the Uposatha service.)\par
- \pali{bhagav\=a \ldots \=asane nis\=idi, saddhi\d m \textbf{bhikkhusa\. nghena}}\footnote{Mv\,6.276} (The Blessed One \ldots sat on the seat, together with a group of monks.)\par
- \pali{\textbf{sahassena} sama\d m mit\=a}\footnote{S1\,32 (SN\,1)} (measured as 1,000)\par
- \pali{ala\d m te idha \textbf{v\=asena}}\footnote{Buv1\,436} (That's enough for you with the living here.)\par
- \pali{Ala\d m, vakkali, ki\d m te \textbf{imin\=a p\=utik\=ayena di\d t\d thena}}\footnote{S3\,87 (SN\,22)} (That's enough, Vakkali, in what [benefit] for you with the seeing of this rotten body?)\par

\paragraph*{(4) As `together with'} (Sadd\,593)\par
Even without \pali{saha} or \pali{saddhi\d m}, the third \pali{vibhatti} can has such meaning.\par
- \pali{devadatto r\=ajagaha\d m p\=avisi \textbf{kok\=alikena pacch\=asama\d nena}} (Devadatta entered R\=ajagaha together with Kok\=alika as a follower.)\par
- \pali{Dukkho \textbf{b\=alehi} sa\d mv\=aso}\footnote{Dhp\,15.207} (Association with foolish people [is] suffering.)\par

\paragraph*{(5) As causes} (Kacc\,289, R\=upa\,297, Sadd\,601, Mogg\,2.19)\par
- \pali{\textbf{annena} vasati} (Because of food, [one can] lives)\par
- \pali{\textbf{Saddh\=aya} tarati ogha\d m}\footnote{S1\,246 (SN\,10)} (Because of faith, [one can] cross the torrent [of suffering].)\par
- \pali{\textbf{yena} te bhikkh\=u \textbf{ten}upasa\.nkami}\footnote{Buv1\,471} (Because of [the place where] those monks [stay], [\=Ananda] approaches that place.)\par
- \pali{Na \textbf{jacc\=a}\footnote{This is an ins.\ form of j\=ati. See also the declension of \pali{bodhi} in page \pageref{decl:bodhi}.} vasalo hoti}\footnote{Snp\,1.136} ([One] is not an outcaste because of birth.)\par
- \pali{\textbf{kena nimittena}} (Because of what sign?)\par
- \pali{\textbf{kena hetun\=a}} (Because of what reason?)\par
- \pali{\textbf{kena\d t\d thena}} (Because of what benefit.)\par
- \pali{\textbf{kena paccayena}} (Because of what factor?)\par

\paragraph*{(6) As locatives} (Kacc\,290, R\=upa\,298, Sadd\,602)\par
- \pali{\textbf{tena samayena}}\footnote{passim in the Vinaya, e.g.\ Buv1\,1} (in that occasion)\par
- \pali{\textbf{tena k\=alena}}\footnote{Ja\,16:137} (in that time)\par
- \pali{\textbf{k\=alena} dhammassavana\d m}\footnote{A4\,146} (listening to the Dhamma in time [suitable])\par
- \pali{Yo vo, \=ananda, may\=a dhammo ca vinayo ca desito pa\~n\~natto, so vo \textbf{mamaccayena} satth\=a}\footnote{D2\,216 (DN\,16)} (\=Ananda, which teaching and discipline preached and designated by me for you [all], that [will be] your teacher in the time after my death.)\par
- \pali{\textbf{dakkhi\d nena} vir\=u\d lhako}\footnote{D2\,336 (DN\,20)} (in the south, Vir\=u\d lhaka)\par

\paragraph*{(7) As accusatives} (Sadd\,595)\par
- \pali{\textbf{tilehi khette vappati}}\footnote{In this instance, \pali{tilehi} means \pali{til\=ani}. However, Aggava\d msa doubts that it might come from \pali{tile} and particle \pali{hi}. This example is found in R\=upa\,293 as \pali{tilehi khette vapati}.} ([One] sows sesame seeds in the field)\par
- \pali{sa\d mvibhajetha no \textbf{rajjena}}\footnote{D2\,306 (DN\,19)} (Divide the kingdom for us.)\par

\paragraph*{(8) As ablatives} (Sadd\,596)\par
- \pali{Sumutt\=a maya\d m \textbf{tena mah\=asama\d nena}}\footnote{Cv\,11.437} (We are well free from that great ascetic.)\par

\paragraph*{(9) As nominatives} (Sadd\,597)\par
- \pali{\textbf{ma\d nin\=a} me attho}\footnote{Buv1\,344} (The jewel [is] beneficial for me.)\par

\paragraph*{(10) Disabled organs} (Kacc\,291, R\=upa\,299, Sadd\,603, Mogg\,2.18)\par
- \pali{\textbf{akkhin\=a} k\=a\d no} (blind in the eye)\par
- \pali{\textbf{hatthena} ku\d n\=i} (crooked in the hand)\par
- \pali{\textbf{p\=adena} kha\~njo} (lame in the foot)\par
- \pali{\textbf{pi\d t\d thiy\=a khujjo}} (humped in the back)\par

\paragraph*{(11) As adverbials and modifiers} (Kacc\,292, R\=upa\,300, Sadd\,604, Mogg\,2.16)\par
- \pali{Bhagav\=a, m\=aris\=a, khattiyo \textbf{j\=atiy\=a} khattiyakule uppanno}\footnote{D2\,91 (DN\,14)} (The Blessed One, sirs, is of the warrior caste by birth, born in a warrior family)\par
- \pali{\textbf{sippena} na\d lak\=aro so} (By craft, he is a basket maker.)\par
- \pali{Ek\=unati\d mso \textbf{vayas\=a} subhadda}\footnote{D2\,214 (DN\,16)} ([I] am twenty-nine by age, Subhadda.)\par
- \pali{\textbf{vijj\=aya} s\=adhu} (good by knowledge)\par
- \pali{\textbf{tapas\=a} uttamo} (excellent by austerity)\par
- \pali{\textbf{suva\d n\d nena} abhir\=upo} (beautiful by a golden look)\par
- \pali{\textbf{pakatiy\=a} abhir\=upo} (always beautiful)\par
- \pali{\textbf{visamena} dh\=avati} ([One] runs unevenly)\par
- \pali{\textbf{dvido\d nena} dha\~n\~na\d m ki\d n\=ati} ([One] buys grain two do\d nas [$\approx$1/4th of a bushel].)\par

\paragraph*{(12) As signs} (Sadd\,598, Mogg\,2.18)\par
- \pali{\textbf{tida\d n\d dakena} paribb\=ajakamadakkhi} ([One] saw a wandering ascetic by a [sign of] trident.)\par
- \pali{\textbf{setacchattena} r\=aj\=anamadakkhi} ([One] saw a king by a [sing of] white parasol.)\par

\paragraph*{(13) In quick actions} (Sadd\,599)\par
- \pali{\textbf{ekahen}eva b\=ar\=a\d nas\=i p\=ay\=asi} ([One] went to Banares [in] just one day.)\par
- \pali{\textbf{navahi m\=asehi} vih\=ara\d m ni\d t\d th\=apesi} ([One] had a temple built [in] just six months.)\par

\paragraph*{(14) Relation to \pali{pubba}, etc.} (Sadd\,600)\par
- \pali{\textbf{m\=asena} pubbo} (one month before)\par
- \pali{\textbf{pitar\=a} sadiso} (similar to father)\par
- \pali{\textbf{m\=atar\=a} samo} (similar to mother)\par
- \pali{\textbf{kah\=apa\d nen}\=uno} (one lacking money)\par
- \pali{\textbf{asin\=a} kalaho} (a dispute with sword)\par
- \pali{\textbf{\=ac\=arena} nipu\d no} (one elegant by conduct)\par
- \pali{\textbf{tilena} missako} (mixed with sesame seeds)\par
- \pali{\textbf{v\=ac\=aya} sakhilo} (kind with speech)\par

\paragraph*{(15) Relation to \pali{samaya}} (Sadd\,662)\par
As we have seen above, some instances taken from the canon use \pali{samaya} (occasion) in narrations. Typically, in the Vinaya, it takes ins.\ form, e.g.\ \pali{\textbf{tena samayena}} (in that occasion). In the Suttanta, it takes acc.\ form, e.g.\ \pali{\textbf{eka\d m samaya\d m}} (in one occasion). In the Abhidhamma, it takes loc.\ form, e.g.\ \pali{\textbf{yasmi\d m samaye}} (in which occasion). All these have locative meaning.

\phantomsection
\addcontentsline{toc}{subsection}{The Fourth \pali{Vibhatti} (Dat.)}
\subsection*{The Fourth \pali{Vibhatti} (Dat.)}

This \pali{vibhatti} has a close relation to \pali{Sampad\=anak\=araka}. It mainly marks indirect object of the action, particularly giving. It has a couple of uses as shown below, but see also the section on \pali{Sampad\=anak\=araka} above.

\paragraph*{(1) Indirect objects} (Kacc\,293, R\=upa\,301, Sadd\,605, Mogg\,2.24)\par
- \pali{\textbf{buddhassa} d\=ana\d m deti} ([One] gives alms to the Buddha.)\par
- \pali{\textbf{atth\=aya hit\=aya sukh\=aya} manuss\=ana\d m} (for the benefit, welfare, happiness of human beings)\par
- \pali{\textbf{y\=up\=aya} t\=aru} (wood for [building] a sacrificial post)\par
- \pali{n\=ala\d m \textbf{d\=arabhara\d n\=aya}} (not fit for taking care of a wife)\par

\paragraph*{(2) Relation to \pali{namo}, etc.} (Kacc\,294, R\=upa\,305, Sadd\,606)\par
- \pali{Namo \textbf{te} buddha v\=iratthu}\footnote{S1\,90 (SN\,2)} (May the veneration [goes] for you, [my] brave Buddha.)\par
- \pali{sotthi \textbf{janapadassa}}\footnote{D1\,274 (DN\,3)} (May people be blessed.)\par
- \pali{\textbf{te} sv\=agata\d m r\=aja}\footnote{Ja\,19:68} (Your majesty, may the well-coming be for you.)\par

\phantomsection
\addcontentsline{toc}{subsection}{The Fifth \pali{Vibhatti} (Abl.)}
\subsection*{The Fifth \pali{Vibhatti} (Abl.)}

We have met various uses of \pali{Apad\=anak\=araka} in the above section. Some will be repeated here, but the main focus is on the \pali{vibhatti} itself.

\paragraph*{(1) As \pali{Apad\=anak\=araka}} (Kacc\,295, R\=upa\,307, Sadd\,607, Mogg\,2.26)\par
- \pali{\textbf{p\=ap\=a} citta\d m niv\=araye}\footnote{Dhp\,9.116} ([One] should protect the mind from evils.)\par
- \pali{\textbf{bhay\=a} muccati so naro} (That person is free from danger.)\par
- \pali{\textbf{abbh\=a} muttova candim\=a}\footnote{Dhp\,13.172} (Like the moon was free from cloud.)\par

\paragraph*{(2) As causes} (Kacc\,296, R\=upa\,314, Sadd\,608, Mogg\,2.21)\par
- \pali{Catunna\d m, bhikkhave, ariyasacc\=ana\d m \textbf{ananubodh\=a appa\d tive\-dh\=a} evamida\d m d\=ighamaddh\=ana\d m sandh\=avita\d m sa\d msarita\d m mama\~nceva tumh\=aka\~nca}\footnote{D2\,155 (DN\,16)} (Monks, because of not understanding, not penetrating the four noble truths, we have wondered and transmigrated for such a long time.)\par
- \pali{\textbf{Avijj\=apaccay\=a}, bhikkhave, sa\.nkh\=ar\=a}\footnote{S2\,1 (SN\,12)} (Because of ignorance as the cause, monks, conditioned things [arise].)\par

\paragraph*{(3) As source of knowledge} (Sadd\,647)\par
- \pali{\textbf{upajjh\=ay\=a} adh\=ite} ([One] learns from the preceptor.)\par
- \pali{\textbf{upajjh\=ay\=a} su\d noti} ([One] listens from the preceptor.)\par
- \pali{\textbf{Yamh\=a} dhamma\d m vij\=aneyya}\footnote{Dhp\,26.392} (From whom [one] should learn the Dhamma.)\par

\paragraph*{(4) With \pali{k\=ara\d na}, etc.} (Sadd\,648)\par
When \pali{k\=ara\d na} (reason, cause) is accompanied with \pali{ya\d m, ta\d m,} or \pali{ki\d m}, it takes abl., but sometimes acc., e.g.\ \pali{ki\d m k\=ara\d na\d m}. Other term that can denote cause is \pali{nid\=ana}, see examples below. However, in Sadd\,655 another line of thought is proposed. The idioms denoting causes, as shown below plus \pali{ta\d m kissa hetu}, can be seen as indeclinable units. This means the declension of them is not taken into consideration.\par
- \pali{\textbf{ya\d mk\=ara\d n\=a}}\footnote{In Sadd\,653, it is said that \pali{ya\d m, ta\d m,} and \pali{ki\d m} in these idioms take nom., and \pali{k\=ara\d na} takes abl.} (from which reason)\par
- \pali{\textbf{ta\d mk\=ara\d n\=a}} (from that reason)\par
- \pali{\textbf{ki\d mk\=ara\d n\=a}} (from what reason?)\par
- \pali{\textbf{tato}nid\=ana\d m}\footnote{From Sadd\,654, in this instance \pali{-to} marks abl., and \pali{nid\=ana} takes nom.} (from that reason)\par
- \pali{\textbf{yato}nid\=ana\d m} (from which reason)\par

\paragraph*{(5) As instruments with \pali{saha}, etc.} (Sadd\,657)\par
Normally we use ins.\ with \pali{saha, saddhi\d m,} etc., but occasionally we can find the following instances.\par
- \pali{Parinibbute bhagavati saha \textbf{parinibb\=an\=a} brahm\=asahampati ima\d m g\=atha\d m abh\=asi}\footnote{D2\,220 (DN\,16)} (When the Blessed One attained the final release, together with the attaining the Great Brahma said \ldots)\par
- \pali{Ahampi nacirasseva, saddhi\d m \textbf{s\=avakasa\.nghato}; Idheva parin\-ibbissa\d m}\footnote{Bv\,27:22} (Even I, not long, with disciples will die [without being reborn] here.)\par
- \pali{\~N\=atisa\.ngh\=a vin\=a hoti}\footnote{Snp\,3.594} ([One] is departed from relatives.)\par

\paragraph*{(6) As `till' with \pali{y\=ava}}\label{par:abltill} \ \par
When abl.\ is accompanied with \pali{y\=ava}, it means `(un)till' or `up to,' not `from.'\par
- \pali{y\=ava \textbf{mara\d nak\=al\=a}}\footnote{Buv1\,172} (till the time of death)\par
- \pali{Sukha\d m y\=ava \textbf{jar\=a} s\=ila\d m}\footnote{Dhp\,23.333} (Morality [brings] happiness till the old age.)\par
- \pali{y\=ava \textbf{brahmalok\=a} saddo abbhuggacchi}\footnote{Buv1\,36} (The sound rose up to the Brahma world.)\par

\phantomsection
\addcontentsline{toc}{subsection}{The Sixth \pali{Vibhatti} (Gen.)}
\subsection*{The Sixth \pali{Vibhatti} (Gen.)}

We are familiar with this as the possessive marker, but it can denote other things too as described below.

\paragraph*{(1) Possession} (Kacc\,301, R\=upa\,315, Sadd\,609--14, Mogg\,2.39)\par
- \pali{\textbf{tassa bhikkhuno} patto} (the bowl of that monk)\par
- \pali{\textbf{attano} mukha\d m} (one's own face)\par
- \pali{\textbf{ra\~n\~no} dhana\d m} (the king's wealth)\par
- \pali{\textbf{ambavanassa} avid\=ure} ([a place] not far of the mango forest)\par
- \pali{r\=asi \textbf{suva\d n\d nassa}} (a heap of gold)\par
- \pali{sakko \textbf{dev\=anam}indo}\footnote{S1\,247 (SN\,11)} (Sakka the ruler of deities)\par
- \pali{\textbf{ra\~n\~no purohitassa d\=aso}} (a male slave of an religious advisor of the king)\par
- \pali{\textbf{ra\~n\~no} purisena}\footnote{This instance shows that gen.\ can be used with other cases, see Sadd\,613--4.} (by a man of the king)\par

\paragraph*{(2) As instruments and locatives} (Sadd\,635--9, Mogg\,2.40)\par
- \pali{\textbf{ghatassa} aggi\d m yajati}\footnote{This is equal to ``\pali{ghatena aggi\d m yajati.}''} ([One] sacrifies for the fire with ghee.)\par
- \pali{Dh\=iro p\=urati \textbf{pu\~n\~nassa}}\footnote{Dhp\,9.122. This can also be ``\pali{\ldots pu\~n\~nena}.''} (A wise person is full of righteousness.)\par
- \pali{\textbf{pitassa/m\=atuy\=a} tulyo/sadiso}\footnote{This can be ``\pali{pitar\=a/m\=atar\=a \ldots}''} (like father/mother)\par
- \pali{Ki\d m tettha \textbf{catuma\d t\d thassa}}\footnote{Ja\,2:74}\footnote{This can also be ``\pali{\ldots catuma\d t\d thena.}''} (What is the use in that fine four things?)\par
- \pali{kusalo tva\d m rathassa \textbf{a\.ngapacca\.ng\=ana\d m}}\footnote{M2\,87 (MN\,58)} (You are skillful in the major and minor parts of the cart.)\par

\paragraph*{(3) As accusatives and ablatives} (Kacc\,309, R\=upa\,318, Sadd\,640)\par
- \pali{sahas\=a \textbf{kammassa} katt\=aro} (a sudden-action doer)\par
- \pali{\textbf{amatassa} d\=at\=a}\footnote{M1\,203 (MN\,18)} (a deathless-teaching giver)\par
- \pali{\textbf{catunna\d m mah\=abh\=ut\=ana\d m} up\=ad\=aya pas\=ado}\footnote{Dhs\,3:596} (Hanging on to the four great elements, the faculty [exists].)\par
- \pali{\textbf{m\=atu} sarati} ([One] remembers mother)\par
- \pali{Na \textbf{tesa\d m} koci sarati, \textbf{satt\=ana\d m} kammapaccay\=a}\footnote{Khp\,7:2} (Anyone does not remember those beings because of action.)\par
- \pali{\textbf{puttassa} icchati} ([One] wishes for a son.)\par
- \pali{\textbf{ka\d n\d dassa} patikurute} ([One] adjusts an arrow.)\par
- \pali{assavanat\=a \textbf{dhammassa} parih\=ayanti}\footnote{D2\,66 (DN\,14)} ([Ones] fall away from the Dhamma because of not listening.)\par

\paragraph*{(4) Distinction of parts} (Sadd\,615)\par
- \pali{\textbf{gimh\=ana\d m} pacchime m\=ase}\footnote{M1\,263 (MN\,25)} (in the last month of the summer)\par
- \pali{\textbf{vass\=ana\d m} tatiye m\=ase} (in the third month of the rainy season)\par
- \pali{\textbf{kappassa} tatiye bh\=ago} (the third part of the eon)\par

\paragraph*{(5) As unseparated parts} (Sadd\,616)\par
- \pali{\textbf{sil\=aputtassa} sar\=ira\d m} (a small part of a grinding stone)\par
- \pali{P\=as\=a\d nas\=ara\d m kha\d nasi, \textbf{ka\d nik\=arassa} d\=arun\=a}\footnote{Ja\,20:8} (Dig into a stone with a piece of wood.)\footnote{This can be wood from \pali{ka\d nik\=ara} tree. Supaphan Na Bangchang suggests that it is the tools's handle unseparated from it, \citep[p.~327]{supaphan:pali}}\par

\paragraph*{(6) With \pali{chavas\=isa}} (Sadd\,617)\par
- \pali{\textbf{chavas\=isassa} patto}\footnote{Cv\,5.255} (a bowl made of a skull)\par

\paragraph*{(7) Separation of the united} (Sadd\,618)\par
- \pali{\textbf{sandhino} mokkho} (a release from the union)\par

\paragraph*{(8) With \pali{rujati}} (Sadd\,619)\par
- \pali{\textbf{devadattassa} rujati} (Devadatta gets pain.)\par

\paragraph*{(9) Relation to measurement} (Sadd\,620)\par
- \pali{\textbf{til\=ana\d m} mu\d t\d thi} (a handful of sesame seeds)\par
- \pali{\textbf{Sippik\=ana\d m} sata\d m natthi} (There is no 100 of oysters [cowrie shell used as money].)\par

\paragraph*{(10) Relation to indeclinables} (Sadd\,621)\par
- \pali{\textbf{vasalassa} katv\=a} (having done to an outcaste)\par
- \pali{\textbf{bhagavato} purato p\=aturahosi}\footnote{S1\,172 (SN\,6)} ([The Brahma] appears before the Blessed One.)\par
- \pali{\textbf{tassa} pacchato} (behind of that [person])\par
- \pali{\textbf{nagarassa} dakkhito} (south of the city)\par

\paragraph*{(11) Relation to \pali{pada}} (Sadd\,622)\par
- \pali{pam\=ado \textbf{maccuno} pada\d m}\footnote{Dhp\,2.21} (Carelessness [is] a path of death.)\par
- \pali{\textbf{sabbadhamm\=ana\d m} pada\d m s\=ila\d m} (Moral [is] the base of all teaching.)\par

\paragraph*{(12) State of being (\pali{bh\=ava})} (Sadd\,623)\par
- \pali{\textbf{pa\~n\~n\=aya} pa\d tubh\=avo} (the state of skillfulness of wisdom)\par
- \pali{\textbf{r\=upassa} lahut\=a}\footnote{Dhs\,3:584} (lightness of form)\par

\paragraph*{(13) Relation to \pali{hetu}, etc.} (Sadd\,624, 652, Mogg\,2.22)\par
- \pali{\textbf{buddhassa} hetu vasati} ([One] lives because of the Buddha.)\par
- \pali{\textbf{Ekassa} k\=ara\d n\=a mayha\d m hi\d mseyya bahuko jano}\footnote{Ja\,22:1898} (Many people may hurt me because of one person.)\par
- \pali{ta\d m \textbf{kissa} hetu}\footnote{From Sadd\,652, this is equal to ``\pali{kena k\=ara\d nena}.'' Having no meaning, \pali{ta\d m} is just a filler. This instance is used as an idiomatic unit.} (by what reason)\par

\paragraph*{(14) With \pali{ki\d m}} (Sadd\,650)\par
- \pali{Ta\d m \textbf{kissa} hetu}\footnote{M1\,2 (MN\,1)} (by what reason)\par
- \pali{\textbf{kissa} tumhe kilamatha} (Why are you exhausted?)\par

\paragraph*{(15) Relation to \pali{ujjh\=apana}, etc.} (Sadd\,625)\par
- \pali{\textbf{mah\=asen\=apat\=ina\d m} ujjh\=apetabba\d m vikkanditabba\d m viravitabba\d m}\footnote{D3\,282 (DN\,32)} ([One] should complain, shout, cry out [=report] to the great general.)\par
- \pali{\textbf{pa\d tivissak\=ana\d m} ujjh\=apesi}\footnote{M2\,226 (DN\,21)} ([K\=al\=i] complained to the neighbor.)\par
However, sometimes gen.\ is not used, for example, ``\pali{Ujjh\=apetv\=ana bh\=ut\=ani, tamh\=a \d th\=an\=a apakkami}''\footnote{Ja\,22:844} (Having complained to deities, [Somadatta] went away from that place.)\par

\paragraph*{(16) Relation to \pali{Bh\=avas\=adhana}} (Sadd\,626)\par
This is a use with \pali{kita} nouns generated from \pali{yu-paccaya}, etc.\ (see Appendix \ref{chap:kita}).\par
- \pali{\textbf{r\=upassa} upacayo}\footnote{Dhs\,3:584} (accumulation of form)\par
- \pali{\textbf{khandh\=ana\d m} bhedo}\footnote{D2\,390 (DN\,22)} (disunion of the aggregate)\par
- \pali{\textbf{tesa\d m satt\=ana\d m} tamh\=a k\=ay\=a cuti}\footnote{A4\,171} (the passing away from that body of those beings)\par
- \pali{\textbf{\=asav\=ana\d m} khayo}\footnote{S5\,7 (SN\,45)} (destruction of spirits)\par
- \pali{natthi \textbf{n\=as\=aya} r\=uhan\=a}\footnote{Ja\,3:33} (There is no growing of the nose.)\par
- \pali{\textbf{K\=am\=anam}eta\d m nissara\d na\d m yadida\d m nekkhamma\d m}\footnote{It\,72} (This departure from pleasures [is] thus renunciation.)\par

\paragraph*{(17) Relation to \pali{yu, \d nvu, tu}} (Sadd\,627)\par
This use is different from the previous one. The nouns in this case is the object of certain actions.\par
- \pali{moho \textbf{\~neyyass}\=avara\d no} (Stupidity [is] the hindrance of knowledge.)\par
- \pali{\textbf{va\d nass}\=aropana\d m tela\d m} (Oil [is] a wound healer.)\par
- \pali{\textbf{rukkhassa} chedano parasu} (A hatchet [is] a cutting tool of tree.)\par
- \pali{\textbf{Kammassa} k\=arako natthi}\footnote{Vism\,19.689} (There is no actor of the action.)\par

\paragraph*{(18) As objects of fear} (Sadd\,628)\par
This use can be alternatively of acc.\ and ins. Aggava\d msa also has an interesting remark on this. As you may recall, when verb \pali{bh\=ayati} (to fear) is used, it takes abl.\ object (see Chapter \ref{chap:abl}). This is true, he asserts, only when `arising' is implied, for example, ``\pali{yato khema\d m tato bhaya\d m}''\footnote{Ja\,9:58} (From where safety [comes], from that fear [arises]). The verb left out here is \pali{j\=ayati}. In other cases, gen., acc., and ins.\ are mostly found.\par
- \pali{M\=a, bhikkhave, \textbf{pu\~n\~n\=ana\d m} bh\=ayittha}\footnote{It\,22} (Monks, do not be afraid of merit.)\par
- \pali{puriso bh\=ito \textbf{catunna\d m \=as\=ivis\=ana\d m}}\footnote{S4\,238 (SN\,35)} (A person was frightened of four kinds of snakes.)\par
- \pali{\textbf{mus\=av\=adassa} ottapa\d m}\footnote{S1\,184 (SN\,6)} (remorse of telling lies)\par
- \pali{Sabbe tasanti \textbf{da\d n\d dassa}, sabbe bh\=ayanti \textbf{maccuno}}\footnote{Dhp\,10.129} (All beings tremble at punishment, all are frightened of death.)\par

\paragraph*{(19) Grammatical insertion and transformation} (Sadd\,628)\par
This use is found in gramatical textbooks.\par
- \pali{\textbf{puthassa} g\=agamo}\footnote{Sadd\,53} (insertion of \pali{ga} in \pali{putha})\par
- \pali{o \textbf{avassa}}\footnote{Kacc\,50, R\=upa\,45, Sadd\,126} ([change] \pali{ava} to \pali{o})\par

\paragraph*{(20) Relation to \pali{s\=am\=i}, etc. [also loc.]} (Kacc\,303, R\=upa\,321, Sadd\,631)\par
This use is shared with loc., so the examples below will show both of them.\par
- \pali{\textbf{go\d n\=ana\d m/go\d nesu} s\=am\=i} (the master of cattle)\par
- \pali{\textbf{go\d n\=ana\d m/go\d nesu} issaro} (the lord of cattle)\par
- \pali{\textbf{go\d n\=ana\d m/go\d nesu} adhipati} (the ruler of cattle)\par
- \pali{\textbf{go\d n\=ana\d m/go\d nesu} d\=ay\=ado} (an offspring of cattle)\par
- \pali{\textbf{go\d n\=ana\d m/go\d nesu} sakkh\=i} (a witness of cattle)\par
- \pali{\textbf{go\d n\=ana\d m/go\d nesu} patibh\=u} (the master of cattle)\par
- \pali{\textbf{go\d n\=ana\d m/go\d nesu} pasuto} (an expert of cattle)\par
- \pali{\textbf{go\d n\=ana\d m/go\d nesu} kusalo} (skillful in cattle)\par
- \pali{\textbf{atth\=ana\d m/atthesu} kovido} (clever in benefits)\par

\paragraph*{(21) Singling out (\pali{niddh\=ara\d na}) [also loc.]} (Kacc\,304, R\=upa\,322, Sadd\,632, Mogg\,2.36)\par
In Sadd\,632, this use is called \pali{ubb\=ahana}. It can be used both with gen.\ or loc. This seems to be called \emph{partitive genitive} (or locative) in grammatical terms.\footnote{\citealp[p.~31, 36]{collins:grammar}}\par
- \pali{\textbf{manuss\=ana\d m/manussesu} khattiyo s\=uratamo} (Of/in human beings, warrior [is] the bravest.)\par
- \pali{ka\d nh\=a \textbf{g\=av\=ina\d m/g\=av\=isu} sampannakh\=iratam\=a} (Of/in [these] cows, the black one [is] the most productive milker.)\par
- \pali{s\=am\=a \textbf{n\=ar\=ina\d m/n\=ar\=isu} dassan\=iyatam\=a} (Of/in [these] women, S\=am\=a [is] the most beautiful.)\par
- \pali{\textbf{pathik\=ana\d m/pathikesu} dh\=avanto s\=ighatamo} (Of/in pedestrians, the running one [is] the fastest.)\par

\paragraph*{(22) Absolute construction (\pali{an\=adara}) [also loc.]} (Kacc\,305, R\=upa\,323, Sadd\,633, Mogg\,2.35)\par
This use is often found in the texts. It forms a subordinate action that happens simultaneously with the main action. In English it is usually marked by `when' or `while.' This can be used with gen.\ or loc. Technically speaking, this is called \emph{genitive absolute}, or in case of loc., \emph{locative absolute}.\footnote{See \citealp[p.~58, 103]{warder:intro}; \citealp[p.~37, 38]{collins:grammar}. The accusative can also form absolute construction (see above). In these absolutes, locative forms are mostly found, genitive forms are rare, and accusative forms are even rarer (\citealp[p.~336]{perniola:grammar}).}\par
- \pali{\textbf{rudato d\=arakassa/rudantasmi\d m d\=arake} pabbaji} (While the child was crying, [he] went forth.)\par
- \pali{\=Ako\d tayanto te neti, \textbf{sivir\=ajassa pekkhato}}\footnote{Ja\,22:2122} (Hitting those [children], [J\=ujaka] leads them, while king Sivi [Vessantara] is watching.)\par
- \pali{maccu gacchati \=ad\=aya \textbf{pekkham\=ane mah\=ajane}} (Taking [his life], death goes, while people are watching [= He dies before watching people].)\par

\phantomsection
\addcontentsline{toc}{subsection}{The Seventh \pali{Vibhatti} (Loc.)}
\subsection*{The Seventh \pali{Vibhatti} (Loc.)}

We normally use this \pali{vibhatti} to mark a point in space and time, but it can be used in other ways as well.

\paragraph*{(1) In \pali{ok\=asak\=araka}} (Kacc\,302, R\=upa\,319, Sadd\,630, Mogg\,2.32)\par
- \pali{\textbf{gambh\=ire} g\=adhamedhati}\footnote{S1\,200 (SN\,7)} ([One] gets a foothold in deep [water].)\par
- \pali{\textbf{p\=apasmi\d m} ramat\=i mano}\footnote{Dhp\,9.116} (The mind is delighted in evil [deed].)\par
- \pali{\textbf{bhagavati} brahmacariya\d m vussati kulaputto} (A young man practices religious life in [according to] the Buddha.)\par
- \pali{\textbf{ka\d msap\=atiya\d m} bhu\~njati} ([One] eat in a bronze plate.)\par

\paragraph*{(2) As objects, instruments, and signs} (Kacc\,310, R\=upa\,324, Sadd\,641, Mogg\,2.33)\par
- \pali{sundar\=a kho ime, \=avuso, \=aj\=ivak\=a ye ime \textbf{bhikkh\=usu} abhiv\=adenti}\footnote{Buv1\,517} (Venerable, these good wandering ascetics salute to monks.)\par
- \pali{\textbf{Hatthesu} pi\d n\d d\=aya caranti}\footnote{Mv\,1.118} ([Monks] travel for alms with hands.)\par
- \pali{\textbf{pathesu} gacchanti} ([People] go by roads.)\par
- \pali{d\=ipi \textbf{cammesu} ha\~n\~nate} (A panther is killed by the sign [because] of [its] hide.)\par
- \pali{ku\~njaro \textbf{dantesu} ha\~n\~nate}\footnote{In Ja\,22:305, it is ``\pali{n\=ago dantehi ha\~n\~nate}.''} (An elephant is killed by the sign [because] of [its] tusks.)\par

\paragraph*{(3) As indirect objects} (Kacc\,311, R\=upa\,325, Sadd\,642)\par
- \pali{\textbf{sa\.nghe} dinna\d m mahapphala\d m}\footnote{Vv\,634} (A gift to the Sangha [is] very fruitful.)\par
- \pali{\textbf{sa\.nghe}, gotami, dehi}\footnote{M3\,376 (MN\,142)} (Give to the Sangha, Gotam\=i.)\par
- \pali{\textbf{Sa\.nghe} te dinne aha\~nceva p\=ujito bhaviss\=ami}\footnote{M3\,376 (MN\,142)} (When you give to the Sangha, I will also be venerated.)\par

\paragraph*{(4) As ablatives} (Kacc\,312, R\=upa\,326, Sadd\,643)\par
- \pali{\textbf{kadal\=isu} gaje rakkhanti} ([They] prevent elephants from banana trees.)\par

\paragraph*{(5) Time marking} (Kacc\,313, R\=upa\,327, Sadd\,644, Mogg\,2.34)\par
This use has two senses. The first denotes time of the action in general. This is shown by the first two examples. The second marks time of other actions. Technically, this is called \pali{bh\=avalakkha\d na}. This normally forms a kind of a subordinate clause, with help of a verbal \pali{kita}. In English it looks like `when' or `while' clause. This is shown by the third example onwards. This structure is worth noting, because it is found quite often.\par
- \pali{\textbf{pubba\d nhasamaye} gato} ([One] went in the morning.)\par
- \pali{\textbf{s\=aya\d nhasamaye} \=agato} ([One] came in the evening.)\par
- \pali{\textbf{bhikkh\=usu bhojiyam\=anesu} gato} (While monks are eating, [he] has gone.)\par
- \pali{\textbf{bhuttesu} \=agato} (When [monks] had eaten, [he] came.)\par
- \pali{\textbf{gosu duyham\=an\=asu} gato} (When cows is being milked, [he] has gone.)\par
- \pali{\textbf{duddh\=asu} \=agato} (When [cows] had been milked, [he] came.)\par

\paragraph*{(6) Relation to \pali{upa} and \pali{adhi}} (Kacc\,314, R\=upa\,328, Sadd\,645, Mogg\,2.14--5)\par
- \pali{upa \textbf{kh\=ariya\d m} do\d no}\footnote{\pali{kh\=ariy\=a do\d no adhikoti attho.}} (1 Kh\=ar\=i plus 1 do\d na)\par
- \pali{upa \textbf{nikkhe} kah\=apa\d na\d m}\footnote{\pali{nikkhassa kah\=apa\d na\d m adhikanti.}} (1 Nikkha plus 1 Kah\=apa\d na)\par
- \pali{adhi \textbf{devesu} buddho} (The buddha [is] above deities.)\par
- \pali{adhi \textbf{nacce} gotam\=i} (Gotam\=i [is] great in dancing.)\par
- \pali{adhi \textbf{brahmadatte} pa\~nc\=al\=a}\footnote{\pali{brahmadattissar\=a pa\~nc\=al\=ati attho.} It is worth noting that this instance and \pali{adhi devesu buddho} above are reverse in logic, thanks to Costanzo to point this out. I follow a Thai translation of this, so I maintain it as such to respect the source. To P\=ali students, you have to ponder about it and make a judgment. Knowing its context may be helpful.} (Brahmadatta [is] over people of Pa\~nc\=ala.)\par

\paragraph*{(7) Relation to `bright' and `zealous' [also ins.]} (Kacc\,315, R\=upa\,329, Sadd\,646)\par
- \pali{\textbf{\~n\=a\d nena/\~n\=a\d nasmi\d m} pas\=idito} ([One] became bright with/ in wisdom.)\par
- \pali{\textbf{\~n\=a\d nena/\~n\=a\d nasmi\d m} ussukko} ([one] zealous with/in wisdom)\par

\paragraph*{(8) As nominative [also ins.]} (Sadd\,659)\par
- \pali{Idampissa hoti \textbf{s\=ilasmi\d m}}\footnote{D1\,194 (DN\,2)} (Yet, this is a moral action of that [monk])\par

\paragraph*{(9) As instruments} (Sadd\,661)\par
- \pali{\textbf{ma\d nimhi} passa nimmita\d m}\footnote{Ja\,22:1394} (Look at the created with the jewel.)\par

\paragraph*{(10) With \pali{saha}, etc.} (Sadd\,658)\par
- \pali{Saha\textbf{sacce kate} mayha\d m}\footnote{Cp\,3:82} (together with my declaration on oath)\par

\phantomsection
\addcontentsline{toc}{section}{Deviations (\pali{Vipall\=asa})}
\section*{Deviations (\pali{Vipall\=asa})}

Traditional grammarians were not unaware to irregularity in the system they laid down. Once rules are formed, recalcitrant instances are visible. Then they tried to make rules from the oddities, as we have seen that some rules might be formed just to address a single instance found in the canon. But some instances are so strange that an attempt to posit certain rule out of that might destabilize the whole system (rendering that anything goes). So they are better seen as anomalies. They are occasionally found here and there, particularly in verses. Constrained by their meter, verses have fixed structures. To fulfil this condition, some words are intentionaly bent to make them fit the meter. That can explain a cause of deviations, if we see that they are deliberately created, not just an error. This also means poetics trumps the integrity of grammar, from the user's point of view.

In this last section we will learn all deviations recorded by the traditon. In Sadd\,672, six kinds of them are mentioned. I think they are just conspicuous ones. As a matter of fact, they should be much more than these. That is not a big point, however. The real merit of this matter is it reminds us that nothing is perfect. Language is a human enterprise. When used, it grows, it is mixed, and it is changed. That is the very nature of any language. Moreover, peculiarities can come from the medium used. Every time texts are reproduced, errors occur. They can be just faults. I should stress that all you see here are not good examples. Do not ever copy these in your own uses, unless you have a very very good reason.

\paragraph*{(1) Deviation of genders}\ \par
- \pali{sivi \textbf{putt\=ani} avhaya}\footnote{Ja\,22:2235} (King Sivi [Vessandara], please call the children)\par
- \pali{Eva\d m \textbf{dhamm\=ani} sutv\=ana, vippas\=idanti pa\d n\d dit\=a}\footnote{Dhp\,6.82} (Having listened to the teachings, thus wise persons become satisfied.)\par
From the examples above, \pali{putta} (m.) should be \pali{putte}, and \pali{dhamma} (m.) should be \pali{dhamme}, but nt.\ is used instead.

\paragraph*{(2) Deviation of cases}\ \par
- \pali{yo ma\d m gahetv\=ana \textbf{dak\=aya} neti}\footnote{Ja\,6:97} (Grasping me, which person leads me to the water.)\par
- \pali{appo \textbf{sagg\=aya} gacchati}\footnote{Dhp\,13.174} (Few [people] go to heaven.)\par
- \pali{S\=a n\=una kapa\d n\=a amm\=a, \textbf{ciraratt\=aya} rucchati}\footnote{Ja\,22:317} (That mother [Madd\=i] cries miserably throughout the long night.)\par
This three examples use dat.\ in the place of acc. They should be \pali{daka\d m}, \pali{sagga\d m}, and \pali{ciraratti\d m}.\par
- \pali{asakkat\=a casma \textbf{dhana\~njay\=aya}}\footnote{Ja\,4:113} (We were treated with disrespect by King Dhana\~njaya.)\par
- \pali{\textbf{pu\~n\~n\=aya} sugati\d m yanti, \textbf{c\=ag\=aya} vipula\d m dhana\d m}\footnote{Kacc\,275} ([People] reach a happy state by merit, [reach] great wealth by giving up.)\par
This two use dat.\ in the place of ins. They should be \pali{dhana\~njayena}, \pali{pu\~n\~nena}, and \pali{c\=agena}.\par
- \pali{viramath\=ayasmanto mama \textbf{vacan\=aya}}\footnote{Buv1\,425} (Sirs, abstain from words for me [= don't lesson me].)\par
This example uses dat.\ in the place of abl. It should be \pali{vacanato}.\par
- \pali{\textbf{Mah\=aga\d n\=aya} bhatt\=a me}\footnote{Ja\,21:105} (My [Dhatara\d t\d tha] is the leader of the great group [of swans].)\par
This example uses dat.\ in the place of gen. It should be \pali{mah\=aga\d nassa}.\par
- \pali{ko nu kho hetu, ko paccayo bhagavato sitassa \textbf{p\=atukamm\=aya}}\footnote{M2\,282 (MN\,81)} (What is the cause, what is the factor in making visible the smile of the Blessed One?)\par
This example uses dat.\ in the place of loc. It should be \pali{p\=atukamme}.\par

\paragraph*{(3) Deviation of numbers}\ \par
- \pali{\textbf{Najjo} c\=anupariy\=ati, n\=an\=apupphadum\=ayut\=a}\footnote{Ja\,22:529} (A river is surrounded by various flowers.)\par
In this example, \pali{najjo} is plural of \pali{nad\=i} but used as singular.

\paragraph*{(4) Deviation of tenses}\ \par
- \pali{chabbass\=ani n\=ama muggay\=usakulatthay\=usahare\d nuy\=us\=ad\=ina\d m pasa\d tamattena \textbf{y\=apessati}}\footnote{Pps1\,139 (MN-a\,11)} (The Bodhisatta fed himself with a handful of green pea's juice, etc.\ throughout six years.)\par
This example uses future tense (\pali{y\=apessati}) in the narration. It should be past (\pali{y\=apesi}).

\paragraph*{(5) Deviation of persons}\ \par
- \pali{Putta\d m \textbf{labhetha} varada\d m}\footnote{Ja\,22:1661} (May [I] have a son who gives the best thing.)\par
This example has `I' as the subject, so the verb should be \pali{labheyya\d m} or \pali{labheyy\=ami}. See also Sadd\,1099.

\paragraph*{(6) Deviation of letters}\ \par
- \pali{Yath\=a bal\=aka\textbf{yonimhi}, na vijjati pumo sad\=a}\footnote{Ap1\,1:511} (As in gender of cranes, there is always no male.)\par
Since the gender of \pali{yoni} is f., the word should be \pali{yoniya\d m}. This is counted as a deviation of letters.\footnote{Why not gender deviation, I still wonder.} I think this kind of discrepancy is not rare, so in Sadd\,673 there is a rule that in verses f.\ nouns can be in \pali{mhi} form, e.g.\ ``\pali{kus\=avatimhi nagare}''\footnote{Cp\,1:28} (in the city of Kus\=avat\=i). And in Sadd\,674, it is said that in prose in can also be found, e.g.\ \pali{sandhimhi}, \pali{pa\d tisandhimhi}.


