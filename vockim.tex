\chapter{\headhl{Boy, who} are you?}\label{chap:vockim}

As we have learned so far, we cannot yet make a conversation, even a short one. That is because a dialogue has turn taking signaled by interrogation. We have to know how to ask a question first, then we can engage in a conversation. In this chapter we will learn two things. The first is how to address people. This is accomplished by the last case---\emph{vocative}. The second is the widely used question word in P\=ali---the interrogative pronoun \pali{ki\d m}.

\phantomsection
\addcontentsline{toc}{section}{Declension of Vocative Case}
\section*{Declension of Vocative Case}

In P\=ali, as we find in the scriptures, addressing the interlocutor is extensively used. In English, we address people by calling their name, such as Mr./Mrs./Miss/Ms.\ Somebody, usually by their last name for politeness. Other words can also be used to show respect, such as Sir, Madam, Your Excellency, Venerable, Reverend, etc. P\=ali use addressing (\pali{\=alapana}) by two ways, vocative case and some particles. Declension of vocative case is shown in Table \ref{tab:vocreg}. Be careful with those highlighted. Most pronouns in P\=ali have no vocative forms. This means you cannot address people by just calling ``You.''

\begin{table}[!hbt]
\centering
\caption{Vocative case endings of regular nouns}
\label{tab:vocreg}
\bigskip
\begin{tabular}{@{}>{\bfseries}l*{5}{>{\itshape}l}@{}} \toprule
\multirow{2}{*}{G. Num.} & \multicolumn{5}{c}{\bfseries Endings} \\ \cmidrule(l){2-6}
& a & i & \=i & u & \=u \\ \midrule
m. sg. & a & i & \replacewith{\=i}{i} & u & \replacewith{\=u}{u} \\
m. pl. & \replacewith{a}{\=a} & \replacewith{i}{\=i} & \=i & \replacewith{u}{\=u} & \=u \\
& & \replacewith{i}{ayo} & \replacewith{\=i}{ino} & \replacewith{u}{avo} & \replacewith{\=u}{uno} \\
& & & & \texthl{\replacewith{u}{ave}} & \\
\midrule
nt. sg. & a & i & & u & \\
nt. pl. & \replacewith{a}{\=ani} & \replacewith{i}{\=i} & & \replacewith{u}{\=u} & \\
& & \replacewith{i}{\=ini} & & \replacewith{u}{\=uni} & \\
\midrule
& \=a & i & \=i & u & \=u \\
\midrule
f. sg. & \texthl{\replacewith{\=a}{e}} & i & \replacewith{\=i}{i} & u & \replacewith{\=u}{u} \\
f. pl. & \=a & \replacewith{i}{\=i} & \=i & \replacewith{u}{\=u} & \=u \\
& \=ayo & iyo & \replacewith{\=i}{iyo} & uyo & \replacewith{\=u}{uyo} \\
\bottomrule
\end{tabular}
\end{table}

Apart from addressing by vocative case of nouns, some indeclinables are also used likewise. Particles that can be used for vocative function are listed in Table \ref{tab:vocind} (see also Appendix \ref{chap:nipata}, page \pageref{nip:voc}).

\begin{table}[!hbt]
\centering
\caption{Vocative particles}
\label{tab:vocind}
\bigskip
\begin{tabular}{@{}>{\itshape}lcp{0.48\linewidth}@{}} \toprule
\bfseries\upshape Particle & \bfseries Address to & \bfseries Description \\ \midrule
bhante & superiors & Reverend Sir, O Lord \\
bhadante & superiors & Reverend Sir, O Lord \\
bha\d ne & equals or inferiors & \rdelim{\}}{4}{0.45\linewidth}[more polite than the below]\\
ambho & equals or inferiors & \\
hambho & equals or inferiors & \\
\=avuso & equals or inferiors & \\
re & equals or inferiors & \rdelim{\}}{3}{0.45\linewidth}[less polite]\\
are & equals or inferiors & \\
hare & equals or inferiors & \\
he & equals or inferiors & to people, animals and things\\
je & inferiors & to a female servant\\
\bottomrule
\end{tabular}
\end{table}

I also list some words often used, or only used, as vocative in Table \ref{tab:vocother}. The group of \pali{bho} (vocative form of \pali{bhavanta}, see page \pageref{decl:bhavanta}) is general-purpose for addressing human beings. It is a kind of official addressing form preceding voc.\ of nouns as we find in traditional accounts, e.g.\ \pali{bho purisa}.\footnote{There is a discussion on this in Sadd-Pad Ch.\,5.} For things and animals, we use \pali{he} in this case. However, Aggava\d msa explains that \pali{bho} can also be a particle (\pali{nip\=ata}), so it can be used both in sg.\ and pl., also used with f.\ and inanimate things.\footnote{\pali{P\=a\d liya\~nhi a\d t\d thakath\=asu ca nip\=atabh\=uto bhosaddo ekavacanabahuvacanavasena dvidh\=a dissati, \ldots} (Sadd-Pad Ch.\,7).} In this use, \pali{ayyo} can be a voc.\ form in both sg.\ and pl.\footnote{\pali{Ettha ayyo iti saddo paccattavacanabh\=ave ekavacana\d m, \=alapanavacanabh\=ave ekavacana\~nceva bahuvacana\~nca} (Sadd-Pad Ch.\,5).}, while \pali{ayye} is voc.\ of \pali{ayyā} (lady mistress). However, \pali{ayya/ayyā} as voc.\ can be used normally with both genders.

\begin{table}[!hbt]
\centering
\caption{Some other vocative words}
\label{tab:vocother}
\bigskip
\begin{tabular}{@{}>{\itshape}lcp{0.6\linewidth}@{}} \toprule
\bfseries\upshape Voc. & \bfseries G. Num. & \bfseries Description \\ \midrule
bho & m. sg. & \rdelim{\}}{5}{0.6\linewidth}[general terms for addressing people] \\
bhavanto & m. pl. & \\
bhonto & m. pl. & \\
bhoti & f. sg. & \\
bhotiyo & f. pl. & \\
ayyo & m. sg. pl. & Master \\
ayye & f. sg. & My Lady \\
amma & f. sg. & (to a girl, daughter) \\
samma & m. sg. & My Dear (only in voc.) \\
m\=arisa & m. sg. pl. & Sir, Sirs (only in voc.) \\
\bottomrule
\end{tabular}
\end{table}

Here are some examples showing how voc.\ words are used. Some of these also use \pali{ko} or \pali{ka\d m} which will be discussed in the next section. And some others use things that will be learned in future lessons.

\begin{quote}
	\pali{d\=arike, so d\=arako tuyha\d m ida\d m khajjaka\d m dad\=ati.}\\
	``Girl, that boy gives this candy to you.''\\[1.5mm]
	\pali{anuj\=an\=ami, bhikkhave, \=ar\=ama\d m}\footnote{Mv\,1.59}\\
	``[I] allow you, monks, a monastery.''\\[1.5mm]
	\pali{No heta\d m, bhante}\footnote{Mv\,1.21}\\
	``No, it isn't, sir.''\\[1.5mm]
	\pali{Ka\d msi tva\d m, \=avuso, uddissa pabbajito?}\footnote{Mv\,1.11}\\
	``To whom, my dear friend, you went forth?''\\[1.5mm]
	\pali{ko, bha\d ne, gil\=ano, ka\d m tikicch\=ami?}\footnote{Mv\,8.330}\\
	``Who, mister, is sick? Whom I will cure?.''\\[1.5mm]
	\pali{handa, je, ima\d m sappi\d m picun\=a ga\d nh\=ahi}\footnote{Mv\,8.330}\\
	``Here! maid, take up this ghee with the cutton.''\\[1.5mm]
	\pali{katamo pana so, bho \=ananda, ariyo s\=ilakkhandho}\footnote{Dī 1.10.450 (DN 10)}\\
	``What is, Ven.\ \=Ananda, that (group of) noble morality?''\\[1.5mm]
	\pali{nanu tva\d m, ayyo, bhikkh\=usu pabbajito ahosi?}\footnote{Mv\,1.89}\\
	``Were you, Venerable, ordained among the monks?''\\[1.5mm]
	\pali{vejjo, ayye, \=agato; so ta\d m da\d t\d thuk\=amo}\footnote{Mv\,8.330}\\
	``The physician, my lady, has come. He wants to see you.''\\[1.5mm]
	\pali{sace kho tva\d m, ayya, pabbajissasi, eva\d m mayampi pabbajiss\=ama}\footnote{Mv\,1.99}\\
	``If you, my friend, will go forth, we will go forth likewise.''
\end{quote}

\phantomsection
\addcontentsline{toc}{section}{Interrogative Pronoun}
\section*{Interrogative Pronoun}

P\=ali has only one interrogative pronoun---\pali{ki\d m}.\footnote{In dictionaries, this term is often listed as \pali{ka} (see PTSD and \citealp[pp.~600--3]{cone:dict1}). That is right when we treat \pali{ka} as its stem form. But the tradition calls this \pali{ki\d msadda}---the sound \pali{ki\d m} (e.g.\ Sadd\,498).} This can be used in all senses of English question words: \emph{who}, \emph{whom}, \emph{whose}, \emph{what}, \emph{which}, \emph{when}, \emph{where}, \emph{why}, and \emph{how}. The way that \pali{ki\d m} can express various kinds of question is the use of corresponding cases. For example, the question of `whose' clearly asks for gen. The question of time and place can be in loc. But it is not always so, because the destination of the action is marked by acc., whereas the source of the action is marked by abl. Sometimes dat.\ is used if it is about a purpose. The question of `why' and `how' can be seen in line with causal or instrumental expression which can be in abl., ins., or loc.

So, you have to understand the question clearly and match it to a suitable case. Before we see some examples, you have to remember the declension of \pali{ki\d m} as shown in Table \ref{tab:kim}. The cases in the table is ordered as the tradition does. Many forms in the table are repeated, such as m.\ and nt.\ use the same pattern except nom.\ and acc. In all genders, dat.\ and gen.\ use exactly the same forms.

\begin{table}[!hbt]
\centering\small
\caption{Declension of interrogative pronoun}
\label{tab:kim}
\bigskip
\begin{tabular}{@{}l*{6}{>{\itshape}l}@{}} \toprule
\multirow{2}{*}{\bfseries\upshape Case} & \multicolumn{2}{c}{\bfseries\upshape m.} & \multicolumn{2}{c}{\bfseries\upshape f.} & \multicolumn{2}{c}{\bfseries\upshape nt.} \\
\cmidrule(lr){2-3} \cmidrule(lr){4-5} \cmidrule(lr){6-7} 
& \bfseries\upshape sg. & \bfseries\upshape pl. & \bfseries\upshape sg. & \bfseries\upshape pl. & \bfseries\upshape sg. & \bfseries\upshape pl. \\
\midrule
1. nom. & ko & ke & k\=a & k\=a & ki\d m & k\=ani \\
2. acc. & ka\d m & ke & ka\d m & k\=a & ki\d m & k\=ani \\
3. ins. & kena & kehi & k\=aya & k\=ahi & kena & kehi \\
& & kebhi & & k\=abhi & & kebhi \\
4. dat. & kassa & kesa\d m & kass\=a & k\=asa\d m & kassa & kesa\d m \\
& kissa & & & & kissa & \\
5. abl. & kasm\=a & kehi & k\=aya & k\=ahi & kasm\=a & kehi \\
& & kebhi & & k\=abhi & & kebhi \\
6. gen. & kassa & kesa\d m & kass\=a & k\=asa\d m & kassa & kesa\d m \\
& kissa & & & & kissa & \\
7. loc. & kasmi\d m & kesu & kassa\d m & k\=asu & kasmi\d m & kesu \\
& kismi\d m & & & & kismi\d m & \\
\bottomrule
\end{tabular}
\end{table}

When we make a question, we just use this question word in the place of the unknown with corresponding case. For the gender of the question word, if it is known, use the corresponding gender, if not use m.\ for personal agent otherwise nt. Therefore, asking for m.\ nom., ``Boy, who are you?'' can be said as:

\palisample{ko hosi, kum\=ara.}

I put the vocative term at the end to make this agreeable with a typical style---``The vocative case is never put at the beginning of a sentence in the Pali canonical language.''\footnote{\citealp[p.~304]{perniola:grammar}} More often you find the vocative are put near the beginning but not the starter. So, it is more fashionable to say ``\pali{ko, kum\=ara, hosi}.'' However, if you insist to say ``\pali{kum\=ara, ko hosi},'' it is still acceptable for its understandability. You just keep in mind that this is not the way the tradition did it.\footnote{You can also find this in the canon, ``\pali{\=avuso, k\=idisa\d m te bha\d n\d da\d m}'' (see towards the end of this chapter).}

You might be curious why a question mark is not used in the question. Traditionally speaking, P\=ali has no use of that symbol. It is indeed unnecessary. However, in modern P\=ali compilation, question marks are inserted to help the readers. But it is not always so. Then I prefer not to use question marks in my instruction here. This makes students more familiar with textual materials. You have to read from the text, not just rely on a symbol that may mislead you, so to speak. However, in the exercise and other chapters, question marks are used as usual because they really have a great benefit.

In the above example we suppose the interlocutor is a boy. When it is a girl, the question will be ``Girl, who are you?'' Hence we get this:

\palisample{k\=a hosi, kum\=ari.}

Now let us try various ways of questioning. ``Who is going to school?'' also asks for nom.

\palisample{ko p\=a\d thas\=ala\d m gacchati.}

In some situation, nt.\ form is used because we may be asking whether some unknown being are going there, hence ``\pali{ki\d m p\=a\d thas\=ala\d m gacchati}'' (What is going to school?). This sentence is ambiguous because it can also mean ``Which school does he/she go?''\ when \pali{ki\d m} is seen as a pronominal adjective, a modifier of school. So, be careful with this.

This is a question to ask for a name, \pali{n\=ama} (nt.): ``What is your name?''

\palisample{(tuyha\d m) ki\d m n\=ama\d m hoti.}

``What is that man's name?''

\palisample{tassa purisassa ki\d m n\=ama\d m hoti.}

``What is that woman's name?''

\palisample{tass\=a itthiy\=a ki\d m n\=ama\d m hoti.}

Practically, \pali{ki\d m} and \pali{n\=ama} are often found as a compound \pali{ki\d mn\=ama} or \pali{kinn\=ama} (what name), which declines correspondingly to gender of the person, for example, \pali{ki\d mn\=amo} (m.), \pali{ki\d mn\=am\=a} (f.), and \pali{ki\d mn\=ama\d m} (nt.). So, ``What is your name?'' (literally ``You are what name?'') can be said as (for m.):

\palisample{(tva\d m) ki\d mn\=amo hosi.\sampleor[or using \pali{asi}]ki\d mn\=amo asi.\sampleor[or more often in a terse joining form]kinn\=amo'si}

``What is that (woman's) name?''

\palisample{s\=a (itth\=i) ki\d mn\=am\=a hoti.}

``What is that (family's) name?''

\palisample{ta\d m (kula\d m) ki\d mn\=ama\d m hoti.}

Aggava\d msa (Sadd\,459) tells us that the compound can also take the form of \pali{kon\=ama}. So, it is alright to use \pali{kon\=amo}, \pali{kon\=am\=a}, and \pali{kon\=ama\d m} respectively in the above examples.

\pali{N\=amena} can be used as ins.\ in the sense of ``by name.'' So, ``What is your name?'' is equivalent to ``Who are you by name?''

\palisample{ko/k\=a n\=amena hosi.}

In common usage, \pali{n\=ama}\footnote{PTSD says this term takes acc.\ form (see the entry). Childers says it is used as adv.\ \citep[p.~257]{childers:dict}. Cone classifies it as ind.\ \citep[p.~526]{cone:dict2}.} without declension can also be used as an particle to mean ``by name'' or ``called.'' So, ``What is your name?'' or ``What are you called?'' can simply be:

\palisample{ki\d m nama hosi.}

This is a general, quick way to ask for a name. To answer the question, for example, ``My name is \=Ananda'' or literally ``I by name is \=Ananda'' or ``I am called \=Ananda,'' you can say this:

\palisample{(aha\d m) \=Anando n\=ama (homi).}

Or you can put the name in a compound form.

\palisample{(aha\d m) \=Anandan\=amo (homi).}

And this is for ``That is a country called America.''

\palisample{eta\d m America n\=ama ra\d t\d tha\d m hoti.}

This is not a good way to deal with foreign names. Normally, we form a compound to make it end with P\=ali.\footnote{For a treatment of foreign names see Sentence No.\,\ref{conv:bangkok}, page \pageref{conv:bangkok}.} So, it is more suitable to say as follows:

\palisample{eta\d m America-n\=ama\d m ra\d t\d tha\d m hoti.\sampleor eta\d m America-ra\d t\d tha\d m hoti.}

``Whose book is this?'' asks for gen.

\palisample{kassa aya\d m potthako hoti.\sampleor kassa ida\d m potthaka\d m hoti.}

``Whom do you give this book to?'' asks for dat.

\palisample{(tva\d m) kassa ima\d m potthaka\d m desi.}

``For what benefit do you go to school?'' also asks for dat.\ by using \pali{ki\d m} as a pronominal adjective.

\palisample{(tva\d m) kassa hitassa p\=a\d thas\=ala\d m gacchasi.\footnote{This can also be translated as ``For whose benefit do you go to school?''}}

To avoid ambiguity, the above question usually uses \pali{kimatth\=aya} (ind.) instead in the sense of ``for what purpose?'' So, the question should be:

\palisample{(tva\d m) kimatth\=aya p\=a\d thas\=ala\d m gacchasi.}

``Where are you going?'' asks for acc.

\palisample{(tva\d m) ki\d m\footnote{To avoid ambiguity, indeclinables like \pali{kattha} or \pali{katra} or \pali{kuhi\d m} are more often used. See Chapter \ref{chap:ind-to}.} gacchasi.}

``Where do you come from?'' asks for abl.

\palisample{(tva\d m) kasm\=a\footnote{More often, \pali{kuto} (ind.) is used to make this clearer, see Chapter \ref{chap:ind-to}.} \=agacchasi.}

``Where do you live?'' asks for loc.

\palisample{(tva\d m) kasmi\d m vasasi.}

``When do you go to school?'' also asks for loc.

\palisample{(tva\d m) kasmi\d m\footnote{To avoid ambiguity, \pali{kad\=a} (ind.) is often used, see Chapter \ref{chap:ind-to}.} p\=a\d thas\=ala\d m gacchasi.}

``Why do you do this?'' asks for motivation or reason, so we use abl.\ or ins.

\palisample{(tva\d m) kasm\=a ima\d m (kamma\d m) karosi.\sampleor kena ima\d m karosi.}

``With whom do you go to the city?'' asks for ins.

\palisample{(tva\d m) kena saddhi\d m nagara\d m gacchasi.}

``How do you go to school?'' also asks for ins.

\palisample{(tva\d m) kena p\=a\d thas\=ala\d m gacchasi.}

This question can be asked for `why' as well in the sense of ``by what reason.''

Now I will add addressing terms. Let us start with ``Teacher, what are you saying?''

\palisample{ki\d m, \=acariya, kathesi.}

We can combine with addressing particles as:

\palisample{ki\d m, \=acariya bhante, kathesi.}

Practically, there is a sociocultural preference when talking with superiors. Plural verb forms are preferred even if we talk to a single person. So, it is proper to say:

\palisample{ki\d m, \=acariya bhante, kathetha.}

The addressing words, both in ind.\ and voc.\ form, can be used when we are not familiar or know little about the interlocutor. For example, ``Sir/Madam, for what purpose do you come?'' or the common English addressing question ``May I help you?'' can be put in this way:

\palisample{kassa, bho, \=agacchasi. \textup{\normalsize(m.)}\sampleor kassa, bhoti, \=agacchasi. \textup{\normalsize(f.)}}

It is more common to use particle \pali{katha\d m} (why or how) or \pali{kimatth\=aya} (for what purpose) in this context. So, the previous sentence can become:

\palisample{katha\d m, bho(ti), \=agacchasi.\sampleor kimatth\=aya, bho(ti), \=agacchasi.}

The last keyword should be introduced here is \pali{k\=idisa}. It is used to ask a question like `how/what about?' or `what kind?' or `what like?' In conversation we often use this kind of question. The term is formed by primary derivation (see Appendix \ref{chap:kita}, page \pageref{par:kiidisa}). We can use it as an adjective. Let us see some examples from the canon.

\begin{quote}
\pali{K\=idiso tesa\d m vip\=ako, sampar\=ayo ca k\=idiso}\footnote{S1\,49 (SN\,1)}\\
``What kind of their [karmic] result, and what kind of future state?''\\[1.5mm]
\pali{\=avuso, k\=idisa\d m te bha\d n\d da\d m}\footnote{Buv2\,506}\\
``Sir, what does your article look like?''\\[1.5mm]
\pali{k\=idis\=a n\=ama t\=a, ayyaputta, acchar\=ayo y\=asa\d m tva\d m hetu brahmacariya\d m carasi}\footnote{Buv1\,35}\\
``What kind of nymphs, Venerable, do you practice the religious life for?''\\[1.5mm]
\end{quote}

Now let us try this question, ``What kind of book are you reading?''

\palisample{tva\d m k\=idisa\d m potthaka\d m pa\d thasi.}

Another simple way to ask this question is to create a compound with \pali{ki\d m}, hence we can say this also:

\palisample{tva\d m ki\d mpotthaka\d m pa\d thasi.}

However, I do not recommend you to do as such because it makes the question ambiguous, particular when you say it. With one space inserted the meaning of the sentence can be changed. If it is said, instead, ``\pali{ki\d m potthaka\d m pa\d thasi},'' it can mean ``Are you reading a book?'' So, using \pali{k\=idisa} is more suitable.

It seems enough for this chapter. We will learn more about questioning in Chapter \ref{chap:ques}. Do not forget to do our exercise.

\section*{Exercise \ref{chap:vockim}}
Ask these in P\=ali.
\begin{compactenum}
\item Who is the man you talk to?
\item Who is crossing the street?, with who?
\item Where does she buy this thing?
\item By which bus do you go to school?
\item Why do you not go to school today?
\item What do they read that book for?
\item What animal do you fear?
\item Whose friend do you go to the theater with?
\item How your life is going on nowadays?
\item Do you know what your future looks like?
\end{compactenum}
