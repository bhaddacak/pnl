\chapter{Introduction to Traditional Grammar Books}\label{chap:textbook}

In the beginning of our course, I use my own approach to make new students easy to start. Referencing to traditional textbooks is kept minimum at that stage. When the readers go deeper, it is inevitable to refer to traditional works. And they are used heavily in theoretical explanation, particularly in the Appendices. My main purpose to bring traditional textbooks into play is to make students of P\=ali familiar with the sources as much as possible. At the end I hope, all students can consult, or argue if the case might be, the textbooks by their own terms. That is, I think, the best way to learn the language. 

One form of authority comes from accessibility of sacred sources. In this age, anyone can be, and should be, an authority of P\=ali and decide by oneself whether a certain thing is true or not, worthy to believe or not. That is the only way to liberate us from the manipulation of textual monopoly. But becoming any kind of authority needs rigorous study and systematic thinking. I can do only providing you sufficient information. For the rest of the path you have to walk yourselves through.

Since this book is not about traditional approach to P\=ali, not directly at least, it is proper to put this introduction as an appendix. I will introduce the three main schools of P\=ali grammar, namely \emph{Kacc\=ayana}, \emph{Moggall\=ana}, and \emph{Saddan\=iti} school. After these, I also add a lexical work.

\paragraph*{Kacc\=ayanaby\=akara\d na} The oldest of all three schools, Kacc\=ayana provided a model followed by all other schools. The textbook is used in P\=ali courses until today, rigorously in Myanmar, and in a lesser extent in Thailand.\footnote{Before the reformation of ecclesiastic education in 1893, Thai monks learned P\=ali from this book (a rearranged version, to be precise). Then it has been put aside and forgotten for nearly a century. In recent decades the study of the book has been revived, but it is still not the main stream.}

Who is this Kacc\=ayana is a matter of dispute. In traditional view, he is one of the direct disciples of the Buddha, known as Mah\=akacc\=ayana. This renders the book, or parts of it, dates back to the initial time of the religion. Some Western scholars also hold this view. In the introduction of a translation of the book, James d'Alwis writes this:

\begin{quote}
I apprehend, very clear that Kachch\=ayana, the author of \pali{Sandhi-kappa}, was one of the eighty eminent disciples of Gotama. As such, he must have flourished in the latter-half of the sixth century before Christ.\footnote{\citealp[p.~xxx]{dalwis:kach}}
\end{quote}

In line with the traditional view, d'Alwis has a strong belief that P\=ali is M\=aghadh\=i, the language used at the time. So, it is unsurprised to say that many adherents of the religion still believe as such. However, it sounds improbable if we look to the text itself. So, another line of account goes like this: In fact Ven.\,Mah\=akacc\=ayana provided us only the terse formulas, and all other part came from followers of that tradition.\footnote{See `Kacc\=ayana-vy\=akarana' in \emph{Dictionary of P\=ali Proper Names} of G.\,P.\ Malalasekera, available in \textsc{P\=ali\,Platform}.}

Putting faith aside, not trying to make it look pristine unnecessarily, and studying it objectively, modern scholars have a reasonable doubt on that view. First, the book itself is not original in its structure. It is modelled after a Sanskrit grammar book named \emph{K\=atantra}.\footnote{This is mentioned in Malalasekera's dictionary. See also \citealp[p.~163]{norman:literature}.} Some formulas look very close to each other. Second, In Sadd\,833, Aggava\d msa criticizes that in Kacc\,395 the order of huge numbers is not in line with the P\=ali texts. Third, in Kacc\,251, there is an example going like this: ``\pali{Kva gatosi tva dev\=ana\d mpiyatissa}'' (Dev\=anampiyatissa, where did you go?). As you may feel, `Dev\=anampiyatissa' sounds rather Sinhalese than Magadhian.\footnote{In Malalasekera's dictionary, Dev\=anampiyatissa was a king of Ceylon (247--207 BC).} Fourth, in Kacc\,281, an example goes ``\pali{upaguttena m\=aro bandho}'' (The demon was bound by Upagutta). The monk named Upagutta first appeared in Asoka era. All these show it is unlikely that the book dates back to the Buddha's time. It might be of the 5th century\footnote{It is said ``to have been carried into Burma early in the fifth century A.D.'' \citep[p.~622]{law:history}.}, or later in the 10th to 11th century after the period of commentaries, but before the period of subcommentaries.\footnote{K.\,R. Norman believes that this Kacc\=ayana must be later than Buddhaghosa because the name was not mentioned in the Buddhaghosa's works. See \citealp[p.~163]{norman:literature}.} It is evident that terminology used for cases is different in the commentaries and in the textbook. Table \ref{tab:caseterms} shows the differences.\footnote{\citealp[p.~10]{supaphan:pali}}

\begin{table}[!hbt]
\centering
\caption{Grammatical terms for cases}
\label{tab:caseterms}
\bigskip
\begin{tabular}{@{}l*{2}{>{\itshape}l}@{}} \toprule
\bfseries Cases & \bfseries\upshape In the commentaries & \bfseries\upshape In Kacc\=ayana and \\
& & \bfseries\upshape the subcommentaries \\
\midrule
nom. & paccatta & pa\d tham\=a \\
acc. & upayoga & dutiy\=a \\
ins. & kara\d na & tatiy\=a \\
dat. & sampad\=ana & catutth\=i \\
abl. & nissakka & pa\~ncam\=i \\
gen. & s\=ami & cha\d t\d th\=i \\
loc. & bhumma & sattam\=i \\
voc. & \=alapana & \=alapana \\
\bottomrule
\end{tabular}
\end{table}

Whenever the textbook is written does not matter for us the language learners. It is undeniable that its impact is significant. Supaphan Na Bangchang counts the literature related to Kacc\=ayana as follows: 7 are written in India and Sri Lanka, 52 are written in Burma, and 6 are written in Thailand.\footnote{See the list in \citealp[pp.~10--18]{supaphan:pali}.} The most important commentary on Kacc\=ayana is R\=upasiddhi or Padar\=upasiddhi, written by Buddhappiya D\=ipa\.nkara in southern India around the second half of the 13th century.\footnote{\citealp[p.~51]{geiger:literature}} When students say they learn Kacc\=ayana, it normally means they learn R\=upasiddhi altogether, because both are tightly linked by the teaching system. Another one is B\=al\=avat\=ara, written by Dhammakitti in Sri Lanka towards the end of the 14th century. I do not use this one in our course.

\paragraph*{Moggall\=anaby\=akara\d na} From the 12th century, this work was written by a monk named Moggall\=ana in Sri Lanka. The writer also wrote his own commentary named Moggall\=anapa\~ncik\=a. There are eight related works written by followers of this school. A noted one is Payogasiddhi by Vanaratana Medha\d mkara around 1300 A.D. Another recent one is Niruttid\=ipan\=i, written by Le\d d\=i Say\=a\d do in the 19th century. Comparing to Kacc\=ayana, Moggall\=ana is less Sanskritized and has more precise formulas. Geiger says that this work is superior to Kacc\=ayana because the writer ``deals with the linguistic material more exhaustively and with greater understanding of the essence and character of Pali.''\footnote{\citealp[p.~53]{geiger:literature}}

\paragraph*{Saddan\=itippakara\d na} This work is written by Aggava\d msa in Myanmar. Scholars give us that 1154 A.D.\ is the year of the writing. This seems incorrect because the work makes use of R\=upasiddhi extensively, particularly exemplified sentences. Aggava\d msa even mentions it in Sadd-Pad Ch.\,6 as ``\pali{Kacc\=ayanar\=upasiddhiganthesu}'' (In Kacc\=ayana and R\=upasiddhi). If this is the case, Saddan\=iti should be written in the 13th century, after R\=upasiddhi at least.\footnote{The probable year is in between 1234--1250 A.D.\ \citep[see][p.~23]{supaphan:pali}.} There are a few related works of this textbook. This may come from two reasons. First, Saddan\=iti itself is so comprehensive and self-explained that no further commentary is needed. And second, unlike Moggall\=ana, Saddan\=iti does not establish a distinct line of grammatical explanation. It mostly follows Kacc\=ana with its own perspective, even disagreement of Kacc\=ayana is often seen. That is to say, we can logically put Saddan\=iti in the group of Kacc\=ayana's related works. However, with its highly scholarly value and unique characteristic, most scholars set it apart as a different school.

\paragraph*{Abhidh\=anappad\=ipik\=a} Another work often used as a learning resource together with grammatical textbooks is P\=ali dictionary. The oldest one is Abhidh\=anappad\=ipik\=a by another Moggall\=ana, written toward the end of 12th century.\footnote{\citealp[p.~56]{geiger:literature}} The work was composed in verses, 1203 in total. The large part of it deals with synonyms. So, it looks more like a thesaurus. Words are arranged by groups not order. This makes it very difficult to use as a handy reference like modern dictionary. The work is not original. It is modelled after a Sanskrit lexicon named Amarako\'sa. Many words are taken from Sanskrit and converted to P\=ali equivalents. So, they are `artificial' in a way. Thus K.\,R.\ Norman writes this:

\begin{quote}
A proportion of the vocabulary in the Abhidh\=anappadipika is therefore artificial, in the sense that it had no existence in P\=ali until it had been specially coined for inclusion in the dictionary.\footnote{\citealp[p.~167]{norman:literature}}
\end{quote}

\section*{Contents of Grammatical Works}

To make better understanding, now we will look into the contents of the textbooks.

\paragraph*{Contents of Kacc\=ayanaby\=akara\d na} Kacc\=ayana is divided into four parts, i.e.\ alphabets \& sandhi, nouns, verbs, and \pali{kita}. These can be arranged into 8 chapters (\pali{kappa}), 23 sections (\pali{ka\d n\d a}), depicted in Table \ref{tab:contkacc}.\footnote{adapted from \citealp[p.~28]{supaphan:pali}}

\bigskip
\begin{longtable}[c]{@{}>{\raggedright\arraybackslash\small}p{0.15\linewidth}%
	>{\raggedright\arraybackslash\small}p{0.25\linewidth}%
	>{\raggedleft\arraybackslash\small}p{0.1\linewidth}%
	>{\raggedleft\arraybackslash\small}p{0.08\linewidth}%
	>{\raggedleft\arraybackslash\small}p{0.1\linewidth}%
	>{\raggedright\arraybackslash\small}p{0.06\linewidth}}
\caption{Contents of Kacc\=ayana}\label{tab:contkacc}\\
\toprule
\bfseries Part & \bfseries Chapter & \bfseries \mbox{Section} & \multicolumn{2}{c}{\bfseries Sutta} & \\ 
\cmidrule(lr){4-5}
& & & \bfseries 1st & \bfseries Total \\ \midrule
\endfirsthead
\multicolumn{6}{c}{\tablename\ \thetable: Contents of Kacc\=ayana (contd\ldots)}\\
\toprule
\bfseries Part & \bfseries Chapter & \bfseries \mbox{Section} & \multicolumn{2}{c}{\bfseries Sutta} & \\ 
\cmidrule(lr){4-5}
& & & \bfseries 1st & \bfseries Total \\ \midrule
\endhead
\bottomrule
\ltblcontinuedbreak{6}
\endfoot
\bottomrule
\endlastfoot
%
\mbox{Alphabets} & 1. Sandhikappa & 1 & 1 & 11 & \rdelim{\}}{5}{\linewidth}[51] \\
\& Sandhi & & 2 & & 11 & \\
& & 3 & & 7 & \\
& & 4 & & 12 & \\
& & 5 & & 10 & \\
\newpage
Nouns & 2. N\=amakappa & 1 & 52 & 68 & \rdelim{\}}{5}{\linewidth}[219] \\
& & 2 & & 41 & \\
& & 3 & & 50 & \\
& & 4 & & 36 & \\
& & 5 & & 24 & \\
& \mbox{3. K\=arakakappa} & 6 & 271 & 45 & \\
& \mbox{4. Sam\=asakappa} & 7 & 316 & 28 & \\
& \mbox{5. Taddhitakappa} & 8 & 344 & 62 & \\
Verbs & \mbox{6. \=Akhy\=atakappa} & 1 & 406 & 26 & \rdelim{\}}{4}{\linewidth}[118] \\
& & 2 & & 26 & \\
& & 3 & & 24 & \\
& & 4 & & 42 & \\
\pali{Kita} & 7. Kitakappa & 1 & 524 & 26 & \rdelim{\}}{5}{\linewidth}[100] \\
& & 2 & & 21 & \\
& & 3 & & 19 & \\
& & 4 & & 17 & \\
& & 5 & & 17 & \\
& \mbox{8. U\d n\=adikappa} & 6 & 624 & 50 & \\
\midrule
4 & 8 & 23 & & 673 & \\
\end{longtable}

\paragraph*{Contents of Moggall\=anaby\=akara\d na} With a more cryptic naming scheme, Moggall\=ana is divided into 7 chapters (\pali{ka\d n\d da}), namely Sa\~n\~n\=adi, Sy\=adi (\pali{si, etc.}), Sam\=asa, \d N\=adi (\pali{\d na, etc.}), Kh\=adi (\pali{kha, etc.}), Ty\=adi (\pali{ti, etc.}), and \d Nv\=adi (\pali{\d nu, etc.}).\footnote{At first, \d Nv\=adi part, called \pali{\d nv\=adivutti}, is not a part of the book. It is treated as a kind of different book. Without it, the treatise will not be complete, then it is incorporated into the book as a chapter.} I summarize the contents in Table \ref{tab:contmogg}.

\bigskip
\begin{longtable}[c]{@{}>{\raggedright\arraybackslash\small}p{0.17\linewidth}%
	>{\raggedleft\arraybackslash\small}p{0.15\linewidth}%
	>{\raggedright\arraybackslash\small}p{0.55\linewidth}}
\caption{Contents of Moggall\=ana}\label{tab:contmogg}\\
\toprule
\bfseries Chapter & \bfseries Sutta & \bfseries Description \\ 
\midrule
\endfirsthead
\multicolumn{3}{c}{\tablename\ \thetable: Contents of Moggall\=ana (contd\ldots)}\\
\toprule
\bfseries Chapter & \bfseries Suttas & \bfseries Description \\ \midrule
\endhead
\bottomrule
\ltblcontinuedbreak{3}
\endfoot
\bottomrule
\endlastfoot
%
1. Sa\~n\~n\=adi & 58 & Alphabets \& Sandhi, plus \pali{paribh\=as\=a}\\
2. Sy\=adi & 241 & First 39 suttas are about \pali{k\=araka}, the rest 202 suttas are about \pali{n\=ama}. So, partly this is equal to K\=araka- kappa plus N\=amakappa of Kacc\=ayana.\\
3. Sam\=asa & 110 & This is a mixed-up. There are 74 suttas comparable to Sam\=asa- kappa. The rest 36 suttas are about \pali{n\=ama, taddhita,} and \pali{kita}.\\
4. \d N\=adi & 142 & There are 124 suttas comparable to Taddhitakappa. The rest 18 suttas are about \pali{n\=ama}.\\
5. Kh\=adi & 179 & There are 75 suttas comparable to \=Akhy\=atakappa, and 104 suttas comparable to Kitakappa and U\d n\=adikappa.\\
6. Ty\=adi & 78 & All these are about verbal \pali{vibhatti}. So, it should be compare in part with \=Akhy\=atakappa.\\
7. \d Nv\=adi & 229 & This is equivalent to U\d n\=adikappa. \\
\midrule
& 1,037 & \\
\end{longtable}

\paragraph*{Contents of Saddan\=itippakara\d na} This monumental work is divided into three volumes, namely Padam\=al\=a, Dh\=atum\=al\=a, and Suttam\=al\=a. Only the last one can be compared with other work by its structure. I summarize the whole contens of the book in Table \ref{tab:contsadd}.\footnote{adapted from \citealp[pp.~31--3]{supaphan:pali}}

\bigskip
\begin{longtable}[c]{@{}>{\raggedright\arraybackslash\small}p{0.32\linewidth}%
	>{\raggedright\arraybackslash\small}p{0.58\linewidth}}
\caption{Contents of Saddan\=iti}\label{tab:contsadd}\\
\toprule
\bfseries Chapter & \bfseries Description \\ \midrule
\endfirsthead
\multicolumn{2}{c}{\tablename\ \thetable: Contents of Saddan\=iti (contd\ldots)}\\
\toprule
\bfseries Chapter & \bfseries Description \\ \midrule
\endhead
\bottomrule
\ltblcontinuedbreak{2}
\endfoot
\bottomrule
\endlastfoot
%
\multicolumn{2}{c}{1. Padam\=al\=a} \\
\midrule
\multicolumn{2}{l}{1. Savikara\d n\=akhy\=atavibh\=aga} \\
& about root-group \pali{paccaya} and verbal conjugation \\
\multicolumn{2}{l}{2. Bhavatikriy\=apadam\=al\=avibh\=aga} \\
& about verbal conjugation of 8 root-groups \\
\multicolumn{2}{l}{3. Paki\d n\d nakavinicchaya} \\
& miscellaneous terms' explanation \\
\multicolumn{2}{l}{4. Bh\=udh\=atumayan\=amikar\=upavibh\=aga} \\
& about nominal declension of nouns created from \pali{bh\=u} \\
\multicolumn{2}{l}{5. Ok\=arantapulli\.ngan\=amikapadam\=al\=a} \\
& about masculine nouns ending with \pali{o}\\
\multicolumn{2}{l}{6. \=Ak\=arantapulli\.ngan\=amikapadam\=al\=a} \\
& about masculine nouns ending with \pali{\=a}\\
\multicolumn{2}{l}{7. Niggah\=itantapulli\.ngan\=amikapadam\=al\=a} \\
& \mbox{about masculine nouns ending with \pali{\d m}}\\
\multicolumn{2}{l}{8. Itthili\.ngan\=amikapadam\=al\=a} \\
& about feminine nouns \\
\multicolumn{2}{l}{9. Napu\d msakali\.ngan\=amikapadam\=al\=a} \\
& about neuter nouns \\
\multicolumn{2}{l}{10. Li\.ngattayamissakan\=amikapadam\=al\=a} \\
& about gender-mixed nouns \\
\multicolumn{2}{l}{11. V\=acc\=abhidheyyali\.ng\=adiparid\=ipanan\=amikapadam\=al\=a} \\
& about declension of adjectives \\
\multicolumn{2}{l}{12. Sabban\=amata\d msadisan\=aman\=amikapadam\=al\=a} \\
& about declension of pronouns and the like \\
\multicolumn{2}{l}{13. Savinicchayasa\.nkhy\=an\=aman\=amikapadam\=al\=a} \\
& about numerals \\
\multicolumn{2}{l}{14. Atthattikavibh\=aga} \\
& about \pali{bh\=uta}, and terms ending with \pali{tu\d m} and \pali{tv\=a}\\
\midrule
\multicolumn{2}{c}{2. Dh\=atum\=al\=a} \\
\midrule
\multicolumn{2}{l}{15. Saravaggapa\~ncakantika suddhassaradh\=atu} \\
& about all-voweled root and roots ending with a character of the five main groups (\pali{vagga}) \\
\cmidrule{2-2}
& all-voweled root: \pali{i} \\
& root ending with \pali{ka}: \pali{ku}, etc. \\
& root ending with \pali{kha}: \pali{kh\=a}, etc. \\
& root ending with \pali{ga}: \pali{gu}, etc. \\
& root ending with \pali{gha}: \pali{gh\=a}, etc. \\
& root ending with \pali{ca}: \pali{suca}, etc. \\
& root ending with \pali{cha}: \pali{chu}, etc. \\
& root ending with \pali{ja}: \pali{ji}, etc. \\
& root ending with \pali{jha}: \pali{jhe}, etc. \\
& root ending with \pali{\~na}: \pali{\~n\=a}, etc. \\
& root ending with \pali{\d ta}: \pali{so\d ta}, etc. \\
& root ending with \pali{\d tha}: \pali{\d th\=a}, etc. \\
& root ending with \pali{\d da}: \pali{\d di}, etc. \\
& root ending with \pali{\d dha}: \pali{va\d d\d dha}, etc. \\
& root ending with \pali{\d na}: \pali{a\d na}, etc. \\
& root ending with \pali{ta}: \pali{te}, etc. \\
& root ending with \pali{tha}: \pali{th\=a}, etc. \\
& root ending with \pali{da}: \pali{d\=a}, etc. \\
& root ending with \pali{dha}: \pali{dh\=a}, etc. \\
& root ending with \pali{na}: \pali{n\=i}, etc. \\
& root ending with \pali{pa}: \pali{p\=a}, etc. \\
& root ending with \pali{pha}: \pali{puppha}, etc. \\
& root ending with \pali{ba}: \pali{bhabba}, etc. \\
& root ending with \pali{bha}: \pali{bh\=a}, etc. \\
& root ending with \pali{ma}: \pali{m\=a}, etc. \\
\cmidrule{2-2}
\multicolumn{2}{l}{16. Bh\=uv\=adiga\d nikapariccheda} \\
& about roots ending with \pali{avagga} group and unsorted ones \\
\cmidrule{2-2}
& root ending with \pali{ya}: \pali{y\=a}, etc. \\
& root ending with \pali{ra}: \pali{r\=a}, etc. \\
& root ending with \pali{la}: \pali{l\=a}, etc. \\
& root ending with \pali{va}: \pali{v\=a}, etc. \\
& root ending with \pali{sa}: \pali{s\=a}, etc. \\
& root ending with \pali{ha}: \pali{h\=a}, etc. \\
& root ending with \pali{\d la}: \pali{bi\d la}, etc. \\
& unsorted: \pali{h\=u, bh\=u, gamu}, etc. \\
\cmidrule{2-2}
\multicolumn{2}{l}{17. Rudh\=adichakka} \\
& about roots of 6 groups i.e.\ \pali{rudhi}-group, \pali{divu}-group, \pali{su}-group, \pali{k\=i}-group, \pali{gaha}-group, and \pali{tanu}-group \\
\multicolumn{2}{l}{18. Cur\=adiga\d naparid\=ipana} \\
& about roots of \pali{cura}-group and other things \\
\multicolumn{2}{l}{19. Sabbaga\d navinicchaya} \\
& discussion of some terms and roots \\
\cmidrule{2-2}
& totally 1,686 roots mentioned \\
\midrule
\multicolumn{2}{c}{3. Suttam\=al\=a} \\
\midrule
20. Sandhikappa & about alphabets and Sandhi \\
& (191 suttas, started with 1) \\
21. N\=amakappa & about nouns \\
& (355 suttas, started with 192) \\
22. K\=arakakappa & about \pali{k\=araka} (cases) \\
& (128 suttas, started with 547) \\
23. Sam\=asakappa & about compounds \\
& (76 suttas, started with 675) \\
24. Taddhitakappa & about secondary derivation \\
& (114 suttas, started with 751) \\
25. \=Akhy\=atakappa & about verbs \\
& (241 suttas, started with 865) \\
26. Kitakappa & about \pali{kita} and \pali{u\d n\=adi} \\
& (242 suttas, started with 1106) \\
\cmidrule{2-2}
& 1,347 suttas in total \\
\cmidrule{2-2}
27. Catupadavibh\=aga & about terms in 4 groups: nouns, \pali{upasagga} (prefixes), \pali{nip\=ata} (particles), and verbs \\
\mbox{28. P\=a\d linay\=adisa\.ngaha} & about styles in canonical texts, commentaries, subcommentaries, and other texts \\
\end{longtable}

\section*{Structure of a Grammatical Sutta}

Reading grammatical textbooks by yourselves is encouraged for P\=ali students of all levels, despite its difficulty. Even I often have a hard time to read them, or better, to decipher them, but I try nonetheless. I am perhaps more lucky than Western learners, because it is not difficult for me to find a decent translation of the works. Yet, some are hard to find, for example, there is no full translation of Moggall\=ana available to me. I have to grope by myself in that case. I also encourage you to do so.

To ease the learners, I will show you what a sutta in grammatical works looks like. Even though each textbook has its own approach to the language, they all use the same structure as I show in Table \ref{tab:suttastruct}.

\begin{table}[!hbt]
\centering
\caption{Structure of a grammatical sutta}
\label{tab:suttastruct}
\bigskip
\begin{tabular}{@{}>{\raggedright\arraybackslash}p{0.25\linewidth}%
	>{\raggedright\arraybackslash}p{0.6\linewidth}@{}} \toprule
\bfseries Item & \bfseries Description \\
\midrule
1. Formula & The essence of sutta in terse form, often unintelligible to read or understand by itself \\
2. \pali{Vutti} & The explanation of the formula \\
3. \pali{Ud\=ahara\d na} & Examples related to the formula \\
4. \pali{Payoga} & Discussion: additional explanation, analysis, or questions \& answers \\
\bottomrule
\end{tabular}
\end{table}

Now I will show examples of one sutta that explains the same thing across all textbooks, for you can see the comparison. The sutta is about applying \pali{si} over \pali{a}-ending masculine nouns.

\begin{quote}
[From Kacc] \\
104, 66. \pali{So.} $\leftarrow$ \fbox{\small formula} \hspace{11mm}$\downarrow$\ \fbox{\small explanation} \\
\pali{Tasm\=a ak\=arato sivacanassa ok\=ar\=adeso hoti.} \\
\pali{Sabbo, yo, so, ko, amuko, puriso.} $\leftarrow$ \fbox{\small examples} \\
\pali{S\=iti kimattha\d m? Puris\=ana\d m.} $\leftarrow$ \fbox{\small discussion} \\
\pali{Atoti kimattha\d m? Sayambh\=u.} \\[2mm]
(In formula) \\
104 is sutta number in Kacc. \\
66 is related sutta number in R\=upa. \\
\pali{So} is the formula. It is read \pali{si} + \pali{o}. \\[1.5mm]
(In explanation) \\
``Thus, from \pali{a}-ending there is transformation of \pali{si} to \pali{o}.'' \\[1.5mm]
(In examples) \\
``\pali{sabbo} [\pali{sabba + si}] (all), \pali{yo} [\pali{ya + si}] (which), \pali{so} [\pali{ta + si}] (that), \pali{amuko} [\pali{amuka + si}] (over there), \pali{puriso} [\pali{purisa + si}] (a man).'' \\[1.5mm]
(In discussion) \\
Q: ``What [is] \pali{si} for?'' \\
A: ``[To prevent other \pali{vibhatti} from making \pali{o}-ending, such as] \pali{puris\=ana\d m} [\pali{purisa + na\d m}] (of/for men).'' \\
Q: ``What [is] \pali{a} (\pali{ato}) for?'' \\
A: ``[To prevent other endings from becoming \pali{o}, such as] \pali{sayambh\=u} [\pali{sayambh\=u + si}] (the Creator).'' \\
\end{quote}

As you may realize, without any help from previous study of former learners you can go clueless. Then you inevitably have to do some guesswork, as illustrated in the discussion part. That is to say, the given explanations are not always clear, because of their succinct form. Sometimes they are redundant or even senseless (to us). Let us see how Aggava\d msa deals with this material.

\begin{quote}
[From Sadd-Sut] \\
272. \pali{Sissa o.} $\leftarrow$ \fbox{\small formula} \\
\pali{Ak\=arato sivacanassa ok\=aro hoti.} $\leftarrow$ \fbox{\small explanation} \\
\pali{Sabbo, yo, so, ko, amuko, puriso.} $\leftarrow$ \fbox{\small examples} \\
\end{quote}

From the same content, the formula body is changed to be less cryptic and more understandable. The formula \pali{sissa o} can be analyzed to ``\pali{o sissa parassa ato hoti}'' (There is [transformation to] \pali{o} from \pali{a}-ending of \pali{si} [application]). The explanation part looks cleaner. The examples are taken verbatim. And the redundant discussion is removed. In other suttas, Aggava\d msa may give us a lengthy discussion, but for this one it is better to keep quiet because everything is clear enough. Now, let us see the tersest of all.

\begin{quote}
[From Mogg] \\
109. \pali{Sisso.} $\leftarrow$ \fbox{\small formula} \\
\pali{Ak\=arantato n\=amasm\=a sissa o hoti,} $\leftarrow$ \fbox{\small explanation} \\
\pali{buddho,} $\leftarrow$ \fbox{\small examples} \\
\pali{atotveva? Aggi.} $\leftarrow$ \fbox{\small discussion} \\
\end{quote}

You can see different wording used by Moggall\=ana on the same matter. The explanation is readable in this sutta. Examples are reduced to just one. And the discussion part is retained partly, to assert that \pali{a}-ending has differentiating function, say, to tell it apart from \pali{i}-ending. It looks obvious, you may think, why bother?\footnote{This discussion part is not mentioned neither in Payogasiddhi, nor in Niruttid\=ipan\=i.} In general, very short form of formulas and explanations are used in Mogg. Sometimes they are also difficult to decrypt the message because the idiomatic use does not belong to our time.

\section*{Concluding Remarks}

In \emph{P\=ali Literature and Language}, Wilhelm Geiger writes about P\=ali grammatical textbooks as follows:

\begin{quote}
They are not based on the direct know­ledge of P\=ali as a living and spoken language. The authors have drawn their material from the literature just as we too have to do to-day. Their method also is not based on any homogeneous tradi­tion reaching back to the days when P\=ali was actually spoken. Moreover they slavishly imitate the model works of Sanskrit grammar and lexicography and take over their system mechanically into P\=ali. Grammatical forms and words of Pali which are found in the text-books have therefore to be treated with the greater caution so long as they are not proved actually to occur in literature. In all these cases the possibility is ever there that we have before us merely artificial constructions in imitation of Sanskrit.\footnote{\citealp[p.~50]{geiger:literature}} 
\end{quote}

Geiger tells us that P\=ali textbooks do not come from, as the learners should expect, studies done with spoken P\=ali. Instead, the textbooks use textual analysis from existing literature combining with a rework of Sanskrit grammar imposing upon the P\=ali texts. Thus they look `artificial' in Geiger's view.

Let me put in this way. How can new learners learn the language without textbooks in modern languages available? The only option is to learn from its linguistic kin. Many scholars learn P\=ali through Sanskrit. Unlike the traditional way of learning, after having some starter course the language learners study P\=ali texts directly. Which is better between studying P\=ali from pure Sanskrit or studying it with its own texts (with certain influence from Sanskrit)? To my view, learning from P\=ali textbooks are easier and more suitable because they have been `tuned' to some degree. They may be far from perfect, but these are the best we have so far. The only caution I concern is ``do not take the textbooks too seriously.'' It is good to know them all, but not good to believe everything said by them. This is true for all P\=ali texts as well. Texts are an object of our study. Our task is to master them, not to be mastered by them. Apart from knowing texts and their limitation, having a good critical judgement is indispensable characteristic of modern P\=ali scholars.
