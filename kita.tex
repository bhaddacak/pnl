\chapter{\headhl{Kita} (Primary Derivation)}\label{chap:kita}

This is quite a big topic in P\=ali grammar. It is all about word formation. In this appendix, I will describe the topic traditionally. For essential uses of some \pali{kita} verbs, I explain them practically in the lessons. What is this kind of word formation anyway? When we learn about verbs, we know that verbs are created from a root plus some additional parts (\pali{paccaya/vibhatti}). We can call this process roughly `derivation.'\footnote{``The process by which affixes are added to roots and stems to build up new lexical words'', \citealp[pp.~128--9]{brownmiller:dict}.}

When we talk about verbs in P\=ali, we usually mean the main verbs (\pali{\=Akhy\=ata}) that is the essential part of a sentence, even if it can be omitted. That kind of verbs have their process of formation which we have learned gradually from the start of our lessons, and I summarize the principle in Chapter \ref{chap:vform}. Normally we do not call the process of main verb formation as derivation. Therefore, this is not we are going to talk about here, because you have learned a lot of it previously.

Broadly speaking, derivation has two kinds, primary and secondary. Primary derivation operates on roots or stems with sets of \pali{paccaya}s, ending up with verbs and nouns. We usually call these \pali{kita} verbs (\pali{kiriy\=akita}) and \pali{kita} nouns (\pali{n\=amakita}) respectively. That will be explained in detail here. Secondary derivation operates on nouns already derived primarily or secondarily, producing nouns with modified meaning and a number of indeclinables. We call this group \pali{Taddhita}. You can learn about secondary derivation in Appendix \ref{chap:taddhita}.

In P\=ali, derivative process mainly uses suffixes as the instrument. We call these suffixes \pali{paccaya}s (see also Chapter \ref{chap:ind-intro}). The main approach of traditional P\=ali textbooks is to learn how each \pali{paccaya} works. We will learn all of them here. For new students, this can be overwhelming with trivial things. However, in practice there are just a handful of \pali{paccaya}s you have to master, i.e.\ \pali{ta, anta, m\=ana, an\=iya, tabba, tv\=a,} and \pali{tu\d m}. So, you should not be discouraged and try to catch the big things.

Before we embark on our tour, we have to know some preliminaries. First, there are 2--3 forms we have to deal with. Active form (\pali{kattu}) focuses on the agent or doer. Passive form focuses on the target of the action. This has two types in P\=ali: with transitive verbs (\pali{kamma}) and with intransitive verbs (\pali{bh\=ava}). For more detail about these forms, see Chapter \ref{chap:vform}. Technically, we call \pali{paccaya}s for active form \pali{kitapaccaya}\footnote{Kacc\,546, R\=upa\,562, Sadd\,1132; Kacc\,624, R\=upa\,563, Sadd\,1231}, and for passive form \pali{kiccapaccaya}\footnote{Kacc\,545, R\=upa\,548, Sadd\,1131; Kacc\,625, R\=upa\,605, Sadd\,1232}.

Like \pali{Sam\=asa} (compounds), when a new word is formed, you have to explain it with an \emph{analytic sentence} of the term. In textbooks, there are 7 kinds of meaning described by analytic sentences. Technically they are called \pali{s\=adhana}. I will not focus on these much. So I give you the analytic sentence of words only when it is necessary in footnotes. You have to notice by yourselves, if you are curious, which \pali{s\=adhana} is used.

\paragraph*{(1) \pali{Kattus\=adhana}} The terms denote the agent or doer of the action, comparable to nominative case, for example:\par
- \pali{saya\d mbhavat\=iti sayambh\=u} (one who exists by oneself, thus \pali{sayambh\=u}/God).\par
- \pali{dhamma\d m vadati s\=ilen\=ati dhammav\=ad\=i} (one who normally talk the Dhamma, thus a dhamma-talker).\par

\paragraph*{(2) \pali{Kammas\=adhana}} The terms are things done by the action, comparable to accusative case, for example:\par
- \pali{niss\=aya na\d m vasat\=iti nissayo} (a thing on which one live, thus a support)\footnote{This can mean a person such as a teacher.}\par
- \pali{vahitabboti v\=aho} (a thing carried, thus a burden).\par

\paragraph*{(3) \pali{Bh\=avas\=adhana}} The terms denote state of being or verbal nouns, for example:\par
- \pali{gacchiyateti gamana\d m} (a state that one goes, thus a going/journey).\par

\paragraph*{(4) \pali{Kara\d nas\=adhana}} The terms denote instruments used by the action, comparable to instrumental case, for example:\par
- \pali{sarati et\=ay\=ati sati} (one remembers by that, thus mindfulness).\par
- \pali{sa\d mva\d n\d niyati et\=ay\=ati sa\d mva\d n\d nan\=a} ([thing] explained by that, thus exposition).\par

\paragraph*{(5) \pali{Sampadh\=anas\=adhana}} The terms denote indirect objects of the action, comparable to dative case, for example:\par
- \pali{dhana\d m assa bhavat\=uti dhanabh\=uti} (let wealth exist for that one, thus a wealth holder).\par
- \pali{d\=iyate ass\=ati d\=aniyo} (one to whom is given, thus a recipient).\par

\paragraph*{(6) \pali{Apadh\=anas\=adhana}} The terms denote source of the action, comparable to ablative case, for example:\par
- \pali{pa\d thama\d m bhavati etasm\=ati pabhavo} ([thing] originating from this, thus origin/source).\par

\paragraph*{(7) \pali{Adhikara\d nas\=adhana}} The terms denote place where the action takes place, comparable to locative case, for example:\par
- \pali{sayanti etth\=ati sayana\d m} (ones sleep on this, thus a bed).\par
- \pali{pas\=iyati b\=adh\=iyati etth\=ati pacchi} (a thing bound in here, thus a basket).\par

\section*{\pali{Paccaya}s of \pali{Kita}}

Derivative process uses many \pali{paccaya}s to produce words, both primary and secondary kinds. Those are used in primary derivation will be described here. The majority of them generate nouns, and a handful produces verbs. The latter is far more important because they play a significant role in structuring sentences. The former is good to know because they can give us an insight to the meaning of words, but they are not so necessary. In the following sections, all \pali{paccaya}s are described and exemplified. They are grouped in the traditional way. They can be meaning-oriented or function-oriented. A blurry cut between categories can be seen. One \pali{paccaya}, \pali{\d na} for example, can be used in a variety of meaning. The first four groups can be used regardless of time.\footnote{In traditional terms, these can be in three times: past, present, and future (Kacc\,550, R\=upa\,546, Sadd\,1137), for example, \pali{kumbha\d m karoti ak\=asi karissat\=iti kumbhak\=aro} (one makes, made, or will make a pot, thus a potmaker); \pali{karoti ak\=asi karissati ten\=ati kara\d na\d m} (one does, did, or will do with that thing, thus an instrument).}

To help you see the big picture, I list all groups of \pali{paccaya}s in the table below. For the index of them, together with secondary \pali{paccaya}s, you can find in Appendix \ref{chap:paccaya}. On the account of each \pali{paccaya} below, you will find it rather meticulous. One reason for this is that each school has its own way to name \pali{paccaya}s. Sometimes they look very odd and have very specific use. At first you may feel frustrated when you learn all these things. If you do not give up soon, your attempt is indeed rewarding.

\bigskip
\begin{longtable}[c]{@{}%
	>{\raggedleft\arraybackslash}p{0.03\linewidth}%
	>{\raggedright\arraybackslash}p{0.7\linewidth}%
	>{\raggedleft\arraybackslash}p{0.1\linewidth}@{}}
\caption*{Groups of \pali{paccaya}s for \pali{Kita}}\\
\toprule
& \bfseries\upshape Group & \bfseries\upshape Page \\ \midrule
\endfirsthead
\multicolumn{2}{c}{Groups of \pali{paccaya}s for \pali{Kita} (contd\ldots)}\\
\toprule
& \bfseries\upshape Group & \bfseries\upshape Page \\ \midrule
\endhead
\bottomrule
\ltblcontinuedbreak{3}
\endfoot
\bottomrule
\endlastfoot
%
1. & Active \pali{paccaya}s for nouns & \pageref{kita:group1} \\
2. & Active \pali{paccaya}s for nouns of regularity & \pageref{kita:group2} \\
3. & Passive \pali{paccaya}s for verbs & \pageref{kita:group3} \\
4. & Other \pali{paccaya}s for nouns & \pageref{kita:group4} \\
5. & \pali{Paccaya}s for naming & \pageref{kita:group5} \\
6. & \pali{Paccaya}s for feminine nouns & \pageref{kita:group6} \\
7. & \pali{Paccaya}s for infinitives & \pageref{kita:group7} \\
8. & \pali{Paccaya}s for past participles & \pageref{kita:group8} \\
9. & \pali{Paccaya}s for absolutives & \pageref{kita:group9} \\
10. & \pali{Paccaya}s for present participles & \pageref{kita:group10} \\
11. & \mbox{\pali{Paccaya}s for nouns of some particular roots} & \pageref{kita:group11} \\
\end{longtable}

\subsection*{1.\ Active \pali{paccaya}s for nouns}\label{kita:group1}

In Kacc, seven \pali{paccaya}s are mentioned: \pali{\d na, a, \d nvu, tu, \=av\=i, kvi,} and \pali{ra}. In Sadd other two are added: \pali{ro} and \pali{\=a}. Yet \pali{\d n\=i} can also be found in this sense. In Mogg, there are ten of them: \pali{a\d na, a, \d naka, ltu, \=av\=i, kvi, gha\d na, saka, ro,} and \pali{\d nana}.
 
\subparagraph*{\pali{\d Na, a\d na, gha\d na, saka}} (Kacc\,524, 528, R\=upa\,561, 577, Sadd\,1106, 1110, Mogg\,5.41, 5.44, 7.215)\label{pacck1:dna}\label{pacck1:adna}\label{pacck1:ghadna}\label{pacck1:saka}
 
This group of \pali{paccaya} operates on roots which have object of the action in the first part. This results in the doer or maker of that objects. When this occurs to particular roots, the outcomes are abstract or verbal nouns. For a peculiar bahavior of \pali{\d n} component, see a short remark on page \pageref{pacct1:dna}. For more detail, see page \pageref{par:dnapacc}. Here are some examples:

\pali{kamma + kara + \d na} = \palibf{kammak\=ara}\footnote{\pali{kamma\d m karot\=iti kammak\=aro.}} (worker)\par
\pali{kumbha + kara + \d na} = \palibf{kumbhak\=ara} (pot maker)\par
\pali{nagara + kara + \d na} = \palibf{nagarak\=ara}\footnote{\pali{nagara\d m karissat\=iti nagarak\=aro.} This \pali{paccaya} can have future meaning (Kacc\,654, R\=upa\,649, Sadd\,1292), for example, \pali{nagarak\=aro vajati} (one who will build the town goes).} (town builder)\par
\pali{ka\d t\d tha + kara + \d na} = \palibf{ka\d t\d thak\=ara} (timberman)\par
\pali{m\=al\=a + kara + \d na} = \palibf{m\=al\=ak\=ara} (florist)\par
\pali{ratha + kara + \d na} = \palibf{rathak\=ara} (car maker, mechanic)\par
\pali{suva\d n\d na + kara + \d na} = \palibf{suva\d n\d nak\=ara} (goldsmith)\par
\pali{dhamma + kamu + \d na} = \palibf{dhammak\=ama} (one delighted in the Dhamma)\par
\pali{pa + visa + \d na} = \palibf{pavesa}\footnote{\pali{pavissat\=iti paveso.}} (entering)\par
\pali{ruja + \d na} = \palibf{roga}\footnote{In Mogg\,5.44, this is done by \pali{gha\d na}. See also Mogg\,5.98.} (disease)\par
\pali{up + pada + \d na} = \palibf{upp\=ada} (arising)\par
\pali{phusa + \d na} = \palibf{phassa}\footnote{In Mogg\,7.215, this is done by \pali{saka}.} (contact)\par
\pali{bh\=u + \d na} = \palibf{bh\=ava} (being, existing)\par
\pali{sa\d m + budha + \d na} = \palibf{sambodha} (enlightenment)\par
\pali{vi + hara + \d na} = \palibf{vih\=ara} (living)\par

\subparagraph*{\pali{A}} (Kacc\,525, R\=upa\,565, Sadd\,1107)\label{pacck1:a1}

This is used in proper nouns which have an object as the first part. This entails \pali{nu} insertion for some.\footnote{Kacc\,537, R\=upa\,566, Sadd\,1122}

\pali{ari + nu + damu + a} = \palibf{arindama}\footnote{\pali{ari\d m damet\=iti arindamo, r\=aj\=a.}} (Arindama, one taming the enemy)\par
\pali{vessa + nu + tara + a} = \palibf{vessantara}\footnote{\pali{vessa\d m tarat\=iti vessantaro, r\=aj\=a.}} (Vessantara, one crossing the merchant's lane)\par
\pali{pabh\=a + nu + kara + a} = \palibf{pabha\.nkara}\footnote{\pali{pabha\d m karot\=iti pabha\.nkaro, bhgav\=a.}} (Pabha\.nkara, one making light)\par
\pali{pura + d\=a + a} = \palibf{purindada}\footnote{\pali{pured\=ana\d m ad\=as\=iti purindado, r\=aj\=a.} This instance has a dedicated rule, Kacc\,526, R\=upa\,567, Sadd\,1108. See also Mogg\,5.44.} (Purindada, one giving in the past)\par

\subparagraph*{\pali{A, \d nvu, \d naka, tu, ltu, \=av\=i}} (Kacc\,527, R\=upa\,568, Sadd\,1109, Mogg\,5.44, 5.33--4)\label{pacck1:a2}\label{pacck1:dnavu}\label{pacck1:dnaka}\label{pacck1:tu}\label{pacck1:ltu}\label{pacck1:aavii}

This group behaves like above, but they can also be used when the object is absent. From Sanskrit grammar, nouns ending with \pali{tu} are equivalent to \pali{ar} ending, e.g.\ \pali{bh\=asitu = bh\=asitar} (speaker). These are called agent nouns.\footnote{\citealp[p.~209]{warder:intro}} From traditional point of view, we always use \pali{tu} ending when mentioning their stem form, not \pali{ar}.

\pali{ta + kara + a} = \palibf{takkara}\footnote{\pali{ta\d m karot\=iti takkaro.}} (one doing that)\par
\pali{hita + kara + a} = \palibf{hitakkara}\footnote{\pali{hita\d m karot\=iti hitakkaro.}} (one doing beneficial things)\par
\pali{ni + si + a} = \palibf{nissaya}\footnote{\pali{niss\=aya na\d m vasat\=iti nissayo.}} (support, e.g.\ teacher)\par
\pali{ratha + kara + \d nvu} = \palibf{rathak\=araka}\footnote{\pali{ratha\d m karot\=iti rathak\=arako.}} (car maker, mechanic)\par
\pali{anna + d\=a + \d nvu} = \palibf{annad\=ayaka}\footnote{\pali{anna\d m dad\=at\=iti annad\=ayako.}} (one giving food)\par
\pali{kara + \d nvu} = \palibf{k\=araka}\footnote{\pali{karot\=iti k\=arako.} See also Kacc\,622, R\=upa\,570, Sadd\,1228, Mogg\,5.84.} (doer)\par
\pali{kara + \d nvu} = \palibf{k\=araka}\footnote{\pali{karissat\=iti k\=arako.} This can have future meaning (Kacc\,652, R\=upa\,648, Sadd\,1290), for example, \pali{k\=arako vajati} (One who will do goes).} (one who will do)\par
\pali{d\=a + \d nvu} = \palibf{d\=ayaka}\footnote{\pali{dad\=at\=iti d\=ayako.}} (giver)\par
\pali{n\=i + \d nvu} = \palibf{n\=ayaka}\footnote{\pali{net\=iti n\=ayako.}} (leader)\par
\pali{ta + kara + tu} = \palibf{takkattu}\footnote{\pali{ta\d m karot\=iti takkatt\=a.}} (one doing that)\par
\pali{bhojana + d\=a + tu} = \palibf{bhojanad\=atu}\footnote{\pali{bhojana\d m dad\=at\=iti bhojanad\=at\=a.}} (one giving food)\par
\pali{kara + tu} = \palibf{kattu}\footnote{\pali{karot\=iti katt\=a.}} (doer)\par
\pali{sara + tu} = \palibf{saritu}\footnote{\pali{sarat\=iti sarit\=a.}} (rememberer)\par
\pali{bhuja + tu} = \palibf{bhottu}\footnote{Also with \pali{\d nvu}, this can also have future meaning (Kacc\,652, R\=upa\,648, Sadd\,1290), for example, \pali{bhott\=a vajati} (One who will eat goes).} (eater, one who will eat)\par
\pali{bhaya + disa + \=av\=i} = \palibf{bhayadass\=av\=i}\footnote{\pali{bhaya\d m passat\=iti bhayadass\=av\=i.}} (one seeing danger)\par
\pali{pa\d tha + ltu} = \palibf{pa\d thitu}\footnote{Mogg\,5.33} (reciter)\par
\pali{pa\d tha + \d naka} = \palibf{p\=a\d thaka}\footnote{Mogg\,5.33} (reciter)\par

\subparagraph*{\pali{Kvi}} (Kacc\,530, R\=upa\,584, Sadd\,1112, Mogg\,5.47)\label{pacck1:kvi}

When \pali{kvi} is applied, it causes the ending consonant and itself to be deleted (Kacc\,615, R\=upa\,586, Sadd\,1220, Mogg\,5.94; Kacc\,639, R\=upa\,585, Sadd\,1266, Mogg\,5.159).

\pali{sa\d m + bh\=u + kvi} = \palibf{sambh\=u}\footnote{\pali{sambhavat\=iti sambh\=u.}} (self creator, God)\par
\pali{vi + bh\=u + kvi} = \palibf{vibh\=u}\footnote{\pali{visesena bhavat\=iti vibh\=u.}} (exceptional being)\par
\pali{bhuja + gamu + kvi} = \palibf{bhujaga}\footnote{\pali{bhujena gacchat\=iti bhujago.}} (snake, the being that goes by bending)\par
\pali{ura + gamu + kvi} = \palibf{uraga}\footnote{\pali{uras\=a gacchat\=iti urago.}} (snake, the being that goes by the chest)\par
\pali{sa\d m + khanu + kvi} = \palibf{sa\.nkha}\footnote{\pali{sa\d m su\d t\d thu khanat\=iti sa\.nkho.}} (conch, the being that digs well)\par
\pali{loka + vida + kvi} = \palibf{lokavid\=u}\footnote{Kacc\,616, R\=upa\,587, Sadd\,1222. In Mogg this instance is a product of \pali{k\=u}. See below.} (one who knows the world)\par
\pali{masu?\footnote{Roots marked with a question mark like this is questionable, because they do not conform to Sadd-Dh\=a. They may be those that Aggava\d msa overlooked, or they may be in his list but with a different name.} + kvi} = \palibf{macchara/macchera}\footnote{\pali{massat\=iti maccharo.} See Kacc\,630, R\=upa\,654, Sadd\,1239.} (stinginess)\par
\pali{\=a + cara + kvi} = \palibf{acchara/acchera/acchariya}\footnote{\pali{\=a bhuso caritabbanti acchariya\d m.} See Kacc\,631, R\=upa\,655, Sadd\,1240. It is also said in Sadd\,1240 that the term can be counted as a secondary derivation of \pali{acchar\=a} (finger snap). It sounds like the thing is so wonderful that a snap should be given.} (marvel, wonder)\par
\pali{pa\d ti + hi + kvi} = \palibf{p\=a\d tihera/p\=a\d tih\=ira}\footnote{\pali{pa\d tipakkhe madditv\=a gacchati pavattat\=iti p\=a\d tihera\d m, p\=a\d tih\=ira\d m.} See Kacc\,662, R\=upa\,672, Sadd\,1304. In Sadd\,1303, another line of analysis is given: \pali{pa\d tipakkhe harat\=iti p\=a\d tihera\d m, p\=a\d tih\=ira\d m, p\=a\d tih\=ariya\d m.} Hence, the term should come from \pali{pa\d ti + hara + a + iya}, and \pali{p\=a\d tih\=ariya\d m} can also be an outcome.} (miracle)\par

\medskip
The following examples are explained in Kacc\,642, R\=upa\,588, Sadd\,1269.\label{par:kiidisa}

\pali{ima + dusa + kvi} = \palibf{idisa/\=idisa/irasa/\=idikkha/\=irikkha/\=id\=i}\footnote{\pali{imamiva na\d m passat\=iti \=idiso.}} (this kind of person)\par
\pali{ya + dusa + kvi} = \palibf{y\=adisa/y\=arisa/y\=adikkha/y\=ad\=i}\footnote{\pali{yamiva na\d m passat\=iti y\=adiso.}} (which kind of person)\par
\pali{ta + dusa + kvi} = \palibf{t\=adisa/t\=arisa/t\=adikkha/t\=ad\=i} (that kind of person)\par
\pali{amha + dusa + kvi} = \palibf{m\=adisa/m\=arisa/m\=adikkha/m\=ad\=i} (a kind of person like me)\par
\pali{ki\d m + dusa + kvi} = \palibf{k\=idisa/k\=irisa/k\=idikkha/k\=id\=i} (what kind of person?\par
\pali{eta + dusa + kvi} = \palibf{edisa/erisa/edikkha/ed\=i} (this kind of person?)\par
\pali{sam\=ana + dusa + kvi} = \palibf{sadisa/sarisa/sarikkha/s\=adisa/ s\=arisa/s\=adikkha/s\=arikkha/s\=ad\=i}\footnote{In Mogg\,5.125 also \pali{sar\=i, sad\=i,} and \pali{sadikkha} are given, but the process is different, see below.} (the same kind of person)\par

\medskip
However, in Mogg\,5.43 these are products of \pali{r\=i} or \pali{rikkha} or \pali{ka} over root \pali{disa}. The marker \pali{r} (last-syllable killer) and \pali{k} (\pali{vuddhi} preventer) are \pali{anubandha} (see page \pageref{sec:anubandha}). Thus we get as shown above. Furthermore, the process can happen to other bases too as shown below. See also in Mogg\,3.85--90.

\pali{a\~n\~na + disa + r\=i} = \palibf{a\~n\~n\=ad\=i} (other kind of person)\par
\pali{a\~n\~na + disa + rikkha} = \palibf{a\~n\~n\=adikkha} (other kind of person)\par
\pali{a\~n\~na + disa + ka} = \palibf{a\~n\~n\=adisa} (other kind of person)\par
\pali{bh\=u + disa + r\=i} = \palibf{bhav\=ad\=i} (existing kind of person)\par
\pali{bh\=u + disa + rikkha} = \palibf{bhav\=adikkha} (existing kind of person)\par
\pali{bh\=u + disa + ka} = \palibf{bhav\=adisa} (existing kind of person)\par
\pali{tumha + disa + r\=i} = \palibf{ty\=ad\=i} (a kind of person like you)\par
\pali{tumha + disa + rikkha} = \palibf{ty\=adikkha} (a kind of person like you)\par
\pali{tumha + disa + ka} = \palibf{ty\=adisa} (a kind of person like you)\par

\subparagraph*{\pali{Ra}} (Kacc\,538, R\=upa\,595, Sadd\,1123)\label{pacck1:ra}

This \pali{paccaya} has a strange behavior. It change \pali{hana} (kill) to \pali{gha} if preceded by \pali{sa\d m}. When \pali{ra-anubandha} is in operation, it causes the end of the root and itself to be deleted.\footnote{Kacc\,539, R\=upa\,558, Sadd\,1124}

\pali{sa\d m + hana + ra} = \palibf{sa\d mgha}\footnote{\pali{samagga\d m kamma\d m samupagacchati, sammadeva kilesaharathe hant\=iti v\=a sa\d mgho} (ones doing things together, or killing defilement, thus the Sangha). This sounds very specific, perhaps a post hoc explanation. In Mogg\,5.100, this instance and \pali{pa\d tigha} are product of \pali{kvi}.} (the Sangha)\par
\pali{pati + hana + ra} = \palibf{pa\d tigha} (collision, anger)\par
\pali{vi + \=a + hana + ra} = \palibf{byaggha}\footnote{\pali{vividhe satte bhuso hanat\=iti byaggho.}} (tiger)\par
\pali{pari + khanu + ra} = \palibf{parikh\=a}\footnote{\pali{samanatto nagarassa b\=ahire kha\~n\~nt\=iti parikh\=a.}} (ditch, moat)\par
\pali{anta + kara + ra} = \palibf{antaka}\footnote{\pali{anta\d m karot\=iti antako.}} (death, the state the does the end)\par

\subparagraph*{\pali{Ro}} (Sadd\,1115, Mogg\,7.13)\label{pacck1:ro}

\pali{gamu + ro} = \palibf{go}\footnote{\pali{gacchat\=iti go} (a being that goes, thus an ox).} (ox)\par

\subparagraph*{\pali{\=A}} (Sadd\,1116)\label{pacck1:aa}

\pali{su + \=a} = \palibf{s\=a}\footnote{\pali{su\d n\=at\=iti s\=a} (a being that listens, thus a dog).} (dog)\par

\subparagraph*{\pali{\d N\=i}} (Sadd\,1121)\label{pacck1:dnii}

\pali{pa\d n\d dita + mana + \d n\=i} = \palibf{pa\d n\d ditam\=an\=i}\footnote{\pali{pa\d n\d dita\d m att\=ana\d m ma\~n\~nt\=iti pa\d n\d ditam\=an\=i.}} (one recognizing oneself as a wise man)\par
\pali{sattu + gha\d ta + \d n\=i} = \palibf{sattugh\=at\=i} (one killing an enemy)\par
\pali{d\=igha + j\=iva + \d n\=i} = \palibf{d\=ighaj\=iv\=i} (one living long)\par
\pali{dhamma + vada + \d n\=i} = \palibf{dhammav\=ad\=i} (one talking the Dhamma)\par
\pali{s\=iha + nada + \d n\=i} = \palibf{s\=ihan\=ad\=i}\footnote{\pali{s\=iho viya nibbhaya\d m nadat\=iti s\=ihan\=ad\=i.}} (one speaking like a lion roar)\par
\pali{bh\=umi + s\=i + \d n\=i} = \palibf{bh\=umis\=ay\=i}\footnote{\pali{bh\=umiya\d m sayat\=iti bh\=umis\=ay\=i.}} (one lying down on the ground)\par
\pali{k\=ama + bhuja + \d n\=i} = \palibf{k\=amabhog\=i}\footnote{\pali{k\=ame bhu\~njat\=iti k\=amabhog\=i.} In Sadd\,1294, it is said that \pali{\d n\=i} is timeless when used with \pali{gamu}, etc.} (one enjoying pleasure)\par

Apart from marking the agent, \pali{\d n\=i} can also mean `definitely' or `inevitably' or `necessarily' in certain context, for example, \pali{k\=ar\=i asi me kamma\d m avassa\d m} (You are definitely the doer of my work), \pali{h\=ar\=i asi me bh\=ara\d m avassa\d m} (You are definitely the carrier of my burden), \pali{d\=ay\=i asi me sata\d m i\d na\d m} (You are obligatorily my payer of debt of 100), \pali{dh\=ar\=i asi me sahassa\d m i\d na\d m} (You are obligatorily my holder of debt of 1,000).\footnote{Kacc\,636, R\=upa\,659, Sadd\,1245. In the examples, \pali{avassa\d m} (inevitably) is redundant and optional.}

\subparagraph*{\pali{\d Nana}} (Mogg\,4.36--7)\label{pacck1:dnana}

\pali{kara + \d nana} = \palibf{k\=ara\d na}\footnote{\pali{karot\=iti k\=ara\d na\d m.}} (cause)\par
\pali{h\=a + \d nana} = \palibf{h\=ayan\=a/h\=ayana}\footnote{\pali{h\=ayan\=a n\=ama v\=ihayo, h\=ayano sa\d mvaccharo.}} (paddy or year)\par

\subsection*{2.\ Active \pali{paccaya}s for nouns of regularity}\label{kita:group2}

Both Kacc and Sadd give us six: \pali{\d n\=i, tu, \=av\=i, yu, r\=u,} and \pali{\d nuka}. In Mogg, there are five: \pali{\d n\=i, ana, r\=u, k\=u,} and \pali{u}. Furthermore, \pali{\d nu}, \pali{ghi\d n}, and \pali{\=i\d n} are introduced later in Kacc's \pali{U\d n\=adika\d n\d da}. This group of meaning is a bit sloppy because some \pali{paccaya}s also produce the meaning of agency like the previous group.

\subparagraph*{\pali{\d N\=i, tu, \=av\=i}} (Kacc\,532, R\=upa\,590, Sadd\,1114, Mogg\,5.53)\label{pacck2:dnii}\label{pacck2:tu}\label{pacck2:aavii}

\pali{brahma + cara + \d n\=i} = \palibf{brahmac\=ar\=i}\footnote{\pali{brahma\d m caritu\d m s\=ila\d m yassa puggalassa, so hoti puggalo brahmac\=ar\=i.}} (one leading a chaste life)\par
\pali{gamu + \d n\=i} = \palibf{g\=am\=i}\footnote{\pali{\=ayati\d m gamitu\d m s\=ila\d m yassa, so hot\=iti g\=am\=i.} With this root, it has future meaning (Kacc\,651, R\=upa\,647, Sadd\,1289).} (one regularly going further)\par
\pali{bhaja + \d n\=i} = \palibf{bh\=aj\=i} (one regularly sharing)\par
\pali{pasayha + pa + vata + tu} = \palibf{pasayhappavattu}\footnote{\pali{pasayha pavattu\d m s\=ila\d m yassa ra\~n\~no, so hoti r\=aj\=a pasayhappavatt\=a.}} (one who regularly uses force)\par
\pali{bhaya + disa + \=av\=i} = \palibf{bhayadass\=av\=i}\footnote{\pali{bhaya\d m passitu\d m s\=ila\d m yassa, so hoti sama\d no bhayadass\=av\=i.} In Sadd\,1289, \pali{dass\=av\=i} is a product of \pali{\d n\=i}.} (one regularly seeing danger)\par

\subparagraph*{\pali{Yu, ana}} (Kacc\,533, R\=upa\,591, Sadd\,1117, Mogg\,5.48)\label{pacck2:yu}\label{pacck2:ana}

In Mogg, there is no \pali{yu}. Perhaps, it is seen as incomprehensible, so \pali{ana} is used instead.

\pali{ghusa + yu} = \palibf{ghosana}\footnote{\pali{ghosanas\=ilo ghosano.}} (one who regularly shouts, reporter)\par
\pali{bh\=asa + yu} = \palibf{bh\=asana} (one regularly speaking)\par
\pali{kudha + yu} = \palibf{kodhana} (one regularly angry)\par
\pali{ruca + yu} = \palibf{rocana} (one regularly shining)\par
\pali{cala + yu} = \palibf{calana} (one regularly trembling/changing)\par
\pali{va\d d\d dha + yu} = \palibf{va\d d\d dhana} (one regularly growing)\par
\pali{v\=a + yu} = \palibf{v\=ayu}\footnote{\pali{av\=ayi, v\=ayat\=iti v\=ayu.} This instance is not changed to \pali{ana}. It is also said that \pali{yu}, \pali{\d nu}, and \pali{ta} have present and past meaning (Kacc\,650, R\=upa\,651, Sadd\,1288).} (thing regularly going, wind)\par

\subparagraph*{\pali{R\=u, k\=u}} (Kacc\,534--5, R\=upa\,592--3, Sadd\,1118--9, Mogg\,5.38--40, 5.42)\label{pacck2:ruu}\label{pacck2:kuu}

\pali{bhavap\=ara + gamu + r\=u} = \palibf{bhavap\=arag\=u}\footnote{\pali{bhavap\=ara\d m gantu\d m s\=ila\d m yassa purisassa, so hoti puriso bhavap\=arag\=u.} In Mogg\,5.42, when this form of analysis is found, it does not mean regulality, but it signifies an agent, like \pali{kvi}, etc.} (one who regularly goes to the other side of existence)\par
\pali{anta + gamu + r\=u} = \palibf{antag\=u} (one who regularly goes to the end [of suffering])\par
\pali{veda + gamu + r\=u} = \palibf{vedag\=u} (one who regularly goes to knowledge or the Veda)\par
\pali{bhikkha + r\=u} = \palibf{bhikkhu}\footnote{\pali{bhikkhanas\=ilo bhikkhu.} In Mogg\,7.2, this instance is a result of \pali{u}: \pali{Bhikkhat\=iti bhikkhu=sama\d n o}.} (one who regularly begs, monk)\par
\pali{vi + \~n\=a + r\=u/k\=u} = \palibf{vi\~n\~n\=u}\footnote{\pali{vij\=ananas\=ilo vi\~n\~n\=u.} In Mogg\,5.39, this instance is produced by \pali{k\=u} and signifies an agent.} (one who regularly knows)\par
\pali{sabba + \~n\=a + k\=u} = \palibf{sabba\~n\~n\=u} (one who knows all)\par
\pali{k\=ala + \~n\=a + k\=u} = \palibf{k\=ala\~n\~n\=u} (one who knows [proper] time)\par
\pali{vida + k\=u} = \palibf{vid\=u} (one who knows)\par
\pali{loka + vida + k\=u} = \palibf{lokavid\=u} (one who knows the world)\par

\subparagraph*{\pali{\d Nuka}} (Kacc\,536, R\=upa\,594, Sadd\,1120)\label{pacck2:dnuka}

\pali{\=a + hana + \d nuka} = \palibf{\=agh\=atuka}\footnote{\pali{\=ahananas\=ilo \=agh\=atuko.}} (one who regularly kills)\par
\pali{kara + \d nuka} = \palibf{k\=aruka}\footnote{\pali{kara\d nas\=ilo k\=aruko.}} (one who regularly does)\par

\subparagraph*{\pali{\d Nu}} (Kacc\,650, R\=upa\,651, Sadd\,1288)\label{pacck2:dnu}

\pali{kara + \d nu} = \palibf{k\=aru}\footnote{\pali{ak\=asi, karot\=iti k\=aru.}} (maker)\par

\subparagraph*{\pali{Ghi\d n}} (Kacc\,651, R\=upa\,647, Sadd\,1289)\label{pacck2:ghidn}

This \pali{paccaya} has future meaning. See also \pali{\d n\=i} with this root above.

\pali{gamu + ghi\d n} = \palibf{g\=ami}\footnote{\pali{\=ayati\d m gamitu\d m s\=ilamass\=ati g\=ami.}} (one regularly going further)\par

\subparagraph*{\pali{\=I\d na}} (Mogg\,7.11)\label{pacck2:iidna}

This is equivalent to \pali{\d n\=i} of Kacc/Sadd school which has future meaning.

\pali{gamu + \=i\d na} = \palibf{g\=am\=i}\footnote{\pali{gamissat\=iti g\=am\=i gamissam\=ano.}} (one who will go)\par
\pali{bh\=u + \=i\d na} = \palibf{bh\=av\=i} (one who will be)\par

\subsection*{3.\ Passive \pali{paccaya}s for verbs}\label{kita:group3}\label{par:passpaccaya}

In Kacc five are mentioned: \pali{tabba, an\=iya, \d nya, teyya,} and \pali{ricca}. In Sadd \pali{tabya} is added. In Mogg, there are five: \pali{tabba, an\=iya, ghya\d na, ya,} and \pali{yaka}.

\subparagraph*{\pali{Tabba, an\=iya}} (Kacc\,540, R\=upa\,545, Sadd\,1125, Mogg\,5.27)\label{pacck3:tabba}\label{pacck3:aniiya}

This group can be of transtive or intransitive verbs. We has a lesson on these, see Chapter \ref{chap:pass}.

\pali{bh\=u + tabba/an\=iya} = \palibf{bhavitabba/bhavan\=iya} (be been)\par
\pali{pada + tabba/an\=iya} = \palibf{pajjitabba/pajjan\=iya} (be attained)\par
\pali{kara + tabba/an\=iya} = \palibf{kattabba/kara\d n\=iya} (be done)\par
\pali{gamu + tabba/an\=iya} = \palibf{gantabba/gaman\=iya} (be gone)\par

\subparagraph*{\pali{\d Nya, teyya, ghya\d na, ya, yaka}} (Kacc\,541, 544, R\=upa\,552, 556, Sadd\,1126, 1129, Mogg\,5.28--30, 5.32)\label{pacck3:dnya}\label{pacck3:teyya}\label{pacck3:ghyadna}\label{pacck3:ya1}\label{pacck3:yaka}

In Mogg, \pali{ghya\d na} or \pali{ya} or \pali{yaka} is used instead of \pali{\d nya}.

\pali{ji + \d nya} = \palibf{jeyya}\footnote{\pali{jetabba\d m jeyya\d m.}} (be won)\par
\pali{n\=i + \d nya} = \palibf{neyya} (be led)\par
\pali{kara + \d nya} = \palibf{k\=ariya} (be done)\par
\pali{bh\=u + \d nya} = \palibf{bhabba}\footnote{Kacc\,543, R\=upa\,555, Sadd\,1128. \pali{bhavitabbo bhabbo.}} (be been)\par
\pali{\~n\=a + teyya} = \palibf{\~n\=ateyya} (be known)\par
\pali{vada + \d nya} = \palibf{vajja}\footnote{For this and the followings are from Kacc\,544, R\=upa\,556, Sadd\,1129. In Mogg\,5.30, these are products of \pali{ya}.} (be said)\par
\pali{mada + \d nya} = \palibf{majja} (be intoxicated)\par
\pali{gamu + \d nya} = \palibf{gamma} (be gone)\par
\pali{yuja + \d nya} = \palibf{yogga} (be put together)\par
\pali{garaha + \d nya} = \palibf{g\=arayha} (be reproached)\par
\pali{d\=a + \d nya} = \palibf{deyya}\footnote{In Mogg\,5.29, this is produced by \pali{ghya\d na}.} (be given)\par
\pali{p\=a + \d nya} = \palibf{peyya} (be drunk)\par
\pali{h\=a + \d nya} = \palibf{heyya} (be discarded)\par
\pali{m\=a + \d nya} = \palibf{meyya} (be honored)\par
\pali{\~n\=a + \d nya} = \palibf{\~neyya} (be known)\par
\pali{guh\=u + yaka} = \palibf{guyha}\footnote{Mogg\,5.32} (be hidden)\par

\subparagraph*{\pali{Ricca, ya}} (Kacc\,542, R\=upa\,557, Sadd\,1127, Mogg\,5.31)\label{pacck3:ricca}\label{pacck3:ya2}

In Mogg, \pali{ya} is used instead of \pali{ricca}.

\pali{kara + ricca/ya} = \palibf{kicca}\footnote{\pali{k\=atabba\d m kicca\d m.}} (be done)\par

\subparagraph*{\pali{Tapya}} (Sadd\,1130)\label{pacck3:tapya}

Supaphan Na Bangchang notes that this form may actually be \pali{tabba} influenced by Sanskrit.\footnote{\citealp[p.~601]{supaphan:pali}}

\pali{p\=a + tapya} = \palibf{p\=atapya} (be eaten, worth eating)\par

\subsection*{4.\ Other \pali{paccaya}s for nouns}\label{kita:group4}

Unlike above which are mainly of \pali{Kattus\=adhana}, this group has meaning in other \pali{s\=adhana}s. In Kacc and Sadd five \pali{paccaya}s are mentioned: \pali{\d na, ramma, yu, ina,} and \pali{kha}. In Mogg, there are six of them: \pali{gha\d na, ma, ana, naka, ina,} and \pali{a}.

\subparagraph*{\pali{\d Na, gha\d na}} (Kacc\,529, R\=upa\,580, Sadd\,1111, Mogg\,5.44)\label{pacck4:dna}\label{pacck4:ghadna}

In Mogg, \pali{gha\d na} is used instead of \pali{\d na}. This group can have future meaning when used as dative case\footnote{Kacc\,653, R\=upa\,306, Sadd\,1291}, for example, \pali{p\=ak\=aya vajati} (He/she goes for cooking), \pali{bhog\=aya vajati} (He/she goes for eating). The products of \pali{\d na} as verbal nouns are of masculine gender (Sadd\,1346).

\pali{paca + \d na} = \palibf{p\=aka}\footnote{\pali{paccate pacana\d m v\=a p\=ako.}} (be cooked, cooking)\par
\pali{caja + \d na} = \palibf{c\=aga} (be given up, giving up)\par
\pali{bh\=u + \d na} = \palibf{bh\=ava} (be been, being)\par
\pali{yaja + \d na} = \palibf{y\=aga} (be honored, honoring)\par
\pali{yuja + \d na} = \palibf{yoga} (be put together, putting together)\par
\pali{bhaja + \d na} = \palibf{bh\=aga} (be associated, asociation)\par
\pali{bhuja + \d na} = \palibf{bhoga} (be eaten, eating)\par

\subparagraph*{\pali{Ramma, ma}} (Kacc\,531, R\=upa\,589, Sadd\,1113, Mogg\,7.136)\label{pacck4:ramma}\label{pacck4:ma}

In Mogg, \pali{ma} is used instead of \pali{ramma}.

\pali{dh\=a + ramma} = \palibf{dhamma}\footnote{\pali{yath\=anusi\d t\d tha\d m pa\d tipajjam\=ane cat\=usu ap\=ayesu apatam\=ane satte dh\=aret\=iti dhammo, dharati ten\=ati v\=a dhammo.}} (Dhamma, the state that holds or keeps [the practitioners from unwholesomeness])\par
\pali{kara + ramma} = \palibf{kamma}\footnote{\pali{kar\=iyate tanti kamma\d m.}} (work)\par

\subparagraph*{\pali{Yu}} (Kacc\,547--8, R\=upa\,596--7, Sadd\,1133--4)\label{pacck4:yu}

As said above, In Mogg this is regarded as \pali{ana} not \pali{yu}.

\pali{nanda + yu} = \palibf{nandana}\footnote{\pali{nand\=iyate nandana\d m} or \pali{nanditabba\d m nandana\d m.}} (rejoicing)\par
\pali{gaha + yu} = \palibf{gaha\d na}\footnote{After \pali{ha} and \pali{ra}, \pali{na} becomes \pali{\d na} (Kacc\,549, R\=upa\,550, Sadd\,1135, Mogg\,5.171) But \pali{vagahana, udakagahana, kalalagahana} stay unchanged (Sadd\,1136, see also Mogg\,5.172).} (taking)\par
\pali{cara + yu} = \palibf{cara\d na} (behaving)\par
\pali{raja + hara + yu} = \palibf{rajohara\d na}\footnote{\pali{raja\d m harat\=iti rajohara\d na\d m.}} (thing removing dust, water)\par
\pali{kara + yu} = \palibf{kara\d na}\footnote{\pali{karoti ten\=ati kara\d na\d m.}} (thing by which one does, tool)\par
\pali{\d th\=a + yu} = \palibf{\d th\=ana}\footnote{\pali{ti\d t\d thanti tasminti \d th\=ana\d m.}} (place on which people stand, standing point, status)\par

\subparagraph*{\pali{Ina, naka}} (Kacc\,558--9, R\=upa\,602--3, Sadd\,1145--6, Mogg\,7.102--3, 7.105)\label{pacck4:ina}\label{pacck4:naka}

\pali{ji + ina} = \palibf{jina}\footnote{\pali{p\=apake akusale dhamme jin\=ati ajini jinissat\=iti jino.} In Mogg\,7.105, this instance is a product of \pali{naka}.} (winner, the Buddha who wins unwholesome natures)\par
\pali{supa + ina} = \palibf{supina}\footnote{\pali{supat\=iti supino} or \pali{supiyate supina\d m.} In Mogg\,7.103, this is also a product of \pali{ina}.} (sleeper, sleeping)\par
\pali{aja + ina} = \palibf{ajina}\footnote{This is from Mogg\,7.102.} (animal hide)\par

\subparagraph*{\pali{Kha, a}} (Kacc\,560, R\=upa\,604, Sadd\,1147, Mogg\,5.44)\label{pacck4:kha}\label{pacck4:a}

In Mogg, \pali{a} is used instead of \pali{kha}.

\pali{\=isa\d m + s\=i + kha} = \palibf{\=isassaya} (little slept)\par
\pali{du + s\=i + kha} = \palibf{dussaya} (difficultly slept)\par
\pali{su + s\=i + kha} = \palibf{sussaya} (easily slept)\par
\pali{\=isa\d m + kara + kha} = \palibf{\=isakkara} (little done)\par
\pali{du + kara + kha} = \palibf{dukkara} (difficultly done)\par
\pali{su + kara + kha} = \palibf{sukara} (easily done)\par

\subsection*{5.\ \pali{Paccaya}s for naming}\label{kita:group5}

This group results in nouns, some are abstract naming, some are proper names. In Kacc and Sadd, \pali{i} and \pali{ti} are mentioned, in Mogg \pali{i} and \pali{aka}.

\subparagraph*{\pali{I}} (Kacc\,551, R\=upa\,598, Sadd\,1138, Mogg\,5.45)\label{pacck5:i}

\pali{\=a + d\=a + i} = \palibf{\=adi}\footnote{\pali{pa\d thama\d m \=ad\=iyat\=iti \=adi.}} (beginning, thing taken first)\par
\pali{udaka + dh\=a + i} = \palibf{udadhi}\footnote{\pali{udaka\d m dadh\=at\=iti udadhi.}} (ocean, place holding water)\par
\pali{sa\d m + \=a + dh\=a + i} = \palibf{sam\=adhi}\footnote{\pali{samm\=a sama\d m v\=a citta\d m \=adadh\=at\=iti sam\=adhi.}} (concentration, state that keeps the mind right or even)\par

\subparagraph*{\pali{Ti}} (Kacc\,552, R\=upa\,609, Sadd\,1139)\label{pacck5:ti}

Some verbal \pali{paccaya}s, such as \pali{ta, m\=ana}, can also be used in this meaning.

\pali{jina + budha + ti} = \palibf{jinabuddhi}\footnote{\pali{jino ena\d m bujjhat\=uti jinabuddhi} (May the Buddha knows this one, thus \pali{Jinabuddhi}).} (Jinabuddhi)\par
\pali{dhana + bh\=u + ti} = \palibf{dhanabh\=uti}\footnote{\pali{dhana\d m assa bhavat\=uti dhanabh\=uti} (May wealth be of this one, thus \pali{Dhanabh\=uti}).} (Dhanabh\=uti)\par
\pali{dhamma + d\=a + ta} = \palibf{dhammadinna}\footnote{\pali{dhammo ena\d m dad\=at\=uti dhammadinno} (May the Dhamma gives this one, thus \pali{Dhammadinna}).} (Dhammadinna)\par
\pali{va\d d\d dha + m\=ana} = \palibf{va\d d\d dham\=ana}\footnote{\pali{va\d d\d dhat\=uti va\d d\d dham\=ano.} (May this one grows, thus \pali{Va\d d\d dham\=ana}).} (Va\d d\d dham\=ana)\par

\subparagraph*{\pali{Aka}} (Mogg\,5.35)\label{pacck5:aka}

\pali{j\=iva + aka} = \palibf{j\=ivaka}\footnote{\pali{j\=ivat\=uti j\=ivako} (May this one lives long, thus \pali{J\=ivaka}).} (J\=ivaka)\par
\pali{nanda + aka} = \palibf{nandaka}\footnote{\pali{nandat\=uti nandako} (May this one rejoices, thus \pali{Nandaka}).} (Nandaka)\par

\subsection*{6.\ \pali{Paccaya}s for feminine nouns}\label{kita:group6}

In Kacc and Sadd four are mentioned: \pali{a, ti, yu,} and \pali{ririya}. In Mogg, there are nine: \pali{a, \d na, kti, ka, yaka, ya, ana, ririya,} and \pali{ni}.

\subparagraph*{\pali{A, ti, yu, \d na, kti, ka, yaka, ya, ana}} (Kacc\,553, R\=upa\,599, Sadd\,1140, Mogg\,5.49)\label{pacck6:a}\label{pacck6:ti}\label{pacck6:yu}\label{pacck6:dna}\label{pacck6:kti}\label{pacck6:ka}\label{pacck6:yaka}\label{pacck6:ya}\label{pacck6:ana}

\pali{j\=ira + a} = \palibf{jar\=a}\footnote{\pali{j\=irati j\=irana\d m v\=a jar\=a.}} (old age, decay)\par
\pali{pati + sa\d m + bhidi + a} = \palibf{pa\d tisambhid\=a}\footnote{\pali{pa\d tisambhijjat\=iti pa\d tisambhid\=a.}} (discriminating knowledge)\par
\pali{pati + pada + a} = \palibf{pa\d tipad\=a}\footnote{\pali{pa\d tipajjati et\=ay\=ati pa\d tipad\=a.}} (way by which one practices)\par
\pali{upa + \=a + d\=a + a} = \palibf{up\=ad\=a}\footnote{\pali{up\=adiyat\=iti up\=ad\=a.}} (attachment)\par
\pali{cinta + a} = \palibf{cint\=a}\footnote{\pali{cintana\d m cint\=a.}} (thought)\par
\pali{pati + \d th\=a + a} = \palibf{pati\d t\d th\=a}\footnote{\pali{pati\d t\d th\=ana\d m pati\d t\d th\=a.}} (support)\par
\pali{sikkha + a} = \palibf{sikkh\=a}\footnote{\pali{sikkhana\d m sikkhiyat\=iti v\=a sikkh\=a.}} (learning, education)\par
\pali{bhikkha + a} = \palibf{bhikkh\=a} (begging, alms)\par
\pali{mana + ti} = \palibf{mati}\footnote{\pali{manati j\=an\=at\=iti mati manana\d m v\=a mati.}} (knowledge, thought)\par
\pali{sara + ti} = \palibf{sati} (mindfulness, reflection)\par
\pali{cinta + yu} = \palibf{jetan\=a}\footnote{\pali{cetayat\=iti cetan\=a.}} (intention)\par
\pali{vida + yu} = \palibf{vetan\=a}\footnote{\pali{vetayat\=iti vetan\=a.}} (feeling)\par

In Mogg\,5.49, various \pali{paccaya}s are exemplified: \palibf{a}---\pali{titikkh\=a, v\=ima\d ms\=a, jigucch\=a, pip\=as\=a, putt\=iy\=a, \=ih\=a, bhikkh\=a, \=apad\=a, medh\=a, godh\=a}; \palibf{\d na}---\pali{k\=ar\=a, h\=ar\=a, t\=ar\=a, dh\=ar\=a, \=ar\=a}; \palibf{kti}---\pali{i\d t\d thi, si\d t\d thi, bhitti, bhatti, tanti, bh\=uti}; \palibf{ka}---\pali{guh\=a, ruj\=a, mud\=a}; \palibf{yaka}---\pali{vijj\=a, ijj\=a}; \palibf{ya}---\pali{seyy\=a, samajj\=a, pabbajj\=a, paricariy\=a, j\=agariy\=a}; \palibf{ana}---\pali{k\=ara\d n\=a, h\=ara\d n\=a, vedan\=a, vandan\=a, up\=asan\=a}.

\subparagraph*{\pali{Ririya}} (Kacc\,554, R\=upa\,601, Sadd\,1141, Mogg\,5.51)\label{pacck6:ririya}

\pali{kara + ririya} = \palibf{kiriy\=a}\footnote{\pali{kattabb\=a kiriy\=a, kara\d na\d m kiriy\=a.} This can also be nt.: \pali{kara\d n\=iya\d m kiriya\d m}.} (action)\par

\subparagraph*{\pali{Ni}} (Mogg\,5.50)\label{pacck6:ni}

\pali{h\=a + ni} = \palibf{h\=ani/j\=ani} (loss, deprivation)\par

\subsection*{7.\ \pali{Paccaya}s for infinitives}\label{kita:group7}\label{par:kitatum}

In Kacc and Sadd \pali{tave} and \pali{tu\d m} are mentioned. In Mogg \pali{t\=aye} is added. For their use, we have a dedicated lesson in Chapter \ref{chap:inf}. 

\subparagraph*{\pali{Tave, tu\d m, t\=aye}} (Kacc\,561--3, R\=upa\,636--9, Sadd\,1148--9, Mogg\,5.61)\label{pacck7:tave}\label{pacck7:tudm}\label{pacck7:taaye}

\pali{kara + tave} = \palibf{k\=atave}\footnote{For example, \pali{pu\~n\~n\=ani k\=atave icchati} (One desires to make merit). In Mogg\,5.61 these examples are given: \pali{k\=atu\d m gacchati, katt\=aye gacchati, k\=atave gacchati} (one goes to do).} (to do)\par
\pali{su + tu\d m} = \palibf{sotu\d m}\footnote{For example, \pali{saddhamma\d m sotumicchati} (One desires to hear the true doctrine).} (to hear)\par
\pali{nida + tu\d m} = \palibf{ninditu\d m}\footnote{For example, \pali{ko ta\d m ninditumarahati} (One is suitable to blame that person).} (to blame)\par
\pali{ji + tu\d m} = \palibf{jetu\d m}\footnote{For example, \pali{sakk\=a jetu\d m dhanena v\=a} (Or capable to win with wealth).} (to win)\par
\pali{d\=a + tu\d m} = \palibf{d\=atu\d m}\footnote{For example, \pali{alameva d\=an\=ani d\=atu\d m} (Suitable only to give gifts).} (to give)\par
\pali{kara + tu\d m} = \palibf{k\=atu\d m}\footnote{For example, \pali{alameva pu\~n\~n\=ani k\=atu\d m} (Suitable only to make merits).} (to do)\par

In Mogg\,5.61 these are also given, for \pali{bhuja + tu\d m}: \pali{icchati bhottu\d m} (one desires to eat), \pali{sakkoti bhottu\d m} (one is able to eat), \pali{k\=alo bhottu\d m} (time to eat), \pali{arahati bhottu\d m} (one is suitable to eat), \pali{ala\d m bhottu\d m} (suitable to eat).

\subsection*{8.\ \pali{Paccaya}s for past participles}\label{kita:group8}

In Kacc and Sadd \pali{ta}, \pali{tavantu}, and \pali{t\=av\=i} are given. In Mogg \pali{kta}, \pali{ktavantu}, and \pali{kt\=av\=i} is mentioned instead. Moggall\=ana adds \pali{k-anubandha} to mark that no vowel \pali{vuddhi} will be applied. See Chapter \ref{chap:pp} for the use of these verbs.

\subparagraph*{\pali{Ta, tavantu, t\=av\=i}} (Kacc\,555--7, R\=upa\,612--4, Sadd\,1142--4, Mogg\,5.55--60)\label{pacck8:ta}\label{pacck8:kta}\label{pacck8:tavantu}\label{pacck8:ktavantu}\label{pacck8:taavii}\label{pacck8:ktaavii}

Only \pali{ta/kta} can be in both active and passive forms. The rest are only for active form. The products of \pali{ta} as verbal nouns are of neuter gender (Sadd\,1347).

\pali{hu + ta/tavantu/t\=av\=i} = \palibf{huta/hutavantu/hut\=av\=i}\footnote{For example, \pali{aggi\d m huto, hutav\=a, hut\=av\=i} (one who offered to fire).} (offered)\par
\pali{bhuja + ta/tavantu/t\=av\=i} = \palibf{bhutta/bhuttavantu/bhutt\=av\=i}\footnote{For example, \pali{odana\d m bhutto, bhuttav\=a, bhutt\=av\=i} (one who ate boiled rice).} (eaten)\par

Here are examples for passive \pali{ta}: \pali{tena bh\=asita\d m} ([words] said by that [person]), \pali{tena desita\d m} ([Dhamma] preached by that [person]). With intransitive verbs, it sounds like verbal nouns, for example, \pali{tassa g\=ita\d m} (his singing), \pali{tassa nacca\d m} (his dancing), \pali{tassa hasita\d m} (his laughing). Here are examples from Mogg\,5.59--60, \pali{aya\d m tehi y\=ato patho} (This way was gone by them), \pali{iha te y\=at\=a} (They went here), \pali{iha tehi y\=ata\d m} (Here was gone by them), \pali{odano tehi bhutto} (Boiled rice was eaten by them), \pali{iha tehi bhutta\d m} (Here [food] was eaten by them).

Furthermore \pali{ta} can be used regardless of time for certain roots. This ends up with nouns, for example, \pali{buddha/\~n\=ata} (knower), \pali{sara\d na\.ngata} (one going to refuge), \pali{samatha\.ngata} (one going to tranquility), \pali{amata\.ngata} (one going to the deathless state). In Kacc\,650, R\=upa\,651, and Sadd\,1288, it is said that \pali{ta} has present and past meaning, e.g.\ \pali{bh\=uta}\footnote{abhavi, bhavat\=iti bh\=uta\d m.} (state of being).

\subsection*{9.\ \pali{Paccaya}s for absolutives}\label{kita:group9}

In Kacc and Sadd \pali{tuna}\footnote{In Thai tradition, this is normally called \pali{t\=una}.}, \pali{tv\=ana}, and \pali{tv\=a} are mentioned. As in Mogg but a little differently, \pali{tuna}, \pali{ktv\=ana}, and \pali{ktv\=a} is given. All these \pali{paccaya}s produce uninflected verb form, i.e.\ \emph{absolutives}. Some scholars call the products of these \pali{gerund}. This is a misnomer because, as Kacc\=ayana asserts, the products of these \pali{paccaya}s, also \pali{tave} and \pali{tu\d m}, will never be nouns (Kacc\,601, R\=upa\,334). In our lessons, I mention these in Chapter \ref{chap:pp}.

\subparagraph*{\pali{Tuna, tv\=ana, tv\=a}} (Kacc\,564, R\=upa\,640, Sadd\,1150--6, Mogg\,5.62--3)\label{pacck9:tuna}\label{pacck9:tuuna}\label{pacck9:tvaana}\label{pacck9:tvaa}\label{pacck9:ktvaana}\label{pacck9:ktvaa}

\pali{kara + tuna} = \palibf{k\=atuna}\footnote{For example, \pali{k\=atuna kamma\d m gacchati} (Having done work, one goes).} (having done)\par
\pali{na + kara + tuna} = \palibf{ak\=atuna}\footnote{For example, \pali{ak\=atuna pu\~n\~na\d m kilissanti satt\=a} (Not having done merit, beings are blemished).} (not having done)\par
\pali{su + tv\=ana} = \palibf{sutv\=ana}\footnote{For example, \pali{dhamma\d m sutv\=ana modanti} (Having listened to the Dhamma, [people] delight).} (having listened)\par
\pali{su + tv\=a} = \palibf{sutv\=a}\footnote{For example, \pali{sutv\=a j\=aniss\=ama} (Having listened, [then we] know).} (having listened)\par

Sometimes these \pali{paccaya}s mark verbs that simultaneously act with the main verb (Sadd\,1151), for example, \pali{andhak\=ara\d m nihantv\=ana uditoya\d m div\=akaro} (This sun rose and killed the dark).

Sometimes the verbs act after the main verb (Sadd\,1152), for example, \pali{dv\=aram\=avaritv\=a pavisati} (He/she enters then shuts the door).

Sometimes these verbs and the main verb take different subjects (Sadd\,1153), for example, \pali{s\=iha\d m disv\=a bhaya\d m hoti} (Having seen a lion, fear arises [= he/she is frightened]).

Sometimes these verbs are used in a mutual structure without any main verb (Sadd\,1154), for example, \pali{appatv\=a nadi\d m pabbato, atikkamma pabbata\d m nad\=i} (The mountain does not reach the river, the river runs through the mountain).

Sometimes these verbs denote the cause or sign of the action (Sadd\,1155), for example, \pali{s\=iha\d m disv\=a bhaya\d m hoti} (Because of seeing the lion, he/she is scared), \pali{ghata\d m pivitv\=a bala\d m j\=ayate} (Because of eating ghee, power rises).

Sometimes these \pali{paccaya}s mark modifiers (Sadd\,1156), for example, \pali{up\=ad\=ayar\=upa} (dependent form), \pali{nh\=atv\=agamana} (bath-going).

Sometimes these have negative meaning when used with \pali{ala\d m} and \pali{khalu} (Mogg\,5.62), for example, \pali{ala\d m/khalu sotuna/sutv\=ana/ sutv\=a} (had enough to hear, useless to hear).

\subsection*{10.\ \pali{Paccaya}s for present participles}\label{kita:group10}

There are two \pali{paccaya}s in this group, namely \pali{anta} and \pali{m\=ana}. In Mogg \pali{anta} becomes \pali{nta}. Also \pali{\=ana} is mentioned somewhere else. For their use, see Chapter \ref{chap:prp}.

\subparagraph*{\pali{M\=ana, \=ana, anta}} (Kacc\,565, R\=upa\,646, Sadd\,1157--8, Mogg\,5.64--7; Kacc\,655, R\=upa\,650, Sadd\,1293)\label{pacck10:maana}\label{pacck10:aana}\label{pacck10:anta}\label{pacck10:nta}

\pali{sara + m\=ana} = \palibf{saram\=ana}\footnote{For example, \pali{saram\=ano rodati} ([While] remembering, one cries).} (remembering)\par
\pali{kara + m\=ana} = \palibf{kurum\=ana} (doing)\par
\pali{kara + \=ana} = \palibf{kar\=ana} (doing)\par
\pali{gamu + anta} = \palibf{gacchanta}\footnote{For example, \pali{gacchanto ga\d nh\=ati} ([While] going, one carries [a thing]).} (going)\par

Sometimes \pali{anta} is used regardless of time (Sadd\,1158), for example, \pali{so mahanto hoti} (he honors), \pali{so mahanto ahosi} (he honored), \pali{so mahanto bhavissati} (he will honor).

Sometimes these are used in passive form (Mogg\,5.66), for example, \pali{\d th\=iyam\=ana\d m} ([place] stood [by him/her]), \pali{paccam\=ano odano} (boiled rice being cooked [by him/her]).

Sometimes these can be used with \pali{ssa} to mark the future (Mogg\,5.67), for example, \pali{\d thassanto/\d thassam\=ano} ([He/she] will stand), \pali{\d th\=iyissam\=ana\d m} ([place] on where he/she will stand), \pali{paccissam\=ano odano} (boiled rice that he/she will cook).

Sometimes \pali{m\=ana}, \pali{\=ana} and \pali{anta} can have future meaning (Kacc\,655, R\=upa\,650, Sadd\,1293), for example, \pali{kamma\d m karonto, kamma\d m kurum\=ano, kamma\d m kar\=ano vajati} (one who will do the work goes).

\subsection*{11.\ \pali{Paccaya}s for nouns of some particular roots}\label{kita:group11}

In Kacc and Sadd five additional \pali{paccaya}s are mentioned: \pali{ratthu, ritu, r\=atu, tuka,} and \pali{ika}. In Mogg two are mentioned: \pali{tu} (equivalent to \pali{ritu} and \pali{r\=atu}) and \pali{kika} (equivalent to \pali{ika}).

\subparagraph*{\pali{Ratthu}} (Kacc\,566, R\=upa\,574, Sadd\,1159)\label{pacck11:ratthu}

\pali{s\=asa + ratthu} = \palibf{satthu}\footnote{\pali{sadevaka\d m loka\d m s\=asat\=iti satth\=a} (One who teaches the worldlings together with gods, thus teacher).} (teacher)\par

\subparagraph*{\pali{Ritu, tu}} (Kacc\,567, R\=upa\,565, Sadd\,1160, Mogg\,7.72)\label{pacck11:ritu}\label{pacck11:tu1}

\pali{p\=a + ritu} = \palibf{pitu}\footnote{\pali{p\=ati puttanti pit\=a} (One who protects [his] child, thus father).} (father)\par
\pali{dh\=a + ritu} = \palibf{dh\=itu}\footnote{\pali{m\=at\=apit\=uhi dh\=ariyateti dh\=it\=a} (One being protected by parents, thus daughter).} (daughter)\par

\subparagraph*{\pali{R\=atu, tu}} (Kacc\,568, R\=upa\,576, Sadd\,1161, Mogg\,7.72)\label{pacck11:raatu}\label{pacck11:tu2}

\pali{m\=ana + r\=atu} = \palibf{m\=atu}\footnote{\pali{dhammena putta\d m manet\=iti m\=at\=a} (One loves [her] child by nature, thus mother).} (mother)\par
\pali{bh\=asa + r\=atu} = \palibf{bh\=atu}\footnote{\pali{pubbe bh\=ast\=iti bh\=at\=a} (One speaks first, thus [elder] brother). Or \pali{pacch\=a bh\=ast\=iti bh\=at\=a} (One speaks later, thus [younger] brother).} (brother)\par

\subparagraph*{\pali{Tuka}} (Kacc\,569, R\=upa\,610, Sadd\,1162)\label{pacck11:tuka}

\pali{\=a + gamu + tuka} = \palibf{\=agantuka}\footnote{\pali{\=agacchat\=iti \=agantuko.}} (guest, comer)\par

\subparagraph*{\pali{Ika, kika}} (Kacc\,570, R\=upa\,611, Sadd\,1163, Mogg\,7.21)\label{pacck11:ika}\label{pacck11:kika}

\pali{gamu + ika} = \palibf{gamika}\footnote{\pali{gamissati, gantu\d m bhabboti gamiko, bhikkhu.}} (one who will goes, or one suitable to go)\par

\bigskip
The following section is a part of this group beside the aforementioned. There are other \pali{paccaya}s which produce nouns for some roots in a particular manner. They are so numerous, actually overwhelming, that I cannot list them first. Some are the component of many familiar terms. Some are trivial. I try to list all of them, but very trivial things are intentionally neglected. This list seems in order, but it is not always so. I mainly follow Dr.\,Supaphan's order (\citealp{supaphan:pali}) with an attempt to merge things together (but it turns out to be unfulfilled though). When Mogg is brought into consideration together with Kacc and Sadd, it breaks the smooth flow inevitably. Sometimes, you have to jump around to compare the \pali{paccaya} of the same name but from different sources. Mogg has a precise way to name \pali{paccaya}s by adding transformative markers (\pali{anubandha}) into them. The often found \pali{anubandha}s are \pali{\d na} (\pali{vuddhi} marker), \pali{ka} (\pali{vuddhi} preventer), and \pali{ra} (last-syllable killer). Sometimes these are added to the end, sometimes to the beginning of the \pali{paccaya}s. That is the reason why they seem messy when you see from English perspective. I arrange all of these \pali{paccaya}s into a familiar order in Appendix \ref{chap:paccaya}. You can consult that part when you want to find a specific thing. 

Another issue worth mentioning is the root of the terms analyzed. There is no strict rule of that, so you can see a variety of them. Sometimes a root is called with slightly different names, e.g.\ Mogg's \pali{kama} is Kacc/Sadd's \pali{kamu}. That is easy to identify. But many of roots mentioned by Mogg, even by Kacc or Sadd itself, are not found in Sadd-Dh\=a. I mark these with a question (?). They can be the missing ones, or the result of certain transformation of existing ones. I have not enough effort to investigate into this, so I leave them to you as such. Furthermore, I follow Moggall\=ana in the CSCD collection which the name of \pali{paccaya}s always ends with a vowel, mostly \pali{a}. Whereas in Kacc/Sadd several \pali{paccaya}s end with an \pali{anubandha} consonant, \pali{tra\d n, man} for example.

The final remark here is it is undoubted that the traditional grammarians exert a great effort to expose words' origin and put them into order. However, recalcitrant instances can be found here and there. Do not be surprised or panic when you see things not in place, or when you hope to see an intelligible explanation but none is found. That is natural, not esoteric. No one can know everything about this. Even great grammarians cast doubts, and sometimes make an indigestible judgement.

\subparagraph*{\pali{A}} (Sadd\,1248--9)\label{pacckx:a}

\pali{sa\d m + dh\=a + a} = \palibf{saddh\=a} (faith)\par
\pali{sa\d m + \~n\=a + a} = \palibf{sa\~n\~n\=a} (recognition)\par
\pali{pa + bh\=a + a} = \palibf{pabh\=a}\footnote{In Sadd\,1266, this instance is a product of \pali{kvi}.} (light)\par
\pali{me/dhara + a} = \palibf{medh\=a}\footnote{This term has a confusing origin. In Sadd\,1325, it may come from \pali{me} (to seize) or \pali{dhara} (to hold) plus \pali{a}. In Sadd\,1326, Aggava\d msa entertains that it may come from \pali{midhu} (to hurt) plus \pali{\d na}. There is no such a root listed in Sadd-Dh\=a. The closest is \pali{mida} in the same meaning. The latter idea sounds more plausible to me.} (wisdom)\par

\subparagraph*{\pali{Ka, \d da, dha}} (Kacc\,663--4, R\=upa\,673--4, Sadd\,1305--7, Mogg\,7.58--9, Mogg\,7.98)\label{pacckx:ka1}\label{pacckx:dda}\label{pacckx:dha1}

In Mogg \pali{\d da} and \pali{dha} is used instead of \pali{ka}.

\pali{ka\d di + ka} = \palibf{ka\d n\d da}\footnote{In Mogg\,7.58, this is from root \pali{kamu} (go).} (arrow)\par
\pali{gha\d ti + ka} = \palibf{gha\d n\d ta}\footnote{In dictionaries, f.\ \pali{gha\d n\d t\=a} is found.} (bell)\par
\pali{va\d ti + ka} = \palibf{va\d n\d ta} (stalk)\par
\pali{kara\d di? + ka} = \palibf{kara\d n\d da} (basket)\par
\pali{ma\d di + ka} = \palibf{ma\d n\d da}\footnote{In Mogg\,7.58, this is from root \pali{mana} (know).} (top)\par
\pali{sa\d di + ka} = \palibf{sa\d n\d da}\footnote{In Mogg\,7.58, this is from root \pali{sama} (calm).} (heap)\par
\pali{ku\d thi + ka} = \palibf{ku\d t\d tha} (leprosy)\par
\pali{bha\d di + ka} = \palibf{bha\d n\d da} (goods)\par
\pali{pa\d di + ka} = \palibf{pa\d n\d daka} (eunuch)\par
\pali{da\d di? + ka} = \palibf{da\d n\d da}\footnote{In Mogg\,7.58, this is from root \pali{damu} (tame).} (stick)\par
\pali{ra\d di? + ka} = \palibf{ra\d n\d da}\footnote{In Mogg\,7.58, this is from root \pali{ramu} (play).} (drunkard)\par
\pali{vi + ta\d di + ka} = \palibf{vita\d n\d da}\footnote{\pali{visesena ga\d n\d dati c\=aleti paresa\d m vi\~n\~n\=una\d m hadaya\d m kampet\=iti viga\d n\d do.}} (persuading/agitating speech)\par
\pali{isi\d di? + ka} = \palibf{isi\d n\d da}\footnote{\pali{isi\d n\d dati paresa\d m maddat\=iti isi\d n\d do.}} (subjugator)\par
\pali{ca\d di + ka} = \palibf{ca\d n\d da} (fierce)\par
\pali{ga\d di + ka} = \palibf{ga\d n\d da}\footnote{In Mogg\,7.58, this is from root \pali{gamu} (go).} (swelling)\par
\pali{a\d di? + ka} = \palibf{a\d n\d da}\footnote{In Mogg\,7.58, this is from root \pali{ama} (arise).} (egg)\par
\pali{la\d di? + ka} = \palibf{la\d n\d da}\footnote{In Mogg\,7.58, this is from root \pali{lama?} (hurt).} (dung)\par
\pali{me\d di? + ka} = \palibf{me\d n\d da} (ram)\par
\pali{era\d di? + ka} = \palibf{era\d n\d da} (castor oil plant)\par
\pali{kha\d di + ka} = \palibf{kha\d n\d da}\footnote{In Mogg\,7.58, this is from root \pali{khanu} (dig).} (bit)\par
\pali{kh\=ada + ka} = \palibf{khandha}\footnote{In Mogg\,7.98, this is a product of \pali{dha} appling to root \pali{khanu}.} (bulk of the body)\par
\pali{ama + ka} = \palibf{andha}\footnote{In Mogg\,7.98, this is a product of \pali{dha}.} (blind)\par
\pali{gamu + ka} = \palibf{gandha} (smell)\par
\pali{damu + dha} = \palibf{dandha}\footnote{Mogg\,7.98} (stupid person)\par
\pali{ramu + dha} = \palibf{randha}\footnote{Mogg\,7.98} (cleft)\par

\subparagraph*{\pali{I}} (Kacc\,669, R\=upa\,679, Sadd\,1315, Mogg\,7.7--8)\label{pacckx:i}

\pali{muna + i} = \palibf{muni}\footnote{In Mogg\,7.8, this is from root \pali{mana}.} (monk)\par
\pali{yata + i} = \palibf{yati} (monk)\par
\pali{agga + i} = \palibf{aggi} (fire)\par
\pali{kava + i} = \palibf{kavi} (poet)\par
\pali{suca + i} = \palibf{suci} (cleanness)\par
\pali{ruca + i} = \palibf{ruci} (liking)\par
\pali{asa + i} = \palibf{asi} (sword)\par
\pali{kasa + i} = \palibf{kasi} (ploughing)\par
\pali{masa + i} = \palibf{masi} (soot)\par
\pali{ru + i} = \palibf{ravi} (the sun)\par
\pali{sappa + i} = \palibf{sappi} (ghee)\par
\pali{dh\=a + i} = \palibf{dadhi} (curd)\par

\subparagraph*{\pali{Ki}} (Mogg\,7.9)\label{pacckx:ki}

\pali{K-anubandha} prevents vowel \pali{vuddhi}.

\pali{isa + ki} = \palibf{isi} (sage)\par
\pali{gira + ki} = \palibf{giri} (mountain)\par
\pali{suca + ki} = \palibf{suci} (cleanness)\par
\pali{ruca + ki} = \palibf{ruci} (liking)\par

\subparagraph*{\pali{I\d na}} (Mogg\,7.10)\label{pacckx:idna1}

\pali{\d N-anubandha} entails vowel \pali{vuddhi}.

\pali{vapa + i\d na} = \palibf{v\=api} (water tank)\par
\pali{vara + i\d na} = \palibf{v\=ari} (water)\par
\pali{vasa + i\d na} = \palibf{v\=asi} (knife)\par
\pali{rasa + i\d na} = \palibf{r\=asi} (heap)\par
\pali{nabha + i\d na} = \palibf{n\=abhi} (navel)\par
\pali{hara + i\d na} = \palibf{h\=ari} (attractive)\par
\pali{hana + i\d na} = \palibf{gh\=ati} (weapon)\par
\pali{pa\d na + i\d na} = \palibf{p\=a\d ni} (the hand)\par

\subparagraph*{\pali{Gi}} (Mogg\,7.34)\label{pacckx:gi}

\pali{aga + gi} = \palibf{aggi}\footnote{In Kacc\,669, R\=upa\,679, Sadd\,1315, this is the product of \pali{agga + i}.} (fire)\par

\subparagraph*{\pali{Ati}} (Mogg\,7.69)\label{pacckx:ati}

\pali{p\=a + ati} = \palibf{pati} (master)\par
\pali{vasa + ati} = \palibf{vasati} (dwelling)\par

\subparagraph*{\pali{\=I}} (Mogg\,7.12)\label{pacckx:ii}

\pali{tanda? + \=i} = \palibf{tand\=i}\footnote{\pali{tandana\d m tand\=i \=alasya\d m.}} (laziness)\par
\pali{lakkha + \=i} = \palibf{lakkh\=i} (good luck)\par

\subparagraph*{\pali{U}} (Mogg\,7.2)\label{pacckx:u}

\pali{bhara + u} = \palibf{bhara}\footnote{\pali{bharat\=iti bhara bhatt\=a.} This should be \pali{bharu}, but the term is not found anywhere except in compound forms.} (husband)\par
\pali{mara + u} = \palibf{maru} (sand, deity)\par
\pali{cara + u} = \palibf{caru} (food offered to gods/spirits)\par
\pali{tara + u} = \palibf{taru} (tree)\par
\pali{ara + u} = \palibf{aru} (wound)\par
\pali{gara + u} = \palibf{garu} (teacher)\par
\pali{hana + u} = \palibf{hanu}\footnote{See also Kacc\,671, R\=upa\,681, Sadd\,1317.} (jaw)\par
\pali{tanu + u} = \palibf{tanu} (body)\par
\pali{mana + u} = \palibf{manu} (the creator god)\par
\pali{bhama? + u} = \palibf{bhamu} (eyebrow)\par
\pali{kita + u} = \palibf{ketu} (flag)\par
\pali{dhana + u} = \palibf{dhanu} (bow)\par
\pali{ba\d mha? + u} = \palibf{bahu} (many)\par
\pali{kamba? + u} = \palibf{kambu} (bangle, conch)\par
\pali{amba? + u} = \palibf{ambu} (water)\par
\pali{cakkha + u} = \palibf{cakkhu} (eye)\par
\pali{bhikkha + u} = \palibf{bhikkhu}\footnote{In Kacc\,535, R\=upa\,593, Sadd\,1119, this instance comes from \pali{bhikkha + r\=u}.} (monk)\par
\pali{sa\.nka? + u} = \palibf{sa\.nku} (spike)\par
\pali{inda? + u} = \palibf{indu} (the moon)\par
\pali{anda? + u} = \palibf{andu} (fetter)\par
\pali{yaja + u} = \palibf{yaju} (Yajur Veda)\par
\pali{pa\d ta + u} = \palibf{pa\d tu} (clever)\par
\pali{a\d na + u} = \palibf{a\d nu} (particle, atom)\par
\pali{asa + u} = \palibf{asu/asava} (life, breath)\par
\pali{vasa + u} = \palibf{vasu} (wealth)\par
\pali{pasa + u} = \palibf{pasu} (cattle)\par
\pali{pa\d msa + u} = \palibf{pa\d msu} (dust)\par
\pali{bandha + u} = \palibf{bandhu} (relation)\par

\subparagraph*{\pali{\d Nu}} (Mogg\,7.1)\label{pacckx:dnu1}

This means \pali{u} with \pali{\d n-anubandha}, so \pali{vuddhi} is expected.

\pali{cara + \d nu} = \palibf{c\=aru} (beautiful)\par
\pali{dara + \d nu} = \palibf{t\=aru} (wood)\par
\pali{kara + \d nu} = \palibf{k\=aru} (craftsman, maker god)\par
\pali{raha + \d nu} = \palibf{r\=ahu} (eclipse)\par
\pali{jana + \d nu} = \palibf{j\=a\d nu} (knee)\par
\pali{sana + \d nu} = \palibf{s\=anu} (table land)\par
\pali{tala + \d nu} = \palibf{t\=alu} (palate)\par
\pali{s\=ada? + \d nu} = \palibf{s\=adu} (sweet)\par
\pali{s\=adha + \d nu} = \palibf{s\=adhu} (good person)\par
\pali{kasa + \d nu} = \palibf{k\=asu} (pit)\par
\pali{asa + \d nu} = \palibf{\=asu} (quickly)\par
\pali{ca\d ta + \d nu} = \palibf{c\=a\d tu} (pleasant)\par
\pali{aya + \d nu} = \palibf{\=ayu} (age)\par
\pali{v\=a + \d nu} = \palibf{v\=ayu} (wind)\par

\subparagraph*{\pali{Ku}} (Mogg\,7.5--6)\label{pacckx:ku}

This is \pali{u} with \pali{k-anubandha}.

\pali{tapa + ku} = \palibf{tipu} (lead, tin)\par
\pali{usa + ku} = \palibf{usu} (arrow)\par
\pali{vidha + ku} = \palibf{vidhu} (the moon)\par
\pali{kura + ku} = \palibf{kuru} (Kuru)\par
\pali{putha + ku} = \palibf{puthu} (thick)\par
\pali{muda + ku} = \palibf{mudu} (soft)\par
\pali{sanda + ku} = \palibf{sindhu} (river)\par
\pali{b\=adha + ku} = \palibf{b\=ahu} (the arm)\par
\pali{ra\d mgha? + ku} = \palibf{raghu} (king Raghu)\par
\pali{vida + ku} = \palibf{bindu} (dot)\par
\pali{mana + ku} = \palibf{madhu} (sweet)\par
\pali{rapa? + ku} = \palibf{ripu} (enemy)\par
\pali{sasa + ku} = \palibf{susu} (young man)\par
\pali{ara + ku} = \palibf{uru} (large)\par
\pali{\=a + khanu + ku} = \palibf{\=akhu} (rat)\par
\pali{tara + ku} = \palibf{tharu} (hilt, handle)\par
\pali{la\d mgha? + ku} = \palibf{laghu/lahu} (light, quick)\par
\pali{pa + bhaja + ku} = \palibf{pabha\.ngu} (sprout, brittle)\par
\pali{su + \d th\=a + ku} = \palibf{su\d t\d thu} (good)\par
\pali{du + \d th\=a + ku} = \palibf{du\d t\d thu} (bad)\par

\subparagraph*{\pali{\=U}} (Mogg\,7.3--4)\label{pacckx:uu}

\pali{bandha + \=u} = \palibf{vadh\=u} (woman)\par
\pali{jan\=i + \=u} = \palibf{jamb\=u} (rose-apple tree)\par
\pali{kara + \=u} = \palibf{kakkandh\=u} (jujube tree)\par
\pali{\=a + lamba? + \=u} = \palibf{al\=ab\=u} (long white gourd)\par
\pali{sara + \=u} = \palibf{sarabh\=u} (river Sarabh\=u)\par
\pali{sara + \=u} = \palibf{sarab\=u} (gecko)\par
\pali{cama + \=u} = \palibf{cam\=u} (army)\par
\pali{tanu + \=u} = \palibf{tanu} (body)\par

\subparagraph*{\pali{Ka}} (Kacc\,661, R\=upa\,671, Sadd\,1302, Mogg\,7.14--5)\label{pacckx:ka2}

\pali{susa + ka} = \palibf{sukka} (white)\par
\pali{suca + ka} = \palibf{soka} (grief)\par
\pali{vaka + ka} = \palibf{vakka} (kidney)\par
\pali{i + ka} = \palibf{eka} (one)\par
\pali{bh\=i + ka} = \palibf{bheka} (frog)\par
\pali{k\=a? + ka} = \palibf{k\=aka} (crow)\par
\pali{kara + ka} = \palibf{kakka} (paste)\par
\pali{ara + ka} = \palibf{akka} (the sun)\par
\pali{saka + ka} = \palibf{sakka} (king of the gods)\par
\pali{v\=a + ka} = \palibf{v\=aka} (bark)\par
\pali{\=uha + ka} = \palibf{\=uk\=a} (louse)\par
\pali{unda? + ka} = \palibf{udaka} (water)\par
\pali{saka + ka} = \palibf{sikk\=a}\footnote{This exactly means a basket carried by a stick with loads on two ends.} (string of a balance)\par
\pali{h\=a + ka} = \palibf{h\=aka} (anger)\par
\pali{samba + ka} = \palibf{sambuka} (oyster)\par
\pali{putha + ka} = \palibf{puthuka} (foolish person)\par
\pali{suca + ka} = \palibf{sukka} (semen)\par
\pali{upa + ci + ka} = \palibf{upacik\=a} (termite)\par
\pali{kampa? + ka} = \palibf{pa\.nka} (mud)\par
\pali{usa + ka} = \palibf{ukk\=a} (torch)\par
\pali{usa + ka} = \palibf{ummuka} (firebrand)\par
\pali{vama? + ka} = \palibf{vammika} (anthill)\par
\pali{masa + ka} = \palibf{matthaka} (the head)\par

\subparagraph*{\pali{Aka}} (Mogg\,7.18)\label{pacckx:aka}

\pali{kara + aka} = \palibf{karaka} (drinking vessel)\par
\pali{kara + aka} = \palibf{karak\=a} (hail)\par
\pali{sara + aka} = \palibf{saraka} (drinking vessel)\par
\pali{nara + aka} = \palibf{naraka} (hell)\par
\pali{tara + aka} = \palibf{taraka} (boat, raft)\par
\pali{vara + aka} = \palibf{varaka}\footnote{In Thai translation, it is Job's tears, a kind of beadlike grains.} (wall, a kind of grain)\par
\pali{jana + aka} = \palibf{janaka} (father)\par
\pali{kana + aka} = \palibf{kanaka} (gold)\par
\pali{ka\d ta + aka} = \palibf{ka\d taka} (city)\par
\pali{kura + aka} = \palibf{koraka} (bud)\par
\pali{thu + aka} = \palibf{thavaka} (garland)\par

\subparagraph*{\pali{\=Aka}} (Mogg\,7.19--20)\label{pacckx:aaka}

\pali{pala + \=aka} = \palibf{bal\=ak\=a} (crane)\par
\pali{pata + \=aka} = \palibf{bat\=ak\=a} (flag)\par
\pali{s\=a + \=aka} = \palibf{s\=am\=ak\=a} (millet)\par
\pali{p\=a + \=aka} = \palibf{pin\=ak\=a} (bow of the great one)\par
\pali{gu + \=aka} = \palibf{guv\=ak\=a} (fruit of areca palm)\par
\pali{pa\d ta + \=aka} = \palibf{pa\d t\=ak\=a}\footnote{\pali{pa\d tati y\=at\=iti pa\d t\=ak\=a vejayant\=i.} This might be also a kind of plant.} (the Inda's mansion or chariot)\par
\pali{sala + \=aka} = \palibf{sal\=ak\=a} (medical instruments)\par
\pali{vida + \=aka} = \palibf{vid\=ak\=a} (wise person)\par
\pali{pa\d na + \=aka} = \palibf{pi\d n\d n\=ak\=a} (sesame paste)\par

\subparagraph*{\pali{\=Anaka}} (Mogg\,7.16)\label{pacckx:aanaka}

\pali{bh\=i + \=anaka} = \palibf{bhay\=anaka}\footnote{\pali{bh\=ayanti etasm\=ati bhay\=anako bhayajanako.} This means thing that frightens you.} (horrible)\par

\subparagraph*{\pali{\=A\d nika, \=a\d taka}} (Mogg\,7.17)\label{pacckx:aadnika}\label{pacckx:aadtaka}

\pali{si\.ngha? + \=a\d nika} = \palibf{si\.ngh\=a\d nik\=a} (nasal mucus)\par
\pali{si\.ngha? + \=a\d taka} = \palibf{si\.ngh\=a\d taka} (crossroad)\par

\subparagraph*{\pali{Kika}} (Mogg\,7.21--2)\label{pacckx:kika}

This is actually \pali{ika} with \pali{k-anubandha}.

\pali{viccha + kika} = \palibf{vicchika} (scorpion)\par
\pali{ala + kika} = \palibf{alika} (lie)\par
\pali{gamu + kika} = \palibf{gamika} (goer)\par
\pali{musa + kika} = \palibf{musika} (rat)\par
\pali{ka\d na + kika} = \palibf{k\=ika\d nik\=a} (bell)\par
\pali{muda + kika} = \palibf{muddik\=a} (ring)\par
\pali{maha + kika} = \palibf{mahik\=a} (frost, snow)\par
\pali{kala + kika} = \palibf{kalik\=a} (bud)\par
\pali{sappa + kika} = \palibf{sippik\=a} (oyster)\par

\subparagraph*{\pali{K\=ika}} (Mogg\,7.23)\label{pacckx:kiika}

This is \pali{\=ika} with \pali{k-anubandha}.

\pali{isa + k\=ika} = \palibf{is\=ik\=a} (brush)\par

\subparagraph*{\pali{\d Nuka}} (Mogg\,7.24)\label{pacckx:dnuka}

This is \pali{uka} with \pali{\d n-anubandha}.

\pali{kamu + \d nuka} = \palibf{k\=amuka} (sweetheart)\par
\pali{pada + \d nuka} = \palibf{p\=aduka} (shoes)\par

\subparagraph*{\pali{\d N\=uka}} (Mogg\,7.25--6)\label{pacckx:dnuuka}

This is \pali{\=uka} with \pali{\d n-anubandha}.

\pali{ma\d n\d da? + \d n\=uka} = \palibf{ma\d n\d d\=uka} (frog)\par
\pali{sala + \d n\=uka} = \palibf{s\=al\=uka} (the root of water lily)\par
\pali{ula? + \d n\=uka} = \palibf{ul\=uka} (owl)\par
\pali{mana + \d n\=uka} = \palibf{madh\=uka} (a kind of plant)\par
\pali{jala + \d n\=uka} = \palibf{jal\=uk\=a} (leech)\par

\subparagraph*{\pali{Tika}} (Mogg\,7.28)\label{pacckx:tika}

\pali{kara + tika} = \palibf{kattika} (month of Kattik\=a, November)\par

\subparagraph*{\pali{Saka}} (Mogg\,7.27)\label{pacckx:saka1}

\pali{kasa + saka} = \palibf{kassaka} (farmer)\par

\subparagraph*{\pali{\d Thakana}} (Mogg\,7.29)\label{pacckx:dthakana}

\pali{isa + \d thakana} = \palibf{i\d t\d thak\=a} (brick)\par

\subparagraph*{\pali{Kha}} (Mogg\,7.30--1)\label{pacckx:kha}

\pali{sama + kha} = \palibf{sa\.nkha}\footnote{In Kacc\,530, R\=upa\,584, Sadd\,1112, this is the product of \pali{sa\d m + khanu + kvi}.} (conch)\par
\pali{muna + kha} = \palibf{mukha} (face)\par
\pali{si + kha} = \palibf{sikh\=a} (crest)\par
\pali{vi + si + kha} = \palibf{visikh\=a} (street)\par
\pali{ni + kana + kha} = \palibf{nikkha} (big gold coin)\par
\pali{maya + kha} = \palibf{may\=ukha} (ray of light)\par
\pali{l\=u + kha} = \palibf{l\=ukha} (coarse)\par
\pali{ala + kha} = \palibf{akkha} (axle)\par
\pali{yasa + kha} = \palibf{yakkha} (demon)\par
\pali{ruha + kha} = \palibf{rukkha} (tree)\par
\pali{usa + kha} = \palibf{ukkha} (ox)\par
\pali{saha + kha} = \palibf{sakh\=a} (friend)\par

\subparagraph*{\pali{Gaka}} (Mogg\,7.32--3)\label{pacckx:gaka}

\pali{aja + gaka} = \palibf{agga} (the highest)\par
\pali{vaja + gaka} = \palibf{vagga} (group)\par
\pali{muda + gaka} = \palibf{mugga} (green peas)\par
\pali{gada + gaka} = \palibf{gagga} (sage Gagga)\par
\pali{gamu + gaka} = \palibf{ga\.nga} (the Ganges)\par
\pali{s\=i + gaka} = \palibf{si\.nga} (horn)\par
\pali{phura? + gaka} = \palibf{phuli\.nga} (buring charcoal)\par
\pali{u + cala + gaka} = \palibf{ucc\=ali\.nga} (caterpillar)\par
\pali{kala + gaka} = \palibf{kali\.nga} (Kali\.nga country)\par
\pali{bhama? + gaka} = \palibf{bhi\.nga} (wasp)\par
\pali{pa\d ta + gaka} = \palibf{pa\d ta\.nga} (grasshopper)\par

\subparagraph*{\pali{Gu}} (Mogg\,7.35--6)\label{pacckx:gu}

\pali{y\=a + gu} = \palibf{y\=agu} (rice-gruel)\par
\pali{vala + gu} = \palibf{vaggu} (pleasant)\par
\pali{phala + gu} = \palibf{pheggu} (sapwood, worthless thing)\par
\pali{bhara + gu} = \palibf{bhagu} (sage Bhagu)\par
\pali{hi + gu} = \palibf{hi\.ngu} (asafetida)\par
\pali{kama + gu} = \palibf{ka\.ngu} (millet)\par

\subparagraph*{\pali{Gha}} (Mogg\,7.37--8)\label{pacckx:gha}

\pali{jana + gha} = \palibf{ja\.ngh\=a} (the lower leg)\par
\pali{miha + gha} = \palibf{megha} (cloud)\par
\pali{muha + gha} = \palibf{mogha} (empty, useless)\par
\pali{s\=i + gha} = \palibf{s\=igha} (fast)\par
\pali{ni + daha + gha} = \palibf{nid\=agha} (drought, summer)\par
\pali{maha + gha} = \palibf{magh\=a} (a constellation)\par

\subparagraph*{\pali{Ca}} (Mogg\,7.39--40)\label{pacckx:ca}

\pali{cu + ca} = \palibf{coca} (wild banana)\par
\pali{sara + ca} = \palibf{sacca}\footnote{In Sadd\,1260, this is the product of \pali{sata + tya}.} (truth)\par
\pali{vara + ca} = \palibf{vacca} (excrement)\par
\pali{mara + ca} = \palibf{macca}\footnote{In Sadd\,1254, this is the product of \pali{mara + ratya}.} (human, the mortal)\par

\subparagraph*{\pali{Cu, \=ici}} (Mogg\,7.40)\label{pacckx:cu}\label{pacckx:iici}

\pali{mara + cu} = \palibf{maccu}\footnote{In Sadd\,1253, this is the product of \pali{musa + tyu}.} (death)\par
\pali{mara + \=ici} = \palibf{mar\=ici} (ray of light, mirage)\par

\subparagraph*{\pali{Ccha, cch\=ana}} (Sadd\,1251)\label{pacckx:ccha}\label{pacckx:cchaana}

\pali{tira + ccha/cch\=ana} = \palibf{tiraccha/tiracch\=ana} (beast)\par

\subparagraph*{\pali{Cha}} (Sadd\,1250, Mogg\,7.43--4)\label{pacckx:cha}

When \pali{cha} is applied, the last consonant of the roots is changed to \pali{ca} (Sadd\,1262).

\pali{ruja + cha} = \palibf{rucch\=a} (pain)\par
\pali{rica + cha} = \palibf{ricch\=a} (purging)\par
\pali{kita + cha} = \palibf{tikicch\=a} (healing)\par
\pali{sa\d m + kuca + cha} = \palibf{sa\d mkucch\=a} (bending)\par
\pali{mada + cha} = \palibf{macch\=a} (intoxication)\par
\pali{labha + cha} = \palibf{lacch\=a} (gain)\par
\pali{rada + cha} = \palibf{racch\=a} (path)\par
\pali{tira + cha} = \palibf{tiracch\=a} (beast)\par
\pali{sa\d m + gamu + cha} = \palibf{s\=agacch\=a} (going together)\par
\pali{du + bhaja + cha} = \palibf{dobhacch\=a} (bad consuming)\par
\pali{du + rusa + cha} = \palibf{dorucch\=a} (bad anger)\par
\pali{muha + cha} = \palibf{mucch\=a} (confusion)\par
\pali{vasa + cha} = \palibf{vacch\=a} (living)\par
\pali{kaca + cha} = \palibf{kacch\=a} (prospering)\par
\pali{sa\d m + katha + cha} = \palibf{s\=akacch\=a} (conversation)\par
\pali{tuda + cha} = \palibf{tucch\=a} (oppressing)\par
\pali{visa + cha} = \palibf{vicch\=a} (entering)\par
\pali{tatha + cha} = \palibf{taccha}\footnote{Strickly speaking, this is a secondary derivation.} (truth)\par
\pali{vi + ge + cha} = \palibf{vigaccha} (untuned song)\par
\pali{asa + cha} = \palibf{accha}\footnote{This instance and the following come from Mogg\,7.43--4.} (bear)\par
\pali{masa + cha} = \palibf{maccha} (fish)\par
\pali{vada + cha} = \palibf{vaccha} (calf)\par
\pali{kuca + cha} = \palibf{koccha} (rattan chair)\par
\pali{kaca + cha} = \palibf{kaccha} (armpit)\par
\pali{gupa + cha} = \palibf{guccha} (bouquet)\par
\pali{tusa + cha} = \palibf{tuccha} (lie)\par
\pali{pusa + cha} = \palibf{puccha} (tail)\par

\subparagraph*{\pali{Chika}} (Mogg\,7.41)\label{pacckx:chika}

\pali{kusa + chika} = \palibf{kucchi} (belly)\par
\pali{pasa + chika} = \palibf{pacchi} (basket)\par

\subparagraph*{\pali{Chuka}} (Mogg\,7.42)\label{pacckx:chuka}

\pali{kasa + chuka} = \palibf{kacchu} (itch, scab)\par
\pali{usa + chuka} = \palibf{ucchu} (sugarcane)\par

\subparagraph*{\pali{Ja}} (Sadd\,1259)\label{pacckx:ja}

\pali{aja + ja} = \palibf{ajj\=a}\footnote{\pali{aja gatikkhepane}. So, this should mean stopping, not going.} (stop)\par
\pali{sada + ja} = \palibf{sajj\=a} (sitting)\par

\subparagraph*{\pali{Ju, u\d ta}} (Mogg\,7.45--6)\label{pacckx:ju}\label{pacckx:udta1}

\pali{ara + ju/u\d ta} = \palibf{uju} (straight)\par
\pali{rudha? + ju} = \palibf{rajju} (rope)\par
\pali{mana + ju} = \palibf{ma\~nju} (charming)\par

\subparagraph*{\pali{Jhaka}} (Mogg\,7.47--8)\label{pacckx:jhaka}

\pali{gidha? + jhaka} = \palibf{gijjha} (vulture)\par
\pali{vana + jhaka} = \palibf{va\~njha/va\~njh\=a} (barren [tree/woman])\par
\pali{sa\d mja? + jhaka} = \palibf{sajjha} (silver)\par

\subparagraph*{\pali{\~Na}} (Mogg\,7.49--50)\label{pacckx:dna1}

\pali{kama + \~na} = \palibf{ka\~n\~n\=a} (girl)\par
\pali{yaja + \~na} = \palibf{ya\~n\~na} (sacrifice)\par
\pali{pu/pu\d na + \~na} = \palibf{pu\~n\~na} (merit)\par

\subparagraph*{\pali{A\~n\~na}} (Mogg\,7.51)\label{pacckx:aynyna}

\pali{ara + a\~n\~na} = \palibf{ara\~n\~na} (forest)\par
\pali{h\=a + a\~n\~na} = \palibf{hira\~n\~na} (gold)\par

\subparagraph*{\pali{A\d ta}} (Mogg\,7.53)\label{pacckx:adta}

\pali{saka + a\d ta} = \palibf{saka\d ta} (cart, wagon)\par
\pali{kasa + a\d ta} = \palibf{kasa\d ta} (nasty)\par
\pali{kara + a\d ta} = \palibf{kara\d ta} (crow)\par
\pali{makka? + a\d ta} = \palibf{makka\d ta} (monkey)\par
\pali{deva + a\d ta} = \palibf{deva\d ta} (sage Deva\d ta)\par
\pali{kama + a\d ta} = \palibf{kama\d ta} (dwarf)\par

\subparagraph*{\pali{U\d ta, \=a\d ta\d na, \=a\d ta, ku\d taka}} (Mogg\,7.54)\label{pacckx:udta2}\label{pacckx:aadtadna}\label{pacckx:aadta}\label{pacckx:kudtaka}

The markers of \pali{\d na} and \pali{ka} show that whether vowel \pali{vuddhi} will be applied or not.

\pali{ma\d mki? + u\d ta} = \palibf{maku\d ta} (crown)\par
\pali{ava + \=a\d ta\d na} = \palibf{\=av\=a\d ta} (pit)\par
\pali{ku + \=a\d ta} = \palibf{kav\=a\d ta} (window)\par
\pali{kuka + ku\d taka} = \palibf{kukku\d ta} (cock)\par

\subparagraph*{\pali{K\=i\d ta}} (Mogg\,7.52)\label{pacckx:kiidta}

\pali{kira + k\=i\d ta} = \palibf{kir\=i\d ta} (crown)\par
\pali{tara + k\=i\d ta} = \palibf{tir\=i\d ta} (garment for wrap)\par

\subparagraph*{\pali{\d Tha}} (Kacc\,672, R\=upa\,682, Sadd\,1318, Mogg\,7.55--6)\label{pacckx:dtha1}

\pali{ku\d ta + \d tha} = \palibf{ku\d t\d tha}\footnote{In Kacc\,663, R\=upa\,673, Sadd\,1305, this instance is the product of \pali{ku\d thi + ka}. In Mogg\,7.56 the root of this is \pali{kusa}. When used as nt.\ it means the disease, when used as m.\ it means the person who has the disease.} (leprosy)\par
\pali{ku\d ta + \d tha} = \palibf{ko\d t\d tha}\footnote{In Mogg\,7.55 the root of this is \pali{kusa}.} (store room)\par
\pali{ka\d ta + \d tha} = \palibf{ka\d t\d tha}\footnote{In Mogg\,7.55 the root of this is \pali{kasa}.} (timber)\par
\pali{kama + \d tha} = \palibf{ka\d n\d tha} (neck)\par
\pali{usa + \d tha} = \palibf{o\d t\d tha} (mouth, camel)\par
\pali{ku\d na + \d tha} = \palibf{ku\d n\d tha} (blunt)\par
\pali{da\d msa + \d tha} = \palibf{d\=a\d th\=a} (fang)\par
\pali{kama + \d tha} = \palibf{kama\d tha} (begging bowl, dwarf, turtle)\par
\pali{phassa? + \d tha} = \palibf{phu\d t\d tha}\footnote{According to Sadd-Dh\=a this should be from root \pali{phusa}.} (touch)\par

\subparagraph*{\pali{A\d n\d da}} (Mogg\,7.57)\label{pacckx:adndda}

\pali{vara + a\d n\d da} = \palibf{vara\d n\d da} (pimple)\par
\pali{kara + a\d n\d da} = \palibf{kara\d n\d da} (casket)\par

\subparagraph*{\pali{\d Dha, \d d\d dha, \d tha, \d t\d tha}} (Kacc\,659, R\=upa\,669, Sadd\,1299--300)\label{pacckx:ddha}\label{pacckx:ddddha}\label{pacckx:dtha2}\label{pacckx:dtdtha}

\pali{usu + \d dha/\d d\d dha} = \palibf{u\d d\d dha} (heat)\par
\pali{da\d msa + \d dha} = \palibf{da\d d\d dha}\footnote{In Sadd\,1300, it is suggested that the term should be from root \pali{daha} with certain transformation. See also Kacc\,576, R\=upa\,607, Sadd\,1179, Mogg\,5.146.} (burning)\par
\pali{ranja + \d tha/\d t\d tha} = \palibf{ra\d t\d tha} (country)\par

\subparagraph*{\pali{\d Na}} (Mogg\,7.65)\label{pacckx:dna2}

\pali{ku + \d na} = \palibf{ko\d na} (corner)\par
\pali{su + \d na} = \palibf{so\d na}\footnote{In Kacc\,647, R\=upa\,663, the term comes from root \pali{suna} and then transforms to \pali{su\d na, sv\=ana, suv\=ana, s\=una, sunakha, suna, s\=a,} and \pali{s\=ana.} In Sadd\,1285, \pali{s\=u\d na} and \pali{su\d na} are given instead of \pali{s\=una} and \pali{suna}. In Sadd\,1286, another line of thought is proposed, i.e.\ \pali{so\d na = su + o\d na}, \pali{sv\=ana = su + v\=ana}, and \pali{suv\=ana = su + uv\=ana}.} (dog)\par
\pali{du + \d na} = \palibf{do\d na} (1/8 bushel)\par
\pali{vara + \d na} = \palibf{va\d n\d na} (color)\par
\pali{kara + \d na} = \palibf{ka\d n\d na} (ear)\par
\pali{pa\d na + \d na} = \palibf{pa\d n\d na} (leaf)\par
\pali{t\=a + \d na} = \palibf{t\=a\d na} (protection)\par
\pali{l\=i + \d na} = \palibf{le\d na} (cave)\par

\subparagraph*{\pali{\d Naka}} (Mogg\,7.66--7)\label{pacckx:dnaka}

Marked by \pali{ka}, the vowel \pali{vuddhi} is not applied here. Also \pali{\d na} is retained.

\pali{su + \d naka} = \palibf{su\d na} (dog)\par
\pali{v\=i + \d naka} = \palibf{v\=i\d n\=a} (lute)\par
\pali{tija + \d naka} = \palibf{ti\d na} (grass)\par
\pali{l\=i + \d naka} = \palibf{lo\d na} (salt)\par
\pali{gamu + \d naka} = \palibf{go\d na} (ox)\par
\pali{hara + \d naka} = \palibf{hari\d na} (deer)\par
\pali{\=ira + \d naka} = \palibf{iri\d na} (barren soil)\par
\pali{thu + \d naka} = \palibf{th\=u\d na} (city)\par

\subparagraph*{\pali{A\d na}} (Mogg\,7.68)\label{pacckx:adna}

A bit confusing, vowel \pali{vuddhi} by \pali{\d na} is prevented by the leading \pali{a}.

\pali{rava? + a\d na} = \palibf{rava\d na} (cuckoo)\par
\pali{vara + a\d na} = \palibf{vara\d na} (wall)\par
\pali{p\=ura + a\d na} = \palibf{p\=ura\d na} (filling)\par

\subparagraph*{\pali{Y\=a\d na, l\=a\d na}} (Kacc\,633, R\=upa\,657, Sadd\,1242)\label{pacckx:yaadna}\label{pacckx:laadna}

\pali{kala + y\=a\d na} = \palibf{kaly\=a\d na} (goodness)\par
\pali{kala + l\=a\d na} = \palibf{kall\=a\d na} (good person)\par
\pali{pati + sala + y\=a\d na} = \palibf{pa\d tisaly\=a\d na} (seclusion)\par
\pali{pati + sala + l\=a\d na} = \palibf{pa\d tisall\=a\d na}\footnote{In Sadd\,1242, alternatively this can come from \pali{pati + sa\d m + l\=i + yu}.} (seclusion)\par

\subparagraph*{\pali{Kkhi\d na}} (Sadd\,1344)\label{pacckx:kkhidna}

\pali{d\=a + kkhi\d na} = \palibf{dakkhi\d na}\footnote{\pali{d\=atabb\=a dakkhi\d n\=a.}} (oblation)\par

\subparagraph*{\pali{I\d na, ki\d na}} (Sadd\,1345, Mogg\,7.60)\label{pacckx:idna2}\label{pacckx:kidna}

\pali{dakkha + i\d na} = \palibf{dakkhi\d na}\footnote{\pali{dakkhanti va\d d\d dhanti satt\=a et\=ay\=ati dakkhi\d n\=a,} from Sadd\,1245.} (oblation)\par
\pali{dakkha + ki\d na} = \palibf{dakkhi\d na}\footnote{\pali{dakkhati vuddhi\d m gacchati et\=ay\=ati dakkhi\d n\=a kusala\d m,} from Mogg\,7.60} (goodness)\par
\pali{tija + ki\d na} = \palibf{tikhi\d na} (sharp)\par
\pali{kasa + ki\d na} = \palibf{kasi\d na} (whole, no remaining)\par
\pali{tasa + ki\d na} = \palibf{tasi\d na} (craving)\par

\subparagraph*{\pali{\d Ni}} (Mogg\,7.61)\label{pacckx:dni}

\pali{v\=i + \d ni} = \palibf{ve\d ni} (braid of hair)\par
\pali{si + \d ni} = \palibf{se\d ni} (guild)\par
\pali{ni + si + \d ni} = \palibf{nise\d ni} (stairs)\par
\pali{su + \d ni} = \palibf{so\d ni} (the hip, waist)\par
\pali{du + \d ni} = \palibf{do\d ni} (boat)\par
\pali{k\=i + \d ni} = \palibf{ke\d ni} (buying)\par
\pali{s\=a + \d ni} = \palibf{s\=a\d ni} (curtain, screen)\par

\subparagraph*{\pali{A\d ni}} (Mogg\,7.62)\label{pacckx:adni}

This group has no \pali{vuddhi}.

\pali{gaha + a\d ni} = \palibf{gaha\d ni} (gestation, digestion)\par
\pali{ara + a\d ni} = \palibf{ara\d ni} (wood used for kindling)\par
\pali{dhara + a\d ni} = \palibf{dhara\d ni} (ground)\par
\pali{sara + a\d ni} = \palibf{sara\d ni} (path)\par
\pali{tara + a\d ni} = \palibf{tara\d ni} (ship, the sun)\par

\subparagraph*{\pali{Ru\d na}} (Sadd\,1321--3)\label{pacckx:rudna}

\pali{kara/kira + ru\d na} = \palibf{karu\d n\=a}\footnote{In Sadd\,1322, this may come from \pali{ka + rudhi + \d na}. In Mogg\,7.101, this comes from \pali{kara + kuna}, se below.} (compassion)\par

\subparagraph*{\pali{\d Nu}} (Kacc\,671, R\=upa\,681, Sadd\,1317, Mogg\,7.63--4)\label{pacckx:dnu2}

\pali{hana + \d nu} = \palibf{ha\d nu/hanu}\footnote{In Mogg\,7.2, \pali{hanu} is the product of \pali{hana + u}.} (jaw)\par
\pali{jana + \d nu} = \palibf{j\=a\d nu} (knee)\par
\pali{bh\=a + \d nu} = \palibf{bh\=a\d nu/bh\=anu} (the sun)\par
\pali{ri + \d nu} = \palibf{re\d nu} (dust, pollen)\par
\pali{khanu + \d nu} = \palibf{kh\=a\d nu} (stump)\par
\pali{ama + \d nu} = \palibf{a\d nu}\footnote{In Mogg\,7.2, this is the product of \pali{a\d na + u}.} (particle, atom)\par
\pali{ve + \d nu} = \palibf{ve\d nu} (bamboo)\par

\subparagraph*{\pali{Tu}} (Kacc\,667, R\=upa\,677, Sadd\,1313; Kacc\,671, R\=upa\,681, Sadd\,1317; Mogg\,7.70--1)\label{pacckx:tu}

\pali{sasu + tu} = \palibf{sattu} (enemy)\par
\pali{dh\=a + tu} = \palibf{dh\=atu} (element)\par
\pali{si + tu} = \palibf{setu} (bridge)\par
\pali{ki + tu} = \palibf{ketu} (flag)\par
\pali{hi + tu} = \palibf{hetu} (cause)\par
\pali{tana? + tu} = \palibf{tantu} (string)\par
\pali{jana + tu} = \palibf{jantu} (creature)\par
\pali{jara + tu} = \palibf{jattu} (shoulder)\par
\pali{gamu + tu} = \palibf{gantu} (goer)\par
\pali{saca + tu} = \palibf{sattu} (parched flour)\par
\pali{ara + tu} = \palibf{utu} (season)\par

\subparagraph*{\pali{Ratu}} (Mogg\,7.73)\label{pacckx:ratu}

This is actually \pali{tu} with \pali{ra} which entailed the last syllable deletion.

\pali{jana + ratu} = \palibf{jatu} (sealing wax)\par
\pali{kara + ratu} = \palibf{katu} (sacrifice)\par

\subparagraph*{\pali{Unta}} (Mogg\,7.74)\label{pacckx:unta}

\pali{saka + unta} = \palibf{sakunta} (bird)\par

\subparagraph*{\pali{Ota}} (Mogg\,7.75)\label{pacckx:ota}

\pali{kapa + ota} = \palibf{kapota/kapo\d ta} (pigeon)\par

\subparagraph*{\pali{Anta}} (Mogg\,7.76--7)\label{pacckx:anta}

\pali{vasa + anta} = \palibf{vasanta} (spring season)\par
\pali{ruha + anta} = \palibf{ruhanta} (tree)\par
\pali{bhadda + anta} = \palibf{bhadanta} (venerable person)\par
\pali{nanda + anta} = \palibf{nandant\=i} (female friend)\par
\pali{j\=iva + anta} = \palibf{j\=ivant\=i} (medicine)\par
\pali{su + anta} = \palibf{savant\=i} (river)\par
\pali{ruda + anta} = \palibf{rodant\=i} (medicine)\par
\pali{ava + anta} = \palibf{avant\=i} (a country)\par
\pali{hi + anta} = \palibf{hemanta} (winter)\par
\pali{s\=i + anta} = \palibf{s\=imanta}\footnote{\pali{sayanti ettha \=uk\=a kusum\=adayo c\=ati s\=imanto kesamaggo.} This means a place in hair that a flower can be put on, or louses can live in.} (path in hair)\par

\subparagraph*{\pali{Ita}} (Mogg\,7.78)\label{pacckx:ita}

\pali{hara + ita} = \palibf{harita} (green, vegetable)\par
\pali{ruha + ita} = \palibf{rohita} (a kind of fish)\par
\pali{ruha + ita} = \palibf{lohita} (blood)\par
\pali{gula + ita} = \palibf{kolita} (a name)\par

\subparagraph*{\pali{Ata}} (Mogg\,7.79)\label{pacckx:ata}

\pali{bhara + ata} = \palibf{bharata} (actor)\par
\pali{ra\d mja? + ata} = \palibf{rajata} (silver)\par
\pali{yaja + ata} = \palibf{yajata} (fire)\par
\pali{paca + ata} = \palibf{pacata} (cook)\par

\subparagraph*{\pali{\=Ataka}} (Mogg\,7.80)\label{pacckx:aataka}

The marker \pali{ka} confirms that no \pali{vuddhi} is applied here.

\pali{kira + \=ataka} = \palibf{kir\=ata} (jungleman)\par
\pali{ala + \=ataka} = \palibf{al\=ata} (firebrand)\par
\pali{cila + \=ataka} = \palibf{cil\=ata} (a kind of fish)\par

\subparagraph*{\pali{Ta, tra\d n, atta, taka}} (Kacc\,656, R\=upa\,666, Sadd\,1295--6, Mogg\,7.81--4)\label{pacckx:ta}\label{pacckx:tradn}\label{pacckx:atta}\label{pacckx:taka}

In Sadd\,1296, Aggava\d msa seems to disagree with the use of \pali{tra\d n}. Perhaps, it looks too much like Sanskrit and it is rarely found in the scriptures. In Mogg, the Sanskrit-like forms are not mentioned, but to be more precise there are three \pali{paccaya}s in this group: \pali{ta, atta,} and \pali{taka} (= \pali{ta} without \pali{vuddhi}).

\pali{chada + ta/tra\d n} = \palibf{chatta/chatra} (umbrella)\par
\pali{cinta + ta/tra\d n} = \palibf{citta/citra} (mind)\par
\pali{su + ta/tra\d n} = \palibf{sutta/sutra} (thread)\par
\pali{n\=i + ta/tra\d n} = \palibf{netta/netra} (thread)\par
\pali{pa+vida/pu + ta/tra\d n} = \palibf{pavitta/pavitra} (cleanness)\par
\pali{pada/pata + ta/tra\d n} = \palibf{patta/patra} (bowl)\par
\pali{tanu + ta/tra\d n} = \palibf{tanta/tantra} (thread)\par
\pali{yata + ta/tra\d n} = \palibf{yatta/yatra} (effort)\par
\pali{ada? + ta/tra\d n} = \palibf{atta/atra}\footnote{In Mogg\,7.82 \pali{atta} comes from \pali{ata + ta}.} (self)\par
\pali{mada + ta/tra\d n} = \palibf{matta/matra} (intoxicated)\par
\pali{yuja + ta/tra\d n} = \palibf{yotta/yotra} (rope)\par
\pali{vata + ta/tra\d n} = \palibf{vatta/vatra}\footnote{In Mogg\,7.83 \pali{vatta} comes from \pali{vara + taka}.} (duty)\par
\pali{mida + ta/tra\d n} = \palibf{mitta/mitra} (friend)\par
\pali{mida + ta/tra\d n} = \palibf{mett\=a/metr\=a} (friendliness)\par
\pali{m\=a + ta/tra\d n} = \palibf{matt\=a/matr\=a} (measure)\par
\pali{pu + ta/tra\d n} = \palibf{putta/putra} (child, son)\par
\pali{kala + ta/tra\d n} = \palibf{kalatta/kalatra} (wife)\par
\pali{vara + ta/tra\d n} = \palibf{varatta/varatra} (strap)\par
\pali{vepu? + ta/tra\d n} = \palibf{vetta/vetra} (cane, twig)\par
\pali{gupa + ta/tra\d n} = \palibf{gutta/gutra/gotta/gotra} (thing worth protecting)\par
\pali{d\=a + ta/tra\d n} = \palibf{d\=atta/d\=atra} (sickle)\par
\pali{ama + atta} = \palibf{amatta} (small earthen vessel)\par
\pali{v\=a + ta} = \palibf{v\=ata} (wind)\par
\pali{t\=a + ta} = \palibf{t\=ata} (father)\par
\pali{dama + ta} = \palibf{danta} (tooth)\par
\pali{ama + ta} = \palibf{anta} (end, intestine)\par
\pali{si + ta} = \palibf{seta} (white)\par
\pali{su + ta} = \palibf{sota} (the ear, stream)\par
\pali{pu + ta} = \palibf{pota} (child)\par
\pali{gaha + ta} = \palibf{gatta} (body)\par
\pali{ata + ta} = \palibf{att\=a} (self)\par
\pali{khipa + ta} = \palibf{khetta} (field, plot of land)\par
\pali{ghara + taka} = \palibf{ghata} (ghee)\par
\pali{si + taka} = \palibf{sita} (white)\par
\pali{d\=u? + taka} = \palibf{d\=uta} (envoy)\par
\pali{vida + taka} = \palibf{vitta} (wealth, property)\par
\pali{kara + taka} = \palibf{kutta} (action)\par
\pali{kama + taka} = \palibf{kunta} (lance)\par
\pali{su + rama + taka} = \palibf{surata} (well-living person)\par
\pali{p\=ala + taka} = \palibf{palita} (grey hair)\par
\pali{mhi + taka} = \palibf{mihita/sita} (smile)\par
\pali{kusa + taka} = \palibf{kus\=ita} (lazy)\par
\pali{si + taka} = \palibf{s\=it\=a} (furrow)\par

\subparagraph*{\pali{\d Nitta}} (Kacc\,657, R\=upa\,667, Sadd\,1297)\label{pacckx:dnitta}

This \pali{paccaya} signifies group.

\pali{vada + \d nitta} = \palibf{v\=aditta} (musical band)\par
\pali{cara + \d nitta} = \palibf{c\=aritta} (custom, group of practices)\par
\pali{vara + \d nitta} = \palibf{v\=aritta} (group of guards)\par

\subparagraph*{\pali{Tti, ti}} (Kacc\,658, R\=upa\,668, Sadd\,1298)\label{pacckx:tti}\label{pacckx:ti}

\pali{mida + tti} = \palibf{metti} (love)\par
\pali{pada + tti} = \palibf{patti} (foot-soldier)\par
\pali{ranja + tti} = \palibf{ratti} (night)\par
\pali{tanu + ti} = \palibf{tanti} (secret text)\par
\pali{dh\=a + ti} = \palibf{dh\=ati} (nanny)\par

\subparagraph*{\pali{Tha, atha, thaka}} (Kacc\,628, R\=upa\,653, Sadd\,1236, Mogg\,7.85--88; Kacc\,660, R\=upa\,670, Sadd\,1301)\label{pacckx:tha}\label{pacckx:atha}\label{pacckx:thaka}

In Mogg \pali{atha} and \pali{thaka} are given. The former retains the root forms, whereas the latter can cause certain transformation.

\pali{samu + tha/atha} = \palibf{samatha} (calm)\par
\pali{dama + tha/atha} = \palibf{damatha} (training)\par
\pali{dara + tha/atha} = \palibf{daratha} (anxiety)\par
\pali{raha + tha} = \palibf{ratha}\footnote{In Mogg\,7.87 this is the product of \pali{rama + thaka}.} (car)\par
\pali{sapa + tha/atha} = \palibf{sapatha} (oath)\par
\pali{\=a + vasa + tha/atha} = \palibf{\=avasatha} (dwelling)\par
\pali{yu + tha/thaka} = \palibf{y\=utha} (herd)\par
\pali{kilama? + atha} = \palibf{kilamatha} (weariness)\par
\pali{upa + vasa + atha} = \palibf{uposatha} (Buddhist Sabbath day)\par
\pali{tara + thaka} = \palibf{tittha} (harbor)\par
\pali{sica + thaka} = \palibf{sittha} (beeswax)\par
\pali{hasa + thaka} = \palibf{hattha} (hand)\par
\pali{ge + thaka} = \palibf{g\=ath\=a} (verse)\par
\pali{ara + thaka} = \palibf{attha} (wealth)\par
\pali{gupa + thaka} = \palibf{g\=utha} (excretion)\par
\pali{s\=u + tha} = \palibf{sattha}\footnote{See also Mogg\,5.144.} (weapon)\par
\pali{vu + tha} = \palibf{vattha} (cloth)\par
\pali{asa + tha} = \palibf{attha} (meaning)\par

\subparagraph*{\pali{Thu, athu}} (Kacc\,644, R\=upa\,661, Sadd\,1271, Mogg\,5.46, 7.89)\label{pacckx:thu}\label{pacckx:athu}

\pali{vepu? + thu/athu} = \palibf{vepathu} (a sickness causing shivering)\par
\pali{s\=i + thu/athu} = \palibf{sayathu} (a sickness causing swelling)\par
\pali{dava? + thu/athu} = \palibf{davathu} (a sickness causing heat)\par
\pali{vamu + thu/athu} = \palibf{vamathu} (a sickness causing vomiting)\par
\pali{vasa + thu} = \palibf{vatthu} (matter, story)\par
\pali{masa + thu} = \palibf{matthu} (clear liquid of curd)\par
\pali{kusa + thu} = \palibf{kotthu} (jackal)\par

\subparagraph*{\pali{Thi}} (Mogg\,7.90)\label{pacckx:thi}

\pali{saka + thi} = \palibf{satthi} (thigh)\par
\pali{vasa + thi} = \palibf{vatthi} (bladder)\par

\subparagraph*{\pali{Thika}} (Mogg\,7.91)\label{pacckx:thika}

This is actually \pali{thi} without \pali{vuddhi}.

\pali{v\=i + thika} = \palibf{v\=ithi} (street)\par

\subparagraph*{\pali{Rathi}} (Mogg\,7.92)\label{pacckx:rathi}

\pali{sara + rathi} = \palibf{s\=arathi} (driver)\par

\subparagraph*{\pali{Ithi}} (Mogg\,7.93)\label{pacckx:ithi}

\pali{t\=a + ithi} = \palibf{tithi} (lunar day)\par
\pali{ata + ithi} = \palibf{atithi} (guest)\par

\subparagraph*{\pali{Th\=i}} (Mogg\,7.94)\label{pacckx:thii}

\pali{isa + th\=i} = \palibf{itth\=i} (woman)\par

\subparagraph*{\pali{Da, idda, daka}} (Kacc\,661, R\=upa\,671, Sadd\,1302, Mogg\,7.95--6)\label{pacckx:da}\label{pacckx:idda}\label{pacckx:daka}

In Mogg \pali{da} is called \pali{daka} instead, to mark that no \pali{vuddhi} will be applied.

\pali{sa\d m + udi + da/daka} = \palibf{samudda} (ocean)\par
\pali{idi + da} = \palibf{inda} (king, ruler)\par
\pali{cadi + da} = \palibf{canda} (the moon)\par
\pali{madi + da} = \palibf{manda} (little)\par
\pali{khuda + da/daka} = \palibf{khudda} (little)\par
\pali{chidi + da/daka} = \palibf{chidda} (hole)\par
\pali{ruda + da} = \palibf{rudda} (cruel)\par
\pali{dala + idda} = \palibf{dalidda} (poor)\par
\pali{ruda + daka} = \palibf{rudda} (a deity)\par
\pali{muda + daka} = \palibf{mudd\=a} (engraved ring)\par
\pali{mada + daka} = \palibf{madda} (a country)\par
\pali{s\=uda + daka} = \palibf{sudda} (S\=udra caste)\par
\pali{sapa + daka} = \palibf{sadda} (sound)\par
\pali{kama + daka} = \palibf{kanda} (tuber)\par
\pali{kama + daka} = \palibf{kunda} (jusmine)\par
\pali{mana + daka} = \palibf{manda} (stupid)\par
\pali{vu\d na? + daka} = \palibf{bunda} (root)\par
\pali{ninda? + daka} = \palibf{nidd\=a} (sleep)\par
\pali{unda? + daka} = \palibf{udda} (otter)\par
\pali{pula + daka} = \palibf{pulinda} (savage)\par

\subparagraph*{\pali{Du}} (Kacc\,667, R\=upa\,677, Sadd\,1313, Mogg\,7.97)\label{pacckx:du}

\pali{dada? + du} = \palibf{daddu} (a skin eruption)\par
\pali{ada? + du} = \palibf{addu} (jail)\par
\pali{mada + du} = \palibf{maddu} (drunkard)\par

\subparagraph*{\pali{Dha}} (Kacc\,661, R\=upa\,671, Sadd\,1302, Mogg\,7.98--9)\label{pacckx:dha2}

\pali{ranja + dha} = \palibf{randha}\footnote{In Mogg\,7.98 the root of this instance is \pali{rama}.} (hole, cleft)\par
\pali{dama + dha} = \palibf{dandha} (foolish person)\par
\pali{muda + dha} = \palibf{muddh\=a} (the head)\par
\pali{ara + dha} = \palibf{addh\=a} (path, time)\par
\pali{gidha + dha} = \palibf{gaddha} (vulture)\par
\pali{vidha + dha} = \palibf{viddha} (clean)\par

\subparagraph*{\pali{Dhuka}} (Mogg\,7.100)\label{pacckx:dhuka}

\pali{s\=i + dhuka} = \palibf{s\=idhu} (a kind of liquor)\par

\subparagraph*{\pali{Kuna}} (Mogg\,7.101)\label{pacckx:kuna}

\pali{vara + kuna} = \palibf{varu\d na} (a deity)\par
\pali{ara + kuna} = \palibf{aru\d na} (the sun)\par
\pali{kara + kuna} = \palibf{karu\d n\=a}\footnote{In Sadd\,1321, this is the product of \pali{kara/kira + ru\d na}.} (compassion)\par
\pali{tara + kuna} = \palibf{taru\d na} (youth)\par
\pali{dara + kuna} = \palibf{t\=aru\d na} (cruel)\par
\pali{yama + kuna} = \palibf{yamuna} (a river)\par
\pali{ajja + kuna} = \palibf{ajjuna} (a kind of plant)\par
\pali{mitha? + kuna} = \palibf{mithuna} (sexual couple)\par
\pali{saka + kuna} = \palibf{sakuna/saku\d na/saku\d n\=i} (bird)\par

\subparagraph*{\pali{Kana}} (Mogg\,7.104)\label{pacckx:kana}

\pali{kira? + kana} = \palibf{kira\d n\=a} (ray)\par

\subparagraph*{\pali{Na}} (Mogg\,7.106--7)\label{pacckx:na}

\pali{si + na} = \palibf{sena/sen\=a} (hawk/army)\par
\pali{dh\=a + na} = \palibf{dh\=an\=a} (popped rice)\par
\pali{v\=i + na} = \palibf{vena} (ignoble person)\par
\pali{v\=a + na} = \palibf{v\=ana} (craving)\par
\pali{\=uha + na} = \palibf{\=una} (deficient)\par
\pali{hi + na} = \palibf{h\=ina} (inferior, despicable)\par
\pali{ci + na} = \palibf{c\=ina} (a country)\par
\pali{hana + na} = \palibf{jaghana} (loin, buttocks)\par
\pali{\d th\=a + na} = \palibf{thena} (thief)\par
\pali{unda + na} = \palibf{odana} (boiled rice)\par
\pali{ra\d mja + na} = \palibf{rajana} (color)\par
\pali{ra\d mja + na} = \palibf{rajan\=i} (night)\par
\pali{pada + na} = \palibf{pajjunna} (cloud, rain-god)\par
\pali{gama + na} = \palibf{gagana} (sky)\par

\subparagraph*{\pali{Tana}} (Mogg\,7.108)\label{pacckx:tana}

\pali{v\=i + tana} = \palibf{vetana} (wage)\par
\pali{pata + tana} = \palibf{pattana} (city)\par

\subparagraph*{\pali{Tanaka}} (Mogg\,7.109)\label{pacckx:tanaka}

\pali{rama + tanaka} = \palibf{ratana} (jewel, cubit)\par

\subparagraph*{\pali{Nu, nuka}} (Kacc\,671, R\=upa\,681, Sadd\,1317, Mogg\,7.110--1)\label{pacckx:nu}\label{pacckx:nuka}

\pali{dhe + nu} = \palibf{dhenu}\footnote{In Mogg\,7.111 the root of this is \pali{dh\=a}.} (cow)\par
\pali{s\=u + nuka} = \palibf{s\=unu} (child)\par
\pali{bh\=a + nuka} = \palibf{bh\=unu} (the sun)\par

\subparagraph*{\pali{\=Ani}} (Kacc\,645, R\=upa\,662, Sadd\,1281)\label{pacckx:aani}

This \pali{paccaya} is used to reproach with negative sense, for example, \pali{agam\=ani te jamma deso} (Bad guy, that place is not [for you] to go). Aggava\d msa explains further that with \pali{\=ani} the terms are used as indeclinables, i.e.\ their forms are retained for sg.\ and pl.\ and all all genders, like \pali{seyyo}. If it is not a reproach, \pali{\=ani} is not appplied, so as when \pali{na} is not present.

\pali{na + gamu + \=ani} = \palibf{agam\=ani}\footnote{\pali{agam\=ani = na gamitabbo.}} (not [good] to go)\par
\pali{na + kara + \=ani} = \palibf{akar\=ani}\footnote{\pali{akar\=ani = na kattabba\d m.}} (not [good] to do)\par

\subparagraph*{\pali{Ani}} (Mogg\,7.112)\label{pacckx:ani}

\pali{vatta? + ani} = \palibf{vattani} (shuttle stem)\par
\pali{vatta? + ani} = \palibf{vattan\=i} (path)\par
\pali{a\d ta + ani} = \palibf{a\d tani} (frame of a bed)\par
\pali{ava + ani} = \palibf{avani} (ground)\par
\pali{dhama + ani} = \palibf{dhamani} (vein)\par
\pali{asa + ani} = \palibf{asani} (thunderbolt)\par

\subparagraph*{\pali{Ni}} (Mogg\,7.113)\label{pacckx:ni}

\pali{yu + ni} = \palibf{yoni} (female genital)\par

\subparagraph*{\pali{Pa}} (Mogg\,7.114--5)\label{pacckx:pa}

\pali{cama + pa} = \palibf{camp\=a} (a city)\par
\pali{apa + pa} = \palibf{appa} (small)\par
\pali{p\=a + pa} = \palibf{p\=apa} (evil)\par
\pali{vapa + pa} = \palibf{vappa} (arable land)\par
\pali{yu + pa} = \palibf{y\=upa} (sacrificial post)\par
\pali{thu + pa} = \palibf{th\=upa} (pagoda)\par
\pali{ku + pa} = \palibf{k\=upa} (well)\par

\subparagraph*{\pali{Paka}} (Mogg\,7.116--7)\label{pacckx:paka}

\pali{khipa + paka} = \palibf{khippa} (quick)\par
\pali{supa + paka} = \palibf{suppa} (winnowing basket)\par
\pali{n\=i + paka} = \palibf{n\=ipa} (a kind of tree)\par
\pali{s\=u + paka} = \palibf{s\=upa} (curry)\par
\pali{p\=u + paka} = \palibf{p\=upa} (cake)\par
\pali{sapa + paka} = \palibf{sippa} (craft, art)\par
\pali{vapa + paka} = \palibf{vippa} (brahman)\par
\pali{vama + paka} = \palibf{bappa} (tear)\par
\pali{chupa? + paka} = \palibf{cheppa} (tail)\par
\pali{rupa? + paka} = \palibf{r\=upa} (form)\par

\subparagraph*{\pali{Apa}} (Mogg\,7.118--9)\label{pacckx:apa}

\pali{s\=asa + apa} = \palibf{s\=asapa} (mustard seed)\par
\pali{va\d ta + apa} = \palibf{vi\d tapa} (branch, fork of a tree)\par
\pali{kutha + apa} = \palibf{ku\d napa} (corpse)\par
\pali{ma\d n\d da? + apa} = \palibf{ma\d n\d dapa} (temporary shed or pavilion)\par

\subparagraph*{\pali{Pha}} (Mogg\,7.120)\label{pacckx:pha}

\pali{gupa + pha} = \palibf{goppha} (ankle)\par

\subparagraph*{\pali{Ba}} (Mogg\,7.121--2)\label{pacckx:ba}

\pali{gara + ba} = \palibf{gabba} (conceit)\par
\pali{sara + ba} = \palibf{sabba} (all)\par
\pali{ama + ba} = \palibf{amba} (mango)\par
\pali{ama + ba} = \palibf{amb\=a} (mother)\par
\pali{nama + ba} = \palibf{nimba} (margosa tree)\par
\pali{vama + ba} = \palibf{bimba} (the body)\par
\pali{kusa + ba} = \palibf{kosamba} (a kind of tree)\par
\pali{kada + ba} = \palibf{kadamba} (a kind of tree)\par
\pali{ku\d ta + ba} = \palibf{ku\d tumba} (family property)\par
\pali{ka\d n\d da? + ba} = \palibf{ku\d duba} (a kind of container)\par

\subparagraph*{\pali{Bi}} (Mogg\,7.123)\label{pacckx:bi}

\pali{dara + bi} = \palibf{dabbi} (spoon, ladle)\par

\subparagraph*{\pali{Abha}} (Mogg\,7.124)\label{pacckx:abha}

\pali{kara + abha} = \palibf{karabha} (the wrist, camel)\par
\pali{sara + abha} = \palibf{sarabha} (a kind of deer)\par
\pali{sala + abha} = \palibf{salabha} (grasshopper)\par
\pali{kala + abha} = \palibf{kalabha} (young elephant)\par
\pali{valla + abha} = \palibf{vallabha} (favourite)\par
\pali{vasa + abha} = \palibf{vasabha} (ox)\par

\subparagraph*{\pali{Rabha}} (Mogg\,7.125)\label{pacckx:rabha}

\pali{gada + rabha} = \palibf{gadrabha} (donkey)\par

\subparagraph*{\pali{Kabha}} (Mogg\,7.126)\label{pacckx:kabha}

\pali{usa + kabha} = \palibf{usabha} (noble)\par
\pali{r\=asa + kabha} = \palibf{r\=asabha} (donkey)\par

\subparagraph*{\pali{Bhaka}} (Mogg\,7.127)\label{pacckx:bhaka}

\pali{i + bhaka} = \palibf{ibha} (elephant)\par

\subparagraph*{\pali{Bha}} (Mogg\,7.128--9)\label{pacckx:bha}

\pali{gara + bha} = \palibf{gabbha} (room, womb)\par
\pali{ava + bha} = \palibf{abbha} (cloud)\par
\pali{sada + bha} = \palibf{sobbha} (pit, pool)\par
\pali{kama + bha} = \palibf{kumbha} (water pot)\par
\pali{kusa + bha} = \palibf{kusumbha} (safflower, gold)\par

\subparagraph*{\pali{Man, ma}} (Kacc\,627, R\=upa\,652, Sadd\,1234--5, Mogg\,7.136--7)\label{pacckx:man}\label{pacckx:ma1}

In Kacc/Sadd the marker \pali{n} is a sign of \pali{vuddhi}, but in Mogg it is seen as just \pali{ma}.

\pali{kh\=i + man/ma} = \palibf{khema} (full of peace)\par
\pali{bh\=i + man} = \palibf{bhema/bh\=ima} (demon)\par
\pali{su + man/ma} = \palibf{soma} (the moon)\par
\pali{ru + man} = \palibf{roma} (body hair)\par
\pali{hu + man/ma} = \palibf{homa} (oblation)\par
\pali{v\=a + man} = \palibf{v\=ama} (agreeable)\par
\pali{dh\=u + man} = \palibf{dh\=uma} (smoke)\par
\pali{hi + man/ma} = \palibf{hema} (gold)\par
\pali{l\=u + man/ma} = \palibf{loma} (body hair)\par
\pali{p\=i + man} = \palibf{pema} (love)\par
\pali{ada? + man} = \palibf{atta/\=atuma}\footnote{In Mogg\,7.82 \pali{atta} comes from \pali{ata + ta}.} (self)\par
\pali{v\=i + ma} = \palibf{vema} (shuttle)\par
\pali{g\=a + ma} = \palibf{g\=ama} (village)\par
\pali{s\=a + ma} = \palibf{s\=ama} (black)\par
\pali{khu + ma} = \palibf{khoma} (linen cloth)\par
\pali{mara + ma} = \palibf{mamma} (vital spot of the body)\par
\pali{dhara + ma} = \palibf{dhamma}\footnote{In Kacc\,531, R\=upa\,589, Sadd\,1113, \pali{dhamma} comes from \pali{dh\=a + ramma} and \pali{kamma} comes from \pali{kara + ramma}.} (Dhamma)\par
\pali{kara + ma} = \palibf{kamma} (action)\par
\pali{ghara + ma} = \palibf{ghamma} (heat, summer)\par
\pali{jama + ma} = \palibf{jamma} (degraded one)\par
\pali{ama + ma} = \palibf{amma} (mother)\par
\pali{sama + ma} = \palibf{samma} (my dear!)\par
\pali{asa + ma} = \palibf{asm\=a} (stone)\par
\pali{asa + ma} = \palibf{adhama} (ignoble)\par
\pali{visa + ma} = \palibf{vesma} (dwelling)\par
\pali{bh\=i + ma} = \palibf{bhesma} (cause of fear, terrible)\par
\pali{kara + ma} = \palibf{kumma} (turtle)\par

\subparagraph*{\pali{Ma, maka}} (Kacc\,628, R\=upa\,653, Sadd\,1236, Mogg\,7.134)\label{pacckx:ma2}\label{pacckx:maka}

In Kacc/Sadd this \pali{ma} does not entails vowel \pali{vuddhi}, but in Mogg it is given with the preventer \pali{ka} instead.

\pali{du + ma} = \palibf{duma} (tree)\par
\pali{hi + ma/maka} = \palibf{hima} (snow)\par
\pali{si + ma} = \palibf{s\=ima/s\=im\=a} (boundary)\par
\pali{bh\=i + ma/maka} = \palibf{bh\=ima} (demon)\par
\pali{d\=a + ma} = \palibf{d\=ama} (rope)\par
\pali{y\=a + ma} = \palibf{y\=ama} (time)\par
\pali{s\=a + ma} = \palibf{s\=ama} (gold)\par
\pali{\d th\=a + ma} = \palibf{th\=ama} (power)\par
\pali{bhasa + ma} = \palibf{bhasma} (ashes)\par
\pali{br\=uha + ma} = \palibf{brahma} (god Brahma)\par
\pali{usa + ma} = \palibf{usuma} (heat)\par
\pali{dh\=u + maka} = \palibf{dh\=uma} (smoke)\par

\subparagraph*{\pali{R\=isana}} (Mogg\,7.135)\label{pacckx:riisana}

\pali{bh\=i + r\=isana} = \palibf{bh\=isana} (demon)\par

\subparagraph*{\pali{Ama, ima}} (Kacc\,666, R\=upa\,676, Sadd\,1309--12, Mogg\,7.133)\label{pacckx:ama}\label{pacckx:ima}

\pali{putha + ama} = \palibf{puthuv\=i, pathav\=i, pa\d thav\=i} (the earth)\par
\pali{putha + ama} = \palibf{pathama, pa\d thama}\footnote{In Mogg\,7.133 the root of this is \pali{pa\d tha}.} (first, excellent)\par
\pali{cara + ima} = \palibf{carima}\footnote{Mogg\,7.133} (the last)\par

\subparagraph*{\pali{Ttima}} (Kacc\,644, R\=upa\,661, Sadd\,1272)\label{pacckx:ttima}

\pali{bh\=u + ttima} = \palibf{bhottima} (thing arising from existence)\par
\pali{ku + ttima} = \palibf{kuttima}\footnote{See also Sadd\,1275--6.} (thing arising from action, counterfeit)\par
\pali{d\=a + ttima} = \palibf{dattima} (thing arising from giving)\par

\subparagraph*{\pali{\d Nima}} (Kacc\,644, R\=upa\,661, Sadd\,1273)\label{pacckx:dnima}

\pali{o + hu + \d nima} = \palibf{oh\=avima} (thing arising from honoring)\par

\subparagraph*{\pali{Kuma}} (Mogg\,7.130--1)\label{pacckx:kuma}

\pali{usa + kuma} = \palibf{usuma} (heat)\par
\pali{kusa + kuma} = \palibf{kusuma} (flower)\par
\pali{pada + kuma} = \palibf{paduma} (lotus)\par
\pali{sukha + kuma} = \palibf{sukhuma} (fine, subtle)\par
\pali{vaja + kuma} = \palibf{va\d tuma} (path)\par
\pali{silisa + kuma} = \palibf{silesuma} (phlegm)\par
\pali{kama + kuma} = \palibf{ku\.nkuma} (saffron)\par

\subparagraph*{\pali{Uma}} (Mogg\,7.132)\label{pacckx:uma}

\pali{gudha + uma} = \palibf{godhuma} (wheat)\par

\subparagraph*{\pali{Mi}} (Mogg\,7.138--9)\label{pacckx:mi}

\pali{n\=i + mi} = \palibf{nemi} (rim of a wheel)\par
\pali{\=uha + mi} = \palibf{\=umi} (wave)\par
\pali{bh\=u + mi} = \palibf{bh\=umi} (ground)\par
\pali{rasa + mi} = \palibf{rasmi} (rope)\par

\subparagraph*{\pali{Tyu, \d t\d tu}} (Sadd\,1253, 1264)\label{pacckx:tyu}\label{pacckx:dtdtu}

\pali{musa + tyu/\d t\d tu} = \palibf{maccu/mu\d t\d tu}\footnote{In Mogg\,7.40, \pali{maccu} comes from \pali{mara + cu}.} (death)\par

\subparagraph*{\pali{Ratya}} (Sadd\,1254)\label{pacckx:ratya}

With \pali{ra} marker, the last syllable is deleted.

\pali{mara + ratya} = \palibf{macca}\footnote{In Mogg\,7.40, this comes from \pali{mara + ca}.} (human, the mortal)\par

\subparagraph*{\pali{Tya}} (Sadd\,1255, 1265, 1260)\label{pacckx:tya}

\pali{u + dh\=u + tya} = \palibf{uddhacca}\footnote{In Sadd\,1256, Aggava\d msa entertains that the term may be seen as a secondary derivative of \pali{uddhata + \d nya} (\pali{uddhatassa bh\=avo uddhacc\d m}).} (distraction, agitation)\par
\pali{ku + kara + tya} = \palibf{kukkucca}\footnote{In Sadd\,1258, 1261, this may a secondary derivative of \pali{kukata + \d nya}.} (remorse, worry)\par
\pali{sata? + tya} = \palibf{sacca}\footnote{In Mogg\,7.39, this comes from \pali{sara + ca}.} (truth)\par
\pali{nata + tya} = \palibf{nacca} (dancing)\par
\pali{niti? + tya} = \palibf{nicca} (permanent)\par

\subparagraph*{\pali{Ya}} (Kacc\,632, R\=upa\,656, Sadd\,1241, Mogg\,7.140--2)\label{pacckx:ya}

\pali{ala + ya} = \palibf{alya} (new, wet)\par
\pali{kala + ya} = \palibf{kalya} (comfortable, proper)\par
\pali{sala + ya} = \palibf{salya} (arrow)\par
\pali{m\=a + ya} = \palibf{m\=ay\=a} (fraud, jugglery)\par
\pali{ch\=a + ya} = \palibf{ch\=ay\=a} (shadow)\par
\pali{jana + ya} = \palibf{j\=ay\=a} (wife)\par
\pali{hara + ya} = \palibf{hadaya} (mind)\par
\pali{tana + ya} = \palibf{tanaya} (child)\par
\pali{sara + ya} = \palibf{s\=uriya} (the sun)\par
\pali{hara + ya} = \palibf{hammiya} (storied building)\par
\pali{kasa + ya} = \palibf{kisalaya} (young leaf, sprout)\par

\subparagraph*{\pali{Raka}} (Mogg\,7.143--6)\label{pacckx:raka}

The actual ending is \pali{ra} and \pali{ka} is a \pali{vuddhi} preventer.

\pali{kh\=i + raka} = \palibf{k\=ira} (milk)\par
\pali{si + raka} = \palibf{sira} (the head)\par
\pali{si + raka} = \palibf{sir\=a} (tendon, vein)\par
\pali{n\=i + raka} = \palibf{n\=ira} (water)\par
\pali{s\=i + raka} = \palibf{s\=ira} (plough)\par
\pali{su + raka} = \palibf{sur\=a} (liquor)\par
\pali{su + raka} = \palibf{sura} (deity)\par
\pali{su + raka} = \palibf{s\=ura} (the sun, hero)\par
\pali{v\=i + raka} = \palibf{v\=ira} (hero)\par
\pali{ku + raka} = \palibf{kura/k\=ura}\footnote{In Kacc\,670, R\=upa\,680, Sadd\,1316, \pali{k\=ura} comes from \pali{ku + \=ura}.} (boiled rice)\par
\pali{hi + raka} = \palibf{h\=ira} (diamond)\par
\pali{ci + raka} = \palibf{c\=ira} (bark)\par
\pali{du + raka} = \palibf{d\=ura}\footnote{In Kacc\,670, R\=upa\,680, Sadd\,1316, this comes from \pali{du + \=ura}.} (far)\par
\pali{mi + raka} = \palibf{m\=ira} (ocean)\par
\pali{dh\=a + raka} = \palibf{dh\=ira} (wise person)\par
\pali{t\=a + raka} = \palibf{t\=ira} (shore, riverbank)\par
\pali{bhadda? + raka} = \palibf{bhadra} (good, lucky)\par
\pali{bh\=i + raka} = \palibf{bher\=i} (drum)\par
\pali{vi + cita + raka} = \palibf{vicitra} (variegated)\par
\pali{y\=a + raka} = \palibf{y\=atr\=a} (travel, voyage)\par
\pali{gupa + raka} = \palibf{gotra} (clan)\par
\pali{bhasa + raka} = \palibf{bhastr\=a} (blower)\par
\pali{usa + raka} = \palibf{ura} (the chest)\par

\subparagraph*{\pali{\=Ura}} (Kacc\,670, R\=upa\,680, Sadd\,1316, Mogg\,7.171--2)\label{pacckx:uura}

\pali{vida + \=ura} = \palibf{vid\=ura/ved\=ura} (distant [village])\par
\pali{valla + \=ura} = \palibf{vall\=ura} (dried meat)\par
\pali{masa + \=ura} = \palibf{mas\=ura} (animal hide, a kind of grain)\par
\pali{sida + \=ura} = \palibf{sind\=ura} (red lead)\par
\pali{kapu + \=ura} = \palibf{kapp\=ura} (camphor)\par
\pali{ma + y\=a + \=ura} = \palibf{may\=ura} (peacock)\par
\pali{udi + \=ura} = \palibf{und\=ura} (rat)\par
\pali{khajja + \=ura} = \palibf{khajj\=ura/khajj\=ur\=i} (date palm)\par
\pali{kura + \=ura} = \palibf{kur\=ura}\footnote{In Mogg\,7.172 the root of this is \pali{kara}.} (cruel one)\par

\subparagraph*{\pali{Ura}} (Mogg\,7.147--8)\label{pacckx:ura}

\pali{manda? + ura} = \palibf{mandur\=a} (horse pen)\par
\pali{a\.nka? + ura} = \palibf{a\.nkura} (sprout, bud)\par
\pali{sasa + ura} = \palibf{sasura} (father-in-law)\par
\pali{asa + ura} = \palibf{asura} (demon)\par
\pali{matha + ura} = \palibf{mathura} (a city)\par
\pali{cata? + ura} = \palibf{catura} (clever)\par
\pali{vidha + ura} = \palibf{vidhura} (destitute, lonely)\par
\pali{unda? + ura} = \palibf{undura} (rat)\par
\pali{ma\.nka? + ura} = \palibf{makura} (mirror, car, powder, fish)\par
\pali{kuka + ura} = \palibf{kukkura} (dog)\par
\pali{ma\.nga? + ura} = \palibf{ma\.ngura} (a kind of fish)\par

\subparagraph*{\pali{Ira, kira}} (Kacc\,661, R\=upa\,671, Sadd\,1302, Mogg\,7.149--50)\label{pacckx:ira}\label{pacckx:kira}

For Mogg, it is \pali{kira} with \pali{k-anubandha}.

\pali{vaja + ira/kira} = \palibf{vajira} (thunderbolt)\par
\pali{tima + kira} = \palibf{timira} (darkness, water)\par
\pali{ruha + kira} = \palibf{ruhira} (blood)\par
\pali{rudha + kira} = \palibf{rudhira} (blood)\par
\pali{badha + kira} = \palibf{badhira} (deaf)\par
\pali{mada + kira} = \palibf{madir\=a} (liquor)\par
\pali{manda? + kira} = \palibf{mandira} (house)\par
\pali{aja + kira} = \palibf{ajira} (courtyard)\par
\pali{ruca + kira} = \palibf{rucira} (beautiful)\par
\pali{kasa + kira} = \palibf{kasira} (misery)\par
\pali{\d th\=a + kira} = \palibf{thira} (stable)\par
\pali{s\=isa? + kira} = \palibf{sisira} (winter)\par
\pali{kh\=ada + kira} = \palibf{khadira} (a kind of tree)\par

\subparagraph*{\pali{Dura}} (Mogg\,7.151)\label{pacckx:dura}

\pali{dada? + dura} = \palibf{daddura} (frog)\par

\subparagraph*{\pali{Bhara}} (Mogg\,7.151)\label{pacckx:bhara}

\pali{gara + bhara} = \palibf{gabbhara} (cave)\par

\subparagraph*{\pali{Cara}} (Mogg\,7.152)\label{pacckx:cara}

\pali{cara + cara} = \palibf{caccara} (crossroad, courtyard)\par

\subparagraph*{\pali{Dara}} (Mogg\,7.152)\label{pacckx:dara}

\pali{dara + dara} = \palibf{daddara} (an instrument, drum)\par

\subparagraph*{\pali{Jara}} (Mogg\,7.152)\label{pacckx:jara}

\pali{jara + jara} = \palibf{jajjara} (old age)\par

\subparagraph*{\pali{Gara}} (Mogg\,7.152)\label{pacckx:gara}

\pali{gara + gara} = \palibf{gaggara} (bellow)\par

\subparagraph*{\pali{Mara}} (Mogg\,7.152)\label{pacckx:mara}

\pali{mara + mara} = \palibf{mammara} (dried leaf, sound of leaves or cloth)\par

\subparagraph*{\pali{\=Ivara, kvara}} (Kacc\,668, R\=upa\,678, Sadd\,1314, Mogg\,7.153--4)\label{pacckx:iivara}\label{pacckx:kvara}

In Mogg, \pali{kvara} is given instead of \pali{\=ivara}.

\pali{ci + \=ivara/kvara} = \palibf{c\=ivara} (robe)\par
\pali{p\=a + \=ivara} = \palibf{p\=ivara} (full, fat, turtle)\par
\pali{dh\=a + \=ivara/kvara} = \palibf{dh\=ivara} (fisherman)\par
\pali{p\=i + kvara} = \palibf{p\=ivara} (fat)\par
\pali{sama + kvara} = \palibf{sa\d mvar\=i} (night)\par
\pali{t\=a + kvara} = \palibf{t\=ivara} (ignoble one)\par
\pali{n\=i + kvara} = \palibf{n\=ivara} (house)\par

\subparagraph*{\pali{Krara}} (Mogg\,7.155)\label{pacckx:krara}

\pali{ku + krara} = \palibf{kurara/kurar\=i} (osprey)\par

\subparagraph*{\pali{Chara}} (Mogg\,7.156)\label{pacckx:chara1}

\pali{vasa + chara} = \palibf{vacchara} (year)\par
\pali{sa\d m + vasa + chara} = \palibf{sa\d mvacchara} (year)\par
\pali{asa + chara} = \palibf{acchar\=a} (nymph, finger snap)\par

\subparagraph*{\pali{Chera, chara}} (Mogg\,7.157)\label{pacckx:chera}\label{pacckx:chara2}

\pali{masa + chera} = \palibf{macchera} (stinginess)\par
\pali{masa + chara} = \palibf{macchara} (stinginess)\par

\subparagraph*{\pali{Sara}} (Mogg\,7.158)\label{pacckx:sara}

\pali{dh\=u + sara} = \palibf{dh\=usara} (dust-colored, yellowish)\par
\pali{v\=a + sara} = \palibf{v\=asara} (day)\par

\subparagraph*{\pali{Ara}} (Mogg\,7.159--62)\label{pacckx:ara}

\pali{bhama? + ara} = \palibf{bhamara} (wasp, bee)\par
\pali{tasa + ara} = \palibf{tasara} (shuttle)\par
\pali{manda? + ara} = \palibf{mandara} (a mountain)\par
\pali{kanda + ara} = \palibf{kandara} (glen, cave)\par
\pali{diva + ara} = \palibf{devara} (brother-in-law)\par

\subparagraph*{\pali{Ara\d na}} (Mogg\,7.163)\label{pacckx:aradna}

\pali{vaka + ara\d na} = \palibf{v\=akar\=a} (snare, net)\par

\subparagraph*{\pali{\=Ara}} (Mogg\,7.164--6)\label{pacckx:aara}

\pali{si\.ngi? + \=ara} = \palibf{si\.ng\=ara} (erotic sentiment)\par
\pali{a\.nga? + \=ara} = \palibf{a\.ng\=ara} (charcoal, embers)\par
\pali{aga + \=ara} = \palibf{ag\=ara} (house)\par
\pali{majja + \=ara} = \palibf{majj\=ara} (cat)\par
\pali{kala + \=ara} = \palibf{ka\d l\=ara} (brown, tawny)\par
\pali{ala + \=ara} = \palibf{a\d l\=ara} (arc, curve)\par
\pali{kama + \=ara} = \palibf{kum\=ara} (child)\par
\pali{bhara + \=ara} = \palibf{bhi\.ng\=ara} (golden water-jug)\par
\pali{kleda? + \=ara} = \palibf{ked\=ara} (arable land, field)\par
\pali{ku + vida + \=ara} = \palibf{kovi\d l\=ara} (a kind of tree with double leaves)\par

\subparagraph*{\pali{M\=ara}} (Mogg\,7.167)\label{pacckx:maara}

\pali{kara + m\=ara} = \palibf{kamm\=ara} (blacksmith)\par

\subparagraph*{\pali{Khara}} (Mogg\,7.168)\label{pacckx:khara}

\pali{pusa + khara} = \palibf{pokkhara} (lotus)\par
\pali{sara + khara} = \palibf{sakkhar\=a} (sugar)\par

\subparagraph*{\pali{K\=ira}} (Mogg\,7.169--70)\label{pacckx:kiira}

\pali{sara + k\=ira} = \palibf{sar\=ira} (the body)\par
\pali{vasa + k\=ira} = \palibf{us\=ira} (a kind of plant)\par
\pali{kala + k\=ira} = \palibf{kal\=ira} (shoot, sprout)\par
\pali{gama + k\=ira} = \palibf{gambh\=ira/gabh\=ira} (deep)\par
\pali{kula + k\=ira} = \palibf{ku\d l\=ira} (crab)\par

\subparagraph*{\pali{Ora}} (Mogg\,7.173--4)\label{pacckx:ora}

\pali{ka\d tha + ora} = \palibf{ka\d thora} (rough)\par
\pali{caka + ora} = \palibf{cakora} (francolin partridge)\par
\pali{m\=i + ora} = \palibf{mora} (peacock)\par
\pali{kasa + ora} = \palibf{kisora} (young horse)\par
\pali{maha + ora} = \palibf{mahora} (anthill)\par

\subparagraph*{\pali{Eraka}} (Mogg\,7.175)\label{pacckx:eraka}

\pali{ku + eraka} = \palibf{kuvera} (a deity)\par

\subparagraph*{\pali{Rika}} (Mogg\,7.176)\label{pacckx:rika}

\pali{bh\=u + rika} = \palibf{bh\=uri} (plenty)\par
\pali{bh\=u + rika} = \palibf{bh\=ur\=i} (wisdom)\par
\pali{s\=u + rika} = \palibf{s\=uri} (wise one)\par

\subparagraph*{\pali{Ru}} (Mogg\,7.177)\label{pacckx:ru}

\pali{m\=i + ru} = \palibf{meru} (the Sineru)\par
\pali{ka + s\=i + ru} = \palibf{kaseru} (a kind of plant, water chestnut)\par
\pali{n\=i + ru} = \palibf{neru} (a mountain)\par

\subparagraph*{\pali{Eru}} (Mogg\,7.178)\label{pacckx:eru}

\pali{sin\=a? + eru} = \palibf{sineru} (the king of mountains)\par

\subparagraph*{\pali{Ruka}} (Mogg\,7.179)\label{pacckx:ruka}

\pali{bh\=i + ruka} = \palibf{bh\=iru} (frightening)\par
\pali{ru + ruka} = \palibf{ruru} (a kind of deer)\par

\subparagraph*{\pali{La}} (Kacc\,632, R\=upa\,656, Sadd\,1241; Kacc\,634, R\=upa\,658)\label{pacckx:la1}

\pali{ala + la} = \palibf{alla} (new, wet)\par
\pali{kala + la} = \palibf{kalla} (comfortable, proper)\par
\pali{sala + la} = \palibf{salla} (arrow)\par
\pali{matha + la} = \palibf{malla/mallaka} (wrestler)\par

\subparagraph*{\pali{Ala}} (Kacc\,665, R\=upa\,675, Sadd\,1308, Mogg\,7.182)\label{pacckx:ala}

\pali{pa\d ta + ala} = \palibf{pa\d tala} (covering, group)\par
\pali{ma\.nga? + ala} = \palibf{ma\.ngala} (auspicious)\par
\pali{kama + ala} = \palibf{kamala} (lotus)\par
\pali{samba + ala} = \palibf{sambala} (provision)\par
\pali{saba? + ala} = \palibf{sabala} (spotted)\par
\pali{saka + ala} = \palibf{sakala} (all)\par
\pali{vasa + ala} = \palibf{vasala} (ignoble one)\par
\pali{pisa + ala} = \palibf{pesala} (one having good conduct)\par
\pali{keva? + ala} = \palibf{kevala} (total)\par
\pali{kala + ala} = \palibf{kalala} (mud, mire)\par
\pali{palla? + ala} = \palibf{pallala} (marshy ground, small lake)\par
\pali{ka\d tha + ala} = \palibf{ka\d thala} (pebble)\par
\pali{ku\d n\d da? + ala} = \palibf{ku\d n\d dala} (earring)\par
\pali{ma\d n\d da? + ala} = \palibf{ma\d n\d dala} (circle)\par

Other examples do not have any analytic part, so I just list the words here: \pali{kusala} (wholesome), \pali{kadala} (banana tree), \pali{bhagandala} (ulcer), \pali{mekhala/mekhal\=a} (girdle), \pali{vakkala} (bark), \pali{takkala} (resin), \pali{saddala} (grass), \pali{mul\=ala} (lutus's root), \pali{pil\=ala} (salt), \pali{vid\=ala} (a kind of plant), \pali{ca\d n\d d\=ala} (outcaste), \pali{v\=a\d la} (snake), \pali{v\=ala} (water), \pali{macala} (thief), \pali{musala} (pestle), \pali{kotthula} (jackal), \pali{puthula} (thick, wide), \pali{bahula} (plenty), \pali{bahala} (many, thick), \pali{kambala} (wool), \pali{agga\d la/aggala} (bolt, latch).

\subparagraph*{\pali{Kala}} (Mogg\,7.183--5)\label{pacckx:kala}

\pali{musa + kala} = \palibf{musala} (pestle)\par
\pali{\d th\=a + kala} = \palibf{thala} (dry ground)\par
\pali{u + p\=a + kala} = \palibf{uppala} (waterlily)\par
\pali{pata + kala} = \palibf{p\=a\d tala} (fruit, pink)\par
\pali{ba\d mhi? + kala} = \palibf{bahala} (thick)\par
\pali{cupa + kala} = \palibf{capala} (unsteady, fickle)\par
\pali{kula + kala} = \palibf{kulala} (hawk, vulture)\par

\subparagraph*{\pali{K\=ala}} (Mogg\,7.185--6)\label{pacckx:kaala}

\pali{kula + k\=ala} = \palibf{kul\=ala} (pot maker)\par
\pali{m\=ila + k\=ala} = \palibf{mu\d l\=ala} (lotus's root)\par
\pali{bala + k\=ala} = \palibf{bi\d l\=ala} (cat)\par
\pali{kappa + k\=ala} = \palibf{kap\=ala} (potsherd)\par
\pali{p\=i + k\=ala} = \palibf{piy\=ala} (a kind of tree)\par
\pali{ku\d na + k\=ala} = \palibf{ku\d n\=ala} (big pond)\par
\pali{visa + k\=ala} = \palibf{vis\=ala} (large)\par
\pali{pala + k\=ala} = \palibf{pal\=ala} (straw)\par
\pali{sara + k\=ala} = \palibf{sig\=ala} (jackal)\par

\subparagraph*{\pali{\d N\=ala}} (Mogg\,7.187)\label{pacckx:dnaala}

\pali{ca\d n\d da? + \d n\=ala} = \palibf{ca\d n\d d\=ala} (outcaste)\par
\pali{pata + \d n\=ala} = \palibf{p\=at\=ala} (abyss)\par

\subparagraph*{\pali{La}} (Mogg\,7.188)\label{pacckx:la2}

\pali{m\=a + la} = \palibf{m\=al\=a} (garland)\par
\pali{i + la} = \palibf{el\=a} (saliva)\par
\pali{p\=i + la} = \palibf{pel\=a} (a kind of basket)\par
\pali{d\=u + la} = \palibf{dol\=a} (swing, palanquin)\par
\pali{kala + la} = \palibf{kalla} (suitable)\par

\subparagraph*{\pali{Chilla}} (Sadd\,1252)\label{pacckx:chilla}

\pali{pisa + chilla} = \palibf{picchilla} (grinding)\par

\subparagraph*{\pali{B\=ula}} (Mogg\,7.180))\label{pacckx:buula}

\pali{tama + b\=ula} = \palibf{tamb\=ala} (betel-leaf)\par

\subparagraph*{\pali{Laka, v\=ala}} (Mogg\,7.181)\label{pacckx:laka}\label{pacckx:vaala}

\pali{si + laka} = \palibf{sil\=a} (stone)\par
\pali{si + laka} = \palibf{sel\=a} (mountain)\par
\pali{si + v\=ala} = \palibf{sev\=ala} (moss, slime)\par

\subparagraph*{\pali{Ila}} (Mogg\,7.189)\label{pacckx:ila}

\pali{ana + ila} = \palibf{anila} (wind)\par
\pali{sala + ila} = \palibf{salila} (water)\par
\pali{kala + ila} = \palibf{kalila} (dense)\par
\pali{kuka + ila} = \palibf{kokila} (cuckoo)\par
\pali{sa\d tha + ila} = \palibf{sa\d thila} (cheat)\par
\pali{maha + ila} = \palibf{mahil\=a} (woman)\par

\subparagraph*{\pali{Kila}} (Mogg\,7.190--1)\label{pacckx:kila}

\pali{ku\d ta + kila} = \palibf{ku\d tila} (crooked, curve)\par
\pali{saha + kila} = \palibf{sithila} (unsteady)\par
\pali{kampa? + kila} = \palibf{kapila} (sage Kapila)\par
\pali{matha + kila} = \palibf{mithil\=a} (Mithil\=a city)\par

\subparagraph*{\pali{Kula}} (Mogg\,7.192--3)\label{pacckx:kula}

\pali{ca\d ta + kula} = \palibf{ca\d tula} (flatterer)\par
\pali{ka\d n\d da? + kula} = \palibf{ka\d n\d dula} (tree)\par
\pali{va\d t\d ta + kula} = \palibf{va\d t\d tula} (round, circle)\par
\pali{putha + kula} = \palibf{puthula} (broad, large)\par
\pali{tama + kula} = \palibf{tumula} (great)\par
\pali{tama + kula} = \palibf{ta\d n\d dula} (rice-grain)\par
\pali{ni + ci + kula} = \palibf{nicula} (a kind of plant)\par

\subparagraph*{\pali{Ola}} (Mogg\,7.194)\label{pacckx:ola}

\pali{kalla + ola} = \palibf{kallola} (billow, big wave, tsunami)\par
\pali{kapa + ola} = \palibf{kapola} (the cheek)\par
\pali{takka + ola} = \palibf{takkola} (a kind of pepper)\par
\pali{pa\d ta + ola} = \palibf{pa\d tola} (snake-gourd)\par

\subparagraph*{\pali{Ula, uli}} (Mogg\,7.195)\label{pacckx:ula}\label{pacckx:uli}

\pali{a\.nga? + ula} = \palibf{a\.ngula} (a measure)\par
\pali{a\.nga? + uli} = \palibf{a\.nguli} (finger)\par

\subparagraph*{\pali{Ali}} (Mogg\,7.196)\label{pacckx:ali}

\pali{a\~nja + ali} = \palibf{a\~njali} (putting hands into lotus shape)\par

\subparagraph*{\pali{Li}} (Mogg\,7.197--8)\label{pacckx:li}

\pali{chada + li} = \palibf{challi} (bark, skin)\par
\pali{ara + li} = \palibf{alli} (a kind of tree)\par
\pali{n\=i + li} = \palibf{n\=ili} (a kind of tree)\par
\pali{p\=ala + li} = \palibf{p\=ali} (row, line)\par
\pali{p\=ala + li} = \palibf{palli} (hut)\par
\pali{cuda + li} = \palibf{culli} (stove)\par

\subparagraph*{\pali{Ava}} (Mogg\,7.199--200)\label{pacckx:ava}

\pali{pila + ava} = \palibf{pelava} (light, soft)\par
\pali{palla? + ava} = \palibf{pallava} (young leaf)\par
\pali{pa\d na + ava} = \palibf{pa\d nava} (small drum)\par
\pali{sala + ava} = \palibf{s\=a\d lava} (salad)\par
\pali{kita + ava} = \palibf{kitava} (gambler, thief)\par
\pali{mu? + ava} = \palibf{mutava} (outcaste)\par
\pali{vala + ava} = \palibf{va\d lav\=a} (female horse)\par
\pali{mula + ava} = \palibf{murava} (drum)\par

\subparagraph*{\pali{\=Ava}} (Mogg\,7.201)\label{pacckx:aava}

\pali{sara + \=ava} = \palibf{sar\=ava} (cup, saucer)\par

\subparagraph*{\pali{\d Nuva}} (Mogg\,7.202)\label{pacckx:dnuva}

\pali{ala + \d nuva} = \palibf{\=aluva} (shrub)\par
\pali{mala + \d nuva} = \palibf{m\=aluva} (a kind of plant)\par
\pali{pila + \d nuva} = \palibf{peluva} (a kind of plant)\par

\subparagraph*{\pali{\=Iva}} (Mogg\,7.203)\label{pacckx:iiva}

\pali{g\=a + \=iva} = \palibf{g\=iv\=a} (the neck)\par

\subparagraph*{\pali{Kva, kv\=a}} (Mogg\,7.204--5)\label{pacckx:kva}\label{pacckx:kvaa}

\pali{su + kva} = \palibf{suva} (parrot)\par
\pali{su + kv\=a} = \palibf{suv\=a} (dog)\par
\pali{vida + kv\=a} = \palibf{vidv\=a} (wise person)\par

\subparagraph*{\pali{Riva}} (Mogg\,7.207)\label{pacckx:riva}

\pali{sama + riva} = \palibf{siva} (god Shiva)\par

\subparagraph*{\pali{Reva}} (Mogg\,7.206)\label{pacckx:reva}

\pali{thu + reva} = \palibf{theva} (water drop)\par

\subparagraph*{\pali{Ravi}} (Mogg\,7.208)\label{pacckx:ravi}

\pali{chada + ravi} = \palibf{chavi} (skin)\par

\subparagraph*{\pali{Ussa, nusa, isa}} (Kacc\,673, R\=upa\,683, Sadd\,1319)\label{pacckx:ussa}\label{pacckx:nusa}\label{pacckx:isa}

\pali{manu + ussa} = \palibf{manussa} (human being)\par
\pali{manu + nusa} = \palibf{m\=anusa} (human being)\par
\pali{p\=ura + isa} = \palibf{purisa}\footnote{In Mogg\,7.209 this comes from \pali{p\=ura + kisa}.} (man)\par
\pali{p\=ura + isa} = \palibf{posa} (man)\par
\pali{su\d na? + isa} = \palibf{su\d nis\=a}\footnote{In Mogg\,7.216 this comes from \pali{su + \d nisaka}.} (daughter-in-law)\par
\pali{ku + isa} = \palibf{kar\=isa}\footnote{In Mogg\,7.210 this comes from \pali{kara + \=isa}.} (excrement)\par
\pali{su + isa} = \palibf{s\=uriya} (the sun)\par
\pali{hi\d msa + isa} = \palibf{sir\=isa}\footnote{In Mogg\,7.211 this comes from \pali{sara + \=isa}.} (a kind of tree)\par
\pali{ila + isa} = \palibf{illisa} (depressed one)\par
\pali{ala + isa} = \palibf{alasa}\footnote{In Mogg\,7.217 this comes from \pali{ala + asa}.} (lazy person)\par
\pali{maha + isa} = \palibf{mahisa} (buffalo)\par
\pali{s\=i + isa} = \palibf{s\=isa}\footnote{In Mogg\,7.214 this comes from \pali{s\=i + saka}.} (the head)\par
\pali{ki + isa} = \palibf{kisa} (thin, skinny)\par

\subparagraph*{\pali{Kisa}} (Mogg\,7.209)\label{pacckx:kisa}

\pali{p\=ura + kisa} = \palibf{purisa} (man)\par
\pali{tima + kisa} = \palibf{timisa} (dark)\par

\subparagraph*{\pali{\=Isa}} (Mogg\,7.210--1)\label{pacckx:iisa}

\pali{kara + \=isa} = \palibf{kar\=isa} (excrement)\par
\pali{sara + \=isa} = \palibf{sir\=isa} (a kind of plant)\par
\pali{p\=ura + \=isa} = \palibf{pur\=isa} (excrement)\par
\pali{tala + \=isa} = \palibf{t\=al\=isa} (a kind of herb)\par

\subparagraph*{\pali{Saka}} (Mogg\,7.214--5)\label{pacckx:saka2}

\pali{\=ami + saka} = \palibf{\=amisa} (food)\par
\pali{thu + saka} = \palibf{thusa} (chaff)\par
\pali{ku + saka} = \palibf{kusa} (a kind of grass)\par
\pali{s\=i + saka} = \palibf{s\=isa} (the head, lead)\par
\pali{phusa + saka} = \palibf{phassa}\footnote{In Kacc\,528, R\=upa\,577, Sadd\,1110, this comes from \pali{phusa + \d na.}} (contact)\par
\pali{phusa + saka} = \palibf{phussa} (a constellation)\par
\pali{pusa + saka} = \palibf{pussa} (a kind of fruit)\par
\pali{bh\=u + saka} = \palibf{bhusa} (chaff)\par
\pali{a\.nka? + saka} = \palibf{a\.nkusa} (hook for controlling an elephant)\par
\pali{pa + ph\=aya? + saka} = \palibf{papph\=asa} (lung)\par
\pali{kala + saka} = \palibf{kamm\=asa} (blemished, spotted)\par
\pali{kula + saka} = \palibf{kumm\=asa} (junket, a kind of sweet)\par
\pali{kula + saka} = \palibf{kulisa} (thunderbolt)\par
\pali{mana + saka} = \palibf{ma\~nj\=us\=a} (casket, box)\par
\pali{p\=i + saka} = \palibf{p\=iy\=usa} (elixir)\par
\pali{bala + saka} = \palibf{ba\d lisa} (fishhook)\par
\pali{maha + saka} = \palibf{mahes\=i} (queen)\par

\subparagraph*{\pali{\d Nisaka}} (Mogg\,7.216)\label{pacckx:dnisaka}

\pali{su + \d nisaka} = \palibf{su\d nis\=a} (daughter-in-law)\par

\subparagraph*{\pali{Asa}} (Mogg\,7.217)\label{pacckx:asa}

\pali{veta? + asa} = \palibf{vetasa} (a kind of tree)\par
\pali{ata + asa} = \palibf{atasa} (a kind of tree)\par
\pali{yu + asa} = \palibf{yavasa} (grass for cattle)\par
\pali{pana + asa} = \palibf{panasa} (jackfruit)\par
\pali{ala + asa} = \palibf{alasa} (lazy person)\par
\pali{kala + asa} = \palibf{kalasa} (water pot)\par
\pali{cama + asa} = \palibf{camasa} (ladle for offering)\par

\subparagraph*{\pali{Ribbisa}} (Mogg\,7.212)\label{pacckx:ribbisa}

\pali{kara + ribbisa} = \palibf{kibbisa} (wrong action)\par

\subparagraph*{\pali{Sa}} (Mogg\,7.213)\label{pacckx:sa}

\pali{sasa + sa} = \palibf{sassa} (crop)\par
\pali{asa + sa} = \palibf{assa} (horse)\par
\pali{vasa + sa} = \palibf{vassa} (year)\par
\pali{visa + sa} = \palibf{vessa} (the merchant caste)\par
\pali{hana + sa} = \palibf{ha\d msa} (swan)\par
\pali{vana + sa} = \palibf{va\d msa} (clan, bamboo)\par
\pali{mana + sa} = \palibf{ma\d msa} (flesh)\par
\pali{ana + sa} = \palibf{a\d msa} (part, shoulder)\par
\pali{kama + sa} = \palibf{ka\d msa} (a measure, bronze)\par

\subparagraph*{\pali{asa\d na, asaka, p\=asa, kasa}} (Mogg\,7.218)\label{pacckx:asadna}\label{pacckx:paasa}\label{pacckx:kasa}

\pali{vaya + asa\d na} = \palibf{v\=ayasa} (crow)\par
\pali{diva + asaka} = \palibf{divasa} (day)\par
\pali{kara + p\=aka} = \palibf{kapp\=asa} (cotton)\par
\pali{kara + kasa} = \palibf{kakkasa} (rough, harsh)\par

\subparagraph*{\pali{Su}} (Mogg\,7.219)\label{pacckx:su}

\pali{sasa + su} = \palibf{sassu} (mother-in-law)\par
\pali{masa + su} = \palibf{massu} (beard)\par
\pali{da\d msa + su} = \palibf{dassu} (thief)\par
\pali{asa + su} = \palibf{assu} (tear)\par

\subparagraph*{\pali{Dusuka}} (Mogg\,7.220)\label{pacckx:dusuka}

\pali{vida + dusuka} = \palibf{viddasu} (wise person)\par

\subparagraph*{\pali{R\=iha}} (Mogg\,7.221)\label{pacckx:riiha}

\pali{sasa + r\=iha} = \palibf{s\=iha} (lion)\par

\subparagraph*{\pali{Ha}} (Mogg\,7.222--3)\label{pacckx:ha}

\pali{j\=iva + ha} = \palibf{jivh\=a} (tongue)\par
\pali{ama + ha} = \palibf{amha} (stone)\par
\pali{pa + ama + ha} = \palibf{pamha} (eyelash)\par
\pali{tasa + ha} = \palibf{ta\d nh\=a} (craving)\par
\pali{kasa + ha} = \palibf{ka\d nha} (black)\par
\pali{juta + ha} = \palibf{ju\d nh\=a} (moonlight)\par
\pali{m\=ila + ha} = \palibf{m\=i\d lha} (excrement)\par
\pali{g\=aha + ha} = \palibf{g\=a\d lha} (strong)\par
\pali{daha + ha} = \palibf{da\d lha} (stable)\par
\pali{baha + ha} = \palibf{b\=a\d lha} (stable)\par
\pali{gama + ha} = \palibf{gimha} (hot)\par
\pali{pa\d ta + ha} = \palibf{pa\d taha} (war drum, kettledrum)\par
\pali{kala + ha} = \palibf{kalaha} (dispute)\par
\pali{ka\d ta + ha} = \palibf{ka\d t\=aha} (receptacle, cauldron)\par
\pali{vara + ha} = \palibf{var\=aha} (pig)\par
\pali{l\=u + ha} = \palibf{loha} (metal)\par

\subparagraph*{\pali{Hi, h\=i}} (Mogg\,7.224)\label{pacckx:hi}\label{pacckx:hii}

\pali{pa\d na + hi} = \palibf{pa\d nhi} (the heel)\par
\pali{u + saha + h\=i} = \palibf{usso\d lh\=i} (effort)\par

\subparagraph*{\pali{\d La}} (Mogg\,7.225--6)\label{pacckx:dla}

\pali{kh\=i + \d la} = \palibf{khe\d la} (saliva)\par
\pali{mi + \d la} = \palibf{me\d l\=a} (soot)\par
\pali{p\=i + \d la} = \palibf{pe\d l\=a} (a kind of basket)\par
\pali{cu + \d la} = \palibf{c\=u\d l\=a} (crest)\par
\pali{m\=a + \d la} = \palibf{m\=a\d la} (a kind of pavilion)\par
\pali{v\=a + \d la} = \palibf{v\=a\d la} (beast)\par
\pali{k\=a + \d la} = \palibf{k\=a\d la} (black)\par
\pali{gu + \d la} = \palibf{go\d la} (dwarf)\par

\subparagraph*{\pali{\d Laka}} (Mogg\,7.226--7)\label{pacckx:dlaka}

\pali{gu + \d laka} = \palibf{gu\d la} (sugar)\par
\pali{kha\~nja + \d laka} = \palibf{pa\.ngu\d la} (cripple)\par
\pali{kara + \d laka} = \palibf{kakkha\d la} (rough, harsh)\par
\pali{kuka + \d laka} = \palibf{kukku\d la} (a hell)\par
\pali{ma\d mka? + \d laka} = \palibf{maku\d la} (bud)\par

\subparagraph*{\pali{\d Li}} (Mogg\,7.228)\label{pacckx:dli}

\pali{p\=a + \d li} = \palibf{p\=a\d li} (line, P\=ali)\par

\subparagraph*{\pali{\d Lu}} (Mogg\,7.229)\label{pacckx:dlu}

\pali{v\=i + \d lu} = \palibf{ve\d lu} (bamboo)\par

\section*{Other minor matters}

There are some trivial things mentioned in the textbooks that I do not want to skip them for they might have some merit. Some of these are not directly relevant to derivation. Some are idiosyncratic ways of analyzing words. It is good to have them in one place.

\paragraph*{\pali{I} and \pali{\=i} insertion} When compounds are formed with derivatives of \pali{bh\=u} and \pali{kara}, \pali{i} and \pali{\=i} can be inserted (Sadd\,1338, Mogg\,4.119), for example, \pali{s\=itibh\=uta} (having become cool/calm), \pali{byantikata}\footnote{In a dictionary you may find \pali{vyant\=ikata} instead.} (having abolished [something]), \pali{y\=anikata} (having made a habit of), \pali{bahulikata} (having done a lot), \pali{cittikata} (having done/put in mind), \pali{sammukh\=ibh\=uta} (having become face to face with), \pali{kaddam\=ibh\=uta} (having become mud), \pali{ekodak\=ibh\=uta} (having become united), \pali{sara\d n\=ibh\=uta} (having become refuge), \pali{bhasm\=ikata} (having made ashes). However, some are not so, for example, \pali{manussabh\=uta} (having become a human being), \pali{kammak\=ara} (worker).

\paragraph*{\pali{Uddha + mukha = udukkhala}} From Sadd\,1339, this is analyzed as ``\pali{uddha\d m mukhamass\=ati udukkhala\d m}'' (thing having a mouth on the top, thus a mortar).

\paragraph*{\pali{V\=ariv\=ahaka $\rightarrow$ val\=ahaka}} From Sadd\,1340, the former can be changed into the latter. Both can be used to mean `rain cloud'--- thing carrying water (\pali{v\=ar\=i vahat\=iti v\=ariv\=ahako}). If the ending is not \pali{v\=ahaka}, the change to \pali{l} will not occur, for example, \pali{v\=arivaho p\=uro} (full river).

\paragraph*{\pali{Chavasayana $\rightarrow$ sus\=ana}} From Sadd\,1341, this is analyzed as ``\pali{chav\=ana\d m sayana\d m chavasayana\d m}'' (lying place of corpses, thus cemetery).

\paragraph*{\pali{Br\=u + saha = bhis\=i}} From Sadd\,1342, the analytic sentence is ``\pali{br\=uvanto etissa\d m s\=idant\=iti bhis\=i}'' (on that place [people] sit talking, thus a cushion).

\paragraph*{\pali{Bhava + gamana + vanta = bhagav\=a}} From Sadd\,1343, this is analyzed as ``\pali{bhavesu gamana\d m vantoti bhagav\=a}'' (renouncing the going in state of existence, thus the blessed one [the Buddha]).

\paragraph*{\pali{\d Na-paccaya} produces masculine verbal nouns} From Sadd\,1346, \pali{pavisna\d m \textbf{paveso}} (entering) and \pali{phusana\d m \textbf{phasso}} (contact) are exemplified.

\paragraph*{\pali{Ta-paccaya} produces neuter verbal nouns} From Sadd\,1347, \pali{gamana\d m \textbf{gata\d m}} (going), \pali{supana\d m \textbf{sutta\d m}} (sleeping), \pali{\=as\=isana\d m \textbf{\=asi\d t\d tha\d m}} (hoping), and \pali{bujjhana\d m \textbf{buddha\d m}} (knowing) are exemplified.
