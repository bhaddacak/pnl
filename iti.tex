\chapter{I say ``\headhl{P\=ali is not so difficult}''}\label{chap:iti}

\phantomsection
\addcontentsline{toc}{section}{Direct and Indirect Speech}
\section*{Direct and Indirect Speech}

This is the last chapter that we have a task to fulfill. This marks the end of primer function of the book. After this chapter, there will be description and explanation of principles. In this chapter we will focus only on one particle---\pali{iti}. We already have touched upon this particle in Chapter \ref{chap:ind-intro}, and I have used it several times in our former exercises. As the most used particle of all, \pali{iti} is the only thing that can create direct speech in P\=ali. It is really important because without knowing this we have no clue whatsoever to make sense out of word strings in the scriptures. Religious text makes use of direct and indirect speech thoroughly, because there are many stories to be told and retold. So, mastering \pali{iti} is essential.

Let us be familiar with \pali{iti} first. This term is a particle which is used in a variety of ways. It is quite rare to see this in full form. Most of the time, \pali{iti} is welded (\pali{sandhi}) with the preceding word making it appear only as \pali{-ti}.\footnote{These can be \pali{-\=ati, -\=iti, -\=uti, eti, oti,} and \pali{-nti}. The first five tell us that the ending of the preceding word is \pali{a} or \pali{\=a}, \pali{i} or \pali{\=i}, \pali{u} or \pali{\=u}, \pali{e}, and \pali{o} respectively. The last one tells us that the preceding word ends with \pali{\d m}.} So, you have to recognize it first. With untrained eyes, one can mistake it easily, because verbs also end with \pali{ti}. Fortunately for modern learners, in newly compiled texts a quotation mark is inserted to mark out \pali{iti}.\footnote{Not every instance is done so. You still have to make a decision by your own sometimes.} So, it is relatively easy nowadays to single out \pali{iti} sentences.

What \pali{iti} does in direct speech is to mark out the speech reported in sentences. It is equivalent to quotation marks in English, so it has no meaning by itself. There are viable verbs that \pali{iti} can be used with. Most of them have things to do with utterance, e.g.\ to say, to ask, to reply, to complain, to reproach, and so on. Some are mental activity, e.g.\ to think, to wish, to plan, to remember, and so on. Sometimes no specific verb is mentioned; the speech is marked by the context. Here are some simple examples:

\begin{quote}
\pali{`ya\d mn\=un\=aha\d m buddha\d m paccakkheyyan'ti vadati \\vi\~n\~n\=apeti.}\footnote{Buv1\,45}\\
{[A monk] says, makes [another] know, ``I should give up [following] the Buddha.''}\\[1.5mm]
\pali{sapatto sapattassa eva\d m icchati -- `aho vat\=aya\d m \\dubba\d n\d no ass\=a'ti}\footnote{A7\,64}\\
{A foe wishes this to [his] foe, ``May this [person] be ugly.''}\\[1.5mm]
\pali{`P\=apa\d m me katan'ti tappati}\footnote{Dhp\,1.17}\\
{[He] suffers [thinking] ``Evil has been done by me.''}\\[1.5mm]
\pali{Buddho buddhoti cintento, magga\d m sodhemaha\d m tad\=a}\footnote{Bv\,2:44}\\
{In that time, thinking `Buddho, Buddho,' I am sweeping the path.}\\[1.5mm]
\pali{Kodhanoya\d m, bhikkhave, purisapuggalo kodh\=abhibh\=uto kodhapareto, anatthampi gahetv\=a `attho me gahito'ti ma\~n\~nati, atthampi gahetv\=a `anattho me gahito'ti \\ma\~n\~nati.}\footnote{A7\,64}\\
{An angry person, monks, overpowered by anger, afflicted by anger, [when] having had a disadvantage, thinks `The advantage was taken by me'; [when] having had an advantage, thinks `The disadvantage is taken by me.'}\\[1.5mm]
\pali{Atha kho bhagav\=a tassa addham\=asassa accayena \\pa\d tisall\=an\=a vu\d t\d thito \=ayasmanta\d m \=ananda\d m \=amantesi -- `ki\d m nu kho, \=ananda, tanubh\=uto viya bhikkhusa\.ngho'ti?}\footnote{Buv1\,164}\\
{When that fortnight has passed, the Buddha, having emerged from seclusion, called the Venerable \=Ananda [and asked], ``Why, \=Ananda, does the community have less monks?''}\\[1.5mm]
\end{quote}

To be familiar with a narrative form in the Suttanta, let us see this excerpt:

\begin{quote}
\pali{Eva\d m me suta\d m -- eka\d m samaya\d m bhagav\=a s\=avatthiya\d m viharati jetavane an\=athapi\d n\d dikassa \=ar\=ame. Tatra kho bhagav\=a bhikkh\=u \=amantesi -- `bhikkhavo'ti. `Bhadante'ti te bhikkh\=u bhagavato paccassosu\d m. Bhagav\=a etadavoca --``Dhammad\=ay\=ad\=a me, bhikkhave, bhavatha, m\=a \=amisad\=ay\=ad\=a. \ldots'ti. Idamavoca bhagav\=a. Ida\d m vatv\=ana sugato u\d t\d th\=ay\=asan\=a vih\=ara\d m p\=avisi.}\footnote{M1\,29--30 (MN\,3)}\\[1.5mm]
{It is heard by me thus -- In one occasion, the Buddha is living in Park Jeta Temple of An\=athapi\d n\d dika, S\=avatth\=i. In that time the Buddha called monks ``Bhik\-khus.'' Monks responded to the Buddha ``Sir.'' [Then] the Buddha said, ``Be my heirs of teaching, monks; not material things.'' \ldots\ The Buddha said in this way. Having said thus, the Buddha, having risen from the seat, [then] entered into [his] place.}
\end{quote}

This is a typical form of a discourse in the canon. Without using \pali{iti}, \pali{eva\d m me suta\d m} marks the beginning of the narration. Dialogues and speeches are marked by \pali{iti}. Even so they are embedded in the narration seamlessly.

In grammatical textbooks, \pali{iti} is often used in definition or analytical parts (of compounds, for example). In an analytic sentence of \pali{mah\=apuriso}, you can see this: ``\pali{mahanto ca so puriso c\=ati mah\=apuriso}'' ([The person is] great and [the person is] a man, hence a great man). For more detail, see Appendix \ref{chap:samasa}.

If you have no problem with all examples mentioned above, now we can do our heading task, ``I say `P\=ali is not so difficult'.'' Here we go:

\palisample{aha\d m vad\=ami `P\=alibh\=as\=a t\=adis\=a kicch\=a na hot\=i'ti.}

Let us see another example, ``I say `Give me that book'.'' This sentence uses imperative mood in the speech, hence we get this:

\palisample{aha\d m vad\=ami `ta\d m me potthaka\d m deh\=i'ti.}

The interlocutor in this case is singular second person (`you'). If it is plural, the verb becomes \pali{detha}. If we change the sentence to indirect speech, thus ``I say to him he must give me that book,'' we can convert it to P\=ali straightly as ``\pali{aha\d m tassa vad\=ami so me ta\d m potthaka\d m detu}.'' This is ill-formed because, in English grammar's terms, there are two verbs in one sentence. It is better to use participles instead. In this case, a verb in \pali{tabba} form is suitable, but we have to say it in passive voice. Therefore, the sentence should be rewritten as ``I say to him the book must be given to me by you.'' Then we get this P\=ali:

\palisample{aha\d m tassa vad\=ami tay\=a me ta\d m potthaka\d m d\=atabba\d m.}

It is still better to have \pali{iti} in the sentence, hence:

\palisample{aha\d m tassa vad\=ami `tay\=a me ta\d m potthaka\d m d\=atabban'ti.}

When \pali{iti} is used, active structure turns to be valid as well. So, it is equivalent to say this:

\palisample{aha\d m tassa vad\=ami `ta\d m me potthaka\d m deh\=i'ti.\sampleor aha\d m vad\=ami `so me ta\d m potthaka\d m det\=u'ti.}

This makes the English equivalent rebounds to direct speech, ``I say to him `the book must be given to me by you'.'' As you may see along my experiment that direct speech is very natural to say in P\=ali. That is why this form of speech is used overwhelmingly in the texts. Whereas indirect speech is exceedingly rare.\footnote{\citealp[p.~36]{warder:intro}} Here are some examples of indirect speech suggested by Vito Perniola.\footnote{\citealp[p.~395]{perniola:grammar}}

\begin{quote}
\pali{Addasa\d msu kho gop\=alak\=a pasup\=alak\=a kassak\=a path\=avino bhagavanta\d m d\=uratova \=agacchanta\d m.}\footnote{Buv2\,326. This sentence is truncated in Perniola's book.}\\
Cowherds, cattlemen, farmers, and travellers saw the Buddha coming from a faraway [place].\\[1.5mm]
\pali{Sa\.ng\=a sa\.ng\=amaji\d m mutta\d m,\\tamaha\d m br\=umi br\=ahma\d na\d m}\footnote{Ud\,1.8}\\
I call a winner of the war, who is free from attachment, [as a] brahman.\\[1.5mm]\footnote{Antonio Costanzo gave me an Ireland's translation of this as follows: ``Sa\.ng\=amaji, freed from ties, Him I call a brahmin.''}
\pali{Tassime pa\~nca n\=ivara\d ne pah\=ine attani \\samanupassato p\=amojja\d m j\=ayati.}\footnote{D1\,466 (DN\,10)}\\
When that [monk] sees the five hindrances having been destroyed by himself, joy arises.
\end{quote}

In the older strata of texts, indirect speech appears in compound form, for example (Please study these carefully):\footnote{\citealp[pp.~395--6]{perniola:grammar}}

\begin{quote}
\pali{Disv\=a vijitasa\.ng\=ama\d m}\footnote{It\,82}\\
Having seen [a disciple] who won the war, [gods] \ldots\\[1.5mm]
\pali{Ta\d m ve kaly\=a\d napa\~n\~noti, \=ahu bhikkhu\d m an\=asava\d m}\footnote{It\,97. This sentence is truncated in Perniola's book.}\\
{[Buddhas] call a monk who is free from defilement `one who has beautiful wisdom.'}\\[1.5mm]
\pali{\=ahu sabbapah\=ayina\d m}\footnote{It\,97. Here is an explanation from Antonio Costanzo about this sentence and the previous one: In both sentences, there is, indeed, indirect speech. Perniola points out that two words, two bahubb\=ihi compounds, contain one whole spoken sentence each. So, these words have to be seen like an indirect speech. As an example, the word `\pali{an\=avasa\d m}' in the first sentence means: ``They say that (such a monk) is without defiling tendencies.'' The compound expresses indirectly the meaning of the direct speech sentence between quotes.}\\
{[Buddhas] call a monk who has all [defilements] destroyed [`one who has beautiful wisdom.']}\\[1.5mm]
\end{quote}

Another frequent use of \pali{iti} is much like we use quotation marks for defining things or quoting passages. For example, ``This [thing] is called `book'\,'' can be rendered as ``\pali{ida\d m [vatthu\d m] potthakan'ti vuccati}.'' Here are some examples from the canon:

\begin{quote}
\pali{Ida\d m dukkhanti kho, po\d t\d thap\=ada, may\=a by\=akata\d m}\footnote{D1\,420 (DN\,9)}\\
This has been declared by me as suffering, Po\d t\d thap\=ada.\\[1.5mm]
\pali{ahet\=u appaccay\=a purisassa sa\~n\~n\=a uppajjantipi nirujjhantip\=i'ti, \=aditova tesa\d m aparaddha\d m.}\footnote{D1\,412 (DN\,9)}\\
{[The view] of those as ``a man's perceptions arise and cease without a reason, without a cause,'' is wrong from the beginning.}\\[1.5mm]
\end{quote}

Let us move on by seeing a more complex example:

\begin{quote}
\pali{Atha kho corassa a\.ngulim\=alassa etadahosi ``ime kho sama\d n\=a sakyaputtiy\=a saccav\=adino saccapa\d ti\~n\~n\=a. Atha pan\=aya\d m sama\d no gaccha\d m yev\=aha `\d thito aha\d m, \\a\.ngulim\=ala, tva\~nca ti\d t\d th\=a'ti. Ya\d mn\=un\=aha\d m ima\d m sama\-\d na\d m puccheyyan''ti.}\footnote{M2\,348 (MN\,86)}\\
{Then [a thought] happened to robber A\.ngulim\=ala, ``These ascetics of S\=akya [normally] say truth and keep a promise. Yet this ascetic while going but said `I stood, A\.ngulim\=ala, you must stand [too].' I should ask this ascetic.''}
\end{quote}

In the above example, there are two layers of \pali{iti}. The outer is in thought, the inner in speech. You can find such complexity quite often, even in the very first paragraph of the canon. Do not panic. You just try to single out \pali{iti} clauses and identify the accompanying verbs. It is not so difficult unless you mistake a verb as an \pali{iti} marker. If you take texts from a modern collection, there should not be such a problem.

Before we end this section, we should know that \pali{iti} can do more than what we have seen. This is rather theoretical. So, it is good to know, but do not worry too much about how to put the following account into practice. Aggava\d msa summarizes functions of \pali{iti} as follows:\footnote{Sadd-Dh\=a\,1, from \pali{Id\=ani yath\=araha\d m nip\=at\=akhy\=atan\=amikapariy\=apann\=ana\d m itiito} onwards. See also \citealp[p.~142]{collins:grammar}.}

\paragraph*{Denoting cause or reason}\label{par:iti} For example:

\begin{quote}
\pali{Ruppat\=iti kho, bhikkhave, tasm\=a `r\=upan'ti vuccati.}\footnote{S3\,79 (SN\,22). In this instance, there are two \pali{iti}s. The first one is in \pali{ruppat\=iti} (\pali{ruppati + iti}), the second in \pali{r\=upanti} (\pali{r\=upa\d m + iti}).}\\
Because [it is] changed, monks, so it is called `body.'\\
\end{quote}

\paragraph*{Marking the end of expression} For example:

\begin{quote}
\pali{Atthi me tumhesu anukamp\=a -- `kinti me s\=avak\=a dhammad\=ay\=ad\=a bhaveyyu\d m, no \=amisad\=ay\=ad\=a'ti.}\footnote{M1\,29 (MN\,3)}\\
I have compassion for you [by thinking that] `How might my disciples become heirs of the teaching, not material things?'\\
\end{quote}

\paragraph*{Exemplifying or `such as'} For example:

\begin{quote}
\pali{iti v\=a iti evar\=up\=a vis\=ukadassan\=a pa\d tivirato}\footnote{D1\,13 (DN\,1)}\\
{[One] abstained from suchlike visiting shows and so on.}\\
\end{quote}

\paragraph*{Marking a near-synonym} For example:

\begin{quote}
\pali{M\=aga\d n\d diyoti tassa br\=ahma\d nassa n\=ama\d m sa\.nkh\=a \\sama\~n\~n\=a pa\~n\~natti voh\=aro}\footnote{Nidd1\,73}\\
Of that brahman, `M\=aga\d n\d diya' is a name, definition, designation, concept, expression.\\
\end{quote}

\paragraph*{As `in this manner'} For example:

\begin{quote}
\pali{Iti kho, bhikkhave, sappa\d tibhayo b\=alo, appa\d tibhayo pa\d n\d dito; saupaddavo b\=alo, anupaddavo pa\d n\d dito; \\saupasaggo b\=alo, anupasaggo pa\d n\d dito.}\footnote{M3\,124 (MN\,115)}\\
In this manner, monks, a fool [has] fear, a wise man [has] no fear; a fool [undergoes] misfortune, a wise man [undergoes] no misfortune; a fool [encounters] danger, a wise man [encounters] no danger.\\
\end{quote}

\paragraph*{As `only'} Technically, this is called \pali{avadh\=ara\d na}. It is like a simile, but it stresses more on `only.' See page \pageref{par:samasa-avadh} for some explanation. Here is a given example:

\begin{quote}
\pali{Atthi idappaccay\=a jar\=amara\d nan'ti iti pu\d t\d thena sat\=a, \=ananda, atth\=atissa vacan\=iya\d m. \\`Ki\d mpaccay\=a jar\=amara\d nan'ti iti ce vadeyya, \\`j\=atipaccay\=a jar\=amara\d nan'ti iccassa vacan\=iya\d m.}\footnote{D2\,96 (DN\,15).}\\
\=Ananda, were a wise person questioned in this way, `Does aging-and-death exist because of a cause?' One may say to him `It does.' If [he] asks [further] thus `From what cause, does aging-and-death exist?' One may reply thus `\textbf{Only} from birth as cause, aging-and-death exists.\footnote{I cannot say I fully understand Aggava\d msa's point on this matter. I stress `only' because that is the way Thai scholars translate it by applying the notion of \pali{avadh\=ara\d na}.}\\
\end{quote}

\paragraph*{Illustrating} For example:

\begin{quote}
\pali{`Sabba\d m atth\=i'ti kho, kacc\=ana, ayameko anto. `Sabba\d m natth\=i'ti aya\d m dutiyo anto.}\footnote{S2\,15 (SN\,12)}\\
This `Everything exists,' Kacc\=ana, is one extreme. This `Everything does not exist' is the second extreme.
\end{quote}

\phantomsection
\addcontentsline{toc}{section}{Some Minor Matters}
\section*{Some Minor Matters}

There are some minor things that there is no suitable place to put in. These include some assorted idioms that it is too early to be put in previous lessons. I describe them here.

\paragraph*{\pali{Pe} = etc.} If you see terms by frequency, you will find that \pali{pe} has many occurrences but it is not grouped with particles. What is this then? It is not even a word. This is the abbreviation of \pali{peyy\=ala}. It has nothing to do with grammar. It is a redactor's tool to represent an omission of repetitive portions of texts, hence ellipsis (\dots) or \textit{et cetera} (etc.).

\paragraph*{Action nouns can have an object.} Not only verbs can take an object, i.e.\ an accusative or genitive instance. Action or verbal nouns also do likewise. This is common to English too when a gerund takes an object, for example `doing something.' These nouns are normally nominal \pali{kita} formed by \pali{yu} or \pali{ana} (see page \pageref{pacck2:yu}), for example \pali{dassana} (seeing, sight). Here are examples from the canon:\footnote{See also \citealp[p.~138]{warder:intro}; \citealp[pp.~381--2]{perniola:grammar}.}

\begin{quote}
\pali{kaha\d m nu kho, bho, etarahi so bhava\d m gotamo viharati? Ta\~nhi maya\d m bhavanta\d m gotama\d m dassan\=aya idh\=upasa\.nkant\=a.}\footnote{D1\,259 (DN\,3)}\\
Sir, where does that Ven.\,Gotama stay now? We came here for seeing that Ven.\,Gotama.\\[1.5mm]
\pali{bhikkhuno \ldots sam\=adhi hoti dibb\=ana\d m r\=up\=ana\d m dassan\=aya \ldots, no ca kho dibb\=ana\d m sadd\=ana\d m savan\=aya}\footnote{D1\,366 (DN\,6). Objects of \pali{dassana} and \pali{savana} are in genitive form.}\\
There is meditation of a monk for seeing divine images, not for hearing divine sounds.\\
\end{quote}

\paragraph*{\pali{Kuto pana} = let alone (still less)} Literally this means `whence' or `from where.' It can be used generally as `why' or `how.' In certain contexts, accompanying with \pali{pi}, it fits to `let alone' or `still less' nicely, for example:

\begin{quote} 
\pali{Dasavass\=ayukesu, bhikkhave, manussesu kusalantipi na bhavissati, kuto pana kusalassa k\=arako.}\footnote{D3\,103 (DN\,26). Using \pali{iti} in \pali{kusalantipi} (\pali{kusala\d m + iti + pi}) is interesting here.}\\
``In the era that humans have [only] 10 years of lifespan, monks, among human beings even `good' does not exist, let alone a doer of goodness.''\\[1.5mm]
\pali{itthiratana\d m r\=aj\=ana\d m mah\=asudassana\d m manas\=api no aticari, kuto pana k\=ayena.}\footnote{D2\,249 (DN\,17)}\\
``Woman-jewel of king Mah\=asudassana did not commit adultery even with the mind, let alone with the body.''\\[1.5mm]
\pali{Yopissa so satth\=a sopi ma\d m neva khippa\d m j\=aneyya, kuto pana ma\d m aya\d m s\=avako j\=anissati}\footnote{M1\,506 (MN\,10)}\\
``Even the master would not know me quickly, why this disciple will know me?''\\[1.5mm]
\end{quote}

\paragraph*{\pali{Pageva} = let alone (still more)} This is somehow the reverse of \pali{kuto pana}, but sometimes they seem identical. In English we use `let alone' nonetheless. Here are examples:

\begin{quote}
\pali{Anuj\=an\=ami, bhikkhave, pa\~ncanna\d m satt\=ahakara\d n\=iyena appahitepi gantu\d m, pageva pahite.}\footnote{Mv\,3.193.}\\
``I allow you, monks, to go by a seven-day leave even when no one sent [to invite] by the five co-religionists, let alone having someone sent.''\\[1.5mm]
\pali{manasi k\=atumpi me es\=a, bhikkhave, dis\=a na ph\=asu hoti, pageva gantu\d m}\footnote{A3\,125}\\
``Monks, it is not comfortable for me even to think of that region, let alone to go [there].''\\[1.5mm]
\pali{ko nu kho, bho gotama, hetu ko paccayo, yena kad\=aci d\=igharatta\d m sajjh\=ayakat\=api mant\=a nappa\d tibhanti, pageva asajjh\=ayakat\=a?}\footnote{A5\,193}\\
``Why, Venerable Gotama, incantations which was recited for a long time do not become clear, let alone the unrecited ones?''\\[1.5mm]
\end{quote}

\section*{Exercise \ref{chap:iti}}
Translate this excerpt into P\=ali.\footnote{This is taken from the beginning part of Chapter 5 of Lewis Carroll's \emph{Alice's Adventures in Wonderland} (1865). This version is from the Project Gutenberg EBook (\url{http://gutenberg.org/ebooks/928}). The excerpt is not in its full form. I have cut some parts out to make it short but still connected. It is better to read the book yourselves.}
\begin{compactenum}
\item The Caterpillar and Alice looked at each other for some time in silence: at last the Caterpillar took the hookah out of its mouth and addressed Alice in a languid, sleepy voice.
\item ``Who are \emph{you}?'' said the Caterpillar.
\item This was not an encouraging opening for a conversation. Alice replied, rather shyly, ``I---I hardly know, sir, just at present---at least I know who I \emph{was} when I got up this morning, but I think I must have changed several times since then.''
\item ``What do you mean by that?'' said the Caterpillar, sternly. ``Explain yourself!''
\item ``I can't explain \emph{myself}, I'm afraid, sir,'' said Alice, ``because I'm not myself, you see.''
\item ``I don't see,'' said the Caterpillar.
\item Alice replied very politely ``\ldots being so many different sizes in a day is very confusing.''
\item ``It isn't,'' said the Caterpillar.
\item She drew herself up and said very gravely, ``I think you ought to tell me who \emph{you} are, first.''
\item ``Why?'' said the Caterpillar.
\item As Alice could not think of any good reason and the Caterpillar seemed to be in a \emph{very} unpleasant state of mind, she turned away.
\item ``Come back!'' the Caterpillar called after her. ``I've something important to say!''
\item Alice turned and came back again.
\item ``Keep your temper,'' said the Caterpillar.
\item ``Is that all?'' said Alice, swallowing down her anger as well as she could.
\item ``No,'' said the Caterpillar. It unfolded its arms, took the hookah out of its mouth again, and said, ``So you think you're changed, do you?''
\item ``I'm afraid, I am, sir,'' said Alice. ``I can't remember things as I used---and I don't keep the same size for ten minutes together!''
\item ``What size do you want to be?'' asked the Caterpillar.
\item ``Oh, I'm not particular as to size,'' Alice hastily replied, ``only one doesn't like changing so often, you know. I should like to be a little larger, sir, if you wouldn't mind,'' said Alice. ``Three inches is such a wretched height to be.''
\item ``It is a very good height indeed!'' said the Caterpillar angrily, rearing itself upright as it spoke (it was exactly three inches high).
\item In a minute or two, the Caterpillar got down off the mushroom and crawled away into the grass, merely remarking, as it went, ``One side will make you grow taller, and the other side will make you grow shorter.''
\item ``One side of \emph{what}? The other side of \emph{what}?'' thought Alice to herself.
\item ``Of the mushroom,'' said the Caterpillar, just as if she had asked it aloud; and in another moment, it was out of sight.
\end{compactenum}
