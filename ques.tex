\chapter{Are you going home?}\label{chap:ques}

\phantomsection
\addcontentsline{toc}{section}{More about Questioning}
\section*{More about Questioning}

In Chapter \ref{chap:vockim} we learn to ask questions using \pali{ki\d m} in various cases. In this chapter we will learn other ways of questioning. Other aspects concerning asking questions will be covered in this chapter. 

Like English, in a way, moving verb to the beginning of the sentence can form a simple close question. For example, ``Are you going home?'' can simply be:

\palisample{gacchasi ag\=ara\d m?}

Or alternatively, you can start the sentence with \pali{ki\d m} to mark the questioning. So, we can also put it in this way:

\palisample{ki\d m (tva\d m) ag\=ara\d m gacchasi?}

Both ways are useful in a conversational situation, for the context determines whether the utterance is question or not. When used in writing, this form of question can be ambiguous because the meaning of the sentence is not really controlled by its arrangement.\footnote{In ordination ceremony, the candidate is asked, among other questions, ``\pali{manusso'si}'' (Are you a human being?). This question is in normal order (\pali{manusso + asi}). To mark it as a question, the interrogators raise the voice in the last syllable.} To clarify the sentence P\=ali uses particles to facilitate the intended meaning. In Table \ref{tab:indques} particles used to mark interrogation are summarized. Some of these are also explained in Appendix \ref{chap:nipata}, page \pageref{nip:ques}.

\begin{table}[!hbt]
\centering
\caption{Interrogative particles}
\label{tab:indques}
\bigskip
\begin{tabular}{>{\itshape}lp{0.5\linewidth}} \toprule
\bfseries\upshape Particle & \bfseries Description \\ \midrule
ki\d msu, ki\d m & what? \\
katha\d m & how?, why?, for what reason? \\
kinnu & why?, is it? \\
kacci & is it? \\
nu (kho) & is it? \\
nanu & is it not? \\
ud\=ahu & \ldots or\ldots? \\
seyyath\=ida\d m & such as what? \\
\bottomrule
\end{tabular}
\end{table}

To make a close question, which `yes' or `no' is expected as an answer, we add \palibf{nu} to sentences to make it clearer. For the above question, so we get this:

\palisample{gacchasi nu (tva\d m) ag\=ara\d m?}

Often \pali{nu} is accompanied with \palibf{kho}, a filler particle. It does not add anything new to the meaning, just an emphasis like `indeed' or `really.' You can use this when you feel that only \pali{nu} is a bit too short. In a way, when \pali{kho} is used, it denotes a reflective doubt. Like you have a question in your mind.

\palisample{gacchasi nu kho ag\=ara\d m?}

When answering the question, you can use particles listed in Table \ref{tab:indans}. Some of these have explanation on page \pageref{nip:resp}.

\begin{table}[!hbt]
\centering
\caption{Answering particles}
\label{tab:indans}
\bigskip
\begin{tabular}{>{\itshape}ll} \toprule
\bfseries\upshape Particle & \bfseries description \\ \midrule
\=ama & yes \\
\=amant\=a & yes \\
eva\d m & yes, in that way \\
s\=adhu & yes, alright, well done \\
s\=ahu & yes, alright \\
na \ldots & not \ldots \\
\bottomrule
\end{tabular}
\end{table}

Therefore, a suitable affirmative answer to the question ``Are you going home?'' is \palibf{\=ama}, hence:

\palisample{\=ama.\sampleor[or with the verb repeated]\=ama (aha\d m) gacch\=ami.}

In very formal situation, \pali{\=amant\=a} can be used instead. In the canon, this word is found only in the Abbhidhamma. When responding with a negative answer, \palibf{na} with the verb is used:

\palisample{na gacch\=ami.}

Asking whether something exists or not, for example ``Do you have a book?'' You can put it like this:

\palisample{atthi nu (kho tuyha\d m) potthaka\d m?}

Use \pali{\=ama} to say `yes' and \pali{natthi} (\pali{na+atthi}) to say `no.' The full sentence of negative answer is:

\palisample{mayha\d m potthaka\d m natthi.\sampleor[or, to emphasize]natthi mayha\d m potthaka\d m.}

In general, \pali{na} is used to negate a verb by preceding it. To negate the whole sentence, \pali{na} can be put at the beginning. To learn more about negation see page \pageref{nip:neg}. 

Even though \palibf{nanu} has negative meaning, it can replace \pali{nu} in most cases. For example, ``Don't you go to school?'' is equivalent to:

\palisample{gacchasi nanu p\=a\d thas\=ala\d m?}

If you really go to school, the expected answer is `yes.' Hence, \pali{nu} and \pali{nanu} can be used interchangeably. Things can go a little complicated if you add another \pali{na} in front of the verb, like:

\palisample{na gacchasi nanu p\=a\d thas\=ala\d m.}

This means, a kind of, ``You don't go to school; is it true?'' So, if you really go to school, the expected answer is `no.'

In a close question, \palibf{kacci} can be used instead of \pali{nu} or \pali{nanu}, but this normally appears at the beginning of a sentence, for example:

\begin{quote}
\pali{kacci ma\d m, samma j\=ivaka, na va\~ncesi?}\footnote{D1\,159 (DN\,2)}\\
``J\=ivaka, my friend, don't you deceive me?''\\
\end{quote}

So, we can use this in our going-home example as follows:

\palisample{kacci (nu kho) gacchasi ag\=ara\d m?}

Now \pali{nu kho} is optional. If it sounds better, you can keep it. In the canon, you can find this quite often, for example:

\begin{quote}
\pali{kacci nu kho aha\d m p\=ar\=ajika\d m \=apatti\d m \=apanno}\footnote{Buv1\,67. In this sentence, past participle is used. To learn more about this, see Chapter \ref{chap:pp}.}\\
``Did I violate the gravest offense?''
\end{quote}

Apart from \pali{\=ama}, other terms that can be used in affirmative response are \palibf{eva\d m}, \palibf{s\=adhu}, and \palibf{s\=ahu}. When \pali{eva\d m} is used in response, it is more than just saying `yes.' It sounds like ``It is so'' or ``I agree with that'' or ``That is the case'' or ``What I will say is what you have said'' or ``I accept that as such.'' And when \pali{s\=adhu} or \pali{s\=ahu} is used, it has a positive tone of acceptance, like ``That is good'' or ``It is alright'' or ``It is well done.''

For open questioning, an explanation is expected as the response. This function is accomplished mainly by \pali{ki\d m}, as we have seen in Chapter \ref{chap:vockim}. There are some other particles that can be used in certain questions.

We can use \palibf{katha\d m}\footnote{In PTSD, there is some useful information of this, see the entry.} to ask `how' or `why' questions. For example, ``How do you go to school?'' can be asked as follows:

\palisample{katha\d m tva\d m p\=a\d thas\=ala\d m gacchasi?}

We can use \palibf{kinnu} (\pali{ki\d m+nu}) in reflective question, like you are deciding to do something. Here is an example from the canon:

\begin{quote}
\pali{Kinnu kho aha\d m sa\.nghassa veyy\=avacca\d m kareyya\d m?}\footnote{Buv1\,380}\\
``How should I do a service for the Sangha?''
\end{quote}

We can use \palibf{ud\=ahu} to ask that among options we have, which one should be selected. It is normally translated as `or.' See some examples on page \pageref{nip:udaahu}. You can use this, say, when you ask your friend ``Do you go to school by bus or by train?'' Here is its P\=ali:

\palisample{tva\d m p\=a\d thas\=ala\d m mah\=arathena ud\=ahu dh\=umarathena gacchasi?}

If the question asks between a binary option, for example, ``Will you go to school or not?'' We use \palibf{v\=a} in this case:

\palisample{gacchissasi v\=a no/na v\=a tva\d m p\=a\d thas\=ala\d m?}

Although \palibf{seyyath\=ida\d m} is not meant to be used in questions, it can mark interrogation by the context. Consider this dialogue:

\begin{quote}
A: \pali{ahampi bahul\=ani kusal\=ani karomi.}\\
(I even do many good things.)\\[1.5mm]
B: \pali{seyyath\=ida\d m?}\\
(Such as what?)\\[1.5mm]
A: \mbox{\pali{sunakh\=ana\d m \=ah\=ara\d m demi, te na padena pahar\=ami ca.}}\\
(I give food to dogs, and I do not kick them.)\\[1.5mm]
\end{quote}

For questioning about numbers, we use \pali{kita} and \pali{kittaka} as we have seen in Chapter \ref{chap:num}.

Now I come back to our protagonist \pali{ki\d m}, sometimes used as \pali{kim\d su}. Aggava\d msa summarizes that the term can express several things as follows:\footnote{Sadd-Pad Ch.\,12, from \pali{Etthetassa atthuddh\=aro vuccate} onwards, \citealp[p.~279]{smith:sadd1}.}

\paragraph*{(1) \pali{Garahane} (in reproach)}\label{par:kim} Much like English, or other language in this matter, questions can be treated as rebuke, for example, ``\pali{ki\d m r\=aj\=a yo loka\d m na rakkhati}'' (What kind of king who do not protect the world?). An example from the canon is ``\pali{Ki\d m nu kho n\=ama tumhe, \=avuso, ma\d m vattabba\d m ma\~n\~natha?}''\footnote{Vbh 2.424} (Guys, do you think I should be told/blamed?).

\paragraph*{(2) \pali{Animaye} (in uncertainty)} Aggava\d msa puts this as an example, ``\pali{ya\d m ki\~nci r\=upa\d m at\=it\=an\=agatapaccuppanna\d m}''\footnote{Mv\,1.22} (whatever form, past, future, or present). Learn more about this in Chapter \ref{chap:pron-misc}.

\paragraph*{(3) \pali{Nippayojane} (in uselessness)} Here is an example from the canon, ``\pali{vakkali, ki\d m te imin\=a p\=utik\=ayena di\d t\d thena?}''\footnote{S3\,87 (SN\,22)} (Vakkali, what's the use with this rotten body you've seen?).

\paragraph*{(4) \pali{Sampa\d t\d thicchane} (in acceptance)} This sounds like an affirmation of a promise, for example, ``\pali{ki\d m na k\=ah\=ami te vaco}''\footnote{Ja\,20:72} (Won't I do after your word? [Have I ever let you down?]).

\paragraph*{(5) \pali{Pucch\=aya\d m} (in interrogation)} That is the main use of the term we have learned so far. Apart from what we have learned in Chapter \ref{chap:vockim}, as an indeclinable \pali{ki\d m} can form questions in various ways. It is often accompanied with \pali{nu}. It can ask for `why?', sometimes with \palibf{k\=ara\d n\=a} (from reason), for example:

\begin{quote}
\pali{Ki\d m nu santaram\=anova, k\=asu\d m kha\d nasi s\=arathi}\footnote{Ja\,22:3} \\
``Charioteer, why do you dig a hole so quickly?'' \\[1.5mm]
\pali{Ki\d m nu j\=ati\d m na rocesi}\footnote{Thig\,7.190} \\
``Why don't you like birth?'' \\[1.5mm]
\pali{ki\d m nu bh\=itova ti\d t\d thasi}\footnote{S1\,90 (SN\,2)} \\
``Why do you stand frightened?'' \\[1.5mm]
\pali{amma, ki\d m k\=ara\d n\=a rodasi}\footnote{Dhp-a\,26.415} \\
``My dear lady, why do you cry?'' \\[1.5mm]
\pali{Ki\d m k\=ara\d n\=a amma tuva\d m pamajjasi}\footnote{Dhp-a\,8.112} \\
``My dear lady, why are you negligent?''
\end{quote}

It can be used for `what about?' or `how about?' or a kind of ``How's that going?,'' for example, ``\pali{Ki\d mcitto tva\d m, bhikkhu}''\footnote{Vbh 1.135; It can be used in compounds like this.} (How's your mind going, monk?). To ask for status of a person related to someone, you say ``\pali{es\=a te itth\=i ki\d m hoti}''\footnote{Sadd-Pad Ch.\,12} (What/How is this woman for/of you?).

\pali{Ki\d m} and \pali{ki\d msu} can be used to ask `what' in general, for example:

\begin{quote}
\pali{Ki\d msu chetv\=a sukha\d m seti, ki\d msu chetv\=a na socati}\footnote{S1\,71 (SN\,1)} \\
``What to be cut, [for] one sleeps happily, what to be cut, [for] one does not grieve.''
\end{quote}

It even can form a simple yes-no question like \pali{nu}, for example, ``\pali{Kh\=adasi ki\d m pivasi ki\d m}''\footnote{Sadd-Pad Ch.\,12} (Will you eat?, will you drink?).

\bigskip
Apart from the various ways of asking questions described above, there are some other idiomatic uses that can denote interrogation.\footnote{Vito Perniola also has a nice summary of how questions are formed in P\=ali, see \citealp[pp.~388--90]{perniola:grammar}. In the following part, I take some points from Perniola's ideas that I have never mentioned before.}

\paragraph*{\pali{Sacca\d m kira}} This means like ``Is it true?'' It is often found in the Vinaya when the Buddha asks monks whether they commit a certain offense. Here are some examples:

\begin{quote}
\pali{sacca\d m kira, bhikkhave, bhikkh\=u anupasampannena sahaseyya\d m kappenti}\footnote{Buv2\,49}\\
``Is it true, monks, that [some] monks sleep in the same place with a lay person?''\\[1.5mm]
\pali{sacca\d m kira tva\d m, ud\=ayi, m\=atug\=amassa dhamma\d m desesi}\footnote{Buv2\,60}\\
``Is it true, Ud\=ay\=i, that you teach the Dhamma to a woman?''\\[1.5mm]
\pali{sacca\d m kira tva\d m, ambho purisa, paresa\d m adinna\d m theyyasa\.nkh\=ata\d m \=adiyi}\footnote{D3\,91 (DN\,26)}\\
``Is it true, man, that you have taken ungiven [things] of others like a thief?''\\[1.5mm]
\pali{Sacca\d m kira tva\d m, nanda, sambahul\=ana\d m bhikkh\=una\d m evam\=arocesi?}\footnote{Ud\,3.22}\\
``Is it true, Nanda, that you have spoken to many monks in this way \ldots?''\\[1.5mm]
\end{quote}

\paragraph*{\pali{Atthi n\=ama}} This can mark a question with a surprise or rebuke. It may sound like ``Is it possible?'' or ``Is it true?'' The use of this is quite rare, for example:

\begin{quote}
\pali{atthi n\=ama, t\=ata sudinna, \=abhidosika\d m kumm\=asa\d m paribhu\~njissasi}\footnote{Buv1\,32}\\
``Sudinna my son, will you eat stale rice?''\\[1.5mm]
\pali{atthi n\=ama, \=ananda, thera\d m bhikkhu\d m vihesiyam\=ana\d m ajjhupekkhissatha}\footnote{A5\,166}\\
``Is it possible, \=Ananda, that you [all] look at a senior monk who is being harassed without taking any action?''\\[1.5mm]
\end{quote}


\section*{Exercise \ref{chap:ques}}
Say these in P\=ali.
\begin{compactenum}
\item Papa, why's the sky blue?
\item It is hard to understand, son.
\item Mama said it mirrors the ocean. Is that true?
\item No, don't tell anybody like that.
\item Maybe the space is blue, isn't it?
\item No, the space is dark.
\item Tell me why it is blue then.
\item The sunlight hits air molecules. By scattering of the light, the blue color dominates other colors because of higher frequency.
\item Your answer is useless. Asking mom is better.
\item How about rainbow, papa, where's it from?
\item It's from treasure-pots at the horizon.
\item That's nonsense.
\end{compactenum}
