\chapter{Verb Classes Summarized}\label{chap:vclass}

This chapter, together with Chapter \ref{chap:cases}, describes P\=ali grammar in depth. We will wrap up what we have learned about verbs, and go deeper into P\=ali verbal system. The approach in this chapter, like in Chapter \ref{chap:cases}, is tradition-wise. I will use traditional materials to explain the matter extensively. It is supposed to be difficult to new students, so we have not talked in this way at the beginning. Now I expect all readers to be mature enough to digest the real stuff. By this understanding, you can go on studying or researching into P\=ali grammar in the traditional way on your own.

Generally speaking there are two kinds of verb in P\=ali: \pali{\=akhy\=ata} and verbal \pali{kita} (primary derivation). I occasionally call the former `main verb' particularly when both kinds of verb are present together. I am reluctant to call them `finite'\footnote{They are verbs that are marked for tense, person, and number \citep[p.~172]{brownmiller:dict}.} and `non-finite'\footnote{They are verbs that have no mark of tense, person, and number, e.g.\ infinitives, participles \citep[p.~312]{brownmiller:dict}.} verbs, because in P\=ali both can complete the sentences. When present together \pali{\=akhy\=ata} dominates verbal \pali{kita} and functions as the main verb of the sentences. When \pali{\=akhy\=ata} is not present or left out, verbal \pali{kita} can perform the verb function. Moreover, as we have seen, in certain situation verbs can be left out altogether, and we still regard bundle of noun phrases as sentences.

In this chapter we will talk only about \pali{\=akhy\=ata}. For verbal \pali{kita}, it has several concerns that are divided into chapters as you have learned along the way. The meaning of \pali{\=akhy\=ata} given by Aggava\d msa is ``\pali{kiriya\d m akkh\=ayat\=iti \=akhy\=ata\d m, kiriy\=apada\d m}''\footnote{before Sadd\,865} ([Term] tells the action, thus \pali{\=akhy\=ata}, action-term). If I do not say otherwise, from now on `verb' means only \pali{\=akhy\=ata}.

Here is the big picture. A verb has components or characteristics as follows: Each verb is composed from \pali{dh\=atu} + \pali{paccaya} + verbal \pali{vibhatti}, for example, \pali{bhavati} = \pali{bh\=u + a + ti}. Verbal \pali{vibhatti} has 8 classes, i.e.\ present tense (\pali{vattam\=an\=a}), imperative mood (\pali{pa\~ncam\=i}), optative mood (\pali{sattam\=i}), perfect tense (\pali{parokkh\=a}), imperfect tense (\pali{hiyyattan\=i}), aorist tense (\pali{ajjatan\=i}), future tense (\pali{bhavissanti}), and conditional mood (\pali{k\=al\=atipatti}). We have already learned all these classes along in our course. Moreover, verbal \pali{vibhatti} can be divided into 12 groups. The first 6 groups is called \pali{parassapada}\footnote{Kacc\,406, R\=upa\,429, Sadd\,865} (term for other), and the last 6 groups \pali{attanopada}\footnote{Kacc\,407, R\=upa\,439, Sadd\,866} (term for oneself). In most cases when we use by ourselves and when we read from texts, \pali{parassapada} is far more common. In 6 groups of each, 3 are for singular, and other 3 are for plural. And in these groups of three, they are divided into persons: third person (\pali{pa\d thamapurisa}), second person (\pali{majjhimapurisa}), and first person (\pali{uttamapurisa}).\footnote{Kacc\,408, R\=upa\,431, Sadd\,867} The order of persons is reversed to those of English. To illustrate the point, Table \ref{tab:vibhvatt} show all \pali{vibhatti}s of present tense (\pali{vattam\=an\=a}). For all classes of verb, see Appendix \ref{chap:conj}. 

\begin{table}[!hbt]
\centering
\caption{\pali{Vattam\=an\=avibhatti}}
\label{tab:vibhvatt}
\bigskip
\begin{tabular}{>{\bfseries}l*{4}{>{\itshape}c}} \toprule
Person & \multicolumn{2}{c}{\bfseries\itshape Parassapada} & \multicolumn{2}{c}{\bfseries\itshape Attanopada} \\
\cmidrule(l){2-3}\cmidrule(l){4-5}
& \bfseries\upshape sg. & \bfseries\upshape pl. &  \bfseries\upshape sg. &  \bfseries\upshape pl. \\
\midrule
3rd & ti & nti & te & nte \\
2nd & si & tha & se & vhe \\
1st & mi & ma & e & mhe \\
\bottomrule
\end{tabular}
\end{table}

What baffles new students most is the difference between \pali{parassapada} and \pali{attanopada}. The former means the action that the subject does affects other entity, for example, ``\pali{jano kum\=ara\d m paharati}'' (A person hits a boy). Technically speaking, \pali{parassapada} is used with the agent of active structure (\pali{kattuk\=araka}).\footnote{Kacc\,456, R\=upa\,430, Sadd\,937. For the \pali{k\=araka} thing, see Chapter \ref{chap:cases}.} On the other hand, the action of the latter affects the subject itself, for example, ``\pali{kum\=aro janena pahariyate}'' (A boy is hit by a person). That is to say, \pali{attanopada} is used in passive structure (\pali{kammak\=araka \& bh\=avak\=araka}).\footnote{Kacc\,453, R\=upa\,444, Sadd\,934} However, \pali{attanopada} in active structure can also be the case\footnote{Kacc\,454, R\=upa\,440, Sadd\,935}, for example, \pali{ma\~n\~nate} ([One] deems), \pali{rojate} ([One] prospers), \pali{j\=ayate} ([One] is born).

Comparing to other ancient languages, like Greek and Sanskrit, the two modes are named `active' and `middle' voice by scholars. William Whitney paraphrases these as `transitive' and `reflexive.'\footnote{\citealp[\S529:p.~200]{whitney:grammar}} That sounds more sensible. As in Sanskrit, the exact distinction between the two is blurred or even undiscernible. Middle voice left its trace mostly in verses. Whitney wrote this:

\begin{quote}
In the epics there is much effacement of the distinction between active and middle, the choice of voice being very often determined by metrical considerations alone.\footnote{\citealp[\S529:p.~200]{whitney:grammar}. See also \citealp[p.~117]{geiger:grammar}.}
\end{quote}

The point of this matter for practical concern is ``Don't be serious with the distinction.'' As Geiger noted, in the oldest period of the language passive verb forms already have active endings.\footnote{\citealp[p.~117]{geiger:grammar}} That is the reason why you did not see verbs in middle form at the beginning of our lessons. You have to know this when you read texts, but when you make your compositions, decision is yours. For me, simplicity is the best policy. Furthermore, not every root has middle forms. Unlike active forms, you cannot render verbs into middle forms in full range, so to speak.

\label{par:multiactors}The main task of us concerning \pali{vibhatti} is to choose the right ending (\pali{vibhatti}) according to \pali{pada}, as mentioned above, and person. A problematic case is when multiple actors do the same action. Which person should we use? In P\=ali grammar, verbs agree to the last actor\footnote{Kacc\,409, R\=upa\,441, Sadd\,868, Mogg\,1.22}, for example, ``\pali{so ca pacati, tva\~nca pacasi, tumhe pacatha}'' (He cooks, you cook too, [thus] you [all] cook), ``\pali{so ca pacati, tva\~nca pacasi, aha\~nca pac\=ami, maya\d m pac\=ama}'' (He cooks, you also cook, I cook too, [thus] we cook). It is logical to use plural verb form, but sometimes you can see singular nevertheless. As you have often seen, even when the subject is not present, the verb has to be agreed with person implied in the sentence.\footnote{Kacc\,410--2, R\=upa\,432, 436--7, Sadd\,869--71} Sometimes discrepancy can be the case\footnote{Sadd\,1099}, for example, ``\pali{Putta\d m \textbf{labhetha} varada\d m}''\footnote{Ja\,22:1661} (May [I] have a son who gives the best thing). In the example, the implied subject is first person, thus \pali{labheyy\=ami} should be used instead of \pali{labhetha}.

However, Vito Perniola explains the use of multiple subjects in this way: ``If the subjects contain different persons, the verb agrees with the first person in preference to the second and third, and with the second in preference to the third.''\footnote{\citealp[p.~341]{perniola:grammar}} Then he shows us this example:

\begin{quote}
\pali{Aha\~nca, \=ananda, im\=ani ca pa\~nca bhikkhusat\=ani sabbeva \=ane\~njasam\=adhin\=a nis\=idimha}\footnote{Ud\,3.23}\\
``I and 500 monks, \=Ananda, all sat in motionless meditation.''\\
\end{quote}

In this instance, the verb \pali{nis\=idimha} (aorist, 1st person pl.) is used in the sense that English users are familiar, against the explanation in the traditional textbooks (but see below shortly). But if disjunctive particle \pali{v\=a} is used instead, the verb agrees with the (preceding) nearest subject, for example:

\begin{quote}
\pali{Ya\d mn\=un\=aha\d m v\=a pabbajeyya\d m, anuruddho v\=a}\footnote{Cv\,7.330}\\
``What if I or Anuruddha were to go forth.''\\
\end{quote}

The sentence above is a speculation. The verb, \pali{pabbajeyya\d m} (optative, middle voice, 1st person sg.), agrees with \pali{aha\d m}. According to the explanation if the verb is shifted to the last position, it would be ``\pali{Ya\d mn\=un\=aha\d m v\=a anuruddho v\=a pabbajeyya}.'' Now \pali{pabbajeyya} agrees with \pali{anuruddho}.

In Sadd-Pad Ch.\,2, Aggava\d msa mentions the use of multiple subjects by inference from the meaning (\pali{atthanaya}).\footnote{Sadd-Pad Ch.\,2, from \pali{Aparopi atthanayo vuccati} onwards.} He illustrates by these examples:

\begin{quote}
\pali{tva\~nca atthakusalo bhavasi, so ca atthakusalo bhavati, tumhe atthakusal\=a bhavatha}\\
``You are clever in beneficial seeking. He is also clever in beneficial seeking. You [all] are clever in beneficial seeking.''\\[1.5mm]
\pali{aha\~nca atthakusalo bhav\=ami, so ca atthakusalo bhavati, mayamatthakusal\=a bhav\=am\=a}\\
``I am clever in beneficial seeking. He is also clever in beneficial seeking. We are clever in beneficial seeking.''\\[1.5mm]
\end{quote}

By this account, we can feel at home when using multiple subjects. The lesson here is when we say something just do what makes us feel right. Language should agree with our natural tendency.

In the following sections we will go through each class of verbs in more detail.

\phantomsection
\addcontentsline{toc}{section}{Present Tense (\pali{Vattam\=an\=a})}
\section*{Present Tense (\pali{Vattam\=an\=a})}

When composed with \pali{vattam\=an\=avibhatti}, `to go' (from root \pali{gamu}) in present tense can be seen in Table \ref{tab:exvatt}.\footnote{Sadd-Dh\=a\,845} According to the tradition, this tense can be used in various way concerning time as follows:

\begin{table}[!hbt]
\centering
\caption{Present forms of `to go' (\pali{gamu})}
\label{tab:exvatt}
\bigskip
\begin{tabular}{@{}>{\bfseries}l*{4}{>{\itshape}l}@{}} \toprule
Person & \multicolumn{2}{c}{\bfseries\itshape Parassapada} & \multicolumn{2}{c}{\bfseries\itshape Attanopada} \\
\cmidrule(l){2-3}\cmidrule(l){4-5}
& \bfseries\upshape sg. & \bfseries\upshape pl. &  \bfseries\upshape sg. &  \bfseries\upshape pl. \\
\midrule
3rd & gacchati & gacchanti & gacchate & gacchante \\
2nd & gacchasi & gacchatha & gacchase & gacchavhe \\
1st & gacch\=ami & gacch\=ama & gacche & gacch\=amhe \\
\bottomrule
\end{tabular}
\end{table}

\paragraph*{(1) In present time} (Kacc\,414, R\=upa\,428, Sadd\,872)\par
This, also some of the following, is equivalent to simple present tense of English. Also present continuous tense, or progressive aspect, can be used in this sense.\par
- \pali{p\=a\d taliputta\d m \textbf{gacchati}.} ([One] goes to P\=a\d taliputta [Patna].)\par
- \pali{bhagav\=a s\=avatthiya\d m \textbf{viharati} jetavane.}\footnote{Ud\,4.36} (The Blessed One lives in Jetavana nearby S\=avatth\=i.)\par

\paragraph*{(2) In near past} (Sadd\,873)\par
- \pali{kuto nu tva\d m bhikkhu \textbf{\=agacchasi}.} (From where, monk, do you come?)\par

\paragraph*{(3) In near future with \pali{y\=ava, pure, pur\=a}} (Sadd\,874)\par
- \pali{Y\=avadeva anatth\=aya, \~natta\d m b\=alassa \textbf{j\=ayati}.}\footnote{Dhp\,5.72} (Knowledge of a foolish will arise only for uselessness.)\par
- \pali{Pure adhammo \textbf{dippati}}\footnote{Cv\,12.450} (Before false teaching will prosper.)\par
- \pali{dante ime chinda pur\=a \textbf{mar\=ami.}}\footnote{Ja\,16:127} (Cut these tusks before I die.)\par

\paragraph*{(4) In the future that has certainty} (Sadd\,875)\par
- \pali{Niraya\d m n\=una \textbf{gacch\=ami}, ettha me natthi sa\d msayo.}\footnote{Apad\=a 2-1.48} ([I] certainly will go to hell. There is no doubt for me in this.)\par
- \pali{dhuva\d m buddho \textbf{bhav\=am}aha\d m.}\footnote{Bv\,2:109} (I certainly will be an Enlightened One)\par
- \pali{Manas\=a ce padu\d t\d thena, \textbf{bh\=asati} v\=a \textbf{karoti} v\=a.}\footnote{Dhp\,1.1. This instance is of unspecified condition (\pali{aniyamattha}).} (If [one] says or does with the mind corrupted.)\par

\paragraph*{(5) In the future with \pali{kad\=a, karahi}} (Sadd\,876)\par
- \pali{kad\=a \textbf{gacchati}?} (When does [he/she] go?)\par
- \pali{karahi \textbf{gacchati}?} (In what time does [he/she] go?)\par
It is also logical to use future tense here, thus \pali{kad\=a/karahi gamissati}.

\paragraph*{(6) In the past with \pali{nanu}} (Sadd\,877)\par
- \pali{ak\=asi ka\d ta\d m devadatta? nanu \textbf{karomi} bho.} (Did you make the mat, Devadatta? Haven't I made it, sir?)\par

\paragraph*{(7) In the past with \pali{na, nu}} (Sadd\,878)\par
- \pali{ak\=asi ka\d ta\d m devadatta? na \textbf{karomi} bho.} (\ldots I haven't done that, sir.)\par
- \pali{\ldots, aha\d m nu \textbf{karomi}.} (\ldots Have I done that?)\par
It is also logical to use past tense here, hence \pali{n\=ak\=asi\d m, nvak\=asi\d m}.

\paragraph*{(8) In the past as narration} (Sadd\,879)\par
- \pali{Bhaya\d m tad\=a na \textbf{bhavati}.}\footnote{Bv\,2:100} (There was no danger in that time.)\par

\phantomsection
\addcontentsline{toc}{section}{Imperative Mood (\pali{Pa\~ncam\=i})}
\section*{Imperative Mood (\pali{Pa\~ncam\=i})}

Table \ref{tab:expanc}\footnote{Sadd-Dh\=a\,845. The presence of \pali{gacchassu} in \pali{parassapada} is unusual, but the term does exist in the list by Aggava\d msa. This may show that the normal use of this form is seeable.} shows imperative forms of `to go.' Several forms of these are identical to the present forms. It is good, for you do not need to remember many things. It is bad, for you have to make a judgement when you come across an ambiguous one. Generally this mood is used for making an order or a wish in unspecified time or near the present.\footnote{Kacc\,415, R\=upa\,451, Sadd\,880} For the uses of this mood in detail, see Chapter \ref{chap:imp}, additionally see the section of optative mood below.

\begin{table}[!hbt]
\centering
\caption{Imperative forms of `to go' (\pali{gamu})}
\label{tab:expanc}
\bigskip
\begin{tabular}{@{}>{\bfseries}l*{4}{>{\itshape}l}@{}} \toprule
Person & \multicolumn{2}{c}{\bfseries\itshape Parassapada} & \multicolumn{2}{c}{\bfseries\itshape Attanopada} \\
\cmidrule(l){2-3}\cmidrule(l){4-5}
& \bfseries\upshape sg. & \bfseries\upshape pl. &  \bfseries\upshape sg. &  \bfseries\upshape pl. \\
\midrule
3rd & gacchatu & gacchantu & gacchata\d m & gacchanta\d m \\
2nd & gacch\=ahi, & gacchatha & gacchassu & gacchavho \\
& gaccha, & & & \\
& (gacchassu) & & & \\
1st & gacch\=ami & gacch\=ama & gacche & gacch\=amase \\
\bottomrule
\end{tabular}
\end{table}

\phantomsection
\addcontentsline{toc}{section}{Optative Mood (\pali{Sattam\=i})}
\section*{Optative Mood (\pali{Sattam\=i})}

This mood is used for making a permission, supposition, and instruction in unspecified time.\footnote{Kacc\,416, R\=upa\,454, Sadd\,881, Mogg\,6.9, 6.12} In a way, it is similar to imperative mood. In some contexts they are even used interchangeably. For the uses in detail, please see Chapter \ref{chap:opt}. Optative forms of `to go' are shown in Table \ref{tab:exsatt}.\footnote{Sadd-Dh\=a\,845}

\begin{table}[!hbt]
\centering\small
\caption{Optative forms of `to go' (\pali{gamu})}
\label{tab:exsatt}
\bigskip
\begin{tabular}{@{}>{\bfseries}l*{4}{>{\itshape}l}@{}} \toprule
Per. & \multicolumn{2}{c}{\bfseries\itshape Parassapada} & \multicolumn{2}{c}{\bfseries\itshape Attanopada} \\
\cmidrule(l){2-3}\cmidrule(l){4-5}
& \bfseries\upshape sg. & \bfseries\upshape pl. &  \bfseries\upshape sg. &  \bfseries\upshape pl. \\
\midrule
3rd & gaccheyya, & gaccheyyu\d m & gacchetha & gacchera\d m \\
& gacche & & & \\
2nd & gaccheyy\=asi & gaccheyy\=atha & gacchetho & gaccheyy\=avho \\
1st & gaccheyy\=ami & gaccheyy\=ama, & gaccheyya\d m & gaccheyy\=amhe \\
& & gacchemu & & \\
\bottomrule
\end{tabular}
\end{table}

Apart from the uses described in Chapter \ref{chap:opt}, there are some other concerns as follows:

\paragraph*{(1) Making an order, instruction, and time reminding} (Sadd\,882)\par
This formula is also applied to the imperative.\par
- \pali{bhava\d m khalu ka\d ta\d m \textbf{karotu}.} (You definitely have to make a mat.)\par
- \pali{bhava\d m khalu ka\d ta\d m \textbf{kareyya}.} (You definitely should make a mat.)\par
- \pali{Pu\~n\~n\=ani \textbf{kayir\=atha} sukh\=avah\=ani}\footnote{S1\,3 (SN\,1)} (You should make merit that brings happiness)\par
- \pali{aya\d m te saccak\=alo, sacca\d m \textbf{vadeyy\=asi}.} (This is your time of truth. You should say the truth.)\par

\paragraph*{(2) Time telling with \pali{ya\d m}} (Sadd\,883)\par
- \pali{ya\d m \textbf{bhu\~njeyya} bhava\d m.} (You should eat in which time.)\par

\paragraph*{(3) In suitability and capability} (Sadd\,884, Mogg\,6.11)\par
- \pali{bhava\d m khalu ka\~n\~na\d m \textbf{gaheyya}, bhava\d m etamarahati.} (You definitely should seize the girl, you deserve this.)\par
- \pali{iha bhava\d m vattu\d m \textbf{sakku\d neyya}} (In here, you are capable of saying)\par

\phantomsection
\addcontentsline{toc}{section}{Perfect Tense (\pali{Parokkh\=a})}
\section*{Perfect Tense (\pali{Parokkh\=a})}

The use of this tense is rare in P\=ali texts. Do not confuse this with perfect tense in English. It has nothing to do with that. Some modern P\=ali grammar books do not even mention it at all. Some even make it look confusing.\footnote{For example, in \citealp[p.~80]{collins:grammar}, whereas aorist is mentioned, perfect \pali{vibhatti} is described.} The main cause of this is about English grammatical terms we use for P\=ali which do not exactly fit. To understand this and P\=ali past tenses in general, we have to invest some effort to unravel the crux of this matter.

A grammatical term that has to be introduced here is `aspect'---``An indication of whether the action is ongoing, completed, or not specified''.\footnote{\citealp[p.~113]{fairbairn:understanding}} Verbs that denote ongoing actions have \emph{imperfect} or \emph{progressive} aspect. Verbs denoting completed actions have \emph{perfect} aspect. And verbs that describe the actions as a whole, with no reference to whether they are completed or not, have \emph{simple} or \emph{indefinite} aspect.\footnote{\citealp[p.~110]{fairbairn:understanding}} The last one may be called \emph{habitual} aspect that denotes a habit or regular pattern.\footnote{\citealp[p.~204]{brownmiller:dict}} These aspects can be of three times: past, present, and future. In English usage, we can grasp these in Table \ref{tab:engaspect}.\footnote{This is adapted from the table in \citealp[p.~118]{fairbairn:understanding}.}

\begin{table}[!hbt]
\centering
\caption{Time and aspect in English}
\label{tab:engaspect}
\bigskip
\begin{tabular}{@{}>{\bfseries}llll@{}} \toprule
Time & \multicolumn{3}{c}{\bfseries Aspect} \\
\cmidrule{2-4}
& \bfseries Ongoing & \bfseries Completed & \bfseries Unspecified \\
& (Progressive) & (Perfect) & (Simple) \\
\midrule
Past & I was doing & I had done & I did \\
Present & I am doing & I have done & I do \\
Future & I will be doing & I will have done & I will do \\
\bottomrule
\end{tabular}
\end{table}

When Greek and Latin are taken into consideration, terminology used is a bit confusing as I show in Table \ref{tab:gltenses}.\footnote{This is adapted from the table in \citealp[p.~123]{fairbairn:understanding}.} After you see this table, you will know that the very cause of confusion comes from grammatical terms used to describe P\=ali equivalents. The use was started by Sanskrit scholars who saw similarity between Greek and Sanskrit. And we use Greek grammatical terms since then. For \pali{parokkh\=a}, by traditional explanation, it is used to mark past actions with indefinite time.\footnote{Kacc\,417, R\=upa\,460, Sadd\,885} That means `aorist' in Greek and `perfect' in Latin. Modern scholars use `perfect' for \pali{parokkh\=a}. It has the sense of completeness of events done in remote past, unperceived by the narrator.\footnote{\citealp[p.~134]{williams:grammar}} Some P\=ali scholars, e.g.\ A.\,P.\,Buddhadatta, use `preterite'\footnote{This is ``equivalent to Simple Past'' \citep[p.~357]{brownmiller:dict}.} for \pali{parokkh\=a}. But we will not follow that.

\begin{table}[!hbt]
\centering\small
\caption{Tenses in Greek and Latin}
\label{tab:gltenses}
\bigskip
\begin{tabular}{@{}llp{0.12\linewidth}>{\raggedright\arraybackslash}p{0.28\linewidth}@{}} \toprule
\bfseries Greek name & \bfseries Latin name & \bfseries Time & \bfseries Aspect \\
\midrule
Present & Present & Present & Ongoing \\
Imperfect & Imperfect & Past & Ongoing \\
Future & Future & Future & Ongoing\linebreak (or unspecified) \\
Aorist & Perfect & Past & Unspecified\linebreak (or completed) \\
Perfect & Perfect & Past/ Present & Completed but with continuing results \\
Pluperfect & Pluperfect & Past & completed \\
Future perfect & Future perfect & Future & completed \\
\bottomrule
\end{tabular}
\end{table}

In Table \ref{tab:exparo}\footnote{Sadd-Dh\=a\,845}, perfect forms of `to go' is shown, for you can get the idea what they look like. Here are some examples found in the texts:

- \pali{Codako \textbf{\=aha} \=apannoti.}\footnote{Pvr\,359} (The plaintiff said, ``[He] offended.'')\par
- \pali{\textbf{\=Ahu} bya\~njananimittakovid\=a.}\footnote{Dh\=i 3.7.209 (DN 30)} (Said diviners [who are] well-versed in signs)\par
- \pali{eva\d m kira por\=a\d n\=a \textbf{\=ahu}\footnote{Sadd\,885. In Kacc\,417, it is ``\pali{eva\d m kila por\=a\d n\=ahu.}''}} (Former [teachers] said thus)\par

\begin{table}[!hbt]
\centering
\caption{Perfect forms of `to go' (\pali{gamu})}
\label{tab:exparo}
\bigskip
\begin{tabular}{@{}>{\bfseries}l*{4}{>{\itshape}l}@{}} \toprule
Person & \multicolumn{2}{c}{\bfseries\itshape Parassapada} & \multicolumn{2}{c}{\bfseries\itshape Attanopada} \\
\cmidrule(l){2-3}\cmidrule(l){4-5}
& \bfseries\upshape sg. & \bfseries\upshape pl. &  \bfseries\upshape sg. &  \bfseries\upshape pl. \\
\midrule
3rd & gaccha & gacchu, & gacchittha, & gacchire \\
& & ga\~nchu & ga\~nchittha & \\
2nd & gacche & gacchittha, & gacchittho & gacchivho \\
& & ga\~nchittha & & \\
1st & gaccha\d m & gacchimha, & gacchi\d m, & gacchimhe \\
& & ga\~nchimha & ga\~nchi\d m & \\
\bottomrule
\end{tabular}
\end{table}

\phantomsection
\addcontentsline{toc}{section}{Imperfect Tense (\pali{Hiyyattan\=i})}
\section*{Imperfect Tense (\pali{Hiyyattan\=i})}

This tense is used for the past events that happened yesterday with time specified or not.\footnote{Kacc\,418, R\=upa\,456, Sadd\,886} As you have seen above, calling this `imperfect' is really a mismatch from P\=ali grammarians' point of view. There is no sense of `ongoingness' or `progressiveness' in this tense whatsoever.\footnote{Monier Williams notes that Sanskrit past tenses ``are not very commonly used to represent the completeness of the action'' \citep[p.~134]{williams:grammar}. This means they do not express the progressiveness either. However, Williams also explains that this tense corresponds to the imperfect of Greek that refers to recent past but before the current day. It may denote continuity or be used like Greek aorist.} To ease our life, we follow the terminology nonetheless. Like perfect tense, imperfect tense in P\=ali is rare. And both are virtually identical in meaning.\footnote{Once these two had different denotation, but the difference has been lost even in Classical Sanskrit \citep[p.~271]{ruppel:sanskrit}.} Exemplified forms of this are shown in Table \ref{tab:exhiyys}\footnote{Sadd-Dh\=a\,845}, and alternatively in Table \ref{tab:exhiyym}\footnote{Mogg\,6.5}. The forms of this tense are mostly prefixed with \pali{a} (augment). Some examples are as follows:

- \pali{so \textbf{agam\=a} magga\d m.} (He went the path.)\par
- \pali{te \textbf{agam\=u} magga\d m.} (They went the path.)\par
- \pali{\textbf{Agam\=a} r\=ajagaha\d m buddho}\footnote{Snp\,3.410. This can also be seen as aorist.} (Went to R\=ajagaha the Buddha)\par

\begin{table}[!hbt]
\centering
\caption{Imperfect forms of `to go' (\pali{gamu})}
\label{tab:exhiyys}
\bigskip
\begin{tabular}{@{}>{\bfseries}l*{4}{>{\itshape}l}@{}} \toprule
Per. & \multicolumn{2}{c}{\bfseries\itshape Parassapada} & \multicolumn{2}{c}{\bfseries\itshape Attanopada} \\
\cmidrule(l){2-3}\cmidrule(l){4-5}
& \bfseries\upshape sg. & \bfseries\upshape pl. &  \bfseries\upshape sg. &  \bfseries\upshape pl. \\
\midrule
3rd & agacch\=a & agacch\=u & agacchatha & agacchatthu\d m \\
2nd & agaccho & agacchatha & agacchase & agacchavha\d m \\
1st & agaccha\d m & agacchamha & agacchi\d m, & agacchamhase \\
& & & aga\~nchi\d m & \\
\bottomrule
\end{tabular}
\end{table}

\begin{table}[!hbt]
\centering
\caption{Imperfect forms of `to go' (alternative)}
\label{tab:exhiyym}
\bigskip
\begin{tabular}{@{}>{\bfseries}l*{4}{>{\itshape}l}@{}} \toprule
Person & \multicolumn{2}{c}{\bfseries\itshape Parassapada} & \multicolumn{2}{c}{\bfseries\itshape Attanopada} \\
\cmidrule(l){2-3}\cmidrule(l){4-5}
& \bfseries\upshape sg. & \bfseries\upshape pl. &  \bfseries\upshape sg. &  \bfseries\upshape pl. \\
\midrule
3rd & agam\=a & agam\=u & agamattha & agamatthu\d m \\
2nd & agamo & agamattha & agamase & agamavha\d m \\
1st & agama & agamamh\=a & agami\d m & agamamhase \\
\bottomrule
\end{tabular}
\end{table}

\phantomsection
\addcontentsline{toc}{section}{Aorist Tense (\pali{Ajjatan\=i})}
\section*{Aorist Tense (\pali{Ajjatan\=i})}

In traditional account this tense is used in the near past, events that happen today, with time specified or not.\footnote{Kacc\,419, R\=upa\,469, Sadd\,887} Modern scholars call this `aorist' that has nothing to do with traditional account. We still use this for convenience, so we can make a distinction to other past tenses. In fact, the distinction is only the names of them, because they are identical in use.\footnote{Geiger notes that imperfect and aorist ``are no longer sharply distinguished in Pali. Both of them have coincided in the pret[erite] which is mostly called 
`aorist'\,'' \citep[p.~117]{geiger:grammar}.} For verbs used in past, aorist forms are far more common than the previous two. I show examples of a verb in Table \ref{tab:exajjs}\footnote{Sadd-Dh\=a\,845}, and alternatively Table \ref{tab:exajjm}\footnote{Mogg\,6.4}. About the \pali{a} prefix, in aorist case, as well as conditional mood, it is uncertain---meaning that you can find both forms, with and without \pali{a}, for example, \pali{agacchi} and \pali{gacchi}.\footnote{Sadd-Dh\=a\,845} Both forms can be identical in all respects. In practice, for 3rd person sg.\ we often see \pali{i} ending rather than \pali{\=i}, and alternative or irregular forms of this tense are quite various. 

\begin{table}[!hbt]
\centering
\caption{Aorist forms of `to go' (\pali{gamu})}
\label{tab:exajjs}
\bigskip
\begin{tabular}{@{}>{\bfseries}l*{4}{>{\itshape}l}@{}} \toprule
Per. & \multicolumn{2}{c}{\bfseries\itshape Parassapada} & \multicolumn{2}{c}{\bfseries\itshape Attanopada} \\
\cmidrule(l){2-3}\cmidrule(l){4-5}
& \bfseries\upshape sg. & \bfseries\upshape pl. &  \bfseries\upshape sg. &  \bfseries\upshape pl. \\
\midrule
3rd & agacchi, & agacchu\d m, & agacch\=a, & agacch\=u \\
& aga\~nchi & aga\~nchu\d m & & \\
2nd & agaccho & agacchittha, & agacchase & agacchivha\d m \\
& & aga\~nchittha & & \\
1st & agacchi\d m, & agacchimh\=a & agaccha\d m, & agacchimhe \\
& aga\~nchi\d m & aga\~nchimh\=a & & \\
\bottomrule
\end{tabular}
\end{table}

\begin{table}[!hbt]
\centering
\caption{Aorist forms of `to go' (alternative)}
\label{tab:exajjm}
\bigskip
\begin{tabular}{@{}>{\bfseries}l*{4}{>{\itshape}l}@{}} \toprule
Person & \multicolumn{2}{c}{\bfseries\itshape Parassapada} & \multicolumn{2}{c}{\bfseries\itshape Attanopada} \\
\cmidrule(l){2-3}\cmidrule(l){4-5}
& \bfseries\upshape sg. & \bfseries\upshape pl. &  \bfseries\upshape sg. &  \bfseries\upshape pl. \\
\midrule
3rd & agam\=i & agamu\d m, & agam\=a & agam\=u \\
2nd & agamo & agamittha & agamise & agamivha\d m \\
1st & agami\d m & agamimh\=a & agama & agamimhe \\
\bottomrule
\end{tabular}
\end{table}

Here are some simple examples of use:\par
- \pali{so magga\d m \textbf{agam\=i}.} (He went the path.)\par
- \pali{te magga\d m \textbf{agamu\d m}.} (They went the path.)\par
- \pali{va\.nka\d m \textbf{agamu} pabbata\d m.}\footnote{Cp\,1:106. This can be seen as a contracted form of imperfect or aorist.} ([They] went to mount Va\.nka.)\par
- \pali{\textbf{upagacchu\d m} buddhasantike.}\footnote{Bv\,12:16} ([They] approached the Buddha's dwelling.)\par
The verb can have a special form, e.g.\ ``\pali{te \textbf{gu\d m}}'' (They went).\footnote{Sadd-Dh\=a\,845}\par

There are some other concerns about aorist and other past tense relating to \pali{m\=a}.\label{par:vclassma} Normally, particle \pali{m\=a} is used to make a prohibition. It logically agrees with imperative mood, but as found in the texts imperfect and aorist tense are used mostly.\footnote{Kacc\,420, R\=upa\,471, Sadd\,888, Mogg\,6.13. But in the commentaries, imperative has more use (Sadd\,889).} Here are some examples:\par
- \pali{kha\d no vo m\=a \textbf{upaccag\=a}.}\footnote{Dhp\,22.315.} (Don't let the moment run away.)\par
- \pali{M\=a vo \textbf{ruccittha} gamana\d m.}\footnote{Ja\,22:1891} (Don't be pleased with the going.)\par
- \pali{m\=a dhamma\d m r\=aja \textbf{p\=amado}.}\footnote{Ja\,17:48. The word should be \pali{pam\=ado}.} (Your Majesty, don't be negligent in the teaching)\par
- \pali{M\=a\textbf{kattha} p\=apaka\d m kamma\d m.}\footnote{Ud\,5.44} (Don't do evil deed.)\par
In Sadd\,890, it is said that perfect and present tense are even less than imperative mood to be found in the canon. Some examples are given nonetheless:\par
- \pali{m\=a \textbf{kisittho} may\=a vin\=a.}\footnote{Ja\,22:1713. In some editions, it is \pali{kisittha}.} (Don't be exhausted without me.)\par
- \pali{M\=a deva \textbf{paridevesi}}\footnote{Ja\,22:1857} (Dear god, don't lament.)\par

\phantomsection
\addcontentsline{toc}{section}{Future Tense (\pali{Bhavissanti})}
\section*{Future Tense (\pali{Bhavissanti})}

This tense is easy to deal with. It denotes future events.\footnote{Kacc\,421, R\=upa\,473, Sadd\,892} In Table \ref{tab:exbham}\footnote{Mogg\,6.2}, typical future forms of `to go' are shown, and in Table \ref{tab:exbhas}\footnote{Sadd-Dh\=a\,845} alternative rendition is shown. There are some minor concerns about this tense that I have already explained in Chapter \ref{chap:fut}. Here are some simple examples:

- \pali{so \textbf{gacchissati}} (He will go)\par
- \pali{so \textbf{karissati}} (He will do)\par

\begin{table}[!hbt]
\centering
\caption{Future forms of `to go' (\pali{gamu})}
\label{tab:exbham}
\bigskip
\begin{tabular}{@{}>{\bfseries}l*{4}{>{\itshape}l}@{}} \toprule
Per. & \multicolumn{2}{c}{\bfseries\itshape Parassapada} & \multicolumn{2}{c}{\bfseries\itshape Attanopada} \\
\cmidrule(l){2-3}\cmidrule(l){4-5}
& \bfseries\upshape sg. & \bfseries\upshape pl. &  \bfseries\upshape sg. &  \bfseries\upshape pl. \\
\midrule
3rd & gamissati & gamissanti & gamissate, & gamissante \\
2nd & gamissasi & gamissatha & gamissase & gamissavhe \\
1st & gamiss\=ami & gamiss\=ama & gamissa\d m & gamiss\=amhe \\
\bottomrule
\end{tabular}
\end{table}

\begin{table}[!hbt]
\centering\small
\caption{Future forms of `to go' (alternative)}
\label{tab:exbhas}
\bigskip
\begin{tabular}{@{}>{\bfseries}l*{4}{>{\itshape}l}@{}} \toprule
Per. & \multicolumn{2}{c}{\bfseries\itshape Parassapada} & \multicolumn{2}{c}{\bfseries\itshape Attanopada} \\
\cmidrule(l){2-3}\cmidrule(l){4-5}
& \bfseries\upshape sg. & \bfseries\upshape pl. &  \bfseries\upshape sg. &  \bfseries\upshape pl. \\
\midrule
3rd & gacchissati & gacchissanti & gacchissate & gacchissante \\
2nd & gacchissasi & gacchissatha & gacchissase & gacchissavhe \\
1st & gacchiss\=ami & gacchiss\=ama & gacchissa\d m & gacchiss\=amhe \\
\bottomrule
\end{tabular}
\end{table}

\phantomsection
\addcontentsline{toc}{section}{Conditional Mood (\pali{K\=al\=atipatti})}
\section*{Conditional Mood (\pali{K\=al\=atipatti})}

This tense, in a way, is like in English when we talk about an action that should have done, but it did not. Its forms look like a combination of past and future. I show typical forms of `to go' in Table \ref{tab:exkalm}\footnote{Mogg\,6.7}, and alternatively in Table \ref{tab:exkals}\footnote{Sadd-Dh\=a\,845}. The \pali{a}-prefix is mostly present, but sometimes it is left out. By traditional account, this tense is used to mark actions that do not really happen.\footnote{Kacc\,422, R\=upa\,475, Sadd\,895} For more detail of conditionals, see Chapter \ref{chap:cond}. An example can be:

- \pali{so ce y\=ana\d m \textbf{alabhiss\=a, agacchiss\=a}.}\footnote{Sadd\,895} (If he had got a vehicle, he would have gone.)\par

\begin{table}[!hbt]
\centering\footnotesize
\caption{Conditional forms of `to go' (\pali{gamu})}
\label{tab:exkalm}
\bigskip
\begin{tabular}{@{}>{\bfseries}l*{4}{>{\itshape}l}@{}} \toprule
Per. & \multicolumn{2}{c}{\bfseries\itshape Parassapada} & \multicolumn{2}{c}{\bfseries\itshape Attanopada} \\
\cmidrule(l){2-3}\cmidrule(l){4-5}
& \bfseries\upshape sg. & \bfseries\upshape pl. &  \bfseries\upshape sg. &  \bfseries\upshape pl. \\
\midrule
3rd & agamiss\=a, & agamissa\d msu, & agamissatha, & agamissi\d msu \\
2nd & agamisse & agamissatha, & agamissase & agamissavhe \\
1st & agamissa\d m, & agamiss\=amh\=a & agamissi\d m, & agamiss\=amhase \\
\bottomrule
\end{tabular}
\end{table}

\begin{table}[!hbt]
\centering\scriptsize
\caption{Conditional forms of `to go' (alternative)}
\label{tab:exkals}
\bigskip
\begin{tabular}{@{}>{\bfseries}l*{4}{>{\itshape}l}@{}} \toprule
Per. & \multicolumn{2}{c}{\bfseries\itshape Parassapada} & \multicolumn{2}{c}{\bfseries\itshape Attanopada} \\
\cmidrule(l){2-3}\cmidrule(l){4-5}
& \bfseries\upshape sg. & \bfseries\upshape pl. &  \bfseries\upshape sg. &  \bfseries\upshape pl. \\
\midrule
3rd & agacchiss\=a, & agacchissa\d msu, & agacchissatha, & agacchissi\d msu \\
2nd & agacchisse & agacchissatha, & agacchissase & agacchissavhe \\
1st & agacchissa\d m, & agacchiss\=amh\=a & agacchissi\d m, & agacchiss\=amhase \\
\bottomrule
\end{tabular}
\end{table}

Concerning \pali{i} insertion, as you see in \pali{gam\textbf{i}ssati} but not in \pali{gacchati}, the tradition has an explanation that imperfect tense, imperative mood, optative mood, and present tense do not have this insertion, whereas the rest of them, i.e.\ perfect tense, aorist tense, future tense, and conditional mood have it.\footnote{Kacc\,431, R\=upa\,458, Sadd\,904} It is better for you to observe this yourselves.



