\chapter{I \headhl{went} to school}\label{chap:past}

\phantomsection
\addcontentsline{toc}{section}{Introduction to Past Tense}
\section*{Introduction to Past Tense}

At this point, it is suitable to introduce other verb forms. To be healthy, let us cope with the bitterest now---past tense(s). Before we come to that, it is better to talk about verb in general first. Previously, we have met verb `to be' in Chapter \ref{chap:verb-be} and verb `to go' in Chapter \ref{chap:verb-go}. These two verbs, and their relatives, are among the most used. Even though we can use these and other verbs to say some simple things, it is by no means enough to make a normal conversation. We have to learn more, and there are a lot to learn.

Before you read any further, there is a task you should do first. In Appendix \ref{vocab:verb}, page \pageref{vocab:verb} onwards, I list a number of common verbs essential to our learning process. Now you are supposed to read through the table one time, at least. You may not understand what you see there, but this makes you familiar with P\=ali verb forms. You will know what is waiting in future lessons. Try to grasp the patterns. 

\begin{center}
Please do the task. I am waiting.\\
\ldots
\end{center}

Now you come back, and find out that patterns of verb formation can be discerned to some degree. There are several irregularities found. That is exactly what I want you to see. Now you have prepared your mind to meet oddities in P\=ali grammar and other chaotic stuffs. With this awareness, I choose a simpler method to introduce P\=ali verb system, unlike the traditional approach.

Influenced by Sanskrit grammar, the tradition learns verbs from their \emph{root}.\footnote{``The central part of a word which cannot be broken up into smaller morphs'' \citep[p.~389]{brownmiller:dict}.} We can call this `bottom-up' approach. To explain this, let me exemplify with an English verb---\emph{conversed}. We can break it down into three parts, called \emph{morphs}\footnote{``[T]he smallest chunks into which spoken or written words can be divided'' \citep[p.~294]{brownmiller:dict}.}, \emph{con-} + \emph{verse} + \emph{-ed}. The central part of this word is \emph{verse} which came from Latin \emph{vers\=are}, `to turn.' This is the root of the word. Other constituent parts, \emph{con-} and \emph{-ed}, are called \emph{affix}. To be precise, an affix added at the beginning is called \emph{prefix}, whereas the one added at the end is \emph{suffix}. Affixes modify the meaning and function of the word. For example, \emph{con-} meaning ``together with (other people)'' makes \emph{converse} ``to turn together with other people'' which refers to ``to engage in a spoken exchange of thoughts, ideas, or feelings.''\footnote{Meaning has two sides: \emph{sense} and \emph{reference}. Sometimes word formation makes an unintelligible sense but refers to a simple thing or action. Be careful of this distinction, particularly in religious context. Confusion between the two often ends up with a dispute, mostly a nonsense one. It is just an argument over words.} Another affix part, \emph{-ed} makes the verb function as past tense.

As a far relative to English, P\=ali also utilizes the same kind of tactic in verb formation and word formation in general, but much more elaborate. If you learn P\=ali from a traditional school, you have to know roots first. Then you learn how they can be composed and transformed into various words under certain rules. For example, to say `to go' you have to learn that root \pali{gam} belongs to root group I\footnote{According to Kacc and Sadd, roots can be divided into eight groups. In that tradition, this root is called \pali{gamu}.} which has \pali{-a} as its group suffix (\pali{paccaya}). Then you can form a present tense, 3rd person, singular, active voiced verb as \pali{gam + a + ti}, resulting in \pali{gacchati}. Why \pali{gam} becomes \pali{gacch} is enigmatic to me until now.\footnote{Linguists may have an explanation on this. It might have something to do with phonetics. Or it is a mix-up with another root. I am not sure about that.}

At this point, we can differentiate two terms---\emph{root} and \emph{stem}\footnote{``Any chunk of a word to which an affix can be added'' \citep[p.~416]{brownmiller:dict}}. Root is more fundamental than stem. When a word is formed, the root can undergo changes under certain phonetic rules before it is annexed with affixes. What we really see in this case is stem not root. As exemplified above, \pali{gam} is root, whereas \pali{gacch} is stem. Sometimes, when root is not changed, its stem takes the same form.\footnote{In fact, \pali{gamati} has its uses in P\=ali texts, but far rarer than \pali{gacchati}. From \textsc{Pal\=i\,Platform} 3, in the CSTR collection \pali{gacchati} has 4,741 occurrences, whereas \pali{gamati} has only 2.}

Let me sum up my point. By traditional approach, we have to learn rules of verb formation from the ground up. That takes time and effort and is somewhat daunting. The hard part is that rules do not always work. There are many exceptions and irregularities as you can see when you examine our verb table. You do not only remember rules, but also their exceptions. They are too overwhelming to new students.

My approach here is simpler. Let me call it `top-down' approach. We will learn verbs as children learn to speak. As a child, we do not care how the words we use come. We just use the words as we hear them. When we come across words many times, we, or our brain, can detect the patterns and formulate grammatical rules in our mind.\footnote{Noam Chomsky would say that those grammatical rules are innate.} This is a natural way to learn a language. Therefore, you do not need to know roots. You only have to recognize stems. However, if your goal is more than just to speak or read, say, to be a grammarian, you have to follow the traditional way.

Here is my strategy to cope with oddities in P\=ali grammar.
\begin{compactenum}[(1)]
\item Be familiar with irregularities. That is the very reason I suggest you to examine our verb list first.
\item Learn the patterns. You do not need to remember a great number of rules. Studying from patterns of word formation is quicker. 
\item Follow simple rules. We, nonetheless, have some general rules to learn. When you create a word, use generic patterns first.
\item Remember some conspicuous oddities. It is worth remembering very weird forms. They are not so many. Most odd words are common to use in every day life. Those words also happen frequently in the scriptures because they are very ancient ones. That is the reason why they are still there. When generic forms do not look quite right, they may take irregular forms. If you are familiar with oddities, you can recall them instantly.
\end{compactenum}

You might wonder why I have to introduce so long. How to simply say ``I went to school'' anyway? In fact, there are many things yet to discuss. We will learn P\=ali verb system by traditional way in Chapter \ref{chap:vclass} and Chapter \ref{chap:vform}. To the point, there are two ways to say things in past: using main verb forms, and using derivative verb forms, past participle in this case. We will talk about past participle later in Chapter \ref{chap:pp}. Now we will deal only with the main verb forms that are categorized precisely into eight classes: five tenses and three moods, traditionally ordered as (1) present tense, (2) imperative mood, (3) optative mood, (4) perfect tense, (5) imperfect tense, (6) aorist tense, (7) future tense, and (8) conditional mood. You can see all verbal conjugations in Appendix \ref{chap:conj}. In principle, you can say things in past by using three tenses: perfect, or imperfect, or aorist. In practice, only aorist tense is widely used, and the remaining two are virtually absent from the scriptures.

Therefore, the main lesson in this chapter is how to use aorist tense (\pali{Ajjattan\=i}). As we have learned from present tense conjugation in Chapter \ref{chap:verb-go}, we have to know person and number of the actor before we apply the endings to verb's stems. In our verb list, I give you only 3rd-person, singular, active-voiced forms. So, you have to work out by yourselves to render the verbs properly. As an example, I show you the aorist conjugation of verb `to go' (\pali{gacchati}) in Table \ref{tab:aorgacch}.\footnote{R\=upa\,470} Only active forms (\pali{parassapada}) are presented here.

\bigskip
\begin{longtable}[c]{@{}>{\raggedright\arraybackslash}p{0.15\linewidth}%
	>{\raggedright\arraybackslash\itshape}p{0.3\linewidth}%
	>{\raggedright\arraybackslash\itshape}p{0.4\linewidth}@{}}
\caption{Aorist conjugation of \pali{gacchati}}\label{tab:aorgacch}\\
\toprule
\bfseries\upshape Person & \bfseries\upshape Singular & \bfseries\upshape Plural \\ \midrule
\endfirsthead
\multicolumn{3}{c}{\tablename\ \thetable: Aorist conjugation of \pali{gacchati} (contd\ldots)}\\
\toprule
\bfseries\upshape Person & \bfseries\upshape Singular & \bfseries\upshape Plural \\ \midrule
\endhead
\bottomrule
\ltblcontinuedbreak{3}
\endfoot
\bottomrule
\endlastfoot
%%
3rd & gacchi, gacch\=i & gacchi\d msu, gacchu\d m \\
2nd & gacchi, gaccho & gacchittha \\
1st & gacchi\d m & gacchimha, gacchimh\=a \\
\midrule
3rd & agacchi, agacch\=i & agacchi\d msu, agacchu\d m \\
2nd & agacchi, agaccho & agacchittha \\
1st & agacchi\d m & agacchimha, agacchimh\=a \\
\midrule
3rd & ga\~nchi, ga\~nch\=i & ga\~nchi\d msu, ga\~nchu\d m \\
2nd & ga\~nchi, ga\~ncho & ga\~nchittha \\
1st & ga\~nchi\d m & ga\~nchimha, ga\~nchimh\=a \\
\newpage
3rd & aga\~nchi, aga\~nch\=i & aga\~nchi\d msu, aga\~nchu\d m \\
2nd & aga\~nchi, aga\~ncho & aga\~nchittha \\
1st & aga\~nchi\d m & aga\~nchimha, aga\~nchimh\=a \\
\midrule
3rd & gami, gam\=i, (gam\=asi) & gami\d msu, gama\d msu, gamu\d m \\
2nd & gami, gamo & gamittha, gamuttha \\
1st & gami\d m & gamimha, gamumha, gamimh\=a \\
\midrule
3rd & agami, agam\=i, agam\=asi & agami\d msu, agama\d msu, agamu\d m \\
2nd & agami, agamo & agamittha, agamuttha \\
1st & agami\d m & agamimha, agamumha, agamimh\=a \\
\end{longtable}

You might feel panic right now when you find that in the vocabulary (Appendix \ref{vocab:verb}) I give you only \pali{gacchi} but the tradition gives you several. ``How can I know this?,'' you might also grumble. To understand the situation, let us exercise some thought with me. Considering that ``How did the tradition know all these?,'' you might be more pacified. When there were no grammatical book like we have nowadays in the past, the language learners had to examine the texts thoroughly and recorded distinct forms of terms. When certain patterns were detected, they were put into formulas. However, by sedimentary nature of the texts, terms used sometimes resisted the formulation. Terms were formed in a variety of ways, showing that they came from a variety of sources. We can also see this effect in nominal forms because there are plenty of irregularity, but in a manageable degree. Considering verbal forms, you will see that they are indeed much diverse than nouns. No textbook can list you all the possible verbal forms. Textbooks can only give you some typical cases. For the rest you have to experiment by yourselves under the given rules.

To pep you up a little bit, \pali{gacchati} is one in a handful of terms that has a great variety, because it is a very common verb. So, we have not many like this to deal with. If you see it as the worst case, you may feel better now. To simply use it, you just follow our principle of using verbs: be aware of person and number. And here we go for ``I went to school.''

\palisample{(aha\d m) p\=a\d thas\=ala\d m (a)gacchi\d m.\sampleor p\=a\d thas\=ala\d m (a)ga\~nchi\d m.\sampleor p\=a\d thas\=ala\d m (a)gami\d m.}

These are for ``You went to school.''

\palisample{(tva\d m) p\=a\d thas\=ala\d m (a)gacchi/(a)gaccho.\sampleor p\=a\d thas\=ala\d m (a)ga\~nchi/(a)ga\~ncho.\sampleor p\=a\d thas\=ala\d m (a)gami/(a)gamo.}

Finally, ``He/She went to school.''

\palisample{(so/s\=a) p\=a\d thas\=ala\d m (a)gacchi/(a)gacch\=i.\sampleor p\=a\d thas\=ala\d m (a)ga\~nchi/(a)ga\~nch\=i.\sampleor p\=a\d thas\=ala\d m (a)gami/(a)gam\=i/(a)gam\=asi.}

A question now pops up in your mind: ``What is the leading \pali{a-} for?'' In fact, it adds nothing to the meaning. If you really curious, here is a kind of explanation from Aggava\d msa:

\begin{quote}
\pali{Tattha ajjataniy\=a k\=al\=atipattiy\=a ca ak\=ar\=agama\d m sabbesu purisesu sabbesu vacanesu labbham\=anampi s\=asane aniyat\=a hutv\=a labbhat\=iti da\d t\d thabba\d m. Tath\=a hi ``agacchi, gacchi, agacchiss\=a, gacchiss\=a''ti\=adin\=a dve dve r\=up\=ani dissanti.}\footnote{Sadd-Dh\=a\,845}\\
``In that matter, it is worth seeing that in \pali{ajjatan\=i} and \pali{k\=al\=atipatti}, obtaining \pali{a}-prefixed [terms happens] in all persons, in all numbers, but in the teaching [this] obtaining [is] uncertain. It is so, because dual forms such as `\pali{agacchi, gacchi; agacchiss\=a, gacchiss\=a}' are seen.''
\end{quote}

Aggava\d msa says nothing about the meaning of the prefix \pali{a}.\footnote{In fact, this is called `augment' by linguists.} He just admits that we find both instances, with and without that prefix. To see a clearer picture, I list aor.\ of \pali{gacchati} (only 3rd-person sg.) counted by \textsc{P\=ali\,Platform} 3 using the CSTR collection in Table \ref{tab:gacchi}.

\begin{table}[!hbt]
\centering
\caption{List of aor.\ of \pali{gacchati}}
\label{tab:gacchi}
\bigskip
\begin{tabular}{>{\itshape}lrr} \toprule
Term & Total Freq & G\=ath\=a Freq \\ 
\midrule
gacchi & 8 & 3 \\
agacchi & 5 & 3 \\
gacch\=i, agacch\=i & 0 & 0 \\
ga\~nchi & 3 & 0 \\
aga\~nchi & 5 & 3 \\
ga\~nch\=i, aga\~nch\=i & 0 & 0 \\
gami & 16 & 10 \\
agami & 1 & 0 \\
gam\=i & 5 & 3 \\
agam\=i & 3 & 2 \\
gam\=asi & 0 & 0 \\
agam\=asi & 1,553 & 27 \\
\bottomrule
\end{tabular}
\end{table}

I will leave the analysis of these data to you. If you are more curious, experiment yourselves with other forms. It is obvious that some forms are more fashionable in verses, e.g.\ \pali{gami}. And \pali{agam\=asi} is overwhelmingly popular aor.\ form of \pali{gacchati}. All these tell you that do not take alternative verb forms as well as the prefix \pali{a} seriously. You have to know the variation when you read texts. When you use it by yourselves, in speaking or writing, it is a matter of style.

For those who have good eyes, you may think of a discrepancy here. Whereas the rule says that the ending of 3rd-person sg., of aor.\ is \pali{\=i} (see Appendix \ref{chap:conj}), why \pali{i} is more used. It is true that aor.\ ending with \pali{\=i} is rarely found. I do not know the real reason of this. It seems that those who use the language prefer short sound over long one.\footnote{Sadd\,1041 and Mogg\,6.33 just say that long ending vowels sometimes are shortened.} It might just be easier to pronounce, using less energy. Perhaps, explanation from linguists/philologists can illuminate this more. And why does it become \pali{agam\=asi} then? I cannot explain this either. The tradition just says sometimes \pali{s} is added without giving any informative reason. Maybe, those who have a good knowledge of Sanskrit can explain this. Now you know why learning P\=ali verbs is difficult. Teaching this in a digestible way is even more difficult. With my method, I hope new students are able to grasp the subject easier and quicker (as well as having more fun in learning, may I add).

Before we depart this lesson, I leave you Table \ref{tab:conjasi} showing aorist conjugation of verb \pali{atthi}\footnote{R\=upa\,500} which has a wild irregularity.

\begin{table}[!hbt]
\centering
\caption{Aorist conjugation of \pali{atthi}}
\label{tab:conjasi}
\bigskip
\begin{tabular}{l*{2}{>{\itshape}l}} \toprule
\bfseries Person & \bfseries\upshape Singular & \bfseries\upshape Plural \\ \midrule
3rd & \=asi & \=asi\d msu, \=asu\d m \\
2nd & \=asi & \=asittha \\
1st & \=asi\d m & \=asimha \\
\bottomrule
\end{tabular}
\end{table}

Now it is your turn to do the exercise.

\section*{Exercise \ref{chap:past}}
Say these in P\=ali using verbs in the vocabulary.
\begin{compactenum}
\item Why did you not go to school yesterday?
\item I was sick and I went to the hospital.
\item What did the doctor say to you?
\item He told me, ``Going to school is not suitable.''
\item Did you do your homework?
\item The doctor also said, ``Lying in bed is better.''
\end{compactenum}
