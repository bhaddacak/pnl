\markboth{}{Preface}
\cleardoublepage
\phantomsection
\addcontentsline{toc}{chapter}{Preface}
\chapter*{Preface}

I write this book with a hope that it can be an easy starter to P\=ali studies which I wish it had been my first book on P\=ali language. The main theme of this book comes from a simple question: ``What is the quickest way to learn a language?'' It seems to me that the best answer is ``by using it in conversation.'' This is true for many living languages in the world, but not quite so in a dead language like P\=ali.

Whether P\=ali is a dead language can be a point of controversy, because in some situations the language is really used in conversation. When no other common language available, Theravada monks from different countries converse in P\=ali. This is true only for those who are well-versed in the language. That is to say, traditionally monks learn P\=ali mainly for translating scriptures, not for conversation. Those who are able to speak P\=ali are more or less near the level of P\=ali expert. They have to master the grammar and vocabulary first by many years of study.

The approach of this book is the reverse of that. We will start learning the language by simple conversation. The aim is not to make P\=ali a daily language (it is pointless to do that), but to make sense of the language in a familiar way. In the traditional way of learning, students have to remember many grammatical rules before they translate a portion of text. That is a big waste of time. Many rules have exceptions, some rules occur seldom in the text, and some rules have conflicting stances according to traditional grammarians. Why do new learners have to remember all of those? How about if we focus on crucial grammatical points first by using them in a simple context? That is the main idea of my approach.

Surprisingly, conversational approach to P\=ali is not really popular. You can find rare books on the topic. The noted one is \emph{Aids to Pali Conversation and Translation} by A.\,P.\,Buddhadatta Mah\=athera.\footnote{\citealp{buddhadatta:aids}} The book looks much like a traveling guide with typical situations and few grammatical guidelines. That means it is not suitable for beginners to start learning the language, but rather it is a supplement for those who are quite familiar with the language to some extent. Nevertheless, this book inspires me and helps me form some ideas of the present book. My aim is different, not to make a better guidebook, but to use conversation as the beginning point leading to grammatical explanation. By this way of learning, the language makes sense at the start, unlike in the traditional way that new learners have to be bombarded with rules long before they can grasp the ideas. The drawback of conversational approach is that we cannot touch upon every bit of rules. We can just talk about some big rules and try to use them. That is a digestible way of learning a new language. It also makes more fun.

\bigskip
The previous part as you have read so far was written before the first chapter is formulated. The following part is written after all chapters are done one year later.

At first, I thought the book would not grow this big. My primary intention is fulfilled perfectly. The book finally can be used as a P\=ali primer (like one I dreamed of). Furthermore, with my decision to incorporate traditional accounts into the book, it now becomes a reference manual at the same time.\footnote{The main factor that enables me to do this is Dr.\,Supaphan Na Bangchang's book on traditional P\=ali grammar (\citealp{supaphan:pali}). Without this work I cannot quickly capture the essence of P\=ali grammar of three main schools. Other notable Thai translations I use are Saddan\=iti by Phra Maha Nimit Dhammas\=aro and Chamroon Thammada, Niruttid\=ipan\=i by Sompob Sa-nguanpanit, and so many more that I cannot list them here.} That is a kind of book I also sought after when I investigated into the language, but very few was accessible. Now my dream is bundled into your hand.

Put it another way, this book serves two purposes. First, as a primer, it uses conversational approach to introduce new learners to P\=ali in a less intimidating way. You can find this part in the lessons. Second, as a reference, it contains most of materials used by the tradition in learning process. You can find this part mostly in the appendices. That means the book is self-contained. You just need only one book to learn the language at starter level. In addition, with a companion program, \textsc{P\=ali\,Platform}, you have most of P\=ali literature in hand together with a powerful search function and useful tools. Knowledgeable teachers can quicken your learning process, but they are not really necessary because you have all of digestible materials here. The only thing you need to learn the language successfully is perseverance, maybe plus some motivation.

My target readers, apart from those who want to study the language academically, are ones adherent to the tradition who want to learn the language in a more effective and healthy way. Let me make clear why I stress on healthy way of learning. To the tradition, P\=ali is a sacred language because it preserves what the Buddha taught. No one disputes on that. And any sacred language is supposed to be difficult. This means few experts can understand it properly. A consequence of this is those few specialists determine what the rest should believe and practice. From my background of religious studies, one major factor that sets the direction of religious tenets is politics. This means many things Buddhists believe are just for political purposes.\footnote{This sounds quite modern to see that religion and other domains of life are separate areas. In fact, there is no such separation in the ancient mind. See it another way, religion is the only effective tool in the past to keep the society in order. But now the situation is different because religion begins to lose its power, and the close link between religion and politics is now visible resulting in attempts to separate these two areas. Yet, in the modern world many religions are still powerful in keeping the social order.} Healthy learning thus means you can learn to read the sources by yourselves and decide that whether it is worth believing or observing or not. That is the only way a religion can serve the public for their own benefits, not just for benefits of executive few.\footnote{It is idealistic, in my view, to establish a purely democratic society like this. Once religion becomes a cultural component of the society, it is really hard to challenge, even if its original tenets are inconsistent, distorted, or even outright wrong. Even though it is hard to do, we still have to acknowledge it as such and try our best.}

I have a short treatment on the point that Buddhism, or any religion in this matter, always has political dimension. If I ask ``What is the main purpose of Buddhism?,'' everyone should answer that it is about soteriology or salvation or liberation. The next question is ``How much do you need to know for liberation?'' A straight answer of this is ``Not much.'' And you even do not need to know P\=ali or any sacred scripture thoroughly. You just need to know how to observe yourselves properly. That is all for practical purpose. Then the main reason why we have many things to learn about the religion and to entrench them in our belief is all about politics, both in the religion's own sphere and governmental sphere.\footnote{Talking about this issue can be a book-length discussion. It might be more accurate to say that all religions are economics-based. That sounds rather Marxist, but I think a purely soteriological religion is really hard to find on earth. I do not mean religious people seek after wealth (except Weber's Protestantism, perhaps), but people must have things to eat first. That makes social structure and political system indispensable. Then the idea of salvation comes along. However, religion makes us believe that soteriological goal in the cosmic order exists from the beginning. Then politics and social order accommodate people to that goal. Some readers might think we have many things to learn because they are an intellectual enterprise of human beings. That is true, but scholastic endeavor has little to do with real salvation. It just maintains the tradition, hence the economic and politic bases.}

I have to make myself clear at the beginning why I am so critical to our object of learning, as you shall see throughout our course. My point is that any good knowledge should have a liberating effect. When we really know something by ourselves, not just by being told, it can change us in a subtle way. If you have enough integrity, the change will be in a good way and liberating. That is the practical\footnote{You may expect `spiritual' for this word. I do not like the term because it sounds spooky. I have a down-to-earth and realistic view on a religious journey. For me, a spiritually awakening life is just a healthy life in its entirety.} purpose of this book, apart from the scholarly one.

Here is my future plan. The next volume from this one will be about how to read texts. I will bring various theories into play with P\=ali translation, such as semiotics, hermeneutics, literary theories, translation theories, and so on. I will not just write a manual of P\=ali translation, but I will go deeper as far as modern knowledge can guide us.\footnote{Now this book is retitled to \emph{P\=ali Text Reading: A Handbook}. See \url{bhaddacak.github.io/ptr}.} And, if possible, the third book will be about P\=ali composition and prosody.\footnote{I have written a short treatment of Pāli prosody in \textit{Pāli\,Platform's User Manual}. See \url{bhaddacak.github.io/ppmanual}.} That will be less theoretical and focus more on practical technique. That last one is not quite appealing to me to write, but it can make the series complete. I assert no strong commitment of that, but for the second one I have already prepared some materials.

Some might be curious why I am motivated to do difficult things, even though they bring me no financial gain whatsoever. That is my healthy way of living. It is simple: you set goals, and finish them one by one. If you have difficult goals, you just have a few big things to do in your life. Take your time and enjoy your life. When finished, they will be great. The outcome does not matter much really. You just have an opportunity to focus on one thing at a time. I am lucky to have not many desires. Hence, I have little distraction. People give me food, and that is enough for me to live happily without worrying about making a living.

\subsection*{How to use the book}

The book is roughly divided into two parts: lessons and appendices. For the lesson part, it is meant to be learned sequentially. Chapter 1 is about introduction to P\=ali language. New learners may find this too difficult, or too critical. You can skip this one if you like. Chapter 2--35 are the primer part. You are supposed to go through these one by one. From Chapter 3 onwards, there is an exercise at the end of each chapter. You are encouraged to exercise your knowledge before you go to its answer keys (Appendix \ref{chap:keys}).

Chapter 36--38 are theoretical summaries, mostly about the verb system and cases. They are essential but too difficult to learn at the beginning. I place them at the end after you know how to use the language. In the traditional way, you have to learn all these before you start reading texts. You can feel how tough traditional students are.

Chapter 39 is all about conversation. All knowledge you learn will be applied here. This chapter is not necessary to read as the last lesson. You can go to this if you are curious how to put things into practice. There are cross references to related lessons in this chapter. But if you do not hurry, make it the last one is better, like you eat a pudding at the end of the main course.

The other half of the book is additional materials. You can read them in any order. They are supposed to be read after you are familiar with the language to some degree, and you want to be equipped with additional information unprovided in the lessons. All these materials are not necessary to know at the start. If you can read all of them, however, you will know the heart of the traditional approach, and you will know where to find further materials.

My writing style goes between formal and casual extreme. If you have heavily academic mind, please tolerate my playful moments. English is not my native tongue, so you have to tolerate it too. My main concern is how to make the book reader-friendly and enjoyable to write.

The author's website by now is \url{bhaddacak.github.io}. Please check there for a new revision and other works. The life cycle of electronic publishing is short, so a new edition can come out quickly. You can contact the author personally by emailing to \texttt{bhaddacak} at \texttt{proton} dot \texttt{me}.\footnote{FYI: The author does not use any kind of social media, and he is not online all the time. Normally, he connects to the Internet when necessary, sometimes once in one or two weeks or even in a month or more. Mostly, he lives in seclusion spending time with reading, writing, and meditating. That is a peaceful way of living in the modern world.}

\subsection*{Notes on 2.0 edition}

After I have finished \textsc{P\=ali\,Platform} 2 and written its manual, I feel I should do something with this book too. First, I remove the appendix on \textsc{P\=ali\,Platform} because we already have the full user's manual now. And second, I update all numbers, mostly term frequencies, taken from the first version of the program to the newer ones.

In \textsc{P\=ali\,Platform} 2, I utilize a different approach to tokenize the text, so the term frequencies change as a result. The change is not drastic, except in the case of \pali{ti} (an elided form of \pali{iti}). Many instances that have \pali{ti} (or \pali{nti}) annexed are now split, if detectable. So, you see that the frequency of \pali{ti} doubles in Chapter \ref{chap:ind-intro} (now the numbers are changed in accordance with \textsc{P\=ali\,Platform} 3).

One big change, due to the change of my working environment, is the overall format of the book. When I use a new version of \TeX\ Live, the fonts look smaller and line spacing is narrower. So, in this release the total number of pages shrinks to under 800 (from more than 1,000).

Another minor matter is I decide to remove the graphic cover from the book, and change the cover to a simple text-based design. The graphic cover makes the file unnecessarily bigger, and it looks too commercial to me.

Finally, I apply a Creative Commons public license to the document (now it is changed to \emph{GNU Free Documentation License}). It is suitable to an electronic publication like this. My side intention of this is to demonstrate how knowledge is shared in the modern environment. With the current technology, if you have an effort to learn new things, you can do by yourself what most people deem impossible.

I also fixed some errors that I came across. I have not yet had plenty of time and energy to reread the whole thing and overhaul the editing. Just formatting the book this time takes days. So, if you want a perfect book, please help me spot defects.

\subsection*{Notes on 3.0 edition}

The reasons for updating the book are partly like the previous release. I have just finished \textsc{P\=ali\,Platform} 3, which now incorporates a number of P\=ali text corpora. In this version, the old collection of CSCD is now removed, and \emph{CST Restructured}\footnote{See its web site at \url{bhaddacak.github.io/cst-kit}.} is preferably used. That means all statistical information used in this book has to be updated.

Another big change is on text referencing. When the structure of a text collection is changed, the way we refer to its elements is inevitably changed. In the new collection, I also utilize new abbreviations. So, we should make the book conform with the naming system of the corpus.

Another reason, Antonio Costanzo who lives in Iceland has invested time and efforts in correcting the whole book. Thanks to Costanzo, now the book has less errors. Still, I am likely to make new errors when editing the book, and many errors are yet to be detected. Correcting process is ever ongoing.

Lastly, as some people asked me for translating my books to another language, I therefore change the book's license to the \emph{GNU Free Documentation License}. This means now the book is fully open-source.\footnote{In the old paradigm of book publication, open-sourcing a book sounds a little unusual. This means anyone can produce the book with the same quality and distribute it. That will be the way to go in the future for non-profitable electronic books like this one. I myself hardly read printed books nowadays. However, it is not really everyone can do it. To deal with \LaTeX\ source files, you must have some programming skill.} Anyone can access its source code and make changes on it (with the conditions described in the license). Even though I do not encourage translation (learning better English and reading it in English is a lot better because now we have many good resources and facilities in English), I see open-sourcing the book is worthwhile. When a book is not author-locked-in, it will have life of its own and everlastingly grow.

\subsection*{Acknowledgment}

Now, many people have seen and read the book. With my agile model\footnote{The agile model of development has this simple process: release quickly and frequently, listen to feedback, and improve the work along the way. This means, in our case, there is no final, perfect version of the book. It is an ever-improving product.} of authoring, there are still many things to be done. Translations need to be improved, errors need to be fixed, typos need to be detected and corrected, missing materials need to be added, ideas have to be polished, etc. Readers can help me by taking notes when you read the book, and send me back. You will be a part of this intellectual heritage.

Apart from Antonio Costanzo mentioned above who has valuable contribution to the book editing, I would like to thank those who remarked the work and gave me error lists. Most of them are unidentifiable, though. Some recognizable ones are listed as follows: Waiyin Chow, Gaoyang Li.
