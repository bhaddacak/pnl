\chapter{I \headhl{have you give} me a book}\label{chap:caus}

\phantomsection
\addcontentsline{toc}{section}{Introduction to Causatives}
\section*{Introduction to Causatives}

Now we will learn a little complicated form of verbs. It is used when someone makes another one do something. We call this kind of structure \emph{causative}. For better understanding, you are supposed to master Chapter \ref{chap:pass} before coming to this. In English we construct the causative by using certain verbs to mark this condition plus a target verb in infinitive form (with or without `to' depending on the main verb). Here are some common uses in English:

\begin{quote}
- A teacher \textbf{has} a student read a book.\\
- A teacher \textbf{makes} a student read a book.\\
- A teacher \textbf{gets} a student \textbf{to} read a book.\\
- A teacher \textbf{causes} a student \textbf{to} read a book.\\
\end{quote}

Even all these sentences have slightly different implication, they go in the same pattern. It is quite easy. You just remember when `to' is used. In P\=ali the task is more complicate than that because different verb forms have to be used. That is to say, in P\=ali we do not need helping verbs like English, but we instead change the target verb into causative form. This is the main subject of this chapter. Before we go to that, let us talk about object first.

In basic sentence, we use SVO (subject-verb-object) form, or SOV (subject-object-verb) form in typical P\=ali. To mark an object in a sentence we normally use accusative case, albeit other cases can be used as well, particularly genitive and instrumental case.\footnote{In ``\pali{sunakhehipi kh\=ad\=apenti}'' [M1\,169 (MN\,13)] ([The king] makes dogs eat [him]), instrumental case is used as object. For genitive object, it is more frequent to be found.} In a simple active sentence with a transitive verb, normally there is one object, for example, ``\pali{sisso potthaka\d m pa\d thati}'' (A student reads a book). Some verbs can take two objects.\footnote{As described in Sadd-Dh\=a Ch.\,19, these roots sometimes take two objects: \pali{duha, kara, vaha, puccha, y\=aca, bhikkha, ni, br\=u, bha\d na, vada, vaca, bh\=asa, s\=asa, daha, n\=atha, rudha, ji,} and \pali{ci}, for instance. See \pali{Duhikaravahipucchi} onwards.} Here are some examples of them:\footnote{All are from Sadd-Dh\=a Ch.\,19.}

\begin{quote}
\pali{G\=avi\d m kh\=ira\d m duhati gop\=alad\=arako.}\\
``A cowherd boy milks a cow [for] milk.''\\[1.5mm]
\pali{suva\d n\d na\d m ka\d taka\d m karoti.}\\
``[One] makes gold into a bracelet.''\\[1.5mm]
\pali{r\=ajapuris\=a ratha\d m g\=ama\d m vahanti.}\\
``King's men lead a cart to a village.''\\[1.5mm]
\pali{Aya\d m r\=aj\=a ma\d m n\=ama\d m pucchati.}\\
``This king asks me the name.''\\[1.5mm]
\pali{T\=apaso kula\d m bhojana\d m bhikkhati.}\\
``A hermit asks a family [for] food.''\\[1.5mm]
\pali{Aja\d m g\=ama\d m neti.}\\
``[One] leads a goat to a village.''\\[1.5mm]
\pali{Bhikkhu mah\=ar\=aj\=ana\d m dhamma\d m bha\d nati.}\\
``A monk talks the Dhamma to a great king.''\\[1.5mm]
\end{quote}

When more than one terms take accusative case simultaneously, there is a thing to be concerned. When composed carelessly, a sentence can be ambiguous. For example, ``\pali{aja\d m d\=araka\d m neti}'' can mean one leads a goat to a child, leads a child to a goat, or leads both to somewhere else.\footnote{Maintain a proper order of words can be a treatment of this, but in principle there is no guarantee. Encouraging a good style of writing can be a viable solution.}

Now let us try out a causative sentence. Basically, this structure has two objects. A P\=ali equivalent of the English examples above can be written as below. Please note on the verb form.

\palisample{\=acariyo sissa\d m potthaka\d m p\=a\d th\=apeti.}

It is possible that when used in causative structure, some verbs take more than two objects, for example:

\begin{quote}
\pali{Issaro gop\=ala\d m gava\d m payo duh\=apeti.}\footnote{Sadd-Dh\=a Ch.\,19. In this instance, Aggava\d msa tells us that \pali{payo} is in acc.}\\
``A master has a cowherd milk a cow [for] milk.''\\[1.5mm]
\pali{suva\d n\d na\d m ka\d taka\d m poso, k\=areti purisa\d m}\footnote{Sadd-Pad Ch.\,1}\\
``A person makes [another] person make gold into a bracelet.''\\[1.5mm]
\pali{puriso purise g\=ama\d m, ratha\d m v\=aheti}\footnote{Sadd-Pad Ch.\,1}\\
``A person make people lead a cart to a village.''\\[1.5mm]
\end{quote}

There are four \pali{paccaya}s that can mark a verb as causative: \palibf{\d ne}, \palibf{\d naya}, \palibf{\d n\=ape}, and \palibf{\d n\=apaya}. To learn how these work see page \pageref{pacca:dne2}. It is crucial to know that before we go on. If you have not read it yet, do it now.

So, you understood how \pali{p\=a\d th\=apeti} (pa\d tha + \d n\=ape + ti) comes.

Now we are ready to do our heading task. Thus we can say ``I have you give me a book'' in P\=ali as follows:

\palisample{aha\d m tva\d m (mayha\d m) potthaka\d m d\=apeti.}

If we leave out \pali{mayha\d m}, it can mean that you give the book to someone else.

Now let us consider intransitive verbs. When verbs requires no object, in causative structure you just drop one object. Thus, only one remains. For example, ``A teacher makes a student stand'' can be rendered as follows:

\palisample{\=acariyo sissa\d m \d th\=apeti/ti\d t\d theti.}

Finding this verb used in the canon, even in the whole P\=ali collection, is difficult. So, I guess these forms are probable. I found another verb, \pali{mara} (to die), which is used in this structure, but it comes from a commentary.

\begin{quote}
\pali{na, bhikkhave, so ime sattadivase s\=ukare m\=areti}\footnote{Dhp-a\,1.15}\\
``Monks, he does not make pigs die in these seven days [= he does not kill pigs].''\\
\end{quote}

Causative in passive structure is extremely rare in the canon. We call this \emph{casual passive}. Let us try to compose one from ``A teacher makes a student read a book.'' First, converting this sentence into passive voice, I get this one: ``A book is read by a student who is ordered by a teacher.'' Even though this sounds a bit odd in English, it is natural to say this in P\=ali because there is a particular structure for this. If you understand P\=ali passive structure well, you can guess this has something to do with \pali{ya}. That is right. To translate this into P\=ali, first you have to apply \pali{ya} to the verb (with \pali{i} or \pali{\=i} in most cases) after \pali{\d ne}, etc. Then you change the case of nouns involved accordingly. I use \pali{potthaka} as m.\ to make it clearer. Here is my result:

\palisample{\=acariyena sissa\d m potthako p\=a\d thapiyate.}

We use nominative case for `a book' because it is the patient, and this is the subject of the sentence. Instrumental case is used for `by a teacher.' And accusative case is used for `a student' because it is seen as the object of the teacher's order. If the focus of this sentence is shifted to student, hence ``A student is ordered by a teacher to read a book.'' The cases used now are different, but the verb stays the same. Thus we get this:

\palisample{\=acariyena sisso potthaka\d m p\=a\d thapiyate.\footnote{Amusingly, Antonio Costanzo told me he understood this sentence as ``A student is `bookly' being read by the teacher.'' In this very sense, the sentence should be \pali{\=acariyena sisso potthaka\d m pa\d thiyate}.}}

Now `a book' becomes acc.\ and `a student' becomes nom. You can see how effective this structure is. For intransitive verbs, like the pig example above, it can be done likewise. Hence we get this:

\palisample{tena s\=ukar\=a m\=ar\=apiyante.}

This is read ``Pigs are made die by him.'' An important lesson here is when a verb is used in passive form, cases of nouns related to this verb have to be composed accordingly. This is quite a little confusing for new students. Fortunately, we can say it is quite safe if you do not master this, because the structure itself is rarely used in the texts, and you do not need to give yourself a headache by saying in a difficult manner. Rephrasing passive sentences to active structure is the best practice of all time. If you insist to play difficult postures for better score, using verbal \pali{kita} may help (see below).

Before we move to another topic, I would like to remind the learners that verbs in causative form are not supposed to take two objects---to make someone do something. In some uses, a causative verb may be required when an intransitive verb is changed to transitive one, or the active and passive role of a verb is reversed. Here are some examples:

\begin{quote}
\pali{Yasmi\d m kho pana, bhikkhave, padese cakkaratana\d m \textbf{pati\d t\d th\=ati} tattha r\=aj\=a cakkavatt\=i v\=asa\d m upeti}\footnote{M3\,256 (MN\,129)}\\
``Monks, in which place the Jewel Wheel stands firmly, in that place the universal monarch obtains habitation.''\\[1.5mm]
\pali{na sama\d nabr\=ahma\d nesu uddhaggika\d m dakkhi\d na\d m \\\textbf{pati\d t\d th\=apeti}}\footnote{S1\,130 (SN\,3)}\\
``[A fool] does not establish offering [for future benefit] in ascetics and priests.''\\[1.5mm]
\end{quote}

In the first example above, \pali{pati\d t\d th\=ati} is used as an intransitive verb meaning `to stand firmly' or `to be established.' In the second example, \pali{pati\d t\d th\=apeti}, a causative form, now is a transitive verb meaning `to establish something' or `to make something stand firmly.' Let us see another pair:

\begin{quote}
\pali{antalikkh\=a dha\~n\~nassa dh\=ar\=a opatitv\=a dha\~n\~n\=ag\=ara\d m \textbf{p\=ureti}}\footnote{Mv\,6.296}\\
``A stream of grains, having fallen from the sky, fills the granary.''\\[1.5mm]
\pali{B\=alo \textbf{p\=urati} p\=apassa, thoka\d m thokampi \=acina\d m}\footnote{Dhp\,9.121}\\
``A fool is full of evil, litle by little collectively.''\\[1.5mm]
\end{quote}

In the first sentence, \pali{p\=ureti} is causative meaning `to make full' or `to fill,' whereas in the second, \pali{p\=urati} means `to be full' or 'to be filled.' The former has active meaning, the latter passive. Other pairs of verbs that works in the same way are, for example, `to grow' = \pali{va\d d\d dhati} (v.i.)/\pali{va\d d\d dheti} (v.t.), `to rise/to raise' = \pali{u\d t\d thahati} (v.i.)/\pali{u\d t\d th\=apeti} (v.t.). And some active/passive pairs are `to learn/to teach' = \pali{ugga\d nh\=ati}/\pali{ugga\d nh\=apeti} or \pali{sikkhati}/ \pali{sikkh\=apeti}; `to know/to inform' = \pali{paj\=an\=ati}/\pali{pa\~n\~n\=apeti}; `to be lost/to destroy' = \pali{(vi)nassati}/\pali{(vi)n\=aseti}. You can find some more by yourselves along the way of your study.

\phantomsection
\addcontentsline{toc}{section}{Using \pali{Kita} in Causatives}
\section*{Using \pali{Kita} in Causatives}

Some \pali{kita} forms are useful in creating causative structure. For active causatives, we can use \pali{anta} and \pali{m\=ana} in present meaning, and \pali{ta} (also \pali{tavantu} and \pali{t\=av\=i}) in past meaning. For causal passive structure, we can use \pali{m\=ana} in present meaning (not \pali{anta}), \pali{an\=iya} and \pali{tabba} in imperative or optative meaning, and \pali{ta} in past meaning. Verbs in \pali{tv\=a} form can be used in all structures. Here is a brief guideline.

\begin{enumerate}
\item Choose a verb to use. Aware of its root or stem.
\item Apply \pali{\d ne, \d naya, \d n\=ape,} or \pali{\d n\=apaya} to the verb. This marks it as causative.
\item For passive voice, apply \pali{ya} preceded with \pali{i} or \pali{\=i} after the causative marker.
\item Apply other \pali{paccaya} corresponding to the function intended.
\end{enumerate}

Not every form of verbs described above can be easily found in the texts. Here are some examples from the canon:

\begin{quote}
\pali{aha\d m kho imasmi\d m vanasa\d n\d de kammanta\d m \textbf{k\=ar\=apento} ram\=ami.}\footnote{S1\,203 (SN\,7)}\\
``I enjoys myself having [people] work in this jungle.''\\[1.5mm]
\pali{Mahallaka\d m vih\=ara\d m \textbf{k\=ar\=apento} tisso \=apattiyo \=apajjati.}\footnote{Pvr\,161}\\
``Having [someone] make a big building, [a monk] gets into three offenses.''\\[1.5mm]
\pali{na ekaccassa kes\=a chedetabb\=a, na ekaccena kes\=a \\\textbf{ched\=apetabb\=a}}\footnote{Mv\,1.66}\\
``The hair of someone should not be cut [by the monk in penance]. The hair [of the monk] should not be cut by someone.''\\[1.5mm]
\pali{Pa\d thama\d m upajjha\d m \textbf{g\=ah\=apetabbo}. \\Upajjha\d m \textbf{g\=ah\=apetv\=a} pattac\=ivara\d m \=acikkhitabba\d m}\footnote{Mv\,1 126}\\
``First, a preceptor shall be taken [by the ordination candidate]. [After] the preceptor has been taken, about robe and bowl shall be informed.''\\[1.5mm]
\pali{Pa\d thama\d m khetta\d m \textbf{kas\=apetabba\d m}. \textbf{Kas\=apetv\=a \\vap\=apetabba\d m}. \textbf{Vap\=apetv\=a} udaka\d m \textbf{abhinetabba\d m}.}\footnote{Cv\,7.330}\\
``First, the field has to be ploughed. Having made [the field] ploughed, [paddy] has to be sowed. Having made [the paddy] sowed, water has to be brought in.''\\[1.5mm]
\pali{karoto kho, mah\=ar\=aja, \textbf{k\=arayato}, chindato \textbf{ched\=apayato}, pacato \textbf{p\=ac\=apayato}, socayato \textbf{soc\=apayato}, kilamato \textbf{kilam\=apayato}, phandato \textbf{phand\=apayato} \ldots}\footnote{D\=i 1.2.166 (DN\,2)}\\
``Your Majesty, [a person], having done [or] having made [someone] do, having cut [or] having made [someone] cut, having boiled [or] having made [someone] boil, having lamented [or] having made [someone] lament, having made oneself in trouble [or] having made [other] in trouble, having trembled [or] having made [someone] trembled, \ldots''\\[1.5mm]
\pali{Tena kho pana samayena bhagav\=a s\=ayanhasamaya\d m pa\d tisall\=an\=a vu\d t\d thito pacch\=atape nisinno hoti pi\d t\d thi\d m \textbf{ot\=apayam\=ano}.}\footnote{S5\,511 (SN\,48)}\\
``By that occasion in one evening, there is the Buddha, having emerged from seclusion, having sat down making [his] back exposed to the heat of the sun.''\\[1.5mm]
\end{quote}

Now let us try to do it by ourselves. If our heading is rephrased to ``There is I who have you give me a book,'' we can put it like this (suppose the speaker is male):

\palisample{aha\d m tva\d m mayha\d m potthaka\d m d\=apayanto homi.}

If you compare this sentence to that one we get earlier, you can see their similarity in structure. To make it valid, I just add verb `to be' to complete the sentence. Alternatively, \pali{d\=apayam\=ano} can also do the job. In past tense, you can do likewise. And here is for ``There is I who have you gave me a book'' using \pali{ta}:

\palisample{aha\d m tva\d m mayha\d m potthaka\d m d\=apeto homi.}

According to the principle, we do not need `to be' here. So, the sentence can be more straightforward, hence ``I had you give me a book.'' Here is its P\=ali:

\palisample{aha\d m tva\d m mayha\d m potthaka\d m d\=apeto.}

To get the benefit of using \pali{ta}, it is more suitable, or fashionable, to be constructed in passive voice. Then we get ``A book was given to me by you [who was ordered].''

\palisample{potthako tay\=a mayha\d m d\=apeto.}

In present tense, we can use \pali{m\=ana} in passive structure. Then the sentence becomes ``There is a book that is being given to me by you [who is ordered].''

\palisample{potthako tay\=a mayha\d m d\=apayam\=ano hoti.}

In passive imperative sense, we can use \pali{tabba} (or rarely \pali{an\=iya}). In this case, the sentence becomes ``A book has to be made given to me by you.''

\palisample{potthako tay\=a mayha\d m d\=apetabbo.}

Let us try one with \pali{tv\=a}. Saying ``Having made given (to me by you), I read a book'' can be as follows:

\palisample{aha\d m (tay\=a mayha\d m) potthaka\d m d\=apetv\=a pa\d th\=ami.}

And for its passive equivalent ``Having made given, a book is read by me.''

\palisample{may\=a potthako d\=apetv\=a pa\d thayati.}

You can play around more on this by yourselves to get better understanding. Things might look complicated. But you can master them by gradually adding up components and shuffling things around. Do not leave out a single thing you do not understand. Once you are familiar with its nature, learning P\=ali can bring a lot of fun.

\enlargethispage{3\baselineskip}
\section*{Exercise \ref{chap:caus}}
Say these in P\=ali. They are challenging, even for me, but worth pondering upon.\footnote{I spent about six hours for writing and translating this short dialogue. Do not take the content seriously. Try to grasp how to deal with difficult terminology and structure. My solution is by no means the best. You may come up with better ones.}
\begin{compactenum}
\item I made myself confused by thinking that if there is no one's true self, what does transmigrate?
\item This problem is very old and perennial. It existed even in the Buddha's lifetime. The argument on the issue continues to these days.
\item Maybe it is just a poor reasoning, I think.
\item Although academic discussions of the issue may happen, for Buddhists there is no such a problem. People accept what is told without thinking about it.
\item What does make one's identity persist over time then?
\item Many explanations are given so far. If you want to know, read those books. But I think it is not a metaphysical problem that needs deliberate explanation. It is true even in modern psychology that our ego makes us suffer. I think this is the very point the Buddha tried to say.
\item I see. The belief makes the Order survive and make the government runs smoothly. It has social function, while the doctrine of no-self has psychological function for individuals.
\end{compactenum}
