\chapter{Declensional Paradigms}\label{chap:decl}

In the traditional way of learning, remembering nomical declension from examples or paradigms is at the heart of the method. It helps us see the final forms of terms quickly. However, I do not quite follow the method in our lessons, particularly those concerning nouns. That is the reason I add them all here as an appendix for the sake of referencing. Another reason is to make those who are familiar with traditional method feel comfortable. But I leave out the explanations how each form comes to be in shape. To me most parts of that are not explanation, they are just descriptions. They answer the question `how' not `why.' So, it is better to see a lot of typical examples, rather than to figure out why or how rules work.

In traditional view, the whole business of learning declension is to know about \pali{vibhatti} `classification' (see also Chapter \ref{chap:ind-intro}). Nominal \pali{vibhatti} has 14 instances, namely \pali{si yo a\d m yo n\=a hi sa na\d m sm\=a hi sa na\d m smi\d m su}.\footnote{Kacc\,55, R\=upa\,63, Sadd\,200, Mogg\,2.1, Niru\,61}. In these 7 pairs, the first part is singular, the second plural.\footnote{Sadd\,201} So we get 7 cases respectively. But \pali{si yo} can also perform addressing function. The eighth pair is then added.\footnote{Sadd\,709} We call this last one \pali{\=alapana}. This explains why vocative and nominative forms look similar in most cases. I summarize all nominal \pali{vibhatti} in Table \ref{tab:nomvibhatti}.

\begin{table}[!hbt]
\centering
\caption{Nominal \pali{vibhatti}}
\label{tab:nomvibhatti}
\bigskip
\begin{tabular}{l>{\itshape}ll*{2}{>{\itshape}l}} \toprule
\multicolumn{3}{c}{\bfseries Case} & \upshape\bfseries Singular & \upshape\bfseries Plural  \\
\midrule
1. & pa\d tham\=a & nominative & si & yo \\
2. & dutiy\=a & accusative & a\d m & yo \\
3. & tatiy\=a & instrumental & n\=a & hi \\
4. & catutth\=i & dative & sa & na\d m \\
5. & pa\~ncam\=i & ablative & sm\=a & hi \\
6. & cha\d t\d th\=i & genitive & sa & na\d m \\
7. & sattam\=i & locative & smi\d m & su \\
\=a. & \=alapana & vocative & si & yo \\
\bottomrule
\end{tabular}
\end{table}
What are these \pali{vibhatti}s after all? They look like forms of word ending. In a way, yes, they are. But this way of explaining is quite misleading, for you will never find some forms of them, for example, \pali{si} or \pali{yo}. It is better to see all of these as names of distinct word processing methods. Each has things to do with raw words, or \pali{sadda} as I explain in Chapter \ref{chap:ind-intro}.\footnote{The tradition calls raw words \pali{li\.nga} (Sadd\,192, 196--7). But for the sake of clarity, I will not follow this terminology.} Raw words are those term that do not get any meaning yet, because they are not composed in a sentence. We can find raw words, especially nouns and adjectives, in a dictionary. Sometimes I call these dictionary form of words.

To see a clearer picture, let us look at an example. I have a raw word, say, \pali{jana} (m., = person). When I want to use it in singular nominative case, it has to be processed with \pali{si}. In textbooks, there are procedural explanations of this, but I skip all of them. Let us take it simply as follows. The \pali{si} process determines whether the term is irregular or not. If it is irregular, it use irregular tables. If not, it looks for the gender of the term and its ending, then follows the regular paradigms. Since \pali{jana} is a regular masculine word, it becomes \pali{jano} (a person) in nominative singular. Likewise, nominative plural uses \pali{yo} process. As a result, we get \pali{jan\=a} (people). By this explanation, it is not necessary, believe me, to know why or how \pali{si} produces \pali{jano}\footnote{Kacc\,104, R\=upa\,66, Sadd\,272, Mogg\,2.109} or \pali{yo} produces \pali{jan\=a}\footnote{Kacc\,107, R\=upa\,69, Sadd\,275, Mogg\,2.41}. You just follow the provided paradigms. In most cases, one \pali{vibhatti} can produce more than one form, so you can see several of them sometimes.

The major part of nominal paradigms is taken from Padar\=upasiddhi (R\=upa) with some adaptation and addition from other textbooks, particularly Saddan\=iti Padam\=al\=a. The list has a good coverage, but some peculiar, trivial terms are left out.  To save the space and make tables less dense, I leave out some repetitions as described below.
\begin{compactenum}[(1)]
\item In m.\ and nt.\ sg.\ of abl.\ \pali{-mh\=a} ending is omitted, for it can replace \pali{-sm\=a} ending in every place.
\item In m.\ and nt.\ sg.\ of loc.\ \pali{-mhi} ending is omitted, for it can replace \pali{-smi\d m} ending in every place.
\item In pl.\ of ins.\ and abl.\ \pali{-bhi} ending is omitted, for it can replace \pali{-hi} ending in every place.
\item In voc.\ generic addressing words are omitted, namely \pali{bho}, \pali{bhavanto}, \pali{bhoti}, \pali{bhotiyo}, and \pali{he}.\footnote{Padar\=upasiddhi, following Kacc\=ayana, exemplifies voc.\ in double form, e.g.\ \pali{bho purisa, bhavanto puris\=a, bhoti ka\~n\~ne, bhotiyo kan\~n\~n\=a, he nama, he man\=a}. See, for example, R\=upa\,74. Aggava\d masa discusses this issue in Sadd-Pad Ch.\,5.}
\end{compactenum}
I also reorder and rearrange the lists to make them easier to follow. Moreover, I make some words highlighted with color to remind us to pay more attention on them. Normally these words are worth remembering.

\raggedright
\section{Regular Masculine Nouns}

\begin{decltable}{Paradigm of regular m.\ \pali{a} [\pali{purisa}]}
1. nom. & \texthl{puriso}\footnote{In rare cases, the ending of nom.\ sg.\ becomes \pali{e} instead of \pali{o}, also instead \pali{a\d m} in nt.\ nouns. This is said to be Magadhism (see \citealp[p.~73]{geiger:grammar}), for example, ``\pali{b\=ale ca pa\d n\d dite ca}'' [D1\,168 (DN\,2)] (the fool and the wise man).} & \texthl{puris\=a} \\
2. acc. & purisa\d m & \texthl{purise} \\
3. ins. & purisena & purisehi \\
4. dat. & purisassa, puris\=aya, purisattha\d m & puris\=ana\d m \\
5. abl. & purisasm\=a, \texthl{puris\=a} & purisehi \\
6. gen. & purisassa & puris\=ana\d m \\
7. loc. & purisasmi\d m, \texthl{purise} & purisesu \\
\=a. voc. & purisa, puris\=a & puris\=a \\
\end{decltable}

\begin{decltable}{Paradigm of regular m.\ \pali{i} [\pali{aggi}]}
1. nom. & aggi, \texthl{aggini} & agg\=i, aggayo \\
2. acc. & aggi\d m & agg\=i, aggayo \\
3. ins. & \texthl{aggin\=a} & agg\=ihi, aggihi \\
4. dat. & aggissa, \texthl{aggino} & agg\=ina\d m, aggina\d m \\
5. abl. & aggism\=a, \texthl{aggin\=a} & agg\=ihi, aggihi \\
6. gen. & aggissa, \texthl{aggino} & agg\=ina\d m, aggina\d m \\
7. loc. & aggismi\d m & agg\=isu, aggisu \\
\=a. voc. & aggi & agg\=i, aggayo \\
\end{decltable}

\begin{decltable}{Paradigm of regular m.\ \pali{\=i} [\pali{da\d n\d d\=i}]}
1. nom. & da\d n\d d\=i & da\d n\d d\=i, \texthl{da\d n\d dino} \\
2. acc. & da\d n\d di\d m, \texthl{da\d n\d dina\d m} & da\d n\d d\=i, \texthl{da\d n\d dino} \\
3. ins. & \texthl{da\d n\d din\=a} & da\d n\d d\=ihi \\
4. dat. & da\d n\d dissa, \texthl{da\d n\d dino} & da\d n\d d\=ina\d m \\
5. abl. & da\d n\d dism\=a, \texthl{da\d n\d din\=a} & da\d n\d d\=ihi \\
6. gen. & da\d n\d dissa, \texthl{da\d n\d dino} & da\d n\d d\=ina\d m \\
7. loc. & da\d n\d dismi\d m, \texthl{da\d n\d dini} & da\d n\d d\=isu \\
\=a. voc. & da\d n\d di & da\d n\d d\=i, \texthl{da\d n\d dino} \\
\end{decltable}

\begin{decltable}{Paradigm of regular m.\ \pali{u} [\pali{bhikkhu}]}
1. nom. & bhikkhu & bhikkh\=u, bhikkhavo \\
2. acc. & bhikkhu\d m & bhikkh\=u, bhikkhavo \\
3. ins. & \texthl{bhikkhun\=a} & bhikkh\=uhi, bhikkhuhi \\
4. dat. & bhikkhussa, \texthl{bhikkhuno} & bhikkh\=una\d m, bhikkhuna\d m \\
5. abl. & bhikkhusm\=a, \texthl{bhikkhun\=a} & bhikkh\=uhi, bhikkhuhi \\
6. gen. & bhikkhussa, \texthl{bhikkhuno} & bhikkh\=una\d m, bhikkhuna\d m \\
7. loc. & bhikkhusmi\d m & bhikkh\=usu, bhikkhusu \\
\=a. voc. & bhikkhu & bhikkh\=u, bhikkhavo, \texthl{bhikkhave} \\
\end{decltable}

\begin{listtableT}{Some slight variations}
hetu & in pl.\ nom.\ \& acc.\ also \pali{\texthl{hetuyo}} \\
jantu & in pl.\ nom.\ \& acc.\ also \pali{jantuyo}, \pali{\texthl{jantuno}} \\
\end{listtableT}

\begin{decltable}{Paradigm of regular m.\ \pali{\=u} [\pali{sabba\~n\~n\=u}]}
1. nom. & sabba\~n\~n\=u & sabba\~n\~n\=u, \texthl{sabba\~n\~nuno} \\
2. acc. & sabba\~n\~nu\d m & sabba\~n\~n\=u, \texthl{sabba\~n\~nuno} \\
3. ins. & sabba\~n\~nun\=a & sabba\~n\~n\=uhi \\
4. dat. & sabba\~n\~nussa, sabba\~n\~nuno & sabba\~n\~n\=una\d m \\
5. abl. & sabba\~n\~nusm\=a, sabba\~n\~nun\=a & sabba\~n\~n\=uhi \\
6. gen. & sabba\~n\~nussa, sabba\~n\~nuno & sabba\~n\~n\=una\d m \\
7. loc. & sabba\~n\~nusmi\d m & sabba\~n\~n\=usu \\
\=a. voc. & sabba\~n\~nu & sabba\~n\~n\=u, \texthl{sabba\~n\~nuno} \\
\end{decltable}

\begin{listtableT}{Some slight variations}
abhibh\=u & \rdelim{\}}{4}{\linewidth}[in pl.\ nom.\ \& acc.\ \& voc. as \pali{abhibh\=u, \texthl{abhibhuvo}}] \\
\mbox{par\=abhibh\=u} & \\
vessabh\=u & \\
sayambh\=u & \\
sahabh\=u & as above plus \pali{\texthl{sahabhuno}} \\
\end{listtableT}

\clearpage
\section{Regular Feminine Nouns}

\begin{decltable}{Paradigm of regular f.\ \pali{\=a} [\pali{ka\~n\~n\=a}]}
1. nom. & ka\~n\~n\=a & ka\~n\~n\=a, \texthl{ka\~n\~n\=ayo} \\
2. acc. & ka\~n\~na\d m & ka\~n\~n\=a, \texthl{ka\~n\~n\=ayo} \\
3. ins. & ka\~n\~n\=aya & ka\~n\~n\=ahi \\
4. dat. & ka\~n\~n\=aya & ka\~n\~n\=ana\d m \\
5. abl. & ka\~n\~n\=aya & ka\~n\~n\=ahi \\
6. gen. & ka\~n\~n\=aya & ka\~n\~n\=ana\d m \\
7. loc. & ka\~n\~n\=aya, \texthl{ka\~n\~n\=aya\d m} & ka\~n\~n\=asu \\
\=a. voc. & \texthl{ka\~n\~ne} & ka\~n\~n\=a, \texthl{ka\~n\~n\=ayo} \\
\end{decltable}

\begin{decltable}{Paradigm of regular f.\ \pali{i} [\pali{ratti}]}
1. nom. & ratti & ratt\=i, rattiyo \\
2. acc. & ratti\d m & ratt\=i, rattiyo \\
3. ins. & rattiy\=a & ratt\=ihi, rattihi \\
4. dat. & rattiy\=a & ratt\=ina\d m, rattina\d m \\
5. abl. & rattiy\=a & ratt\=ihi, rattihi \\
6. gen. & rattiy\=a & ratt\=ina\d m, rattina\d m \\
7. loc. & rattiy\=a, rattiya\d m & ratt\=isu, rattisu \\
\=a. voc. & ratti & ratt\=i, rattiyo \\
\end{decltable}

\begin{listtableT}{Some slight variations}
ratti & in pl.\ nom.\ also \pali{ratyo}, in sg.\ abl.\ also \pali{raty\=a}, and in sg.\ loc.\ also \pali{raty\=a, ratya\d m, ratti\d m, ratto} \\
\end{listtableT}

\begin{decltable}{Paradigm of regular f.\ \pali{\=i} [\pali{itth\=i}]}
1. nom. & itth\=i & itth\=i, itthiyo \\
2. acc. & itth\=i\d m & itth\=i, itthiyo \\
3. ins. & itthiy\=a & itth\=ihi \\
4. dat. & itthiy\=a & itth\=ina\d m \\
5. abl. & itthiy\=a & itth\=ihi \\
6. gen. & itthiy\=a & itth\=ina\d m \\
7. loc. & itthiy\=a, itthiya\d m & itth\=isu \\
\=a. voc. & itthi & itth\=i, itthiyo \\
\end{decltable}

\begin{decltable}{Paradigm of regular f.\ \pali{u} [\pali{y\=agu}]}
1. nom. & y\=agu & y\=ag\=u, y\=aguyo \\
2. acc. & y\=agu\d m & y\=ag\=u, y\=aguyo \\
3. ins. & y\=aguy\=a & y\=ag\=uhi, y\=aguhi \\
4. dat. & y\=aguy\=a & y\=ag\=una\d m, y\=aguna\d m \\
5. abl. & y\=aguy\=a & y\=ag\=uhi, y\=aguhi \\
6. gen. & y\=aguy\=a & y\=ag\=una\d m, y\=aguna\d m \\
7. loc. & y\=aguy\=a, y\=aguya\d m & y\=ag\=usu, y\=agusu \\
\=a. voc. & y\=agu & y\=ag\=u, y\=aguyo \\
\end{decltable}

\newpage
\begin{decltable}{Paradigm of regular f.\ \pali{\=u} [\pali{jamb\=u}]}
1. nom. & jamb\=u & jamb\=u, jambuyo \\
2. acc. & jambu\d m & jamb\=u, jambuyo \\
3. ins. & jambuy\=a & jamb\=uhi \\
4. dat. & jambuy\=a & jamb\=una\d m \\
5. abl. & jambuy\=a & jamb\=uhi \\
6. gen. & jambuy\=a & jamb\=una\d m \\
7. loc. & jambuy\=a, jambuya\d m & jamb\=usu \\
\=a. voc. & jambu & jamb\=u, jambuyo \\
\end{decltable}

\section{Regular Neuter Nouns}

\begin{decltable}{Paradigm of regular nt.\ \pali{a} [\pali{citta}]}
1. nom. & \texthl{citta\d m} & \texthl{citt\=ani}, citt\=a \\
2. acc. & citta\d m & \texthl{citte}, \texthl{citt\=ani} \\
3. ins. & cittena & cittehi \\
4. dat. & cittassa & citt\=ana\d m \\
5. abl. & cittasm\=a, \texthl{citt\=a} & cittehi \\
6. gen. & cittassa & citt\=ana\d m \\
7. loc. & cittasmi\d m, \texthl{citte} & cittesu \\
\=a. voc. & citta & \texthl{citt\=ani}, citt\=a \\
\end{decltable}

\begin{decltable}{Paradigm of regular nt.\ \pali{i} [\pali{a\d t\d thi}]}
1. nom. & a\d t\d thi & a\d t\d th\=i, a\d t\d th\=ini \\
2. acc. & a\d t\d thi\d m & a\d t\d th\=i, a\d t\d th\=ini \\
3. ins. & a\d t\d thin\=a & a\d t\d th\=ihi, a\d t\d thihi \\
4. dat. & a\d t\d thissa, a\d t\d thino & a\d t\d th\=ina\d m, a\d t\d thina\d m \\
5. abl. & a\d t\d thism\=a, a\d t\d thin\=a & a\d t\d th\=ihi, a\d t\d thihi \\
6. gen. & a\d t\d thissa, a\d t\d thino & a\d t\d th\=ina\d m, a\d t\d thina\d m \\
7. loc. & a\d t\d thismi\d m & a\d t\d th\=isu, a\d t\d thisu \\
\=a. voc. & a\d t\d thi & a\d t\d th\=i, a\d t\d th\=ini \\
\end{decltable}

\begin{decltable}{Paradigm of regular nt.\ \pali{u} [\pali{\=ayu}]}
1. nom. & \=ayu & \=ay\=u, \=ay\=uni \\
2. acc. & \=ayu\d m & \=ay\=u, \=ay\=uni \\
3. ins. & \=ayun\=a, \texthl{\=ayus\=a} & \=ay\=uhi, \=ayuhi \\
4. dat. & \=ayussa, \=ayuno & \=ay\=una\d m, \=ayuna\d m \\
5. abl. & \=ayusm\=a, \=ayun\=a, \texthl{\=ayus\=a} & \=ay\=uhi, \=ayuhi \\
6. gen. & \=ayussa, \=ayuno & \=ay\=una\d m, \=ayuna\d m \\
7. loc. & \=ayusmi\d m & \=ay\=usu, \=ayusu \\
\=a. voc. & \=ayu & \=ay\=u, \=ay\=uni \\
\end{decltable}

\clearpage
\section{Irregular Nouns}\label{decl:irrn}

Regarding the irregular nouns listed below, you can see further explanation in Chapter \ref{chap:irrn}.

\medskip
\begin{decltable}{Declension of m.\ \pali{mana}\label{decl:mana}}
1. nom. & mano & man\=a \\
2. acc. & mana\d m & mane \\
3. ins. & manena, \texthl{manas\=a} & manehi \\
4. dat. & manassa, \texthl{manaso} & man\=ana\d m \\
5. abl. & manasm\=a, man\=a & manehi \\
6. gen. & manassa, \texthl{manaso} & man\=ana\d m \\
7. loc. & manasmi\d m, mane, \texthl{manasi} & manesu \\
\=a. voc. & mana, man\=a & man\=a \\
\end{decltable}

\begin{decltable}{Declension of nt.\ \pali{mana}}
1. nom. & mana\d m & man\=ani, man\=a \\
2. acc. & mana\d m & man\=ani, mane \\
3. ins. & manena & manehi \\
4. dat. & manassa, manaso & man\=ana\d m \\
5. abl. & manasm\=a, man\=a & manehi \\
6. gen. & manassa, manaso & man\=ana\d m \\
7. loc. & manasmi\d m, mane, manasi & manesu \\
\=a. voc. & mana & man\=ani, man\=a \\
\end{decltable}

\begin{listtableE}{Words declining as \pali{mana}, only m.}
aya & aha & ura & ceta & chanda & tapa & tama & teja \\
mana & paya & yasa & raha & vaca & vaya & sara & sira \\
\end{listtableE}

\begin{decltable}{Declension of m.\ \pali{r\=aja}\label{decl:raaja}\footnote{Compounds ending with \pali{r\=aj\=a} can decline in both ways, like regular noun, e.g.\ \pali{mah\=ar\=ajo}, and like in this table, e.g.\ \pali{mah\=ar\=aj\=a}. See a detailed discussion in Sadd-Pad Ch.\,6.}}
1. nom. & \texthl{r\=aj\=a} & r\=aj\=ano \\
2. acc. & r\=aj\=ana\d m, r\=aja\d m & r\=aj\=ano \\
3. ins. & r\=ajena, \texthl{ra\~n\~n\=a} & r\=ajehi, \texthl{r\=aj\=uhi} \\ 
4. dat. & \texthl{r\=ajino}, \texthl{ra\~n\~no}, ra\~n\~nassa & r\=aj\=ana\d m, \texthl{r\=aj\=una\d m}, \texthl{ra\~n\~na\d m} \\
5. abl. & \texthl{ra\~n\~n\=a} & r\=ajehi, \texthl{r\=aj\=uhi} \\ 
6. gen. & \texthl{r\=ajino}, \texthl{ra\~n\~no}, ra\~n\~nassa & r\=aj\=ana\d m, \texthl{r\=aj\=una\d m}, \texthl{ra\~n\~na\d m} \\
7. loc. & \texthl{r\=ajini}, \texthl{ra\~n\~ne} & r\=ajesu, \texthl{r\=aj\=usu} \\
\=a. voc. & r\=aja, r\=aj\=a & r\=aj\=ano \\
\end{decltable}

\begin{decltable}{Declension of m.\ \pali{brahma}\label{decl:brahma}}
1. nom. & \texthl{brahm\=a} & brahm\=ano \\
2. acc. & brahm\=ana\d m, brahma\d m & brahm\=ano \\
3. ins. & brahmena, \texthl{brahmun\=a} & brahmehi \\ 
4. dat. & brahmassa, \texthl{brahmuno} & brahm\=ana\d m, brahm\=una\d m\\
5. abl. & \texthl{brahmun\=a} & brahmehi \\ 
6. gen. & brahmassa, \texthl{brahmuno} & brahm\=ana\d m, brahm\=una\d m\\
7. loc. & brahmani & brahmesu \\
\=a. voc. & \texthl{brahme} & brahm\=ano \\
\end{decltable}

\begin{decltable}{Declension of m.\ \pali{sakha}\label{decl:sakha}\footnote{Compounds ending with \pali{sakha} decline as regular nouns.}}
1. nom. & \texthl{sakh\=a} & sakh\=a, sakh\=ayo, sakh\=ano, sakh\=aro, sakhino \\
2. acc. & sakha\d m, sakh\=ana\d m, sakh\=ara\d m & sakhe, sakh\=ayo, sakh\=ano, sakh\=are, sakhino \\
3. ins. & sakhin\=a & sakhehi, sakh\=arehi \\
4. dat. & sakhissa, sakhino & sakh\=ina\d m, sakh\=ana\d m, sakh\=ar\=ana\d m \\
5. abl. & sakhin\=a, sakh\=arasm\=a, sakhism\=a, sakhasm\=a, sakh\=a, sakh\=ar\=a & sakhehi, sakh\=arehi \\
6. gen. & sakhissa, sakhino & sakh\=ina\d m, sakh\=ana\d m, sakh\=ar\=ana\d m \\
7. loc. & sakhe & sakhesu, sakh\=aresu \\
\=a. voc. & sakha, sakh\=a, sakhi, sakh\=i, sakhe & sakh\=a, sakh\=ayo, sakh\=ano, sakh\=aro, sakhino \\
\end{decltable}

\begin{decltable}{Declension of m.\ \pali{atta}\label{decl:atta}\footnote{Compounds ending with \pali{atta} decline as regular nouns.}\label{decl:atta}}
1. nom. & \texthl{att\=a} & att\=a, att\=ano  \\
2. acc. & att\=ana\d m, atta\d m & att\=ano \\
3. ins. & attena, \texthl{attan\=a} & attanehi, attehi \\ 
4. dat. & \texthl{attano}, attassa & att\=ana\d m \\
5. abl. & \texthl{attan\=a} & attanehi, attehi \\ 
6. gen. & \texthl{attano}, attassa & att\=ana\d m \\
7. loc. & attani & attesu \\
\=a. voc. & atta, att\=a & att\=ano \\
\end{decltable}

\newpage
\begin{decltable}{Declension of m.\ \pali{\=atuma}\label{decl:aatuma}}
1. nom. & \texthl{\=atum\=a} & \=atum\=a, \=atum\=ano  \\
2. acc. & \=atuma\d m, \=atum\=ana\d m  & \=atum\=ano \\
3. ins. & \=atumena & \=atumehi \\
4. dat. & \=atumassa, \=atum\=aya, \=atumattha\d m & \=atum\=ana\d m \\
5. abl. & \=atumasm\=a, \=atum\=a & \=atumehi \\
6. gen. & \=atumassa & \=atum\=ana\d m \\
7. loc. & \=atumasmi\d m, \=atume & \=atumesu \\
\=a. voc. & \=atuma, \=atum\=a & \=atum\=ano \\
\end{decltable}

\begin{decltable}{Declension of m.\ \pali{puma}\label{decl:puma}\footnote{\pali{Puma} can also decline as regular nouns.}}
1. nom. & \texthl{pum\=a} & pum\=a, pum\=ano \\
2. acc. & pum\=ana\d m & pum\=ane, pum\=ano \\
3. ins. & pumena, pum\=an\=a, pumun\=a & pum\=anehi \\
4. dat. & pumassa, pumuno & pum\=ana\d m \\
5. abl. & pum\=an\=a, pumun\=a & pum\=anehi \\
6. gen. & pumassa, pumuno & pum\=ana\d m \\
7. loc. & pum\=ane, pume & pum\=anesu, pum\=asu \\
\=a. voc. & puma, puma\d m & pum\=a, pum\=ano \\
\end{decltable}

\begin{decltable}{Declension of m.\ \pali{yuva}\label{decl:yuva}}
1. nom. & \texthl{yuv\=a}, yuv\=ano & yuv\=a, yuv\=ano, yuv\=an\=a \\
2. acc. & yuv\=ana\d m, yuva\d m & yuve, yuv\=ane \\
3. ins. & yuv\=an\=a, yuvena, yuv\=anena & yuv\=anehi, yuvehi \\
4. dat. & yuv\=anassa, yuvassa, yuvino & yuv\=an\=ana\d m, yuvana\d m \\
5. abl. & yuv\=anasm\=a, yuv\=an\=a & yuv\=anehi, yuvehi \\
6. gen. & yuv\=anassa, yuvassa, yuvino & yuv\=an\=ana\d m, yuvana\d m \\
7. loc. & yuv\=anasmi\d m, yuv\=ane, yuvasmi\d m, yuve & yuv\=anesu, yuv\=asu, yuvesu \\
\=a. voc. & yuva, yuv\=ana & yuv\=ano, yuv\=an\=a \\
\end{decltable}

\begin{listtableF}{Words declining as \pali{yuva} (m.)}
maghava\footnote{\pali{Maghava} or \pali{maghavantu} can also decline like \pali{gu\d navantu} (see below).} & & & & \\
\end{listtableF}

\newpage
\begin{decltable}{Declension of m.\ \pali{raha}\label{decl:raha}}
1. nom. & \texthl{rah\=a} & rah\=a, rahino \\
2. acc. & rah\=ana\d m & rah\=ane \\
3. ins. & rahin\=a & rahinehi \\
4. dat. & rahassa & rah\=ana\d m \\
5. abl. & rah\=a & rahinehi \\
6. gen. & rahassa & rah\=ana\d m \\
7. loc. & rah\=ane & rah\=anesu \\
\=a. voc. & raha & rah\=a, rahino \\
\end{decltable}

\begin{decltable}{Declension of m.\ \pali{vattaha}\label{decl:vattaha}}
1. nom. & \texthl{vattah\=a} & vattah\=ano \\
2. acc. & vattah\=ana\d m & vattah\=ane \\
3. ins. & vattah\=an\=a & vattah\=anehi \\
4. dat. & vattahino, vattah\=ano & vattah\=ana\d n, vattah\=an\=ana\d m \\
5. abl. & vattah\=an\=a & vattah\=anehi \\
6. gen. & vattahino, vattah\=ano & vattah\=ana\d n, vattah\=an\=ana\d m \\
7. loc. & vattah\=ane & vattah\=asu \\
\=a. voc. & vattaha & vattah\=ano \\
\end{decltable}

\begin{decltable}{Declension of m.\ \pali{vuttasira}\label{decl:vuttasira}}
1. nom. & \texthl{vuttasir\=a} & vuttasir\=a, vuttasir\=ano \\
2. acc. & vuttasir\=ana\d m & vuttasir\=ane \\
3. ins. & vuttasir\=an\=a & vuttasir\=anehi \\
4. dat. & vuttasirassa & vuttasir\=ana\d m \\
5. abl. & vuttasir\=an\=a & vuttasir\=anehi \\
6. gen. & vuttasirassa & vuttasir\=ana\d m \\
7. loc. & vuttasir\=ane & vuttasir\=anesu \\
\=a. voc. & vuttasira & vuttasir\=ano \\
\end{decltable}

\begin{decltable}{Declension of m.\ \pali{addha}\label{decl:addha}}
1. nom. & \texthl{addh\=a} & addh\=a, addh\=ano \\
2. acc. & addh\=ana\d m & addh\=ane \\
3. ins. & addhun\=a & addh\=anehi \\
4. dat. & addhuno & addh\=ana\d m \\
5. abl. & addhun\=a & addh\=anehi \\
6. gen. & addhuno & addh\=ana\d m \\
7. loc. & addhani, addh\=ane & addh\=anesu \\
\=a. voc. & addha & addh\=a, addh\=ano \\
\end{decltable}

\newpage
\begin{decltable}{Declension of m.\ \pali{muddha}\label{decl:muddha}}
1. nom. & \texthl{muddh\=a} & muddh\=a, muddh\=ano \\
2. acc. & muddha\d m & muddhe, muddh\=ane \\
3. ins. & muddh\=an\=a & muddhehi \\
4. dat. & muddhassa & muddh\=ana\d m \\
5. abl. & muddh\=an\=a & muddhehi \\
6. gen. & muddhassa & muddh\=ana\d m \\
7. loc. & muddhani & muddhanesu \\
\=a. voc. & muddha & muddh\=a, muddh\=ano \\
\end{decltable}

\begin{decltable}{Declension of nt.\ \pali{kamma}\label{decl:kamma}}
1. nom. & kamma\d m & kamm\=a, kamm\=ani \\
2. acc. & kamma\d m & kamme, kamm\=ani \\
3. ins. & kammena, \texthl{kammun\=a}, \texthl{kamman\=a} & kammehi \\
4. dat. & kammassa, \texthl{kammuno} & kamm\=ana\d m \\
5. abl. & kammasm\=a, kamm\=a, \texthl{kammun\=a} & kammehi \\
6. gen. & kammassa, \texthl{kammuno} & kamm\=ana\d m \\
7. loc. & kammasmi\d m, kamme, \texthl{kammani} & kammesu \\
\=a. voc. & kamma & kamm\=a, kamm\=ani \\
\end{decltable}

\begin{decltable}{Declension of m.\ \pali{s\=a} (dog)\label{decl:saa}\footnote{R\=upa\,144}}
1. nom. & s\=a & s\=a \\
2. acc. & sa\d m & se \\
3. ins. & sena & s\=ahi \\ 
4. dat. & sassa, s\=aya & s\=ana\d m \\
5. abl. & sasm\=a, s\=a & s\=ahi \\ 
6. gen. & sassa & s\=ana\d m \\
7. loc. & sasmi\d m, se & s\=asu \\
\=a. voc. & sa, s\=a & s\=a \\
\end{decltable}

\begin{decltable}{Declension of m.\ \pali{s\=a} (dog)\footnote{Sadd-Pad Ch.\,6}}
1. nom. & s\=a & s\=a, s\=ano \\
2. acc. & s\=ana\d m & s\=ane \\
3. ins. & s\=an\=a & s\=anehi \\ 
4. dat. & s\=assa, s\=anassa & s\=ana\d m \\
5. abl. & s\=an\=a & s\=anehi \\  
6. gen. & s\=assa, s\=anassa & s\=ana\d m \\
7. loc. & s\=ane & s\=anesu \\
\=a. voc. & s\=a, s\=ana & s\=ano \\
\end{decltable}

\newpage
\begin{decltable}{Declension of nt.\ \pali{assaddh\=a}\label{decl:assaddhaa}}
1. nom. & assaddha\d m & assaddh\=a, assaddh\=ani \\
2. acc. & assaddha\d m & assaddhe, assaddh\=ani \\
3. ins. & assaddhena & assaddhehi \\
4. dat. & assaddhassa & assaddh\=ana\d m \\
5. abl. & assaddhasm\=a, assaddh\=a & assaddhehi \\
6. gen. & assaddhassa & assaddh\=ana\d m \\
7. loc. & assaddhasmi\d m, assaddhe & assaddhesu \\
\=a. voc. & assaddha & assaddh\=a, assaddh\=ani \\
\end{decltable}

\begin{decltable}{Declension of f.\ \pali{bodhi}\footnote{This paradigm is from Sadd-Pad Ch.\,8. Some peculiar forms appear in other terms as well. For example, \pali{pokkhara\d n\=i} has \pali{pokkara\~n\~no, pokkara\~n\~n\=a, pokkara\~n\~na\d m}; \pali{d\=as\=i} has \pali{d\=asyo, d\=asy\=a, d\=asya\d m}; \pali{br\=ahma\d n\=i} has \pali{br\=ahma\d nyo, br\=ahma\d ny\=a}; and \pali{nad\=i} has \pali{najjo, najj\=a, najja\d m}. I also find \pali{jacc\=a} as ins.\ of \pali{j\=ati}.}\label{decl:bodhi}}
1. nom. & bodhi & bodh\=i, bodhiyo, bojjho \\
2. acc. & bodhi\d m, bodhiya\d m, bojjha\d m & bodh\=i, bodhiyo, bojjho \\
3. ins. & bodhiy\=a, bojjh\=a & bodh\=ihi \\
4. dat. & bodhiy\=a, bojjh\=a & bodh\=ina\d m \\
5. abl. & bodhiy\=a, bojjh\=a & bodh\=ihi \\
6. gen. & bodhiy\=a, bojjh\=a & bodh\=ina\d m \\
7. loc. & \mbox{bodhiy\=a, bojjh\=a, bhodiya\d m, bojjha\d m} & bodh\=isu \\
\=a. voc. & bodhi & bodh\=i, bodhiyo, bojjho \\
\end{decltable}

\begin{decltable}{Declension of nt.\ \pali{sukhak\=ar\=i}\label{decl:sukhakaarii}}
1. nom. & sukhak\=ari & sukhak\=ar\=i, sukhak\=ar\=ini \\
2. acc. & sukhak\=ari\d m, sukhak\=arina\d m & sukhak\=ar\=i, sukhak\=ar\=ini \\
3. ins. & sukhak\=arin\=a & sukhak\=ar\=ihi \\
4. dat. & sukhak\=arissa, sukhak\=arino & skkhak\=ar\=ina\d m \\
5. abl. & sukhak\=arism\=a, sukhak\=arin\=a & sukhak\=ar\=ihi \\
6. gen. & sukhak\=arissa, sukhak\=arino & sukhak\=ar\=ina\d m \\
7. loc. & sukhak\=arismi\d m, sukhak\=arini & sukhak\=ar\=isu \\
\=a. voc. & sukhak\=ari & sukhak\=ar\=i, sukhak\=ar\=ini \\
\end{decltable}

\begin{decltable}{Declension of nt.\ \pali{gotrabh\=u}\label{decl:gotrabhuu}}
1. nom. & gotrabhu & gotrabh\=u, gotrabh\=uni \\
2. acc. & gotrabhu\d m & gotrabh\=u, gotrabh\=uni \\
3. ins. & gotrabhun\=a & gotrabh\=uhi \\
4. dat. & gotrabhussa, gotrabhuno & gotrabh\=una\d m \\
5. abl. & gotrabhusm\=a, gotrabhun\=a & gotrabh\=uhi \\
6. gen. & gotrabhussa, gotrabhuno & gotrabh\=una\d m \\
7. loc. & gotrabhusmi\d m & gotrabh\=usu \\
\=a. voc. & gotrabhu & gotrabh\=u, gotrabh\=uni \\
\end{decltable}

\begin{listtableF}{Words declining as \pali{gotrabh\=u} (nt.)}
abhibh\=u & dhamma\~n\~n\=u & sayambh\=u & & \\
\end{listtableF}

\begin{decltable}{Declension of m.\ \& f.\ \pali{go}\label{decl:go}}
1. nom. & go & g\=avo, gavo \\
2. acc. & g\=avu\d m, g\=ava\d m, gava\d m & g\=avo, gavo \\
3. ins. & g\=avena, gavena & gohi \\ 
4. dat. & g\=avassa, gavassa & gava\d m, gunna\d m, gona\d m \\
5. abl. & g\=avasm\=a, g\=av\=a, gavasm\=a, gav\=a & gohi \\ 
6. gen. & g\=avassa, gavassa & gava\d m, gunna\d m, gona\d m \\
7. loc. & g\=avasmi\d m, g\=ave, gavasmi\d m, gave & g\=avesu, gavesu, gosu \\
\=a. voc. & go & g\=avo, gavo \\
\end{decltable}

\begin{decltable}{Declension of nt.\ \pali{cittago}\label{decl:cittago}}
1. nom. & cittagu & cittag\=u, cittag\=uni \\
2. acc. & cittagu\d m & cittag\=u, cittag\=uni \\
3. ins. & cittagun\=a & cittag\=uhi, cittaguhi \\
4. dat. & cittagussa, cittaguno & cittag\=una\d m, cittaguna\d m \\
5. abl. & cittagusm\=a, cittagun\=a & cittag\=uhi, cittaguhi \\
6. gen. & cittagussa, cittaguno & cittag\=una\d m, cittaguna\d m \\
7. loc. & cittagusmi\d m & cittag\=usu, cittagusu \\
\=a. voc. & cittagu & cittag\=u, cittag\=uni \\
\end{decltable}

\begin{decltable}{Declension of m.\ \pali{satthu}\label{decl:satthu}}
1. nom. & \texthl{satth\=a} & \texthl{satth\=aro} \\
2. acc. & \texthl{satth\=ara\d m} & satth\=are, satth\=aro \\
3. ins. & \texthl{satth\=ar\=a}, \texthl{satthun\=a} & satth\=arehi\\
4. dat. & satthussa, satthuno, \texthl{satthu} & satth\=ana\d m, \texthl{satth\=ar\=ana\d m} \\
5. abl. & \texthl{satth\=ar\=a} & satth\=arehi\\
6. gen. & satthussa, satthuno, \texthl{satthu} & satth\=ana\d m, \texthl{satth\=ar\=ana\d m} \\
7. loc. & \texthl{satthari} & satth\=aresu \\
\=a. voc. & sattha, satth\=a & satth\=aro \\
\end{decltable}

\newpage
\begin{decltable}{Declension of m.\ \pali{kattu}\label{decl:kattu}}
1. nom. & \texthl{katt\=a} & \texthl{katt\=aro} \\
2. acc. & \texthl{katt\=ara\d m} & katt\=are, katt\=aro \\
3. ins. & \texthl{katt\=ar\=a}, \texthl{kattun\=a} & katt\=arehi\\
4. dat. & kattussa, kattuno, \texthl{kattu} & katt\=ana\d m, \texthl{katt\=ar\=ana\d m}, \texthl{katt\=una\d m}, \texthl{kattuna\d m} \\
5. abl. & \texthl{katt\=ar\=a}, \texthl{kattun\=a} & katt\=arehi\\
6. gen. & kattussa, kattuno, \texthl{kattu} & katt\=ana\d m, \texthl{katt\=ar\=ana\d m}, \texthl{katt\=una\d m}, \texthl{kattuna\d m} \\
7. loc. & \texthl{kattari} & katt\=aresu, \texthl{katt\=usu}, \texthl{kattusu} \\
\=a. voc. & katta, katt\=a, katte & katt\=aro \\
\end{decltable}

\begin{listtableF}{Words declining as \pali{kattu}\footnote{I follow R\=upa here, and add some more from Sadd-Pad Ch.\,6. However, in Sadd-Pad most terms follow the paradigm of \pali{satthu}, and \pali{kattu} is treated as a special case. No one can really say which is right, for we cannot find all forms of all terms in the collection. Aggava\d msa seems to miss some forms, i.e.\ \pali{satth\=are} and \pali{satthun\=a}. And \pali{katte} is not found in R\=upa. Yet, other peculiar forms can also be found occasionally. In practice, I suggest that we can merge two paradigms into one when we use with other terms than \pali{satthu} and \pali{kattu}. Following Sadd-Pad Ch.\,8, feminine words of this group decline as f.\ \pali{nattu} below.}}
akkh\=atu & \mbox{abhibhavitu} & u\d t\d th\=atu & upp\=adetu & okkamitu \\
k\=aretu & khattu & khantu & gajjitu & gantu \\
cetu & chettu & jetu & \~n\=atu & tatu \\
t\=atu & d\=atu & dh\=atu & nattu & netu \\
nettu & pa\d tisedhitu & pa\d tisevitu & panattu & pabr\=uhetu \\
pucchitu & bhattu & bh\=asitu & bhettu & bhoddhu \\
bhodhetu & metu & mucchitu & vattu & vassitu \\
vi\~n\~n\=apetu & vinetu & sandassetu & sahitu & s\=avetu \\
sotu & hantu & & & \\
\end{listtableF}

\begin{decltable}{Declension of m.\ \pali{pitu (pitar)}\label{decl:pitu}}
1. nom. & \texthl{pit\=a} & \texthl{pitaro} \\
2. acc. & \texthl{pitara\d m} & pitre, pitaro \\
3. ins. & \texthl{pitar\=a}, \texthl{pitun\=a}, pety\=a & pitarehi, pit\=uhi, pituhi \\
4. dat. & pitussa, pituno, \texthl{pitu} & pitar\=ana\d m, pit\=ana\d m, pit\=una\d m, pituna\d m \\
5. abl. & \texthl{pitar\=a}, pety\=a & pitarehi, pit\=uhi, pituhi \\
6. gen. & pitussa, pituno, \texthl{pitu} & pitar\=ana\d m, pit\=ana\d m, pit\=una\d m, pituna\d m \\
7. loc. & \texthl{pitari} & pitaresu, pit\=usu, pitusu \\
\=a. voc. & pita, pit\=a & pitaro \\
\end{decltable}

\begin{listtableF}{Words declining as \pali{pitu}\footnote{Words ending with \pali{bh\=atu} do not have the form of \pali{pety\=a} and \pali{pit\=una\d m}. See Sadd-Pad Ch.\,6, \pali{Ettha pana ``pety\=a, pit\=unan''ti \ldots}}}
\mbox{ka\d ni\d t\d thabh\=atu} & c\=ulapitu & j\=am\=atu & \mbox{je\d t\d thabh\=atu} & bh\=atu \\
\end{listtableF}

\begin{decltable}{Declension of f.\ \pali{m\=atu (m\=atar)}\label{decl:maatu}\footnote{In Sadd-Pad Ch.\,8, \pali{m\=at\=a} can also be used as pl., both in nom.\ and voc. And \pali{maty\=a} can be used from ins.\ to loc.}}
1. nom. & m\=at\=a & m\=ataro \\
2. acc. & m\=atara\d m & m\=atare, m\=ataro \\
3. ins. & m\=atar\=a, m\=atuy\=a, \texthl{maty\=a} & m\=atarehi, m\=at\=uhi, m\=atuhi \\
4. dat. & m\=atussa, m\=atuy\=a, m\=atu & m\=atar\=ana\d m, m\=at\=ana\d m, m\=at\=una\d m, m\=atuna\d m \\
5. abl. & m\=atar\=a, m\=atuy\=a & m\=atarehi, m\=at\=uhi, m\=atuhi \\
6. gen. & m\=atussa, m\=atuy\=a, m\=atu & m\=atar\=ana\d m, m\=at\=ana\d m, m\=at\=una\d m, m\=atuna\d m \\
7. loc. & m\=atari, m\=atuya\d m, \texthl{matya\d m} & m\=ataresu, m\=at\=usu, m\=atusu \\
\=a. voc. & m\=ata, m\=at\=a & m\=ataro \\
\end{decltable}

\begin{listtableF}{Words declining as \pali{m\=atu}\footnote{The forms of \pali{maty\=a} and \pali{matya\d m} do not apply here. From Sadd-Pad Ch.\,8, \pali{dh\=ita\d m} can be used as acc.\ sg.}}
c\=ulam\=atu & dh\=itu & duhitu & bh\=atudh\=itu & \\
\end{listtableF}

\begin{decltable}{Declension of f.\ \pali{nattu}\label{decl:nattu}}
1. nom. & natt\=a & natt\=a, natt\=aro \\
2. acc. & natta\d m, natt\=ara\d m & natt\=aro \\
3. ins. & natt\=ar\=a, nattuy\=a & natt\=uhi \\
4. dat. & nattu, nattuy\=a & natt\=ar\=ana\d m, natt\=ana\d m, natt\=una\d m \\
5. abl. & natt\=ar\=a, nattuy\=a & natt\=uhi \\
6. gen. & nattu, nattuy\=a & natt\=ar\=ana\d m, natt\=ana\d m, natt\=una\d m \\
7. loc. & nattari, nattuy\=a, nattuya\d m & natt\=usu \\
\=a. voc. & natta, natt\=a & natt\=a, natt\=aro \\
\end{decltable}

\newpage
\begin{decltable}{Declension of m.\ \pali{gu\d navantu} (\pali{gu\d navant})\label{decl:gunavm}\footnote{This term is often listed in dictionaries in its Sanskrit stem form as \pali{gu\d navant} \citep[see][p.~58]{collins:grammar}. However, this form is not used in traditional textbooks. I follow the rules of Padar\=upasiddhi e.g.\ R\=upa\,98--99 for nom. The form of \pali{gu\d navanto} is not used as singular except some are found in verses (Sadd\,252). The form of \pali{gu\d nav\=a} can also be plural (Sadd\,297).}}
1. nom. & \texthl{gu\d nav\=a} & \mbox{gu\d navanto, gu\d navant\=a} \\
2. acc. & gu\d navanta\d m & gu\d navante \\
3. ins. & gu\d navantena, \texthl{gu\d navat\=a} & gu\d navantehi \\
4. dat. & gu\d navantassa, \texthl{gu\d navato} & gu\d navant\=ana\d m, \texthl{gu\d navata\d m} \\
5. abl. & gu\d navantasm\=a, gu\d navant\=a, \texthl{gu\d navat\=a} & gu\d navantehi \\
6. gen. & gu\d navantassa, \texthl{gu\d navato} & gu\d navant\=ana\d m, \texthl{gu\d navata\d m} \\
7. loc. & gu\d navantasmi\d m, gu\d navante \texthl{gu\d navati} & gu\d navantesu \\
\=a. voc. & \texthl{gu\d nav\=a}, gu\d nava, gu\d nava\d m & \mbox{gu\d navanto, gu\d navant\=a} \\
\end{decltable}

\begin{decltable}{Declension of nt.\ \pali{gu\d navantu}\label{decl:gunavnt}}
1. nom. & \texthl{gu\d nava\d m} & \texthl{gu\d navanti}, gu\d navant\=ani \\
2. acc. & gu\d navanta\d m & gu\d navante, gu\d navant\=ani \\
3. ins. & \rdelim{\}}{5}{\linewidth}[as m.\ \pali{gu\d navantu}] & \\
4. dat. & & \\
5. abl. & & \\
6. gen. & & \\
7. loc. & & \\
\=a. voc. & gu\d nava\d m, gu\d nava, gu\d nav\=a & \texthl{gu\d navanti}, gu\d navant\=ani \\
\end{decltable}

\begin{listtableF}{Words declining as \pali{gu\d navantu}}
atthavantu & katavantu & kulavantu & ga\d navantu & c\=agavantu \\
\mbox{cetan\=avantu} & \mbox{th\=amavantu} & \mbox{dhanavantu} & dhitivantu & \mbox{dhutavantu} \\
\mbox{pa\~n\~navantu} & phalavantu & balavantu & \mbox{bhagavantu} & \mbox{massuvantu} \\
yatavantu & yasavantu & \mbox{yasassivantu} & \mbox{rasmivantu} & vidvantu \\
\mbox{vedan\=avantu} & \mbox{sa\~n\~n\=avantu} & \mbox{saddh\=avantu} & sabb\=avantu & s\=ilavantu \\
sutavantu & hitavantu & & & \\
\end{listtableF}

\begin{decltable}{Declension of f.\ \pali{gu\d navat\=i}\label{decl:gunavf}}
1. nom. & gu\d navat\=i & gu\d navat\=i, gu\d navatiyo \\
2. acc. & gu\d navati\d m, gu\d navatiya\d m & gu\d navat\=i, gu\d navatiyo \\
3. ins. & gu\d navatiy\=a & gu\d navat\=ihi \\
4. dat. & gu\d navatiy\=a & gu\d navat\=ina\d m \\
5. abl. & gu\d navatiy\=a & gu\d navat\=ihi \\
6. gen. & gu\d navatiy\=a & gu\d navat\=ina\d m \\
7. loc. & gu\d navatiy\=a, gu\d navatiya\d m & gu\d navat\=isu \\
\=a. voc. & gu\d navati & gu\d navat\=i, gu\d navatiyo \\
\end{decltable}

\begin{listtableF}{Words declining as \pali{gu\d navat\=i}}
gu\d navant\=i & gacchant\=i & & & \\
\end{listtableF}

\begin{decltable}{Declension of m.\ \pali{himavantu}\label{decl:himavantu}}
1. nom. & himav\=a, \texthl{himavanto} & \mbox{himavanto, himavant\=a} \\
2. acc. & \rdelim{\}}{7}{\linewidth}[as \pali{gu\d navantu}] & \\
3. ins. & & \\ 
4. dat. & & \\
5. abl. & & \\ 
6. gen. & & \\
7. loc. & & \\
\=a. voc. & & \\
\end{decltable}

\begin{listtableF}{Words declining as \pali{himavantu}\footnote{In Sadd-Pad Ch.\,6, all these and those of \pali{vantu} ending are of the same group, declining in the same way. I follow R\=upa here by dividing these into two groups. First, \pali{vantu} group follows \pali{gu\d navantu} paradigm. And second, \pali{mantu} group follows \pali{himavantu} paradigm. The two paradigms are mostly the same, except there is no form like \pali{gu\d navanto} as nom.\ sg.}}
\mbox{atthadassimantu} & & \=ayasmantu & kalimantu & kasimantu \\
\mbox{kh\=a\d numantu} & gatimantu & gomantu & \mbox{cakkhumantu} & \mbox{cantimantu} \\
jutimantu & thutimantu & dhitimantu & dh\=imantu & p\=apimantu \\
puttimantu & balimantu & \mbox{bh\=a\d numantu} & \mbox{buddhimantu} & matimantu \\
mutimantu & \mbox{muttimantu} & yatimantu & ratimantu & r\=ahumantu \\
rucimantu & vasumantu & vijjumantu & sirimantu & sucimantu \\
setumantu & hirimantu & hetumantu & & \\
\end{listtableF}

\begin{decltable}{Declension of m.\ \pali{satimantu}\label{decl:satimantu}}
1. nom. & satim\=a, satimanto & \mbox{satimanto, satimant\=a} \\
2. acc. & satimanta\d m, \texthl{satima\d m} & satimante \\
3. ins. & satimantena, satimat\=a & satimantehi \\ 
4. dat. & satimantassa, satimato, \texthl{satimassa} & satimant\=ana\d m, satimata\d m\\
5. abl. & satimantasm\=a, satimant\=a, satimat\=a & satimantehi \\ 
6. gen. & satimantassa, satimato, \texthl{satimassa} & satimant\=ana\d m, satimata\d m\\
7. loc. & satimantasmi\d m, satimante, satimati & satimantesu \\
\=a. voc. & satim\=a, satima, satima\d m & \mbox{satimanto, satimant\=a} \\
\end{decltable}

\begin{listtableF}{Words declining as \pali{satimantu}}
\mbox{bandhumantu} & & & & \\
\end{listtableF}

\begin{decltable}{Declension of m.\ \pali{gacchanta}\label{decl:gacchanta}}
1. nom. & \texthl{gaccha\d m}, gacchanto & gacchanto, gacchant\=a \\
2. acc. & gacchanta\d m & gacchante, gacchanto \\
3. ins. & gacchantena, gacchat\=a & gacchantehi \\ 
4. dat. & gacchantassa, gacchato & gacchant\=ana\d m, gacchata\d m\\
5. abl. & gacchantasm\=a, gacchant\=a, gacchat\=a & gacchantehi \\ 
6. gen. & gacchantassa, gacchato & gacchant\=ana\d m, gacchata\d m\\
7. loc. & gacchantasmi\d m, gacchante, gacchati & gacchantesu \\
\=a. voc. & gacch\=a, gaccha, gaccha\d m & gacchanto, gacchant\=a \\
\end{decltable}

\begin{decltable}{Declension of nt.\ \pali{gacchanta}}
1. nom. & gaccha\d m, gacchanta\d m & gacchant\=a, gacchant\=ani \\
2. acc. & gacchanta\d m & gacchante, gacchant\=ani \\
2. acc. & gacchanta\d m & gacchante \\
3. ins. & \rdelim{\}}{4}{\linewidth}[as m.\ \pali{gacchanta}] & \\
4. dat. & & \\
5. abl. & & \\
6. gen. & & \\
7. loc. & as m.\ \pali{gacchanta} & \\
\=a. voc. & gacchanta & gacchant\=a, gacchant\=ani \\
\end{decltable}

\begin{listtableF}{Words declining as \pali{gacchanta}}
kubbanta & caranta & cavanta & japanta & jayanta \\
j\=iranta & ti\d t\d thanta & dadanta & pacanta & bhu\~njanta \\
mahanta & m\=iyanta & vajanta & saranta & su\d nanta \\
\end{listtableF}

\begin{decltable}{Declension of m.\ \pali{bhavanta}\label{decl:bhavanta}}
1. nom. & \texthl{bhava\d m} & bhavanto, bhavant\=a, \texthl{bhonto} \\
2. acc. & bhavanta\d m & bhavante, bhonte \\
3. ins. & bhavantena, bhavat\=a, \texthl{bhot\=a} & bhavantehi \\ 
4. dat. & bhavantassa, bhavato, \texthl{bhoto} & bhavant\=ana\d m, bhavata\d m\\
5. abl. & bhavantasm\=a, bhavant\=a, bhavat\=a, \texthl{bhot\=a} & bhavantehi \\ 
6. gen. & bhavantassa, bhavato, \texthl{bhoto} & bhavant\=ana\d m, bhavata\d m\\
7. loc. & bhavantasmi\d m, bhavante, bhavati & bhavantesu \\
\=a. voc. & \texthl{bho}, \texthl{bhante}, \texthl{bhonta}, \texthl{bhont\=a} & bhavanto, bhavant\=a, \texthl{bhonto} \\
\end{decltable}

\begin{decltable}{Declension of m.\ \pali{karonta}\label{decl:karonta}}
1. nom. & \texthl{kara\d m} & karonto, karont\=a \\
2. acc. & karonta\d m & karonte \\
3. ins. & karontena, \texthl{karot\=a} & karontehi \\
4. dat. & karontassa, \texthl{karoto} & karont\=ana\d m, karota\d m \\
5. abl. & karotasm\=a, \texthl{karot\=a}, karont\=a  & karontehi \\
6. gen. & karontassa, \texthl{karoto} & karont\=ana\d m, karota\d m \\
7. loc. & karontasmi\d m, karonte & karontesu \\
\=a. voc. & karonta & karont\=a \\
\end{decltable}

\begin{decltable}{Declension of adj.\ \pali{arahanta}\label{decl:arahanta}\footnote{As an adjective, this means `worth worshiping.' When used as a noun denoting an arhant, the nom.\ sg.\ form is \pali{arah\=a}. See Sadd-Pad Ch.\,7.}}
1. nom. & \texthl{araha\d m} & arahanto \\
2. acc. & arahanta\d m & arahante \\
3. ins. & arahantena, \texthl{arahat\=a} & arahantehi \\
4. dat. & arahantassa, \texthl{arahato} & arahant\=ana\d m, arahata\d m \\
5. abl. & arahantasm\=a, \texthl{arahat\=a}, arahant\=a  & arahantehi \\
6. gen. & arahantassa, \texthl{arahato} & arahant\=ana\d m, arahata\d m \\
7. loc. & arahantasmi\d m, arahante & arahantesu \\
\=a. voc. & arahanta & arahanto \\
\end{decltable}

\begin{decltable}{Declension of m.\ \pali{santa} (righteous person)\label{decl:santa1}}
1. nom. & \texthl{sa\d m}, santo & santo, sant\=a \\
2. acc. & sa\d m, santa\d m & sante \\
3. ins. & santena, \texthl{sat\=a} & santehi, \texthl{sabbhi} \\ 
4. dat. & santassa, \texthl{sato} & sant\=ana\d m, sata\d m\\
5. abl. & santasm\=a, \texthl{sat\=a}, sant\=a & santehi, \texthl{sabbhi} \\ 
6. gen. & santassa, \texthl{sato} & sant\=ana\d m, sata\d m\\
7. loc. & santasmi\d m, sante, sati & santesu \\
\=a. voc. & sa\d m, sa, s\=a, santa & santo, sant\=a \\
\end{decltable}

\begin{decltable}{Declension of m.\ \pali{santa} (existing)\label{decl:santa2}\footnote{This is used as an adjective. When \pali{santa} is used to mean `weary', `calmed', or `ceased', it decline as regular nouns.}}
1. nom. & santo & santo, sant\=a \\
2. acc. & santa\d m & sante \\
3. ins. & santena, \texthl{sat\=a} & santehi \\
4. dat. & santassa, \texthl{sato} & sant\=ana\d m, sata\d m\\
5. abl. & santasm\=a, \texthl{sat\=a}, sant\=a & santehi \\
6. gen. & santassa, \texthl{sato} & sant\=ana\d m, sata\d m\\
7. loc. & santasmi\d m, sante, sati & santesu \\
\=a. voc. & santa & santo, sant\=a \\
\end{decltable}

\begin{decltable}{Declension of m.\ \pali{mahanta}\label{decl:mahanta}\footnote{This paradigm is proposed by Aggava\d msa in Sadd-Pad Ch.\,7. It is somehow an extended version of the regular paradigm. For f., we use \pali{mahat\=i} or \pali{mahant\=a} with regular paradigm, as well as nt.\ which declines to \pali{mahanta\d m, mahant\=ani,} etc. In addition to the regular declension, \pali{mah\=a} can be used as nom.\ sg.\ in all genders. Unlike \pali{gu\d navant\=i}, Aggava\d msa maintains that \pali{mahant\=i} is not a correct form of f. Still, we can find its uses, but only in postcanonical texts.}}
1. nom. & \texthl{maha\d m}, \texthl{mah\=a}, mahanto & mahant\=a \\
2. acc. & mahanta\d m & mahante \\
3. ins. & mahantena, mahat\=a & mahantehi \\
4. dat. & mahantassa, mahato & mahant\=ana\d m, mahata\d m \\
5. abl. & mahantasm\=a, mahat\=a, mahant\=a & mahantehi \\
6. gen. & mahantassa, mahato & mahant\=ana\d m, mahata\d m \\
7. loc. & mahantasmi\d m, mahati, mahante & mahantesu \\
\=a. voc. & maha, mah\=a & mahanto \\
\end{decltable}

\section{Pronouns}\label{decl:pron}

\begin{decltable}{Declension of \pali{amha}}
1. nom. & aha\d m & maya\d m, amhe, no \\
2. acc. & ma\d m, mama\d m & amh\=aka\d m, amhe, no \\
3. ins. & may\=a, me & amhehi, no \\
4. dat. & mayha\d m, amha\d m, mama, mama\d m, me & amha\d m, amh\=aka\d m, asm\=aka\d m, no \\
5. abl. & may\=a & amhehi \\
6. gen. & mayha\d m, amha\d m, mama, mama\d m, me & amha\d m, amh\=aka\d m, asm\=aka\d m, no \\
7. loc. & mayi & amhesu \\
\end{decltable}

\begin{decltable}{Declension of \pali{tumha}}
1. nom. & tva\d m, tuva\d m & tumhe, vo  \\
2. acc. & tva\d m, tuva\d m, tava\d m, ta\d m & tumh\=aka\d m, tumhe, vo \\
3. ins. & tay\=a, tvay\=a, te & tumhehi, vo  \\
4. dat. & tuyha\d m, tumha\d m, tava, te &  tumha\d m, tumh\=aka\d m, vo \\
5. abl. & tay\=a & tumhehi \\
6. gen. & tuyha\d m, tumha\d m, tava, te &  tumha\d m, tumh\=aka\d m, vo \\
7. loc. & tayi, tvayi & tumhesu \\
\end{decltable}

\begin{decltable}{Declension of m.\ \pali{ta}}
1. nom. & so & te, ne \\
2. acc. & ta\d m, na\d m & te, ne \\
3. ins. & tena, nena & tehi, nehi \\
4. dat. & tassa, nassa, assa & tesa\d m, tes\=ana\d m, nesa\d m, nes\=ana\d m \\
5. abl. & tasm\=a, nasm\=a, asm\=a & tehi, nehi \\
6. gen. & tassa, nassa, assa & tesa\d m, tes\=ana\d m, nesa\d m, nes\=ana\d m \\
7. loc. & tasmi\d m, nasmi\d m, asmi\d m & tesu, nesu \\
\end{decltable}

\begin{decltable}{Declension of f.\ \pali{ta}}
1. nom. & s\=a & t\=a, t\=ayo, n\=a, n\=ayo \\
2. acc. & ta\d m, na\d m & t\=a, t\=ayo, n\=a, n\=ayo \\
3. ins. & t\=aya, n\=aya, tass\=a, tiss\=a & t\=ahi, n\=ahi \\
4. dat. & t\=aya, tass\=a, tass\=aya, tiss\=a, tiss\=aya, n\=aya, nass\=a, nass\=aya, ass\=a, ass\=aya & t\=asa\d m, t\=as\=ana\d m, n\=asa\d m, n\=as\=ana\d m \\
5. abl. & t\=aya, n\=aya & t\=ahi, n\=ahi \\
6. gen. & t\=aya, tass\=a, tass\=aya, tiss\=a, tiss\=aya, n\=aya, nass\=a, nass\=aya, ass\=a, ass\=aya & t\=asa\d m, t\=as\=ana\d m, n\=asa\d m, n\=as\=ana\d m \\
7. loc. & t\=aya\d m, tassa\d m, tissa\d m, n\=aya\d m, nassa\d m, assa\d m & t\=asu, n\=asu \\
\end{decltable}

\begin{decltable}{Declension of nt.\ \pali{ta}}
1. nom. & ta\d m, na\d m & t\=ani, n\=ani \\
2. acc. & ta\d m, na\d m & t\=ani, n\=ani \\
3. ins. & \rdelim{\}}{5}{\linewidth}[as m.\ \pali{ta}] & \\
4. dat. & & \\
5. abl. & & \\
6. gen. & & \\
7. loc. & & \\
\end{decltable}

\begin{decltable}{Declension of m.\ \pali{eta}}
1. nom. & eso & te \\
2. acc. & eta\d m, ena\d m & ete, ene \\
3. ins. & etena & etehi \\
4. dat. & etassa & etesa\d m, etes\=ana\d m \\
5. abl. & etasm\=a & teehi \\
6. gen. & etassa & etesa\d m, etes\=ana\d m \\
7. loc. & etasmi\d m & etesu \\
\end{decltable}

\begin{decltable}{Declension of f.\ \pali{eta}}
1. nom. & es\=a & et\=a, et\=ayo \\
2. acc. & eta\d m & et\=a, et\=ayo \\
3. ins. & et\=aya & et\=ahi \\
4. dat. & et\=aya, etiss\=a, etiss\=aya & et\=asa\d m, et\=as\=ana\d m \\
5. abl. & et\=aya & et\=ahi \\
6. gen. & et\=aya, etiss\=a, etiss\=aya & et\=asa\d m, et\=as\=ana\d m \\
7. loc. & et\=aya\d m, etissa\d m & et\=asu \\
\end{decltable}

\begin{decltable}{Declension of nt.\ \pali{eta}}
1. nom. & eta\d m & et\=ani \\
2. acc. & eta\d m & et\=ani \\
3. ins. & \rdelim{\}}{4}{\linewidth}[as m.\ \pali{eta}] & \\
4. dat. & & \\
5. abl. & & \\
6. gen. & & \\
7. loc. & as m.\ \pali{eta} & \\
\end{decltable}

\begin{decltable}{Declension of m.\ \pali{ima}}
1. nom. & aya\d m & ime \\
2. acc. & ima\d m & ime \\
3. ins. & imin\=a, anena & imehi, ehi \\
4. dat. & imassa, assa & imesa\d m, imes\=ana\d m, esa\d m, es\=ana\d m \\
5. abl. & imasm\=a, asm\=a & imehi, ehi \\
6. gen. & imassa, assa & imesa\d m, imes\=ana\d m, esa\d m, es\=ana\d m \\
7. loc. & imasmi\d m, asmi\d m & imesu, esu \\
\end{decltable}

\begin{decltable}{Declension of f.\ \pali{ima}}
1. nom. & aya\d m & im\=a, im\=ayo \\
2. acc. & ima\d m & im\=a, im\=ayo \\
3. ins. & im\=aya & im\=ahi \\
4. dat. & im\=aya, imiss\=a, imiss\=aya, ass\=a, ass\=aya & im\=asa\d m, im\=as\=ana\d m, \=asa\d m \\
5. abl. & im\=aya & im\=ahi \\
6. gen. & im\=aya, imiss\=a, imiss\=aya, ass\=a, ass\=aya & im\=asa\d m, im\=as\=ana\d m, \=asa\d m \\
7. loc. & im\=aya\d m, imiss\=a, imissa\d m, assa\d m & im\=asu \\
\end{decltable}

\begin{decltable}{Declension of nt.\ \pali{ima}}
1. nom. & ida\d m, ima\d m & im\=ani \\
2. acc. & ida\d m, ima\d m & im\=ani \\
3. ins. & \rdelim{\}}{5}{\linewidth}[as m.\ \pali{ima}] & \\
4. dat. & & \\
5. abl. & & \\
6. gen. & & \\
7. loc. & & \\
\end{decltable}

\begin{decltable}{Declension of m.\ \pali{amu}}
1. nom. & asu & am\=u \\
2. acc. & amu\d m & im\=u \\
3. ins. & amun\=a & am\=uhi, amuhi \\
4. dat. & amussa, (a)dussa\footnote{In R\=upa\,224 \pali{adussa} is listed, but in Sadd-Pad Ch.\,12 \pali{dussa} is listed. \textsc{P\=ali\,Platform} shows that \pali{adussa} is only found in the A\~n\~na (Extra) group of the CST4 corpus. There is no occurrence in the main texts.} & am\=usa\d m, am\=us\=ana\d m, amusa\d m, amus\=ana\d m \\
5. abl. & amusm\=a & am\=uhi, amuhi \\
6. gen. & amussa, (a)dussa & am\=usa\d m, am\=us\=ana\d m, amusa\d m, amus\=ana\d m \\
7. loc. & amusmi\d m & am\=usu, amusu \\
\end{decltable}

\begin{decltable}{Declension of f.\ \pali{amu}}
1. nom. & asu & am\=u, amuyo \\
2. acc. & amu\d m & im\=u, amuyo \\
3. ins. & amuy\=a & am\=uhi \\
4. dat. & amuy\=a, amuss\=a & am\=usa\d m, am\=us\=ana\d m \\
5. abl. & amuy\=a & am\=uhi \\
6. gen. & amuy\=a, amuss\=a & am\=usa\d m, am\=us\=ana\d m \\
7. loc. & amuy\=a, amuya\d m, amussa\d m & am\=usu \\
\end{decltable}

\begin{decltable}{Declension of nt.\ \pali{amu}}
1. nom. & adu\d m, amu\d m & am\=uni, am\=u \\
2. acc. & adu\d m, amu\d m & am\=uni, am\=u \\
3. ins. & \rdelim{\}}{5}{\linewidth}[as m.\ \pali{amu}] & \\
4. dat. & & \\
5. abl. & & \\
6. gen. & & \\
7. loc. & & \\
\end{decltable}

\newpage
\begin{decltable}{Declension of m.\ \pali{asuka}\label{decl:asuka}}
1. nom. & asuko & asuk\=a \\
2. acc. & asuka\d m & asuke \\
3. ins. & asukena & asukehi \\
4. dat. & asukassa & asuk\=ana\d m \\
5. abl. & asukasm\=a, asuk\=a & asukehi \\
6. gen. & asukassa & asuk\=ana\d m \\
7. loc. & asukasmi\d m, asuke & asukesu \\
\end{decltable}

\begin{decltable}{Declension of f.\ \pali{asuka}}
1. nom. & asuk\=a & asuk\=a, asuk\=ayo \\
2. acc. & asuka\d m & asuk\=a, asuk\=ayo \\
3. ins. & asuk\=aya & asuk\=ahi \\
4. dat. & asuk\=aya & asuk\=ana\d m \\
5. abl. & asuk\=aya & asuk\=ahi \\
6. gen. & asuk\=aya & asuk\=ana\d m \\
7. loc. & asuk\=aya\d m & asuk\=asu \\
\end{decltable}

\begin{decltable}{Declension of nt.\ \pali{asuka}}
1. nom. & asuka\d m & asuk\=ani, asuk\=a \\
2. acc. & asuka\d m & asuk\=ani, asuke \\
3. ins. & \rdelim{\}}{5}{\linewidth}[as m.\ \pali{asuka}] & \\
4. dat. & & \\
5. abl. & & \\
6. gen. & & \\
7. loc. & & \\
\end{decltable}

\begin{decltable}{Declension of m.\ \pali{ya}\label{decl:ya}}
1. nom. & yo & ye \\
2. acc. & ya\d m & ye \\
3. ins. & yena & yehi \\
4. dat. & yassa & yesa\d m, yes\=ana\d m \\
5. abl. & yasm\=a & yehi \\
6. gen. & yassa & yesa\d m, yes\=ana\d m \\
7. loc. & yasmi\d m & yesu \\
\end{decltable}

\begin{decltable}{Declension of f.\ \pali{ya}}
1. nom. & y\=a & y\=a, y\=ayo \\
2. acc. & ya\d m & y\=a, y\=ayo \\
3. ins. & y\=aya & y\=ahi \\
4. dat. & y\=aya, yass\=a & y\=asa\d m, y\=as\=ana\d m \\
5. abl. & y\=aya & y\=ahi \\
6. gen. & y\=aya, yass\=a & y\=asa\d m, y\=as\=ana\d m \\
7. loc. & y\=aya\d m, yassa\d m & y\=asu \\
\end{decltable}

\newpage
\begin{decltable}{Declension of nt.\ \pali{ya}}
1. nom. & ya\d m & y\=ani \\
2. acc. & ya\d m & y\=ani \\
3. ins. & \rdelim{\}}{5}{\linewidth}[as m.\ \pali{ya}] & \\
4. dat. & & \\
5. abl. & & \\
6. gen. & & \\
7. loc. & & \\
\end{decltable}

\begin{decltable}{Declension of m.\ \pali{ki\d m} (\pali{ka})}
1. nom. & ko & ke \\
2. acc. & ka\d m & ke \\
3. ins. & kena & kehi \\
4. dat. & kassa, kissa & kesa\d m, kes\=ana\d m \\
5. abl. & kasm\=a & kehi \\
6. gen. & kassa, kissa & kesa\d m, kes\=ana\d m \\
7. loc. & kasmi\d m, kismi\d m & kesu \\
\end{decltable}

\begin{decltable}{Declension of f.\ \pali{ki\d m}}
1. nom. & k\=a & k\=a, k\=ayo \\
2. acc. & ka\d m & k\=a, k\=ayo \\
3. ins. & k\=aya & k\=ahi \\
4. dat. & k\=aya, kass\=a & k\=asa\d m, k\=as\=ana\d m \\
5. abl. & k\=aya & k\=ahi \\
6. gen. & k\=aya, kass\=a & k\=asa\d m, k\=as\=ana\d m \\
7. loc. & k\=aya\d m, kassa\d m & k\=asu \\
\end{decltable}

\begin{decltable}{Declension of nt.\ \pali{ki\d m}}
1. nom. & ka\d m & k\=ani \\
2. acc. & ka\d m & k\=ani \\
3. ins. & \rdelim{\}}{4}{\linewidth}[as m.\ \pali{ki\d m}] & \\
4. dat. & & \\
5. abl. & & \\
6. gen. & & \\
7. loc. & as m.\ \pali{ki\d m} & \\
\end{decltable}

\begin{decltable}{Declension of m.\ \pali{ki\d m + ci}\label{decl:koci}}
1. nom. & koci & keci, kecana \\
2. acc. & ka\~nci, ki\~nci, ki\~ncana\d m & keci, kecana \\
3. ins. & kenaci & kehici \\
4. dat. & kassaci & kesa\~nci \\
5. abl. & kasm\=aci & kehici \\
6. gen. & kassaci & kesa\~nci \\
7. loc. & kasmi\~nci, kismi\~nci & kesuci \\
\end{decltable}

\newpage
\begin{decltable}{Declension of f.\ \pali{ki\d m + ci}}
1. nom. & k\=aci & k\=aci \\
2. acc. & ka\~nci, ki\~nci & k\=aci \\
3. ins. & k\=ayaci & k\=ahici \\
4. dat. & k\=ayaci, kass\=aci & k\=asa\~nci \\
5. abl. & k\=ayaci & k\=ahici \\
6. gen. & k\=ayaci, kass\=aci & k\=asa\~nci \\
7. loc. & k\=ayaci & k\=asuci \\
\end{decltable}

\begin{decltable}{Declension of nt.\ \pali{ki\d m + ci}}
1. nom. & ki\~nci & k\=anici \\
2. acc. & ki\~nci & k\=anici \\
3. ins. & \rdelim{\}}{2}{\linewidth}[as m.\ \pali{ki\d m + ci}] & \\
4. dat. & & \\
5. abl. & \rdelim{\}}{3}{\linewidth}[as m.\ \pali{ki\d m + ci}] & \\
6. gen. & & \\
7. loc. & & \\
\end{decltable}

\begin{decltable}{Declension of m.\ \pali{ya + ki\d m + ci}\label{decl:yokoci}}
1. nom. & yo koci & ye keci \\
2. acc. & ya\d m ka\~nci, ya\d m ki\~nci & ye keci \\
3. ins. & yena kenaci & yehi kehici \\
4. dat. & yassa kassaci & yesa\d m kesa\~nci \\
5. abl. & yasm\=a kasm\=aci & yehi kehici \\
6. gen. & yassa kassaci & yesa\d m kesa\~nci \\
7. loc. & yasmi\d m kasmi\~nci & yesu kesuci \\
\end{decltable}

\begin{decltable}{Declension of f.\ \pali{ya + ki\d m + ci}}
1. nom. & y\=a k\=aci & y\=a k\=aci \\
2. acc. & ya\d m ka\~nci, ya\d m ki\~nci & y\=a k\=aci \\
3. ins. & y\=aya k\=ayaci & y\=ahi k\=ahici \\
4. dat. & y\=aya k\=ayaci, kass\=aci & y\=asa\d m k\=asa\~nci \\
5. abl. & y\=aya k\=ayaci & y\=ahi k\=ahici \\
6. gen. & y\=aya k\=ayaci, kass\=aci & y\=asa\d m k\=asa\~nci \\
7. loc. & y\=aya k\=ayaci & y\=asu k\=asuci \\
\end{decltable}

\begin{decltable}{Declension of nt.\ \pali{ya + ki\d m + ci}}
1. nom. & ya\d m ki\~nci & y\=ani k\=anici \\
2. acc. & ya\d m ki\~nci & y\=ani k\=anici \\
3. ins. & \rdelim{\}}{5}{\linewidth}[as m.\ \pali{ya + ki\d m + ci}] & \\
4. dat. & & \\
5. abl. & & \\
6. gen. & & \\
7. loc. & & \\
\end{decltable}

\newpage
\begin{decltable}{Declension of m.\ \pali{sabba}\label{decl:sabba}}
1. nom. & sabbo & sabbe \\
2. acc. & sabba\d m & sabbe \\
3. ins. & sabbena & sabbehi \\
4. dat. & sabbassa & sabbesa\d m, sabbes\=ana\d m \\
5. abl. & sabbasm\=a, sabb\=a & sabbehi \\
6. gen. & sabbassa & sabbesa\d m, sabbes\=ana\d m \\
7. loc. & sabbasmi\d m, sabbe & sabbesu \\
\=a. voc. & sabba, sabb\=a & sabbe \\
\end{decltable}

\begin{decltable}{Declension of f.\ \pali{sabba}}
1. nom. & sabb\=a & sabb\=a, sabb\=ayo \\
2. acc. & sabba\d m & sabb\=a, sabb\=ayo \\
3. ins. & sabb\=aya, sabbass\=a & sabb\=ahi \\
4. dat. & sabb\=aya, sabbass\=a & sabb\=asa\d m, sabb\=as\=ana\d m \\
5. abl. & sabb\=aya, sabbass\=a & sabb\=ahi \\
6. gen. & sabb\=aya, sabbass\=a & sabb\=asa\d m, sabb\=as\=ana\d m \\
7. loc. & sabb\=aya\d m, sabbass\=a, sabbassa\d m & sabb\=asu \\
\=a. voc. & sabbe & sabb\=a, sabb\=ayo \\
\end{decltable}

\begin{decltable}{Declension of nt.\ \pali{sabba}}
1. nom. & sabba\d m & sabb\=ani \\
2. acc. & sabba\d m & sabb\=ani \\
3. ins. & \rdelim{\}}{5}{\linewidth}[as m.\ \pali{sabba}] &  \\
4. dat. &  &  \\
5. abl. &  &  \\
6. gen. &  &  \\
7. loc. &  &  \\
\=a. voc. & sabba & sabb\=ani \\
\end{decltable}

\begin{listtableF}{Words declining as \pali{sabba}}
katara & katama & ubhaya & itara & a\~n\~na \\
a\~n\~natara & a\~n\~natama & & & \\
\end{listtableF}

\begin{decltable}{Declension of m.\ \pali{pubba}\label{decl:pubba}}
1. nom. & pubbo & pubbe, pubb\=a \\
2. acc. & pubba\d m & pubbe \\
3. ins. & pubbena & pubbehi \\
4. dat. & pubbassa & pubbesa\d m, pubbes\=ana\d m \\
5. abl. & pubbasm\=a, pubb\=a & pubbehi \\
6. gen. & pubbassa & pubbesa\d m, pubbes\=ana\d m \\
7. loc. & pubbasmi\d m, pubbe & pubbesu \\
\=a. voc. & pubba & pubbe, pubb\=a \\
\end{decltable}

\newpage
\begin{decltable}{Declension of f.\ \pali{pubba}}
1. nom. & pubb\=a & pubb\=a, pubb\=ayo \\
2. acc. & pubba\d m & pubb\=a, pubb\=ayo \\
3. ins. & pubb\=aya & pubb\=ahi \\
4. dat. & pubb\=aya, pubbass\=a & pubb\=asa\d m, pubb\=as\=ana\d m \\
5. abl. & pubb\=aya & pubb\=ahi \\
6. gen. & pubb\=aya, pubbass\=a & pubb\=asa\d m, pubb\=as\=ana\d m \\
7. loc. & pubb\=aya\d m, pubbassa\d m & pubb\=asu \\
\=a. voc. & pubbe & pubb\=a, pubb\=ayo \\
\end{decltable}

\begin{decltable}{Declension of nt.\ \pali{pubba}}
1. nom. & pubba\d m & pubb\=ani \\
2. acc. & pubba\d m & pubb\=ani \\
3. ins. & \rdelim{\}}{5}{\linewidth}[as m.\ \pali{pubba}] & \\
4. dat. & & \\
5. abl. & & \\
6. gen. & & \\
7. loc. & & \\
\=a. voc. & pubba & pubb\=ani \\
\end{decltable}

\begin{listtableF}{Words declining as \pali{pubba}}
para & apara & dakkhi\d na & uttara & adhara\\
\end{listtableF}

\footnotesize
\begin{longtable}{|p{0.12\linewidth}%
	>{\raggedright\arraybackslash\itshape}p{0.23\linewidth}%
	>{\raggedright\arraybackslash\itshape}p{0.23\linewidth}%
	>{\raggedright\arraybackslash\itshape}p{0.23\linewidth}|}%
\multicolumn{4}{@{}p{0.8\linewidth}@{}}{\small Declension of \pali{eka} (sg.)}\label{decl:one}\\*
\hline
\bfseries\upshape case & \bfseries\upshape m. & \bfseries\upshape f. & \bfseries\upshape nt.\\
\hline
1. nom. & eko & ek\=a & eka\d m \\
2. acc. & eka\d m & eka\d m & eka\d m \\
3. ins. & ekena & ek\=aya & \rdelim{\}}{5}{\linewidth}[as m.] \\
4. dat. & ekassa & ek\=aya, ekiss\=a & \\
5. abl. & ekasm\=a & ek\=aya & \\
6. gen. & ekassa & ek\=aya, ekiss\=a & \\
7. loc. & ekasmi\d m & ek\=aya\d m, ekissa\d m & \\
\hline
\end{longtable}

\begin{longtable}{|p{0.12\linewidth}%
	>{\raggedright\arraybackslash\itshape}p{0.23\linewidth}%
	>{\raggedright\arraybackslash\itshape}p{0.23\linewidth}%
	>{\raggedright\arraybackslash\itshape}p{0.23\linewidth}|}%
\multicolumn{4}{@{}p{0.8\linewidth}@{}}{\small Declension of \pali{eka} (pl.)}\label{decl:onepl}\\*
\hline
\bfseries\upshape case & \bfseries\upshape m. & \bfseries\upshape f. & \bfseries\upshape nt.\\
\hline
1. nom. & eke & ek\=a, ek\=ayo & ek\=ani \\
2. acc. & eke & ek\=a, ek\=ayo & ek\=ani \\
3. ins. & ekehi & ek\=ahi & \rdelim{\}}{7}{\linewidth}[as m.] \\
4. dat. & ekesa\d m, ekes\=ana\d m & ek\=asa\d m, ek\=as\=ana\d m & \\
5. abl. & ekehi & ek\=ahi & \\
6. gen. & ekesa\d m, ekes\=ana\d m & ek\=asa\d m, ek\=as\=ana\d m & \\
7. loc. & ekesu & ek\=asu & \\
\hline
\end{longtable}

\begin{longtable}{|p{0.12\linewidth}%
	>{\raggedright\arraybackslash\itshape}p{0.37\linewidth}%
	>{\raggedright\arraybackslash\itshape}p{0.37\linewidth}|}%
\multicolumn{3}{@{}p{0.8\linewidth}@{}}{\small Declension of \pali{dvi} \& \pali{ubha} all genders (only pl.)}\label{decl:two}\\*
\hline
\bfseries\upshape case & \bfseries dvi & \bfseries ubha \\
\hline
1. nom. & dve, duve & ubho, ubhe \\
2. acc. & dve, duve & ubho, ubhe \\
3. ins. & dv\=ihi & ubhohi, ubhehi \\
4. dat. & dvinna\d m, duvinna\d m & ubhinna\d m \\
5. abl. & dv\=ihi & ubhohi, ubhehi \\
6. gen. & dvinna\d m, duvinna\d m & ubhinna\d m \\
7. loc. & dv\=isu & ubhosu, ubhesu \\
\hline
\end{longtable}

\begin{longtable}{|p{0.12\linewidth}%
	>{\raggedright\arraybackslash\itshape}p{0.23\linewidth}%
	>{\raggedright\arraybackslash\itshape}p{0.23\linewidth}%
	>{\raggedright\arraybackslash\itshape}p{0.23\linewidth}|}%
\multicolumn{4}{@{}p{0.8\linewidth}@{}}{\small Declension of \pali{ti} (only pl.)}\label{decl:three}\\*
\hline
\bfseries\upshape case & \bfseries\upshape m. & \bfseries\upshape f. & \bfseries\upshape nt.\\
\hline
1. nom. & tayo & tisso & t\=i\d ni \\
2. acc. & tayo & tisso & t\=i\d ni \\
3. ins. & t\=ihi & t\=ihi & \rdelim{\}}{7}{\linewidth}[as m.] \\
4. dat. & ti\d n\d na\d m, ti\d n\d nanna\d m & tissanna\d m & \\ 
5. abl. & t\=ihi & t\=ihi & \\
6. gen. & ti\d n\d na\d m, ti\d n\d nanna\d m & tissanna\d m & \\
7. loc. & t\=isu & t\=isu & \\
\hline
\end{longtable}

\begin{longtable}{|p{0.12\linewidth}%
	>{\raggedright\arraybackslash\itshape}p{0.23\linewidth}%
	>{\raggedright\arraybackslash\itshape}p{0.23\linewidth}%
	>{\raggedright\arraybackslash\itshape}p{0.23\linewidth}|}%
\multicolumn{4}{@{}p{0.8\linewidth}@{}}{\small Declension of \pali{catu} (only pl.)}\label{decl:four}\\*
\hline
\bfseries\upshape case & \bfseries\upshape m. & \bfseries\upshape f. & \bfseries\upshape nt.\\
\hline
1. nom. & catt\=aro, caturo & catasso & catt\=ari \\
2. acc. & catt\=aro, caturo & catasso & catt\=ari \\
3. ins. & cat\=uhi, catubbhi & cat\=uhi, catubbhi & \rdelim{\}}{5}{\linewidth}[as m.] \\
4. dat. & catunna\d m & cattassanna\d m & \\
5. abl. & cat\=uhi, catubbhi & cat\=uhi, catubbhi & \\
6. gen. & catunna\d m & cattassanna\d m & \\
7. loc. & cat\=usu & cat\=usu & \\
\hline
\end{longtable}

\normalsize\justifying
