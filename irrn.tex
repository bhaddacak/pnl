\chapter{My daughter is wise}\label{chap:irrn}

\phantomsection
\addcontentsline{toc}{section}{Irregular Nouns}
\section*{Irregular Nouns}

As we have gone so far from the beginning, you may realize that at the foundamental level knowing how to decline nouns to intended cases is essential. Most of nouns, adjectives included, in P\=ali are friendly to us. They follow the same pattern according to their ending. Although pronouns use different patterns, we have finite number of them. So, pronouns and regular nouns are quite manageable when you can remember some basic rules. Apart from summarized forms that I give you in the corresponding chapters, I also list all regular paradigms of nominal declension in Appendix \ref{chap:decl}, and paradigms of pronominal declension are in Appendix \ref{decl:pron}. You can consult those tables when you have a certain doubt about declension. That is the way the tradition learns to decline nouns, adjectives, and pronouns.

However, there are a number of nouns that defy regularity. They decline so differently that new students can be baffled. In this chapter we will deal with some of these nouns, just to remind you that you should be aware of this group also. The full list of irregular paradigms is shown in Appendix \ref{decl:irrn}. It is not necessary to bring all of them here.

The reason why we have this group of nouns, I think, is historical one. Some of them are very common in the scriptures, such as, \pali{satthu} (the Buddha), \pali{r\=aja} (king), \pali{pitu} (father), \pali{m\=atu} (mother), \pali{atta} (self), and \pali{mana} (mind). This means these terms are of very old layers of the scriptures which follow very ancient rules.\footnote{Some scholars do not see these as irregularity, but rather another group of stems. For example, Steven Collins says that there are two basic kinds of stem: unchangeable and changeable stems \citep[p.~52]{collins:grammar}. What I call irregular forms are those of changeable stems.} Our job here is to recognize all of them as many as possible. I list several of them in Table \ref{tab:irrn} together with their nom.\ form and the page of paradigm used, so that you can get familiar with them more easily. To use these terms in other specific cases, you have to consult Appendix \ref{decl:irrn} directly. Despite its good coverage, the table is by no means exhaustive. There are endless terms that can be generated on purpose by derivation, markedly by secondary derivation (see Appendix \ref{chap:taddhita}) using \pali{vantu} and \pali{mantu} (see page \pageref{pacct10:vantu}), and by primary derivation (see Appendix \ref{chap:kita}) using \pali{tu} (see page \pageref{pacck1:tu}, also \pageref{pacck11:ratthu}) and \pali{anta} (see page \pageref{pacck10:anta}).

\begin{longtable}[c]{@{}%
	>{\itshape\raggedright\arraybackslash}p{0.2\linewidth}%
	>{\raggedright\arraybackslash}p{0.04\linewidth}%
	>{\itshape\raggedright\arraybackslash}p{0.18\linewidth}%
	>{\raggedright\arraybackslash}p{0.27\linewidth}%
	>{\raggedright\arraybackslash}p{0.1\linewidth}@{}}
\caption{Irregular nouns}\label{tab:irrn}\\
\toprule
\bfseries\upshape Term & \bfseries\upshape G. & \bfseries\upshape Nom. & \bfseries\upshape Meaning & \bfseries\upshape Page \\ \midrule
\endfirsthead
\multicolumn{5}{c}{\tablename\ \thetable: Irregular nouns (contd\ldots)}\\
\toprule
\bfseries\upshape Term & \bfseries\upshape G. & \bfseries\upshape Nom. & \bfseries\upshape Meaning & \bfseries\upshape Page \\ \midrule
\endhead
\bottomrule
\ltblcontinuedbreak{5}
\endfoot
\bottomrule
\endlastfoot
%%
mana\footnote{There are some other words that have some forms like this \pali{mana}-group, but do not count as the group, for example, \pali{p\=ada} (nt., foot), \pali{mukha} (nt., mouth). The forms found are, for instance, \pali{p\=adaso, p\=adas\=a, mukhas\=a}. In Sadd-Pad Ch.\,5, \pali{pila} (nt., pipe, vent) is also added, but I found none of its.} & m. & mano & mind & \pageref{decl:mana} \\
aya & m. & ayo & iron & \pageref{decl:mana} \\
aha & m. & aho & day & \pageref{decl:mana} \\
ura & m. & uro & chest & \pageref{decl:mana} \\
ceta & m. & ceto & mind & \pageref{decl:mana} \\
chanda & m. & chando & prosody, will & \pageref{decl:mana} \\
tapa & m. & tapo & penance & \pageref{decl:mana} \\
tama & m. & tamo & darkness & \pageref{decl:mana} \\
teja & m. & tejo & heat & \pageref{decl:mana} \\
paya & m. & payo & milk & \pageref{decl:mana} \\
yasa & m. & yaso & fame & \pageref{decl:mana} \\
raha & m. & raho & secret place & \pageref{decl:mana} \\
vaca & m. & vaco & word & \pageref{decl:mana} \\
vaya & m. & vayo & age\footnote{If the meaning of \pali{vaya} is used as `decay,' it declines as a regular noun.} & \pageref{decl:mana} \\
sara & m. & saro & pond\footnote{If the meaning of \pali{sara} is used as `sound' or `arrow,' it declines as a regular noun.} & \pageref{decl:mana} \\
sira & m. & siro & the head & \pageref{decl:mana} \\
r\=aja & m. & r\=aj\=a & king & \pageref{decl:raaja} \\
brahma & m. & brahm\=a & the Brahma & \pageref{decl:brahma} \\
sakha & m. & sakh\=a & friend & \pageref{decl:sakha} \\
atta & m. & att\=a & self & \pageref{decl:atta} \\
\=atuma & m. & \=atum\=a & self & \pageref{decl:aatuma} \\
puma & m. & pum\=a & male & \pageref{decl:puma} \\
yuva & m. & yuv\=a & youth & \pageref{decl:yuva} \\
maghava & m. & maghav\=a & the Indra & \pageref{decl:yuva} \\
raha & m. & rah\=a & evil nature\footnote{\pali{Rah\=a vuccati p\=apadhammo} (Sadd-Pad Ch.\,6).} & \pageref{decl:raha} \\
vattaha & m. & vattah\=a & the Indra & \pageref{decl:vattaha} \\
vuttasira & m. & vuttasir\=a & one who shaved & \pageref{decl:vuttasira} \\
addha & m. & addh\=a & path, time & \pageref{decl:addha} \\
muddha & m. & muddh\=a & top, summit & \pageref{decl:muddha} \\
kamma & nt. & kamma\d m & action & \pageref{decl:kamma} \\
s\=a & m. & s\=a & dog & \pageref{decl:saa} \\
assaddh\=a & nt. & assaddha\d m & faithless person & \pageref{decl:assaddhaa} \\
bodhi & f. & bodhi & supreme knowledge\footnote{If \pali{bodhi} denotes a Bo tree, it can be in two genders, m.\ and f. Each declines as regular nouns (Sadd-Pad Ch.\,11).} & \pageref{decl:bodhi} \\
sukhak\=ar\=i\footnote{From \pali{sukhak\=ar\=i} to \pali{sayambh\=u}, they are normally m.\ nouns that decline regularly. But sometimes they are used as adjectives, so in case of nt.\ the ending is shorten to \pali{u}.} & nt. & sukhak\=ari & normally happy doer & \pageref{decl:sukhakaarii} \\
gotrabh\=u & nt. & gotrabhu & \mbox{borderline mind}\footnote{This term is very technical to the Buddhist doctrine, especially the Abhidhamma. It means the borderline between worldly state and transcendent state. It happens when a person is about to be enlightened. The term can be an adjective modifying mind or knowledge. Aggava\d msa discusses \pali{gotrabh\=u} briefly near the end of Sadd-Pad Ch.\,4, ``\pali{Gotrabh\=uti pa\~n\~natt\=aramma\d na\d m \ldots}''} & \pageref{decl:gotrabhuu} \\
abhibh\=u & nt. & abhibhu & \mbox{overcoming mind} & \pageref{decl:gotrabhuu} \\
dhamma\~n\~n\=u & nt. & dhamma\~n\~nu & nature-knowing mind & \pageref{decl:gotrabhuu} \\
sayambh\=u & nt. & sayambhu & \mbox{self-knowing mind} & \pageref{decl:gotrabhuu} \\
go & m. & go & cattle\footnote{When refering to cow (f.) and ox (m.), the term use the same paradigm. For cow, \pali{g\=av\=i} can also be used as a regular noun. However, \pali{g\=av\=i} can also be masculine (Sadd\,225). For ox, \pali{go\d na} with regular declension is an alternative.} & \pageref{decl:go} \\
cittago & nt. & cittagu & dappled cow & \pageref{decl:cittago} \\
\midrule
satthu\footnote{This term and the followings sometimes can be seen in a dictionary as \pali{satthar}. That stem form is never used in traditional textbooks. Aggava\d msa discusses this in Sadd-Pad Ch.\,6 concerning that \pali{satth\=aradassana\d m} is found. He also explains that how \pali{u} becomes \pali{\=ara}.} & m. & satth\=a & teacher, \mbox{the Buddha} & \pageref{decl:satthu} \\
kattu\footnote{This term and the like are formed by primary derivation using \pali{tu} process (see page \pageref{pacck1:tu}).} & m. & katt\=a & doer & \pageref{decl:kattu} \\
akkh\=atu & m. & akkh\=at\=a & preacher & \pageref{decl:kattu} \\
\mbox{abhibhavitu} & m. & \mbox{abhibhavit\=a} & one who overcomes & \pageref{decl:kattu} \\
u\d t\d th\=atu & m. & u\d t\d th\=at\=a & energetic actor & \pageref{decl:kattu} \\
upp\=adetu & m. & upp\=adet\=a & producer & \pageref{decl:kattu} \\
okkamitu & m. & okkamit\=a & one who goes down into & \pageref{decl:kattu} \\
k\=aretu & m. & k\=aret\=a & one who causes to do & \pageref{decl:kattu} \\
khattu & m. & khatt\=a & attendant & \pageref{decl:kattu} \\
khantu & m. & khant\=a & digger & \pageref{decl:kattu} \\
gajjitu & m. & gajjit\=a & roarer & \pageref{decl:kattu} \\
gantu & m. & gant\=a & goer & \pageref{decl:kattu} \\
cetu & m. & cet\=a & collector & \pageref{decl:kattu} \\
chettu & m. & chett\=a & one who cuts & \pageref{decl:kattu} \\
jetu & m. & jet\=a & winner & \pageref{decl:kattu} \\
\~n\=atu & m. & \~n\=at\=a & knower & \pageref{decl:kattu} \\
tatu & m. & tat\=a & spreader & \pageref{decl:kattu} \\
t\=atu & m. & t\=at\=a & protector & \pageref{decl:kattu} \\
d\=atu & m. & d\=at\=a & giver & \pageref{decl:kattu} \\
dh\=atu & m. & dh\=at\=a & holder & \pageref{decl:kattu} \\
nattu & m. & natt\=a & grandson & \pageref{decl:kattu} \\
netu & m. & net\=a & leader & \pageref{decl:kattu} \\
nettu & m. & nett\=a & leader & \pageref{decl:kattu} \\
pa\d tisedhitu & m. & pa\d tisedhit\=a & denier & \pageref{decl:kattu} \\
pa\d tisevitu & m. & pa\d tisevit\=a & pursuer & \pageref{decl:kattu} \\
panattu & m. & panatt\=a & great grandson & \pageref{decl:kattu} \\
pabr\=uhetu & m. & pabr\=uhet\=a & raiser & \pageref{decl:kattu} \\
pucchitu & m. & pucchit\=a & questioner & \pageref{decl:kattu} \\
bhattu & m. & bhatt\=a & husband & \pageref{decl:kattu} \\
bh\=asitu & m. & bh\=asit\=a & sayer & \pageref{decl:kattu} \\
bhettu & m. & bhett\=a & destroyer & \pageref{decl:kattu} \\
bhoddhu & m. & bhoddh\=a & knower & \pageref{decl:kattu} \\
bhodhetu & m. & bhodhet\=a & one who causes to know & \pageref{decl:kattu} \\
metu & m. & met\=a & measurer & \pageref{decl:kattu} \\
mucchitu & m. & mucchit\=a & one who faints & \pageref{decl:kattu} \\
vattu & m. & vatt\=a & speaker & \pageref{decl:kattu} \\
vassitu & m. & vassit\=a & crier, rain & \pageref{decl:kattu} \\
vi\~n\~n\=apetu & m. & vi\~n\~n\=apet\=a & one who causes to know & \pageref{decl:kattu} \\
vinetu & m. & vinet\=a & teacher & \pageref{decl:kattu} \\
sandassetu & m. & sandasset\=a & pointer & \pageref{decl:kattu} \\
sahitu & m. & sahit\=a & endurer & \pageref{decl:kattu} \\
s\=avetu & m. & s\=avet\=a & one who cause to listen & \pageref{decl:kattu} \\
sotu & m. & sot\=a & listener & \pageref{decl:kattu} \\
hantu & m. & hant\=a & killer & \pageref{decl:kattu} \\
pitu & m. & pit\=a & father & \pageref{decl:pitu} \\
c\=ulapitu & m. & c\=ulapit\=a & paternal uncle & \pageref{decl:pitu} \\
bh\=atu & m. & bh\=at\=a & brother & \pageref{decl:pitu} \\
ka\d ni\d t\d tha\-bh\=atu & m. & ka\d ni\d t\d tha\-bh\=at\=a & younger brother & \pageref{decl:pitu} \\
j\=am\=atu & m. & j\=am\=at\=a & son-in-law & \pageref{decl:pitu} \\
je\d t\d thabh\=atu & m. & \mbox{je\d t\d thabh\=at\=a} & elder brother & \pageref{decl:pitu} \\
m\=atu & f. & m\=at\=a & mother & \pageref{decl:maatu} \\
c\=ulam\=atu & f. & c\=ulam\=at\=a & paternal uncle's wife & \pageref{decl:maatu} \\
dh\=itu & f. & dh\=it\=a & daughter & \pageref{decl:maatu} \\
duhitu & f. & duhit\=a & daughter & \pageref{decl:maatu} \\
bh\=atudh\=itu & f. & bh\=atudh\=it\=a & \mbox{brother's daughter} & \pageref{decl:maatu} \\
\midrule
\mbox{gu\d navantu}\footnote{This term and its group can be used as nouns or adjectives, so it can be rendered into three genders. For f., it becomes \pali{gu\d navant\=i} or \pali{gu\d navat\=i} and decline as regular f.\ nouns. Following the tradition, we will never refer to its stem form \pali{gu\d navant}. To be convenient, the meaning I give for this group can be either noun or adjective or both.} & m. & gu\d nav\=a & virtuous person & \pageref{decl:gunavm} \\
gu\d navantu & nt. & gu\d nava\d m & virtuous & \pageref{decl:gunavnt} \\
atthavantu & m. & atthav\=a & beneficial & \pageref{decl:gunavm} \\
katavantu & m. & katav\=a & \mbox{one who has done} & \pageref{decl:gunavm} \\
kulavantu & m. & kulav\=a & one who has a good family & \pageref{decl:gunavm} \\
ga\d navantu & m. & ga\d nav\=a & one who has a following & \pageref{decl:gunavm} \\
th\=amavantu & m. & th\=amav\=a & powerful person & \pageref{decl:gunavm} \\
c\=agavantu & m. & c\=agav\=a & generous person & \pageref{decl:gunavm} \\
cetan\=avantu & m. & cetan\=av\=a & having volition & \pageref{decl:gunavm} \\
dhanavantu & m. & dhanav\=a & wealthy person & \pageref{decl:gunavm} \\
dhitivantu & m. & dhitiv\=a & resolute person & \pageref{decl:gunavm} \\
dhutavantu & m. & dhutav\=a & one practicing austerity & \pageref{decl:gunavm} \\
pa\~n\~navantu & m. & pa\~n\~nav\=a & wise person & \pageref{decl:gunavm} \\
phalavantu & m. & phalav\=a & fruitful person & \pageref{decl:gunavm} \\
balavantu & m. & balav\=a & powerful person & \pageref{decl:gunavm} \\
bhagavantu & m. & bhagav\=a & lucky person & \pageref{decl:gunavm} \\
massuvantu & m. & massuv\=a & having beard & \pageref{decl:gunavm} \\
yatavantu & m. & yatav\=a & careful person & \pageref{decl:gunavm} \\
yasavantu & m. & yasav\=a & glorious person & \pageref{decl:gunavm} \\
\mbox{yasassivantu} & m. & yasassiv\=a & \mbox{glorious person}\footnote{It is in the sense of having retinue: \pali{Yasassino pariv\=arabh\=ut\=a jan\=a assa atth\=iti yasassiv\=a} (Sadd-Pad Ch.\,6).} & \pageref{decl:gunavm} \\
rasmivantu & m. & rasmiv\=a & luminous & \pageref{decl:gunavm} \\
vidvantu & m. & vidv\=a & wise person & \pageref{decl:gunavm} \\
\mbox{vedan\=avantu} & m. & vedan\=av\=a & having feeling & \pageref{decl:gunavm} \\
sa\~n\~n\=avantu & m. & sa\~n\~n\=av\=a & \mbox{having perception} & \pageref{decl:gunavm} \\
\mbox{saddh\=avantu} & m. & saddh\=av\=a & faithful person & \pageref{decl:gunavm} \\
sabb\=avantu & m. & sabb\=av\=a & having all & \pageref{decl:gunavm} \\
s\=ilavantu & m. & s\=ilav\=a & virtuous person & \pageref{decl:gunavm} \\
sutavantu & m. & sutav\=a & learned person & \pageref{decl:gunavm} \\
hitavantu & m. & hitav\=a & beneficial & \pageref{decl:gunavm} \\
himavantu & m. & himav\=a & the Himalaya, having snow & \pageref{decl:himavantu} \\
atthadassi\-mantu & m. & atthadas\-sim\=a & foresighted person & \pageref{decl:himavantu} \\
\=ayasmantu & m. & \=ayasm\=a & aging-well, Venerable\footnote{This term is often used for addressing monks, like `Venerable' used in English. When addressing two monks, we use \pali{\=ayasmant\=a}. More than that, we use \pali{\=ayasmanto}. \pali{Apicettha `\=ayasmant\=a'ti dvinna\d m vattabbavacana\d m, `\=ayasmanto'ti bah\=una\d m vattabbavacananti ayampi viseso veditabbo} (Sadd-Pad Ch.\,6).} & \pageref{decl:himavantu} \\
kalimantu & m. & kalim\=a & sinful person & \pageref{decl:himavantu} \\
kasimantu & m. & kasim\=a & having a plough & \pageref{decl:himavantu} \\
ketumantu & m. & ketum\=a & having a flag & \pageref{decl:himavantu} \\
\mbox{kh\=a\d numantu} & m. & kh\=a\d num\=a & stumpful & \pageref{decl:himavantu} \\
gatimantu & m. & gatim\=a & wise & \pageref{decl:himavantu} \\
gomantu & m. & gom\=a & having cattle & \pageref{decl:himavantu} \\
\mbox{cakkhumantu} & m. & cakkhum\=a & having eyes & \pageref{decl:himavantu} \\
candimantu & m. & candim\=a & the moon & \pageref{decl:himavantu} \\
jutimantu & m. & jutim\=a & radiant & \pageref{decl:himavantu} \\
thutimantu & m. & thutim\=a & praiseful & \pageref{decl:himavantu} \\
dhitimantu & m. & dhitim\=a & resolute & \pageref{decl:himavantu} \\
dh\=imantu & m. & dh\=im\=a & wise & \pageref{decl:himavantu} \\
p\=apimantu & m. & p\=apim\=a & sinful & \pageref{decl:himavantu} \\
puttimantu & m. & puttim\=a & having a child & \pageref{decl:himavantu} \\
balimantu & m. & balim\=a & offerer & \pageref{decl:himavantu} \\
\mbox{bh\=a\d numantu} & m. & bh\=a\d num\=a & luminous & \pageref{decl:himavantu} \\
\mbox{buddhimantu} & m. & buddhim\=a & wise & \pageref{decl:himavantu} \\
matimantu & m. & matim\=a & wise & \pageref{decl:himavantu} \\
mutimantu & m. & mutim\=a & wise & \pageref{decl:himavantu} \\
muttimantu & m. & muttim\=a & wise & \pageref{decl:himavantu} \\
yatimantu & m. & yatim\=a & effortful & \pageref{decl:himavantu} \\
ratimantu & m. & ratim\=a & having pleasure & \pageref{decl:himavantu} \\
r\=ahumantu & m. & r\=ahum\=a & \mbox{eclipsed, the moon} & \pageref{decl:himavantu} \\
rucimantu & m. & rucim\=a & delightful & \pageref{decl:himavantu} \\
vasumantu & m. & vasum\=a & having wealth & \pageref{decl:himavantu} \\
vijjumantu & m. & vijjum\=a & lightningful & \pageref{decl:himavantu} \\
sirimantu & m. & sirim\=a & lucky\footnote{If \pali{sirim\=a} denotes a female name, it decline as regular f.\ nouns.} & \pageref{decl:himavantu} \\
sucimantu & m. & sucim\=a & clean & \pageref{decl:himavantu} \\
setumantu & m. & setum\=a & having a bridge & \pageref{decl:himavantu} \\
hirimantu & m. & hirim\=a & shameful & \pageref{decl:himavantu} \\
hetumantu & m. & hetum\=a & having a cause & \pageref{decl:himavantu} \\
satimantu & m. & satim\=a & mindful person & \pageref{decl:satimantu} \\
\mbox{bandhumantu} & m. & \mbox{bhandhum\=a} & having relatives & \pageref{decl:satimantu} \\
\midrule
gacchanta & m. & gaccha\d m & \mbox{one who is going}\footnote{This m.\ noun with nom.\ \pali{a\d m} ending and its group are described in Sadd-Pad Ch.\,7. They mean one who is doing something. The words are of present participle form, so it can be used as adjectives. To save the space, most meanings are cut short.} & \pageref{decl:gacchanta} \\
kubbanta & m. & kubba\d m & doing & \pageref{decl:gacchanta} \\
caranta & m. & cara\d m & travelling & \pageref{decl:gacchanta} \\
cavanta & m. & cava\d m & moving, dying & \pageref{decl:gacchanta} \\
japanta & m. & japa\d m & reciting & \pageref{decl:gacchanta} \\
jayanta & m. & jaya\d m & winning & \pageref{decl:gacchanta} \\
j\=iranta & m. & j\=ira\d m & aging & \pageref{decl:gacchanta} \\
ti\d t\d thanta & m. & ti\d t\d tha\d m & standing & \pageref{decl:gacchanta} \\
dadanta & m. & dada\d m & giving & \pageref{decl:gacchanta} \\
pacanta & m. & paca\d m & cooking & \pageref{decl:gacchanta} \\
bhu\~njanta & m. & bhu\~nja\d m & eating & \pageref{decl:gacchanta} \\
mahanta & m. & maha\d m & worshiping & \pageref{decl:gacchanta} \\
m\=iyanta & m. & m\=iya\d m & dying & \pageref{decl:gacchanta} \\
vajanta & m. & vaja\d m & going & \pageref{decl:gacchanta} \\
saranta & m. & sara\d m & remembering & \pageref{decl:gacchanta} \\
su\d nanta & m. & su\d na\d m & listening & \pageref{decl:gacchanta} \\
\midrule
gu\d navat\=i & f. & gu\d navat\=i & virtuous person & \pageref{decl:gunavf}\footnote{This paradigm is in fact like regular \pali{\=i}-ending f.} \\
gu\d navant\=i & f. & gu\d navant\=i & virtuous person & \pageref{decl:gunavf} \\
gacchant\=i & f. & gacchant\=i & \mbox{one who is going} & \pageref{decl:gunavf} \\
bhavanta & m. & bhava\d m & \mbox{prosperous person} & \pageref{decl:bhavanta} \\
karonta & m. & kara\d m & \mbox{one who is doing} & \pageref{decl:karonta} \\
arahanta & adj. & araha\d m & \mbox{worth venerating} & \pageref{decl:arahanta} \\
arahanta & m. & arah\=a & arhant & \pageref{decl:arahanta} \\
santa & m. & sa\d m & \mbox{righteous person}\footnote{The feminine form of this is \pali{sat\=i}, declining as regular nouns. The term can be negated as \pali{asa\d m} and decline likewise.} & \pageref{decl:santa1} \\
santa & adj. & santo & existing & \pageref{decl:santa2} \\
mahanta & m. & maha\d m, mah\=a & great & \pageref{decl:mahanta} \\
\end{longtable}

Now let us consider our heading task, to say ``My daughter is wise.'' We have two common irregular terms here, \pali{dh\=itu} (daughter) and \pali{pa\~n\~navantu} (having wisdom, wise). You can use other terms that means the same. But, as far as I know, they are also as irregular as these. So, let us do with the commonest terms. Considering the cases to use, in this sentence they are all nom. Then it goes simply as follows:

\palisample{mama dh\=it\=a pa\~n\~navat\=i hoti.\sampleor mama dh\=it\=a pa\~n\~navant\=i hoti.}

Since the main noun is feminine, \pali{pa\~n\~nava(n)t\=i} is used here, following the regular f.\ paradigm like \pali{gu\d nava(n)t\=i}. For the possessive pronoun, when we use its enclitic form (see Chapter \ref{chap:pron-person}, page \pageref{par:enclitic}), it goes as follows:

\palisample{dh\=it\=a me pa\~n\~navat\=i hoti.}

I show this to remind you that when the short form of pronouns is used, it never occupies the first position of the sentence, and it has to associate with other term somehow. We often find this use in the scriptures, because it is really handy to use. The downside of this is it increases ambiguity, because enclitic forms can be used in several cases, and they can mean other things as well.

Another example for a male noun is ``My younger brother is wise.'' We can say this as follows:

\palisample{ka\d ni\d t\d thabh\=at\=a me pa\~n\~nav\=a hoti.}

And here is for its plural version: ``My younger brothers are wise.''

\palisample{ka\d ni\d t\d thabh\=ataro me pa\~n\~navanto honti.\sampleor \ldots pa\~n\~navant\=a honti.}

Now let make it more complex by saying ``My smart daughter has useful books.'' Ready, here we go.

\palisample{mama pa\~n\~navatiy\=a dh\=itu atthavant\=a(ni) potthak\=a(ni) santi.\sampleor \ldots dh\=ituy\=a atthavant\=a potthak\=a santi.\sampleor \ldots dh\=itussa atthavant\=a potthak\=a santi.}

You can also use \pali{hitavant\=a(ni)} for `useful.' It has the same meaning. Now you can talk about your family members. Several of them are irregular nouns. Here is another example: ``I have foreign parents.'' In P\=ali the word `parent' is in compound form, \pali{mat\=apitu} (mother and father). The word declines as \pali{pitu} but only plural. For `foreign' we use \pali{vides\=i} or \pali{videsika}. Then we get this, for instance:

\palisample{mama m\=at\=apitaro videsik\=a honti.}

Our exercise is not so hard. Let us do it.

\section*{Exercise \ref{chap:irrn}}
Say these in P\=ali using nouns and adjectives listed in the table above, if available. For declensional paradigms, consult Appendix \ref{decl:irrn}.
\begin{compactenum}
\item This beautiful moon is luminous. 
\item Your generous mother is faithful.
\item This young king has virtuous mind.
\item A (male) friend of my elder brother is rich.
\item My (maternal) aunt's husband is powerful.
\end{compactenum}
