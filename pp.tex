\chapter{\headhl{Having gone to town}, I bought you a book}\label{chap:pp}

\phantomsection
\addcontentsline{toc}{section}{Introduction to Past Participles}
\section*{Introduction to Past Participles}

If you feel that P\=ali past verbs are hard to deal with, here is good news. As verbal \pali{kita}, verbs in \pali{ta} form can do the same job equally (see Appendix \ref{chap:kita}, page \pageref{pacck8:ta} for more information). They are relatively easier to render, although some irregular forms have to be remembered. And they are very handy to use, versatile like a Swiss army knife. They can be used in all kinds of structure: active, passive, causative, etc. They can also be used as a noun or modifier.\footnote{Vito Perniola has a very good summary of how past participles are used \citep[see][pp.~360--7]{perniola:grammar}.} That is why \pali{ta} form is extensively used in the scriptures. Scholars call these \emph{past participles}. The name does not fit well, because it can do more than that, but we use it nonetheless. In this chapter our main focus is on active structure. For more about passive, see Chapter \ref{chap:pass}; and for causative, see Chapter \ref{chap:caus}.

In principle \pali{ta} can be used in active structure\footnote{Kacc\,626, R\=upa\,634, Sadd\,1233. In the formulas, \pali{kta} is mentioned. The actual \pali{paccaya} is \pali{ta}, but \pali{k-anubandha} is given to stress that no \pali{vuddhi} is applied.}, also in passive structure as both transitive or intransitive verb\footnote{Kacc\,625, R\=upa\,605, Sadd\,1232} (see Chapter \ref{chap:pass} for explanation). When used as intransitive verb (called impersonal passive), the verbs take neuter gender. We often find that only verbs in \pali{ta} appear in a sentence without a main verb. This means, as P\=ali teachers tell us, \pali{ta} can finish sentences like a normal verb. No participle is supposed to do likewise in English. I summarize how to use \pali{ta} as a guideline below:

\begin{enumerate}
\item Choose a verb to use by its root, or its present form. Apply \pali{ta} to it. Be aware of its irregular form.
\item Determine the doer of the verb. Be aware of its gender and number.
\item Decline the \pali{ta} verb corresponding to gender and number of the doer. For example, in nominative case the term's ending will be \pali{to/t\=a} (m.), \pali{t\=a/t\=ayo} (f.), and \pali{ta\d m/ t\=ani} (nt.). 
\item Compose all components in a proper order.
\end{enumerate}

Here are some simple examples adapted from textbooks:

\begin{quote}
\pali{d\=ana\d m dinno kum\=aro.}\\
``A boy gave alms.''\\[1.5mm]
\pali{d\=ana\d m dinn\=a kum\=ar\=a.}\\
``Boys gave alms.''\\[1.5mm]
\pali{d\=ana\d m dinn\=a kum\=ar\=i.}\\
``A girl gave alms.''\\[1.5mm]
\pali{d\=ana\d m dinna\d m.}\\
``Alms is given.''\\[1.5mm]
\pali{d\=ana\d m dinna\d m kum\=arena.}\\
``Alms is given by a boy.''\\[1.5mm]
\pali{d\=ana\d m dinna\d m kum\=ariy\=a.}\\
``Alms is given by a girl.''\\[1.5mm]
\pali{sayita\d m kum\=arena.}\\
``Sleeping was done by a boy.''\\[1.5mm]
\pali{sayita\d m sayana\d m kum\=arena.}\\
``A bed has been slept (on) by a boy.''\\[1.5mm]
\pali{pacita\d m s\=udena.}\\
``Cooking was done by a chef.''\\[1.5mm]
\pali{pacito odano s\=udena.}\\
``Rice has been cooked by a chef.''\\[1.5mm]
\pali{anusi\d t\d tho so may\=a}\\
``He was taught by me.''\\[1.5mm]
\pali{di\d t\d tha\d m me r\=upa\d m}\\
``An image was seen by me.''\\[1.5mm]
\end{quote}

In these examples, you can also see \pali{ta} form as modifier, so you treat the sentences like those with verb `to be' left out. Hence, for example, \pali{d\=ana\d m dinno kum\=aro (hoti)} can be translated as ``A boy is one who gave alms.'' In English the two ways of reading are not exactly the same, but in P\=ali the sense is identical.

Now you can feel more comfortable with past tense in P\=ali. To ease the use, you has to master variation of \pali{ta} form first, see page \pageref{sec:irrprod} for more detail. In our vocabulary verbs in \pali{ta} form are also given, see page \pageref{vocab:verb}.

Now we can do half of our heading task, ``I bought you a book.'' We find that verb `to buy' is \pali{k\=i} by root or \pali{k\=i\d n\=ati} by present form. Its \pali{ta} form is \pali{k\=ita}. Then we get this:

\palisample{aha\d m tuyha\d m potthaka\d m k\=ito. \sampleor[or, if the speaker is female] aha\d m tuyha\d m potthaka\d m k\=it\=a.}

For more understanding, we have to learn instances from the canon.

\begin{quote}
\pali{Te cittakath\=a \textbf{bahussut\=a}, Kome gotamas\=avak\=a \textbf{gat\=a}}\footnote{S1\,224 (SN\,9). In this, \pali{bahussuta} is used as a noun meaning literally one who has listened a lot.}\\
``They are brilliant speakers [and] very learned. Where did these disciples of Gotama go?''\\[1.5mm]
\pali{Amh\=aka\d m pana saki\d m \textbf{kat\=ani} santhat\=ani pa\~ncapi chapi vass\=ani honti}\footnote{Buv1\,557}\\
``There are our mats that were made once, [lasted for] 5--6 years.''\\[1.5mm]
\end{quote}

Apart from \pali{ta}, in rare occasions we find that \pali{t\=av\=i} and \pali{tavantu} can also be used in past meaning, but only in active structure. Here are some examples from the canon:

\begin{quote}
\pali{Bhikkh\=u \textbf{bhutt\=av\=i} pav\=arit\=a \~n\=atikul\=ani gantv\=a ekacce bhu\~nji\d msu ekacce pi\d n\d dap\=ata\d m \=ad\=aya agama\d msu.}\footnote{Buv2\,236}\\
``Having eaten and been satisfied, [then] having gone to relative families, some monks ate [again], some monks, having taken [other] food, went.''\\[1.5mm]
\pali{Yo hoti bhikkhu araha\d m \textbf{kat\=av\=i}, Kh\=i\d n\=asavo antimadehadh\=ar\=i;}\footnote{S1\,25 (SN\,1)}\\
``Which monk made [himself] an arhant, free from mental obsessions, [just] the holder of the final body.''\\[1.5mm]
\pali{Turiyehi ma\d m bh\=arata \textbf{bhuttavanta\d m}}\footnote{Ja\,17:167. Verbs in \pali{tavantu} is extremely hard to find. When these are used, they decline irregularly like \pali{gu\d navanta} (see page \pageref{decl:gunavm}).}\\
``Bh\=arata, [those women please] me, who had eaten, with musical instruments ''\\[1.5mm]
\end{quote}

Like \pali{anta} and \pali{m\=ana} (see Chapter \ref{chap:prp}), in relative clauses \pali{ta} can be used to denote past events, for example:

\begin{quote}
\pali{Tassa ta\d m \=av\=asa\d m \textbf{gatassa} eva\d m hoti}\footnote{Mv\,7.323}\\
``When that [monk] went to that temple, [a thought] arises thus \ldots''\\
\end{quote}

Past participles can appear along side with present participles. This can give us a sense of sequential events, like this example:

\begin{quote}
\pali{Tena kho pana samayena bhagav\=a mahatiy\=a paris\=aya \textbf{parivuto} dhamma\d m desento \textbf{nisinno} hoti.}\footnote{Buv1\,24}\\
``By that occasion, there is the Buddha, surrounded by a mass of people, having sat down, preaching the Dhamma.''\\
\end{quote}

By the previous example, now you have an idea how to finish our heading task. You can use \pali{ta} in adjective clauses. So, we get this for ``Having gone to town, I bought you a book'' (suppose the speaker is a male).

\palisample{aha\d m nagara\d m gato tuyha\d m potthaka\d m k\=ito.}

That makes sense, but it is not the best way to do if you want to show the succession of events. In P\=ali a more suitable thing to do the job exists.

\clearpage
\phantomsection
\addcontentsline{toc}{section}{Introduction to Absolutives}
\section*{Introduction to Absolutives}

Here I will not explain, in grammatical terms, what `absolutive' means, because it is likely to make things more confusing. I just use this as most scholars do to call verbal \pali{kita} in form of \pali{tv\=a, tv\=ana,} and \pali{tuna (t\=una)}.\footnote{See \citealp[p.~114]{collins:grammar} for some explanation. A.\,K.\,Warder calls this \emph{gerund} \citep[p.~48]{warder:intro}. That makes us a little more confused.} I will more often call these verbs in \pali{tv\=a} form, because this form is mostly seen. This verb form works like participles but with a different implication. So, sometimes I call it roughly a participle too. Fortunately for students, this verb form stays intact when used like indeclinables, but you have to remember some irregular forms of it anyway (see page \pageref{sec:irrprod}, and \pali{tv\=a} forms are also given in our vocabulary, see page \pageref{vocab:verb}).

The main use of this is to mark a prior action, or sometimes a simultaneous action, and a successive action, of the main verb. This gives us a sense of sequence. By using this, we will know what happens successively. For more information, see Appendix \ref{chap:kita}, page \pageref{pacck9:tuna}. Let us see a real example:

\begin{quote}
\pali{Atha kho bhagav\=a kumbhak\=ar\=avesana\d m \textbf{pavisitv\=a} \\ekamanta\d m ti\d nasanth\=araka\d m \textbf{pa\~n\~n\=apetv\=a} nis\=idi \\palla\.nka\d m \textbf{\=abhujitv\=a} uju\d m k\=aya\d m \textbf{pa\d nidh\=aya} \\parimukha\d m sati\d m \textbf{upa\d t\d thapetv\=a}.}\footnote{M3\,342 (MN\,140)}\\
``Then the Blessed One, having entered the potter's workshop, having spread a mat of grass on one side, sat down, crossing the legs, keeping the body straight, keeping the mindfulness alert.''\\
\end{quote}

In the example above, the main aorist verb is \pali{nis\=idi} (sat down). Other \pali{tv\=a} verbs give us a series of pictures like a scene in a movie. We see an irregular form here, \pali{pa\d nidh\=aya}.\footnote{Its present verb is \pali{pa\d nidahati}. And its normal absolutive form, \pali{pa\d nidahitv\=a}, can also be used.} Let us see another good example: 
 
\begin{quote}
\pali{Atha kho s\=a paris\=a bhagavat\=a dhammiy\=a kath\=aya sandassit\=a sam\=adapit\=a samuttejit\=a sampaha\d msit\=a \textbf{u\d t\d th\=ay\=a}san\=a bhagavanta\d m \textbf{abhiv\=adetv\=a} padakkhi\d na\d m \textbf{katv\=a} pakk\=ami.}\footnote{Buv1\,24}\\
``That mass of people, having been explained, encouraged, instigated, and delighted by the religious speech of the Buddha; having risen from the seat, bowed down to the Buddha, circumambulated him, then went away.''\\
\end{quote}

In this example, you can see that how \pali{ta} and \pali{tv\=a} work together. The subject of the sentence is \pali{paris\=a}, and the main verb in aorist is \pali{pakk\=ami}. So, the main idea of this sentence is just ``people went away.'' Between the subject and verb, there are clauses of participles, both in \pali{ta} (\pali{sandassit\=a} \ldots \pali{sampaha\d msit\=a}) and \pali{tv\=a} (\pali{u\d t\d th\=aya}\footnote{This is an absolutive form of \pali{u\d t\d th\=apeti} (\pali{u\d t\d th\=ay\=asan\=a} = \pali{u\d t\d th\=aya + \=asan\=a} [having risen from the seat]). A more straight form of this is \pali{u\d t\d th\=apetv\=a}.} \ldots \pali{katv\=a}). In \pali{ta} group, they are used in passive voice, marked by instrumental case of \pali{bhagavat\=a} (see Chapter \ref{chap:pass} for why ins.\ has a thing to do with passive voice).

The key difference between \pali{ta} and \pali{tv\=a} clause is the latter gives us a sense of order. We can see actions run successively in \pali{tv\=a} clauses. On the other hand, in \pali{ta} clauses each verb shows a different aspect of the same thing. All those qualities can happen at the same time, or regardless of order, in the past. Another difference to keep in mind is that \pali{tv\=a} cannot end sentences, like \pali{ta}.

Now for our heading task, ``Having gone to town, I bought you a book,'' we can use \pali{tv\=a} to show the sequence of event as follows:

\palisample{aha\d m nagara\d m gantv\=a tuyha\d m potthaka\d m ki\d ni.}

This means I bought the book after I went to town. Verbs in \pali{tv\=a} form are by no means limited to past actions. They can be used with present tense as well. For example, in ``Going to town, I buy you a book,'' we can put it likewise:

\palisample{aha\d m nagara\d m gantv\=a tuyha\d m potthaka\d m ki\d n\=ami.}

This has a better sense than using present participles, like ``\pali{aha\d m nagara\d m gacchanto(t\=a) tuyha\d m potthaka\d m ki\d n\=ami.}'' Because using present participles can mean that I buy the book on the way of going, not at the town.

Verbs in \pali{tv\=a} form can even be used in future events. So, you can say ``Going to town, I will buy you a book'' as follows:

\palisample{aha\d m nagara\d m gantv\=a tuyha\d m potthaka\d m ki\d niss\=ami.\sampleor[or, comparing to]aha\d m nagara\d m gacchanto(t\=a) tuyha\d m potthaka\d m ki\d niss\=ami.}

\section*{Exercise \ref{chap:pp}}
Say these in P\=ali. Aorist verbs are not allowed to use.
\begin{compactenum}
\item What's wrong with your car, customer sir?
\item While I was coming here, the engine stopped several times.
\item Having driven recently, did you have any accident or any unusual use?
\item Not a serious one, having gone to a mountain, I drove it over some streams.
\item It is not suitable for your car in such a situation. Your car needs an overhaul checking.
\item That will cost me a lot. Why don't you just make it run normally. It's obviously about the engine, isn't it?
\item In that case, sir, I will raise the engine out of your car, clean it up inside and outside, put it back, and make it run.
\item That means I have to pay you a lot anyway.
\item It is our service, customer sir.
\end{compactenum}
