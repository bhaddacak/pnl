\chapter{You are \headhl{the best}}\label{chap:adjcomp}

\phantomsection
\addcontentsline{toc}{section}{Adjective Comparison}
\section*{Adjective Comparison}

In Chapter \ref{chap:abl}, we touch upon adjective comparison using ablative case. For example, when you want to say ``My sister is more beautiful than that girl,'' you have to rephrase it to ``My sister is beautiful from that girl.'' Then we get this in P\=ali:

\palisample{et\=aya ka\~n\~n\=aya mama bhagin\=i sundar\=a hoti.}

A simple alternative of this is to use \palibf{uttara} (higher, over), for example:

\palisample{es\=a ka\~n\~n\=a sundar\=a, mama bhagin\=i (pana) uttar\=a hoti.}

That is a way to say ``That girl is beautiful, (but) my sister is more (beautiful).'' Another alternative is to add some endings to the adjective to make it in comparative degree. The endings are \palibf{-tara}, \palibf{-iya}, and \palibf{-isika}. So, `more beautiful' becomes \pali{sundaratara}, \pali{sundariya} and \pali{sundarisika}. Hence we get this:

\palisample{\ldots, mama bhagin\=i sundaratar\=a hoti.\sampleor \ldots, mama bhagin\=i sundariy\=a hoti.\sampleor \ldots, mama bhagin\=i sundarisik\=a hoti.}

How to say ``My sister is the most beautiful'' then? In the way of \pali{uttara}, we can use \palibf{uttama} (highest, best) in superlative degree. So, we can say it like this:

\palisample{mama bhagin\=i uttam\=a sundar\=a hoti.}

Alternatively, you can use \pali{anuttara} (incomparable, unsurpassed), the negation of \pali{uttara}, in the same meaning. Hence we equally get ``\pali{\ldots anuttar\=a sundar\=a hoti.}'' Yet another alternative is to add superlative endings to the adjective. The endings in this case are \palibf{-tama} and \palibf{-i\d t\d tha}. So, we can equally say like this:

\palisample{mama bhagin\=i sundaratam\=a hoti.\sampleor \ldots sundari\d t\d th\=a hoti.}

As we have gone so far, we can finish our heading task: ``You are the best'' simply as:

\palisample{tva\d m uttamo/uttam\=a hosi.\sampleor tva\d m anuttaro/anuttar\=a hosi.}

Much like English, `better' and `best' are widely used in P\=ali as \palibf{seyya} (better) and \palibf{se\d t\d tha} (best). Using these as adjectives, you can say ``You are the best'' as:

\palisample{tva\d m se\d t\d tho/se\d t\d th\=a hosi.}

Often used as an indeclinable, \pali{seyyo} can be used with all genders. Here are examples from the canon.

\begin{quote}
\pali{Seyyo amitto medh\=av\=i, ya\~nce b\=al\=anukampako;}\footnote{Ja\,1:45} \\
``It is better to have a wise enemy than a foolish compassionate one.''\\[1.5mm]
%
\pali{Es\=ava p\=ujan\=a seyyo}\footnote{Sadd-Pad Ch.\,5; in Dhp\,8.106, it is ``\pali{S\=ayeva p\=ujan\=a seyyo}.''} \\
``One [moment of] homage is better.''\\[1.5mm]
%
\pali{Ek\=aha\d m j\=ivita\d m seyyo, s\=ilavantassa jh\=ayino.}\footnote{Dhp\,8.110} \\
``One-day life of a meditating virtue-holder is better.''
\end{quote}

Before we end this chapter, there is something worth noting here. We can see that certain suffixes can modify meaning of terms, particular nouns and adjectives. In Chapter \ref{chap:ind-intro} we call these \pali{paccaya}. This way of word formation is central to P\=ali grammar. As we have seen from the start, we have learned to compose words into sentences by adding \pali{vibhatti}, a special kind of \pali{paccaya}. Verbs also have their own set of \pali{paccaya/vibhatti} to make them function variously.

This chapter remind us to another category of word formation called \emph{secondary derivation}. This happen to nouns and adjectives like we add \pali{-tara} or \pali{-tama} to adjectives and make them comparative and superlative respectively (see Appendix \ref{chap:taddhita}, page \pageref{par:visesataddhita}). This type of words, like compounds, is quite a big deal in P\=ali grammar because all textbooks have a big chapter for it. I do not incorporate this to our main lessons, for it is too technical to know at the beginning stage. However, knowing this widens your understanding on vocabulary significantly. So, I add it as an appendix. For those who are curious, please see Appendix \ref{chap:taddhita}.

As you might guess, \pali{seyya} and \pali{se\d t\d tha} have something to do with \pali{-iya} and \pali{-i\d t\d tha}, but in a somewhat irregular way.\footnote{It is said that the base word is \pali{pasattha} (praised). When the \pali{paccaya}s is in the process, the whole word becomes just \pali{sa}. For more detail, see Kacc\,262--8, R\=upa\,391--7, Sadd\,511--8, Mogg\,4.135--8, Niru\,555--8.} There are some others that behave in the same way. I summarize these in Table \ref{tab:irrcomp}.

\bigskip
\begin{longtable}[c]{@{}>{\itshape\raggedright\arraybackslash}p{0.17\linewidth}%
	>{\raggedright\arraybackslash}p{0.17\linewidth}%
	>{\itshape\raggedright\arraybackslash}p{0.1\linewidth}%
	>{\itshape\raggedright\arraybackslash}p{0.16\linewidth}%
	>{\itshape\raggedright\arraybackslash}p{0.18\linewidth}@{}}
\caption{Irregular comparative forms}\label{tab:irrcomp}\\
\toprule
\bfseries\upshape Base & \bfseries Meaning & \bfseries\upshape Use & \bfseries\upshape Paccaya & \bfseries\upshape Outcome \\ \midrule
\endfirsthead
\multicolumn{5}{c}{\tablename\ \thetable: Irregular comparative forms (contd\ldots)}\\
\toprule
\bfseries\upshape Base & \bfseries Meaning & \bfseries\upshape Use & \bfseries\upshape Paccaya & \bfseries\upshape Outcome \\ \midrule
\endhead
\bottomrule
\ltblcontinuedbreak{5}
\endfoot
\bottomrule
\endlastfoot
%
vu\d d\d dha & old & ja & iya & jeyya \\
vu\d d\d dha & old & ja & i\d t\d tha & je\d t\d tha \\
pasattha & praised & sa & iya & seyya \\
pasattha & praised & sa & i\d t\d tha & se\d t\d tha \\
antika & near & neda & iya & nediya \\
antika & near & neda & i\d t\d tha & nedi\d t\d tha \\
b\=a\d lha & strong & s\=adha & iya & s\=adhiya \\
b\=a\d lha & strong & s\=adha & i\d t\d tha & s\=adhi\d t\d tha \\
appa & small & ka\d n & iya & ka\d niya \\
appa & small & ka\d n & i\d t\d tha & ka\d ni\d t\d tha \\
yuva & young & ka\d n & iya & ka\d niya\footnote{In Kacc\,267, it is \pali{kaniya} (and \pali{kani\d t\d tha} for \pali{-i\d t\d tha}). In Mogg\,4.137, it can be in both ways.} \\
yuva & young & ka\d n & i\d t\d tha & ka\d ni\d t\d tha \\
gu\d navantu & virtuous & gu\d na\footnote{For words with \pali{-vantu} ending, delete the ending.} & iya & gu\d niya \\
gu\d navantu & virtuous & gu\d na & i\d t\d tha & gu\d ni\d t\d tha \\
satimantu & mindful & sati\footnote{For words with \pali{-mantu} ending, delete the ending.} & iya & satiya \\
satimantu & mindful & sati & i\d t\d tha & sati\d t\d tha \\
medh\=av\=i & wise & medh\=a\footnote{For words with \pali{-v\=i} ending, delete the ending.} & iya & medhiya \\
medh\=av\=i & wise & medh\=a & i\d t\d tha & medhi\d t\d tha \\
\end{longtable}

From the table, now you have learned that adjectives ending with \pali{vantu} and \pali{mantu} as we met in Chapter \ref{chap:irrn} also have irregular comparative and superlative form. Also being formed as secondary derivative, words with \pali{v\=i} ending (see page \pageref{pacct10:vii}) are normally used as a regular noun. When being used as adjectives in comparison, they become irregular. To see a clearer picture, let us do some examples. Here is for ``That man is richer than me.''

\palisample{so puriso may\=a dhaniyo hoti.\sampleor so puriso may\=a vasaviyo hoti.}

The dictionary form of `rich' is \pali{dhanavantu}. We remove the \pali{vantu} ending and add \pali{iya} to it. Hence we get \pali{dhaniya}. Another term in the same meaning with \pali{mantu} ending is \pali{vasumantu}. Then we get \pali{vasu+iya} $\rightarrow$ \pali{vasav+iya} = \pali{vasaviya}.\footnote{This is a typical way to connect different vowels. One side has to be changed to \pali{gu\d na} strength (see the end of Chapter \ref{chap:nuts}). Thus, to maintain \pali{iya}, the preceding \pali{u} is changed to \pali{av}. The result term is not found in any text, let alone in a dictionary. So, it is better to avoid using uncommon terms, unless you provide your own glossary.} These terms are used as normal adjectives. Therefore, ``That woman is richer than me'' is ``\pali{s\=a itth\=i may\=a dhaniy\=a/vasaviy\=a hoti}.'' For `than me' we use ablative case, thus \pali{may\=a}. And here is for ``That man is the richest.''

\palisample{so puriso dhani\d t\d tho hoti.\sampleor so puriso vasavi\d t\d tho hoti.}

Another example is ``You are wiser than me.'' Here is its P\=ali:

\palisample{tva\d m may\=a medhiyo/medhiy\=a hosi.\sampleor tva\d m may\=a pa\~n\~niyo/pa\~n\~niy\=a hosi.\sampleor tva\d m may\=a gatiyo/gatiy\=a hosi.}

And here is ``You are the wisest.''

\palisample{tva\d m may\=a medhi\d t\d tho/medhi\d t\d th\=a hosi.\sampleor tva\d m may\=a pa\~n\~ni\d t\d tho/pa\~n\~ni\d t\d th\=a hosi.\sampleor tva\d m may\=a gati\d t\d tho/gati\d t\d th\=a hosi.}

That seems enough for a guideline to adjective comparison. But how do we say that two things have equal quality? A simple way is to use \palibf{sadisa} (equal), or its adverb form \palibf{sadisa\d m}.\footnote{For derivation of \pali{sadisa} see page \pageref{par:kiidisa}. For more about adverb, see Chapter \ref{chap:adv}.} For example, to say ``You are as rich as I am,'' you have to rephrase the sentence to ``You and I are equal(ly) rich,'' hence:

\palisample{tva\d m aha\d m ca sadis\=a dhanavanto homa.\sampleor tva\d m aha\d m ca sadisa\d m dhanavanto homa.}

As it is implied by \pali{ca}, \pali{sadisa} can even be left out. So, you just say ``You and I are rich.''

\palisample{tva\d m aha\d m ca dhanavanto homa.}

Other variation of using \pali{sadisa} is to use with instrumental case. So, the sentence is rephrased to ``You are equal by wealth to me'':

\palisample{tva\d m me dhanena s\=adiso/s\=adis\=a hosi.}

Another term that can help you emphasize the equality is \palibf{sama} (equal, even). Then you can also say this:

\palisample{tva\d m aha\d m ca sam\=a dhanavanto homa.}

A more fashionable way of using \pali{sama} is to use with instrumental case. So, you can also put it as follows:

\palisample{tva\d m aha\d m ca dhanena sam\=a homa.}

This can be rendered as ``You and I are equal by wealth.'' You can also use \palibf{sama\d m} as an adverb. For example, ``You and I run equally by speed'' can be put like this:

\palisample{tva\d m aha\d m ca vegena/javena sama\d m dh\=av\=ama.}

In this sentence we use \pali{vega} or \pali{java} (speed) in instrumental case with \pali{sama\d m} (see also Appendix \ref{chap:nipata}, page \pageref{nip:samadm}). Alternatively, you can say ``You and I run equally fast'' (\pali{tva\d m aha\d m ca sama\d m s\=igha\d m dh\=av\=ama}). But it is not the time to talk about adverbs now.

\section*{Exercise \ref{chap:adjcomp}}
Say these in P\=ali. For unknown words, see in the vocabulary or in a dictionary.
\begin{compactenum}
\item I am luckier than you, but that man is the luckiest.
\item My elder brother is stronger\footnote{Use \pali{balavantu}.} than me. I am younger than him.
\item That thin pig is heavier that the fattest cat.
\item A mindful moment is the most precious time in our life.
\item P\=ali is easier (to learn) by conversation than by reading.
\end{compactenum}
