\chapter[\headhl{Even though} this book is difficult]{\headhl{Even though} this book is difficult, it is pleasurable to read}\label{chap:pi}

\phantomsection
\addcontentsline{toc}{section}{Concessive Clauses}
\section*{Concessive Clauses}

In this chapter we will exercise our knowledge so far to say the heading above. The main focus here is indeed on concession. Let us do it step by step.

First, what is a concessive clause, anyway? For I am not a linguist, I quote a definition of \emph{concessive} in full:

\begin{quote}
A concessive is an adverbial clause of concession, or a preposition such as \emph{despite}, or a discourse particle such as \emph{though}, signalling that the speaker is conceding some point while maintaining another: \textit{Despite the traffic jams, we reached the airport on time; John is clever -- he's not very hard-working though.}\footnote{\citealp[p.~96]{brownmiller:dict}}
\end{quote}

What particle should we use in this sentence, then? There are some that can be used in contrasting, e.g.\ \pali{ca, pana,} and \pali{(a)pi} (see Appendix \ref{chap:nipata}). All these particles have more than one specific use. However, by rule of thumb we find that the most suitable particle in this situation is \pali{(a)pi}. Because most of the time when we meet \pali{(a)pi}, the sense of `even' can be felt somehow. That is why I put `even' in the sentence, although it looks a little redundant (`though' alone can get the job done).

In general use, \pali{pi} emphasizes the meaning of the preceding term, like `even' does to its immediate follower. Let us look at an example from the canon:

\begin{quote}
\pali{aham\textbf{pi} kho, bhikkhu, na j\=an\=ami, yatthime catt\=aro mah\=abh\=ut\=a aparises\=a nirujjhanti}\footnote{D1\,491 (DN\,11)}\\
``\textbf{Even} I, monk, do not know where these four great elements completely cease.''\\{[or]}\\
``I, monk, \textbf{still} do not know \ldots''\\{[or]}\\
``I, monk, \textbf{indeed} do not know \ldots''\\{[or]}\\
``I, monk, do not know \textbf{so much as} \ldots''\\
\end{quote}

This sentence is not yet a concession because there is no contrasting point. To make a consession, we stress one idea over another, for example:

\begin{quote}
\pali{chinno\textbf{pi} rukkho punareva r\=uhati}\footnote{Dhp\,24.338}\\
``Even being cut, a tree grows again.''\\
\end{quote}

In the above P\=ali sentences, \pali{pi} is used like an adverb. To use \pali{pi} likewise in our task, we have to rephrase our task to ``Even being difficult, the book is pleasurable to read.'' This is easier than the actual heading, so we should tackle this first. For other key terms, I will use \pali{manu\~n\~na} for `pleasurable,' \pali{kiccha} for `difficult,' and \pali{pa\d than\=aya} (dat.) for `to read.' And I use \pali{potthako} (m.) for `book.' Here we go:

\palisample{kiccho pi manu\~n\~na\d m pa\d than\=aya aya\d m potthako hoti.}

Note that, \pali{manu\~n\~na\d m} is used as an adverbial accusative (see Chapter \ref{chap:adv}). Alternatively, we can also use \pali{pana} or \pali{ca} in this sentence instead of \pali{pi}, hence:

\palisample{kiccho pana/ca manu\~n\~na\d m pa\d than\=aya aya\d m potthako hoti.}

Roughly speaking, this P\=ali sentence can be an equivalent to the heading, even though they use different structure. To make them agreeable in structure, we have to make our P\=ali sentence complex. A typical way to do this is to use \pali{ya-ta} structure. Thus we rephrase our heading to ``Which book here is difficult, that [one] is contrastingly pleasurable to read.'' And here is its P\=ali equivalent:

\palisample{yo aya\d m potthako kiccho hoti, so pi manu\~n\~na\d m pa\d than\=aya.\sampleor \ldots, so pana manu\~n\~na\d m pa\d than\=aya.}

We can also use \pali{api ca}\footnote{While \pali{pi} cannot start a sentence or clause, \pali{api} can (see page \pageref{nip:api}).} (but) instead, thus:

\palisample{\ldots, api ca so manu\~n\~na\d m pa\d than\=aya.}

Then we can put \pali{pi} in the first clause and drop the \pali{ya-ta} structure. So, we get this instead:

\palisample{kiccho pi aya\d m potthako hoti, api ca manu\~n\~na\d m pa\d than\=aya.}

Comparing this with this example from the canon, you may get the idea:

\begin{quote} 
\pali{Ahampi kho te, bha\d ne j\=ivaka, m\=atara\d m na j\=an\=ami; api c\=aha\d m te pit\=a; may\=asi pos\=apito}\footnote{Mv\,8.328}\\
``My dear J\=ivaka, even though I do not know your mother, but I am your father, [because you were] fed by me.\\
\end{quote}

Yet another way to compose the sentence is to use \pali{ki\~nc\=api} (although). This particle often works together with \pali{atha kho} or \pali{api ca}. Here are some examples:

\begin{quote}
\pali{Ki\~nc\=api, bho gotama, br\=ahma\d n\=a n\=an\=amagge pa\~n\~napenti, \ldots\ atha kho sabb\=ani t\=ani niyy\=anik\=a}\footnote{D1\,524 (DN\,13)}\\
``Although, Ven.\,Gotama, brahmans declare various paths, \ldots\ those all are leading out to the salvation.''\\[1.5mm]
\pali{Ki\~nc\=api bhava\d m kassapo evam\=aha, atha kho eva\d m me ettha hoti}\footnote{D2\,412 (DN\,23)}\\
``Although the Venerable Kassapa said in that way, this is [still true] for me thus \ldots''\\[1.5mm]
\pali{Ki\~nc\=api, bhante, ayyo anatthiko tena dhammena, apica dussaddh\=apay\=a appasann\=a manuss\=a}\footnote{Buv1\,443}\\
``Although, Venerable, you are not seeking for that matter, but [there are] unpleased people who do not trust [you].''\\
\end{quote}

By these examples, we can revise our task as follows:

\palisample{ki\~nc\=api aya\d m potthako kiccho hoti, atha kho manu\~n\~na\d m pa\d than\=aya.\sampleor \ldots, apica manu\~n\~na\d m pa\d than\=aya.}

This final version is the closest in both meaning and structure. So, we can end this chapter happily.

\section*{Exercise \ref{chap:pi}}
Say these in P\=ali.
\begin{compactenum}
\item Venerable sir, why don't I get rich, even though I made a lot of merit?
\item Such as what, householder?
\item I donated money for building several lodgings in this temple.
\item According to the teaching, you surely will be rich in the next life, even if you don't need it.
\item But I want to be rich in this life, sir.
\item For that matter, you have to work diligently. Even so, you may not be rich as much as you want.
\item What's the use for donating wealth to the religion then?
\item You miss the point of giving completely.
\end{compactenum}
