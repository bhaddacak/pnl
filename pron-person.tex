\chapter{\headhl{It} (is) a book}\label{chap:pron-person}

\phantomsection
\addcontentsline{toc}{section}{Personal Pronouns}
\section*{Personal Pronouns}

There is a close relation between demonstrative and personal pronouns in P\=ali as you might see in the previous chapter. In fact, \pali{ta} plays a dual role, as a demonstrative pronoun and as a personal pronoun---a noun pointing to \emph{person}. Person here does not mean a human being, but it is a grammatical category regarding the ones who engage in the conversation, the interlocutors. There are three persons. \emph{First person} is the one who speaks, represented by \emph{I,} and \emph{we}. \emph{Second person} is the one addressed by the speaker, the interlocutor of first person, represented by \emph{you}. And \emph{third person} is the thing or person that is talked about, represented by \emph{he, she, it,} and \emph{they}.\footnote{In traditional textbooks, the first and third are reversed. I do not follow the traditional scheme.}

Table \ref{tab:nomperson} shows all personal pronouns in nominative case. As you have already seen, \pali{ta} is reproduced from Chapter \ref{chap:pron-demon}. First and second person use the same forms in all genders, so I list them only once. These can be seen as no gender.\footnote{\citealp[p.~62]{collins:grammar}} All these forms should be recalled by heart.

\begin{table}[!hbt]
\centering
\caption{Nominative case of personal pronouns}
\label{tab:nomperson}
\bigskip
\begin{tabular}{@{}*{7}{>{\itshape}l}@{}} \toprule
\multirow{2}{*}{\bfseries\upshape pron.} & \multicolumn{2}{c}{\bfseries\upshape m.} & \multicolumn{2}{c}{\bfseries\upshape f.} & \multicolumn{2}{c}{\bfseries\upshape nt.} \\
\cmidrule(lr){2-3} \cmidrule(lr){4-5} \cmidrule(lr){6-7} 
& \bfseries\upshape sg. & \bfseries\upshape pl. & \bfseries\upshape sg. & \bfseries\upshape pl. & \bfseries\upshape sg. & \bfseries\upshape pl. \\
\midrule
amha \upshape(1st) & aha\d m & maya\d m & & & & \\
& & \pali{no} & & & & \\
tumha \upshape(2nd) & tva\d m & tumhe & & &  & \\
& tuva\d m & vo & & & & \\
ta \upshape(3rd) & so & te & s\=a & t\=a & ta\d m & t\=ani \\
\bottomrule
\end{tabular}
\end{table}

\label{par:enclitic}As you also shall see in the subsequent chapters, first and second person have a very common short (enclitic) forms, i.e.\ \pali{no, vo} (also \pali{me, te} in other chapters). These short forms often cause a confusion, for they are also widely used in other meanings. Practically, these terms ``never come first in a phrase or clause, and almost always refer to what immediately precedes them.''\footnote{See \citealp[p.~64]{collins:grammar}; \citealp[see also][p.~41]{warder:intro}. In Sadd-Pad Ch.\,12, Aggava\d msa wrote, ``\pali{Te me vo noti r\=up\=ani, par\=ani padato yato}'' (Because \pali{te, me, vo, no} [are/depend] on other terms).} Here is a quick example, ``\pali{g\=ama\d m no gaccheyy\=ama}''\footnote{Sadd-Pad Ch.\,12} (Let us go to the village). To new students, I suggest that you should avoid using these short forms of pronouns at this beginning stage. When you see many of them enough, you can figure out how to use them properly.

Therefore ``It (is) a book'' will be simply as:

\palisample{so potthako.\sampleor ta\d m potthaka\d m.}

And ``They (are) books'' is:

\palisample{te potthak\=a.\sampleor t\=ani potthak\=ani.}

To make more sense out of it, let us say this sentence: ``This book (is) big. It (is) heavy.''\footnote{It is better to form the sentence with \pali{ya-ta} structure (see Chapter \ref{chap:yata}).}

\palisample{(yo) aya\d m potthako th\=ulo, so garuko.\sampleor (ya\d m) ida\d m potthaka\d m th\=ula\d m, ta\d m garuka\d m.}

Now let us say ``I (am) an old man. You (are) a young girl.''

\palisample{aha\d m mahallako puriso. tva\d m taru\d n\=a ka\~n\~n\=a.}

Although first and second person have the same form in both gender, the gender of adjectives associated to the speaker and the listener has to be taken from the real gender. Hence if we leave out the nouns in the above sentences, in the same situation (a male speaker talk to a female listener), we will get this:

\palisample{aha\d m mahallako. tva\d m taru\d n\=a.}

Another point comes to my mind concerning gender of nouns. There are a number of words that have two forms to be used with both sexes, e.g.\ kum\=ara/kum\=ar\=i for boy/girl. But many have only one gender form, most of them are masculine, for example \pali{s\=udo} (a cook/chef). What if we want to say ``She is a cook''? I find no clear solution from the traditional point of view. The best and nicest way to deal with this is creating a new word for that gender, for example \pali{s\=ud\=a} or \pali{s\=udak\=a} or \pali{s\=udak\=arin\=i} or even better \pali{bhojanak\=arin\=i}. This solution makes the lexicon bigger, and it takes time to make others accept the use, and some others may reject the new words. Can we bluntly say ``\pali{s\=a s\=udo hoti}''?\footnote{Such a use is called `\pali{vikatikatt\=a}' by the tradition. In English, it is subject complement. Although it looks odd, it is grammatical. This may look better: ``\pali{s\=a m\=atug\=amo gacchati}'' (She, a woman, goes). This use is called \emph{apposition}. Since \pali{m\=atug\=ama} is masculine (see Sadd-Pad Ch.\,8), we cannot go other ways, but I do not find this use in the texts. Incongruence of genders indeed can happen in normal uses, for example when we use numbers (see Chapter \ref{chap:num}). And we can find it in some verses, e.g.\ ``\pali{pam\=ado maccuno pada\d m}'' [Dhp\,2.21] (Carelessness [is] the path of death).}

\phantomsection
\addcontentsline{toc}{section}{\pali{Atta} and other personal representations}
\section*{\pali{Atta} and other personal representations}

Apart from personal pronouns mentioned above, \pali{atta}\footnote{This term declines irregularly, see page \pageref{decl:atta}.} (self) can be used as a reflexive pronoun (one's own self).\footnote{\citealp[pp.~185--6]{warder:intro}} Some examples from the canon (suggested by Warder) are shown below. For these may be too advanced for you now, just make a skim. I put this part here for future referring.

\begin{quote}
\pali{att\=ana\d m sukheti p\=i\d neti}\footnote{D3\,183 (DN\,29)}\\
``[One] makes oneself happy, pleases oneself.''\\[1.5mm]
\pali{S\=a att\=ana\d m ceva j\=ivita\~nca gabbha\~nca s\=apateyya\~nca vin\=a\-sesi.}\footnote{D2\,420 (DN\,23)}\\
``That [woman] destroyed her own life, the fetus, and the property [she would get accordingly].''\\[1.5mm]
\pali{ariyas\=avako \=aka\.nkham\=ano attan\=ava att\=ana\d m by\=akareyya}\footnote{D2\,158 (DN\,16)}\\
``A noble disciple, wishing, should explain himself by himself.''\\[1.5mm]
\pali{j\=an\=asi, \=avuso korakkhattiya, attano gati\d m?}\footnote{D3\,7 (DN\,24)}\\
``Do you know, Korakkhattiya, your own destiny?''\\[1.5mm]
\end{quote}

Some adjectives can be used in the same meaning, such as \pali{sa, saka, nija, niya,} and \pali{niyaka}.\footnote{\citealp[p.~299]{perniola:grammar}} In reflexive use, \pali{saya\d m} and \pali{s\=ama\d m} are also commonly found. Here are some examples:

\begin{quote}
\pali{Ala\d m, mah\=ar\=aja, nis\=ida tva\d m; nisinno aha\d m sake \=asane}\footnote{M2\,303 (MN\,82)}\\
``That's enough [for me], Your Majesty, may you sit [on that one]. I have sat [here] on my own seat.''\\[1.5mm]
\pali{Atha kho, v\=ase\d t\d tha, a\~n\~nataro satto lolaj\=atiko saka\d m bh\=aga\d m parirakkhanto a\~n\~natara\d m bh\=aga\d m adinna\d m \=adiyitv\=a paribhu\~nji.}\footnote{D3\,129 (DN\,27)}\\
``Then, V\=ase\d t\d tha, another greedy being, keeping his own portion, enjoyed other ungiven portion taken.''\\[1.5mm]
\pali{Sehi kammehi dummedho, aggida\d d\d dhova tappati.}\footnote{Dhp\,10.136}\\
``With his own actions, a fool is tormented as if being burnt with fire.''\\[1.5mm]
\pali{Varu\d nassa niya\d m putta\d m, y\=amuna\d m atima\~n\~nasi}\footnote{Ja\,22:787}\\
``[You] scorn Varu\d na's own son, [who was born] in Yamun\=a river.''\\[1.5mm]
\pali{Niyak\=a m\=at\=apitaro, ki\d m pana s\=adh\=ara\d n\=a janat\=a.}\footnote{Thig\,16.471}\\
``[Even] one's own parents [is loathed; as when they die, they are discarded in a cemetery], let alone general people.''\\[1.5mm]
\pali{saya\d mkata\d m makka\d takova j\=ala\d m}\footnote{Dhp\,24.347}\\
``Like a spider [gets caught] in the web itself created.''\\[1.5mm]
\mbox{\pali{Yo pana bhikkhu bhikkhussa s\=ama\d m c\=ivara\d m datv\=a \ldots}}\\
``Whichever monk, himself having given a robe to [another] monk \ldots''\\[1.5mm]
\end{quote}

It is alright if you cannot fully understand the examples above. Just keep in mind and come to these again when you are more ready. Now is the time to do our exercise.

\enlargethispage{\baselineskip}
\section*{Exercise \ref{chap:pron-person}}
Say these in P\=ali.
\begin{compactenum}
\item You (are) evil big enemies.
\item You (are) a tall handsome clever man.
\item We are a great army, strong, brave.
\item Those people (are) Buddhist monks. They (are) thin (and) weak.
\item This object (is) precious. It (is) a blue oval gem.
\end{compactenum}
