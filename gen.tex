\chapter{\headhl{I have} a book}\label{chap:gen}

When we learn to speak English, or any language for this matter, after we know how to call things, the next step is usually to say that someone \emph{has} something. In this chapter, we will learn how to say likewise. But, strangely, P\=ali has no what we call verb `to have' in English.\footnote{The closest term may be \pali{dh\=areti} which means `to bear' or `to hold' or `to wear.' This can be used as `to have' in some context. Another term is \pali{ga\d nh\=ati} which means `to take' or `to seize' or `to hold.'} Instead, we have to change the sentence to ``something of someone exists'' or ``something is someone's.'' So, when we want to say ``I have a book,'' we have to say ``My book exists'' or ``A book is mine.''

Hence, what to learn here is how to make a term possessive.

\phantomsection
\addcontentsline{toc}{section}{Declension of Genitive Case}
\section*{Declension of Genitive Case}

In P\=ali we use \emph{genitive case} to denote possession. It is much like an apostrophe (\emph{'s}) in English. Table \ref{tab:genreg} summarizes the declension of genitive case of regular nouns, including adjectives.

\begin{table}[!hbt]
\centering\small
\caption{Genitive case endings of regular nouns}
\label{tab:genreg}
\bigskip
\begin{tabular}{@{}>{\bfseries}l*{5}{>{\itshape}l}@{}} \toprule
\multirow{2}{*}{G. Num.} & \multicolumn{5}{c}{\bfseries Endings} \\
\cmidrule(l){2-6}
& a & i & \=i & u & \=u\\
\midrule
m. sg. & assa & issa & \replacewith{\=i}{issa} & ussa & \replacewith{\=u}{ussa} \\
& & ino & \replacewith{\=i}{ino} & uno & \replacewith{\=u}{uno} \\
m. pl. & \replacewith{a}{\=ana\d m} & \replacewith{i}{\=ina\d m} & \=ina\d m & \replacewith{u}{\=una\d m} & \=una\d m \\
\midrule
nt. sg. & assa & issa &  & ussa & \\
& & ino & & uno & \\
nt. pl. & \replacewith{a}{\=ana\d m} & \replacewith{i}{\=ina\d m} & & \replacewith{u}{\=una\d m} & \\
\midrule
& \=a & i & \=i & u & \=u\\
\midrule
f. sg. & \=aya & iy\=a & \replacewith{\=i}{iy\=a} & uy\=a & \replacewith{\=u}{uy\=a} \\
f. pl. & \=ana\d m & \replacewith{i}{\=ina\d m} & \=ina\d m & \replacewith{u}{\=una\d m} & \=una\d m \\
\bottomrule
\end{tabular}
\end{table}

Up to now, we have enough knowledge to say ``An elephant has eyes.'' First, we change the sentence to ``An elephant's eyes exist'' or ``Eyes are an elephant's.'' And here is its P\=ali:

\palisample{hatthissa akkh\=ini santi.\sampleor hatthino akkh\=ini santi.}

We normally use verb \pali{atthi} in this context, because it is closer to verb `to have' than \pali{hoti} and \pali{bhavati} which are closer to verb `to be.' Please note that the subject of the sentence is not `elephant' but `eyes.' So, the verb agreeing with this subject is plural. In the sentence, \pali{hatthissa/hatthino} acts like a modifier of the subject. It can be singular or plural. And here is an example for ``Elephants have eyes.''

\palisample{hatth\=ina\d m akkh\=ini santi.}

For a feminine example, this is for ``A girl has beautiful hands.''

\palisample{ka\~n\~n\=aya sundar\=a hatth\=a santi.}

Before we can finish the task of this chapter, we have to know the genitive declension of pronouns first. And these are shown in Table \ref{tab:genpron}. It is worth noting that m.\ and f.\ of 1st and 2nd person pronouns have the same forms. For other pronouns, nt.\ genitives are the same as m. Therefore, you do not need to remember everything in the table. Study it carefully and try to catch its pattern.

\begin{table}[!hbt]
\centering
\caption{Genitive case of pronouns}
\label{tab:genpron}
\bigskip
\begin{tabular}{@{}*{5}{>{\itshape}l}@{}} \toprule
\multirow{2}{*}{\bfseries\upshape Pron.} & \multicolumn{2}{c}{\bfseries\upshape m./nt.} & \multicolumn{2}{c}{\bfseries\upshape f.} \\
\cmidrule(lr){2-3} \cmidrule(lr){4-5}
& \bfseries\upshape sg. & \bfseries\upshape pl. & \bfseries\upshape sg. & \bfseries\upshape pl. \\
\midrule
amha & mayha\d m & amh\=aka\d m & &\\
& amha\d m & no & & \\
& mama & & & \\
& mama\d m & & & \\
& me & & & \\
tumha & tuyha\d m & tumh\=aka\d m & & \\
& tumha\d m & vo & & \\
& tava & & & \\
& te & & & \\
ta & tassa & tesa\d m & tass\=a & t\=asa\d m \\
& assa & nesa\d m & ass\=a & \\
& & & tiss\=a & \\
eta & etassa & etesa\d m & etass\=a & et\=asa\d m \\
& & & etiss\=a & \\
ima & imassa & imesa\d m & imiss\=a & im\=asa\d m \\
& assa & & ass\=a & \\
amu & amussa & am\=usa\d m & amuss\=a & am\=usa\d m \\
& amuno & & & \\
\bottomrule
\end{tabular}
\end{table}

Now we can fulfill our task, to say ``I have a book.''

\palisample{mayha\d m potthako(-ka\d m) atthi.}

We can replace \pali{mayha\d m} with other alternatives, except \pali{me} which is usually not placed at the beginning.\footnote{See page \pageref{par:enclitic}.} We use 3rd person verb here because the book exists not I. Let us play around further. Here is for ``This girl has this big book.''

\palisample{imiss\=a ka\~n\~n\=aya aya\d m th\=ulo potthako atthi. \textup{\normalsize(m.)}\sampleor imiss\=a ka\~n\~n\=aya ida\d m th\=ula\d m potthaka\d m atthi. \textup{\normalsize(nt.)}}

Looking closely to the example above, you will find an important rule concerning the use of declension which I would like to repeat it again: \emph{Modifiers must take the same case as nouns they modify}. In the example, `girl' takes genitive case, so as the first `this.' The second `this' and `big' modify the subject `book,' so they have to take nominative case corresponding to the gender (and number) of the subject.

Another use of gen.\ is in the phrase ``Of those, \ldots'' or ``Among those, \ldots'' It is easier to see an example. When you want to say ``Among those people, you are a clever one,'' you can put it in this way:

\palisample{etesa\d m jan\=ana\d m tva\d m kusalo/kusal\=a hosi.}

If `you' is male, \pali{kusalo} is used, otherwise \pali{kusal\=a}. For a full technical explanation of genitive case, see Chapter \ref{chap:cases}.

It will not be complete if we do not talk about negation here. When you say you have no particular thing, you just use \pali{natthi} (\pali{na+atthi}) instead of \pali{atthi}. Negating this verb (by \pali{na}) means that such a thing does not exist.\footnote{For more information about negative particles, see Appendix \ref{chap:nipata}, page \pageref{nip:neg}.} For example, saying ``I have no book,'' you go simply as:

\palisample{mayha\d m potthako(ka\d m) natthi.}

As you go further, it is a good chance you will meet \pali{atthi} and \pali{natthi} used in plural sense, in stead of \pali{santi} or \pali{na santi}. For example, ``\pali{putt\=a matthi (me+atthi)}''\footnote{Dhp\,5.62} (my children exist) and ``\pali{natthi loke sama\d nabr\=ahma\d n\=a}''\footnote{A3\,118} (no ascetics and brahmans in the world). Traditional textbooks explain that beside taking verb forms, \pali{atthi}, also \pali{natthi} in this case, is regarded as a particle (\pali{nip\=ata}) as well. So, it is used uninflected, and only in nominative case.\footnote{``\pali{Atthi sakk\=a labbh\=a iccete pa\d tham\=aya\d m}'' (these, namely \pali{atthi, sakk\=a, labbh\=a}, [are] in nom.), in R\=upa after 282, Nep\=atikapada toward the end of N\=amaka\d n\d da. And in Sadd-Pad Ch.\,13, ``\pali{Atthinatthisadd\=a hi nip\=atatt\=a ekattepi bahuttepi pavattanti}'' (The words \pali{atthi} and \pali{natthi} go as singular and plural due to [they are] particles).} You will learn more about particles in Chapter \ref{chap:ind-intro}, Chapter \ref{chap:ind-to}, and Appendix \ref{chap:nipata}.

I have some thought about this peculiarity. Language in use and language in the eyes of grammarians sometimes go in different ways. When anomalies occur, grammarians have to find a viable explanation. Whereas, speakers or writers just use them mindlessly in the most convenient way. When aberrations happen frequently, they become norm. Then new rules are established. This is true in all living languages as well, I infer.

Before we close this chapter, let us figure out how to say ``You have my book.'' If you think carefully about this problem, it will give you a good headache and a realization that not every `have' in English can be transformed to P\=ali genitives. I will come to this later in Chapter \ref{chap:yata}. Now you have to finish our exercise.

\section*{Exercise \ref{chap:gen}}
Say these in P\=ali.
\begin{compactenum}
\item This fortune is mine.
\item You have good looking fingers.
\item These lucky women have diligent husbands.
\item Among those frogs, the fat ones have big eyes.
\item These trees have many fruits. They (fruits) belong to those people.
\item I have a brother, no sister.
\end{compactenum}
