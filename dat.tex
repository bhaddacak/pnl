\chapter{I go to school \headhl{for knowledge}}\label{chap:dat}

The next case we are going to talk about is used to mark the destination or purpose of an action as well as the indirect object of it. We call this \emph{dative} case. In English we normally use preposition `for' or `to' to achieve this. This can confuse new students, because for the destination of movement, which is also marked by `to,' we use accusative case (see Chapter \ref{chap:verb-go}) not dative. However, the similarity of meaning makes us see that in several cases they can be used interchangeably. This is often the case when we read the scriptures. When we use in conversation, I suggest, we should use what we intend to mean.

\phantomsection
\addcontentsline{toc}{section}{Declension of Dative Case}
\section*{Declension of Dative Case}

Table \ref{tab:datreg} shows the declension of dative case. As you may recall, dative and genitive forms look alike, except some with highlight. This means you do not have to remember many of them. It also makes text analysis harder. Although they look similar, dative and genitive case work differently. It is worth keeping in mind that \emph{the dative relate verb to noun, whereas the genitive relate noun to noun.} However, we often find that in some ambiguous sentences we can translate in both ways.

\begin{table}[!hbt]
\centering\small
\caption{Dative case endings of regular nouns}
\label{tab:datreg}
\bigskip
\begin{tabular}{@{}>{\bfseries}l*{5}{>{\itshape}l}@{}} \toprule
\multirow{2}{*}{G. Num.} & \multicolumn{5}{c}{\bfseries Endings} \\
\cmidrule(l){2-6}
& a & i & \=i & u & \=u\\
\midrule
m. sg. & assa & issa & \replacewith{\=i}{issa} & ussa & \replacewith{\=u}{ussa} \\
& \texthl{\replacewith{a}{\=aya}} & ino & \replacewith{\=i}{ino} & uno & \replacewith{\=u}{uno} \\
& \texthl{attha\d m} & & & & \\
m. pl. & \replacewith{a}{\=ana\d m} & \replacewith{i}{\=ina\d m} & \=ina\d m & \replacewith{u}{\=una\d m} & \=una\d m \\
\midrule
nt. sg. & assa & issa &  & ussa & \\
& \texthl{\replacewith{a}{\=aya}} & ino & & uno & \\
& \texthl{attha\d m} & & & & \\
nt. pl. & \replacewith{a}{\=ana\d m} & \replacewith{i}{\=ina\d m} & & \replacewith{u}{\=una\d m} & \\
\midrule
& \=a & i & \=i & u & \=u\\
\midrule
f. sg. & \=aya & iy\=a & \replacewith{\=i}{iy\=a} & uy\=a & \replacewith{\=u}{uy\=a} \\
f. pl. & \=ana\d m & \replacewith{i}{\=ina\d m} & \=ina\d m & \replacewith{u}{\=una\d m} & \=una\d m \\
\bottomrule
\end{tabular}
\end{table}

Declension of dative case of pronouns is shown in Table \ref{tab:datpron}. The table is exactly the same as genitive case in Table \ref{tab:genpron} of Chapter \ref{chap:gen}.

\begin{table}[!hbt]
\centering
\caption{Dative case of pronouns}
\label{tab:datpron}
\bigskip
\begin{tabular}{@{}*{5}{>{\itshape}l}@{}} \toprule
\multirow{2}{*}{\bfseries\upshape Pron.} & \multicolumn{2}{c}{\bfseries\upshape m./nt.} & \multicolumn{2}{c}{\bfseries\upshape f.} \\
\cmidrule(lr){2-3} \cmidrule(lr){4-5}
& \bfseries\upshape sg. & \bfseries\upshape pl. & \bfseries\upshape sg. & \bfseries\upshape pl. \\
\midrule
amha & mayha\d m & amh\=aka\d m & & \\
& amha\d m & no & & \\
& mama & & & \\
& mama\d m & & & \\
& me & & & \\
tumha & tuyha\d m & tumh\=aka\d m & & \\
& tumha\d m & vo & & \\
& tava & & & \\
& te & & & \\
ta & tassa & tesa\d m & tass\=a & t\=asa\d m \\
& assa & nesa\d m & ass\=a & \\
& & & tiss\=a & \\
eta & etassa & etesa\d m & etass\=a & et\=asa\d m \\
& & & etiss\=a & \\
ima & imassa & imesa\d m & imiss\=a & im\=asa\d m \\
& assa & & ass\=a & \\
amu & amussa & am\=usa\d m & amuss\=a & am\=usa\d m \\
& amuno & & & \\
\bottomrule
\end{tabular}
\end{table}

With what we know so far, we can say ``I go to school for knowledge'' as:

\palisample{aha\d m vijj\=aya p\=a\d thas\=ala\d m gacch\=ami.}

Knowledge is the purpose of the going, so we use dative case (f.\ form). School is the destination of the going, or direct object of it, so it takes accusative form. For a sentence with indirect object, such as ``I give a book to a boy,'' we can say in P\=ali as:

\palisample{aha\d m kum\=arassa potthaka\d m demi.}

This sentence is equivocal. It can be translated as ``I give a book to a boy'' (dative) or ``I give a boy's book'' (genitive). It might be said that the genitive meaning is not allowed because the book does not belong to me, so I cannot give it to anybody. But try this sentence ``I hold a book for a boy'' which can be rendered as:

\palisample{aha\d m kum\=arassa potthaka\d m dh\=aremi.}

This sentence can be translated equally as ``I hold a boy's book'' which has a close meaning to its dative sense. However, if we take it seriously, dative and genitive cases have a different connotation. Therefore, be aware what you are saying.

To make things less problematic, for singular m.\ and nt.\ nouns with \pali{a} ending, we should use the alternative forms: \pali{kum\=ar\=aya} or \pali{kum\=arattha\d m}. In fact, most nouns in P\=ali fall into this group, and these alternative forms of dative case are used more often than its genitive-like forms. That is the way the tradition solves the ambiguity problem. So, a clearer sentence looks like the following:

\palisample{aha\d m kum\=ar\=aya potthaka\d m demi.\sampleor aha\d m kum\=arattha\d m potthaka\d m demi.}

Instead of taking an accusative object, there are some verbs that take a dative object. A frequently found one is \pali{ruccati}\footnote{\pali{ruca rocane}, Sadd-Dh\=a\,882, 109} (satisfy, delight). You have to change your grammar rule a little when using the term, i.e.\ something satisfies \emph{to} someone. Here are examples:

\begin{quote}
\pali{gamana\d m mayha\d m ruccati}\footnote{Ja\,22:2102} \\
``Going satisfies (to) me.'' \\[1.5mm]
\pali{pabbajj\=a mama ruccati}\footnote{Ja\,22:43} \\
``Going forth satisfies (to) me.'' \\[1.5mm]
\pali{Bhatta\d m me ruccati. Bhattampitassa na ruccati.}\footnote{Sadd-Dh\=a\,882} \\
``Food satisfies (to) me, but food does not satisfy (to) him.''
\end{quote}

\phantomsection
\addcontentsline{toc}{section}{Datives and \pali{bhabba} etc.}
\section*{Datives and \pali{bhabba} etc.}

There are some other terms that relate somehow to dative meaning, for example, \pali{bhabba} (capable of, suitable to), \pali{abhabba} (not capable of, not suitable to), \pali{kalla} (suitable to), and \pali{ala\d m} (enough).\footnote{Instead of using with dative instances, these terms can be used in the same way with infinitives (verbs in \pali{-tu\d m} form). See Chapter \ref{chap:inf} for more detail.} The last one is used as an indeclinable, the rest like adjectives. Some examples are shown below.\footnote{For more terms that relate to dative case, see \citealp[pp.~67--9]{warder:intro}. See also \citealp[pp.~326--7]{perniola:grammar}.} If you feel that the following examples are too difficult because there are many things you have not learned yet, just skip them for now and come back when you feel more ready.

\begin{quote}
\pali{an\=at\=ap\=i anottapp\=i abhabbo sambodh\=aya \ldots \=at\=ap\=i ca kho ottapp\=i 
bhabbo sambodh\=aya}\footnote{S2\,145 (SN\,16)}\\
``One who is not strenuous [and] scrupulous [is] not capable of enlightenment, but one who is strenuous [and] scrupulous [is] capable of enlightenment.'' \\[1.5mm]
\pali{Abhabbo parih\=an\=aya, nibb\=anasseva santike}\footnote{A6\,32}\\
	``[That person who is] not suitable to degeneration, [is] near to nirvana.'' \\[1.5mm]
\pali{Yo so, \=avuso, bhikkhu eva\d m j\=an\=ati eva\d m passati, kalla\d m tasseta\d m vacan\=aya}\footnote{D1\,377, 379 (DN\,6,7)}\\
``Which monk, Venerable, who knows and see thus, that [monk is] suitable for saying this \ldots'' \\[1.5mm]
\pali{Susikkhitosi, bha\d ne j\=ivaka. Ala\d m te ettaka\d m j\=ivik\=aya}\footnote{Mv\,8.329}\\
``You are well-learned, my dear J\=ivaka. That is much enough for your living.'' \\[1.5mm]
\pali{sabbe dhamm\=a n\=ala\d m abhinives\=aya}\footnote{M1\,390 (MN\,37)}\\
``All things are not suitable for adherence.''
\end{quote}

Please test your understanding with this exercise.

\section*{Exercise \ref{chap:dat}}
Say these in P\=ali.
\begin{compactenum}
\item You, a millionaire, give a land to a farmer.
\item I carry my body with me for my benefit.
\item From poor village, those workers come to the city for a fortune.
\item Doctors from hospitals work with their craft for the health of many people.
\item Cooks from a big hotel cook food for students of this school.
\end{compactenum}
