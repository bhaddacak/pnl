\chapter{\headhl{Taddhita} (Secondary Derivation)}\label{chap:taddhita}

In English, when we add `-ian' to `music' we get `musician' meaning ``one who plays music.'' If we add it to `politics' we get `politician' meaning ``one who is involved in politics.'' And if we add it to `Mars' we get `Martian' meaning ``one who lives in Mars.'' The process of adding something to words and modifying meaning of the terms they are produced is called \emph{derivation}. P\=ali has the same kind of process called \pali{Taddhita}. The process is done by a set of suffixes (\pali{paccaya}) as we will learn in detail here.

Secondary derivation occurs when an additional \pali{paccaya} is added to the existing nouns, producing new nouns or adjectives. By `new' here, it is meant only modification like `-ian' example above. It is called `secondary' because it does not happen to root or stem level, but rather to the upper level of existing nouns, which somehow once underwent former derivation. We call this kind of words \emph{secondary derivative} or Taddhita.

Learning about P\=ali Taddhita is mostly about leaning how each \pali{paccaya} works and what kind of meaning it produces. In textbooks there is no clear classification of Taddhita, but from the order of formulas in Kacc and Sadd we can classify it into 13 types according to their meaning. Following this scheme, Supaphan na Bangchang adds another miscellaneous type, making it 14 in all.\footnote{\citealp[p.~399]{supaphan:pali}} I summarize the list in the table below. We will follow this and follow Kacc and Sadd's way of naming \pali{paccaya}s, except some mentioned only by Mogg. All \pali{paccaya}s mentioned here, together with those of primary derivation, are indexed in Appendix \ref{chap:paccaya}. My caveat here is that \pali{paccaya}s look somewhat messy\footnote{The commentator of the Vinaya, ascribed to Buddhaghosa, admits that the usage of \pali{taddhita} is variegated (\pali{Vicitr\=a hi taddhitavutti!}, Sp1\,8). Aggava\d msa repeats this in the last formula of the chapter (Sadd\,864), and says that on one can describe all of them completely because of their profundity, except arhants who have linguistic insight! I do not think the topic is profound in an esoteric way. It is just vast and messy as the nature of a linguistic hotchpotch.} because some of them can mean many things, particularly \pali{\d na}, \pali{\d nika}, \pali{\d neyya}, and \pali{iya}. You should not be discouraged by this difficulty. Your task is not to understand all of these, but to be familiar with them as such. As you have learned so far, you have to realize that order in P\=ali grammar is not what you can expect. Every grammarian from the past put a lot of effort to make it. And these are the best we can get from them.

\bigskip
\begin{longtable}[c]{@{}%
	>{\raggedleft\arraybackslash}p{0.03\linewidth}%
	>{\itshape\raggedright\arraybackslash}p{0.5\linewidth}%
	>{\raggedleft\arraybackslash}p{0.1\linewidth}@{}}
\caption*{Groups of \pali{Taddhita}}\\
\toprule
& \bfseries\upshape Group & \bfseries\upshape Page \\ \midrule
\endfirsthead
\multicolumn{2}{c}{Groups of \pali{Taddhita} (contd\ldots)}\\
\toprule
& \bfseries\upshape Group & \bfseries\upshape Page \\ \midrule
\endhead
\bottomrule
\ltblcontinuedbreak{3}
\endfoot
\bottomrule
\endlastfoot
%
1. & Apaccataddhita & \pageref{tadgroup1} \\
2. & Taraty\=aditaddhita & \pageref{tadgroup2} \\
3. & R\=ag\=aditaddhita & \pageref{tadgroup3} \\
4. & J\=at\=aditaddhita & \pageref{tadgroup4} \\
5. & Sam\=uhataddhita & \pageref{tadgroup5} \\
6. & \d Th\=anataddhita & \pageref{tadgroup6} \\
7. & Bahulataddhita & \pageref{tadgroup7} \\
8. & Bh\=avataddhita & \pageref{tadgroup8} \\
9. & Visesataddhita & \pageref{tadgroup9} \\
10. & Tadassatthitaddhita & \pageref{tadgroup10} \\
11. & Pakatitaddhita & \pageref{tadgroup11} \\
12. & Sa\.nkhy\=ataddhita & \pageref{tadgroup12} \\
13. & Abyayataddhita & \pageref{tadgroup13} \\
14. & Anekatthataddhita & \pageref{tadgroup14} \\
\end{longtable}

Like compounds (Sam\=asa), Taddhita uses analytic sentences to explain the words produced by the process. The sentences generally look easier than those of compounds. It is better to read about that in Appendix \ref{chap:samasa} before you go further, if you have not read it yet. Even though analytic sentences are useful, we will not pay attention to them much, so I will show them only when necessary in footnotes. The gender of the outcomes can be varied. If the words can be used as an adjective, it can be rendered into three genders. If they denote a person, the gender depends. And if they denote states of being, they will be neuter. You can see the intended gender in the analytic sentences.

\subsection*{1.\ \pali{Apaccataddhita}}\label{tadgroup1}

This group means `offspring (of)' (\pali{apacca}). \pali{Paccaya}s used in this group are \pali{\d na, \d n\=ayana, \d n\=ana, \d neyya, \d ni, \d nika, \d nava, \d nera, \d niya, ussa, usa\d n, \~n\~na, ya, iya,} and \pali{\d nya}. The first eight comes from Kacc, the next three are added by Sadd. In Mogg, some agree with other schools. Some have a slightly different name, i.e.\ \pali{ssa} and \pali{sa\d na} are the same as \pali{ussa} and \pali{usa\d n}. Some are newly added, i.e.\ the last four of the list.

\subparagraph*{\pali{\d Na}} (Kacc\,344, R\=upa\,361, Sadd\,752, Mogg\,4.1, Mogg\,4.9)\label{pacct1:dna}

To new students, the most perplexing \pali{paccaya} of all is \pali{\d na} because it entails \pali{vuddhi} strength of the first vowel (see the end of Chapter \ref{chap:nuts}). This means \pali{a} becomes \pali{\=a}; \pali{i} and \pali{\=i} become \pali{e}; \pali{u} and \pali{\=u} become \pali{o}.\footnote{Kacc\,405, R\=upa\,365, Sadd\,751} In fact, \pali{\d na} is only a sign of \pali{vuddhi}. We call this kind of sign \pali{anubandha} (see page \pageref{sec:anubandha}). It means ``just do \pali{vuddhi} thing right here.'' So, normally we will not see \pali{\d na} appears anywhere in the end products unless \pali{\d n} is a part of the base or the \pali{paccaya} itself. However, sometimes in rare cases \pali{\d na} does appear. For more detail, see page \pageref{par:dnapacc}. Here are some typical examples: 

\pali{vasi\d t\d tha + \d na} = \palibf{v\=asi\d t\d tha}\footnote{\pali{vasi\d t\d thassa apacca\d m v\=asi\d t\d tho.} In Sadd\,752, Aggava\d msa explains further that the word can become \pali{v\=ase\d t\d tha} (see also Sadd\,156). It can be \pali{v\=ase\d t\d th\=i} (women of the clan), or \pali{v\=ase\d t\d tha\d m} (the clan itself).} (offspring of \pali{Vasi\d t\d tha}) \par
\pali{gotama + \d na} = \palibf{gotama}\footnote{Like above, this and the followings can be rendered into three genders, i.e.\ \pali{gotamo}, \pali{gotam\=i} and \pali{gotama\d m}. This example shows that if the first vowel is already in \pali{vuddhi} strength, it stays the same.} (offspring of \pali{Gotama})\par
\pali{vasudeva + \d na} = \palibf{v\=asudeva} (offspring of \pali{Vasudeva}) \par
\pali{pa\~nc\=ala + \d na} = \palibf{pa\~nc\=ala}\footnote{If the first vowel precedes double consonants, it stays the same.} (offspring of a Pa\~nc\=ala's dweller, king of Pa\~nc\=ala) \par
\pali{kosala + \d na} = \palibf{kosala} (offspring of a Kosala's dweller, king of Kosala) \par
\pali{magadha + \d na} = \palibf{m\=agadha} (offspring of a Magadha's dweller, king of Magadha) \par

\subparagraph*{\pali{\d N\=ayana, \d n\=ana}} (Kacc\,345, R\=upa\,366, Sadd\,754, Mogg\,4.2)\label{pacct1:dnaayana}\label{pacct1:dnaana}

Like \pali{\d na} above, these \pali{paccaya}s have additional part apart from the \pali{vuddhi} process. They produce the same meaning but used with different group of words, for example:

\pali{vaccha + \d n\=ayana/\d n\=ana} = \palibf{vacch\=ayana/vacch\=ana} (offspring of \pali{Vaccha}) \par
\pali{kacca + \d n\=ayana/\d n\=ana} = \palibf{kacc\=ayana/kacc\=ana} (offspring of \pali{Kacca}) \par
\pali{sa\d mgha + \d n\=ayana/\d n\=ana} = \palibf{sa\d mgh\=ayana/sa\d mgh\=ana} (offspring of the Sangha) \par
\pali{cora + \d n\=ayana/\d n\=ana} = \palibf{cor\=ayana/cor\=ana} (offspring of a thief) \par

\subparagraph*{\pali{\d Neyya}} (Kacc\,346, R\=upa\,367, Sadd\,755, Mogg\,4.3)\label{pacct1:dneyya}

This \pali{paccaya} is used with f.\ nouns, for example:

\pali{kattik\=a + \d neyya} = \palibf{kattikeyya} (offspring of \pali{Kattik\=a}) \par
\pali{vint\=a + \d neyya} = \palibf{venteyya} (offspring of \pali{Vint\=a}) \par
\pali{ga\.ng\=a + \d neyya} = \palibf{ga\.ngeyya} (offspring of \pali{Ga\.ng\=a}) \par

\subparagraph*{\pali{\d Naya}} (Mogg\,4.4, Mogg\,4.10)\label{pacct1:dnaya}

Apart from \pali{vuddhi}, this \pali{paccaya} also entails \pali{ya} process. That is why you see double consonants here. For more about \pali{ya}, see page \pageref{pacca:ya2}.

\pali{diti + \d naya} = \palibf{decca} (offspring of \pali{Diti}) \par
\pali{\=aditi + \d naya} = \palibf{\=adicca} (offspring of \pali{\=Aditi}) \par
\pali{ku\d n\d dan\=i + \d naya} = \palibf{ko\d n\d da\~n\~na} (offspring of \pali{Ku\d n\d dan\=i}) \par
\pali{kuru + \d naya} = \palibf{korabya} (offspring of a Kuru's dweller, king of Kuru) \par
\pali{siv\=i + \d naya} = \palibf{sebya} (offspring of a Siv\=i's dweller, king of Siv\=i) \par

\subparagraph*{\pali{\d Ni}} (Kacc\,347, R\=upa\,368, Sadd\,756, Mogg\,4.5)\label{pacct1:dni}

\pali{dakkha + \d ni} = \palibf{dakkhi} (offspring of \pali{Dakkha}) \par
\pali{vasava + \d ni} = \palibf{v\=asavi} (offspring of \pali{Vasava}) \par
\pali{varu\d na + \d ni} = \palibf{v\=aru\d ni} (offspring of \pali{Varu\d na}) \par
\pali{sakyaputta + \d ni} = \palibf{sakyaputti} (offspring of \pali{Sakya}'s child) \par

\subparagraph*{\pali{\d Nika, \d niya}} (Sadd\,756)\label{pacct1:dnika}\label{pacct1:dniya}

\pali{sakyaputta + \d nika/\d niya} = \palibf{sakyaputtika/sakyaputtiya} (offspring of \pali{Sakya}'s child) \par
\pali{n\=a\d taputta + \d nika/\d niya} = \palibf{n\=a\d taputtika/n\=a\d taputtiya} (offspring of a dancer's child) \par
\pali{jinadatta + \d nika/\d niya} = \palibf{jenadattika/jenadattiya} (offspring of \pali{Jinadatta}) \par

\subparagraph*{\pali{\d Nava}} (Kacc\,348, R\=upa\,371, Sadd\,757)\label{pacct1:dnava}

This \pali{paccaya} is used with \pali{u}-ending nouns.

\pali{upagu\footnote{In Sadd\,757 it is \pali{upaku}.} + \d nava} = \palibf{opagava} (offspring of \pali{Upagu}) \par
\pali{manu + \d nava} = \palibf{m\=anava}\footnote{In Mogg\,4.8 this is a product of \pali{\d na}.} (offspring of \pali{Manu}) \par
\pali{bhaggu + \d nava} = \palibf{bhaggava} (offspring of \pali{Bhaggu}) \par
\pali{pa\d n\d du + \d nava} = \palibf{pa\d n\d dava} (offspring of \pali{Pa\d n\d du}) \par

\subparagraph*{\pali{\d Nera}} (Kacc\,349, R\=upa\,372, Sadd\,758, Mogg\,4.3)\label{pacct1:dnera}

This \pali{paccaya} is used mostly with general nouns, not proper nouns like above.

\pali{vidhav\=a + \d nera} = \palibf{vedhavera} (offspring of a widow) \par
\pali{sama\d na + \d nera} = \palibf{s\=ama\d nera} (offspring of an ascetic) \par

\subparagraph*{\pali{Ussa, usa\d n, ssa, sa\d na}} (Sadd\,753, Mogg\,4.8)\label{pacct1:ussa}\label{pacct1:usadn}\label{pacct1:ssa}\label{pacct1:sadna}

\pali{manu + ussa/usa\d n} = \palibf{manussa/m\=anusa}\footnote{This word means a human being in general. If this is treated as a compound, it can be analyzed to ``\pali{mano usso ussanno etass\=ati manusso}'' (One full of superior mind is human being).} (offspring of \pali{Manu}) \par

\subparagraph*{\pali{\~N\~na}} (Mogg\,4.6)\label{pacct1:ynyna}

\pali{r\=aja + \~n\~na} = \palibf{r\=aja\~n\~na} (royal birth) \par

\subparagraph*{\pali{Ya, iya}} (Mogg\,4.7)\label{pacct1:ya}\label{pacct1:iya}

\pali{khatta + ya/iya} = \palibf{khatya/khattiya} (royal birth) \par

\subsection*{2.\ \pali{Taraty\=aditaddhita}}\label{tadgroup2}

This group means `crossing' (\pali{tara}), etc. In Kacc, only one \pali{paccaya} is mentioned---\pali{\d nika}. In Sadd, other five are added, namely \pali{\d nera, \d neyya, \d niya, \d naya,} and \pali{\=i}. In Mogg, there are totally 24 of them, including \pali{\d nika}. Other are \pali{\d naka, ika, iya, kiya, \d na, tana, acca, ima, ka\d na, \d neyya, \d neyyaka, ya, eyyaka, ttana, \=avantu, rati, r\=iva, r\=ivataka, ita, matta, taggha,} and \pali{r\=aya}.

\subparagraph*{\pali{\d Nika}} (Kacc\,350--1, R\=upa\,373--4, Sadd\,764, Mogg\,4.27--9, etc.)\label{pacct2:dnika}

This \pali{paccaya} produces various kinds of meaning. They are numerous. I try to cover those described in the textbooks as many as possible, nevertheless I have to leave out many trivial instances and some incomprehensible ones. Please learn from the examples below.

\pali{v\=i\d n\=a + \d nika} = \palibf{ve\d nika}\footnote{\pali{v\=i\d n\=a assa sippa\d m ve\d niko.}} (lute player) \par
\pali{mudi\.nga + \d nika} = \palibf{modi\.ngika} (drummer) \par
\pali{va\d msa + \d nika} = \palibf{va\d msika} (flute player) \par
\pali{gadha + \d nika} = \palibf{gandhika}\footnote{\pali{gandho assa bha\d n\d da\d m gandhiko.}} (perfume trader) \par
\pali{tela + \d nika} = \palibf{telika} (oil trader) \par
\pali{gu\d la + \d nika} = \palibf{go\d lika} (sugar trader) \par
\pali{c\=apa + \d nika} = \palibf{c\=apika}\footnote{\pali{c\=apo assa \=avudho c\=apiko.}} (archer) \par
\pali{tomara + \d nika} = \palibf{tomarika} (lancer) \par
\pali{v\=ata + \d nika} = \palibf{v\=atika}\footnote{\pali{v\=ato assa \=ab\=adho v\=atiko.}} (one sick from wind) \par
\pali{semha + \d nika} = \palibf{semhika} (one sick from phlegm) \par
\pali{kumbha + \d nika} = \palibf{kumbhika}\footnote{There are 3 meanings described in Kacc\,351, R\=upa\,374, Sadd\,764: (1) \pali{kumbho assa parim\=a\d na\d m kumbhika\d m}; (2) \pali{kumbhassa r\=asi kumbhika\d m}; (3) \pali{kumbha\d m arahat\=iti kumbhiko}.} (volume of 1 pot, heap of pot, price worth 1 potful) \par
\pali{pa\d msuk\=ula + \d nika} = \palibf{pa\d msuk\=ulika}\footnote{\pali{pa\d msuk\=ulassa dh\=ara\d na\d m pa\d msuk\=ula\d m, pa\d msuk\=ula\d m s\=ilamass\=ati pa\d msuk\=uliko.}} (one wearing discarded robe) \par
\pali{tic\=ivara + \d nika} = \palibf{tec\=ivarika} (one using 3 robes) \par
\pali{upadhi + \d nika} = \palibf{opadhika}\footnote{\pali{upadhippayojanamassa opadhika\d m.} (from Mogg\,4.27)} (having body as benefit) \par
\pali{vinaya + \d nika} = \palibf{venayika}\footnote{\pali{vinayamadh\=ite venayiko}, or, \pali{vinaya\d m deset\=iti venayiko.}} (one knowing or preaching the Vinaya) \par
\pali{suttanta + \d nika} = \palibf{suttantika} (one knowing or preaching the Suttanta) \par
\pali{abhidhamma + \d nika} = \palibf{\=abhidhammika} (one knowing or preaching the Abhidhamma) \par
\pali{by\=akara\d na + \d nika} = \palibf{veyy\=akara\d nika}\footnote{If you are curious, when \pali{\d n-anubandha} is in operation, \pali{by\=akara\d na} $\rightarrow$ \pali{vi\=akara\d na} $\rightarrow$ \pali{veyy\=akara\d na}. See Sadd\,848--50.} (one knowing or teaching grammar) \par
\pali{sata + \d nika} = \palibf{s\=atika}\footnote{\pali{sata\d m arahat\=iti s\=atika\d m.}} (price worth 100) \par
\pali{sahassa + \d nika} = \palibf{s\=ahassika}\footnote{In Mogg\,4.28, \pali{iya} can also be used, hence \pali{sahassiya}.} (price worth 1,000) \par
\pali{ehipassa + \d nika} = \palibf{ehipassika}\footnote{\pali{`ehi pass\=a'ti ima\d m vidhi\d m arahat\=iti ehipassiko.}} (thing worth coming and seeing) \par
\pali{sandi\d t\d tha + \d nika} = \palibf{sandi\d t\d thika} (thing worth seeing by oneself) \par
\pali{antar\=aya + \d nika} = \palibf{antar\=ayika}\footnote{\pali{antar\=aya\d m karot\=iti antar\=ayiko.}} (thing causing danger) \par
\pali{pi\d n\d dap\=ata + \d nika} = \palibf{pi\d n\d dap\=atika}\footnote{\pali{pi\d n\d dap\=ata\d m u\~nchat\=iti pi\d n\d dap\=atiko.}} (one seeking alms) \par
\pali{dhamma + \d nika} = \palibf{dhammika}\footnote{\pali{dhamma\d m carat\=iti dhammiko}, or, \pali{dhamma\d m anuvattat\=iti dhammiko.}} (one practicing dhamma) \par
\pali{upasama + \d nika} = \palibf{opasamika}\footnote{\pali{kiles\=upasama\d m \=avahat\=iti upasamiko.} In Sadd\,764, \pali{upasamiko} seems incorrect.} (thing bringing calmness) \par
\pali{an\=athapi\d n\d da + \d nika} = \palibf{an\=athapi\d n\d dika}\footnote{\pali{an\=ath\=ana\d m pi\d n\d da\d m dad\=at\=iti an\=athapi\d n\d diko.}} (one giving alms to the poor) \par
\pali{urabbha + \d nika} = \palibf{orabbhika}\footnote{\pali{urabbha\d m hantv\=a j\=ivat\=iti orabbhiko.}} (one making a living by killing rams) \par
\pali{s\=ukara + \d nika} = \palibf{sokarika}\footnote{In Mogg\,4.28, \pali{ika} can have the same effect, hence \pali{s\=ukarika}.} (one making a living by killing pigs) \par
\pali{maga + \d nika} = \palibf{m\=agavika} (huntsman) \par
\pali{pakkh\=i + \d nika} = \palibf{pakkhika}\footnote{\pali{pakkhino hant\=iti pakkhiko}. (Mogg\,4.28)} (bird killer) \par
\pali{parad\=ara + \d nika} = \palibf{p\=arad\=arika}\footnote{\pali{parad\=ara\d m gacchat\=iti p\=arad\=ariko}. (Mogg\,4.28)} (one going to other's wife) \par
\pali{tila + \d nika} = \palibf{telika}\footnote{\pali{tilena sa\d msa\d t\d thi\d m bhojana\d m telika\d m.}} (food mixed with sesame seeds) \par
\pali{gu\d la + \d nika} = \palibf{go\d lika} (food mixed with sugar) \par
\pali{ghata + \d nika} = \palibf{gh\=atika} (food mixed with ghee\footnote{In Mogg\,4.29, this can mean ``food seasoned with ghee'' (\pali{ghatena abhisa\.nkhata\d m gh\=atika\d m}).}) \par
\pali{n\=av\=a + \d nika} = \palibf{n\=avika}\footnote{\pali{n\=av\=aya tarat\=iti n\=aviko.}} (sailor, one ferrying) \par
\pali{u\d lumpa + \d nika} = \palibf{o\d lumpika} (one ferrying with a raft) \par
\pali{saka\d ta + \d nika} = \palibf{s\=aka\d tika}\footnote{\pali{saka\d tena carat\=iti s\=aka\d tiko.}} (carter) \par
\pali{patta + \d nika} = \palibf{pattika} (one traveling with a bowl) \par
\pali{da\d n\d d\=i + \d nika} = \palibf{da\d n\d dika} (one traveling with a stick) \par
\pali{p\=ada + \d nika} = \palibf{p\=adika} (one traveling on foot) \par
\pali{s\=isa + \d nika} = \palibf{s\=isika}\footnote{\pali{s\=isena vahat\=iti s\=isiko.}} (one bearing things with the head) \par
\pali{a\d msa + \d nika} = \palibf{a\d msika} (one bearing things with a shoulder) \par
\pali{k\=aya + \d nika} = \palibf{k\=ayika}\footnote{k\=ayena kata\d m kamma\d m k\=ayika\d m.} (action done by the body) \par
\pali{vaca + \d nika} = \palibf{v\=acasika}\footnote{Note that \pali{vaca} and \pali{mana} are of the irregular \pali{mana}-group. That is how \pali{si} comes, I think.} (action done by speech) \par
\pali{mana + \d nika} = \palibf{m\=anasika} (action done by mind) \par
\pali{sutta + \d nika} = \palibf{suttika}\footnote{\pali{suttena baddho suttiko.}} (one tied with thread) \par
\pali{p\=asa + \d nika} = \palibf{p\=asika} (one tied with a snare) \par
\pali{vattha + \d nika} = \palibf{vatthika}\footnote{\pali{vatthena k\=ita\d m bha\d n\d da\d m vatthika\d m.}} (thing bought with cloth) \par
\pali{akkha + \d nika} = \palibf{akkhika}\footnote{\pali{akkhena dibbat\=iti akkhiko.}} (one playing dice\footnote{In Mogg\,4.29, this can also mean ``one who wins with dice'' (\pali{akkhehi jitamakkhika\d m}).}) \par
\pali{j\=ala + \d nika} = \palibf{j\=alika}\footnote{\pali{j\=alena hato j\=aliko.}} (one killed by a net\footnote{In Mogg\,4.29, this can be also in active voice, so it means ``one who kills with a net'' (\pali{j\=alena hant\=iti j\=aliko}).}) \par
\pali{kha\d nitt\=i + \d nika} = \palibf{kh\=a\d nittika}\footnote{\pali{Kha\d nittiy\=a kha\d nat\=iti kh\=a\d nittiko}. (Mogg\,4.29)} (one digging with a spade) \par
\pali{vetana + \d nika} = \palibf{vetanika}\footnote{\pali{Vetanena j\=ivat\=iti vetaniko}. (Mogg\,4.29)} (one living with wage) \par
\pali{do\d na + \d nika} = \palibf{do\d nika}\footnote{\pali{do\d no parim\=a\d namassa do\d niko v\=ihi}. (Mogg\,4.41)} (1/8th bushel of paddy) \par
\pali{r\=ajagaha + \d nika} = \palibf{r\=ajagahika}\footnote{\pali{r\=ajagahe j\=ato r\=ajagahiko}, or, \pali{r\=ajagahe vasat\=iti r\=ajagahiko}} (one born or living in R\=ajagaha) \par
\pali{magadha + \d nika} = \palibf{m\=agadha} (one born or living in Magadha) \par
\pali{s\=avatth\=i + \d nika} = \palibf{s\=avatthika} (one born or living in S\=avatth\=i) \par
\pali{sar\=ira + \d nika} = \palibf{s\=ar\=irika}\footnote{\pali{sar\=ire sannidh\=an\=a vedan\=a s\=ar\=irik\=a.}} ([feeling] based on the body) \par
\pali{dv\=ara + \d nika} = \palibf{dov\=arika}\footnote{\pali{dv\=are niyutto dov\=ariko.} In Sadd\,854, \pali{dv\=ara} becomes \pali{duara} first.} (gatekeeper) \par
\pali{buddha + \d nika} = \palibf{buddhika}\footnote{\pali{buddhe pasanno buddhiko.}} (Buddhist devotee) \par
\pali{loka + \d nika} = \palibf{lokika}\footnote{\pali{loke vidita\d m pariy\=apanna\d m lokika\d m.} It is also in Mogg\,4.30 as \pali{lok\=aya sa\d mvattat\=iti lokiko}.} (belonging to the world) \par
\pali{s\=arada + \d nika} = \palibf{s\=aradika}\footnote{\pali{s\=aradiko divaso, s\=aradik\=a ratti}. (Mogg\,4.26)} ([day or night] in autumn) \par
\pali{punabbhava + \d nika} = \palibf{ponobhavika}\footnote{\pali{punabbhav\=aya sa\d mvattat\=iti ponobhaviko}. (Mogg\,4.30)} (leading to rebirth) \par

\subparagraph*{\pali{\d Nera}} (Sadd\,759)\label{pacct2:dnera}

This denotes object of desire, for example:

\pali{vidhav\=a + \d nera} = \palibf{vedhavera}\footnote{\pali{vidhav\=aya atthiko vedhavero.}} (one desiring a widow) \par
\pali{ka\~n\~n\=a + \d nera} = \palibf{ka\~n\~nera} (one desiring a girl) \par

\subparagraph*{\pali{\d Neyya}} (Sadd\,760)\label{pacct2:S-dneyya}

\pali{suci + \d neyya} = \palibf{soceyya}\footnote{\pali{sucino bh\=avo soceyya\d m.}} (state of pureness) \par
\pali{pabbata + \d neyya} = \palibf{pabbateyya}\footnote{\pali{pabbatato pakkhad\=a nad\=i pabbateyy\=a.}} ([river] running from a mountain) \par
\pali{b\=ar\=a\d nas\=i + \d neyya} = \palibf{b\=ar\=a\d naseyya}\footnote{\pali{b\=ar\=a\d nasiya\d m bhava\d m vattha\d m b\=ar\=a\d naseyya\d m.}} ([cloth] existing in Benares) \par
\pali{kula + \d neyya} = \palibf{koleyya}\footnote{\pali{kule sa\d mva\d d\d dho sunakho koleyyo.}} ([dog] growing in a family) \par

\subparagraph*{\pali{\d Niya}} (Sadd\,761, 763)\label{pacct2:dniya}

\pali{loka + \d niya} = \palibf{lokiya}\footnote{\pali{loke vidita\d m pariy\=apanna\d m, lokena sammata\d m v\=a lokiya\d m.} \pali{\d Nika} also works in the same way, see \pali{lokika} above.} (happening in the world, happening by worldly convention) \par
\pali{inda + \d niya} = \palibf{indriya}\footnote{From Sadd\,763, this term has a number of analytic meanings, for example, \pali{indena bhagavat\=a di\d t\d th\=an\=iti indriy\=ani} (things seen by the Lord); \pali{\=adhipaccasa\.nkh\=atena indriya\d t\d then\=api indriy\=ani} (power or domination).} (faculty) \par

\subparagraph*{\pali{\d Naya}} (Sadd\,766)\label{pacct2:dnaya}

\pali{suva\d n\d na + \d naya} = \palibf{sova\d n\d naya}\footnote{\pali{suva\d n\d n\=ana\d m aya\d m r\=asi sova\d n\d nayo.}} (heap of gold) \par

\subparagraph*{\pali{\=I}} (Sadd\,784)\label{pacct2:ii}

This should be \pali{\d n\=i} because \pali{vuddhi} does happen. This reminds us that sometimes \pali{vuddhi} process is marked by other \pali{paccaya}s as well.

\pali{pura + \=i} = \palibf{por\=i}\footnote{\pali{pure bhav\=a por\=i}, or, \pali{puravadh\=una\d m v\=a es\=ati por\=i} ([speech] of city girls)} (urbane, belonging to city life) \par

\subparagraph*{\pali{\d Na}} (Mogg\,4.20, 4.22)\label{pacct2:dna}

\pali{udaka + \d na} = \palibf{odaka}\footnote{\pali{udake bhavo odako.}} (happening in water) \par
\pali{ura + \d na} = \palibf{orasa} (happening in the breast) \par
\pali{janapada + \d na} = \palibf{j\=anapada} (happening in the countryside) \par
\pali{magadha + \d na} = \palibf{m\=gadha} (happening in Magadha) \par
\pali{pur\=a + \d na} = \palibf{pur\=a\d na}\footnote{This instance is from Mogg\,4.22. It is a bit unusual, because \pali{\d na} is not elided here, and it should be \pali{por\=a\d na}. Both forms are found, but \pali{por\=a\d na} has much more frequency.} (happening in the past) \par

\subparagraph*{\pali{Tana}} (Mogg\,4.21, 4.22)\label{pacct2:tana}

\pali{ajja + tana} = \palibf{ajjatana} (happening today) \par
\pali{sve + tana} = \palibf{sv\=atana} (happening tomorrow) \par
\pali{hiyya + tana} = \palibf{hiyyattana} (happening yesterday) \par
\pali{pur\=a + tana} = \palibf{pur\=atana} (happening in the past) \par

\subparagraph*{\pali{Acca}} (Mogg\,4.23)\label{pacct2:acca}

\pali{am\=a + acca} = \palibf{amacca}\footnote{See also this entry in PTSD.} (privy councillor) \par

\subparagraph*{\pali{Ima}} (Mogg\,4.24)\label{pacct2:ima}

\pali{majjha + ima} = \palibf{majjhima} (middle, moderate) \par
\pali{anta + ima} = \palibf{antima} (last, final) \par

\subparagraph*{\pali{Ka\d na, neyya, neyyaka, ya, iya}} (Mogg\,4.25)\label{pacct2:kadna}\label{pacct2:neyya}\label{pacct2:neyyaka}\label{pacct2:ya}\label{pacct2:iya}\label{pacct2:eyyaka}

In Mogg\,4.25, \pali{eyyaka} is also added at the end.

\pali{kusin\=ar\=a + ka\d na} = \palibf{kosin\=araka}\footnote{\pali{kusin\=ar\=aya bhavo kosin\=arako.}} (happening in Kusin\=ar\=a) \par
\pali{ara\~n\~na + ka\d na} = \palibf{\=ara\~n\~naka} (happening in the forest) \par
\pali{ga\.ng\=a + neyya} = \palibf{ga\.ngeyya} (happening in the river) \par
\pali{pabbata + neyya} = \palibf{pabbateyya} (happening on the mountain) \par
\pali{vana + neyya} = \palibf{v\=aneyya} (happening in the forest) \par
\pali{kula + neyyaka} = \palibf{koleyyaka} (happening in the family) \par
\pali{g\=ama + ya} = \palibf{gamma} (happening in the village) \par
\pali{g\=ama + iya} = \palibf{g\=amiya} (happening in the village) \par
\pali{udara + iya} = \palibf{udariya} (happening in the stomach/womb) \par
\pali{mithil\=a + eyyaka} = \palibf{mithileyyaka} (happening in Mithil\=a) \par

\subparagraph*{\pali{Ttaka}} (Mogg\,4.42)\label{pacct2:ttaka}

In Mogg\,4.42, \pali{\=avataka} is also mentioned.

\pali{ya + ttaka} = \palibf{yattaka}\footnote{\pali{ya\d m parim\=a\d namassa yattaka\d m.}} (however much) \par
\pali{ta + ttaka} = \palibf{tattaka} (that much) \par
\pali{eta + ttaka} = \palibf{ettaka} (this much) \par
\pali{ya + \=avataka} = \palibf{y\=avataka} (as mush as) \par
\pali{ta + \=avataka} = \palibf{t\=avataka} (just so much) \par
\pali{eta + \=avataka} = \palibf{et\=avataka} (just this much) \par

\subparagraph*{\pali{\=Avantu}} (Mogg\,4.43)\label{pacct2:aavantu}

\pali{sabba + \=avantu} = \palibf{sabb\=avantu}\footnote{\pali{sabba\d m parim\=a\d namassa sabb\=avanta\d m.}} (total amount) \par
\pali{ya + \=avantu} = \palibf{y\=avantu} (as many as) \par
\pali{ta + \=avantu} = \palibf{t\=avantu} (as that amount) \par
\pali{eta + \=avantu} = \palibf{et\=avantu} (as this amount) \par

\subparagraph*{\pali{Rati, r\=iva, r\=ivataka, rittaka}} (Mogg\,4.44)\label{pacct2:rati}\label{pacct2:riiva}\label{pacct2:riivataka}\label{pacct2:rittaka}

We do not see \pali{r} in the end products because it is a sign of elision. This \pali{r-anubandha} causes \pali{i\d m} in \pali{ki\d m} to be deleted, \pali{r\=anubandhatt\=a i\d mbh\=agalopo}.\footnote{Payo\,411, see also Kacc\,539, R\=upa\,558, Sadd\,1124, Niru\,500.}

\pali{ki\d m + rati} = \palibf{kati}\footnote{\pali{ki\d m sa\.nkhy\=ana\d m parim\=a\d namesa\d m kati ete.}} (how many) \par
\pali{ki\d m + r\=iva} = \palibf{k\=iva} (how many) \par
\pali{ki\d m + r\=ivataka} = \palibf{k\=ivataka} (how many) \par
\pali{ki\d m + rittaka} = \palibf{kittaka} (how many) \par

\subparagraph*{\pali{Ita}} (Mogg\,4.45)\label{pacct2:ita}

\pali{t\=arak\=a + ita} = \palibf{t\=arakita}\footnote{\pali{t\=arak\=a sa\~nj\=at\=a assa t\=arakita\d m, gagana\d m.}} ([sky] endowed with stars) \par
\pali{puppha + ita} = \palibf{pupphita} ([tree] endowed with flowers) \par

\subparagraph*{\pali{Matta}} (Mogg\,4.46)\label{pacct2:matta}

\pali{hattha + matta} = \palibf{hatthamatta}\footnote{\pali{hattho pam\=a\d namassa hatthamatta\d m.}} (a handful) \par
\pali{sata + matta} = \palibf{satamatta} (amount of 100) \par
\pali{do\d na + matta} = \palibf{do\d namatta} (amount of 1/8 bushel) \par

\subparagraph*{\pali{Taggha}} (Mogg\,4.47, 4.48)\label{pacct2:taggha}

This is used to specify height. Also, \pali{\d na} and \pali{matta} can be used in the same way with \pali{purisa} (Mogg\,4.48), i.e.\ \pali{porisa} and \pali{purisamatta}.

\pali{ja\d n\d nu + taggha} = \palibf{ja\d n\d nutaggha}\footnote{Also \pali{ja\d n\d numatta} has the same meaning.} (as high as the knee) \par
\pali{purisa + taggha} = \palibf{purisataggha} (as high as a man) \par

\subparagraph*{\pali{\d Neyya}} (Mogg\,4.76)\label{pacct2:M-dneyya}

\pali{dakkhi\d na + \d neyya} = \palibf{dakkhi\d neyya}\footnote{\pali{dakkhi\d na\d m arahat\=iti dakkhi\d neyyo.}} (worth offering) \par

\subparagraph*{\pali{R\=aya}} (Mogg\,4.77)\label{pacct2:raaya}

This is used with \pali{-tu\d m}, but \pali{r-anubandha} (see above) causes \pali{u\d m} to be deleted.

\pali{gh\=atetu\d m + r\=aya} = \palibf{gh\=atet\=aya} (worth killing) \par
\pali{pabb\=ajetu\d m + r\=aya} = \palibf{pabb\=ajet\=aya} (worth having to go forth) \par

\subsection*{3.\ \pali{R\=ag\=aditaddhita}}\label{tadgroup3}

This group is mainly about coloring or tinting (\pali{r\=aga}), and it also means many things like above. Kacc and Sadd give us only \pali{\d na}, but \pali{ima} is also given somewhere else. Mogg adds more six, namely \pali{\d nika, kiya, niya, ka, ya,} and \pali{ima}.

\subparagraph*{\pali{\d Na}} (Kacc\,352, R\=upa\,376, Sadd\,765, Mogg\,4.11--9, 4.34)\label{pacct3:dna}

\pali{kas\=ava + \d na} = \palibf{k\=as\=ava}\footnote{\pali{k\=as\=avena ratta\d m vattha\d m k\=as\=ava\d m.}} ([cloth] dyed with orange color) \par
\pali{kusumbha + \d na} = \palibf{kosambha} ([cloth] dyed with safflower) \par
\pali{halidd\=a + \d na} = \palibf{h\=alidda} ([cloth] dyed with turmeric) \par
\pali{ku\.nkuma + \d na} = \palibf{ku\.nkuma} ([cloth] dyed with saffron) \par
\pali{s\=ukara + \d na} = \palibf{sokara}\footnote{\pali{s\=ukarassa ima\d m ma\d msa\d m sokara\d m.}} ([meat] of pig) \par
\pali{mahisa + \d na} = \palibf{m\=ahisa} ([meat] of baffalo) \par
\pali{udumbara + \d na} = \palibf{odumbara}\footnote{\pali{udumbarassa avid\=ure vim\=ana\d m odumbara\d m.}} ([mansion] not far from a fig tree) \par
\pali{vidis\=a + \d na} = \palibf{vedisa}\footnote{\pali{vidis\=aya avid\=ure niv\=aso vediso.}} ([house] not far from a minor direction, e.g.\ Northeast) \par
\pali{mathur\=a + \d na} = \palibf{m\=athura}\footnote{\pali{mathur\=aya j\=ato m\=athuro}, or, \pali{mathur\=aya \=agato m\=athuro}, or, \pali{mathur\=aya assa niv\=aso m\=athuro}, or, \pali{mathur\=aya issaro niv\=aso m\=athuro}} (one born in, came from, living in, or having power in Mathur\=a) \par
\pali{kapilavatthu + \d na} = \palibf{k\=apilavattha}\footnote{\pali{kapilavatthusam\=ipe j\=ata\d m vana\d m k\=apilavattha\d m.}} ([forest] near to Kapila\-vatthu) \par
\pali{kattik\=a + \d na} = \palibf{kattika}\footnote{\pali{kattik\=aya niyutto m\=aso kattiko.}} (month assosiated with the moon passing Kattik\=a constellation, November) \par
\pali{magasira + \d na} = \palibf{m\=agasira} (with Magasira, December) \par
\pali{phussa + \d na} = \palibf{phussa}\footnote{In Mogg\,4.12, there are examples, \pali{phuss\=i ratti, phussa\d m aha\d m} (a night and day in the period of Phussa).} (with Phussa, January) \par
\pali{magh\=a + \d na} = \palibf{m\=agha} (with Magh\=a, February) \par
\pali{phaggun\=i + \d na} = \palibf{phagguna} (with Phaggun\=i, March) \par
\pali{citt\=a + \d na} = \palibf{citto} (with Citt\=a, April) \par
\pali{vis\=akh\=a + \d na} = \palibf{vis\=akha} (with Vis\=akh\=a, May) \par
\pali{je\d t\d th\=a + \d na} = \palibf{je\d t\d tha} (with Je\d t\d th\=a, June) \par
\pali{\=as\=a\d lh\=a + \d na} = \palibf{\=as\=a\d lha} (with \=As\=a\d lh\=a, July) \par
\pali{sava\d na + \d na} = \palibf{s\=ava\d na} (with Sava\d na, August) \par
\pali{bhadda + \d na} = \palibf{bhadda} (with Bhadda, September) \par
\pali{assayuja + \d na} = \palibf{assayuja} (with Assayuja, October) \par
\pali{sikkh\=a + \d na} = \palibf{sikkha}\footnote{\pali{sikkh\=ana\d m sam\=uho sikkho.}} (group of rules) \par
\pali{buddha + \d na} = \palibf{buddha}\footnote{\pali{buddho assa devat\=ati buddho.}} (having the Buddha as a god) \par
\pali{yama + \d na} = \palibf{y\=ama} (having Yama as a god) \par
\pali{soma + \d na} = \palibf{soma} (having the Moon as a god) \par
\pali{sa\d mvacchara + \d na} = \palibf{sa\d mvacchara}\footnote{\pali{sa\d mvaccharamadh\=ite sa\d mvaccharo.}} (one studying year [time calculation]) \par
\pali{nimitta + \d na} = \palibf{nemitta} (one studying omens) \par
\pali{muhutta + \d na} = \palibf{mohutta} (one studying horary astrology) \par
\pali{a\.ngavijja + \d na} = \palibf{a\.ngavijja} (one studying fortunetelling) \par
\pali{veyy\=akara\d na + \d na} = \palibf{veyy\=akara\d na} (one studying grammar) \par
\pali{chanda + \d na} = \palibf{chanda} (one studying prosody) \par
\pali{vas\=ada + \d na} = \palibf{v\=as\=ada}\footnote{\pali{vas\=ad\=ana\d m visayo deso v\=as\=ado.}} (region of Vas\=ada) \par
\pali{udumbara + \d na} = \palibf{odumbara}\footnote{\pali{udumbar\=a assmi\d m padese sant\=iti odumbaro.}} ([country] having fig trees) \par

\subparagraph*{\pali{\d Nika, kiya, niya, ka}} (Mogg\,4.33)\label{pacct3:dnika}\label{pacct3:kiya}\label{pacct3:niya}\label{pacct3:ka}

\pali{sa\d mgha + \d nika} = \palibf{sa\d mghika}\footnote{\pali{sa\d mghassa ida\d m sa\d mghika\d m.}} ([thing] belonging to the Order) \par
\pali{puggala + \d nika} = \palibf{puggalika} ([thing] belonging to a person) \par
\pali{para + kiya} = \palibf{parakiya} ([thing] belonging to other person) \par
\pali{atta + niya} = \palibf{attaniya} ([thing] belonging to oneself) \par
\pali{sa + ka} = \palibf{saka} (one's own) \par
\pali{r\=aja + ka} = \palibf{r\=ajaka} ([thing] belonging to the king) \par

\subparagraph*{\pali{Ya}} (Mogg\,4.35)\label{pacct3:ya}

\pali{go + ya} = \palibf{gabya}\footnote{\pali{gunna\d m ida\d m gabya\d m.}} ([thing] belonging to the cattle) \par

\subparagraph*{\pali{Ima}} (Mogg\,4.63, Sadd\,1276)\label{pacct3:ima}

\pali{p\=aka + ima} = \palibf{p\=akima}\footnote{\pali{p\=akena nibbatta\d m p\=akima\d m.}} ([thing] produced by cooking) \par
\pali{seka + ima} = \palibf{sekima} ([thing] produced by sprinkling) \par
\pali{kutti + ima} = \palibf{kuttima}\footnote{\pali{kara\d na\d m kutti, kuttiy\=a nibbatta\d m kuttima\d m.} (Sadd\,1276)} ([thing] produced by doing) \par

\subsection*{4.\ \pali{J\=at\=aditaddhita}}\label{tadgroup4}

This group denotes things that are born, and means some other things. There are four \pali{paccaya}s in this group, namely \pali{ima, iya, ika,} and \pali{kiya}.

\subparagraph*{\pali{Ima, iya, ika, kiya}} (Kacc\,353, R\=upa\,378, Sadd\,767--9)\label{pacct4:ima}\label{pacct4:iya}\label{pacct4:ika}\label{pacct4:kiya}

\pali{pacch\=a + ima} = \palibf{pacchima}\footnote{\pali{pacch\=a j\=ato pacchimo.}} (one born after) \par
\pali{anta + ima} = \palibf{antima} (one born last) \par
\pali{majjha + ima} = \palibf{majjhima} (one born in the middle) \par
\pali{pura + ima} = \palibf{purima} (one born before) \par
\pali{bodhisattaj\=ati + iya} = \palibf{bodhisattaj\=atiya}\footnote{\pali{bodhisattaj\=atiy\=a j\=ato bodhisattaj\=atiyo.}} (one born as a Boddhisatta) \par
\pali{assaj\=ati + iya} = \palibf{assaj\=atiya} (one born as a horse) \par
\pali{manussaj\=ati + iya} = \palibf{manussaj\=atiya} (one born as a human being) \par
\pali{putta + ima} = \palibf{puttima}\footnote{\pali{putto assa atth\=iti puttimo.} Also \pali{puttiyo} and \pali{puttiko} have the same meaning.} (one having a child) \par

\subsection*{5.\ \pali{Sam\=uhataddhita}}\label{tadgroup5}

This group denotes gathering or collection of things. In Kacc and Sadd, there are three \pali{paccaya}s: \pali{ka\d n, \d na, t\=a}. In Mogg \pali{ka\d na} is given instead and \pali{\d nika} is added. Since these are used in other meaning as well, so be careful and do not haste to conclusion. For example, \pali{m\=anussaka} can also mean ``belonging to human beings'' (see Sadd\,770).

\subparagraph*{\pali{Ka\d n, ka\d na, \d na}} (Kacc\,354, R\=upa\,379, Sadd\,770, Mogg\,4.68)\label{pacct5:kadn}\label{pacct5:kadna}\label{pacct5:dna}

\pali{r\=ajaputta + ka\d n/\d na} = \palibf{r\=ajaputtaka/r\=ajaputta}\footnote{\pali{r\=ajaputt\=ana\d m sam\=uho r\=ajaputtako r\=ajaputto v\=a.}} (group of princes) \par
\pali{manussa + ka\d n/\d na} = \palibf{m\=anussaka/m\=anussa} (group of human beings) \par
\pali{may\=ura + ka\d n/\d na} = \palibf{m\=ay\=uraka/m\=ay\=ura} (group of peacocks) \par
\pali{k\=aka + \d na} = \palibf{k\=aka} (group of crows) \par

\subparagraph*{\pali{T\=a}} (Kacc\,355, R\=upa\,380, Sadd\,771, Mogg\,4.69)\label{pacct5:taa}

\pali{g\=ama + t\=a} = \palibf{g\=amat\=a}\footnote{\pali{g\=am\=ana\d n sam\=uho g\=am\=ana\d m.}} (group of villages) \par
\pali{jana + t\=a} = \palibf{janat\=a} (group of people) \par
\pali{bandhu + t\=a} = \palibf{bandhut\=a} (group of relatives) \par
\pali{sah\=aya + t\=a} = \palibf{sah\=ayat\=a} (group of friends) \par
\pali{n\=agara + t\=a} = \palibf{n\=agarat\=a} (group of city dwellers) \par

As noted in Sadd\,772, sometimes \pali{t\=a} does not change the meaning of the words, for example, \pali{devat\=a} = \pali{devo}, \pali{idappaccayat\=a} = \pali{idappaccay\=a}, and \pali{disat\=a} = \pali{dis\=a}.

\subsection*{6.\ \pali{\d Th\=anataddhita}}\label{tadgroup6}

This group points to base or cause or location of things. In Kacc \pali{iya, \=ayitta,} and \pali{la} are given; in Sadd \pali{iya, \=iya, eyya, \=ayitta,} and \pali{la}; in Mogg \pali{iya, lla, illa}.

\subparagraph*{\pali{Iya}} (Kacc\,356, R\=upa\,381, Sadd\,773--4, Mogg\,4.70)\label{pacct6:iya}

\pali{madana + iya} = \palibf{madaniya}\footnote{\pali{madanassa \d th\=ana\d m madaniya\d m.}} (cause of intoxication) \par
\pali{bandhana + iya} = \palibf{bandhaniya} (cause of attachment) \par
\pali{mucchana + iya} = \palibf{mucchaniya} (cause of obsession) \par
\pali{up\=ad\=ana + iya} = \palibf{up\=ad\=aniya}\footnote{\pali{up\=ad\=ana\d m hita\d m up\=ad\=aniya\d m.}} (contributing to attachment) \par

\subparagraph*{\pali{\=Iya, eyya}} (Sadd\,775)\label{pacct6:iiya}\label{pacct6:eyya}

\pali{dassana + iya/eyya} = \palibf{dassan\=iya/dassaneyya}\footnote{\pali{dassana\d m arahat\=iti dassan\=iya\d m, r\=upa\d m.}} ([image] worth seeing) \par
\pali{vandana + iya/eyya} = \palibf{vandan\=iya/vandaneyya} ([thing/ person] worth saluting) \par
\pali{p\=ujana + iya/eyya} = \palibf{p\=ujan\=iya/p\=ujaneyya} ([thing/person] worth venerating) \par

\subparagraph*{\pali{\=Ayitta}} (Kacc\,357, R\=upa\,382, Sadd\,777)\label{pacct6:aayitta}

\pali{dh\=uma + \=ayitta} = \palibf{dh\=um\=ayitta}\footnote{\pali{dh\=umo viya dissati adu\d m tayida\d m dh\=um\=ayitta\d m.}} ([place] seemingly hazy) \par
\pali{timira + \=ayitta} = \palibf{timir\=ayitta} ([place] seemingly dark) \par

\subparagraph*{\pali{La, lla, illa}} (Kacc\,358, R\=upa\,383, Sadd\,778, Mogg\,4.65)\label{pacct6:la}\label{pacct6:lla}\label{pacct6:illa}

\pali{du\d t\d thu + la} = \palibf{du\d t\d thulla}\footnote{\pali{u\d t\d thu\d t\d th\=ana\d m u\d t\d thulla\d m}, or, \pali{u\d t\d thu nissita\d m u\d t\d thulla\d m.} In Mogg it is \pali{lla} not just \pali{la}.} (cause of badness, [action] depending on badness) \par
\pali{veda + la} = \palibf{vedalla} (cause of insight, depending on insight) \par
\pali{sa\.nkh\=ara + illa} = \palibf{sa\.nkh\=arilla}\footnote{Mogg\,4.65} (depending on conditioned formation) \par

\subsection*{7.\ \pali{Bahulataddhita}}\label{tadgroup7}

This group denotes multitude of things. Only \pali{\=alu} is given here. In Mogg\,4.85 \pali{\=alu} is classified as \pali{Tadassatthitaddhita} meaning ``having \ldots,'' for example, \palibf{abhijjh\=alu} = ``having covetousness,'' \palibf{s\=it\=alu} = ``having coolness.''

\subparagraph*{\pali{\=Alu}} (Kacc\,359, R\=upa\,384, Sadd\,779)\label{pacct7:aalu}

\pali{abhijjh\=a + \=alu} = \palibf{abhijjh\=alu}\footnote{\pali{abhijjh\=a assa pakati abhijjh\=alu, abhijjh\=a assa bahul\=a v\=a abhijjh\=alu.}} ([one] usually or very covetous) \par
\pali{s\=ita + \=alu} = \palibf{s\=it\=alu} ([place] usually or very cool) \par
\pali{dhaja + \=alu} = \palibf{dhaj\=alu} ([place] usually having a flag or many flags) \par
\pali{day\=a + \=alu} = \palibf{day\=alu} ([one] usually or very kind) \par

\subsection*{8.\ \pali{Bh\=avataddhita}}\label{tadgroup8}

This group expresses states of being. In Kacc six \pali{paccaya}s are mentioned: \pali{\d nya, tta, t\=a, ttana, \d na,} and \pali{ka\d n}. In Sadd other four are added: \pali{bya, \d neyya, \d niya,} and \pali{\d lhaka}. In addition, \pali{ima} is mentioned elsewhere. In Mogg yet other three are added: \pali{iya, na\d na,} and \pali{ima}.

\subparagraph*{\pali{\d Nya}} (Kacc\,360, R\=upa\,387, Sadd\,780, Mogg\,4.127)\label{pacct8:dnya}

\pali{alasa + \d nya} = \palibf{\=alasya}\footnote{alasassa bh\=avo \=alasya\d m.} (laziness) \par
\pali{aroga + \d nya} = \palibf{\=arogya} (state devoid of sickness) \par
\pali{brahma + \d nya} = \palibf{brahma\~n\~na} (brahmanhood) \par
\pali{sama\d na + \d nya} = \palibf{s\=ama\~n\~na} (ascetichood) \par
\pali{r\=aja + \d nya} = \palibf{rajja} (kinghood) \par
\pali{kus\=ita + \d nya} = \palibf{kosajja} (laziness) \par
\pali{uju + \d nya} = \palibf{ajjava} (straightness) \par
\pali{suhada + \d nya} = \palibf{sohajja} (friendship) \par
\pali{mudu + \d nya} = \palibf{maddava} (softness) \par
\pali{isi + \d nya} = \palibf{\=arissa}\footnote{\pali{isino ida\d m bh\=avo v\=a \=arissa\d m}, Mogg\,4.127. It is \pali{\=arissya} in Sadd\,857.} (sagehood) \par
\pali{\=ajana? + \d nya} = \palibf{\=aja\~n\~na}\footnote{\pali{\=aj\=an\=iyassa bh\=avo so eva v\=a \=aja\~n\~na\d m}, Mogg\,4.127. In PTSD, this is the contracted form of \pali{\=aj\=aniya}.} (state of being a good bleed) \par
\pali{thena + \d nya} = \palibf{theyya}\footnote{\pali{thenassa bh\=avo kamma\d m v\=a theyya\d m}, Mogg\,4.127.} (theft) \par
\pali{bahussata + \d nya} = \palibf{bahusacca}\footnote{\pali{bahussatassa bh\=avo b\=ahusacca\d m}, Mogg\,4.127.} (state of being erudite) \par

\subparagraph*{\pali{Tta, t\=a, ttana}} (Kacc\,360, R\=upa\,387, Sadd\,780, Mogg\,4.59)\label{pacct8:tta}\label{pacct8:taa}\label{pacct8:ttana}

\pali{pa\d msuk\=ulika + tta} = \palibf{pa\d msuk\=ulikatta}\footnote{\pali{pa\d msuk\=ulikassa bh\=avo pa\d msuk\=ulikatta\d m.}} (state of being one wearing discarded robes) \par
\pali{n\=ila + tta} = \palibf{n\=ilatta} (blueness) \par
\pali{da\d n\d d\=i + tta} = \palibf{da\d n\d ditta} (state of being one holding a stick) \par
\pali{canda + tta} = \palibf{candatta} (state of being the Moon) \par
\pali{go + tta} = \palibf{gotta} (state of being an ox) \par
\pali{nidd\=ar\=ama + t\=a} = \palibf{nidd\=ar\=amat\=a}\footnote{\pali{nidd\=ar\=amassa bh\=avo nidd\=ar\=amat\=a.}} (state of being one who is delighted in sleeping) \par
\pali{kamma\~n\~na + t\=a} = \palibf{kamma\~n\~nat\=a} (state of being fit to work) \par
\pali{lahu + t\=a} = \palibf{lahut\=a} (lightness) \par
\pali{n\=ila + t\=a} = \palibf{n\=ilat\=a} (blueness) \par
\pali{go + t\=a} = \palibf{got\=a} (state of being an ox) \par
\pali{puthujjana + ttana} = \palibf{puthujjanattana}\footnote{\pali{puthujjanassa bh\=avo puthujjanattana\d m.}} (state of being a worldly person) \par
\pali{vedan\=a + ttana} = \palibf{vedanattana} (state of feeling) \par
\pali{j\=ay\=a + ttana} = \palibf{j\=ayattana} (state of being a wife) \par

\subparagraph*{\pali{\d Na}} (Kacc\,361, R\=upa\,388, Sadd\,781, Mogg\,4.59, 4.127)\label{pacct8:dna}

\pali{visama + \d na} = \palibf{vesama}\footnote{\pali{visamassa bh\=avo vesamam\d m.}} (state of being uneven) \par
\pali{suci + \d na} = \palibf{soca} (state of being clean) \par

\subparagraph*{\pali{Ka\d n}} (Kacc\,362, R\=upa\,389, Sadd\,782)\label{pacct8:kadn}

\pali{rama\d n\=iya + ka\d n} = \palibf{r\=aman\=iyaka}\footnote{\pali{rama\d n\=iya bh\=avo r\=ama\d n\=iyaka\d m.}} (state of being delightful) \par
\pali{manu\~n\~na + ka\d n} = \palibf{m\=anu\~n\~naka} (state of being pleasant) \par

\subparagraph*{\pali{Bya}} (Sadd\,780, Mogg\,4.60)\label{pacct8:bya}

\pali{d\=asa + bya} = \palibf{d\=asabya}\footnote{\pali{d\=asassa bh\=avo d\=asabya\d m.}} (state of being a slave) \par
\pali{vaddha + bya} = \palibf{vaddhabya} (state of being old) \par

\subparagraph*{\pali{\d Neyya}} (Sadd\,781, Mogg\,4.59)\label{pacct8:dneyya}

\pali{suci + \d neyya} = \palibf{soceyya} (state of being clean) \par
\pali{adhipati + \d neyya} = \palibf{adhipateyya} (state of being a ruler) \par

\subparagraph*{\pali{\d Niya}} (Sadd\,762, Mogg\,4.59)\label{pacct8:dniya}

\pali{alasa + \d niya} = \palibf{\=alasiya}\footnote{\pali{alasassa bh\=avo \=alasiya\d m.}} (laziness) \par
\pali{kalusa + \d niya} = \palibf{k\=alusiya}\footnote{In Mogg\,4.59 it is \pali{k\=a\d lusiya\d m}.} (impurity, dirtiness) \par

\subparagraph*{\pali{Iya}} (Mogg\,4.59)\label{pacct8:iya}

\pali{adhipati + iya} = \palibf{adhipatiya} (state of being a ruler) \par
\pali{pa\d n\d dita + iya} = \palibf{pa\d n\d ditiya} (state of being a wise person) \par
\pali{bahussuta+ iya} = \palibf{bahussutiya} (state of being a learned person) \par
\pali{nagga+ iya} = \palibf{naggiya} (state of being naked) \par
\pali{s\=ura+ iya} = \palibf{s\=uriya} (state of being courageous) \par

\subparagraph*{\pali{Na\d na}} (Mogg\,4.61)\label{pacct8:nadna}

\pali{yuva + na\d na} = \palibf{yobbana}\footnote{This can also be with other \pali{paccaya}s, i.e.\ \pali{yuvatta}, \pali{yuvat\=a}.} (state of being a youth) \par

\subparagraph*{\pali{Ima}} (Mogg\,4.62, Sadd\,1277)\label{pacct8:ima}

\pali{a\d nu + ima} = \palibf{a\d nim\=a} (state of being small) \par
\pali{mah\=a + ima} = \palibf{mahim\=a} (state of being big) \par
\pali{lahu + ima} = \palibf{lahim\=a/laghim\=a} (state of being light) \par

\subparagraph*{\pali{\d Lhaka}} (Sadd\,840)\label{pacct8:dlhaka}

\pali{dve + \d lhaka} = \palibf{dve\d lhaka}\footnote{\pali{dvebh\=avo dve\d lhaka\d m.} In PTSD this means `doubt.' It may be the sense of uncertainty between two states.} (state of being two) \par

\subsection*{9.\ \pali{Visesataddhita}}\label{tadgroup9}\label{par:visesataddhita}\label{pacct9:tara}\label{pacct9:tama}\label{pacct9:isika}\label{pacct9:issika}\label{pacct9:iya}\label{pacct9:idtdtha}

This group expresses distinction (\pali{visesa}). There are unanimously five \pali{paccaya}s, namely \pali{tara, tama, isika (issika), iya,} and \pali{i\d t\d tha}.\footnote{Kacc\,363, R\=upa\,390, Sadd\,786, Mogg\,4.64} All these are used in comparison (see Chapter \ref{chap:adjcomp}).

\pali{p\=apa + tara} = \palibf{p\=apatara}\footnote{\pali{sabbe ime p\=ap\=a, ayamimesa\d m visesena p\=apoti p\=apataro} (In all these evil people, this person is distinctively evil, thus more evil).} (more evil [person]) \par
\pali{p\=apa + tama} = \palibf{p\=apatama} (the most evil [person]) \par
\pali{p\=apa + isika} = \palibf{p\=apisika} (more evil [person]) \par
\pali{p\=apa + iya} = \palibf{p\=apiya} (more evil [person]) \par
\pali{p\=apa + i\d t\d tha} = \palibf{p\=api\d t\d tha} (the most evil [person]) \par

\subsection*{10.\ \pali{Tadassatthitaddhita}}\label{tadgroup10}

This group has the sense of one's possession of things. In Kacc, there are nine \pali{paccaya}s: \pali{v\=i, sa, s\=i, ika, \=i, ra, vantu, mantu,} and \pali{\d na}. In Sadd other two are added: \pali{imantu,} and \pali{ta}. And in Mogg, yet other eleven are added: \pali{a, ss\=i, bha, ila, va, \=am\=i, uv\=am\=i, na, ima,} and \pali{iya}.

\subparagraph*{\pali{V\=i}} (Kacc\,364, R\=upa\,398, Sadd\,787, Mogg\,4.89)\label{pacct10:vii}

\pali{medh\=a + v\=i} = \palibf{medh\=av\=i}\footnote{\pali{medh\=a yassa atth\=iti medh\=av\=i.}} (one having wisdom) \par
\pali{m\=ay\=a + v\=i} = \palibf{m\=ay\=av\=i} (one having deceit) \par

\subparagraph*{\pali{Sa}} (Kacc\,364, R\=upa\,398, Sadd\,788, Mogg\,4.93)\label{pacct10:sa}

\pali{sumedh\=a + sa} = \palibf{sumedhasa}\footnote{\pali{sumedh\=a yassa atth\=iti sumedhaso.}} (one having wisdom) \par
\pali{bh\=urimedh\=a + sa} = \palibf{bh\=urimedhasa} (one having great wisdom) \par
\pali{loma + sa} = \palibf{lomasa} (one having hair) \par

\subparagraph*{\pali{S\=i, ss\=i}} (Kacc\,365, R\=upa\,399, Sadd\,789, Mogg\,4.81)\label{pacct10:sii}\label{pacct10:ssii}

\pali{tapa + s\=i} = \palibf{tapass\=i}\footnote{\pali{tapo yassa atth\=iti tapass\=i.}} (one practicing austerity) \par
\pali{yasa + s\=i} = \palibf{yasass\=i} (one having fame) \par
\pali{teja + s\=i} = \palibf{tejass\=i}\footnote{In Sadd\,789, it is \pali{tejas\=i}. Aggava\d msa maintains that only this form is found in the canon (\pali{p\=a\d lipotthakesu pana `tejas\=i'ti nissa\~n\~nogapadameva \=agata\d m}). As far as I know, there are both forms in the canon.} (one having power) \par
\pali{mana + s\=i} = \palibf{manass\=i} (one having mind) \par
\pali{paya + s\=i} = \palibf{payass\=i} (one having milk) \par

As you might notice, the terms in above examples are all of \pali{mana}-group. That can explain why \pali{s} plays a role here. See page \pageref{decl:mana}.

\subparagraph*{\pali{Ika, \=i}} (Kacc\,366, R\=upa\,400, Sadd\,790, Mogg\,4.80)\label{pacct10:ika}\label{pacct10:ii}

\pali{da\d n\d da + ika/\=i} = \palibf{da\d n\d dika/da\d n\d d\=i}\footnote{\pali{da\d n\d do yassa atth\=iti da\d n\d diko, da\d n\d d\=i.}} (one having a stick) \par
\pali{m\=al\=a + ika/\=i} = \palibf{m\=alika/m\=al\=i} (one having a garland) \par
\pali{r\=upa + ika/\=i} = \palibf{r\=upika/r\=up\=i} (one having a good look) \par
\pali{dhana + ika/\=i} = \palibf{dhanika/dhan\=i} (one having wealth) \par

\subparagraph*{\pali{Ra}} (Kacc\,367, R\=upa\,401, Sadd\,791, Mogg\,4.82)\label{pacct10:ra}

\pali{madhu + ra} = \palibf{madhura}\footnote{\pali{madhu yassa atth\=iti madhuro.}} (thing having sweet taste) \par
\pali{ku\~nja + ra} = \palibf{ku\~njara}\footnote{In a Thai explanation, this means a being that has a chin, thus elephant. I have not yet found the source of this.} (elephant) \par
\pali{mukha + ra} = \palibf{mukhara} (one having a mouth, talking a lot) \par
\pali{susi? + ra} = \palibf{susira} (thing having holes) \par
\pali{naga + ra} = \palibf{nagara} (place having mountains, city)\footnote{I do not understand the logic of this.} \par

\subparagraph*{\pali{Vantu}} (Kacc\,368, R\=upa\,402, Sadd\,792, Mogg\,4.79)\label{pacct10:vantu}

This \pali{paccaya} is mostly added to terms ending with \pali{a} or \pali{\=a}, otherwise \pali{mantu} is used.

\pali{gu\d na + vantu} = \palibf{gu\d navantu}\footnote{\pali{gu\d no yassa atth\=iti gu\d nav\=a.} For declension of this irregular term and its kin, see Chapter \ref{chap:irrn}, and Appendix \ref{chap:decl}, page \pageref{decl:gunavm} onwards.} (one having virtue) \par
\pali{yasa + vantu} = \palibf{yasavantu} (one having fame) \par
\pali{dhana + vantu} = \palibf{dhanavantu} (one having wealth) \par
\pali{pa\~n\~n\=a + vantu} = \palibf{pa\~n\~navantu} (one having wisdom) \par

\subparagraph*{\pali{Mantu}} (Kacc\,369, R\=upa\,403, Sadd\,793, Mogg\,4.78)\label{pacct10:mantu}

\pali{sati + mantu} = \palibf{satimantu}\footnote{\pali{sati yassa atth\=iti satim\=a.}} (one being mindful) \par
\pali{juti + mantu} = \palibf{jutimantu} (one having brightness) \par
\pali{dhiti + mantu} = \palibf{dhitimantu} (one having wisdom) \par
\pali{cakkhu + mantu} = \palibf{cakkhumantu} (one having eyes) \par
\pali{\=ayu + mantu} = \palibf{\=ayasmantu}\footnote{\pali{\=ayu assa atth\=iti \=ayasm\=a.} For how \pali{u} becomes \pali{as}, see Kacc\,371, R\=upa\,404, Sadd\,797, Mogg\,4.134.} (one having age) \par
\pali{go + mantu} = \palibf{gomantu} (one having cattle) \par

\subparagraph*{\pali{\d Na, a}} (Kacc\,370, R\=upa\,405, Sadd\,795, Mogg\,4.84--5)\label{pacct10:K-dna}\label{pacct10:a}

\pali{saddh\=a + \d na} = \palibf{saddha}\footnote{\pali{saddh\=a yassa atth\=iti saddho.}} (one having faith) \par
\pali{pa\~n\~n\=a + \d na} = \palibf{pa\~n\~na} (one having wisdom) \par
\pali{tapa + \d na} = \palibf{t\=apasa}\footnote{The feminine term of this is \pali{t\=apas\=i}.} (one practicing austerity) \par

\subparagraph*{\pali{Imantu}} (Sadd\,794)\label{pacct10:imantu}

\pali{canda + imantu} = \palibf{candimantu}\footnote{\pali{candavim\=anasa\.nkh\=ato cando assa atth\=iti candim\=a, candadevaputto.}} (one having the moon as a mension, the lunar god) \par
\pali{putta + imantu} = \palibf{puttimantu}\footnote{\pali{putt\=a assa atth\=iti puttim\=a, bahuputto.}} (one having many children) \par
\pali{p\=apa + imantu} = \palibf{p\=apimantu}\footnote{\pali{p\=apa\d m assa atth\=iti p\=apim\=a, k\=amadevo.}} (one having sin, the god of pleasure) \par

\subparagraph*{\pali{Ta}} (Sadd\,796)\label{pacct10:ta}

\pali{pabba + ta} = \palibf{pabbata}\footnote{\pali{pabba\d m assa atthi pabbato, giri.}} (thing having section, mountain) \par
\pali{va\.nka + ta} = \palibf{va\.nkata}\footnote{\pali{va\.nka\d m sa\d n\d th\=ana\d m assa atth\=iti va\.nkato.}} (thing having crooked shaped, name of a mountain) \par

\subparagraph*{\pali{Bha}} (Mogg\,4.83)\label{pacct10:bha}

\pali{tundi + bha} = \palibf{tundibha}\footnote{\pali{tundi vuccati vuddh\=a n\=abhi, tundibho}. (from Niru\,480)} (one having protruded navel) \par
\pali{vali + bha} = \palibf{valibha} (one having wrinkled skin) \par

\subparagraph*{\pali{Ila}} (Mogg\,4.87)\label{pacct10:lla}

\pali{piccha + ila} = \palibf{picchila}\footnote{\pali{piccha\d m t\=ula\d m assa atthi, tasmi\d m v\=a vijjat\=iti picchilo}. (Niru\,483)} ([cotton] having a pod) \par
\pali{phena + ila} = \palibf{phenila} ([water] having foam) \par
\pali{ja\d t\=a + ila} = \palibf{ja\d tila} (one having matted hair) \par
All these can also be fit with \pali{-vantu}, hence \pali{picchav\=a, pheniv\=a, ja\d t\=av\=a}.

\subparagraph*{\pali{Va}} (Mogg\,4.88)\label{pacct10:va}

\pali{s\=ila + va} = \palibf{s\=ilava} (one having virtue) \par
\pali{kesa + va} = \palibf{kesava} (one having hair) \par
Using \pali{-vantu} also works likewise, hence \pali{s\=ilavantu, kesavantu}.

\subparagraph*{\pali{\=Am\=i, uv\=am\=i}} (Mogg\,4.90)\label{pacct10:aamii}\label{pacct10:uvaamii}

\pali{sa + \=am\=i/uv\=am\=i} = \palibf{s\=am\=i/suv\=am\=i} (master, husband) \par

\subparagraph*{\pali{\d Na}} (Mogg\,4.91)\label{pacct10:M-dna}

\pali{lakkh\=i + \d na} = \palibf{lakkha\d na} (having a lucky sign) \par

\subparagraph*{\pali{Na}} (Mogg\,4.92)\label{pacct10:na}

\pali{a\.nga + na} = \palibf{a\.ngana} (one having good figure) \par

\subparagraph*{\pali{Ima, iya}} (Mogg\,4.94)\label{pacct10:ima}\label{pacct10:iya}

\pali{putta + ima/iya} = \palibf{puttima/puttiya} (one having a child) \par
\pali{kitti + ima/iya} = \palibf{kittima/kittiya} (one having fame) \par
\pali{sena + iya} = \palibf{seniya} (one having an army) \par

\subsection*{11.\ \pali{Pakatitaddhita}}\label{tadgroup11}

This group denotes materials that things made from. In Kacc, only \pali{maya} is mentioned. In Sadd, \pali{\=a} and \pali{\=i} are added. And in Mogg, \pali{\d na, \d nika, \d neyya,} and \pali{sa\d na} are added.

\subparagraph*{\pali{Maya}} (Kacc\,372, R\=upa\,385, Sadd\,798--9, Mogg\,4.66)\label{pacct11:maya}

\pali{suva\d n\d na + maya} = \palibf{suva\d n\d namaya}\footnote{\pali{suva\d n\d nena pakata\d m suva\d n\d namaya\d m.}} (thing made of gold) \par
\pali{rajata + maya} = \palibf{rajatamaya} (thing made of silver) \par
\pali{aya + maya} = \palibf{ayomaya} (thing made of iron) \par
\pali{mattik\=a + maya} = \palibf{mattik\=amaya} (thing made of clay) \par
\pali{go + maya} = \palibf{gomaya}\footnote{\pali{gohi nibbatta\d m gomaya\d m.}} (thing arising from an ox) \par
\pali{d\=ana + maya} = \palibf{d\=anamaya}\footnote{\pali{d\=anameva d\=anamaya\d m.} The original meaning is retained.} (giving) \par
\pali{s\=ila + maya} = \palibf{s\=ilamaya} (virtue) \par

\subparagraph*{\pali{\=A}} (Sadd\,800)\label{pacct11:aa}

\pali{s\=ura + \=a} = \palibf{sur\=a}\footnote{\pali{s\=urena n\=ama vanacarakena kat\=a p\=anaj\=ati sur\=a} (drink made by a woodsman called \pali{S\=ura}).} (liquor) \par

\subparagraph*{\pali{\=I}} (Sadd\,801)\label{pacct11:ii}

\pali{varu\d na + \=a} = \palibf{v\=aru\d n\=i}\footnote{\pali{varu\d nena n\=ama duss\=ilat\=apasena kat\=a p\=anaj\=ati v\=aru\d n\=i} (drink made by a bad ascetic called \pali{Varu\d na}).} (liquor) \par

\subparagraph*{\pali{\d Na, \d nika, \d neyya}} (Mogg\,4.66)\label{pacct11:dna}\label{pacct11:dnika}\label{pacct11:dneyya}

\pali{udumbara + \d na} = \palibf{odumbara}\footnote{\pali{Udumbarassa vikati odumbara\d m, bhasm\=a, udumbarassa avayavo odumbara\d m, pa\d n\d n\=adi}. (Niru\,536)} (things made from a fig tree, e.g.\ ashes, or a part of it, e.g.\ leaves) \par
\pali{kapota + \d na} = \palibf{k\=apota} (thing made from a pigeon, e.g.\ meat) \par
\pali{aya + \d na} = \palibf{\=ayasa} (thing made of iron) \par

\subparagraph*{\pali{Sa\d na}} (Mogg\,4.67)\label{pacct11:sadna}

\pali{jatu + sa\d na} = \palibf{j\=atusa}\footnote{\pali{jatuno vik\=aro j\=atusa\d m, jatumaya\d m v\=a}, also \pali{jatumaya}.} (thing made of sealing wax) \par

\subsection*{12.\ \pali{Sa\.nkhy\=ataddhita}}\label{tadgroup12}\label{par:sankhyataddhita}

This group is about numbers. In Kacc there are five \pali{paccaya}s: \pali{tiya, tha, \d tha, ma,} and \pali{\=i}. In Sadd \pali{tha} becomes \pali{ttha} and \pali{\d tha} becomes \pali{\d t\d tha}. In Mogg, other two are added: \pali{\d t\d thama,} and \pali{\d da}. In addition, there are other number-related \pali{paccaya}s, namely \pali{ka, aya,} and \pali{\=ak\=i}.

\subparagraph*{\pali{Tiya}} (Kacc\,385--6, R\=upa\,409--10, Sadd\,817--8)\label{pacct12:tiya}

\pali{dvi + tiya} = \palibf{dutiya}\footnote{\pali{dvinna\d n p\=ura\d no dutiyo.}} (second) \par
\pali{ti + tiya} = \palibf{tatiya} (third) \par

\subparagraph*{\pali{Tha, \d tha, ttha, \d t\d tha, \d t\d thama}} (Kacc\,384, R\=upa\,407, Sadd\,816, Mogg\,4.54)\label{pacct12:tha}\label{pacct12:dtha}\label{pacct12:ttha}\label{pacct12:dtdtha}\label{pacct12:dtdthama}

\pali{catu + tha/ttha} = \palibf{catuttha}\footnote{\pali{catunna\d n p\=ura\d no catuttho.}} (fourth) \par
\pali{cha + \d tha/\d ttha} = \palibf{cha\d t\d tha}\footnote{This can also be \pali{sa\d t\d tha} (Kacc\,374, R\=upa\,408, Sadd\,804). Yet \pali{cha\d t\d thama} can also be found (Sadd\,803, Mogg\,4.54).} (sixth) \par

\subparagraph*{\pali{Ma}} (Kacc\,373, R\=upa\,406, Sadd\,802, Mogg\,4.52--3)\label{pacct12:ma}

\pali{pa\~nca + ma} = \palibf{pa\~ncama}\footnote{\pali{pa\~ncanna\d n p\=ura\d no pa\~ncamo.}} (fifth) \par
\pali{satta + ma} = \palibf{sattama} (seventh) \par
\pali{a\d t\d tha + ma} = \palibf{a\d t\d thama} (eighth) \par
\pali{nava + ma} = \palibf{navama} (ninth) \par
\pali{dasa + ma} = \palibf{dasama} (tenth) \par
\pali{sata + ma} = \palibf{satima} (hundredth) \par
\pali{sahassa + ma} = \palibf{sahassima} (thousandth) \par

\subparagraph*{\pali{\=I}} (Kacc\,375, R\=upa\,412, Sadd\,805)\label{pacct12:ii}

\pali{ek\=adasa + \=i} = \palibf{ek\=adas\=i}\footnote{\pali{ek\=adasanna\d n p\=ura\d n\=i ek\=adas\=i.}} (eleventh) \par
\pali{dv\=adasa + \=i} = \palibf{dv\=adas\=i} (twelfth) \par
\pali{tedasa + \=i} = \palibf{tedas\=i} (thirteenth) \par
\pali{catuddasa + \=i} = \palibf{c\=atuddas\=i} (fourteenth) \par
\pali{pa\~ncadasa + \=i} = \palibf{pa\~ncadas\=i} (fifteenth) \par
\pali{so\d lasa + \=i} = \palibf{so\d las\=i} (sixteenth) \par
\pali{sattarasa + \=i} = \palibf{sattaras\=i} (seventeenth) \par
\pali{a\d t\d th\=arasa + \=i} = \palibf{a\d t\d th\=aras\=i} (eighteenth) \par

\subparagraph*{\pali{\d Da (a)}} (Mogg\,4.50--1))\label{pacct12:dda}

\pali{ek\=adasa + \d da} = \palibf{ek\=adasa/ek\=adasama}\footnote{\pali{ek\=adasanna\d n p\=ura\d no ek\=adaso, ek\=adasamo.}} (eleventh) \par
\pali{v\=isa + \d da} = \palibf{v\=isa/v\=isatima} (twenty/twentieth) \par
\pali{ti\d msa + \d da} = \palibf{ti\d msa/ti\d msatima} (thirty/thirtieth) \par
\pali{catt\=al\=isa + \d da} = \palibf{catt\=al\=isa} (forty) \par
\pali{pa\~n\~n\=asa + \d da} = \palibf{pa\~n\~n\=asa} (fifty) \par
\pali{v\=isa sata + \d da} = \palibf{v\=isa sata}\footnote{\pali{v\=isati adhik\=a asmi\d m steti v\=isa\d m sata\d m}. (from Mogg\,4.50)} (120) \par
\pali{v\=isa sahassa + \d da} = \palibf{v\=isa sahassa} (1,020) \par
\pali{v\=isa satasahassa + \d da} = \palibf{v\=isa satasahassa} (100,020) \par
\pali{ek\=adasa sata + \d da} = \palibf{ek\=adasa sata} (111) \par
\pali{ek\=adasa sahassa + \d da} = \palibf{ek\=adasa sahassa} (1,011) \par

\subparagraph*{\pali{Ka}} (Kacc\,392, R\=upa\,418, Sadd\,831, Mogg\,4.41)\label{pacct12:ka}

\pali{dvi + ka} = \palibf{dvika} (twofold) \par
\pali{ti + ka} = \palibf{tika} (threefold) \par
\pali{catu + ka} = \palibf{catukka} (fourfold) \par
\pali{pa\~nca + ka} = \palibf{pa\~ncaka} (fivefold) \par
\pali{cha + ka} = \palibf{chakka} (sixfold) \par
\pali{satta + ka} = \palibf{sattaka} (sevenfold) \par
\pali{a\d t\d tha + ka} = \palibf{a\d t\d thaka} (eightfold) \par
\pali{nava + ka} = \palibf{navaka} (ninefold) \par
\pali{dasa + ka} = \palibf{dasaka} (tenfold) \par

\subparagraph*{\pali{Aya}} (Mogg\,4.49)\label{pacct12:aya}

\pali{ubha + aya} = \palibf{ubhaya} (twofold) \par
\pali{dvi + aya} = \palibf{dvaya} (twofold) \par
\pali{ti + aya} = \palibf{taya} (threefold) \par

\subparagraph*{\pali{\=Ak\=i}} (Mogg\,4.55)\label{pacct12:aakii}

\pali{eka + \=ak\=a} = \palibf{ek\=ak\=i}\footnote{also \pali{ekaka}, or just \pali{eka}} (alone) \par

\subsection*{13.\ \pali{Abyayataddhita}}\label{tadgroup13}

This group produces indeclinable outcomes by adding these \pali{paccaya}s to existing nouns. Kacc gives us four: \pali{dh\=a, th\=a, thatth\=a,} and \pali{tha\d m}. Sadd adds \pali{jjha} and \pali{so}. Mogg adds \pali{edh\=a} and \pali{kkhattu\d m}.

\subparagraph*{\pali{Dh\=a, edh\=a}} (Kacc\,397, R\=upa\,420, Sadd\,836, Mogg\,4.110, 4.112)\label{pacct13:dhaa}\label{pacct13:edhaa}

\pali{eka + dh\=a} = \palibf{ekadh\=a}\footnote{\pali{ekena vibh\=agena ekadh\=a.}} (in one way) \par
\pali{dvi + dh\=a} = \palibf{dvidh\=a/dvedh\=a} (in two ways) \par
\pali{ti + dh\=a} = \palibf{tidh\=a/tedh\=a} (in three ways) \par
\pali{catu + dh\=a} = \palibf{catudh\=a} (in four ways) \par
\pali{kati + dh\=a} = \palibf{katidh\=a} (in how many ways) \par
\pali{bahu + dh\=a} = \palibf{bahudh\=a} (in many ways) \par

\subparagraph*{\pali{Jjha}} (Sadd\,837, Mogg\,4.111)\label{pacct13:jjha}

\pali{eka + jjha} = \palibf{ekajjha}\footnote{\pali{ekadh\=a karoti ekajjha\d m.}} (in one way) \par
\pali{dvi + jjha} = \palibf{dvijjha} (in two ways) \par

\subparagraph*{\pali{Th\=a, thatth\=a}} (Kacc\,398, R\=upa\,421, Sadd\,844, Mogg\,4.108)\label{pacct13:thaa}\label{pacct13:thatthaa}

\pali{ta + th\=a} = \palibf{tath\=a}\footnote{\pali{so pak\=aro tath\=a, ta\d m pak\=ara\d m tath\=a, tena pak\=arena tath\=a, tassa pak\=arassa tath\=a, tasm\=a pak\=ar\=a tath\=a, tasmi\d m pak\=are tath\=a.}} (in that way) \par
\pali{ya + th\=a} = \palibf{yath\=a} (in which way) \par
\pali{sabba + th\=a} = \palibf{sabbath\=a} (in all ways) \par
\pali{a\~n\~na + th\=a} = \palibf{a\~n\~nath\=a} (in other way) \par
\pali{itara + th\=a} = \palibf{itarath\=a} (in another way) \par
In Kacc and Sadd, \pali{thatth\=a} can be used in the same way, hence we also get \pali{tathatth\=a, yathatth\=a, sabbathatth\=a, a\~n\~nathatth\=a,} and \pali{itarathatth\=a}. Sadd adds that \pali{tathattha\d m} and \pali{a\~n\~nathattha\d m} can also be found.

\subparagraph*{\pali{Tha\d m}} (Kacc\,399, R\=upa\,422, Sadd\,845, Mogg\,4.109)\label{pacct13:thadm}

\pali{ki\d m + tha\d m} = \palibf{katha\d m} (in what way, how) \par
\pali{ima + tha\d m} = \palibf{ittha\d m} (in this way, thus) \par

\subparagraph*{\pali{So}} (Sadd\,838, Mogg\,4.118)\label{pacct13:so}

This \pali{paccaya} has intrumental sense.
	
\pali{sabba + so} = \palibf{sabbaso}\footnote{\pali{sabb\=ak\=arena sabbaso.}} (by all ways, in every respect) \par
\pali{bahu + so} = \palibf{bahuso}\footnote{\pali{bah\=uhi pak\=arehi bahuso.}} (by many ways) \par
\pali{sutta + so} = \palibf{suttaso}\footnote{\pali{suttavibh\=agena suttaso.}} (by sutra's part) \par
\pali{up\=aya + so} = \palibf{up\=ayaso}\footnote{\pali{up\=ayena up\=ayaso.}} (by stratagem) \par
\pali{hetu + so} = \palibf{hetuso}\footnote{\pali{hetun\=a hetuso.}} (by cause) \par
\pali{\d th\=ana + so} = \palibf{\d th\=anaso}\footnote{\pali{ta\.nkha\d neneva\d th\=anaso.}} (by that moment?, by reason) \par
\pali{\~n\=aya + so} = \palibf{yoniso}\footnote{\pali{\~n\=ayena yoniso.}} (by right manner) \par

\subparagraph*{\pali{Kkhattu\d m}} (Mogg\,4.114--7)\label{pacct13:kkhattudm}

This \pali{paccaya} marks number of times. In Kacc\,646, R\=upa\,419, \pali{kkhattu\d m} can be applied to \pali{saki\d m} (once) and \pali{eka} (one), etc. But applying to \pali{saki\d m} is disagreed in Sadd\,1284 because it sounds nonsensical. Applying it to \pali{eka} and so on is acceptable (Sadd\,1282). Sometimes it means division (Sadd\,1283), for example, \pali{ekakkhattu\d m} (one part), \pali{dvikkhattu\d m} (two parts), and \pali{Sahassakkhattumatt\=ana\d m, nimminitv\=ana panthako}\footnote{Thag\,10.563} \\(Ven.\,Panthaka produced himself into 1,000 parts/replicas).

\pali{eka + kkhattu\d m} = \palibf{ekakkhattu\d m}\footnote{also \pali{saki\d m}} (one time) \par
\pali{dvi + kkhattu\d m} = \palibf{dvikkhattu\d m} (two times) \par
\pali{kati + kkhattu\d m} = \palibf{katikkhattu\d m} (how many times) \par
\pali{bahu + kkhattu\d m} = \palibf{bahukkhattu\d m}\footnote{also \pali{bahudh\=a}} (many times) \par

\subsection*{14.\ \pali{Anekatthataddhita}}\label{tadgroup14}

The group combines the remaining miscellaneous things. Some look like post hoc explanation of terms in an idiosyncratic way. So I have to omit some of them.

\subparagraph*{\pali{\d Naya}} (Sadd\,783, Mogg\,4.72--3)\label{pacct14:dnaya}

\pali{kamma + \d naya} = \palibf{kamma\~n\~na}\footnote{\pali{kammani s\=adhu kamma\~n\~na\d m.}} (good in doing, worth doing) \par
\pali{sabh\=a + \d naya} = \palibf{sabbha}\footnote{\pali{sabh\=aya\d m s\=adhu sabbho.}} (good in meeting, worth meeting) \par

\subparagraph*{\pali{Ika}} (Mogg\,4.74, Sadd\,1278)\label{pacct14:ika}

\pali{katha + ika} = \palibf{kathika} (good at talking) \par
\pali{dhammakatha + ika} = \palibf{dhammakathika} (good at talking dhamma) \par
\pali{sa\.ng\=ama + ika} = \palibf{sa\.ng\=amika} (good at fighting) \par
\pali{aha\d m + aha\d m + ika} = \palibf{ahamahamik\=a}\footnote{In this instance \pali{ika} means `I first' (Sadd\,1278). The term is a repetition of `I.' Thus `I first, I first' means like egoistic assertion.} (conceit) \par

\subparagraph*{\pali{\d Nika}} (Sadd\,1279)\label{pacct14:dnika}

\pali{aho + purisa + \d nika} = \palibf{ahopurisik\=a} (arrogance) \par

\subparagraph*{\pali{Lika}} (Sadd\,1280)\label{pacct14:lika}

\pali{putta + lika} = \palibf{puttalik\=a} (doll of a boy's figure) \par
\pali{dh\=itu + lika} = \palibf{dh\=italik\=a} (doll of a girl's figure) \par

\subparagraph*{\pali{\d Neyya}} (Mogg\,4.75)\label{pacct14:dneyya}

\pali{patha + \d neyya} = \palibf{p\=atheyya} (good for travelling, provision) \par
\pali{sakata? + \d neyya} = \palibf{s\=apateyya} (good for the owner, property, wealth) \par

\subparagraph*{\pali{Ka}} (Sadd\,835, Mogg\,4.40)\label{pacct14:ka}

\pali{sama\d na + ka} = \palibf{sama\d naka} (bad ascetic) \par
\pali{itth\=i + ka} = \palibf{itthik\=a} (bad woman) \par
\pali{mu\d n\d da + ka} = \palibf{mu\d n\d daka} (a bald one) \par
\pali{kum\=ara + ka} = \palibf{kum\=araka} (little kid) \par
\pali{putta + ka} = \palibf{puttaka} (little child) \par
\pali{g\=ama + ka} = \palibf{g\=amaka} (small village) \par
\pali{tela + ka} = \palibf{telaka} (little oil) \par
\pali{vin\=ila + ka} = \palibf{vin\=ilaka} (bluish disgusting [corpse]) \par
\pali{h\=ina + ka} = \palibf{h\=inaka}\footnote{The original meaning is retained.} (bad) \par
\pali{pota + ka} = \palibf{potaka} (young) \par
\pali{assa + ka} = \palibf{assaka} (horse of unknown owner) \par
\pali{hatth\=i + ka} = \palibf{hatthika} (elephant-like [toy]) \par
\pali{rukkha + ka} = \palibf{rukkhaka} (shrub) \par
\pali{manussa + ka} = \palibf{m\=anussaka}\footnote{\pali{rasse-m\=anusako}. (in Mogg\,4.40)} (dwarf, human) \par
\pali{mora + ka} = \palibf{moraka} (a man called `peacock') \par

\subparagraph*{\pali{Tara}} (Mogg\,4.56)\label{pacct14:tara}

\pali{vaccha + tara} = \palibf{vacchatara} (small calf) \par

\subparagraph*{\pali{Reyya\d n, cha}} (Mogg\,4.36--7, 4.39)\label{pacct14:reyyadn}\label{pacct14:cha}

\pali{pitu + reyya\d n} = \palibf{petteyya}\footnote{\pali{pitu bh\=at\=a petteyyo.}} (brother of father) \par
\pali{m\=atu + cha} = \palibf{m\=atucch\=a}\footnote{\pali{m\=atu bhagin\=i m\=atucch\=a.}} (sister of mother) \par
\pali{m\=atu + reyya\d n} = \palibf{matteyya} (motherloving) \par
\pali{pitu + reyya\d n} = \palibf{petteyya} (fatherloving) \par

\subparagraph*{\pali{\=Amaha}} (Mogg\,4.38)\label{pacct14:aamaha}

\pali{m\=atu + \=amaha} = \palibf{m\=at\=amah\=i}\footnote{\pali{m\=atu m\=at\=a m\=at\=amah\=i.}} (mother of mother) \par
\pali{m\=atu + \=amaha} = \palibf{m\=at\=amaha}\footnote{\pali{m\=atu pit\=a m\=at\=amaho.}} (father of mother) \par
\pali{pitu + \=amaha} = \palibf{pit\=amah\=i}\footnote{\pali{pitu m\=at\=a pit\=amah\=i.}} (mother of father) \par
\pali{pitu + \=amaha} = \palibf{pit\=amaha}\footnote{\pali{pitu pit\=a pit\=amaha.}} (father of father) \par

\subparagraph*{\pali{Ssa}} (Mogg\,4.71)\label{pacct14:ssa}

\pali{cakkhu + ssa} = \palibf{cakkhussa} (good for eyes) \par
\pali{\=ayu + ssa} = \palibf{\=ayussa} (good for the age) \par

\subparagraph*{\pali{La, iya}} (Mogg\,4.58)\label{pacct14:la}\label{pacct14:iya}

\pali{deva + la/iya} = \palibf{devala/deviya}\footnote{\pali{devena datto devalo, deviyo.}} ([thing/person] given by a god) \par

\subparagraph*{\pali{J\=atiya}} (Mogg\,4.113)\label{pacct14:jaatiya}

\pali{pa\d tu + j\=atiya} = \palibf{pa\d tuj\=atiya} (having sharp property) \par
\pali{mudu + j\=atiya} = \palibf{muduj\=atiya} (having soft property) \par

