\chapter{I read a book \headhl{slowly}}\label{chap:adv}

\phantomsection
\addcontentsline{toc}{section}{Introduction to Adverb}
\section*{Introduction to Adverb}

It might be late to introduce adverb by now. One reason is that P\=ali has no such a word category. In English, what we call adverb is a word or phrase that does adverbial function: modifying adjectives, verbs, other adverbs, and sentences.\footnote{\citealp[p.~13]{brownmiller:dict}} By its form, an adverbial can be an adverb phrase, prepositional phrase, or noun phrase.\footnote{\citealp[\S206]{eastwood:guide}}

Let us see the latter two forms first. When we say ``I will go \textbf{tomorrow},'' the adverbial is a noun phrase. In P\=ali, the sentence is ``\pali{aha\d m \textbf{suve} gamiss\=ami},'' where the adverbial is a particle with locative meaning. And when we say ``I will go \textbf{in the morning},'' the adverbial now is a prepositional phrase. A P\=ali equivalent of this is ``\pali{aha\d m \textbf{pubba\d nhe} gamiss\=ami},'' where the adverbial is a noun in locative case.

You may realize now that why there is no adverb in P\=ali. First, a large number of words that do adverbial function come in form of particles (see Appendix \ref{chap:nipata} for more detail). And second, we can use nouns in various cases to express the idea. What English teachers call `adverb of time' and `adverb of place' are basically nouns in locative case. Let us see these examples:\par
- \pali{d\=arako \textbf{sayane} sayati.} (A boy sleeps on a bed)\par
- \pali{macch\=a \textbf{samudde} honti.} (There are fish in the sea)\par
- \pali{\textbf{raviv\=are} pakkamiss\=ami.} (I will leave on Sunday)\par

As you can think further, other viable cases can do adverbial job as well.\footnote{See also \citealp[p.~124]{collins:grammar}.} Consider these examples:\par
- \pali{\textbf{yojana\d m} d\=igho pabbato}\footnote{Kacc\,298} (a mountain one-yojana high)\par
- \pali{\textbf{pakatiy\=a} abhir\=upo}\footnote{R\=upa\,300} (a naturally beautiful [person])\par
- \pali{\textbf{j\=atiy\=a} so\d lasavasso} ([He is] sixteen by birth.)\par
- \pali{\textbf{tena samayena} buddho bhagav\=a}\footnote{This stock phrase is mostly found in the Vinaya.} (By that time, the Buddha \ldots)\par
- \pali{d\=arak\=a \textbf{sikkh\=aya} p\=a\d thas\=ala\d m gacchanti.} (Children go to school for studying.)\par
- \pali{\textbf{gehasm\=a} p\=a\d thas\=ala\d m gacch\=ami.} (From home, I go to school.)\par
- \pali{So ta\d m pavissa na \textbf{cirassa} n\=ago, dibbena me p\=aturahu\d m janinda}\footnote{Ja\,17:156} (Your Majesty, not long, that serpent entered to that [place]. [Then it] appeared before me [along] with divine [followers].)\par
- \pali{Tena kho pana samayena j\=a\d nusso\d ni br\=ahma\d no sabbasetena va\d lav\=abhirathena s\=avatthiy\=a niyy\=ati \textbf{div\=adivassa}.}\footnote{M1\,288 (MN\,27)} (In that time, Brahman J\=a\d nusso\d ni goes out of S\=avatt\=i with all-white mare-carriage in the noon.)\par

As you have seen, it seems that talking about adverb in P\=ali is a matter of redundancy. However, the real protagonist of this story is terms in accusative case. Much like in English that an adverb can be create by adding `-ly' to an adjective, in P\=ali we can make an adverb simply by putting it into (singular) accusative case.\footnote{Scholars call this \emph{adverbial accusative}, e.g.\ \citealp[p.~116]{warder:intro}.} For demonstration, let us do the heading task right now.

In ``I read a book slowly,'' we have `slow' as the adjective that has to be converted into adverb. In P\=ali there are \pali{dandha} and \pali{manda} given in a dictionary. Those are not quite suitable here, because they have a negative meaning of `stupid.' It is better to use the opposite of `fast,' hence \pali{as\=igha} in this context.

Now we have the word. Changing this to accusative is easy, because this case is one of the most user-friendly. Then we get \pali{as\=igha\d m} as adverb. Now we compose the sentence as follows:

\palisample{aha\d m as\=igha\d m potthaka\d m pa\d th\=ami.}

One possible problem here is when the adverb we use looks like a modifier of other noun. In this case, \pali{as\=igha\d m} can be a modifier of \pali{potthaka\d m}, hence `a slow book' which, fortunately, sounds out of place. But if \pali{dandha} is used instead, it may allow `a stupid book' to be read. Repositioning the word can be a little help. For example, in this case you can separate the two accusatives like this:

\palisample{as\=igha\d m aha\d m potthaka\d m pa\d th\=ami. \sampleor aha\d m potthaka\d m pa\d th\=ami as\=igha\d m.}

However, this still does not guarantee that the unintended meaning will not be rendered.

The adverb used in the previous example is a kind of adverb of manner. Adverbial accusatives, however, can have locative meaning as well, for example:\par
- \pali{Eva\d m me suta\d m \textbf{eka\d m samaya\d m} bhagav\=a \ldots}\footnote{This is a stock phrase found mostly in the Suttanta.} (It is heard by me thus, in one occasion the Buddha \ldots)\par
- \pali{Atha kho bhagav\=a \textbf{pubba\d nhasamaya\d m} niv\=asetv\=a pattac\=ivaram\=ad\=aya r\=ajagaha\d m pi\d n\d d\=aya p\=avisi.}\footnote{Mv\,5.247} (In one morning the Buddha, having dressed himself, having taken bowl and robe, entered R\=ajagaha for alms.)\par

\bigskip
With acc.\ we can also express a duration of time like these examples:\par
- \pali{na, bhikkhave, vassa\d m upagantv\=a purima\d m v\=a \textbf{tem\=asa\d m} pacchima\d m v\=a \textbf{tem\=asa\d m} avasitv\=a c\=arik\=a pakkamitabb\=a}\footnote{Mv\,3.185. See Chapter \ref{chap:pp} to learn about how verbs in \pali{tv\=a} work.} \\(Monks, having undergone the rainy season, [before] the first three months or the last three months [ends], one should not go out for wandering.)\par
- \pali{imasmi\d m vih\=are ima\d m \textbf{tem\=asa\d m} vassa\d m upemi}\footnote{Mv-a\,3.184} (I [will] undergo this three months in rainy season in this temple.)\par

\phantomsection
\addcontentsline{toc}{section}{Repetition}
\section*{Repetition}\label{sec:repetition}

Apart from using a word or a phrase to do adverbial function, repetition of terms can have an adverbial effect. Technically, this is called \pali{vicch\=a}.\footnote{Mogg\,1.54, Niru\,55, see also Sadd-Pad Ch.\,3, from ``\pali{Vicch\=avasena atthavisesal\=abhe}'' onwards.} There are three possible meanings when a word is doubled: (1) individually or every/each, (2) sequentially or gradually, and (3) repetitively or again and again. Here are some examples from Moggall\=ana and Niruttid\=ipan\=i:\par
- \pali{rukkha\d m rukkha\d m si\~ncati} ([One] waters each tree.)\par
- \pali{g\=amo g\=amo rama\d n\=iyo} (Every village is delightful.)\par
- \pali{g\=ame g\=ame sata\d mkumbh\=a} (In each village, [there are] 100 pots.)\par
- \pali{gehe gehe issaro} (the leader in every house)\par
- \pali{rasa\d m rasa\d m bhakkhayati} ([One] eats every taste.)\par
- \pali{kiriya\d m kiriya\d m \=arabhate} (Every action is started.)\par
- \pali{m\=ule m\=ule th\=ul\=a} ([It is] fat gradually in the base.)\par
- \pali{agge agge sukhum\=a} ([It is] subtle gradually on the top.)\par
- \pali{je\d t\d tha\d m je\d t\d tha\d m anupavesetha} ([Please] enter respectively by seniority.)\par
- \pali{imesa\d m devasika\d m m\=asaka\d m m\=asaka\d m dehi} (Do give to these [people] everyday each month.)\par
- \pali{ime jan\=a patha\d m patha\d m accenti} (These people go in each way sequentially.)\par
- \pali{bhatta\d m pacati pacati} ([One] cooks food repeatedly.)\par
- \pali{apu\~n\~na\d m pasavati pasavati} ([One] brings forth demerit again and again.)\par
- \pali{bhutv\=a bhutv\=a nippajjanti} ([They], having eaten, [then] sleep again and again.)\par 
- \pali{pa\d ta\d m pa\d ta\d m karoti} ([One] makes `\pali{pa\d ta}' sound repeatedly.)\par

\bigskip
This one is from the canon:\par
\begin{quote}
\pali{So kho tva\d m, ambho purisa, divase divase t\=ihi t\=ihi sattisatehi ha\~n\~nam\=ano \ldots}\footnote{S5\,1105 (SN\,56)}\\
``Man, you who is being stabbed with 300 spears everyday \ldots''
\end{quote}

And here is an interesting instance where singular \pali{atta} is used in plural meaning to stress the distribution:

\begin{quote}
\pali{katha\~nhi n\=ama bhadant\=a attano attano c\=ivara\d m na sa\~nj\=anissanti}\footnote{Buv2\,367}\\
``Why on earth venerables will not remember their own robe [individually]?''
\end{quote}

\bigskip
Additionally, repetition can have emphatic effect or make the meaning indefinite. For example, \pali{yo yo} (whoever), \pali{yath\=a yath\=a} (in whatever way).\footnote{\citealp[p.~72, 171]{warder:intro}} Sometimes repetition simply means `very.' Here are some examples:

\begin{quote}
\pali{Seyyath\=api n\=ama pakkh\=i saku\d no yena yeneva \d deti, sapattabh\=arova \d deti; evameva bhikkhu santu\d t\d tho hoti}\footnote{M2\,11 (MN\,51)}\\
``Just like wherever a bird flies, it goes only with wings. In the same way, a monk is pleased [only with a robe and alms].''\\[1.5mm]
\pali{sace kho aha\d m yo yo paresa\d m adinna\d m theyyasa\.nkh\=ata\d m \=adiyissati, tassa tassa dhanamanuppadass\=ami, eva\-mida\d m adinn\=ad\=ana\d m pava\d d\d dhissati.}\footnote{D3\,92 (DN\,26)}\\
``If I give out properties to that one whoever will take others' [thing] ungiven like a thief, this taking of ungiven thing will flourish.''\\[1.5mm]
\pali{yath\=a yath\=a v\=a panassa k\=ayo pa\d nihito hoti, tath\=a tath\=a na\d m paj\=an\=ati}\footnote{D2\,375 (DN\,22)}\\
``In whatever way the body of that [monk] is positioned, in that way [he] knows that [position].''\\[1.5mm]
\pali{Seyyath\=api, bhikkhave, daharo kum\=aro mando utt\=anaseyyako dh\=atiy\=a pam\=adamanv\=aya ka\d t\d tha\d m v\=a ka\d thala\d m v\=a mukhe \=ahareyya. Tamena\d m dh\=ati s\=igha\d m s\=igha\d m manasi kareyya;}\footnote{A5\,7}\\
``Just like this, monks, suppose a young child, an infant, puts a piece of wood or a potsherd into his mouth by carelessness of the nursemaid. The nurse should pay attention to that [child] very fast (immediately).''\\[1.5mm]
\end{quote}

Also in Sadd-Pad Ch.\,3, Aggava\d msa summarizes the use of repetition as exclamation (\pali{\=ame\d n\d dita}).\footnote{Sadd-Pad Ch.\,3, from ``\pali{Bhayakodh\=ad\=isu uppannesu kathit\=ame\d ditavacanavasena pana atthavisesal\=abhe ime payog\=a}'' onwards.} Here are examples given:

\begin{quote}
[in fear]\\
\pali{coro coro} (Thief, thief!)\\
\pali{sappo sappo} (Snake, snake!)\\[1.5mm]
[in anger]\\
\pali{vasala vasala} (Outcaste!)\\
\pali{ca\d n\d d\=ala ca\d n\d d\=ala} (Outcaste!)\\
\pali{vijjha vijjha} (Stab [it]!)\\
\pali{pahara pahara} (Beat [it]!)\\[1.5mm]
[in praise]\\
\pali{s\=adhu s\=adhu s\=ariputta}\footnote{M1\,339 (MN\,32)} (Good, good!, S\=ariputta.)\\
\pali{abhikkanta\d m bhante, abhikkanta\d m bhante}\footnote{D1\,441 (DN\,9)} (Fantastic!, Venerable.)\\[1.5mm]
[in haste]\\
\pali{Abhikkama gahapati, abhikkama gahapati}\footnote{S1\,242 (SN\,10)} (Step forward!, householder.)\\
\pali{gaccha gaccha} (Go, go!)\\
\pali{lun\=ahi lun\=ahi} (Reap [it]!)\\[1.5mm]
[in excitement]\\
\pali{\=agaccha \=agaccha} (Come, come!)\\[1.5mm]
[in amazement]\\
\pali{aho buddho aho buddho} (Oh Buddha!)\\[1.5mm]
[in amusement]\\
\pali{aho sukha\d m aho sukha\d m} (Oh happiness!)\\
\pali{aho man\=apa\d m aho man\=apa\d m} (Oh lovely!)\\[1.5mm]
[in grief]\\
\pali{kaha\d m, ekaputtaka, kaha\d m, ekaputtaka}\footnote{M2\,353 (MN\,87)} (Where are you, [my] only son?)\\[1.5mm]
[in faithfulness]\\
\pali{bhavissanti vajj\=i, bhavissanti vajj\=i}\footnote{A5\,58} (Vajj\=i [lords] will flourish, Vajj\=i [lords] will flourish!)\\[1.5mm]
\end{quote}

\section*{Exercise \ref{chap:adv}}
Say these in P\=ali.
\begin{compactenum}
\item If everthing has its previous cause, do we really have free will?
\item It depends on what you mean by `free.'
\item I mean we can do things freely.
\item From the doer's own perspective, I think we have free will because we feel it that way individually.
\item That is what most people see the problem, I guess.
\item From the nature's perspective, on the other hand, everything depends on other things else. Free will is indeed an illusion. From Benjamin Libet's finding, our brain even knows faster than our conscious will.
\item That means I can do evil because it is not my decision really.
\item That is completely a different problem. You have to use your own illusive free will to do good things anyway.
\end{compactenum}
