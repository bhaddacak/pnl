\chapter{\headhl{There is} a book}\label{chap:verb-be}

Now I will introduce an important part of a sentence which we have skipped---\emph{verb}. Verb in P\=ali is really a big topic. It is complicated and difficult if you study it as a linguist or grammarian. If you just want to learn how to use it, you have to overcome only some fundamentals. Still, I have to admit, it is a lot to do. However, not to intimidate you at the first go, I will present you here the commonest verb of all---\emph{to be}.

\phantomsection
\addcontentsline{toc}{section}{Verb to Be}
\section*{Verb to Be}

To say that something exists or is present or has certain quality or has a connection with other thing\footnote{Linguists call this \emph{copula}---``A verb that has no content but simply links two words or phrases'' \citep[p.~112]{brownmiller:dict}.}, P\=ali normally uses three verbs: \pali{hoti}, \pali{bhavati}, and \pali{atthi}. These verbs express the state of being of the subject, like verb `\emph{to exist}' or the phrase `\emph{there is/are}'; or just link to its quality, like verbs `\emph{to be, become}.' These three P\=ali verbs are the most frequently used verbs in the scriptures. In most contexts they can be used somewhat interchangeably.

Like nouns, verbs have to be changed according to its intended function before used. Inflectional transformation of verbs is called \emph{conjugation}. There are four things to be concerned: \emph{tense/mood, person, number}, and \emph{voice}. Basically, P\=ali has three tenses\footnote{Traditionally speaking, there are three past tenses, hence totally we have eight tenses/moods. But only one kind of past is widely used. The other two are seldom found in the texts as remnants of antiquity.} and three moods, i.e.\ present, past, and future tense; and imperative, optative, and conditional mood. There are three persons of subject corresponding to personal pronouns, e.g.\ 1st, 2nd, and 3rd person. Number is how many agents in the subject. It can be singular or plural. Voice in P\=ali is a little confusing. It can be \emph{active}, and \emph{middle} voice. At the present we focus only on active voice.

To make things easier, when we talk about verbs we use their dictionary form---\emph{present, 3rd-person, singular, active-voice}. This means verbs in a dictionary are ready to use only in such a case. In other situations, you have to learn verb conjugation. Table \ref{tab:conj-exist} shows present tense conjugations of the three verbs mentioned above. Verb \pali{atthi} has irregular forms, so please pay more attention on these.

\begin{table}[!hbt]
\centering
\caption{Present tense conjugations of verbs `to be'}
\label{tab:conj-exist}
\bigskip
\begin{tabular}{@{}>{\itshape}lc*{2}{>{\itshape}l}@{}} \toprule
\bfseries\upshape Verb & \bfseries Person & \bfseries\upshape Singular & \bfseries\upshape Plural \\ \midrule
\multirow{3}{*}{\bfseries hoti} & 3rd & hoti & honti \\
\cmidrule(l){2-4} & 2nd & hosi & hotha \\
\cmidrule(l){2-4} & 1st & homi & homa \\
\midrule
\multirow{3}{*}{\bfseries bhavati} & 3rd & bhavati & bhavanti \\
\cmidrule(l){2-4} & 2nd & bhavasi & bhavatha \\
\cmidrule(l){2-4} & 1st & bhav\=ami & bhav\=ama \\
\midrule
\multirow{4}{*}{\bfseries atthi} & 3rd & \texthl{atthi} & \texthl{santi} \\
\cmidrule(l){2-4} & 2nd & \texthl{asi} & \texthl{attha} \\
\cmidrule(l){2-4} & \multirow{2}{*}{1st} & \texthl{amhi} & \texthl{amha} \\
& & \texthl{asmi} & \texthl{asma} \\
\bottomrule
\end{tabular}
\end{table}

Therefore ``There is a book'' in P\=ali can be rendered as:

\palisample{potthako/potthaka\d m hoti.\sampleor potthako/potthaka\d m bhavati.\sampleor potthako/potthaka\d m atthi.}

Here is for ``There are books.''

\palisample{potthak\=a(ni) honti/bhavanti/santi.}

Note that verbs do not care about gender of the subject. Here is for ``There is a beautiful girl.'' And now I will use only \pali{hoti}.

\palisample{sur\=up\=a ka\~n\~n\=a hoti.\sampleor ka\~n\~n\=a hoti sur\=up\=a.\sampleor[or even] hoti ka\~n\~n\=a sur\=up\=a.}

With slightly different meaning, here is for ``A girl is beautiful.''

\palisample{ka\~n\~n\=a sur\=up\=a hoti.}

To be specific, we have to use pronominal adjective \pali{ta} because P\=ali has no article. So, this is for ``The/That girl is beautiful.''

\palisample{s\=a ka\~n\~n\=a sur\=up\=a hoti.}

And this is for its plural version.

\palisample{t\=a ka\~n\~n\=a(yo) sur\=up\=a(yo) honti.}

Now you can say ``I am a fat guy.''

\palisample{aha\d m th\=ulo puriso homi.}

And ``We are fat guys.''

\palisample{maya\d m th\=ul\=a puris\=a homa.}

``You are a wise young woman.''

\palisample{tva\d m pa\~n\~navat\=i taru\d n\=a itth\=i hosi.\sampleor[or more stylistic] tva\d m itth\=i hosi pa\~n\~navat\=i taru\d n\=a.}

``You are wise women.''

\palisample{tumhe pa\~n\~navat\=i itth\=i/itthiyo hotha.}

As we have seen in the preceding chapters, verb `to be' in P\=ali can be omitted if everything is clear. However, I recommend you to put the verb in the sentences you compose until you get used to it. Do not leave without beating our exercise.

\section*{Exercise \ref{chap:verb-be}}
Say these in P\=ali.
\begin{compactenum}
\item Mozart is a great musician.
\item We are powerful wealthy merchants.
\item You are old, feeble, poor beggars.
\item I am a buffalo. I am black, big, fierce.
\item You are a small insect. You are ugly, humble, worthless.
\end{compactenum}
