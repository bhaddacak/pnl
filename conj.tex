\chapter{Verbal Conjugations}\label{chap:conj}

\section{Verbal \pali{Vibhatti}}

All conjugations (verbal \pali{vibhatti}) enumerated by three grammatical schools are listed here. The items marked with asterisk (*) are given by Mogg differently. Please note that the name of person presented here are reversed to those in the traditional textbooks, i.e.\ 1st person = \pali{uttama}, 2nd person = \pali{majjhima}, and 3rd person = \pali{pa\d thama}. However, I maintain the traditional order so that it will be less confusing when you check with the textbooks.

\bigskip
\begin{conjtable}{Conjugation of \pali{Vattam\=an\=a} (present tense)\footnote{Kacc\,423; R\=upa\,426; Sadd\,896; Mogg\,6.1; Niru\,562}}
3rd & ti & nti & te & nte \\
2nd & si & tha & se & vhe \\
1st & mi & ma & e & mhe \\
\end{conjtable}

\begin{conjtable}{Conjugation of \pali{Pa\~ncam\=i} (imperative mood)\footnote{Kacc\,424; R\=upa\,450; Sadd\,897; Mogg\,6.10; Niru\,575}}
3rd & tu & ntu & ta\d m & nta\d m \\
2nd & hi & tha & ssu & vho \\
1st & mi & ma & e & \=amase \\
\end{conjtable}

\begin{conjtable}{Conjugation of \pali{Sattam\=i} (optative mood)\footnote{Kacc\,425; R\=upa\,453; Sadd\,898; Mogg\,6.8; Niru\,577}}
3rd & eyya & eyyu\d m & etha & era\d m \\
2nd & eyy\=asi & eyy\=atha & etho & eyy\=avho\footnote{In Mogg\,6.8 it is \pali{eyyavho}, but \pali{eyy\=avho} in Payo\,533 and Niru\,577.} \\
1st & eyy\=ami & eyy\=ama & eyya\d m & eyy\=amhe \\
\end{conjtable}

\newpage
\begin{conjtable}{Conjugation of \pali{Parokkh\=a} (perfect tense)\footnote{Kacc\,426; R\=upa\,459; Sadd\,899; Mogg\,6.6; Niru\,596}}
3rd & a & u & ttha & re \\
2nd & e & ttha & ttho & vho \\
1st & a\d m, a* & mha & i\d m, i* & mhe \\
\end{conjtable}

\begin{conjtable}{Conjugation of \pali{Hiyyattan\=i} (imperfect tense)\footnote{Kacc\,427; R\=upa\,455; Sadd\,900; Mogg\,6.5; Niru\,584}}
3rd & \=a & \=u & ttha & tthu\d m \\
2nd & o & ttha & se & vha\d m \\
1st & a\d m, a* & mh\=a & i\d m & mhase \\
\end{conjtable}

\begin{conjtable}{Conjugation of \pali{Ajjattan\=i} (aorist tense)\footnote{Kacc\,428; R\=upa\,468; Sadd\,901; Mogg\,6.4; Niru\,587}}
3rd & \=i & u\d m & \=a & \=u \\
2nd & o & ttha & se & vha\d m \\
1st & i\d m & mh\=a & a\d m, a* & mhe \\
\end{conjtable}

\begin{conjtable}{Conjugation of \pali{Bhavissanti} (future tense)\footnote{Kacc\,429; R\=upa\,472; Sadd\,902; Mogg\,6.2; Niru\,601}}
3rd & ssati & ssanti & ssate & ssante \\
2nd & ssasi & ssatha & ssase & ssavhe \\
1st & ss\=ami & ss\=ama & ssa\d m & ss\=amhe \\
\end{conjtable}

\begin{conjtable}{Conjugation of \pali{K\=alatipatti} (conditional mood)\footnote{Kacc\,430; R\=upa\,474; Sadd\,903; Mogg\,6.7; Niru\,604}}
3rd & ss\=a & ssa\d msu & ssatha & ssi\d msu \\
2nd & sse & ssatha & ssase & ssavhe \\
1st & ssa\d m & ss\=amh\=a & \mbox{ssi\d m, ssa\d m} & ss\=amhase \\
\end{conjtable}

\clearpage
\section{Operation of \pali{Vibhatti}}

Like \pali{paccaya}, \pali{vibhatti} is a kind of process used when a verb (\pali{\=akhy\=ata}) is formed. Some of them have a peculiar operation. Sometimes it is so weird that new students have a hard time in recognizing verb forms. I summarize all rules relating to verbal \pali{vibhatti} application here. Some of them have already mentioned in the lessons. We will review all of them again.

\subsection*{\pali{Vibhatti} of pres.\ and imp.}

These two verb classes have several in common, so in the textbooks they are put together.

\paragraph*{(1) Lengthening \pali{a} to \pali{\=a}} (Kacc\,478, R\=upa\,438, Sadd\,959, Mogg\,6.57)\par
When \pali{hi, mi, ma, mhe} are applied, if the preceding ending is \pali{a}, lengthen it to \pali{\=a}, for example, \pali{gacch\=ahi, gacch\=ami, gacch\=ama, gacch\=amhe}.

\paragraph*{(2) Optional \pali{hi}} (Kacc\,479, R\=upa\,452, Sadd\,960, Mogg\,6.48)\par
In imperative mood, \pali{hi} is optional, so sometimes it can be left out, for example, \pali{gaccha/gama} (Go!).

\paragraph*{(3) Changing \pali{nti, nte} to \pali{re}} (Mogg\,6.74)\par
If the preceding vowel is short and it follows a strong syllable, \pali{nti} and \pali{nte} can be changed to \pali{re}, for example, \pali{gacchanti/gacchante} $\rightarrow$ \pali{gacchare} ([They] go).

\subsection*{\pali{Vibhatti} of opt.}

There are two ways \pali{vibhatti} of this verb class are applied. First, the forms of \pali{vibhatti} are maintained. This is easy to recognize, for example, \pali{paceyya, paceyyu\d m, paceyy\=asi, paceyy\=atha, paceyy\=ami, paceyy\=ama}. Second, certain forms can be transformed, for example, \pali{pace, pacu\d m, pacemu}. Here are some explanations.

\paragraph*{(1) Changing \pali{eyya, eyy\=asi, eyy\=ami}\footnote{In Mogg\,6.75, \pali{eyya\d m} is mentioned in stead of \pali{eyy\=ami}.} to \pali{e}} (Sadd\,1088, Mogg\,6.75)\par
This is optional. It looks handy, but less distinct. Here are some examples.\par
- \pali{so \textbf{kare}} (He should do.)\par
- \pali{tva\d m \textbf{kare}} (You should do.)\par
- \pali{aha\d m \textbf{kare}} (I should do.)\par
- \pali{\textbf{bhu\~nje}} ([One] should eat.)\par
- \pali{\textbf{gacche}} ([One] should go.)\par
- \pali{\textbf{care}} ([One] should travel.)\par

\paragraph*{(2) Changing \pali{eyy\=ama} to \pali{emu, omu}} (Sadd\,1070--1, Mogg\,6.78)\par
- \pali{\textbf{vih\=aremu}} ([We] should stay.)\par
- \pali{\textbf{j\=anemu}} ([We] should know.)\par
- \pali{\textbf{pappomu}} ([We] should attain.)\par
- \pali{\textbf{bhavemu/bhaveyy\=amu}}\footnote{Mogg\,6.78} ([We] should be.)\par

\paragraph*{(3) Changing \pali{eyyu\d m} to \pali{u\d m}} (Mogg\,6.47)\par
- \pali{\textbf{gacchu\d m}} ([They] should go.)\par

\subsection*{\pali{Vibhatti} of perf., imperf., aor., fut., and cond.}

These verb classes have serveral things in common explained as follows:

\paragraph*{(1) Insertion of \pali{i}} (Kacc\,516, R\=upa\,466, Sadd\,1030, Mogg\,6.35)\par
It is said that except imperfect tense all tenses and mood mentioned fall into this condition. Examples in Table \ref{tab:exgamu} show how \pali{i} is inserted. They all are for \pali{gamu} (to go).\par

\bigskip
\begin{longtable}[c]{@{}%
	>{\small\raggedright\arraybackslash}p{0.05\linewidth}%
	>{\small\itshape\raggedright\arraybackslash}p{0.19\linewidth}%
	>{\small\itshape\raggedright\arraybackslash}p{0.19\linewidth}%
	>{\small\itshape\raggedright\arraybackslash}p{0.19\linewidth}%
	>{\small\itshape\raggedright\arraybackslash}p{0.19\linewidth}@{}}
\label{tab:exgamu}\\
\toprule
& \multicolumn{2}{c}{\bfseries\itshape Parassapada} & \multicolumn{2}{c}{\bfseries\itshape Attanopada} \\
\cmidrule{2-5}
& \bfseries\upshape sg. & \bfseries\upshape pl. & \bfseries\upshape sg. & \bfseries\upshape pl.\\ \midrule
\endfirsthead
\toprule
& \multicolumn{2}{c}{\bfseries Parassapada} & \multicolumn{2}{c}{\bfseries Attanopada} \\
\cmidrule{2-5}
& \bfseries sg. & \bfseries pl. & \bfseries sg. & \bfseries pl.\\ \midrule
\endhead
\bottomrule
\ltblcontinuedbreak{5}
\endfoot
\bottomrule
\endlastfoot
%
\multicolumn{5}{c}{Perfect tense\footnote{Mogg\,6.6}} \\ \midrule
3rd & jagama & jagamu & jagam\textbf{i}ttha & jagam\textbf{i}re \\
2nd & jagame & jagam\textbf{i}ttha & jagam\textbf{i}ttho & jagam\textbf{i}vho \\
1st & jagama & jagam\textbf{i}mha & jagami & jagam\textbf{i}mhe \\
\newpage
\multicolumn{5}{c}{Aorist tense} \\ \midrule
3rd & agacchi, aga\~nchi & agacchu\d m, aga\~nchu\d m, agacch\textbf{i}\d msu & agacch\=a & agacch\=u \\
2nd & agaccho & agacch\textbf{i}ttha, aga\~nch\textbf{i}ttha & agacchase & \mbox{agacch\textbf{i}vha\d m} \\
1st & agacchi\d m, aga\~nchi\d m & agacch\textbf{i}mh\=a, aga\~nch\textbf{i}mh\=a & agacch\d m & gacch\textbf{i}mhe \\
\midrule
\multicolumn{5}{c}{Future tense} \\ \midrule
3rd & gam\textbf{i}ssati & gam\textbf{i}ssanti & gam\textbf{i}ssate & gam\textbf{i}ssante \\
2nd & gam\textbf{i}ssasi & gam\textbf{i}ssatha & gam\textbf{i}ssase & gam\textbf{i}ssavhe \\
1st & gam\textbf{i}ss\=ami & gam\textbf{i}ss\=ama & gam\textbf{i}ssa\d m & gam\textbf{i}ss\=amhe \\
\midrule
\multicolumn{5}{c}{Conditional mood} \\ \midrule
3rd & agam\textbf{i}ss\=a & \mbox{agam\textbf{i}ssa\d msu} & agam\textbf{i}ssatha & \mbox{agam\textbf{i}ssi\d msu} \\
2nd & agam\textbf{i}sse & agam\textbf{i}ssatha & agam\textbf{i}ssase & agam\textbf{i}ssavhe \\
1st & agam\textbf{i}ssa\d m & \mbox{agam\textbf{i}ss\=amh\=a} & agam\textbf{i}ssi\d m & \mbox{agam\textbf{i}ss\=amhe} \\
\end{longtable}

\paragraph*{(2) Changing \pali{i} insertion to \pali{e}} (Sadd\,1076, Mogg\,5.163)\par
This can be seen occasionally, for example, \pali{aggahesi, aggahesu\d m} (seized).

\paragraph*{(3) Prefixing with \pali{a}} (Kacc\,519, R\=upa\,457, Sadd\,1032, Mogg\,6.15)\par
In grammatical terms, this is called \emph{augment}.\footnote{\citealp[p.~23]{warder:intro}; \citealp[p.~75]{collins:grammar}} This is applied only to imperfect, aroist tense, and conditional mood. It is said that the appearance of \pali{a} is not always so. Examples are \pali{agam\=a} (imperf.), \pali{agam\=i} (aor.), and \pali{agamiss\=a} (cond.). See also in Table \ref{tab:exgamu}.

\paragraph*{(4) Shortening ending vowels} (Sadd\,1041, Mogg\,6.33)\par
It seems that this is a normal practice, for example:\par
- \pali{avoc\=a} $\rightarrow$ \palibf{avoca} (said)\par
- \pali{agacch\=i} $\rightarrow$ \palibf{agacchi} (went)\par
- \pali{gam\=a} $\rightarrow$ \palibf{gama} (went)\par
- \pali{gam\=i} $\rightarrow$ \palibf{gami} (went)\par
- \pali{gamimh\=a} $\rightarrow$ \palibf{gamimha} (went)\par
- \pali{gamissamh\=a}\footnote{In Mogg\,6.7, it is \pali{gamiss\=amh\=a}.} $\rightarrow$ \palibf{gamissamha} (had gone)\par

\paragraph*{(5) Other changes} (Mogg\,6.38)\par
Occasionally, there are some other substitution to be found. To me, these seem not to be a good practice. They are mentioned in Mogg, for example:\par
- \pali{tumhe \pali{bhaveyy\=atha} $\rightarrow$ \palibf{bhaveyy\=atho}} [opt.] (You [all] should be.)\par
- \pali{tva\d m \pali{abhavisse} $\rightarrow$ \palibf{abhavissa}} (You was.)\par
- \pali{aha\d m \pali{abhava} $\rightarrow$ \palibf{abhava\d m}} (I was.)\par
- \pali{so \pali{abhav\=a} $\rightarrow$ \palibf{abhavittha}} (He was.)\par
- \pali{so \pali{abhav\=i} $\rightarrow$ \palibf{abhavittho}} (He was.)\par
- \pali{tumhe \pali{bhavatha} $\rightarrow$ \palibf{bhavathavho}} [imp.] (Let you be.)\par

\paragraph*{(6) Reduplication in perfect verbs} (Mogg\,5.70)\par
A marked characteristic of perfect verbs is reduplication, for example, \pali{jagama} ([One] went). For more information, see Chapter \ref{chap:vform}, page \pageref{sec:redup}.

\paragraph*{(7) Transformation of \pali{u\d m} in aorists} (Kacc\,504, R\=upa\,470, Sadd\,1016--7, Mogg\,6.39--40)\par
For 3rd person plural of aorist verbs, \pali{u\d m} can be changed to \pali{i\d msu, a\d msu, su\d m,} or \pali{\=asu\d m}, for example:\par
- \pali{upa + sa\d m + kamu + a + u\d m} = \pali{\textbf{upasa\.nkami\d msu}} ([They] approached.)\par
- \pali{ni + sada + a + u\d m} = \pali{\textbf{nis\=idi\d msu}} ([They] sat down.)\par
- \pali{disa + a + u\d m} = \pali{\textbf{addas\=asu\d m}} ([They] saw.)\par
- \pali{gamu + a + u\d m} = \pali{\textbf{agamu\d m/agami\d msu/agama\d msu}} ([They] went.)\par
- \pali{n\=i + a + u\d m} = \pali{\textbf{nesu\d m/nayi\d msu}} ([They] led.)\par

\paragraph*{(8) Insertion of \pali{sa} in aorists} (Sadd\,1067, 1075, Mogg\,6.44, 6.46)\par
In some aorist verbs, we sometimes see them with \pali{si} ending. It is explained in Sadd that \pali{sa} is inserted and \pali{\=i} (3rd person sg.) is normally shortened to \pali{i}. In Mogg, it is said that \pali{\=i} itself is changed to \pali{si}. For \pali{i\d m} (1st person sg.), \pali{mh\=a} (1st person pl.), and \pali{ttha} (2nd person pl.), insertion of \pali{si} is normally found (Mogg\,6.46). Here are some examples:\par
- \pali{kara + a + \=i} = \pali{\textbf{ak\=asi}} ([One] did.)\par
- \pali{d\=a + a + \=i} = \pali{\textbf{ad\=asi}} ([One] gave.)\par
- \pali{kara + a + i\d m} = \pali{\textbf{ak\=asi\d m}} ([I] did.)\par
- \pali{kara + a + mh\=a} = \pali{\textbf{ak\=asimh\=a}} ([We] did.)\par
- \pali{kara + a + ttha} = \pali{\textbf{ak\=asittha}} ([You all] did.)\par

\paragraph*{(9) Insertion of \pali{u} in aorists} (Mogg\,6.45)\par
For \pali{mh\=a} and \pali{ttha}, another insertion can be found is \pali{u}, for example:\par
- \pali{gamu + a + mh\=a} = \pali{\textbf{agamumh\=a}} ([We] went.)\par
- \pali{gamu + a + ttha} = \pali{\textbf{agamuttha}} ([You all] went.)\par

\paragraph*{(10) Transformation of \pali{o} in aorists} (Mogg\,6.42--3)\par
In 2rd person singular of aorist verbs, \pali{o} can be changed to other forms, for example:\par
- \pali{bh\=u + a + o} = \pali{tva\d m \textbf{abhavo/abhava/abhavi/\\abhavittha/abhavittho}} ([You] was.)\par
- \pali{h\=u + a + o} = \pali{tva\d m \textbf{ahuvo/ahosi}} ([You] was.)\par

\paragraph*{(11) Changing \pali{i\d m} to \pali{issa\d m} in aorists} (Sadd\,1103)\par
For 1st person singular, this condition occurs in verses. Here are examples from the canon:\par
\begin{quote}
``\pali{sandh\=avissa\d m anibbisa\d m}''\footnote{Dhp\,11.153}\\
(I did not find out, transmigrated.)\\[1.5mm]
``\pali{Uposatha\d m upavasissa\d m}''\footnote{Vv\,130}\\
(I observed the eight precepts.)\\[1.5mm]
``\pali{nirayamhi apaccisa\d m}''\footnote{Thig\,15.438. To maintain the meter, one \pali{s} is dropped.}\\
(I was burned in hell.)
\end{quote}

\paragraph*{(12) Elision of \pali{ssa} in fut.} (Sadd\,1139, Mogg\,6.69)\par
Occasionally, for some roots \pali{ssa} part of the future \pali{vibhatti} can be omitted, for example:\par
- \pali{dakkhissati} $\rightarrow$ \palibf{dakkhati} ([One] will see.)\par
- \pali{sakkhissati} $\rightarrow$ \palibf{sakkhati} ([One] will be able.)\par
- \pali{hehissati} $\rightarrow$ \palibf{hehiti} ([One] will be.)\par
- \pali{hohissati} $\rightarrow$ \palibf{hohiti} ([One] will be.)\par
- \pali{vik\=asissati} $\rightarrow$ \palibf{vik\=asati} ([One] will expand.)\par

\section{Irregular Verb Forms}\label{sec:irrverb}

In P\=ali verb formation, some roots are easy to deal with. For example, \pali{paca} (to cook) is in the top list of verbs exemplified. But many of common verbs are not that easy. Some verbs have several forms, even when composed with the same \pali{paccaya} and \pali{vibhatti}. That can give new students a hard time. As we have seen so far, to learn verb system in P\=ali is mostly to learn the irregularity of it. This is true for noun system as well. It sounds like an irony. Many of rules posited by the tradition can be seen as systematization of irregularity of the language. 

In this section, peculiar verb forms are listed. Only some noteworthy instances wil be shown here. The left-out are supposed to be easy to render in a regular way. Or if they are not found in the texts, it is logical to follow the regular rendition. If tabular form is suitable, I will show verbs in a table. If they are just a few of them, I will show the verbs with their \pali{vibhatti} instead. If it is not stated otherwise, the forms are of active voice (\pali{parassapada}). If you feel unclear about the material presented below, see Chapter \ref{chap:vclass} and \ref{chap:vform} for more detail. For yet more comprehensive information on verb forms, please consult Sadd-Dh\=a directly.

\paragraph*{I (to go)} (Mogg\,6.66)\par
- \pali{i + a + ssati} = \palibf{ehiti} (fut.\ 3rd person sg.)\par

\paragraph*{Asa (to be)} (Kacc\,492--6, 505, R\=upa\,495--99, 500, Sadd\,987--9, 991--9, 1000--2, 1019, Mogg\,6.50--6, 5.130)\par

\newpage
\begin{conjextable}
\multicolumn{3}{@{}l}{Present tense} \\\midrule
3rd & atthi & santi \\
2nd & asi & attha \\
1st & asmi, amhi & asma, amha \\
\midrule
\multicolumn{3}{@{}l}{Imperative mood} \\\midrule
3rd & atthu & santu \\
2nd & ahi & attha \\
1st & asmi, amhi & asma, amha \\
\midrule
\multicolumn{3}{@{}l}{Optative mood} \\\midrule
3rd & siy\=a, assa & siyu\d m, assu, siya\d msu \\
2nd & assa & assatha \\
1st & siya\d m, assa\d m\footnote{For 1st person sg., \pali{vibhatti} of \pali{attanopada} (middle voice) is normally used. I do not see \pali{ass\=ami} used in the texts.} & ass\=ama \\
\midrule
\multicolumn{3}{@{}l}{Perfect tense} \\\midrule
3rd & asa & \\
2nd & & \\
1st & & \\
\midrule
\multicolumn{3}{@{}l}{Aorist tense} \\\midrule
3rd & \=asi & \=asi\d msu, \=asu\d m \\
2nd & \=asi & \=asittha \\
1st & \=asi\d m & \=asimha \\
\end{conjextable}

For future tense and conditional mood of \pali{asa}, corresponding forms of \pali{bh\=u} are used instead, e.g.\ \pali{bhavissati, abhavissa}.\footnote{Kacc\,507, R\=upa\,501, Sadd\,1020, Mogg\,5.128--9}

\paragraph*{Kara (to do)} (Kacc\,512, 491, 481, R\=upa\,522--4, Sadd\,962, 983, 1025--6, 1077--9, 1081--87, 1089, Mogg\,5.177, 6.23--5, 6.70--2)\par

\newpage
\begin{conjextable}
\multicolumn{3}{@{}l}{Present tense (\pali{parassapada}) (method 1)} \\\midrule
3rd & karoti & karonti \\
2nd & karosi & karotha \\
1st & karomi, kummi\footnote{This and \pali{kumma} come from Mogg\,6.23.} & karoma, kumma \\
\midrule
\multicolumn{3}{@{}l}{Present tense (\pali{attanopada}) (method 1)} \\\midrule
3rd & kurute & kubbante\footnote{Interestingly, no \pali{kurunte} is ever found.} \\
2nd & kuruse & kuruvhe \\
1st & kare & karumhe \\
\midrule
\multicolumn{3}{@{}l}{Present tense (\pali{parassapada}) (method 2)} \\\midrule
3rd & kubbati & kubbanti \\
2nd & kubbasi & kubbatha \\
1st & kubb\=ami & kubb\=ama \\
\midrule
\multicolumn{3}{@{}l}{Present tense (\pali{attanopada}) (method 2)} \\\midrule
3rd & kubbate & kubbante \\
2nd & kubbase & kubbavhe \\
1st & kubbe & kubbamhe \\
\midrule
\multicolumn{3}{@{}l}{Present tense (\pali{parassapada}) (method 3)} \\\midrule
3rd & kayirati & kayiranti \\
2nd & kayirasi & kayiratha \\
1st & kayir\=ami & kayir\=ama \\
\midrule
\multicolumn{3}{@{}l}{Present tense (\pali{attanopada}) (method 3)} \\\midrule
3rd & kayirate & kayirante \\
2nd & kayirase & kayiravhe \\
1st & kayire & kayiramhe \\
\midrule\newpage\midrule
\multicolumn{3}{@{}l}{Optative mood (\pali{parassapada})} \\\midrule
3rd & kayir\=a & kayiru\d m \\
2nd & kayir\=asi & kayir\=atha \\
1st & kayir\=ami & kayir\=ama \\
\midrule
\multicolumn{3}{@{}l}{Optative mood (\pali{attanopada})} \\\midrule
3rd & kayir\=atha & kayirera\d m \\
2nd & kayiretho & kayir\=avho \\
1st & kayira\d m & kayir\=amhe \\
\midrule
\multicolumn{3}{@{}l}{Imperfect tense} \\\midrule
3rd & ak\=a\footnote{Sadd\,1089. For example, ``\pali{ak\=a loke sudukkara\d m}'' (Ja\,4:8), ``[He] did a hard thing.''} & \\
2nd & & \\
1st & & \\
\midrule
\multicolumn{3}{@{}l}{Aorist tense (method 1)} \\\midrule
3rd & akari, kari & akari\d msu, kari\d msu, aka\d msu \\
2nd & akaro & akarittha \\
1st & akari\d m, kari\d m & akarimha, karimha \\
\midrule
\multicolumn{3}{@{}l}{Aorist tense (method 2)} \\\midrule
3rd & ak\=asi & ak\=asu\d m \\
2nd & ak\=aso & ak\=asittha \\
1st & ak\=asi\d m & ak\=asimha \\
\midrule
\multicolumn{3}{@{}l}{Future tense (method 1)} \\\midrule
3rd & karissati & karissanti \\
2nd & & \\
1st & & \\
\midrule
\multicolumn{3}{@{}l}{Future tense (method 2)} \\\midrule
3rd & k\=ahati & k\=ahanti \\
2nd & & \\
1st & & \\
\midrule\newpage\midrule
\multicolumn{3}{@{}l}{Future tense (method 3)} \\\midrule
3rd & k\=ahiti & k\=ahinti \\
2nd & & \\
1st & & \\
\end{conjextable}

There are other minor issues with \pali{kara}, for example:\par 
- \pali{kara + a + ssate} = \palibf{kassa\d m}\footnote{Sadd\,1037. An instance found in the canon is ``\pali{ahamapi kassa\d m p\=uja\d m}'' (Pv\,250), ``Even I will do the homage.''} (middle fut.\ 3rd person sg.)\par
- \pali{abhi + sa\d m + kara + a + ti} = \palibf{abhisa\.nkharoti}\footnote{Sadd\,1090, see also Mogg\,5.133--4} ([One] prepares or restores.)\par

\paragraph*{\=A-kusa (to insult)} (Kacc\,498, R\=upa\,480, Sadd\,1004, Mogg\,6.34)\par
The present form of this is \pali{akkosati} (Sadd\,1046). It also has an odd aorist form.\par
- \pali{\=a + kusa + a + \=i} = \palibf{akkocchi} (aor.\ 3rd person sg.)\par

\paragraph*{Gamu (to go)} (Sadd\,1091--5, 1104, Mogg\,6.29--30)\par
Some forms of this root is already demonstrated on page \pageref{tab:exgamu}. There are other forms shown below.

\begin{conjextable}
\multicolumn{3}{@{}l}{Imperfect tense (\pali{parassapada})} \\\midrule
3rd & agacch\=a & agacch\=u \\
2nd & agaccho & agacchattha \\
1st & agaccha\d m & agacchamha \\
\midrule
\multicolumn{3}{@{}l}{Imperfect tense (\pali{attanopada})} \\\midrule
3rd & agacchatha & agacchatthu\d m \\
2nd & agacchase & agacchavha\d m \\
1st & agacchi\d m, aga\~nchi\d m & agacchamhase \\
\multicolumn{3}{@{}l}{Aorist tense (\pali{parassapada})} \\\midrule
3rd & agami, agam\=asi, (ag\=a) & agamu, agama\d msu \\
2nd & agamo & agamittha, agamuttha \\
1st & agami\d m & agamimha, agamumha \\
\midrule\newpage\midrule
\multicolumn{3}{@{}l}{Aorist tense (\pali{attanopada})} \\\midrule
3rd & agam\=a & agamu \\
2nd & agase & agavha\d m \\
1st & aga\d m & agamhe, agamumhe \\
\end{conjextable}

Sometimes \pali{gamu} is shortened to just \pali{ga} (Sadd\,1095) which gives the form of \pali{ag\=a} (imperf.\ and aor.) and the like. Here are some examples of these:\par
- \pali{so dhana\d m \textbf{ajjhag\=a}.}\footnote{This is equal to \pali{adhigacchi}.} (He obtained wealth.)\par
- \pali{te \textbf{ajjhagu}.} (They obtained.)\par
- \pali{sop\textbf{\=ag\=a} samiti\d m vana\d m.}\footnote{D2\,335 (DN\,20)} (Even he went to the forest, the meeting place.)\par
- \pali{Kambalassatar\=a \textbf{\=agu\d m}.}\footnote{D2\,338 (DN\,20)} ([N\=aga] Kambala and Assatara went.)\par
- \pali{ta\d nh\=ana\d m khayam\textbf{ajjhag\=a}.}\footnote{Some use \pali{\=a} ending in 1st person (Sadd\,1104). This instance is from Dhp\,11.154.} ([I] attained the destruction of craving.)\par

Moreover, as noted in Sadd-Dh\=a\,845, there are also special forms of perfect \pali{gamu} described below:

- \pali{so puriso magga\d m \textbf{ga}.} (That man went the path.)\par
- \pali{s\=a itth\=i gharam\textbf{\=aga}.} (That woman came home.)\par
- \pali{te magga\d m \textbf{gu}.} (Those [men] went the path.)\par
- \pali{t\=a gharam\textbf{\=agu}.} (Those [women] came home.)\par
- \pali{tva\d m magga\d m \textbf{ga}.} (You went the path.)\par
- \pali{tva\d m gharam\textbf{\=aga}.} (You came home.)\par
- \pali{tumhe magga\d m \textbf{guttha}.} (You [all] went the path.)\par
- \pali{tumhe gharam\textbf{\=aguttha}.} (You [all] came home.)\par
- \pali{aha\d m magga\d m \textbf{ga\d m}.} (I went the path.)\par
- \pali{aha\d m gharam\textbf{\=aga\d m}.} (I came home.)\par
- \pali{aha\d m ta\d m purisa\d m \textbf{anvaga\d m}.} (I followed that man.)\par
- \pali{mayha\d m magga\d m \textbf{gumha}} (We went the path.)\par
- \pali{mayha\d m ghara\textbf{\=agumha}} (We came home.)\par
- \pali{mayha\d m ta\d m purisa\d m \textbf{anvagumha}.} (We followed that man.)\par
- \pali{sop\textbf{\=aga} samiti\d m vana\d m.}\footnote{D2\,341 (DN\,20), also \pali{\=ag\=a} in 335 and 338.} (Even he went to the forest, the meeting place.)\par
- \pali{\textbf{\=agu\d m} dev\=a yasassino.}\footnote{D2\,340 (DN\,20)} (Came renowned deities.)\par
- \pali{M\=aha\d m k\=akova dummedho, k\=am\=ana\d m vasam\textbf{anvaga\d m}}\footnote{Ja\,19:37} (I won't be foolish as a crow which followed the control of pleasures.)\par

\paragraph*{Chidi (to cut)} (Sadd\,1096, 1098, Mogg\,6.26)\par
- \pali{chidi + a + \=i} = \palibf{acchecchi} (aor.\ 3rd person sg.)\par
- \pali{chidi + a + u\d m} = \palibf{acchecchu\d m} (aor.\ 3rd person pl.)\par
- \pali{chidi + a + o} = \palibf{accheccho} (aor.\ 2rd person sg.)\par
- \pali{chidi + a + ttha} = \palibf{acchecchittha} (aor.\ 2rd person pl.)\par
- \pali{chidi + a + ssati} = \palibf{checchati} (fut.\ 3rd person sg.)\par
- \pali{chidi + a + ssasi} = \palibf{checchasi} (fut.\ 2rd person sg.)\par
- \pali{chidi + a + ss\=a} = \palibf{achecch\=a} (cond.\ 3rd person sg.)\par

\paragraph*{\~N\=a (to know)} (Kacc\,508, R\=upa\,515, Sadd\,1021, Mogg\,6.63--5)

The present form of this root is \pali{j\=an\=ati} (see Chapter \ref{chap:vform}). Some other unusual forms mentioned are:\par
- \pali{\~n\=a + a + eyya} = \palibf{ja\~n\~n\=a, j\=aniy\=a} (opt.\ 3rd person sg.)\par
- \pali{\~n\=a + a + \=i} = \palibf{a\~n\~n\=asi} (aor.\ 3rd person sg.)\par
- \pali{\~n\=a + a + ssati} = \palibf{\~nassati} (fut.\ 3rd person sg.)\par
- \pali{pa + \~n\=a + ya + i + ssati} = \palibf{pa\~n\~n\=ayihiti} (pass.\ fut.\ 3rd person sg.)\par

\paragraph*{Da\d msa\footnote{In Mogg this root is called \pali{\d dansa}.} (to bite)} (Mogg\,6.30)\par
- \pali{da\d msa + a + \=a} = \palibf{a\d da\~nch\=a} (imperf.\ 3rd person sg.)\par
- \pali{da\d msa + a + \=i} = \palibf{a\d da\~nch\=i} (aor.\ 3rd person sg.)\par

\paragraph*{D\=a (to give)} (Kacc\,482, R\=upa\,508, Sadd\,972, 1007--9, Mogg\,6.22)

\newpage
\begin{conjextable}
\multicolumn{3}{@{}l}{Present tense (method 1)} \\\midrule
3rd & dad\=ati & dadanti \\ 
2nd & dad\=asi & dad\=atha \\
1st & dad\=ami & dad\=ama \\
\midrule
\multicolumn{3}{@{}l}{Present tense (method 2)} \\\midrule
3rd & deti & denti \\
2nd & desi & detha \\
1st & demi, dammi & dema, damma \\
\midrule
\multicolumn{3}{@{}l}{Present tense (method 3)} \\\midrule
3rd & dajjati & dajjanti \\
2nd & dajjasi & dajjatha \\
1st & dajj\=ami & dajj\=ama \\
\midrule
\multicolumn{3}{@{}l}{Imperative mood} \\\midrule
3rd & detu & dentu \\
2nd & dehi & detha \\
1st & demi, dammi & dema, damma \\
\midrule
\multicolumn{3}{@{}l}{Optative mood} \\\midrule
3rd & dajjeyya, dajje, dajj\=a & dajjeyyu\d m, dajju\d m\\
2nd & & \\
1st & dajjeyy\=ami, dajja\d m & \\
\end{conjextable}

\paragraph*{Bh\=u (to be)} (Kacc\,475, R\=upa\,469, Sadd\,956, Mogg\,6.17--8)

\begin{conjextable}
\multicolumn{3}{@{}l}{Perfect tense} \\\midrule
3rd & babh\=uva & babh\=uvu \\
2nd & babh\=uve & babh\=uvittha \\
1st & babh\=uva\d m & babh\=uvimha \\
\end{conjextable}

\paragraph*{Br\=u (to say)} (Kacc\,520, R\=upa\,502, Sadd\,1033, Mogg\,6.36; Kacc\,475, R\=upa\,469, Sadd\,956, Mogg\,6.16, 6.19--20, 5.97; Sadd\,984--6)

\begin{conjextable}
\multicolumn{3}{@{}l}{Present tense} \\\midrule
3rd & brav\=iti & brunti \\
2nd & br\=usi & br\=utha \\
1st & br\=umi & br\=uma \\
\midrule
\multicolumn{3}{@{}l}{Perfect tense} \\\midrule
3rd & \=aha & \=ahu, \=aha\d msu\footnote{Mogg\,6.19} \\
2nd & brave & bravittha \\
1st & & \\
\midrule
\multicolumn{3}{@{}l}{Aorist tense} \\\midrule
3rd & abravi\footnote{Mogg\,5.97}, (payirud)\=ah\=asi, (paby)\=ah\=asi\footnote{These forms are mentioned in Sadd\,984--6 concerning \pali{sa} insertion.} & (payirud)\=aha\d msu, (paby)\=aha\d msu \\
2nd & & \\
1st & (payitud)\=ah\=asi\d m, (paby)\=ah\=asi\d m & \\
\end{conjextable}

\paragraph*{Bhidi (to break)} (Sadd\,1097, Mogg\,6.26)\par
- \pali{bhidi + a + \=i} = \palibf{abhecchi} (aor.\ 3rd person sg.)\par
- \pali{bhidi + a + u\d m} = \palibf{abhecchu\d m} (aor.\ 3rd person pl.)\par
- \pali{bhidi + a + o} = \palibf{abheccho} (aor.\ 2rd person sg.)\par
- \pali{bhidi + a + ttha} = \palibf{abhecchittha} (aor.\ 2rd person pl.)\par
- \pali{bhidi + a + ssati} = \palibf{bhecchati} (fut.\ 3rd person sg.)\par
- \pali{bhidi + a + ss\=a} = \palibf{abhecch\=a} (cond.\ 3rd person sg.)\par

\paragraph*{Bhuja (to eat)} (Sadd\,1060--1, Mogg\,6.27)\par
- \pali{bhuja + a + ssati} = \palibf{bhokkhati} (fut.\ 3rd person sg.)\par
- \pali{bhuja + a + ssanti} = \palibf{bhokkhanti} (fut.\ 3rd person pl.)\par
- \pali{bhuja + a + ss\=a} = \palibf{abhokkh\=a} (cond.\ 3rd person sg.)\par

\paragraph*{Muca (to release)} (Mogg\,6.27)\par
- \pali{muca + a + ssati} = \palibf{mokkhati} (fut.\ 3rd person sg.)\par
- \pali{muca + a + ss\=a} = \palibf{amokkh\=a} (cond.\ 3rd person sg.)\par

\paragraph*{Ruda (to cry)} (Sadd\,1045, Mogg\,6.26)\par
A typical present form of this is \pali{rodati}. Thus a typical future form is \pali{rodissati}. Some odd forms are also be found.\par
- \pali{ruda + a + ssati} = \palibf{rucchati} (fut.\ 3rd person sg.)\par
- \pali{ruda + a + ss\=a} = \palibf{arucch\=a} (cond.\ 3rd person sg.)\par

\paragraph*{Ruha (to grow)} (Mogg\,6.34)\par
- \pali{abhi + ruha + a + \=i} = \palibf{abhirucchi} (aor.\ 3rd person sg.)\par

\paragraph*{Labha (to get)} (Kacc\,497, R\=upa\,477, Sadd\,964, 966, 968, 1003, Mogg\,6.26, 6.73)\par

\begin{conjextable}
\multicolumn{3}{@{}l}{Aorist tense} \\\midrule
3rd & alattha\footnote{It is said that \pali{\=i} is changed to \pali{ttha} and the last syllable of the root is deleted. An example from the canon is ``\pali{Alattha kho so\d no ko\d liviso bhagavato santike pabbajja\d m, alattha upasampada\d m}'' (Mv\,5.243), ``So\d na Ko\d livisa got ordination, [and] the highest ordination in the Blessed One's vicinity.''} & \\
2nd & & \\
1st & alattha\d m & \\
\midrule
\multicolumn{3}{@{}l}{Future tense} \\\midrule
3rd & lacchati & lacchanti \\
2nd & lacchasi & lacchatha \\
1st & lacch\=ami & lacch\=ama \\
\midrule
\multicolumn{3}{@{}l}{Conditional mood} \\\midrule
3rd & alacch\=a & \\
2nd & & \\
1st & & \\
\end{conjextable}

\paragraph*{Vaca (to say)} (Kacc\,477, R\=upa\,479, Sadd\,958, 963, 965, 970--1, 1043--4, Mogg\,6.21, 6.27)

\newpage
\begin{conjextable}
\multicolumn{3}{@{}l}{Imperfect tense (\pali{parassapada})} \\\midrule
3rd & avac\=a & avac\=u \\
2nd & avaco & avacuttha \\
1st & avaca\d m & avacumha \\
\midrule
\multicolumn{3}{@{}l}{Imperfect tense (\pali{attanopada})} \\\midrule
3rd & avacuttha & avacutthu\d m \\
2nd & avacase & avacavha\d m \\
1st & avaci\d m & avacamhase \\
\midrule
\multicolumn{3}{@{}l}{Aorist tense (\pali{parassapada})} \\\midrule
3rd & avaci, avoca & avocu\d m, avaci\d msu \\
2nd & avoco & avocuttha \\
1st & avoci\d m & avocumha \\
\midrule
\multicolumn{3}{@{}l}{Aorist tense (\pali{attanopada})} \\\midrule
3rd & avoc\=a & avocu \\
2nd & avacase & avocivha \\
1st & avoca\d m & avocimhe \\
\midrule
\multicolumn{3}{@{}l}{Future tense (\pali{parassapada})} \\\midrule
3rd & vakkhati\footnote{It is said in Sadd\,971 that \pali{vaca} is changed to \pali{vakkha} in future tense. Thus, in normal form it can also be rendered as \pali{vakkhissati, vakkhissanti}, and so on.} & vakkhanti \\
2nd & vakkhasi & vakkhatha \\
1st & vakkh\=ami & vakk\=ama \\
\midrule
\multicolumn{3}{@{}l}{Future tense (\pali{attanopada})} \\\midrule
3rd & vakkhate & vakkhante \\
2nd & vakkhase & vakkhavhe \\
1st & vakkha & vakk\=amhe \\
\midrule
\multicolumn{3}{@{}l}{Conditional mood (\pali{parassapada})} \\\midrule
3rd & avaciss\=a, avakkh\=a\footnote{Mogg\,6.27} & \\
2nd & & \\
1st & & \\
\end{conjextable}

\enlargethispage{2\baselineskip}
\paragraph*{Vada (to say)} (Sadd\,1010--1)

\begin{conjextable}
\multicolumn{3}{@{}l}{Present tense (method 1)} \\\midrule
3rd & vadati & vadanti, vadenti \\
2nd & vadasi & vadatha \\
1st & vad\=ami & vad\=ama \\
\midrule
\multicolumn{3}{@{}l}{Present tense (method 2)} \\\midrule
3rd & vajjati & vajjanti, vajjenti \\
2nd & vajjasi & vajjatha \\
1st & vajj\=ami & vajj\=ama \\
\midrule
\multicolumn{3}{@{}l}{Optative mood} \\\midrule
3rd & & \\
2nd & vajj\=asi & \\
1st & & \\
\end{conjextable}

\paragraph*{Vasa (to live)} (Sadd\,968, Mogg\,6.26)\par
- \pali{vasa + a + ssati} = \palibf{vacchati} (fut.\ 3rd person sg.)\par
- \pali{vasa + a + ss\=a} = \palibf{avacch\=a} (cond.\ 3rd person sg.)\par

\paragraph*{Visa (to enter)} (Sadd\,1047, Mogg\,6.27)\par
- \pali{pa + visa + a + \=i} = \palibf{p\=avekkhi/p\=avisi} (aor.\ 3rd person sg.)\par
- \pali{pa + visa + a + ssati} = \palibf{pavekkhati} (fut.\ 3rd person sg.)\par
- \pali{pa + visa + a + ss\=a} = \palibf{pavekkh\=a} (cond.\ 3rd person sg.)\par

\paragraph*{Saka (to be able)} (Sadd\,1065, Mogg\,6.58--9)\par
- \pali{saka + a + \=i} = \palibf{asakkhi/sakkhi} (aor.\ 3rd person sg.)\par
- \pali{saka + a + u\d m} = \palibf{asakkhi\d msu} (aor.\ 3rd person pl.)\par
- \pali{saka + a + ssati} = \palibf{sakkhissati} (fut.\ 3rd person sg.)\par
- \pali{saka + a + ssanti} = \palibf{sakkhissanti} (fut.\ 3rd person pl.)\par
- \pali{saka + a + ss\=a} = \palibf{sakkhiss\=a} (fut.\ 3rd person sg.)\par
- \pali{saka + a + ssa\d msu} = \palibf{sakkhissa\d msu} (fut.\ 3rd person pl.)\par

\paragraph*{Hana (to kill)} (Sadd\,967, 969, Mogg\,6.67)\par
- \pali{hana + a + ssati} = \palibf{ha\.nkhati} (fut.\ 3rd person sg.)\par
- \pali{hana + a + ss\=ami} = \palibf{ha\~nch\=ami} (pres.\ 1st person sg.)\par
- \pali{pati + hana + a + mi} = \palibf{pa\d tiha\.nkh\=ami} (pres.\ 1st person sg.)\par
- \pali{pati + hana + a + ma} = \palibf{pa\d tiha\.nkh\=ama} (pres.\ 1st person pl.)\par
- \pali{pati + hana + a + ssati} = \palibf{pa\d tiha\.nkhati} (fut.\ 3rd person sg.)\par

\paragraph*{Hara (to carry)} (Sadd\,1038, Mogg\,6.28)\par
- \pali{vi + hara + a + ssati} = \palibf{vihassati}\footnote{For example, ``\pali{appamatto vihassati}'' (S1\,185 (SN\,6)), ``[One] will live carefully.''} ([One] will live)\par
- \pali{hara + a + \=a} = \palibf{ah\=a/ahar\=a} (imperf.\ 3rd person sg.)\par
- \pali{hara + a + \=i} = \palibf{ah\=asi/ahari} (aor.\ 3rd person sg.)\par

\paragraph*{H\=a (to abandon)} (Mogg\,6.68, see also 6.25)\par
A present form of this root is \pali{jahati}, so the future form of it is \pali{jahissati}. This also has an odd form:\par
- \pali{h\=a + a + ssati} = \palibf{h\=ahati} (fut.\ 3rd person sg.)\par

\paragraph*{H\=u (to be)} (Sadd\,1025, 1051, 1053--4, Mogg\,6.41, 6.43)\par
This has typical present forms as \pali{hoti, honti}, and imperative form \pali{hotu, hontu}. Other peculiar forms can also be found.

\begin{conjextable}
\multicolumn{3}{@{}l}{Optative mood} \\\midrule
3rd & huveyya & \\
2nd & & \\
1st & & \\
\midrule
\multicolumn{3}{@{}l}{Perfect tense} \\\midrule
3rd & huva & huvu \\
2nd & & \\
1st & & \\
\midrule
\multicolumn{3}{@{}l}{Imperfect tense} \\\midrule
3rd & ahuv\=a & ahuv\=u \\
2nd & & \\
1st & & \\
\midrule\newpage\midrule
\multicolumn{3}{@{}l}{Aorist tense (\pali{parassapada})} \\\midrule
3rd & ahu\footnote{For example, ``\pali{Yo so ahu r\=aj\=a p\=ay\=asi n\=ama}'' (Pv\,605), ``There was a king called P\=ay\=asi.''}, ahosi & ahavu\d m, ahu\d m, ahesu\d m \\
2nd & ahuvo, ahosi\footnote{For example, ``\pali{kattha ca tva\d m ahosi}'' (Mv\,4.237), ``Where was you?''} & ahuvittha, ahosittha\\
1st & ahuv\=asi\d m, ahu\d m\footnote{For example, ``\pali{Aha\d m keva\d t\d tag\=amasmi\d m, ahu\d m keva\d t\d tad\=arako}'' (Ap1\,39:86), ``I, in a fisherman village, was a fisherboy.''}, ahosi\d m & ahumh\=a, ahosimh\=a \\
\midrule
\multicolumn{3}{@{}l}{Aorist tense (\pali{attanopada})} \\\midrule
3rd & ahuv\=a & ahuvu \\
2nd & ahuvase & ahuvivha \\
1st & ahuva\d m, ahu\d m & ahuvimhe \\
\midrule
\multicolumn{3}{@{}l}{Conditional mood} \\\midrule
3rd & ahuviss\=a & ahuvissa\d msu \\
2nd & & \\
1st & & \\
\end{conjextable}

Future forms of \pali{h\=u} have various renditions (Kacc\,480, R\=upa\,523, Sadd\,961, Mogg\,6.31, 6.69), i.e.\ \pali{hehiti, hehinti}; \pali{hohiti, hohinti}; \pali{heti, henti}; \pali{hehissati, hehissanti}; \pali{hohissati, hohissanti}; \pali{hessati, hessanti}.

