\chapter{Conversations}\label{chap:conver}

Our last lesson here is all about conversation. All knowledge we have learned so far will be applied here. The main guideline of how should we put words into daily speech comes from A.\,P.\,Buddhadatta's \emph{Aids to Pali Conversation and Translation}.\footnote{\citealp{buddhadatta:aids}} I also bring some parts of dialogues presented in the book here. There are many more interesting stories translated into P\=ali in that book. Please consult the book for richer resources. My main purpose is not to make a traveller's phrase book, but rather to show an application of the language in contemporary context. If you understand how these sentences come, it will be easy for you to make your own conversation lines. That is the reason I deliberately put explanation along side with the conversations. Some parts here, however, are not exactly in dialogue form with a connected story, but rather a list of sentences suitable to the context concerned. There are also minor concerns towards the end of this chapter. This chapter does not suppose rigorous understanding in the language. So you can read this before you finish all previous chapters. That is a reason I insert a lot of cross references here.

\phantomsection
\addcontentsline{toc}{section}{First Meeting}
\section*{First Meeting}

Sentences in this section are essential for opening a conversation with someone unfamiliar. The conversation lines in this section mainly come from Ven.\,Buddhadatta's \emph{Aids}\footnote{\citealp[pp.~47--50]{buddhadatta:aids}} with some modification. All explanations are mine.

\bigskip
\setcounter{convnum}{1}
\arabic{convnum}. \pali{Suppabh\=ata\d m!, bhante.}\\
\hspace*{10mm}Good morning!, sir.\\
{\small $\triangleright$ It is not customary in P\=ali speaking world, if there is such a thing, to have this kind of greeting. To make Western learners comfortable, however, we should begin with this starter. The word used here has a few occurrences in the canon, but not in this use. Literally, \pali{suppabh\=ata\d m} (\pali{su + pabh\=ata}) means `a good daybreak.'\footnote{The term is in nominative case, hence nt.} For the use of prefixes, see Appendix \ref{chap:upasagga}. A fuller form of this phrase is \pali{tuyha\d m suppabh\=ata\d m} (Good morning to you!).

By the same method, we can create other greeting phrases as follows: \pali{sv\=apara\d nho} [\pali{su + apara\d nha}] (good afternoon), \pali{susa\~njh\=a} or \pali{sus\=aya\d nho} (good evening). However, in Buddhist culture using \pali{sotthi} for greeting in all time may sound more preferable.

While saying `good morning' is alien to the language, addressing the interlocutor is a common practice. If you address a monk, \pali{bhante} is a suitable word. If the speaker is also a monk, \pali{bhante} is used for addressing a senior monk, for a junior one \pali{\=avuso} is used instead. We can translate these two word as `Venerable,' a common word used to address Theravada monks. For ordinary people, you may use \pali{bho/bhoti} or the name of that person in vocative case. For more information, see Chapter \ref{chap:vockim}.}

\stepcounter{convnum}\medskip
\arabic{convnum}. \pali{Tuyham'pi Suppabh\=ata\d m.}\\
\hspace*{10mm}Good morning to you, too.\\
{\small $\triangleright$ When terms come together, optionally or practically we can phonetically weld them together (see Appendix \ref{chap:sandhi}). So, you see \pali{tuyhampi} rather than \pali{tuyha\d m pi} here. Particle \pali{pi} here means `too' or `also.' It can be used in other way as well (see page \pageref{nip:pi}). On introduction to particles, see Chapter \ref{chap:ind-intro}.}

\stepcounter{convnum}\medskip
\arabic{convnum}. \pali{J\=an\=asi P\=alibh\=asa\d m?}\\
\hspace*{10mm}Do you know P\=ali?\\
{\small $\triangleright$ It is better to check whether our interlocutor is able to understand what we say. About forming yes-no question, see Chapter \ref{chap:ques}. We can also put \pali{nu} or \pali{nu kho} in this sentence, if it sounds better for you, hence ``\pali{J\=an\=asi nu (kho) P\=alibh\=asa\d m?}'' We normally leave out redundant pronouns in conversation. If you want to stress, however, you can put \pali{tva\d m} here, thus ``\pali{J\=an\=asi tva\d m P\=alibh\=asa\d m?}'' A more fashionable way to do is using a proper addressing word. So, it is preferable to say ``\pali{J\=an\=asi, bho, P\=alibh\=asa\d m?}'' (Do you know P\=ali, sir?). This is true for other following lines as well, but I will not remind you again.}

\stepcounter{convnum}\medskip
\arabic{convnum}. \pali{Thoka\d m j\=an\=ami.}\\
\hspace*{10mm}I know a little.\\
{\small $\triangleright$ This is a simple reply to the previous question. If you know a considerable degree, only \pali{j\=anami} is fine. For a negative reply, we can use \pali{na j\=anam\=i}. In this sentence, \pali{thoka\d m} is used as an adverbial accusative (see Chapter \ref{chap:adv}). If you want to add `yes' to the sentence, start it with \pali{\=ama} (see below).}

\stepcounter{convnum}\medskip
\arabic{convnum}. \pali{Sakkosi P\=alibh\=as\=aya sallapitu\d m?}\\
\hspace*{10mm}Can you speak P\=ali?\\
{\small $\triangleright$ This is a more specific question. To ask about capability, normally we use infinitive (see Chapter \ref{chap:inf}). In the sentence, \pali{P\=alibh\=as\=aya} is in instrumental case (see Chapter \ref{chap:ins}). So, literally it means ``Can you speak with P\=ali?''}

\stepcounter{convnum}\medskip
\arabic{convnum}. \pali{\=Ama, thoka\d m sallapitu\d m sakkomi.}\\
\hspace*{10mm}Yes, I can speak a little.\\
{\small $\triangleright$ This can reply to the question above. For a short positive reply, you can say simply ``\pali{\=Ama, [bho,] sakkomi}'' (Yes, [sir,] I can). Or even just ``\pali{\=Ama, [bho]}.'' For a negative reply, you can say ``\pali{Na [sakkomi]}'' (No, [I cannot]). About particle \pali{\=ama}, see page \pageref{nip:aama}. About \pali{na}, see page \pageref{nip:na}.}

\stepcounter{convnum}\medskip
\arabic{convnum}. \pali{Kinn\=amo'si?}\\
\hspace*{10mm}What is your name?\\
{\small $\triangleright$ This is a common way to ask someone's name. For a female interlocutor, we use ``\pali{Kinn\=am\=a'si?}'' The sentence can be broken down to \pali{ki\d m + n\=amo + asi}. Here \pali{asi} is a second-person form of verb `to be' (see Chapter \ref{chap:verb-be}). If you want to ask for a family name, you may go like this, ``\pali{Ki\d m kulassa n\=aman'si?}'' Now, \pali{n\=ama} turns to be nt. If you want to ask a third-person's name, you can say ``Ki\d m so n\=amo atthi?'' (m.) or ``Ki\d m s\=a n\=am\=a atthi?'' (f.). For more information of name asking, see Chapter \ref{chap:vockim}.}

\stepcounter{convnum}\medskip
\arabic{convnum}. \pali{[Aha\d m] \=Anando n\=ama [amhi].}\\
\hspace*{10mm}I am called \=Ananda.\\
{\small $\triangleright$ This is a simple way to tell your name. It may be better to put \pali{aha\d m} here to prevent a mistake. In fact, just \pali{\=Anando amhi} works fine. Ven.\,Buddhadatta suggests a compound form like ``\pali{Aha\d m \=Anandan\=amo'mhi}'' (\pali{\=Anandan\=amo amhi}). This can be more suitable if you have a foreign name which is difficult or impossible to decline into nominative case. For example, you can put bluntly as ``John-n\=amo'mhi.'' It is a little ugly but understandable. In a casual situation, just \pali{John amhi} can do the job.}

\stepcounter{convnum}\medskip
\arabic{convnum}. \pali{Kattha vasasi?}\\
\hspace*{10mm}Where do you live?\\
{\small $\triangleright$ This is a straightforward question to ask one's current place of living. In this sentence, \pali{kattha} is an indeclinable used in locative sense (see Chapter \ref{chap:ind-to}). You can use \pali{kasmi\d m} or \pali{kamhi} instead. For English speakers, it may be more familiar to ask ``Where are you from?'' This can be rendered as ``\pali{Kuto \=agacchasi?} (Where do you come from?).}

\refstepcounter{convnum}\label{conv:bangkok}\medskip
\arabic{convnum}. \pali{Bangkok-nagare vas\=ami.}\\
\hspace*{10mm}I live in Bangkok.\\
{\small $\triangleright$ Here is the simplest way to tell where you live. We use a hybrid compound with locative case (see Chapter \ref{chap:loc}). By adding \pali{-nagara} to a city's name, you can put any town on earth into P\=ali. Do not try to change the name to P\=ali, or use only the name in loc. It will make things confusing. Make it simple, like \pali{New York-nagare, London-nagare, Beijing-nagare}, or whatever. If you want to refer to a country, use \pali{-desa} (region) or \pali{-ra\d t\d tha} (state), for example, \pali{Thai-dese}\footnote{I saw some use \pali{Dayyadesa} or \pali{Dayyara\d t\d tha} for Thailand, sometimes \pali{Dayyabh\=as\=a} for Thai language. I think it is rather confusing when written in Roman script. So, I avoid this transliteration.} (in Thailand), \pali{America-dese} (in America), \pali{Japan-dese} (in Japan).\footnote{Some countries already have their name in P\=ali, for example, \pali{Jambud\=ipa} (India), \pali{La\.nk\=a} or \pali{Sirila\.nk\=a} (Sri Lanka), \pali{C\=inara\d t\d tha} (China), \pali{Marammara\d t\d tha} (Myanmar), \pali{Sy\=amara\d t\d tha} [\pali{Siy\=amara\d t\d tha}] (Siam), \pali{\=A\.ngalara\d t\d tha} [\pali{\=A\.ngal\=iyara\d t\d tha}] (England), \pali{Kampoja} (Cambodia). Some names can be assimilated into P\=ali seamlessly, for example, \pali{It\=aliy\=a} or \pali{It\=aly\=a} (Italy) [These are suggested by Antonio Costanzo].}}

\stepcounter{convnum}\medskip
\arabic{convnum}. \pali{Bangkok-nagar\=a \=agacch\=ami.}\\
\hspace*{12mm}I come from Bangkok.\\
{\small $\triangleright$ If you are asked ``Where do you come from?,'' use this reply instead. Now the place's name is in ablative case (see Chapter \ref{chap:abl}). Alternatively, you can also use \pali{Bangkok-nagarato} (see Chapter \ref{chap:ind-to}).}

\stepcounter{convnum}\medskip
\arabic{convnum}. \pali{Tuyha\d m \=ayupam\=ana\d m kittaka\d m?}\\
\hspace*{12mm}What is your age?\\
For more detail on P\=ali numerals and \pali{kittaka}, see Chapter \ref{chap:num}.

\stepcounter{convnum}\medskip
\arabic{convnum}. \pali{Mayha\d m \=ayupam\=ana\d m pa\d n\d narasa.}\\
\hspace*{12mm}My age is fifteen.

\stepcounter{convnum}\medskip
\arabic{convnum}. \pali{Kativasso'si [\=ayun\=a]?}\\
\hspace*{12mm}How old are you?\\
{\small $\triangleright$ This is another way to ask for the age, an easier one. To make it clearer, \pali{\=ayun\=a} (by age) may be added. For more information about \pali{kati}, see Chapter \ref{chap:num}.}

\stepcounter{convnum}\medskip
\arabic{convnum}. \pali{V\=isativasso'mhi.}\\
\hspace*{12mm}I am twenty years old.

\stepcounter{convnum}\medskip
\arabic{convnum}. \pali{Tuyha\d m bh\=atubhaginiyo pi santi?}\\
\hspace*{12mm}Do you also have brothers and sisters?\\
{\small $\triangleright$ Making compounds in P\=ali on the fly is a powerful feature of the language. You can avoid dealing with a complex sentence by lumping words together, as you shall see more in due course. To learn more about compounds, see Appendix \ref{chap:samasa}.}

\stepcounter{convnum}\medskip
\arabic{convnum}. \pali{\=Ama, mayha\d m catt\=aro bh\=ataro dve bhaginiyo ca santi.}\\
\hspace*{12mm}Yes, I have four brothers and two sisters.\\
{\small $\triangleright$ To say we have something, in P\=ali we use genitive case with verb `to be' (see Chapter \ref{chap:gen}). Note that \pali{bh\=atu} (brother) declines irregularly like \pali{pitu} (father). When one brother is intended, it will be \pali{eko bh\=at\=a} (see page \pageref{decl:pitu}, see also Chapter \ref{chap:irrn}). For a negative reply, you can use ``\pali{natthi}'' ([No,] I have not).}

\stepcounter{convnum}\medskip
\arabic{convnum}. \pali{Te kuhi\d m vasanti?}\\
\hspace*{12mm}Where do they live?\\
{\small $\triangleright$ To be more precise, instead of using \pali{te} you can specify like ``\pali{Tuyha\d m je\d t\d thabh\=at\=a kuhi\d m vasati?}'' (Where does your elder brother live?), ``\pali{Tuyha\d m m\=at\=apitaro kuhi\d m vasanti?}'' (Where do your parents live?). Our vocabulary contains a number of terms concerning our relationship (see Appendix \ref{chap:vocab}).}

\stepcounter{convnum}\medskip
\arabic{convnum}. \pali{Sabbe te p'id\=ani Bangkok-nagare vasanti.}\\
\hspace*{12mm}Yes, they all also live in Bangkok now.

\stepcounter{convnum}\medskip
\arabic{convnum}. \pali{Tava bh\=ataro ki\d m karonti?}\\
\hspace*{12mm}What do your brothers do?

\stepcounter{convnum}\medskip
\parbox[lt]{0.93\linewidth}{\raggedright\arabic{convnum}. \pali{Tesu eko v\=a\d nijo, dutiyo lekhako, dve t\=ava p\=a\d thas\=al\=asu ugga\d nhanti.}\\
\hspace*{6mm}Among them one is a merchant, the second one is a clerk, and the other two still attend schools.}\\[1mm]
{\small $\triangleright$ For more terms about occupation, see vocabulary (Appendix \ref{chap:vocab}). The use of \pali{t\=ava} as `still' is noteworthy here.}

\stepcounter{convnum}\medskip
\arabic{convnum}. \pali{Ki\d m kamma\d m k\=atu\d m icchasi?}\\
\hspace*{12mm}What do you like to do?\\
{\small $\triangleright$ This can be used to ask for aspiration or future occupation. You can apply this question in various way, for example, ``\pali{Ki\d m bhu\~njitu\d m icchasi?}'' (What do you like to eat?), ``\pali{Kuhi\d m gantu\d m icchasi?}'' (Where do you like to go?), ``\pali{Kad\=a apagantu\d m icchasi?}'' (When do you want to leave?). For more detail about the infinitive, see Chapter \ref{chap:inf}.}

\stepcounter{convnum}\medskip
\arabic{convnum}. \pali{Va\d d\d dhak\=i bhavitu\d m icch\=ami.}\\
\hspace*{12mm}I like to become a carpenter (architect).

\stepcounter{convnum}\medskip
\arabic{convnum}. \pali{Kad\=a idha \=agato'si?}\\
\hspace*{12mm}When did you come here?\\
{\small $\triangleright$ This is a practical way to say things in past tense. We normally use past participles, mostly verbs is \pali{ta} form (see Chapter \ref{chap:pp}), with verb `to be' (\pali{asi} in this instance). You can leave out verb `to be' if everything is still understandable. If you use an aorist verb instead, the sentence will look like ``\pali{Kad\=a idha \=agacchi?}'' In this simple question, both ways are equally easy. But with other verbs in variety of person and number, using past participles may cause you less headache than using the aorist.}

\stepcounter{convnum}\medskip
\arabic{convnum}. \pali{Hiyyo idh'\=agato'mhi.}\\
\hspace*{12mm}I came here yesterday.\\
{\small $\triangleright$ If an equivalent aorist verb is used, it becomes ``\pali{Hiyyo idha \=agacchi\d m}.'' Remember that all P\=ali past forms can be translated to either past or perfect tense. So, this can be equally translated as ``I have come here yesterday.''}

\stepcounter{convnum}\medskip
\arabic{convnum}. \pali{Kismi\d m k\=ale p\=apu\d nito'si?}\\
\hspace*{12mm}In what time have you arrived?\\
{\small $\triangleright$ Alternatively, you can use \pali{vel\=a} (f.) for time. Hence, `in what time' will be \pali{kassa\d m vel\=aya\d m} instead. Yet another way to ask for the time is \pali{katigha\d tik\=a}. For example, to ask ``What time is it now?'' we can put it as ``\pali{id\=ani katigha\d tik\=a hoti?}'' More about \pali{gha\d tik\=a}, see below. In Thai tradition, \pali{n\=a\d lik\=a} or \pali{n\=a\d dik\=a} (f.) can be used instead of \pali{gha\d tik\=a}. So, `in what time' can also be put as \pali{katin\=a\d lik\=aya}.}

\refstepcounter{convnum}\label{conv:time}\medskip
\arabic{convnum}. \pali{Apara\d nhe tiggha\d tike p\=apu\d ni\d m.}\\
\hspace*{12mm}I have arrived at 3 p.m.\\
{\small $\triangleright$ We can tell the time roughly in this way. In fact, 60 \pali{gha\d tik\=a} (f.) equal to 24 hours\footnote{Abhidh\=a 74}, but we do not use this astronomical sense. The term can also be used in m.\ (\pali{gha\d tika}) as seen in the sentence. We use this to mean \emph{o'clock} in modern context. For a.m.\ we use \pali{pubba\d nhe}. To be more precise than this, you have to say it in full form by using \pali{vigha\d tik\=a} for `minute,' for example, \pali{apara\d nhe ti-gha\d tik\=a pa\~ncadasa-vigha\d tik\=a ca} (3:15 p.m.), \pali{pa\~ncadasa-gha\d tik\=a ti\d msa-vigha\d tik\=a ca} (15:30). To say it in a more grammatical way, we use past participle, for example, \pali{apara\d nhe tiggha\d tikato ti\d msavigha\d tik\=atikkanto} (half past three p.m.). Literally, this means ``in the afternoon [when the time] went beyond by 30 minutes from 3 o'clock.'' In P\=ali, there is an idiomatic way to say thing with a half (see Table \ref{tab:half} on page \pageref{tab:half}). Therefore, we can say the time in this way also: \pali{diya\d d\d dha-gha\d tik\=a} (1:30), \pali{a\d d\d dhateyya-gha\d tik\=a} (2:30), \pali{a\d d\d dhu\d d\d dha-gha\d tik\=a} (3:30), \pali{a\d d\d dhapa\~ncama-gha\d tik\=a} (4:30), and so on.}

\stepcounter{convnum}\medskip
\arabic{convnum}. \pali{Ki\d m k\=atu\d m idha \=agato'si?}\\
\hspace*{12mm}For what purpose have you come here?\\
{\small $\triangleright$ You may ask a more general question as ``Why do you come?'' This can be rendered as ``\pali{Kasm\=a \=agato'si?}'' Instead of using abl., you can also use \pali{kena} (ins.) or \pali{kasmi\d m} (loc.) for asking a cause or reason. For more detail, see Chapter \ref{chap:vockim}.}

\stepcounter{convnum}\medskip
\arabic{convnum}. \pali{Bha\d n\d d\=ani vikki\d nitu\d m icch\=ami.}\\
\hspace*{12mm}I want to sell some goods.\\

\stepcounter{convnum}\medskip
\arabic{convnum}. \pali{Atthi nu kho idha tava mitto v\=a \~n\=ati v\=a?}\\
\hspace*{12mm}Is there your friend or relative here?\\

\stepcounter{convnum}\medskip
\arabic{convnum}. \pali{Ko idha tava mitto v\=a \~n\=ati v\=a?}\\
\hspace*{12mm}Who is your friend or relative here?\\

\stepcounter{convnum}\medskip
\arabic{convnum}. \pali{Idha keci bhikkh\=u mayha\d m mitt\=a honti.}\\
\hspace*{12mm}Some monks are friends of mine here.\\
{\small $\triangleright$ For \pali{ki\d m+ci}, see Chapter \ref{chap:pron-misc}. For its declension, see page \pageref{decl:koci}}

\stepcounter{convnum}\medskip
\arabic{convnum}. \pali{Kattha kamma\d m karosi?}\\
\hspace*{12mm}Where do you work?\\

\stepcounter{convnum}\medskip
\arabic{convnum}. \pali{Ekasmi\d m mah\=avijj\=alaye kamma\d m karomi.}\\
\hspace*{12mm}I work in a university.\\

\stepcounter{convnum}\medskip
\arabic{convnum}. \pali{Ito kuhi\d m gamissasi?}\\
\hspace*{12mm}Where will you go from here?\\

\stepcounter{convnum}\medskip
\arabic{convnum}. \pali{Ito a\~n\~na\d m nagara\d m gamiss\=ami.}\\
\hspace*{12mm}I will go to another town from here.\\
{\small $\triangleright$ About using future tense, see Chapter \ref{chap:fut}. We also have a lesson on verb `to go' in Chapter \ref{chap:verb-go}.}

\stepcounter{convnum}\medskip
\arabic{convnum}. \pali{Piy\=ayasi nu kho ida\d m \d th\=ana\d m?}\\
\hspace*{12mm}Do you like this place?\\

\stepcounter{convnum}\medskip
\parbox[lt]{0.93\linewidth}{\raggedright\arabic{convnum}. \pali{Piy\=ayeyya\d m ida\d m \d th\=ana\d m, sace ida\d m na ca u\d nha\d m bhaveyya.}\\
\hspace*{6mm}I may like this place if it would not be too hot.}\\[1mm]
{\small $\triangleright$ For a hypothetical statement, we use optative mood (see Chapter \ref{chap:opt}). Note that it is fashionable to use middle voice form (\pali{piy\=ayeyya\d m}) in first person. However, \pali{piy\=ayeyy\=ami} can do the job as well (please check the conjugation table in Appendix \ref{chap:conj}). More about conditionals, see Chapter \ref{chap:cond}.}

\stepcounter{convnum}\medskip
\arabic{convnum}. \pali{Kad\=a saka\d t\d th\=ana\d m gamissasi?}\\
\hspace*{12mm}When will you go home?\\
{\small $\triangleright$ Here \pali{saka\d t\d th\=ana\d m} (\pali{saka + \d t\d th\=ana}) literally means `one's own place.' More about reflexive pronouns, see Chapter \ref{chap:pron-person}.}

\stepcounter{convnum}\medskip
\arabic{convnum}. \pali{Yad\=a paho\d naka\d m m\=ula\d m labhiss\=ami, tad\=a gamiss\=ami.}\\
\hspace*{12mm}I will go when I get enough money.\\
{\small $\triangleright$ To form a complex sentence like this one, using \pali{ya-ta} structure is very common in P\=ali (see Chapter \ref{chap:yata}). Going word by word, we can translate this sentence as ``Which time I will get enough money, that time I will go.''}

\stepcounter{convnum}\medskip
\arabic{convnum}. \pali{Tay\=a sam\=agato bhadda\d m me atthi.}\\
\hspace*{12mm}I am lucky to meet you.\\
{\small $\triangleright$ This is a way to say ``It is good to see you.'' Literally, the sentence means ``Having met with you, I have luck.'' Note that \pali{sam\=agacchati} (to meet) is used with an instrumental object.}

\stepcounter{convnum}\medskip
\arabic{convnum}. \pali{Amh\=aka\d m samosara\d na\d m subha\d m hoti.}\\
\hspace*{12mm}Our meeting is auspicious.\\
{\small $\triangleright$ This is another way to say ``It is nice to meet you.''}

\stepcounter{convnum}\medskip
\arabic{convnum}. \pali{Kara\d n\=iyakicca\d m me atthi. Puna tay\=a sam\=agamana\d m patthemi.}\\
\hspace*{12mm}I have a thing to do. I hope for meeting you again.\\
{\small $\triangleright$ In the first part you can say just ``\pali{Kicca\d m me atthi}'' or ``\pali{Kara\d n\=iya\d m me atthi}.'' You can be more specific on this. For example, you can say ``\pali{Kassaci lekhanakicca\d m me atthi}.'' (I have to do some writing), ``\pali{Vaccaku\d tiy\=a gamanakicca\d m me atthi}'' (I have to go to toilet), ``\pali{Mama adhipatin\=a sam\=agamanakicca\d m me atthi}'' (I have to meet my boss). And here is a simple way to say ``It is the time I have to go'': ``\pali{Mama gamanak\=alo upaka\d t\d tho}'' (My going time is coming). In the second part, you can also use infinitive, hence, ``\pali{Puna tay\=a sam\=agantu\d m patthemi}'' (I hope to meet to you again). A simpler way to say this is ``\pali{Tava pacch\=a dassana\d m icch\=ami}'' (I want seeing you afterwards).}

\stepcounter{convnum}\medskip
\arabic{convnum}. \pali{Sotthi te hotu.}\\
\hspace*{12mm}Goodbye!\\
{\small $\triangleright$ This is a way to say goodbye. It means ``May blessing happen to you.'' Practically, only ``\pali{Sotthi!}'' works fine. In fact, \pali{sotthi} is transliterated to Thai as `sawaddee' (\pali{svasti}) which is used for greeting as well as parting.\footnote{Thai does not have phrases comparable to `good morning' or `goodbye' in English.} This means you can also use \pali{sotthi} when you meet someone, particularly the one who is not a Westerner.}

\stepcounter{convnum}\medskip
\arabic{convnum}. \pali{Subha\d m bhavatu.}\\
\hspace*{12mm}Goodbye!\\
{\small $\triangleright$ This is another way to say goodbye. The meaning is more or less the same as the previous one. You can also use other words that their meaning fits the situation, for example, ``\pali{Kaly\=a\d nak\=alo hotu}'' (Have a good time!), ``\pali{Sukhito/sukhit\=a hotu}'' (May you be happy!).}

\phantomsection
\addcontentsline{toc}{section}{With a Schoolboy}
\section*{With a Schoolboy}

Conversation in this section also comes from the \emph{Aids}\footnote{\citealp[pp.~51--2]{buddhadatta:aids}}, but only the first half of its part. The sentences are slightly modified to make them in line with our lessons.

\stepcounter{convnum}\medskip
\arabic{convnum}. \pali{Kasm\=a tva\d m hiyyo n'\=agato'si?}\\
\hspace*{12mm}Why did you not come yesterday?\\

\stepcounter{convnum}\medskip
\arabic{convnum}. \pali{Hiyyo pitar\=a saddhi\d m mataka\d t\d th\=ana\d m agami\d m.}\\
\hspace*{12mm}Yesterday I went to a funeral with [my] father.\\
{\small $\triangleright$ Normally, particle \pali{saddhi\d m} is used with instrumental case (see Chapter \ref{chap:ins}). You can equally use \pali{saha} instead.}

\stepcounter{convnum}\medskip
\arabic{convnum}. \pali{Kuhi\d m tava potthak\=a lekhanabha\d n\d d\=ani ca?}\\
\hspace*{12mm}Where are your books and writing materials?\\

\stepcounter{convnum}\medskip
\arabic{convnum}. \pali{T\=ani p\=a\d thas\=al\=aya lekhanaphalake \d thapetv\=a \=agato'mhi.}\\
\hspace*{12mm}Having left them on the desk at the school, I came [here].\\
{\small $\triangleright$ This is an example of how to use the absolutive, verbs in \pali{tv\=a} form (see Chapter \ref{chap:pp}).}

\stepcounter{convnum}\medskip
\arabic{convnum}. \pali{Kasm\=a tva\d m ajja cir\=ayitv\=a \=agacchasi?}\\
\hspace*{12mm}Why do you come late today?\\
{\small $\triangleright$ Note that \pali{cir\=ayitv\=a} here works much like an adverb. The term is made from a noun (\pali{cira}). To learn more about denominative verbs, see Chapter \ref{chap:vform}, page \pageref{sec:denomverbs}.}

\stepcounter{convnum}\medskip
\parbox[lt]{0.93\linewidth}{\raggedright\arabic{convnum}. \pali{Antar\=amagge setu\d m bhinnatta\d m taritu\d m asakkonto cir\=ayi\d m.}\\
\hspace*{6mm}On the way, being unable to cross a broken bridge, I delayed.}\\[1mm]
{\small $\triangleright$ Here, we use present participle (\pali{asakkonto}) instead of forming a conditional statement. More on present participles, see Chapter \ref{chap:prp}. The main verb (\pali{cir\=ayi\d m}) is in aorist, first person. More information on past tense, see Chapter \ref{chap:past}. You can alternatively use past participle with verb `to be,' thus ``\pali{cir\=ayito'mhi}.''}

\stepcounter{convnum}\medskip
\arabic{convnum}. \pali{So d\=arako tuyha\d m ki\d m kathesi?}\\
\hspace*{12mm}What did that boy say to you?\\

\refstepcounter{convnum}\label{conv:nimmuu}\medskip
\parbox[lt]{0.93\linewidth}{\raggedright\arabic{convnum}. \pali{Nimm\=ulatt\=a icchite potthake ki\d nitu\d m n\=asakkhin'ti so vadi.}\\
\hspace*{6mm}``Because of having no money, I could not buy necessary books,'' he said.}\\[1mm]
{\small $\triangleright$ We can see direct speech in use here (see more in Chapter \ref{chap:iti}). In the quote, \pali{n\=asakkhi\d m} is negative aorist, first person. An interesting word here is \pali{nimm\=ulatta} (\pali{ni + m\=ula + tta}). It is formed as a secondary derivative with \pali{tta} ending (see Appendix \ref{chap:taddhita}, page \pageref{pacct8:tta}). It denotes a state of being. The prefix \pali{ni} means `free from' (see Appendix \ref{chap:upasagga}, page \pageref{upasagga:ni}). As a unit, \pali{nimm\=ulatta} means `state of having no money.' It is used in ablative case to mark a cause. Alternatively, you can form the word as \pali{Abyay\=ibh\=ava} compounds (see page \pageref{sec:abyayi}), hence \pali{nimm\=ula} (adj). Then you can use this like \pali{nimm\=ulabh\=av\=a} (from state of having no money). Or just \pali{nimm\=ul\=a} can do the job, but a bit vague.}

\stepcounter{convnum}\medskip
\arabic{convnum}. \pali{Kasm\=a so tva\d m pakkosi?}\\
\hspace*{12mm}Why did he send for you?\\

\stepcounter{convnum}\medskip
\parbox[lt]{0.93\linewidth}{\raggedright\arabic{convnum}. \pali{Mama santik\=a eka\d m potthaka\d m laddhu\d m icchanto so ma\d m pakkosi.}\\
\hspace*{6mm}Wishing to get a book from me, he has sent for me.}\\[1mm]
{\small $\triangleright$ Present participles and infinitives can be used together in this way. A conditional clause is not needed here. Normally, \pali{santika} means `vicinity' or `presence.' It is a handy word to denote one's place or possession. With abl.\ in this sentence, it can mean `from my place' or `from my possession' or `from my attendance.'}

\stepcounter{convnum}\medskip
\arabic{convnum}. \pali{Kati potthak\=a k\=it\=a tay\=a?}\\
\hspace*{12mm}How many books have been bought by you?\\
{\small $\triangleright$ Practically, you can translate this into active voice as ``How many books did you buy?'' For more detail on passive voice, see Chapter \ref{chap:pass}.}

\stepcounter{convnum}\medskip
\arabic{convnum}. \pali{May\=a catt\=aro potthak\=a k\=it\=a.}\\
\hspace*{12mm}Four books have been bought by me.\\

\stepcounter{convnum}\medskip
\arabic{convnum}. \pali{Tesa\d m atth\=aya kittaka\d m m\=ula\d m dinna\d m tay\=a?}\\
\hspace*{12mm}How much money was paid by you for them?\\
{\small $\triangleright$ Like a filler, \pali{atth\=aya} more or less means `for the sake of.'\footnote{\citealp[p.~68]{warder:intro}}}

\stepcounter{convnum}\medskip
\arabic{convnum}. \pali{A\d t\d tha r\=upiy\=ani pa\~ncav\=isati-satabh\=age ca aha\d m ad\=asi\d m.}\\
\hspace*{12mm}I gave eight rupees and twenty-five cents.\\

\phantomsection
\addcontentsline{toc}{section}{Between Two Farmers}
\section*{Between Two Farmers}

I took the whole section of this dialogue from the \emph{Aids}.\footnote{\citealp[pp.~54--6]{buddhadatta:aids}} This contains useful ideas and interesting sentence-forming technique. The sentences are left untouched, so you will see personal pronouns in use here.

\stepcounter{convnum}\medskip
\arabic{convnum}. \pali{Suppabh\=ata\d m!}\\
\hspace*{12mm}Good morning!\\

\stepcounter{convnum}\medskip
\arabic{convnum}. \pali{Sundara\d m tay\=a kata\d m idh\=agacchantena.}\\
\hspace*{12mm}It is good of you to have come here.\\
{\small $\triangleright$ A more literal translation of this can go like this: ``Being done by you who is coming here is good.'' On impersonal passive structure, see Chapter \ref{chap:pass}. Note that \pali{gacchantena} here is a present participle working like a noun or an adjective (a modifier of \pali{tay\=a}). The term declines irregularly, see page \pageref{decl:gacchanta}.}

\stepcounter{convnum}\medskip
\parbox[lt]{0.93\linewidth}{\raggedright\arabic{convnum}. \pali{Aha\d m tay\=a sam\=agantu\d m icchanto tav'\=agamana\d m pacc\=asi\d msanto vasi\d m.}\\
\hspace*{6mm}I hoped that you would come as I was anxious to meet you.}\\[1mm]
{\small $\triangleright$ Here is a literal translation: ``I lived, hoping for your coming, wishing to meet with you.''}

\stepcounter{convnum}\medskip
\parbox[lt]{0.93\linewidth}{\raggedright\arabic{convnum}. \pali{Kasm\=a tva\d m cir\=aya idha n'\=agato'si?}\\
\hspace*{6mm}Why did you not come here for a long time?}\\

\stepcounter{convnum}\medskip
\parbox[lt]{0.93\linewidth}{\raggedright\arabic{convnum}. \pali{Gela\~n\~nen'\=abhibh\=uto'ha\d m ekam\=asamatta\d m katthaci pi gantu\d m n\=asakkhi\d m.}\\
\hspace*{6mm}I could not go anywhere for about a month as I was ill.}\\[1mm]
{\small $\triangleright$ My word-by-word translation will go like this: ``Having been overpowered by illness for about a month, I was not able to go even to anywhere.'' About indefinite interrogative particle \pali{ci}, see page \pageref{nip:ci}, and see some uses of it in Chapter \ref{chap:pron-misc}.}

\stepcounter{convnum}\medskip
\parbox[lt]{0.93\linewidth}{\raggedright\arabic{convnum}. \pali{Tava sassa\d m nipphana\d m v\=a no v\=a?}\\
\hspace*{6mm}Was your harvest fruitful?}\\[1mm]
{\small $\triangleright$ To be more accurate, `or not' can be added to the question. About negative particle \pali{no}, see page \pageref{nip:no}.}

\stepcounter{convnum}\medskip
\parbox[lt]{0.93\linewidth}{\raggedright\arabic{convnum}. \pali{Adhikajalena mama sassa\d m vinassi; thoka\d m eva avasi\d t\d tha\d m ahosi.}\\
\hspace*{6mm}My crop was destroyed by an excess of water; only a small quantity is left unharmed.}\\

\stepcounter{convnum}\medskip
\parbox[lt]{0.93\linewidth}{\raggedright\arabic{convnum}. \pali{K\=a bhavato sasse pavatti?}\\
\hspace*{6mm}What about your own crop?}\\[1mm]
{\small $\triangleright$ Do not be confused \pali{pavatti} (f.\ noun = happening) with \pali{pavattati} (v.\ = to move on, to exist). Here \pali{k\=a} is a modifier of \pali{pavatti}, thus f. \pali{Bhavato} is in genitive case (see page \pageref{decl:bhavanta}). This term is a polite way to say `you.' And \pali{sasse} is in loc. You may add verb `to be' like \pali{hoti} in the sentence to make it clearer. Thus, precisely this sentence means ``What is the happening in your crop?''}

\stepcounter{convnum}\medskip
\parbox[lt]{0.93\linewidth}{\raggedright\arabic{convnum}. \pali{Pa\d thama\d m g\=avo vati\d m bhinditv\=a taru\d nasassa\d m kh\=adi\d msu, ath\=avasi\d t\d tha\d m anodakena mil\=ayi.}\\
\hspace*{6mm}At first, some cattle broke the fence and ate the young plants, and then the remainder died of drought.}\\[1mm]
{\small $\triangleright$ I translate the sentence in this way: ``First, having broken the fence, some cattle ate the young plants, then the remainder withered by having no water.''}

\stepcounter{convnum}\medskip
\parbox[lt]{0.93\linewidth}{\raggedright\arabic{convnum}. \pali{Yajjeva\d m, katha\d m tva\d m attano ku\d tumba\d m posetu\d m sakkosi?}\\
\hspace*{6mm}If it is so, how will you feed your family?}\\[1mm]
{\small $\triangleright$ The joined unit of \pali{yajjeva\d m} comes from \pali{yadi + eva\d m}.\footnote{Sadd\,104, R\=upa\,41, Mogg\,1.48, Niru\,44} More about \pali{atta} as a pronoun, see Chapter \ref{chap:pron-person}.}

\stepcounter{convnum}\medskip
\parbox[lt]{0.93\linewidth}{\raggedright\arabic{convnum}. \pali{Aha\d m s\=akapa\d n\d n\=ani bha\d n\d d\=ak\=i-kumbha\d n\d d\=ad\=ini ca vikki\d nitv\=a j\=ivika\d m kappess\=ami.}\\
\hspace*{6mm}I will earn my livelihood by selling pot-herbs, brinjals, pumpkins, etc.}\\[1mm]
{\small $\triangleright$ Being used as an idiom, \pali{j\=ivika\d m kappeti} generally means `to make a living.' This sentence also shows how to use \pali{\=adi} for introducing some samples of things. It normally appears in compounds like this one, \pali{bha\d n\d d\=ak\=ikumbha\d n\d d\=adi} (\pali{bha\d n\d d\=ak\=i + kumbha\d n\d da + \=adi}). The whole unit ends up as nt., thus \pali{\=ini} as acc.\ pl. This means ``brinjals, pumpkins, and so on.''}

\stepcounter{convnum}\medskip
\parbox[lt]{0.93\linewidth}{\raggedright\arabic{convnum}. \pali{Santi tav'uyy\=ane bah\=u jamb\=irarukkh\=a?}\\
\hspace*{6mm}Are there many orange trees in your garden?}\\[1mm]

\stepcounter{convnum}\medskip
\parbox[lt]{0.93\linewidth}{\raggedright\arabic{convnum}. \pali{V\=isati rukkh\=a mam'uyy\=ane ropit\=a honti.}\\
\hspace*{6mm}There are twenty trees in my garden.}\\[1mm]

\stepcounter{convnum}\medskip
\parbox[lt]{0.93\linewidth}{\raggedright\arabic{convnum}. \pali{Ekasmi\d m v\=are tehi kittak\=ani phal\=ani ocin\=asi?}\\
\hspace*{6mm}How many fruits do you gather from those trees in one crop?}\\[1mm]

\stepcounter{convnum}\medskip
\parbox[lt]{0.93\linewidth}{\raggedright\arabic{convnum}. \pali{Ekasmi\d m phalav\=are dvisahassamatt\=ani phal\=ani labh\=ami.}\\
\hspace*{6mm}I get about 2,000 fruits in one crop.}\\[1mm]

\stepcounter{convnum}\medskip
\parbox[lt]{0.93\linewidth}{\raggedright\arabic{convnum}. \pali{Kad\=a tava khetta\d m kasitu\d m icchasi?}\\
\hspace*{6mm}When do you wish to plough your field?}\\[1mm]

\stepcounter{convnum}\medskip
\parbox[lt]{0.93\linewidth}{\raggedright\arabic{convnum}. \pali{Yad\=a go\d ne ca na\.ngal\=ani ca labhiss\=ami tad\=a'ha\d m kasiss\=ami.}\\
\hspace*{6mm}I will plough it when I get oxen and ploughs.}\\[1mm]
{\small $\triangleright$ My translation will go like this: ``Which time I get oxen and ploughs, that time I will plough it.''}

\stepcounter{convnum}\medskip
\parbox[lt]{0.93\linewidth}{\raggedright\arabic{convnum}. \pali{L\=ayane kittak\=a l\=ayak\=a icchitabb\=a?}\\
\hspace*{6mm}How many mowers do you need in reaping?}\\[1mm]
{\small $\triangleright$ From root \pali{l\=a}, \pali{l\=ayana} is a primary derivative by applying \pali{yu} or \pali{ana} to the root (see Appendix \ref{chap:kita}, page \pageref{pacck4:yu}). This is an action noun meaning `reaping.' Also a primary derivative, \pali{l\=ayaka} is a product of \pali{ka} over the same root (see page \pageref{pacckx:ka2}). This means `reaper.' Using future passive participle, a verb in \pali{tabba} form, is noteworthy here (see more in Chapter \ref{chap:pass}). To be precise, this question can be translated as ``How many mowers should be needed in reaping?''}

\stepcounter{convnum}\medskip
\parbox[lt]{0.93\linewidth}{\raggedright\arabic{convnum}. \pali{Dasa l\=ayak\=a dasahi d\=attehi mama sassa\d m l\=ayitu\d m sakkhissanti.}\\
\hspace*{6mm}Ten reapers with ten scythes will be able to reap my harvest.}\\[1mm]

\stepcounter{convnum}\medskip
\parbox[lt]{0.93\linewidth}{\raggedright\arabic{convnum}. \pali{Khale r\=as\=ikatv\=a kat\=ihi go\d nehi madd\=apessasi?}\\
\hspace*{6mm}Having heaped them on the threshing floor, how many oxen do you need for threshing?}\\[1mm]
{\small $\triangleright$ Here we see a causative verb in use. From root \pali{madda}, the normal active form of this verb is \pali{maddati} (to crush). To make this causative, we add \pali{\d n\=ape} to it, hence we get \pali{madd\=apeti} (to have someone crush something). For more detail about causative structure, see Chapter \ref{chap:caus}. Normally, a causative verb needs two objects, one is object of the action, another is object of the order. So, we should see two accusatives here. The object of verb `to thresh' is \pali{sassa\d m} which is left out. And the object of order is somebody unmentioned, not the oxen because the term takes instrumental case, \pali{go\d nehi}. A more precise translation of this sentence can be ``Having heaped them on the threshing floor, how many oxen do you need to have [someone] thresh [the harvest] by them?''}

\stepcounter{convnum}\medskip
\parbox[lt]{0.93\linewidth}{\raggedright\arabic{convnum}. \pali{A\d t\d thahi go\d nehi madd\=apetv\=a pal\=ala\d m uddharitv\=a bhusa\d m pappho\d tetv\=a sukkh\=apetv\=a ca dha\~n\~na\d m geha\d m \=aness\=ami.}\\
\hspace*{6mm}Having got them threshed by eight oxen and having removed straw and chaff, I will bring home the grain after getting it dried.}\\[1mm]
{\small $\triangleright$ Using the absolutive, verbs in \pali{tv\=a} form, gives us a picture of the process in sequence (see Chapter \ref{chap:pp}). This is a typical use of this verb form.}

\phantomsection
\addcontentsline{toc}{section}{Between Two Merchants}
\section*{Between Two Merchants}

This dialogue is also taken verbatim from the \emph{Aids}.\footnote{\citealp[pp.~56--8]{buddhadatta:aids}} You can find several useful ideas here.

\stepcounter{convnum}\medskip
\parbox[lt]{0.93\linewidth}{\raggedright\arabic{convnum}. \pali{Sv\=agata\d m bhavato! Nis\=id\=ah'imasmi\d m \=asane.}\\
\hspace*{6mm}Welcome (to you)! Please sit down here.}\\[1mm]
{\small $\triangleright$ Using \pali{sv\=agata\d m} (\pali{su + \=agata\d m}) as `welcome' is sensible here. An imperative verb (\pali{nis\=id\=ahi}) is used to make a suggestion, but also a command and request (see more in Chapter \ref{chap:imp}).}

\stepcounter{convnum}\medskip
\parbox[lt]{0.93\linewidth}{\raggedright\arabic{convnum}. \pali{Katha\d m tava sar\=irappavatti?}\\
\hspace*{6mm}How are you getting on?}\\[1mm]
{\small $\triangleright$ Literally, this means ``How is the happening of your body?'' There are some other ways to ask ``How are you?,'' for example, ``K\=idisa\d m tuyha\d m ph\=asubh\=ava\d m?''\footnote{For \pali{k\=idisa}, see page \pageref{pacck1:kvi}.} (How about your happy state?), ``Katha\d m tava ph\=asuvih\=aro?'' (How is your happy living?), ``Katha\d m tava sukhadukkha\d m?'' (How is your happiness-unhappiness?), or bluntly ``\pali{K\=idisa\d m tava j\=ivita\d m?}'' (How about your life?).}

\stepcounter{convnum}\medskip
\parbox[lt]{0.93\linewidth}{\raggedright\arabic{convnum}. \pali{Thuti atthu; aha\d m accantanirog\=i vihar\=ami.}\\
\hspace*{6mm}Thank you; I am quite well.}\\[1mm]
{\small $\triangleright$ The idiomatic use of \pali{thuti atthu} as `thank you' is worth remembering. I find that some use \pali{thomayati} to say `thank you.' So, ``[I] thank you'' is ``\pali{Thomay\=ami}.'' A quick word for `thank you' that can be used widely in a variety of contexts is \pali{s\=adhu} (see page \pageref{nip:saadhu}). The second part can be translated as ``I live as an absolutely disease-free person.'' Another way to say ``I am fine'' is \pali{Natthi mayha\d m ki\~nci aph\=asubh\=ava\d m} (I have no any unhappy state), or shortly ``\pali{sukha\d m vas\=ami}'' (I live happily).}

\stepcounter{convnum}\medskip
\parbox[lt]{0.93\linewidth}{\raggedright\arabic{convnum}. \pali{Tava putta-d\=ar\=a pi nirog\=a sukhino?}\\
\hspace*{6mm}Are your wife and children well and happy?}\\[1mm]
{\small $\triangleright$ With \pali{pi} in this sentence, `also' should be added to the translation.}

\stepcounter{convnum}\medskip
\parbox[lt]{0.93\linewidth}{\raggedright\arabic{convnum}. \pali{Eva\d m, te pi app\=ab\=adh\=a c'eva santu\d t\d th\=a ca.}\\
\hspace*{6mm}Yes, they too are in good health and contented.}\\[1mm]

\stepcounter{convnum}\medskip
\parbox[lt]{0.93\linewidth}{\raggedright\arabic{convnum}. \pali{Imasmi\d m m\=ase bha\d n\d davikkatena kittako l\=abho laddho bhavat\=a?}\\
\hspace*{6mm}How much did you gain this month by selling your goods?}\\[1mm]
{\small $\triangleright$ Past participle (\pali{laddho}) used here is in passive voice. Precisely, this means ``In this month, by selling the goods, how much was the gain obtained by you?'' As you may see, \pali{bhavat\=a} is a substitute of `you' in instrumental case (see page \pageref{decl:bhavanta}).}

\stepcounter{convnum}\medskip
\parbox[lt]{0.93\linewidth}{\raggedright\arabic{convnum}. \pali{K\=itam\=ulato pi \=unam\=ulena vikki\d nitatt\=a mayha\d m h\=ani yeva ahosi na va\d d\d dhi.}\\
\hspace*{6mm}There was no gain but only loss as I had to sell many goods at less than the cost price.}\\[1mm]
{\small $\triangleright$ As an indeclinable, \pali{k\=itam\=ulato} (k\=ita + m\=ula + to) has ablative meaning (see Chapter \ref{chap:ind-to}). Together with \pali{\=unam\=ulena}, these two units mean ``by less value than the cost price.'' With emphatic particle \pali{pi}, we can add `even' to the meaning. As a secondary derivative, \pali{vikki\d nitatt\=a} (vikki\d nita + tta) is in ablative case meaning ``from the state of having sold'' (see page \pageref{pacct8:tta}). This term marks the cause of the whole sentence. Another emphatic particle \pali{yeva} means `only' or 'just' in this context. For more understanding, I retranslate this sentence as ``Because of state of having sold [goods] even by less value than the cost price, there was just loss of mine, no gain.''}

\stepcounter{convnum}\medskip
\parbox[lt]{0.93\linewidth}{\raggedright\arabic{convnum}. \pali{P\=arasika-desato k\=ani bha\d n\d d\=ani tay\=a k\=it\=ani?}\\
\hspace*{6mm}What goods did you buy from Persia?}\\[1mm]
{\small $\triangleright$ For foreign countries' name, see Sentence No.\,\ref{conv:bangkok} above. Persia here may refer to Iran today, or maybe roughly the Middle East. This sentence is in fact in passive voice using past participle. So, we can precisely translate it as ``What goods were bought by you from Persia?''}

\stepcounter{convnum}\medskip
\parbox[lt]{0.93\linewidth}{\raggedright\arabic{convnum}. \pali{Aha\d m satthena tattha gantv\=a bah\=uni anagghakojav\=ani o\d t\d thesu \=aropetv\=a \=anesi\d m.}\\
\hspace*{6mm}I went there with a caravan and brought many carpets with the aid of camels}\\[1mm]
{\small $\triangleright$ I will translate the sentence in this way: ``Having gone there by a caravan, I brought many priceless carpets, having put them on camels.''}

\stepcounter{convnum}\medskip
\parbox[lt]{0.93\linewidth}{\raggedright\arabic{convnum}. \pali{Eka\d m kojava\d m kittakena m\=ulena vikki\d nitu\d m icchasi?}\\
\hspace*{6mm}At what price do you want to sell your carpets?}\\[1mm]
{\small $\triangleright$ Precisely, this means ``By what price do you want to sell a carpet?''}

\stepcounter{convnum}\medskip
\parbox[lt]{0.93\linewidth}{\raggedright\arabic{convnum}. \pali{K\=itam\=ulato digu\d nena m\=ulena vikki\d niss\=ami.}\\
\hspace*{6mm}I will sell them at double the cost price.}\\[1mm]

\stepcounter{convnum}\medskip
\parbox[lt]{0.93\linewidth}{\raggedright\arabic{convnum}. \pali{Ek\=a v\=anijan\=av\=a h\=iyo pa\d t\d tana\d m \=agat\=a ti suta\d m may\=a.}\\
\hspace*{6mm}I have heard that a merchant-vessel arrived in the harbour yesterday.}\\[1mm]
{\small $\triangleright$ This sentence is direct speech marked by \pali{ti}, the enclitic form of \pali{iti}. The structure is passive. So, we can also put it in this way: ``It is heard by me thus, `One merchant-vessel has come to the port yesterday.'\,''}

\stepcounter{convnum}\medskip
\parbox[lt]{0.93\linewidth}{\raggedright\arabic{convnum}. \pali{Eva\d m, aha\d m n\=av\=atittha\d m gantv\=a tato bha\d n\d da\d m gahetu\d m saccak\=ara\d m ad\=asi\d m.}\\
\hspace*{6mm}Yes, I went to the harbour and gave some money in advance to buy goods from there.}\\[1mm]

\stepcounter{convnum}\medskip
\parbox[lt]{0.93\linewidth}{\raggedright\arabic{convnum}. \pali{Suve aha\d m dasahi saka\d tehi t\=ani bha\d n\d d\=ani mama \=apa\d na\d m \=ahar\=apess\=ami.}\\
\hspace*{6mm}Tomorrow I will have them brought to my shop in ten carts.}\\[1mm]
{\small $\triangleright$ This sentence has a causative verb with the object of order (them) left out.}

\stepcounter{convnum}\medskip
\parbox[lt]{0.93\linewidth}{\raggedright\arabic{convnum}. \pali{Aha\d m sabba\d m bha\d n\d dar\=asi\d m ki\d nitu\d m icch\=ami.}\\
\hspace*{6mm}I am inclined to buy the whole lot.}\\[1mm]

\stepcounter{convnum}\medskip
\parbox[lt]{0.93\linewidth}{\raggedright\arabic{convnum}. \pali{Sata\d m ambaphal\=ani ekena r\=upiyena ketu\d m sakk\=a.}\\
\hspace*{6mm}A hundred mangoes could be had for a rupee.}\\[1mm]
{\small $\triangleright$ Now here \pali{sakk\=a} is used as an indeclinable (see page \pageref{nip:sakkaa}, also see Chapter \ref{chap:inf}).}

\stepcounter{convnum}\medskip
\parbox[lt]{0.93\linewidth}{\raggedright\arabic{convnum}. \pali{Etassa kambalass'atth\=aya kittaka\d m tay\=a dinna\d m?}\\
\hspace*{6mm}How much did you pay for this blanket?}\\[1mm]
{\small $\triangleright$ As passive voice, albeit a little awkwardly, you can also translate this as ``How much payment was done by you for [the sake of] this blanket?''}

\stepcounter{convnum}\medskip
\parbox[lt]{0.93\linewidth}{\raggedright\arabic{convnum}. \pali{Aha\d m dasa r\=upiy\=ani pa\d n\d n\=asa-satabh\=age ca ad\=asi\d m.}\\
\hspace*{6mm}I gave ten rupees and fifty cents.}\\[1mm]

\phantomsection
\addcontentsline{toc}{section}{With a Person from Burma}
\section*{With a Person from Burma}

This interesting dialogue is also taken from the \emph{Aids}.\footnote{\citealp[pp.~79--82]{buddhadatta:aids}} I retain `Burma' used here, but you can replace it with `Myanmar.' In the dialogue, this island means Sri Lanka.

\stepcounter{convnum}\medskip
\parbox[lt]{0.93\linewidth}{\raggedright\arabic{convnum}. \pali{Tva\d m katara-ra\d t\d thav\=asiko'si?}\\
\hspace*{6mm}What is your native country?}\\[1mm]

\stepcounter{convnum}\medskip
\parbox[lt]{0.93\linewidth}{\raggedright\arabic{convnum}. \pali{K\=a tuyha\d m j\=atabh\=umi?}\\
\hspace*{6mm}What is your birth place?}\\[1mm]

\stepcounter{convnum}\medskip
\parbox[lt]{0.93\linewidth}{\raggedright\arabic{convnum}. \pali{Aha\d m Marammara\d t\d thiko'mhi.}\\
\hspace*{6mm}I am a native of Burma.}\\[1mm]
{\small $\triangleright$ For other country, you can make a suitable compound likewise, for example, \pali{America-ra\d t\d thiko} (a male American), \pali{Thai-ra\d t\d thik\=a} (a female Thai), \pali{Japan-ra\d t\d thiko} (a male Japanese), \pali{C\=inara\d t\d thik\=a} (a female Chinese).}

\stepcounter{convnum}\medskip
\parbox[lt]{0.93\linewidth}{\raggedright\arabic{convnum}. \pali{Aha\d m Marammaj\=atiko'mhi.}\\
\hspace*{6mm}I am a Burman.}\\[1mm]
{\small $\triangleright$ Like in the previous sentence, you can form a compound to denote other nationality. Adding \pali{j\=atika} to the word emphasizes that you are born in that country.}

\stepcounter{convnum}\medskip
\parbox[lt]{0.93\linewidth}{\raggedright\arabic{convnum}. \pali{Tva\d m kad\=a sakara\d t\d thato nikkhanto'si?}\\
\hspace*{6mm}When did you start off from your country?}\\[1mm]

\stepcounter{convnum}\medskip
\parbox[lt]{0.93\linewidth}{\raggedright\arabic{convnum}. \pali{Gatam\=asassa ek\=adasame tato'ha\d m nikkhanto.}\\
\hspace*{6mm}I started from there on the 11th of last month.}\\[1mm]
{\small $\triangleright$ Literally, \pali{gatam\=asa} means `month which has gone,' hence the previous month.}

\stepcounter{convnum}\medskip
\parbox[lt]{0.93\linewidth}{\raggedright\arabic{convnum}. \pali{N\=av\=aya\d m kati-divase v\=itin\=amesi?}\\
\hspace*{6mm}How many days did you spend on board ship?}\\[1mm]

\stepcounter{convnum}\medskip
\parbox[lt]{0.93\linewidth}{\raggedright\arabic{convnum}. \pali{Samudde catt\=ari divas\=ani v\=itin\=amesi\d m.}\\
\hspace*{6mm}I spent four days on the sea.}\\[1mm]

\stepcounter{convnum}\medskip
\parbox[lt]{0.93\linewidth}{\raggedright\arabic{convnum}. \pali{Samuddo upasanto ahosi v\=a no v\=a?}\\
\hspace*{6mm}Was the sea calm or not?}\\[1mm]

\stepcounter{convnum}\medskip
\parbox[lt]{0.93\linewidth}{\raggedright\arabic{convnum}. \pali{Ekad\=a upasanto ahosi, kad\=aci sa\.nkhubhito.}\\
\hspace*{6mm}Sometimes it was calm and sometimes rough.}\\[1mm]

\stepcounter{convnum}\medskip
\parbox[lt]{0.93\linewidth}{\raggedright\arabic{convnum}. \pali{Santi tay\=a saha \=agat\=a a\~n\~ne pi?}\\
\hspace*{6mm}Have others come with you too?}\\[1mm]

\stepcounter{convnum}\medskip
\parbox[lt]{0.93\linewidth}{\raggedright\arabic{convnum}. \pali{A\~n\~ne dve puris\=a eko ca bhikkhu may\=a saddhi\d m \=agat\=a.}\\
\hspace*{6mm}Two others and a Buddhist monk have come with me.}\\[1mm]

\stepcounter{convnum}\medskip
\parbox[lt]{0.93\linewidth}{\raggedright\arabic{convnum}. \pali{Kimatth\=aya tumhe ima\d m d\=ipa\d m \=agat'attha?}\\
\hspace*{6mm}For what purpose did you come to this island?}\\[1mm]

\stepcounter{convnum}\medskip
\parbox[lt]{0.93\linewidth}{\raggedright\arabic{convnum}. \pali{D\=a\d th\=adh\=atu\d m por\=a\d nakacetiy\=ani ca vandanatth\=aya.}\\
\hspace*{6mm}To worship the Tooth Relic and the ancient shrines.}\\[1mm]

\stepcounter{convnum}\medskip
\parbox[lt]{0.93\linewidth}{\raggedright\arabic{convnum}. \pali{Id\=ani kattha v\=as\=upagat'attha?}\\
\hspace*{6mm}Where do you stay now?}\\[1mm]
{\small $\triangleright$ Literally, \pali{v\=as\=upagat\=a} (v\=asa + upagat\=a) means `a taken living place.' The question so precisely means ``Where is your taken living place now?''}

\stepcounter{convnum}\medskip
\parbox[lt]{0.93\linewidth}{\raggedright\arabic{convnum}. \pali{Maya\d m id\=ani Se\.nkha\d d\d dasela-nagare R\=ajav\=ithiya\d m navama\.nke gehe vas\=ama.}\\
\hspace*{6mm}We now stay at No.\ 9, King's Street, Kandy.}\\[1mm]

\stepcounter{convnum}\medskip
\parbox[lt]{0.93\linewidth}{\raggedright\arabic{convnum}. \pali{Kad\=a tumhe cetiyavandanatth\=aya gamissatha?}\\
\hspace*{6mm}When will you go on a pilgrimage to the shrines?}\\[1mm]

\stepcounter{convnum}\medskip
\parbox[lt]{0.93\linewidth}{\raggedright\arabic{convnum}. \pali{Ito dv\=iha-t\=ihaccayena maya\d m Anur\=adhapura\d m gamiss\=ama.}\\
\hspace*{6mm}We shall go to Anur\=adhapura after two or three days.}\\[1mm]

\stepcounter{convnum}\medskip
\parbox[lt]{0.93\linewidth}{\raggedright\arabic{convnum}. \pali{Tumhe Marammara\d t\d the kasmi\d m padese vasatha?}\\
\hspace*{6mm}In which part of Burma do you live?}\\[1mm]

\stepcounter{convnum}\medskip
\parbox[lt]{0.93\linewidth}{\raggedright\arabic{convnum}. \pali{Maya\d m R\=ama\~n\~nama\d n\d dale Moulmein-nagare vas\=ama.}\\
\hspace*{6mm}We live in the city of Moulmein, in R\=ama\~n\~na territory, i.e.\ Lower Burma.}\\[1mm]

\stepcounter{convnum}\medskip
\parbox[lt]{0.93\linewidth}{\raggedright\arabic{convnum}. \pali{Tattha ki\d m kamma\d m karont\=a j\=ivika\d m kappetha?}\\
\hspace*{6mm}How (with what occupation) do you earn your livelihood there?}\\[1mm]
{\small $\triangleright$ Precisely, this can also be translated as ``Doing what work, do you make a living there?''}

\stepcounter{convnum}\medskip
\parbox[lt]{0.93\linewidth}{\raggedright\arabic{convnum}. \pali{Maya\d m kassakakammena d\=arus\=ara-vikkayena ca puttad\=are posema.}\\
\hspace*{6mm}We support our families by agriculture and trading on timber.}\\[1mm]

\stepcounter{convnum}\medskip
\parbox[lt]{0.93\linewidth}{\raggedright\arabic{convnum}. \pali{Kad\=a sakara\d t\d tha\d m pa\d tigamissatha?}\\
\hspace*{6mm}When will you return to your own country?}\\[1mm]

\stepcounter{convnum}\medskip
\parbox[lt]{0.93\linewidth}{\raggedright\arabic{convnum}. \pali{Ito catum\=asaccayena maya\d m sadesa\d m gamiss\=ama.}\\
\hspace*{6mm}We will return to our native land four months hence.}\\[1mm]

\stepcounter{convnum}\medskip
\parbox[lt]{0.93\linewidth}{\raggedright\arabic{convnum}. \pali{Nanu tatth\=api bah\=uni cetiy\=ani santi?}\\
\hspace*{6mm}Are not there many shrines in your country too?}\\[1mm]

\stepcounter{convnum}\medskip
\parbox[lt]{0.93\linewidth}{\raggedright\arabic{convnum}. \pali{\=Ama, Tigumba-mah\=acetiya-pamukh\=ani anekasahassa-cetiy\=ani santi.}\\
\hspace*{6mm}Yes, there are many thousands of pagodas of which the great shrine `Shwe-dagon' is the foremost.}\\[1mm]
{\small $\triangleright$ Note that, even with negative meaning, when we ask with \pali{nanu}, `yes' is expected as a positive response. This means using \pali{nanu} and just \pali{nu} is more or less the same (see Chapter \ref{chap:ques}). While English translation is a little complex, the P\=ali sentence is simple, by using apposition of compounds.}

\stepcounter{convnum}\medskip
\parbox[lt]{0.93\linewidth}{\raggedright\arabic{convnum}. \pali{Tigumbacetiya\d m kasmi\d m \d th\=ane pati\d t\d thita\d m?}\\
\hspace*{6mm}Where is the Shwe-dagon pagoda situated?}\\[1mm]

\stepcounter{convnum}\medskip
\parbox[lt]{0.93\linewidth}{\raggedright\arabic{convnum}. \pali{Ta\d m pana Rangoon-nagarassa uttaras\=im\=asanne pati\d t\d thita\d m.}\\
\hspace*{6mm}It is situated near the northern boundary of the city of Rangoon.}\\[1mm]
{\small $\triangleright$ Showing the power of P\=ali compounds, \pali{uttaras\=im\=asanna} (uttara + s\=im\=a + \=asanna) is a good example. This means `a neighborhood of northern boundary.'}

\stepcounter{convnum}\medskip
\parbox[lt]{0.93\linewidth}{\raggedright\arabic{convnum}. \pali{Ta\d m ki\d m nidahitv\=a kena k\=ar\=apita\d m?}\\
\hspace*{6mm}What was enshrined in it and by whom was it erected?}\\[1mm]
{\small $\triangleright$ A more precise translation of this can be ``Having what deposited [in that], by whom was it made erected?'' Here \pali{k\=ar\=apita} is in causative form. Hence the sentence is in casual passive structure (see Chapter \ref{chap:pass}, \ref{chap:caus}, and \ref{chap:vform}).}

\stepcounter{convnum}\medskip
\parbox[lt]{0.93\linewidth}{\raggedright\arabic{convnum}. \pali{Pa\d thama\d m t\=ava Bhagavato kesadh\=atuyo nidahitv\=a Tapussa-Bhallikan\=amehi dv\=ihi v\=a\d nijehi pati\d t\d th\=apitan'ti vadanti.}\\
\hspace*{6mm}It is said that it was first erected by the two merchants Tapussa and Bhallika, enshrining the hair relics of the Buddha.}\\[1mm]
{\small $\triangleright$ By using an active verb form (\pali{vadanti}), a precise translation will be ``They say that \ldots''}

\stepcounter{convnum}\medskip
\parbox[lt]{0.93\linewidth}{\raggedright\arabic{convnum}. \pali{Pacc\=a pana bah\=uhi r\=aja-r\=aj\=amacc\=ad\=ihi n\=an\=av\=aresu pa\d tisa\.nkhata\d m va\d d\d dhita\~nca.}\\
\hspace*{6mm}Afterwards, on many occasions, it was repaired and enlarged by kings, ministers and other devotees.}\\[1mm]
{\small $\triangleright$ Exactly, the compound \pali{r\=ajar\=aj\=amacc\=adi} (\pali{r\=aja + r\=aj\=amacca + \=adi}) means ``kings, ministers, etc.''}

\stepcounter{convnum}\medskip
\parbox[lt]{0.93\linewidth}{\raggedright\arabic{convnum}. \pali{Tassa cetiyassa \=ak\=ara\d m sa\.nkhepena me kathetu\d m sakkosi?}\\
\hspace*{6mm}Can you give me a short description of that pagoda?}\\[1mm]

\stepcounter{convnum}\medskip
\parbox[lt]{0.93\linewidth}{\raggedright\arabic{convnum}. \pali{(1) Ta\d m pana tiya\d d\d dhasata-ratanubbedha\d m.}\\
\hspace*{6mm}Its height is about 250 cubits.}\\[1mm]
{\small $\triangleright$ In this sentence, \pali{tiya\d d\d dhasata} means `the third hundred with a half,' thus 250. This form is unconventional. As described in textbooks, 250 is \pali{a\d d\d dhateyyasata} (see more in Chapter \ref{chap:num}).}

\stepcounter{convnum}\medskip
\parbox[lt]{0.93\linewidth}{\raggedright\arabic{convnum}. \pali{(2) Bah\=uhi khuddakacetiyehi n\=an\=asata-pa\d tim\=a-gharehi ca pariv\=arita\d m.}\\
\hspace*{6mm}It is surrounded by many small pagodas and many hundreds of image-houses.}\\[1mm]

\stepcounter{convnum}\medskip
\parbox[lt]{0.93\linewidth}{\raggedright\arabic{convnum}. \pali{(3) Aggato y\=ava majjh\=a suva\d n\d napa\d t\d tehi ch\=adita\d m.}\\
\hspace*{6mm}From the top down to the middle it is covered with gold plate.}\\[1mm]

\stepcounter{convnum}\medskip
\parbox[lt]{0.93\linewidth}{\raggedright\arabic{convnum}. \pali{(4) Majjhato y\=ava p\=ad\=a suva\d n\d nena \=alimpita\d m.}\\
\hspace*{6mm}And from the middle to the foot it is overlaid with a thin coating of gold.}\\[1mm]

\stepcounter{convnum}\medskip
\parbox[lt]{0.93\linewidth}{\raggedright\arabic{convnum}. \pali{(5) T\=ihi p\=ak\=arehi parikkhitta\d m ta\d m cetiya\d m rattindiva\d m suva\d n\d napabbato viya virocati.}\\
\hspace*{6mm}Surrounded by three walls, the pagoda shines like a golden mountain, day and night.}\\[1mm]

\phantomsection
\addcontentsline{toc}{section}{Does your dog bite?}
\section*{Does your dog bite?}

This is not exactly a conversation, but a joke. Since they are short and self-contained, jokes are a good starting point for practicing P\=ali composition. Not every joke, however, is easy for cross-language conversion. This one is easy. Please try to read the P\=ali version using a dictionary. The piece won the `best joke submitted by a well-known scientist.'\footnote{For more information, see \url{http://laughlab.co.uk/}.} It is submitted by Nobel laureate, and professor of chemistry, Sir Harry Kroto.

\begin{quote}
\pali{%
Eko puriso v\=ithiya\d m sa\~ncaranto a\~n\~na\d m purisa\d m passati atimahanta\d m sunakha\~nca. So eva\d m pucchati `Kh\=adati nu kho tuyha\d m sunakho'ti? `Mama sunakho na kh\=adat\=i'ti paro vissajjeti. Tato pa\d thamo puriso ta\d m sunakha\d m s\=udara\d m paharati. Tassa hattho sunakhena kh\=adayitv\=a, so ugghoseti `Tuyha\d m sunakho na kh\=adat\=iti ma\~n\~ni\~n'ti.\footnote{Here is two-leveled direct speech, \pali{ugghoseti} ``\,`\pali{Tuyha\d m sunakho na kh\=adati}' \pali{iti ma\~n\~ni\d m}'' \pali{iti}.} Dutiyo puriso vissajjeti `Na mayha\d m so sunakho hot\=i'ti.
}
\end{quote}

\begin{quote}
A man walking down the street sees another man with a very big dog.  The man says: ``Does your dog bite?'' The other man replies: ``No, my dog doesn't bite.''  The first man then pats the dog, has his hand bitten off, and shouts; ``I thought you said your dog didn't bite.'' The other man replies: ``That's not my dog.''
\end{quote}

\phantomsection
\addcontentsline{toc}{section}{Pets}
\section*{Pets}

Another joke is also, sort of, about dog. It is a part of Woody Allen's Standup Comic.\footnote{\url{http://www.ibras.dk/comedy/allen.htm}} It is a bit challenging because the narration is in past tense. But its structure is simple. A discourse marker, like ``y'know,'' is difficult to translate. I use \pali{passasi} (You see) for this. But if you feel it makes things confusing, just ignore it. If you find it is amusing in P\=ali language, your learning is successful. Congratulation! 

\begin{quote}
\pali{%
B\=alakak\=ale aha\d m accanta\d m sunakha\d m icchi\d m. Nimm\=ulatt\=api\footnote{For \pali{nimm\=ulatt\=a}, see Sentence No.\,\ref{conv:nimmuu} above.} maya\d m abhavimh\=a. Aha\d m khuddako d\=arako abhavi\d m. Mama m\=at\=apitaro me sunakha\d m d\=atu\d m n\=asak\-ki\d msu nimm\=ulattena. Tasm\=a sunakha\d t\d th\=ane, `so sunakho hot\=i'ti vadi\d msu, te mayha\d m pip\=ilik\=a ad\=asi\d msu. Ahampi na j\=ani\d m, (passasi,) `so sunakho'ti ma\~n\~ni\d m. Aha\d m dandho d\=arako ahosi\d m. `Spot'ti n\=ama\d m katv\=a ta\d m damesi\d m, (passasi). Ekasmi\d m rattiya\d m Sheldon Finklestein cir\=ayitv\=a geha\d m \=agantv\=a ma\d m hi\d msitu\d m v\=ayami. Spot mama bh\=ag\=i ahosi. `Han\=a'ti vadi\d m, tato Sheldon mama sunakhe akkami.
}
\end{quote}

\begin{quote}
When I was little boy, I wanted a dog desperately, and we had no money. I was a tiny kid, and my parents couldn't get me a dog, 'cause we just didn't have the money, so they got me, instead of a dog -- they told me it was a dog -- they got me an ant. And I didn't know any better, y'know, I thought it was a dog, I was a dumb kid. Called it `Spot'. I trained it, y'know. Coming home late one night, Sheldon Finklestein tried to bully me. Spot was with me. And I said ``Kill!'', and Sheldon stepped on my dog. 
\end{quote}

\phantomsection
\addcontentsline{toc}{section}{In the Woods}
\section*{In the Woods}

When \url{laughlab.co.uk} is mentioned, it will be a big miss if we do not address the world's funniest joke.\footnote{See the report of the research conducted by Richard Wiseman in \url{https://richardwiseman.files.wordpress.com/2011/09/ll-final-report.pdf}, or at \url{http://laughlab.co.uk/}.} This piece is more difficult to translate because the funny point is from the use of an English verb. It may not be funny like you read from the English version. If you feel that it does not sound right enough, it is your turn now to make a better translation of this. If you can do that, my aim of writing this book is perfectly fulfilled.

\begin{quote}
\pali{%
Dve New-Jersey-ra\d t\d thik\=a ludd\=a vanasa\d n\d de gacchanti. Eko luddo bh\=umiya\d m patati. P\=a\d nanena vin\=a viya tassa akkh\=i s\=ise pa\d tivattenti. A\~n\~no luddo s\=igha\d m d\=urabh\=a\-sanayanta\d m n\=iharitv\=a acc\=ayika\d m kicca\d m \=amanteti. \\`Mama mitto mato! ki\d m kattabban'ti?, so dhuran\-dharassa vadati. Dhurandharo upasamena eva\d m vadati `Upasamma, bho. Aha\d m upak\=atu\d m sakkomi. Pa\-\d thama\d m tassa mara\d na\d m niyatatta\d m karoh\=i'ti. Eko tu\d nh\=ibh\=avo atthi. Atha kho aggin\=a\d liy\=a saddo s\=uyati. So luddo tato vadati `kato, id\=ani kin'ti?
}
\end{quote}

\begin{quote}
A couple of New Jersey hunters are out in the woods when one of them falls to the ground. He doesn't seem to be breathing, his eyes are rolled 
back in his head. The other guy whips out his cell phone and calls the emergency services. He gasps to the operator: ``My friend is dead! What can I do?'' The operator, in a calm soothing voice says: ``Just take it easy. I can help. First, let's make sure he's dead.'' There is a silence, then a shot is heard. The guy's voice comes back on the line. He says: ``OK, now what?''
\end{quote}

\phantomsection
\addcontentsline{toc}{section}{Notes on Neologism}
\section*{Notes on Neologism}\label{sec:neologism}

In bringing P\=ali conversation to modern context, one challenging task, or entertaining task for some, is to find a proper term for things that never exist in the P\=ali world. If you know enough basic words, you can compose your own ones. It is quite enjoyable thing to do, and often amusing.

In fact, there are many of P\=ali words used in modern context as we see in Ven.\,Buddhadatta's English-P\=ali dictionary. Some are easy to understand and remember, for example, \pali{vijjubala} (electricity, electrical energy). This makes them widely applicable, for example, \pali{vijjuv\=ijan\=i} [f.] (electric fan), \pali{vijjuratha} [m.] (electric car), \pali{vijjukhula} [nt.] (shaver, electric razor), etc. 

For devices or machines, we can add \pali{yanta} [nt.] to the end of the compounds. For example, \pali{ch\=ay\=ar\=upa} [nt.] means `photograph,' hence \pali{ch\=ay\=ar\=upayanta} means `camera.' In a joke exemplified above, I used \pali{d\=urabh\=asanayanta} for `telephone' suggested by Ven.\,Buddhadatta. The term can also be coined in other way, for example, \pali{d\=urakathanayanta}, \pali{d\=urasaddayanta}. I have seen some call microphone `\pali{saddamaggayanta}' (device of sound path), but Ven.\,Buddhadatta uses `\pali{saddavipph\=arakayanta}' (device for diffusing sound). The latter may be close in meaning but a mouthful. That is to say, to make an understandable word that can capture the modern meaning and be easy to use is a kind of art. Not every good word will be acceptable in use.

While engaging in an immediate conversation, when nothing comes up to your mind, you can use a hybrid compound with a P\=ali term as the last part. For example, for `computer' you can use \pali{computer-yanta} (computer-machine) or \pali{computer-upakara\d na} [nt.] (computer-device) or \pali{computer-bha\d n\d da} [nt.] (computer-ware). A proper P\=ali word for this is \pali{ga\d nakayanta}.

How about `Facebook'? If you do not use its literal translation `\pali{mukhapotthaka},' which has a good chance of misunderstanding, you have to make a compound out of it. We can use \pali{\=ay\=acana} [nt.] as `application.' Thus we get \pali{Facebook-\=ay\=acana} (Facebook-application).

How about `software'? Its literal term `\pali{mudubha\d n\d da}' is, in a way, ridiculous. I have thought this for a while. Then I come up with \pali{niyogam\=al\=a} [f.] (sequence of commands). That is a more correct definition of it, but maybe too technical. If you happen to use unfamiliar words like this one in your P\=ali essay, I suggest that you should also provide the readers with a glossary.

As you have seen, making a new word is a kind of fun. But you have to master all basic things first. So, practice and have fun!
