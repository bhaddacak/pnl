\chapter{(There is) \headhl{a book}}\label{chap:nom}

We will start with an easy task, like a little child: to call a thing out. Before we do it with P\=ali, there are fundamental concepts we need to learn first. Although English and P\=ali are relative to each other, they are different in many respects. The most obvious one is about \emph{word order}. Generally speaking, word order matters in English but not (much) in P\=ali.\footnote{Order of words in P\=ali is mostly about style, not a strictly grammatical function.} In English, ``I run'' and ``I hit a ball'' is grammatical and meaningful, but ``a hit ball I'' is ungrammatical and meaningless, and ``a ball hits me'' is grammatical but carries a different meaning. However, sometimes we can use ``run I'' for a poetic effect.

In P\=ali, you can say ``I run'' or ``run I'' or even just ``run'' to mean the speaker moves on foot quickly. However, basically there is a typical order of words in P\=ali, i.e.\ SV (subject-verb) for intransitive verbs and SOV (subject-object-verb) for transitive verbs. In the latter case English normally uses SVO (subject-verb-object). Therefore, ``I hit a ball'' is typically said in P\=ali as ``I a ball hit.'' Nevertheless, any sequence of words carries the same meaning (but different emphasis). How does P\=ali maintain the word function when its position is changed? The answer is in a technical term---\emph{inflection}.\footnote{``Variation in the form of a lexical word reflecting different morphosyntactic categories.'' \citep[p.~227]{brownmiller:dict}}

Here is a down-to-earth definition: ``The changes in the form of a word as that word assumes different functions in a sentence are called \emph{inflection}.''\footnote{\citealp[p.~44]{fairbairn:understanding}} English does have inflection. As we have seen in ``a ball hits me,'' `hit' changes to `hits' and `I' changes to `me' when their functions change. To agree with the subject `a ball,' the verb `hit' becomes `hits,' and to act as an object, `I' becomes `me.' When the order is changed to ``a ball me hits,'' if the word formation is taken seriously, the correct meaning of the sentence can be obtained, but it is ungrammatical nonetheless. English is a language with limited inflections\footnote{Modern English has only eight inflectional affixes: (1) -s 3rd person singular present, (2) -ed past tense, (3) -ing progressive, (4) -en past participle, (5) -s plural, (6) -'s possessive, (7) -er comparative, and (8) -est superlative  \citep[p.~47]{fromkin:language}.}, whereas P\=ali is highly inflectional language.\footnote{Languages that do not change word formation are called \emph{analytic languages}. Highly analytic languages, for example, are Chinese, Vietnamese, and Thai. On the other hand, \emph{synthetic languages} change word formation normally, for instance, Greek, Latin and their offspring such as French, Spanish, Portuguese, Italian, and Russian in Europe; Sanskrit, Pr\=akrit, P\=ali, and others variations in India. German is moderately synthetic, for it relies heavily also on word order \citep[See][pp.~44--5]{fairbairn:understanding}. Japanese is also a synthetic language with SOV pattern similar to P\=ali.}

There are two kinds of inflection applying to different types of words---\emph{declension}\footnote{``For a given noun, pronoun or adjective, [declension is] the set of its forms, each consisting of a stem and a suffix.'' \citep[p.~122]{brownmiller:dict}} and \emph{conjugation}\footnote{``For a given verb, [conjugation is] all its forms, consisting of a stem and an inflectional affix.'' \citep[p.~99]{brownmiller:dict}}. In this chapter and some followings we will learn to form simple sentences focusing mainly on declension, which applies to nouns, pronouns, and adjectives. Verbs and conjugations are more complicated, so it is better to learn them later.

How to say ``There is a book'' in P\=ali, then? Let us deal with `book' first. In P\=ali and other many inflectional languages, a word that we use to call things (noun) has a gender. It is like dividing words roughly into groups, namely \emph{masculine} (m.), \emph{feminine} (f.), and \emph{neuter} (nt.) (neither the former two). Normally, a word belongs to only one group, or has one gender, but sometimes it has two or three genders. Genders of nouns generally correspond to their natural state. For example, \pali{purisa} (man) is masculine and \pali{ka\~n\~n\=a} (girl) is feminine, but it is not always so.\footnote{Good examples of these are \pali{m\=atug\=ama} (woman), \pali{d\=ara} (wife), and \pali{orodha} (concubine). All are masculine. Aggava\d msa discusses this in Sadd-Pad Ch.\,5, ``\pali{m\=atug\=amasaddo ca orodhasaddo ca d\=arasaddo c\=ati ime itthipadatthav\=acak\=api sam\=an\=a ekantena pulli\.ng\=a bhavanti.}''} You can usually guess genders of obvious words, but it is better to check with a dictionary. 

A word that means `book' in P\=ali is \pali{potthaka}. It is used as both masculine and neuter. That means when you compose a new sentence you have choices, and when you read a text you have to be careful for you may encounter either form. A general clue to tell the gender of words in dictionary or raw form is to see their ending. Table \ref{tab:genders} summarizes the typical endings of each gender.\footnote{Traditional textbooks tend to say that there are more endings than these in each case. For example, \pali{\=a} ending can happen to be masculine (see Sadd-Pad Ch.\,6). I treat this as exceptional cases, because we have not so many of them. In fact, only a handful of masculine words has \pali{\=a} ending as their raw form. We mostly see \pali{\=a} in their final inflected forms.} In practice, however, gender agreement can be less strict. For example, some m.\ nouns when used in plural, its meaning can include both genders, e.g.\ \pali{putt\=a} (sons and daughters).\footnote{Except \pali{brahma, inda, buddha, purisa,} and \pali{m\=atug\=ama} says Sadd\,823. In this formula, Aggava\d msa asserts sexist position by claiming that male is superior to female (\pali{puris\=a hi padh\=an\=a \ldots itthiyo pana appadh\=an\=a}) for two reasons. First, a buddha-to-be never takes female forms. And second, the Brahma gods are for men only. I add this remark for those who may be interested in gender issue in Buddhism.}

\begin{table}[!hbt]
\centering
\caption{Endings of words in raw form of each gender}
\label{tab:genders}
\bigskip
\begin{tabular}{l>{\itshape}c} \toprule
\bfseries Gender & \bfseries\upshape Endings \\ \midrule
masculine & a, i, \=i, u, \=u \\
feminine & \=a, i, \=i, u, \=u \\
neuter & a, i, u \\
\bottomrule
\end{tabular}
\end{table}

How to render then? Words that we find in a P\=ali dictionary are not ready to use, particularly nouns, pronouns, and adjectives.\footnote{These three types of words are all under the same category---\pali{n\=ama}, because they are subject to the same declension rules. They are called \pali{suddhan\=ama} or \pali{n\=aman\=ama, sabban\=ama,} and \pali{gu\d nan\=ama} respectively.} They have to be changed, technically called \emph{decline}, corresponding to their gender and function. Gender is a property of \pali{n\=ama}. Each noun has an intrinsic gender, but some may have more than one. Pronouns and adjectives can be of three genders corresponding to the noun they represent or modify. We will talk about pronouns and adjectives later. Another point to be considered before we compose a P\=ali sentence, apart from gender, is the word's function.

In English and many languages, a word's function is determined by its position in sentences. Subject and object cannot be interchanged in such languages, otherwise the meaning will change. P\=ali does not care (much) about word position. It uses word formation to tell its function, as we mentioned \emph{inflection} earlier. For all words under \pali{n\=ama} group, we call this \emph{declension}. To put it simply, when we use a word, a noun in this case, we have to change its raw form to an inflected form corresponding to its intended function. Inflected words are unlikely to be found in any dictionary, except for irregular ones. Therefore we have to learn to compose and recognize them. This is one of tedious tasks of traditional P\=ali students.

\phantomsection
\addcontentsline{toc}{section}{Declension of Nominative Case}
\section*{Declension of Nominative Case}

P\=ali has eight cases of declension.\footnote{Here are all cases with P\=ali terms: Nominative (\pali{pa\d tham\=a}), Accusative (\pali{dutiy\=a}), Instrumental (\pali{tatiy\=a}), Dative (\pali{catutth\=i}), Ablative (\pali{pa\~ncam\=i}), Genitive (\pali{cha\d t\d th\=i}), Locative (\pali{sattam\=i}), and Vocative (\pali{\=alapana}). We will come to all of these in subsequent lessons.} Nominative case (nom.) is the first one. It is primarily used to identify the subject of sentences. To apply any declension we have to know the word's gender (m., f., or nt.), the word's ending (\pali{a, \=a, i, \=i, u,} or \pali{\=u}) given by a dictionary, and the word's number (singular, sg. or plural, pl.), used in the sentences. Rules for nominative case declension are shown in Table \ref{tab:nomreg}.

\begin{table}[!hbt]
\centering
\caption{Nominative case endings of regular nouns}
\label{tab:nomreg}
\bigskip
\begin{tabular}{@{}>{\bfseries}l*{5}{>{\itshape}l}@{}} \toprule
\multirow{2}{*}{G. Num.} & \multicolumn{5}{c}{\bfseries Endings} \\ \cmidrule(l){2-6}
& a & i & \=i & u & \=u \\ \midrule
m. sg. & \texthl{\replacewith{a}{o}} & i & \=i & u & \=u \\
m. pl. & \replacewith{a}{\=a} & \replacewith{i}{\=i} & \=i & \replacewith{u}{\=u} & \=u \\
& & \texthl{\replacewith{i}{ayo}} & \texthl{\replacewith{\=i}{ino}} & \texthl{\replacewith{u}{avo}} & \texthl{\replacewith{\=u}{uno}} \\
\midrule
nt. sg. & \texthl{a\d m} & i & & u & \\
nt. pl. & \texthl{\replacewith{a}{\=ani}} & \replacewith{i}{\=i} & & \replacewith{u}{\=u} & \\
& & \replacewith{i}{\=ini} & & \replacewith{u}{\=uni} & \\
\midrule
& \=a & i & \=i & u & \=u \\
\midrule
f. sg. & \=a & i & \=i & u & \=u \\
f. pl. & \=a & \replacewith{i}{\=i} & \=i & \replacewith{u}{\=u} & \=u \\
& \=ayo & iyo & \replacewith{\=i}{iyo} & uyo & \replacewith{\=u}{uyo} \\
\bottomrule
\end{tabular}
\end{table}

Unlike traditional approach, I present here in the table only the changes of endings. For paradigmatic approach, see Appendix \ref{chap:decl}. In the table, \pali{\replacewith{a}{o}} means from its raw form you have to change the word's `\pali{a}' ending to `\pali{o}.' The color-highlighted items need more attention for their conspicuous form. These are worth remembering.

Let us focus on singular nominative case first. The rule of nom.\ sg.\ is quite simple because most dictionary forms are retained, except just two points: \pali{a} ending of m.\ and nt. To our mission word `\pali{potthaka}' (book), its nom.\ form therefore is \pali{potthako} (m.) or \pali{potthaka\d m} (nt.). You can use either gender, but be consistent with it. Even though `neuter' book makes more sense, `masculine' book is also found in the scriptures.\footnote{In \textsc{P\=ali\,Platform} 3, the program shows that \pali{potthako} has 35 occurrences and \pali{potthaka\d m} (including acc.) 43 in the whole CSTR collection. Do not take these number too seriously, just hold them as rough count.}

To complete our task, to say ``There is a book,'' we have to put the term into a sentence. Grammatically, a sentence is normally composed of subject and its predicate. In P\=ali, a common way to say something existing or being present at the moment is to use verb `to be,' e.g.\ `\pali{hoti}' (more about this in Chapter \ref{chap:verb-be}). Therefore, the complete sentence is:

\palisample{potthako hoti. {\upshape (m.)}\sampleor potthaka\d m hoti. {\upshape (nt.)}}

This can fulfill our task happily. But practically it is often not put in that way, because P\=ali has a peculiar kind of sentence: verbless sentence---``When it is asserted simply that a thing is something \ldots two nouns (one of them usually an adjective or pronoun) may merely be juxtaposed.''\footnote{\citealp[p.~9]{warder:intro}. This is traditionally called \pali{li\.ngattha} (Kacc\,284; R\=upa\,65, 283; Sadd\,577; Niru\,62). In fact it is not uncommon to ancient languages because ``in Greek and Latin, an idea---especially a state---can be expressed without a verb'' \citep[p.~35]{fairbairn:understanding}.} So, the complete sentence, although it should be with some modifier, can be just:

\palisample{potthako.\sampleor potthaka\d m.}

Declension of proper nouns works in the same way, if you have a name in P\=ali. For example, \pali{\=Ananda} (m.) has nominative form as \pali{\=Anando}. If you do not have a P\=ali name, but you have to use your name in P\=ali, it can be troublesome. That is the reason why all Theravada monks have their P\=ali name. This name has to be recited formally in the ordination ceremony. Normally, the preceptor will give a name to the candidates. In modern context, you have to name your own P\=ali representation. If you choose a word from a dictionary or make a compound out of it, it will be no problem in any case. If you insist to use your native name, you have to adapt it to agree with P\=ali.

\label{par:foreignname}First, the name has to be ended with a vowel: \pali{a, i, \=i, u,} or \pali{\=u} for males and \pali{\=a, i, \=i, u,} or \pali{\=u} for females. Second, letters not belonging to P\=ali have to be change accordingly. For example, \emph{Smith} can be change to \pali{Smitha} yielding nom.\ \pali{Smitho}.\footnote{There is no rule whatsoever about this. You can play around with it, and it makes some fun. Japanese also has a funny way to say foreign words. I am fond of that. Name transformation across languages is common. For example, \emph{Y\=o\d h\=an\=an} (Hebrew) became \emph{I\=oann\=es} (Greek), then \emph{Johannes, Joannes} (Latin), then \emph{Johan, John, Jon} (English), and \emph{Giovanni} (Italian). In P\=ali it can be \pali{Johana} or \pali{Johanna} or \pali{Jona}. Finding Latin origin of your name, if it has one, can be helpful in some cases.} This has no meaning in P\=ali.

Another practical way to deal with foreign names is to form a compound with \pali{n\=ama}, for example, \pali{Smith-n\=ama} (a person named Smith). Then we can decline the word as usual, i.e.\ \pali{Smith-n\=amo [puriso]} (a male Smith), \pali{Smith-n\=am\=a [itth\=i]} (a female Smith), \pali{Smith-n\=ama\d m [kula\d m]} (a Smith family). In an informal situation, a bare foreign name can be used in P\=ali sentences, but this is limited to only nominative case (vocative case can be another possibility).

Before we end this chapter, let us talk about plural. Like English, number matters in P\=ali. To say ``(There are) books,'' you have to make the term plural. As we have seen in the table above, the rule for declining plural nouns is a little complicated. A general idea of making a plural noun is to lengthen its ending's sound as we see in short vowels. Also, an additional sound can be added to mark the plural state. Using plural form of \pali{potthaka}, we can say that briefly in P\=ali as:

\palisample{potthak\=a.\sampleor potthak\=ani.}

I have to say something about declension rules. As a matter of fact, in P\=ali, and all other languages, rules came after the language itself. We have records of language uses in the form of scriptures. Grammarians try to make sense of the language by finding its patterns and formulating rules. This means the rules generally work fine in regular manner, but sometimes they simply do not. We often find anomalies in P\=ali because of its accumulating nature time after time. The peculiar features of the language mostly are the remnants of the far past. Here is the point. There are a number of words that decline irregularly. These are listed in Appendix \ref{decl:irrn}. You should go through this list at least one time to be familiar with its terrain. So, when you use or meet some of peculiar words, you can get an inkling. You can learn more about irregular nouns in Chapter \ref{chap:irrn}.

As you have seen, throughout the book (except conversations in Chapter \ref{chap:conver}) I do not use capital letters in P\=ali sentences. One reason is that they are not really necessary. All local scripts do not have this feature, but scriptures can be read without any difficulty. To mark a sentence, we just use a period. However, in the P\=ali collection we have, capital letters are used normally. So, if the passages taken from the collection are capitalized, they are mostly retained. That means if you see capital letters in some examples, the passages are cut from the beginning of the sentences or stanzas. Otherwise, they are cut in the middle.

Before leaving you should spend your time on the exercise. The first exercise is more or less a hide and seek game. I have listed a number of words in Vocabulary (Appendix \ref{chap:vocab}). These words help you start learning P\=ali quickly. So, you should be familiar with them. Our first exercise is to find words in the list and make them nominative.

\section*{Exercise \ref{chap:nom}}
Say these in P\=ali in all possible forms using word list in Appendix \ref{vocab:noun}.
\begin{multicols}{2}
\RaggedRight
\begin{compactenum}
\item (There is) a tree.
\item (There are) trees.
\item (There is) a gecko.
\item (There are) geckoes.
\item (There is) an elephant.
\item (There are) elephants.
\item (There is) a language.
\item (There are) languages.
\item (There is) a tendon.
\item (There are) tendons.
\item (There is) a broom.
\item (There are) brooms.
\item (There is) a rope.
\item (There are) ropes.
\item (There is) a rainbow.
\item (There are) rainbows.
\item (There is) a bone.
\item (There are) bones.
\item (There is) a \mbox{thunderbolt}.
\item (There are) \mbox{thunderbolts}.
\item (There is) a coconut.
\item (There are) coconuts.
\item (There is) a needle.
\item (There are) needles.
\item (There is) a spoon.
\item (There are) spoons.
\item (There is) a stone.
\item (There are) stones.
\item (There is) a house.
\item (There are) houses.
\end{compactenum}
\end{multicols}
