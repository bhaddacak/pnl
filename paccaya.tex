\chapter{List of \pali{Paccaya}s}\label{chap:paccaya}

As I pointed out several times in the lessons, learning \pali{paccaya}s is the main method of the traditional approach to learn the language. Even though my approach is different, taking \pali{paccaya}s into consideration is inevitable. Digging deeper into P\=ali grammar, at some point you have to deal with these. To ease our learning and for referencing, I put all \pali{paccaya}s into order and gather them in one place. This does not include \pali{vibhatti}s that have more specific functions. For nominal \pali{vibhatti}s please see Appendix \ref{chap:decl}, and for verbal see Appendix \ref{chap:conj}.

In the table below, all \pali{paccaya}s explained in this book are listed, around 360 items. The first column is the name of \pali{paccaya}s. All different forms mentioned by textbooks are listed as many as possible. The second column shows the area of their use: \pali{\textbf{N}\=ama} (noun), \pali{\textbf{S}abban\=ama} (pronoun), \pali{\textbf{\=A}khy\=ata} (verb), \pali{\textbf{K}ita} (primary derivation), and \pali{\textbf{T}addhita} (secondary derivation). The third column refers to the main traditional textbooks: \pali{\textbf{K}acc\=ayana}, \pali{\textbf{M}oggall\=ana}, and \pali{\textbf{S}addan\=iti}. The last column shows the pages that the items are mentioned in this book.

\bigskip
\begin{longtable}[c]{@{}>{\itshape\raggedright\arraybackslash}p{0.15\linewidth}%
	>{\centering\arraybackslash}p{0.08\linewidth}%
	>{\centering\arraybackslash}p{0.08\linewidth}%
	>{\raggedright\arraybackslash}p{0.001\linewidth}%
	>{\raggedleft\arraybackslash}p{0.47\linewidth}@{}}
\caption{All P\=ali \pali{paccaya}s}\label{tab:paccaya}\\
\toprule
\bfseries\upshape \mbox{Paccaya} & \bfseries In & \bfseries Ref. & & \bfseries Page \\ \midrule
\endfirsthead
\multicolumn{5}{c}{\tablename\ \thetable: All P\=ali \pali{paccaya}s (contd\ldots)}\\
\toprule
\bfseries\upshape \mbox{Paccaya} & \bfseries In & \bfseries Ref. & & \bfseries Page \\ \midrule
\endhead
\bottomrule
\ltblcontinuedbreak{5}
\endfoot
\bottomrule
\endlastfoot
%
a & AKT & KMS & & \pageref{pacca:a1}, \pageref{pacca:a2}; \pageref{pacck1:a1}, \pageref{pacck1:a2}, \pageref{pacck4:a}, \pageref{pacck6:a}, \pageref{pacckx:a}; \pageref{pacct10:a} \\
aka & K & M & & \pageref{pacck5:aka}, \pageref{pacckx:aka} \\
acca & T & M & & \pageref{pacct2:acca} \\
ajja & S & KMS & & \pageref{tab:indda} \\
ajju & S & KMS & & \pageref{tab:indda} \\
a\~n\~na & K & M & & \pageref{pacckx:aynyna} \\
a\d ta & K & M & & \pageref{pacckx:adta} \\
a\d na & K & M & & \pageref{pacck1:adna}, \pageref{pacckx:adna} \\
a\d ni & K & M & & \pageref{pacckx:adni} \\
a\d n\d da & K & M & & \pageref{pacckx:adndda} \\
ata & K & M & & \pageref{pacckx:ata} \\
ati & K & M & & \pageref{pacckx:ati} \\
atta & K & M & & \pageref{pacckx:atta} \\
atha & K & M & & \pageref{pacckx:atha} \\
athu & K & M & & \pageref{pacckx:athu} \\
ana & K & M & & \pageref{pacck2:ana}, \pageref{pacck6:ana} \\
ani & K & M & & \pageref{pacckx:ani} \\
an\=iya & K & KMS & & \pageref{pacck3:aniiya} \\
anta & K & KMS & & \pageref{pacck10:anta}, \pageref{pacckx:anta} \\
apa & K & M & & \pageref{pacckx:apa} \\
abha & K & M & & \pageref{pacckx:abha} \\
ama & K & KMS & & \pageref{pacckx:ama} \\
aya & T & M & & \pageref{pacct12:aya} \\
ara & K & M & & \pageref{pacckx:ara} \\
ara\d na & K & M & & \pageref{pacckx:aradna} \\
ala & K & KMS & & \pageref{pacckx:ala} \\
all & K & M & & \pageref{pacckx:ali} \\
ava & K & M & & \pageref{pacckx:ava} \\
asa & K & M & & \pageref{pacckx:asa} \\
asa\d na & K & M & & \pageref{pacckx:asadna} \\
assa & A & M & & \pageref{pacca:assa} \\
\=a & KT & S & & \pageref{pacck1:aa}; \pageref{pacct11:aa} \\
\=aka & K & M & & \pageref{pacckx:aaka} \\
\=ak\=i & T & M & & \pageref{pacct12:aakii} \\
\=a\d ta & K & M & & \pageref{pacckx:aadta} \\
\=a\d taka & K & M & & \pageref{pacckx:aadtaka} \\
\=a\d ta\d na & K & M & & \pageref{pacckx:aadtadna} \\
\=a\d nika & K & M & & \pageref{pacckx:aadnika} \\
\=ataka & K & M & & \pageref{pacckx:aataka} \\
\=ana & K & KS & & \pageref{pacck10:aana} \\
\=anaka & K & M & & \pageref{pacckx:aanaka} \\
\=ani & K & KS & & \pageref{pacckx:aani} \\
\=api & A & M & & \pageref{pacca:aapi} \\
\=amaha & T & M & & \pageref{pacct14:aamaha} \\
\=am\=i & T & M & & \pageref{pacct10:aamii} \\
\=aya & A & KMS & & \pageref{pacca:aaya} \\
\=ayitta & T & KS & & \pageref{pacct6:aayitta} \\
\=ara & K & M & & \pageref{pacckx:aara} \\
\=alu & T & M & & \pageref{pacct7:aalu} \\
\=avantu & T & M & & \pageref{pacct2:aavantu} \\
\=ava & K & M & & \pageref{pacckx:aava} \\
\=av\=i & K & KMS & & \pageref{pacck1:aavii}, \pageref{pacck2:aavii} \\
i & AK & KS & & \pageref{pacca:i}; \pageref{pacck5:i}, \pageref{pacckx:i} \\
ika & KT & KMS & & \pageref{pacck11:ika}; \pageref{pacct4:ika}, \pageref{pacct10:ika}, \pageref{pacct14:ika} \\
i\d t\d tha & T & KMS & & \pageref{pacct9:idtdtha} \\
i\d na & K & MS & & \pageref{pacckx:idna1}, \pageref{pacckx:idna2} \\
ita & KT & M & & \pageref{pacckx:ita}; \pageref{pacct2:ita} \\
ithi & K & M & & \pageref{pacckx:ithi} \\
idda & K & KS & & \pageref{pacckx:idda} \\
ina & K & KMS & & \pageref{pacck4:ina} \\
imantu & T & S & & \pageref{pacct10:imantu} \\
ima & KT & KMS & & \pageref{pacckx:ima}; \pageref{pacct2:ima}, \pageref{pacct3:ima}, \pageref{pacct4:ima}, \pageref{pacct8:ima}, \pageref{pacct10:ima} \\
iya & T & KMS & & \pageref{pacct1:iya}, \pageref{pacct2:iya}, \pageref{pacct4:iya}, \pageref{pacct6:iya}, \pageref{pacct8:iya}, \pageref{pacct9:iya}, \pageref{pacct10:iya}, \pageref{pacct14:iya} \\
ira & K & KS & & \pageref{pacckx:ira} \\
ila & K & M & & \pageref{pacckx:ila} \\
illa & T & KMS & & \pageref{pacct6:illa} \\
isa & K & KS & & \pageref{pacckx:isa} \\
isika & T & KS & & \pageref{pacct9:isika} \\
issika & T & M & & \pageref{pacct9:issika} \\
\=i & AT & KMS & & \pageref{pacca:ii}; \pageref{pacckx:ii}; \pageref{pacct2:ii}, \pageref{pacct10:ii}, \pageref{pacct11:ii}, \pageref{pacct12:ii} \\
\=ici & K & M & & \pageref{pacckx:iici} \\
\=i\d na & K & M & & \pageref{pacck2:iidna} \\
\=iya & AT & KMS & & \pageref{pacca:iiya}; \pageref{pacct6:iiya} \\
\=iva & K & M & & \pageref{pacckx:iiva} \\
\=ivara & K & KS & & \pageref{pacckx:iivara} \\
\=isa & K & M & & \pageref{pacckx:iisa} \\
u & K & M & & \pageref{pacckx:u} \\
u\d ta & K & M & & \pageref{pacckx:udta1}, \pageref{pacckx:udta2} \\
u\d n\=a & A & KS & & \pageref{pacca:udnaa} \\
unta & K & M & & \pageref{pacckx:unta} \\
uma & K & M & & \pageref{pacckx:uma} \\
ura & K & M & & \pageref{pacckx:ura} \\
ula & K & M & & \pageref{pacckx:ula} \\
uli & K & M & & \pageref{pacckx:uli} \\
usa\d n & T & S & & \pageref{pacct1:usadn} \\
ussa & KT & KS & & \pageref{pacckx:ussa}; \pageref{pacct1:ussa} \\
uv\=am\=i & T & M & & \pageref{pacct10:uvaamii} \\
\=u & K & M & & \pageref{pacckx:uu} \\
\=ura & K & KMS & & \pageref{pacckx:uura} \\
e & A & KS & & \pageref{pacca:e} \\
edh\=a & T & M & & \pageref{pacct13:edhaa} \\
eyya & T & S & & \pageref{pacct6:eyya} \\
eyyaka & T & M & & \pageref{pacct2:eyyaka} \\
eraka & K & M & & \pageref{pacckx:eraka} \\
eru & K & M & & \pageref{pacckx:eru} \\
o & A & KMS & & \pageref{pacca:o1}, \pageref{pacca:o2} \\
ota & K & M & & \pageref{pacckx:ota} \\
ora & K & M & & \pageref{pacckx:ora} \\
ola & K & M & & \pageref{pacckx:ola} \\
ka & KT & KMS & & \pageref{pacck6:ka}, \pageref{pacckx:ka1}, \pageref{pacckx:ka2}; \pageref{pacct3:ka}, \pageref{pacct12:ka}, \pageref{pacct14:ka} \\
ka\d n & T & KS & & \pageref{pacct5:kadn}, \pageref{pacct8:kadn} \\
ka\d na & T & M & & \pageref{pacct2:kadna}, \pageref{pacct5:kadna} \\
kana & K & M & & \pageref{pacckx:kana} \\
kabha & K & M & & \pageref{pacckx:kabha} \\
kala & K & M & & \pageref{pacckx:kala} \\
kasa & K & M & & \pageref{pacckx:kasa} \\
k\=ala & K & M & & \pageref{pacckx:kaala} \\
ki & K & M & & \pageref{pacckx:ki} \\
kika & K & M & & \pageref{pacck11:kika}, \pageref{pacckx:kika} \\
ki\d na & K & M & & \pageref{pacckx:kidna} \\
kiya & T & KMS & & \pageref{pacct3:kiya}, \pageref{pacct4:kiya} \\
kira & K & M & & \pageref{pacckx:kira} \\
kila & K & M & & \pageref{pacckx:kila} \\
ku & K & M & & \pageref{pacckx:ku} \\
kuma & K & M & & \pageref{pacckx:kuma} \\
k\=ika & K & M & & \pageref{pacckx:kiika} \\
k\=i\d ta & K & M & & \pageref{pacckx:kiidta} \\
k\=ira & K & M & & \pageref{pacckx:kiira} \\
kudtaka & K & M & & \pageref{pacckx:kudtaka} \\
kula & K & M & & \pageref{pacckx:kula} \\
k\=u & K & M & & \pageref{pacck2:kuu} \\
kha & AK & KMS & & \pageref{pacca:kha}; \pageref{pacck4:kha}, \pageref{pacckx:kha} \\
khara & K & M & & \pageref{pacckx:khara} \\
\mbox{kkhattu\d m} & T & M & & \pageref{pacct13:kkhattudm} \\
kkhi\d na & K & S & & \pageref{pacckx:kkhidna} \\
k\d n\=a & A & M & & \pageref{pacca:kdnaa} \\
k\d no & A & M & & \pageref{pacca:kdno} \\
kta & K & M & & \pageref{pacck8:kta} \\
ktavantu & K & M & & \pageref{pacck8:ktavantu} \\
kt\=av\=i & K & M & & \pageref{pacck8:ktaavii} \\
kti & K & M & & \pageref{pacck6:kti} \\
ktv\=a & K & M & & \pageref{pacck9:ktvaa} \\
ktv\=ana & K & M & & \pageref{pacck9:ktvaana} \\
kn\=a & A & M & & \pageref{pacca:knaa} \\
kya & A & M & & \pageref{pacca:kya} \\
krara & K & M & & \pageref{pacckx:krara} \\
kva & K & M & & \pageref{pacckx:kva} \\
kvara & K & M & & \pageref{pacckx:kvara} \\
kv\=a & K & M & & \pageref{pacckx:kvaa} \\
kvi & K & KMS & & \pageref{pacck1:kvi} \\
ki & K & M & & \pageref{pacckx:ki} \\
kuna & K & M & & \pageref{pacckx:kuna} \\
gaka & K & M & & \pageref{pacckx:gaka} \\
gara & K & M & & \pageref{pacckx:gara} \\
gu & K & M & & \pageref{pacckx:gu} \\
gha & K & M & & \pageref{pacckx:gha} \\
gha\d na & K & M & & \pageref{pacck1:ghadna}, \pageref{pacck4:ghadna} \\
ghi\d n & K & KS & & \pageref{pacck2:ghidn} \\
ghya\d na & K & M & & \pageref{pacck3:ghyadna} \\
ca & K & M & & \pageref{pacckx:ca} \\
ca & K & M & & \pageref{pacckx:cara} \\
cu & K & M & & \pageref{pacckx:cu} \\
ccha & K & S & & \pageref{pacckx:ccha} \\
cch\=ana & K & S & & \pageref{pacckx:cchaana} \\
cha & AKT & KMS & & \pageref{pacca:cha}; \pageref{pacckx:cha}; \pageref{pacct14:cha} \\
chara & K & M & & \pageref{pacckx:chara1}, \pageref{pacckx:chara2} \\
chera & K & M & & \pageref{pacckx:chera} \\
chika & K & M & & \pageref{pacckx:chika} \\
chilla & K & S & & \pageref{pacckx:chilla} \\
chuka & K & M & & \pageref{pacckx:chuka} \\
ja & K & S & & \pageref{pacckx:ja} \\
jara & K & M & & \pageref{pacckx:jara} \\
j\=atiya & T & M & & \pageref{pacct14:jaatiya} \\
ju & K & M & & \pageref{pacckx:ju} \\
jjha & T & MS & & \pageref{pacct13:jjha} \\
jhaka & K & M & & \pageref{pacckx:jhaka} \\
\~n\~na & T & M & & \pageref{pacct1:ynyna} \\
\d t\d tha & KT & KMS & & \pageref{pacckx:dtdtha}; \pageref{pacct12:dtdtha} \\
\d t\d thama & T & M & & \pageref{pacct12:dtdthama} \\
\d t\d tu & K & S & & \pageref{pacckx:dtdtu} \\
\d tha & KT & KMS & & \pageref{pacckx:dtha1}, \pageref{pacckx:dtha2}; \pageref{pacct12:dtha} \\
\d thakana & K & M & & \pageref{pacckx:dthakana} \\
\d da & KT & M & & \pageref{pacckx:dda}; \pageref{pacct12:dda} \\
\d d\d dha & K & KS & & \pageref{pacckx:ddddha} \\
\d dha & K & KS & & \pageref{pacckx:ddha} \\
\d na & KT & KMS & & \pageref{pacck1:dna}, \pageref{pacck4:dna}, \pageref{pacck6:dna}, \pageref{pacckx:dna1}, \pageref{pacckx:dna2}; \pageref{pacct1:dna}, \pageref{pacct2:dna}, \pageref{pacct3:dna}, \pageref{pacct5:dna}, \pageref{pacct8:dna}, \pageref{pacct10:K-dna}, \pageref{pacct10:M-dna}, \pageref{pacct11:dna} \\
\d naka & K & M & & \pageref{pacck1:dnaka}, \pageref{pacckx:dnaka} \\
\d nana & K & M & & \pageref{pacck1:dnana} \\
\d naya & AT & KMS & & \pageref{pacca:dnaya1}, \pageref{pacca:dnaya2},, \pageref{pacca:dnaya3}; \pageref{pacct1:dnaya}, \pageref{pacct2:dnaya}, \pageref{pacct14:dnaya} \\
\d nava & T & KS & & \pageref{pacct1:dnava} \\
\d n\=a & A & KS & & \pageref{pacca:dnaa} \\
\d n\=ana & T & KMS & & \pageref{pacct1:dnaana} \\
\d n\=apaya & A & KS & & \pageref{pacca:dnaapaya} \\
\d n\=api & A & M & & \pageref{pacca:dnaapi} \\
\d n\=ape & A & KS & & \pageref{pacca:dnaape} \\
\d n\=ayana & T & KMS & & \pageref{pacct1:dnaayana} \\
\d n\=ala & K & M & & \pageref{pacckx:dnaala} \\
\d ni & AKT & KMS & & \pageref{pacca:dni1}; \pageref{pacckx:dni}; \pageref{pacct1:dni} \\
\d nika & T & S & & \pageref{pacct1:dnika}, \pageref{pacct2:dnika}, \pageref{pacct11:dnika}, \pageref{pacct14:dnika} \\
\d nitta & K & KS & & \pageref{pacckx:dnitta} \\
\d nima & K & KS & & \pageref{pacckx:dnima} \\
\d niya & T & KMS & & \pageref{pacct1:dniya}, \pageref{pacct2:dniya}, \pageref{pacct8:dniya} \\
\d nisaka & K & M & & \pageref{pacckx:dnisaka} \\
\d n\=i & K & KMS & & \pageref{pacck1:dnii}, \pageref{pacck2:dnii} \\
\d nu & AK & KMS & & \pageref{pacca:dnu}; \pageref{pacck2:dnu}, \pageref{pacckx:dnu1}, \pageref{pacckx:dnu2} \\
\d nuka & K & KMS & & \pageref{pacck2:dnuka}, \pageref{pacckx:dnuka} \\
\d nuva & K & M & & \pageref{pacckx:dnuva} \\
\d n\=uka & K & M & & \pageref{pacckx:dnuuka} \\
\d ne & A & KS & & \pageref{pacca:dne1}, \pageref{pacca:dne2} \\
\d neyya & T & KMS & & \pageref{pacct1:dneyya}, \pageref{pacct2:S-dneyya}, \pageref{pacct2:M-dneyya}, \pageref{pacct8:dneyya}, \pageref{pacct11:dneyya}, \pageref{pacct14:dneyya} \\
\d nera & T & KMS & & \pageref{pacct1:dnera}, \pageref{pacct2:dnera} \\
\d nya & KT & KMS & & \pageref{pacck3:dnya}; \pageref{pacct8:dnya} \\
\d nvu & K & KS & & \pageref{pacck1:dnavu} \\
\d nh\=a & A & KS & & \pageref{pacca:dnhaa} \\
ta & KT & KS & & \pageref{pacck8:ta}, \pageref{pacckx:ta}; \pageref{pacct10:ta} \\
taka & K & M & & \pageref{pacckx:taka} \\
taggha & T & M & & \pageref{pacct2:taggha} \\
tana & KT & M & & \pageref{pacckx:tana}; \pageref{pacct2:tana} \\
tanaka & K & M & & \pageref{pacckx:tanaka} \\
tapya & K & S & & \pageref{pacck3:tapya} \\
tabba & K & KMS & & \pageref{pacck3:tabba} \\
tama & T & KMS & & \pageref{pacct9:tama} \\
tara & T & KMS & & \pageref{pacct9:tara}, \pageref{pacct14:tara} \\
tavantu & K & KS & & \pageref{pacck8:tavantu} \\
tave & K & KMS & & \pageref{pacck7:tave} \\
t\=a & T & KMS & & \pageref{pacct5:taa}, \pageref{pacct8:taa} \\
t\=aye & K & M & & \pageref{pacck7:taaye} \\
t\=av\=i & K & KS & & \pageref{pacck8:taavii} \\
ti & K & KS & & \pageref{pacck5:ti}, \pageref{pacck6:ti}, \pageref{pacckx:ti} \\
tika & K & M & & \pageref{pacckx:tika} \\
tiya & T & KS & & \pageref{pacct12:tiya} \\
tu & K & KMS & & \pageref{pacck1:tu}, \pageref{pacck2:tu}, \pageref{pacck11:tu1}, \pageref{pacck11:tu2}, \pageref{pacckx:tu} \\
tuka & K & KS & & \pageref{pacck11:tuka} \\
tuna & K & KMS & & \pageref{pacck9:tuna} \\
tu\d m & K & KMS & & \pageref{pacck7:tudm} \\
tuuna & K & KMS & & \pageref{pacck9:tuuna} \\
teyya & K & KS & & \pageref{pacck3:teyya} \\
to & NS & KMS & & \pageref{tab:indto} \\
tta & T & KMS & & \pageref{pacct8:tta} \\
ttaka & T & M & & \pageref{pacct2:ttaka} \\
ttana & T & KMS & & \pageref{pacct8:ttana} \\
tti & K & KS & & \pageref{pacckx:tti} \\
ttima & K & KS & & \pageref{pacckx:ttima} \\
ttha & T & S & & \pageref{pacct12:ttha} \\
tya & K & S & & \pageref{pacckx:tya} \\
tyu & K & S & & \pageref{pacckx:tyu} \\
tra & S & KMS & & \pageref{tab:indtra} \\
tra\d n & K & KS & & \pageref{pacckx:tradn} \\
tv\=a & K & KS & & \pageref{pacck9:tvaa} \\
tv\=ana & K & KS & & \pageref{pacck9:tvaana} \\
tha & SKT & KMS & & \pageref{tab:indtra}; \pageref{pacckx:tha}; \pageref{pacct12:tha} \\
thaka & K & M & & \pageref{pacckx:thaka} \\
thatth\=a & T & KMS & & \pageref{pacct13:thatthaa} \\
tha\d m & T & KMS & & \pageref{pacct13:thadm} \\
th\=a & T & KMS & & \pageref{pacct13:thaa} \\
thi & K & M & & \pageref{pacckx:thi} \\
thika & K & M & & \pageref{pacckx:thika} \\
th\=i & K & M & & \pageref{pacckx:thii} \\
thu & K & KS & & \pageref{pacckx:thu} \\
da & K & KS & & \pageref{pacckx:da} \\
daka & K & M & & \pageref{pacckx:daka} \\
dara & K & M & & \pageref{pacckx:dara} \\
d\=a & S & KMS & & \pageref{tab:indda} \\
d\=acana\d m & S & KMS & & \pageref{tab:indda} \\
d\=ani & S & KMS & & \pageref{tab:indda} \\
du & K & KMS & & \pageref{pacckx:du} \\
dura & K & M & & \pageref{pacckx:dura} \\
dusuka & K & M & & \pageref{pacckx:dusuka} \\
dha & SK & KMS & & \pageref{tab:indtra}; \pageref{pacckx:dha1}, \pageref{pacckx:dha2} \\
dh\=a & ST & KMS & & \pageref{pacct13:dhaa} \\
dhi & S & KMS & & \pageref{tab:indtra} \\
dhuka & K & M & & \pageref{pacckx:dhuka} \\
dhun\=a & S & KMS & & \pageref{tab:indda} \\
na & KT & M & & \pageref{pacckx:na}; \pageref{pacct10:na} \\
naka & K & M & & \pageref{pacck4:naka} \\
na\d na & T & M & & \pageref{pacct8:nadna} \\
n\=a & A & KS & & \pageref{pacca:naa} \\
neyya & T & M & & \pageref{pacct2:neyya} \\
neyyaka & T & M & & \pageref{pacct2:neyyaka} \\
ni & K & M & & \pageref{pacck6:ni}, \pageref{pacckx:ni} \\
niya & T & M & & \pageref{pacct3:niya} \\
nta & K & M & & \pageref{pacck10:nta} \\
nu & K & KS & & \pageref{pacckx:nu} \\
nuka & K & M & & \pageref{pacckx:nuka} \\
nusa & K & SK & & \pageref{pacckx:nusa} \\
pa & K & M & & \pageref{pacckx:pa} \\
paka & K & M & & \pageref{pacckx:paka} \\
p\=asa & K & M & & \pageref{pacckx:paasa} \\
ppa & A & KS & & \pageref{pacca:ppa} \\
pha & K & M & & \pageref{pacckx:pha} \\
ba & K & M & & \pageref{pacckx:ba} \\
bi & K & M & & \pageref{pacckx:bi} \\
b\=ula & K & M & & \pageref{pacckx:buula} \\
bya & T & MS & & \pageref{pacct8:bya} \\
bha & KT & M & & \pageref{pacckx:bha}; \pageref{pacct10:bha} \\
bhaka & K & M & & \pageref{pacckx:bhaka} \\
bhara & K & M & & \pageref{pacckx:bhara} \\
ma & KT & KMS & & \pageref{pacck4:ma}, \pageref{pacckx:ma1}, \pageref{pacckx:ma2}; \pageref{pacct12:ma} \\
maka & K & M & & \pageref{pacckx:maka} \\
matta & T & M & & \pageref{pacct2:matta} \\
man & K & KS & & \pageref{pacckx:man} \\
mantu & T & KMS & & \pageref{pacct10:mantu} \\
maya & T & KMS & & \pageref{pacct11:maya} \\
mara & K & M & & \pageref{pacckx:mara} \\
m\=ana & K & KMS & & \pageref{pacck10:maana} \\
m\=ara & K & M & & \pageref{pacckx:maara} \\
mi & K & M & & \pageref{pacckx:mi} \\
ya & AKT & KMS & & \pageref{pacca:ya1}, \pageref{pacca:ya2}; \pageref{pacck3:ya1}, \pageref{pacck3:ya2}, \pageref{pacck6:ya}, \pageref{pacckx:ya}; \pageref{pacct1:ya}, \pageref{pacct2:ya}, \pageref{pacct3:ya} \\
yaka & AK & M & & \pageref{pacca:yaka}; \pageref{pacck3:yaka}, \pageref{pacck6:yaka} \\
y\=a\d na & K & KS & & \pageref{pacckx:yaadna} \\
yira & A & KS & & \pageref{pacca:yira} \\
yu & K & KS & & \pageref{pacck2:yu}, \pageref{pacck4:yu}, \pageref{pacck6:yu} \\
ra & KT & KMS & & \pageref{pacck1:ra}, \pageref{pacct10:ra} \\
raka & K & M & & \pageref{pacckx:raka} \\
rati & T & M & & \pageref{pacct2:rati} \\
ratu & K & M & & \pageref{pacckx:ratu} \\
ratthu & K & KS & & \pageref{pacck11:ratthu} \\
ratya & K & S & & \pageref{pacckx:ratya} \\
rathi & K & M & & \pageref{pacckx:rathi} \\
rabha & K & M & & \pageref{pacckx:rabha} \\
ramma & K & KS & & \pageref{pacck4:ramma} \\
ravi & K & M & & \pageref{pacckx:ravi} \\
raha & S & KMS & & \pageref{tab:indda} \\
rahi & S & KMS & & \pageref{tab:indda} \\
r\=atu & K & SK & & \pageref{pacck11:raatu} \\
r\=aya & T & M & & \pageref{pacct2:raaya} \\
rika & K & M & & \pageref{pacckx:rika} \\
ricca & K & KS & & \pageref{pacck3:ricca} \\
rittaka & T & M & & \pageref{pacct2:rittaka} \\
ritu & K & KS & & \pageref{pacck11:ritu} \\
ribbisa & K & M & & \pageref{pacckx:ribbisa} \\
ririya & K & KMS & & \pageref{pacck6:ririya} \\
riva & K & M & & \pageref{pacckx:riva} \\
r\=iva & T & M & & \pageref{pacct2:riiva} \\
r\=ivataka & T & M & & \pageref{pacct2:riivataka} \\
r\=isana & K & M & & \pageref{pacckx:riisana} \\
r\=iha & K & M & & \pageref{pacckx:riiha} \\
ru & K & M & & \pageref{pacckx:ru} \\
ruka & K & M & & \pageref{pacckx:ruka} \\
ru\d na & K & S & & \pageref{pacckx:rudna} \\
r\=u & K & KMS & & \pageref{pacck2:ruu} \\
reyya\d n & T & M & & \pageref{pacct14:reyyadn} \\
reva & K & M & & \pageref{pacckx:reva} \\
ro & K & MS & & \pageref{pacck1:ro} \\
la & AKT & KMS & & \pageref{pacca:la1}, \pageref{pacca:la2}; \pageref{pacckx:la1}, \pageref{pacckx:la2}; \pageref{pacct6:la}, \pageref{pacct14:la} \\
laka & K & M & & \pageref{pacckx:laka} \\
l\=a\d na & K & KS & & \pageref{pacckx:laadna} \\
li & K & M & & \pageref{pacckx:li} \\
lika & T & S & & \pageref{pacct14:lika} \\
ltu & K & M & & \pageref{pacck1:ltu} \\
lla & T & KMS & & \pageref{pacct6:lla}, \pageref{pacct10:lla} \\
va & ST & KMS & & \pageref{tab:indtra}, \pageref{pacct10:va} \\
vantu & T & KMS & & \pageref{pacct10:vantu} \\
v\=ala & K & M & & \pageref{pacckx:vaala} \\
v\=i & T & KMS & & \pageref{pacct10:vii} \\
sa & AKT & KMS & & \pageref{pacca:sa}; \pageref{pacckx:sa}; \pageref{pacct10:sa} \\
saka & K & M & & \pageref{pacck1:saka}, \pageref{pacckx:saka1}, \pageref{pacckx:saka2} \\
sa\d na & T & M & & \pageref{pacct1:sadna}, \pageref{pacct11:sadna} \\
sara & K & M & & \pageref{pacckx:sara} \\
su & K & M & & \pageref{pacckx:su} \\
so & T & MS & & \pageref{pacct13:so} \\
ssa & T & M & & \pageref{pacct1:ssa}, \pageref{pacct14:ssa} \\
ss\=i & T & M & & \pageref{pacct10:ssii} \\
ha & SK & KMS & & \pageref{tab:indtra}; \pageref{pacckx:ha} \\
ha\d m & S & KMS & & \pageref{tab:indtra} \\
hi & K & M & & \pageref{pacckx:hi} \\
\mbox{hi\~ncana\d m} & S & KMS & & \pageref{tab:indtra} \\
hi\~nci & S & KMS & & \pageref{tab:indtra} \\
hi\d m & S & KMS & & \pageref{tab:indtra} \\
h\=i & K & M & & \pageref{pacckx:hii} \\
\d la & K & M & & \pageref{pacckx:dla} \\
\d laka & K & M & & \pageref{pacckx:dlaka} \\
\d li & K & M & & \pageref{pacckx:dli} \\
\d lu & K & M & & \pageref{pacckx:dlu} \\
\d lhaka & T & S & & \pageref{pacct8:dlhaka} \\
\end{longtable}

\section*{Some difficult \pali{paccaya}s and \pali{anubandha}s}\label{sec:anubandha}

In P\=ali, \pali{paccaya}s are overwhelming, particularly for derivations. Not only the massive number of them is difficult to handle, but also some of them have a strange behavior. This often stuns new students, if not discourages them to give up learning altogether. That happened to me long time ago. In fact, it is just a handful that you have to be aware of their strangeness. I describe some of them here to ease our learning.

I distinguish between \pali{paccaya} and \pali{anubandha}. The former is the whole chunk of them as the given names in the table. The latter is a part of them that causes certain transformation. This part is not normally seen in the final product.\footnote{A.\,K.\,Warder calls this `fictitious addition' an \emph{exponent} \citep[p.~251]{warder:intro}.} For example, \pali{\d n-anubandha} is the most used and the strangest of all. When I mention just a name, it means \pali{paccaya}, otherwise \pali{anubandha} will be shown. For \pali{anubandha}, I list only noteworthy and widely used ones. There are many of them and some of them are used differently by different schools. In Mogg, they are more extensively used.

\paragraph*{\pali{\d N-anubandha}}\label{par:dnapacc} (\pali{vuddhi} marker)

In most case when \pali{\d n} appears in the \pali{paccaya}s, it causes the base to be in \pali{vuddhi} strength, normally the first vowel of it. There are some exceptions in root-group \pali{paccaya}s that have \pali{\d n} in their body.

We can find \pali{\d na} in verb formation and secondary derivation. This \pali{paccaya} has other thing to do than just being added to the base. When used, \pali{\d n} (\pali{\d n-anubandha}) is deleted, then only \pali{a} is left.\footnote{Kacc\,396, R\=upa\,363, Sadd\,834} Furthermore, the first vowel of the base, if not followed by double consonants\footnote{Some can be (Mogg\,4.125). And \pali{vuddhi} sometimes occurs in the middle, e.g.\ \pali{a\d d\d dhateyyo}, \pali{v\=ase\d t\d tho} (Mogg\,4.126).}, has to be in \pali{vuddhi} strength (see the end of Chapter \ref{chap:nuts}).\footnote{Kacc\,400, R\=upa\,364, Sadd\,847} That is to say, \pali{a} is lengthened to \pali{\=a}, \pali{i} and \pali{\=i} to \pali{e}, and \pali{u} and \pali{\=u} to \pali{o}.\footnote{Mogg\,4.124} For example, \pali{vint\=a+\d neyya} becomes \pali{venteyya}, \pali{upadhi+\d nika} becomes \pali{opadhika}, \pali{abhidhamma+\d nika} becomes \pali{\=abhidhammika}, but \pali{suttanta+\d nika} becomes \pali{suttantika}.

That is the general rule of \pali{\d na} processing. There are some cases that do not follow this regularity. Some are very specific, for example, \pali{by\=akara\d na+\d na} = \pali{vi\=akara\d na+\d na} = \pali{veyy\=akara\d na}\footnote{Kacc\,401, R\=upa\,375, Sadd\,848--50}; \pali{sagga+\d nika} = \pali{suagga+\d nika} = \pali{sovaggika}\footnote{Sadd\,851}; \pali{ny\=aya+\d nika} = \pali{ni\=aya+ \d nika} = \pali{neyy\=anika}\footnote{Sadd\,852}; \pali{by\=avaccha+\d na} = \pali{vi\=avaccha+\d na} = \pali{veyy\=avaccha}\footnote{Sadd\,853}; \pali{dv\=ara+\d nika} = \pali{duara+\d nika} = \pali{dov\=arika}\footnote{Sadd\,854}; \pali{byaggha+\d na} = \pali{viaggha+\d na} = \pali{veyyaggha}\footnote{Sadd\,855}; \pali{isi+\d nya} = \pali{\=arissya}, \pali{usabha+\d na} = \pali{\=asabha}\footnote{Kacc\,402, R\=upa\,377, Sadd\,857}. There are also some other things (perhaps almost everything) can happen when \pali{\d na} is in operation, for instance, shortened vowels, lengthened vowels, elision, addition, transformation, and shifted \pali{vuddhi} position.\footnote{Kacc\,403--4, R\=upa\,354, 370, Sadd\,858--9, Mogg\,4.126, 4.128-30, 4.132--3, 4.139--41} Yet \pali{vuddhi} may not happen at all, e.g.\ \pali{abhidhammiko, vinteyyo, u\d lumpiko, ara\~n\~niko}.\footnote{Sadd\,860, 862} Some definitely do, e.g.\ \pali{v\=ase\d t\d tho, b\=aladevo}.\footnote{Sadd\,861} Some never do, e.g.\ \pali{n\=ilavatthiko, p\=itavatthiko}.\footnote{Sadd\,863}

\subparagraph*{\pali{K-anubandha}} (\pali{vuddhi} preventer)

The notion of \pali{\d n} as \pali{vuddhi} marker is used in all grammatical schools. But the use of its preventer is applied only in Moggall\=ana school. It makes things more precise. For example, in Kacc/Sadd \pali{ta} is used in derivation, but in Mogg it is \pali{kta}. This means appling \pali{ta} without any \pali{vuddhi}.

\subparagraph*{\pali{R-anubandha}} (last-syllable killer)

When \pali{r} appears in \pali{paccaya}s, most of the time it cause the last syllable of the base to be deleted. I call this `last-syllable killer.' For example, \pali{anta+gamu+r\=u} becomes \pali{antag\=u}\footnote{Sadd\,1118} (One who normally goes to the end).

\paragraph*{\pali{Kvi}} \ 

We will not find this ending in any words because it causes itself te be deleted (Kacc\,639, R\=upa\,585, Sadd\,1266, Mogg\,5.159), for example, \pali{vi+bh\=u+kvi = vibh\=u} (exceptional being), \pali{saya\d m+bh\=u +kvi = saya\d mbh\=u} (self creator, God), \pali{abhi+ bh\=u+kvi = abhibh\=u} (great being), \pali{sa\d m+dh\=u+kvi = sandhu/sandh\=u} (trembler), \pali{u+dh\=u +kvi = uddhu} (trembler), \pali{vi+bh\=a+ kvi = vibh\=a} (light), \pali{ni+bh\=a+kvi = nibh\=a} (ray), \pali{saha+bh\=a+kvi = sabh\=a} (assembly). Sometimes it can also cause the last consonant of roots to be deleted (Kacc\,615, R\=upa\,586, Sadd\,1220, Mogg\,5.94), for example, \pali{bhuja+gamu+kvi = bhujaga} (snake), \pali{tura+gamu+kvi = turaga} (horse), \pali{vi+yamu+kvi = viyo} (abstainer), \pali{su+mana+kvi = suma} (glad one), \pali{pari+tanu +kvi = parita} (spreader).

\paragraph*{\pali{\d Nvu}} \ 

According to Kacc/Sadd convention, the name of this \pali{paccaya} causes some confusion, unlike in Mogg it is straightly \pali{\d naka}. Apart from being \pali{vuddhi}ed by \pali{\d n-anubandha} (as the examples reveal some of them are not), this also has a particular behavior: it changes itself to \pali{aka} (Kacc\,622, R\=upa\,670, Saddd 1228), sometimes \pali{\=ananaka} (Kacc\,641, R\=upa\,572, Sadd\,1268). This \pali{paccaya} marks the agent of action, for example, \pali{nudaka} (dispeller), \pali{s\=udaka} (cook, sprinkler), \pali{janaka} (father, producer), \pali{s\=avaka} (listener, follower), \pali{l\=avaka} (cutter, reaper), \pali{h\=avaka} (honorer), \pali{p\=avaka} (cleanser, fire), \pali{bh\=avaka} (being), \pali{j\=anaka} (knower), \pali{\=asaka} (eater), \pali{up\=asaka} (near-sitter), \pali{samaka} (leveler). They can be in causative sense, for example, \pali{\=a\d n\=apaka} (commander), \pali{phand\=apaka} (tremble causer), \pali{cet\=apaka} (barterer), \pali{sa\~nj\=ananaka} (demonstrator).

\paragraph*{\pali{Ya}} (passive verb marker)

The main use of \pali{ya} is in verb (\pali{\=akhy\=ata}), but you can find some in derivation with a similar effect. It is the marker of passive verb form, but it also used in active form as a root-group \pali{paccaya}. The marked behavior of \pali{ya} is it cause the last character of the base to be duplicated with some modification. For example, \pali{budha+ya+ti} becomes \pali{bujjhati} ([One] knows). For more detail of its use, see page \pageref{pacca:ya2}. 

\paragraph*{\pali{\d Nya}} \ 

This is actually \pali{ya} with \pali{\d n-anubandha}. But some of the products are \pali{vuddhi}ed, some are not. The following examples are taken from Kacc\,638, R\=upa\,660, Sadd\,1247: \pali{pa+vaja+\d nya = pabbajj\=a} (going forth), \pali{sa\d m+aja+\d nya = samajj\=a} (assembly), \pali{ni+s\=ida+\d nya = nisajj\=a} (sitting), \pali{vi+\~n\=a+\d nya = vijj\=a} (knowing), \pali{vi+saja+\d nya = visajj\=a} (relinquishing), \pali{ni+pada+ \d nya = nipajj\=a} (sleeping), \pali{hana+\d nya = vajjh\=a/vajjha} (killing, person worth killing), \pali{s\=i+\d nya = seyy\=a} (sleeping, bed), \pali{cara+ \d nya = cariy\=a} (conduct), \pali{sada+\d nya = sajj\=a}\footnote{See also Sadd\,1263.} (ending), \pali{pada+ \d nya = pajj\=a} (attaining).

\paragraph*{\pali{Yu}} \ 

In Mogg this is equivalent to \pali{ana} that requires no further explanation. In Kacc/Sadd convention, \pali{yu} changes itself to \pali{ana} (Kacc\,622, R\=upa 670, Saddd 1228), sometimes \pali{\=ana} (Kacc\,641, R\=upa\,572, Sadd\,1268). This \pali{paccaya} can produce terms in three senses: the agent of action (some take the same meaning as \pali{\d nvu}, some are not), the state of action, and the instrument of action. The first sense has male gender generally, sometimes female depending on contexts. The last two normally are neuter. Here are some examples: \pali{s\=udana} (sprinkler, sprinkling, sprinkling tool), \pali{janana} (produced being, producing, instrument of production), \pali{savana} (listener, listening, listening tool), \pali{lavana} (reaper, reaping, reaping tool), \pali{havana} (honorer, honoring, honoring tool), \pali{pavana} (winnower, winnowing, winnowing device), \pali{bhavana} (being, state of being, cause of being), \pali{\~n\=a\d na}\footnote{For the instrumental sense it can be \pali{j\=anana}.} (knower, knowing, knowing tool), \pali{asana} (eater, eating, food), \pali{sama\d na} (tranquil one, state of tranquility, calming tool). Like \pali{\d nvu} they can be in causative sense, for example, \pali{phand\=apana} (agitation), \pali{cet\=apana} (bartering), \pali{\=a\d n\=apana} (commanding).


\section*{Some irregular products}\label{sec:irrprod}

In the following section, I list some peculiar terms under the operation of some \pali{paccaya}s. All of them are primary derivatives. For irregular verb forms, see Appendix \ref{chap:conj}, page \pageref{sec:irrverb}. To save the table space, I have to shorten the references: K = Kacc, R = R\=upa, S = Sadd, M = Mogg. Naming scheme of \pali{paccaya}s in Mogg is discarded. The list is not in a familiar order\footnote{In fact, the list is ordered roughly by sutta numbers in the textbooks. But I try to group things together, then the order is somewhat shaky.}, so you have to go through it one by one.

\bigskip
{\small
\begin{longtable}[c]{@{}>{\raggedright\arraybackslash\itshape}p{0.14\linewidth}%
	>{\raggedright\arraybackslash\itshape}p{0.11\linewidth}%
	>{\raggedright\arraybackslash\itshape}p{0.15\linewidth}%
	>{\raggedright\arraybackslash}p{0.17\linewidth}%
	>{\raggedleft\arraybackslash\footnotesize}p{0.28\linewidth}@{}}
\caption{Irregular products of \pali{paccaya}s}\label{tab:pp}\\
\toprule
\bfseries\upshape Root & \bfseries\upshape Pacc. & \bfseries\upshape \mbox{Product} & \bfseries Meaning & \bfseries\normalsize Ref. \\ \midrule
\endfirsthead
\multicolumn{5}{c}{\tablename\ \thetable: Irregular products of \pali{paccaya}s (contd\ldots)}\\
\toprule
\bfseries\upshape Root & \bfseries\upshape Pacc. & \bfseries\upshape \mbox{Product} & \bfseries Meaning & \bfseries\normalsize Ref. \\ \midrule
\endhead
\bottomrule
\ltblcontinuedbreak{5}
\endfoot
\bottomrule
\endlastfoot
%%
s\=asa & ta & si\d t\d tha & to teach & K572, R625, S1170, M5.117 \\
& ta & sattha & & M5.117, M5.144 \\
disa & ta & di\d t\d tha & to see & K572, R625, S1170 \\
& tabba & da\d t\d thabba & & S1171 \\
& tu\d m, tv\=ana & da\d t\d thu\d m & & S1172, S1174 \\
& tv\=a & di\d t\d th\=a\footnote{This is also \pali{disv\=a}. If it is followed by \pali{patta}, it becomes \pali{di\d t\d thippatta} (Sadd\,1175).} & & S1173 \\
tusa & ta & tu\d t\d tha & to be satisfied & K573, R626, S1176, M5.140 \\
& tv\=a & tu\d t\d thav\=a & & M5.140 \\
& tabba & tu\d t\d thabba & & M5.140 \\
da\d msa & ta & da\d t\d tha & to bite & K573, R626, S1176 \\
puccha & ta & pu\d t\d tha\footnote{But with \pali{tv\=a}, it becomes \pali{pucchitv\=a}.} & to ask & K573, R626, S1176, M5.143 \\
& a & pucch\=a & question & S1249 \\
bhasa & ta & bha\d t\d tha & to fall & K573, R626, S1176, M5.143 \\
hasa & ta & ha\d t\d tha & to laugh & K573, R626, S1176 \\
pa+visa & ta & pavi\d t\d tha & to enter & K573, R626, S1176 \\
yaja & ta & yi\d t\d tha\footnote{In Mogg\,5.113, this can be \pali{i\d t\d tha}.} & to honor & K573, R626, S1176, M5.143, K610, R627, S1215, M5.113 \\
& \d na & y\=aga & honoring & K623, R554, S1229 \\
& \d nvu & y\=ajaka & honorer & K618, R571, S1224 \\
kasa & ta & ki\d t\d tha ka\d t\d tha & to plough & M5.141 \\
vasa & ta & vu\d t\d tha vuttha & to live & K574, R613, S1177, K612, R615, S1217 \\
& ta & u\d t\d tha & & K575, R614, S1178 \\
budha & ta & buddha & to know & K576, R607, S1179 \\
& tv\=a & buddh\=a & & S1206 \\
& tv\=a & bujjhitv\=a & & S1211 \\
va\d d\d dha & ta & vu\d d\d dha & to grow & K576, R607, S1179 \\
& ti & va\d d\d dhi & & M5.158 \\
vaddha & ta & vuddha vaddha & to grow & M5.145, M5.112 \\
labha & ta & laddha & to gain & K576, R607, S1179, K611, R608, S1216, M5.145 \\
& tv\=ana & laddh\=ana & & S1207 \\
upa+labha & tv\=a & \mbox{upalabhitv\=a} \mbox{upaladdh\=a} & to receive & K600, R645, S1205 \\
daha & ta & da\d d\d dha & to burn & K576, R607, S1179, K612, R615, S1217, M5.146 \\
& \d na & \d l\=aha d\=aha & & K614, R581, S1219, M5.127 \\
kudha & ta & kuddha & \mbox{to be angry} & K611, R608, S1216 \\
yudha & ta & yuddha & to fight & K611, R608, S1216 \\
sidhu & ta & siddha & to succeed & K611, R608, S1216 \\
\=a+rabha & ta & \=araddha & to begin & K611, R608, S1216 \\
& tv\=a & \mbox{\=arabhitv\=a} \=araddh\=a \=arabbha & & K600, R645, S1205 \\
sa\d m+naha & ta & \mbox{sannaddha} & to fasten & K611, R608, S1216 \\
duha & ta & duddha & to milk & M5.145 \\
bahi & ta & bu\d d\d dha & to grow & M5.147 \\
\=a+ruha & ta & \=aru\d lha & to ascend & K589, R621, S1193, M5.148 \\
muha & ta & m\=u\d lha\footnote{In Mogg\,5.149, this can also be \pali{muddha}.} & \mbox{to be confused} & K589, R621, S1193, M5.149, M5.106 \\
g\=ahu & ta & g\=a\d lha & to stir & K589, R621, S1193 \\
guh\=u & ta & g\=u\d lha & to hide & M5.148, M5.106 \\
vaha & ta & v\=u\d lha & to carry & M5.148, M5.107 \\
bahi & ta & b\=a\d lha & \mbox{to increase} & M5.148, M5.106 \\
bhanja\footnote{The dictionary form of this verb is \pali{bha\~njati}. Surprisingly, there is no root described in Sadd-Dh\=a for this term, even the term is used once in Sadd-Dh\=a\,169, by the meaning of `to destroy.' If there is a root for this, however, it should be \pali{bha\~nja}, not \pali{bhanja} as given by the textbooks.} & ta & bhagga & to break & K577, R628, S1180, M5.154 \\
& \mbox{tavantu} & \mbox{bhaggavantu} & & M5.154 \\
& \d na & bha\.nga & \mbox{destruction} & K607, R578, S1212 \\
ni+mujja & ta & nimugga & \mbox{to sink down} & M5.154 \\
& \mbox{tavantu} & \mbox{nimuggavantu} & & \\
sa\d m+vida & ta & sa\d mvigga & \mbox{to be found} & M5.154 \\
& \mbox{tavantu} & \mbox{sa\d mviggavantu} & & \\
bhuja & ta & bhutta & to eat & K578, R560, S1181 \\
& \mbox{tavantu} & \mbox{bhuttavantu} & & \\
& t\=av\=i & bhutt\=av\=i & & \\
& tv\=a & bhutv\=a bhu\~njitv\=a & & S1221 \\
caja & ta & catta & to give up & K578, R560, S1181 \\
saja & ta & satta & to attach & K578, R560, S1181 \\
ranja & ta & ratta & to like & K578, R560, S1181 \\
& \d na & r\=aga & lust & K590, R579, S1194 \\
& \d na & ra\.nga & color & K607, R578, S1212 \\
yuja & ta & yutta & to put together & K578, R560, S1181 \\
vi+vica\footnote{No \pali{vica} is listed as a root in Sadd-Dh\=a.} & ta & vivitta & to seclude & K578, R560, S1181; K580, R630, S1183 \\
vaca & ta & vutta\footnote{In Mogg\,5.110--1, this can also be \pali{vu\d t\d tha} or \pali{utta/u\d t\d tha}.} & to say & K579, R629, S1182, M5.110--1 \\
& tv\=a & vivicca & & K598, R643, S1203 \\
su+gupa & ta & sugutta & \mbox{to protect well} & K580, R630, S1183 \\
cinta & ta & citta & to think & K580, R630, S1183 \\
lipa & ta & litta & to smear & K580, R630, S1183 \\
tara & ta & ti\d n\d na & to cross & K581, R616, S1184, M5.153 \\
& \mbox{tavantu} & \mbox{ti\d n\d navantu} & & M5.153 \\
p\=ura & ta & pu\d n\d na & to fill & M5.152 \\
& \mbox{tavantu} & \mbox{pu\d n\d navantu} & & \\
sa\d m+p\=ura & ta & \mbox{sampu\d n\d na} & to fill & K581, R616, S1184 \\
pari+p\=ura & ta & paripu\d n\d na & to be full & K581, R616, S1184 \\
jara & ta & ji\d n\d na & to age & M5.153 \\
& \mbox{tavantu} & \mbox{ji\d n\d navantu} & & \\
pari+jara & ta & \mbox{pariji\d n\d na} & to decay & K581, R616, S1184 \\
kira\footnote{No \pali{kira} is listed as a root in Sadd-Dh\=a.} & ta & ki\d n\d na & to scatter & M5.152 \\
& \mbox{tavantu} & \mbox{ki\d n\d navantu} & & \\
\=a+kira & ta & \=aki\d n\d na & to scatter & K581, R616, S1184 \\
cara & ta & ci\d n\d na & \mbox{to practice} & M5.153 \\
& \mbox{tavantu} & \mbox{ci\d n\d navantu} & & \\
kh\=i & ta & kh\=i\d na & \mbox{to exhaust} & K582, R631, S1185, M5.152 \\
& \mbox{tavantu} & \mbox{kh\=i\d navantu} & & M5.152 \\
bhidi & ta & bhinna & to break & K582, R631, S1185, M5.150 \\
& tabba & \mbox{bhettabba} & & M5.95 \\
& \mbox{tavantu} & \mbox{bhinnavantu} & & M5.150 \\
chidi & ta & chinna & to cut & K582, R631, S1185, M5.150 \\
& \mbox{tavantu} & \mbox{chinnavantu} & & M5.150 \\
d\=a & ta & dinna & to give & K582, R631, S1185, M5.151 \\
& \mbox{tavantu} & \mbox{dinnavantu} & & M5.151 \\
& \d nvu & d\=ayaka & giver & K593, R564, S1197, M5.91 \\
ni+s\=ida\footnote{In Sadd-Dh\=a\,377, \pali{s\=ida} is listed as a root, but in Kacc\,609, R\=upa\,484, and Sadd\,1213 it is supposed to be \pali{sada} transformed to \pali{s\=ida} (see also Sadd\,1214). In Mogg\,5.123, the root is \pali{sada} but with \pali{\=i} insertion.} & ta & nisinna & to sit & K582, R631, S1185 \\
chada & ta & channa & to cover & M5.150 \\
& \mbox{tavantu} & \mbox{channavantu} & & \\
su+chada & ta & suchanna & \mbox{to cover well} & K582, R631, S1185 \\
khidi & ta & khinna & to suffer & K582, R631, S1185 \\
ruda & ta & ru\d n\d na & to cry & K582, R631, S1185 \\
u+pada & ta & uppanna & to arise & M5.150 \\
& tv\=a & \mbox{uppajjitv\=a} uppajja & & K600, R645, S1205, S1211 \\
& \mbox{tavantu} & \mbox{uppannavantu} & & \\
ni+pada & tabba & \mbox{nipajjitabba} & \mbox{\ to lie down} & M5.92 \\
& tu\d m & \mbox{nipajjitu\d m} & & \\
susa & ta & sukkha & \mbox{to make dry} & K583, R617, S1186, M5.155 \\
& \mbox{tavantu} & \mbox{sukkhavantu} & & M5.155 \\
paca & ta & pakka & to cook & K583, R617, S1186, M5.156 \\
& \mbox{tavantu} & \mbox{pakkavantu} & & M5.156 \\
& \d na & p\=aka & cooking & K623, R554, S1229; K640, S1267 \\
& \d nvu & p\=acaka & cooker & K618, R571, S1224 \\
muca & ta & mukka mutta & to release & M5.157 \\
& \mbox{tavantu} & \mbox{mukkavantu} \mbox{muttavantu} & & \\
pa+kamu & ta & pakkanta & \mbox{to go away} & K584, R618, S1187 \\
sa\d m+kamu & ta & sa\.nkanta & to join & K584, R618, S1187 \\
vi+bhama\footnote{No \pali{bhama} or \pali{bhamu} (rotate) is listed as a root in Sadd-Dh\=a.} & ta & \mbox{vibbhanta} & \mbox{to go astray} & K584, R618, S1187 \\
khamu & ta & khanta\footnote{This can be a noun as \pali{khanti} (patience) (Sadd\,1188), also in the same way \pali{santi} (peace), \pali{kanti} (desire).} & to endure & K584, R618, S1187 \\
samu & ta & santa & to calm & K584, R618, S1187 \\
damu & ta & danta & to tame & K584, R618, S1187 \\
nata & ta & nacca na\d t\d ta & to dance & S1166 \\
ni+dh\=a & ta & nihita & to deposit & M5.108 \\
& \mbox{tavantu} & \mbox{nihitavantu} & & \\
jan\=i & ta & j\=ata & \mbox{to be born} & K585, R619, S1189, M5.116 \\
& ti\footnote{Other \pali{paccaya}s apart from \pali{ta} and \pali{ti} do not lengthen \pali{a} to \pali{\=a}, hence \pali{janitv\=a, janit\=a, janitu\d m, janitabba\d m}.} & j\=ati & birth & K585, R619, S1189 \\
gamu & ta & gata & to go & K586, R600, S1190, M5.109 \\
& ta & gamita & & K617, R633, S1223 \\
& ti & gati & going & \\
& tu\d m & gantu\d m gamitu\d m & & K596, R551, S1200 \\
& tabba & gantabba gamitabba & & \\
& tuna & gantuna & & \\
& tv\=ana & gantv\=ana & & \\
\=a+gamu & tv\=a & \=agamitv\=a \=agamma & to come & K600, R645, S1205 \\
khanu & ta & khata & to dig & K586, R600, S1190, M5.109 \\
& ti & khati & digging & \\
& tu\d m & khantu\d m \mbox{khanitu\d m} & & K596, R551, S1200 \\
& tabba & khantabba \mbox{khanitabba} & & M5.96\\
hana & ta & hata & to hurt & K586, R600, S1190, M5.109 \\
& ti & hati & hurting & \\
& tu\d m & hantu\d m hanitu\d m & & K596, R551, S1200; K617, R633, S1223 \\
& tabba & hantabba \mbox{hanitabba} & & \\
& tv\=a & hantv\=a & & S1203 \\
& \d na & gh\=ata & & K591, R544, S1195, M5.99 \\
& \d na & vadha & & K592, R503, S1196 \\
\=a+hana & tv\=a & \=ahacca & & K598, R643, S1203, M5.166 \\
& tv\=a & \=ahanitv\=a & & M5.166 \\
& \d na & \=agh\=ata & & K591, R544, S1195, M5.99 \\
ramu & ta & rata & to enjoy & K586, R600, S1190, M5.109 \\
& ta & ramita & & K617, R633, S1223 \\
& ti & rati & enjoying & \\
mana & ta & mata & to know & K586, R600, S1190, M5.109 \\
& ti & mati & knowing & \\
& tu\d m & mantu\d m manitu\d m & & K596, R551, S1200 \\
& tabba & mantabba \mbox{manitabba} & & \\
kara & ta & kata & to do & K587, R632, S1191, M5.109 \\
& tave & k\=atave & & K595, R637, S1199, M5.118 \\
& tu\d m & k\=atu\d m kattu\d m & & K595, R637, S1199, M5.119; K620, R549, S1226 \\
& tuna & k\=atuna kattuna & & \\
& tabba & k\=atabba kattabba & & K620, R549, S1226, M5.119 \\
& tv\=a & katv\=a & & S1203 \\
& tv\=a & karitv\=a\footnote{For \pali{i} insertion, see Kacc\,605, R\=upa\,547, Sadd\,1210, Mogg\,5.170.} & & K617, R633, S1223 \\
& tv\=a & kacca & & K598, R643, S1203, M5.167 \\
& m\=ana & kar\=a\d na\footnote{In Kacc\,655, R\=upa\,650, Sadd\,1293, this instance is a product of \pali{\=ana}.} \mbox{kurum\=ana} & & M5.162 \\
& tu & kattu & doer & K619, R573, S1225 \\
& \d nvu & k\=araka & doer & K622, R570, S1228, M5.84 \\
pa+kara & ta & pakata & to do & K587, R632, S1191 \\
& ti & pakati & \mbox{natural state} & \\
pura+kara & ta & \mbox{purakkhata} & to put in front & K594, R582, S1198, M5.134 \\
sa\d m+kara & ta & sa\.nkhata & to prepare & K594, R582, S1198 \\
sa\d m+kara & \d na & sa\.nkh\=ara & thing conditioned & K594, R582, S1198, M5.133 \\
upa+kara & ta & \mbox{upakkhata}\footnote{This is more often found as \pali{upakkha\d ta}.} & {\ to put together} & K594, R582, S1198 \\
pari+kara & \d na & parikkh\=ara & accessory & K594, R582, S1198 \\
sara & ta & sata & \mbox{to remember} & K587, R632, S1191 \\
& ta & sarita & & K617, R633, S1223 \\
& ti & sati & \mbox{mindfulness} & \\
\d th\=a & ta & \d thita & to stand & K588, R620, S1192, M5.114 \\
& ti & \d thiti & stability & \\
p\=a & ta & p\=ita & to drink & K588, R620, S1192, M5.115 \\
& ti & p\=iti & joy & \\
ge & ta & g\=ita & to sing & M5.115 \\
& ti & g\=iti & singing & \\
sa\d m+ge & ti & sa\.ng\=iti & rehearsal \mbox{(recite together)} & M5.115 \\
abhi+vanda & {\ tv\=a} & \mbox{abhivanditv\=a} \mbox{abhivandiya} & {\ \ to salute} & K597, R641, S1201 \\
o+h\=a & tv\=a & ohitv\=a oh\=aya & to give up & K597, R641, S1201 \\
upa+n\=i & tv\=a & upanetv\=a upan\=iya & to carry away & K597, R641, S1201 \\
disa & tv\=a & passitv\=a passiya & to see & K597, R641, S1201, M5.169 \\
& & disv\=a & & K599, R644, S1204, M5.169 \\
u+disa & tv\=a & uddisitv\=a uddissa & \mbox{to point out} & K597, R641, S1201 \\
\=a+d\=a & tv\=a & \=adiyitv\=a \=ad\=aya & to grasp & K597, R641, S1201 \\
abhi+bh\=u & tv\=a & \mbox{abhibhavitv\=a} \mbox{abhibh\=uya} & \mbox{\ \ to overcome} & M5.164 \\
anu+bh\=u & tuna & \mbox{anubhavituna} \mbox{anubhaviy\=ana} & \mbox{\ \ \ to undergo} & S1202 \\
abhi+hara & tv\=a & \mbox{abhiharitv\=a} \mbox{abhiha\d t\d thu\d m} & to bring & M5.165 \\
anu+muda & tv\=a & \mbox{anumoditv\=a} \mbox{anumodiy\=ana} & \mbox{\ to appreciate} & M5.165 \\
ni+pata & tv\=a & nipacca & \mbox{to fall down} & K598, R643, S1203 \\
adhi+i & tv\=a & adhicca & to study & M5.168 \\
& tv\=a & adh\=iyitv\=a & & \\
sa\d m+i & tv\=a & samecca & to meet & M5.168 \\
& tv\=a & sametv\=a & & \\
o+kamu & tv\=a & okkamitv\=a okkamma & to enter & K600, R645, S1205 \\
gaha & \d na & ghara\footnote{Aggava\d msa disagrees that this should be from \pali{ghara} (to sprinkle) rather than a transformation of \pali{gaha}.} & house & K613, R583, S1218 \\
pa+gaha & tv\=a & \mbox{pagga\d nhitv\=a} paggayha & \mbox{to hold up} & K600, R645, S1205 \\
vidha & tv\=a & viddh\=a & to pierce & S1206 \\
nanda & yu & nandana & rejoicing & K622, R570, S1228 \\
sanja & \d na & sa\.nga & to cling & K607, R578, S1212 \\
\end{longtable}
}
