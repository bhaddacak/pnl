\chapter{I \headhl{go} to school}\label{chap:verb-go}

In this chapter, we will learn about another common verb which is used very often in conversations. It also appears frequently in the scriptures. The verb is \pali{gacchati} `to go.' In English, we use preposition `to' to mark the destination of going. In P\=ali, it has no use of such a preposition. In fact, it has no individual word that acts like proposition.\footnote{The closest word class in P\=ali that has prepositional function as English is \pali{upasagga}, a kind of indeclinables (see Appendix \ref{chap:upasagga}). Usually, \pali{upasagga} is used as prefix to modify the meaning of verbs and nouns. In few cases, \pali{upasagga} stands alone as a separate word, so it can look like prepositions in English \citep*[see][p.~125]{collins:grammar}.} How to mark the destination then? The answer is in another case of declension---\emph{accusative}.

\phantomsection
\addcontentsline{toc}{section}{Conjugation of Present Tense}
\section*{Conjugation of Present Tense}

Before we talk about accusative case, it is a proper time to introduce the rule of present tense conjugation of common verbs, which is shown in Table \ref{tab:conjpres}. In Chapter \ref{chap:verb-be} we met verbs `to be' in their ready-to-use forms. Here we learn the general formula that can be used with most regular verbs.

\begin{table}[!hbt]
\centering
\caption{Endings of present tense conjugation}
\label{tab:conjpres}
\bigskip
\begin{tabular}{l*{2}{>{\itshape}l}} \toprule
\bfseries Person & \bfseries\upshape Singular & \bfseries\upshape Plural \\ \midrule
3rd & ti & nti \\
2nd & si & tha \\
1st & mi & ma \\
\bottomrule
\end{tabular}
\end{table}

To make a verb present tense, including present continuous tense, we add corresponding endings to its stem form. For regular verbs, we can find their stem forms in dictionaries, which normally list verbs by their canonical form---\emph{present-tense, 3rd-person, singular, active-voice}. For example, `to go' has its dictionary form as `\pali{gacchati}.' We can derive stem form of the verb by a reversed process---removing `\pali{ti}' at the end, then we get `\pali{gaccha}.'\footnote{In fact, it is the stem plus certain ending, \pali{a} in this case, that can be varied according to the group of verb's root. Learning verbs from roots, like the tradition does, is difficult. Learning them from stem forms is much easier. For the traditional account of verb formation, see Chapter \ref{chap:vform}.} Once we get the stem form, we append it with the endings provided. An additional rule for present 1st person conjugation is if the final vowel of the stem form is \pali{a}, lengthen it to \pali{\=a} (\pali{a}$\rightarrow$\pali{\=a} + \pali{mi/ma}).\footnote{Kacc\,478, R\=upa\,438, Sadd\,959, Mogg\,6.57, Niru\,567.} But for 3rd person plural, if the final vowel is long, shorten it, e.g.\ \pali{\=a}$\rightarrow$\pali{a} + \pali{nti}. If the final vowel is \pali{o} or \pali{e}, retain it.

Let us see an example for better understanding. The stem form of `to go' is \pali{gaccha}. Therefore, ``I go'' is \pali{gacch\=ami}, ``We go'' \pali{gacch\=ama}, ``You go'' (sg.) \pali{gacchasi}, ``You go'' (pl.) \pali{gacchatha}, ``He/She/It goes'' \pali{gacchati}, and ``They go'' \pali{gacchanti}. A benefit of learning verb `to go' in P\=ali is you get verb `to come' for free---just prefix it with \pali{\=a} as \pali{\=agacchati}. Everything goes with \pali{gacchati} goes with \pali{\=agacchati} as well.

\phantomsection
\addcontentsline{toc}{section}{Declension of Accusative Case}
\section*{Declension of Accusative Case}

Second to the nominative, accusative case is also the most used declension. The main function of this case is to mark the direct object of transitive verbs. The \emph{object} here has a wider sense than we use in English, as it can be used with `to go.' Table \ref{tab:accreg} summarizes the case endings of regular nouns, including adjectives. The general symbol of acc.\ is \pali{niggah\=ita} (\pali{\d m}). You only have to remember the singular forms, be careful with the highlighted. The plural forms of accusative case are the same as nominatives, except m.\ pl.\ with \pali{a} ending.

\begin{table}[!hbt]
\centering
\caption{Accusative case endings of regular nouns}
\label{tab:accreg}
\bigskip
\begin{tabular}{@{}>{\bfseries}l*{5}{>{\itshape}l}@{}} \toprule
\multirow{2}{*}{G. Num.} & \multicolumn{5}{c}{\bfseries Endings} \\
\cmidrule(l){2-6}
& a & i & \=i & u & \=u\\
\midrule
m. sg. & a\d m & i\d m & \replacewith{\=i}{i\d m} & u\d m & \replacewith{\=u}{u\d m} \\
& & & \texthl{\replacewith{\=i}{ina\d m}} & & \\
m. pl. & \texthl{\replacewith{a}{e}} & \replacewith{i}{\=i} & \=i & \replacewith{u}{\=u} & \=u \\
& & \replacewith{i}{ayo} & \replacewith{\=i}{ino} & \replacewith{u}{avo} & \replacewith{\=u}{uno} \\
\midrule
nt. sg. & a\d m & i\d m &  & u\d m & \\
nt. pl. & \replacewith{a}{\=ani} & \replacewith{i}{\=ini} & & \replacewith{u}{\=uni} & \\
& & \replacewith{i}{\=i} & & \replacewith{u}{\=u} & \\
\midrule
& \=a & i & \=i & u & \=u\\
\midrule
f. sg. & \replacewith{\=a}{a\d m} & i\d m & \replacewith{\=i}{i\d m} & u\d m & \replacewith{\=u}{u\d m} \\
& & & \texthl{\replacewith{\=i}{iya\d m}} & & \\
f. pl. & \=a & \replacewith{i}{\=i} & \=i & \replacewith{u}{\=u} & \=u \\
& \=ayo & iyo & \replacewith{\=i}{iyo} & uyo & \replacewith{\=u}{uyo} \\
\bottomrule
\end{tabular}
\end{table}

We have to learn accusative case of pronouns at this time, for it can be very useful in conversations. Table \ref{tab:accpron} shows declension of both demonstrative and personal pronouns we have learned so far.

\begin{table}[!hbt]
\centering
\caption{Accusative case of pronouns}
\label{tab:accpron}
\bigskip
\begin{tabular}{@{}*{7}{>{\itshape}l}@{}} \toprule
\multirow{2}{*}{\bfseries\upshape Pron.} & \multicolumn{2}{c}{\bfseries\upshape m.} & \multicolumn{2}{c}{\bfseries\upshape f.} & \multicolumn{2}{c}{\bfseries\upshape nt.} \\
\cmidrule(lr){2-3} \cmidrule(lr){4-5} \cmidrule(lr){6-7} 
& \bfseries\upshape sg. & \bfseries\upshape pl. & \bfseries\upshape sg. & \bfseries\upshape pl. & \bfseries\upshape sg. & \bfseries\upshape pl. \\
\midrule
amha & \texthl{ma\d m} & amhe & & & & \\
& \texthl{mama\d m} & no & & & & \\
tumha & tva\d m & tumhe & & & & \\
& tuva\d m & vo & & & & \\
& ta\d m & & & & & \\
ta & ta\d m & te & ta\d m & t\=a & ta\d m & t\=ani \\
& na\d m & ne & na\d m & & na\d m & \\
eta & eta\d m & ete & eta\d m & et\=a & eta\d m & et\=ani\\
& ena\d m & & ena\d m & & ena\d m & \\
ima & ima\d m & ime & ima\d m & im\=a & \texthl{ida\d m} & im\=ani \\
& & & & & ima\d m & \\
amu & amu\d m & am\=u & amu\d m & am\=u & \texthl{adu\d m} & am\=uni \\
\bottomrule
\end{tabular}
\end{table}

Now you can say ``I go to school'' as follows:

\palisample{aha\d m p\=a\d thas\=ala\d m gacch\=ami. \textup{\normalsize(sg.)}}

Alternatively, \pali{sippas\=ala\d m} can do the same job. To be precise, \pali{p\=a\d thas\=al\=a} is the place to learn reading and writing (\pali{p\=a\d tha} = text reading) as general schools do, whereas \pali{sippas\=al\=a} looks more like a school of art or craft (= \pali{sippa}). Generally, the two words can be used interchangeably, because our school system normally incorporates both. And here is for ``We go to school.'' Be careful with the subject and verb agreement.

\palisample{maya\d m p\=a\d thas\=ala\d m gacch\=ama. \textup{\normalsize(pl.)}}

These are for ``You go to school,'' in singular and plural sense.

\palisample{tva\d m p\=a\d thas\=ala\d m gacchasi. \textup{\normalsize(sg.)}\sampleor[and]tumhe p\=a\d thas\=ala\d m gacchatha. \textup{\normalsize(pl.)}}

And the last ones for ``He/she goes to school'' and ``They go to school.''

\palisample{so/s\=a p\=a\d thas\=ala\d m gacchati. \textup{\normalsize(sg.)}\sampleor[and]te/t\=a p\=a\d thas\=ala\d m gacchanti. \textup{\normalsize(pl.)}}

When adjectives are used, they have to take the same case, i.e.\ acc., of the object of \pali{gacchati}. For example, ``I go to a big school'' can be said as:

\palisample{aha\d m mahanta\d m p\=a\d thas\=ala\d m gacch\=ami.}

If you find the verb `to go' understandable, there should be no problem with `to come.' So, ``I come home'' can be easy as:

\palisample{aha\d m geha\d m \=agacch\=ami.}

In P\=ali scriptures, we often find that gen.\ (or dat., as well as loc.), rather than acc., is used to mark the object or destination of the action. So, you can say in this way as well:

\palisample{aha\d m gehassa \=agacch\=ami.}

In practice, I suggest that it is better to stick with acc.\ if you have no good reason to use its alternative. Have fun with our exercise before leaving.

\section*{Exercise \ref{chap:verb-go}}
Say these in P\=ali.
\begin{compactenum}
\item It is a train over there. It goes to the station.
\item This temple has virtuous monks. People go here.
\item You go to a big market. It has a lot of goods.
\item That forest has many trees. I go to that beautiful place.
\item We go to a park with many flowers.
\end{compactenum}
