\chapter{Now, I \headhl{can} speak P\=ali}\label{chap:inf}

As we have learned so far, you should remember that in P\=ali verb system there are tenses (present, past, future), and moods (imperative, optative, conditional). With these forms we can say many things. But what is obviously missing is the expression of ability, as we normally use `can' in English. In this chapter we will address this and some more things.

\phantomsection
\addcontentsline{toc}{section}{Introduction to Infinitives}
\section*{Introduction to Infinitives}

What we will learn from now is called `infinitive' by scholars. Technically speaking, infinitive means ``A verb form denoting an action, process or state not limited to particular participants or a particular time.''\footnote{\citealp[p.~227]{brownmiller:dict}} Together with participle, infinitive is non-finite verb that does not give information of tense, person, and number. In P\=ali, there are some \pali{kita} forms that can be classified as infinitive, i.e.\ verbs in \pali{tu\d m, tave,} and \pali{t\=aye} form (see page \pageref{par:kitatum}). We mostly see \pali{tu\d m} in the texts. Forming \pali{tu\d m} verbs is relatively easy, but some irregular instances has to be remembered though (see page \pageref{sec:irrprod}).

Infinitives in P\=ali can be used in a variety of ways. They can be used in both active and passive structure. For the latter sense, you can see an instrumental actor as a marker. For some explanation, see \citealp[pp.~134--6]{warder:intro}. I will show some common practice of these with examples from the texts.

\paragraph*{Using with `be suitable'} Terms to be used in this sense are several. Some are verbs, e.g.\ \pali{arahati}, \pali{kappati}, and \pali{va\d t\d tati}. Some are used as adjectives, e.g.\ \pali{yutta}, \pali{anucchavika}, \pali{kalla}. Some take indeclinable form, e.g.\ \pali{arah\=a}, \pali{anur\=upa\d m}, \pali{ala\d m} (also `enough'). Here are examples:\par
- \pali{Na arahati bhava\d m so\d nada\d n\d do sama\d na\d m gotama\d m dassan\=aya upasa\.nkamitu\d m.}\footnote{D1\,303 (DN\,4)} (It is not suitable for So\d nada\d n\d da to approach ascetic Gotama to see [him].)\par
- \pali{Na ta\d m arahati sappa\~n\~no, manas\=a anukampitu\d m}\footnote{S1\,236 (SN\,10)} (It is not suitable for a wise person to be moved by mind in that [matter].)\par
- \pali{Ki\d m nu kho, \=avuso, kappati evar\=upa\d m k\=atu\d m}\footnote{Buv1\,234} (Is it suitable, venerable, to do as such?)\par
- \pali{na kappati gu\d lo vik\=ale paribhu\~njitu\d m}\footnote{Mv\,6.272} (Sugar is not suitable to eat in wrong time.)\par
- \pali{amhehi pam\=adac\=ara\d m caritu\d m na va\d t\d tati}\footnote{Dhp-a\,3.35. Using \pali{va\d t\d tati} in this sense is rare in the canon. In commentaries it is widely used, but in an idiomatic way. When this verb comes with \pali{tu\d m}, passive structure is normally used. That is why we see instrumental actor here.} (Practicing carelessly by us is not suitable.)\par
- \pali{bhikkhun\=a n\=ama k\=ay\=ad\=ini rakkhitu\d m va\d t\d tati}\footnote{Dhp-a\,17.231} (Protecting the body, etc., by a monk is suitable.)\par
- \pali{buddhas\=asane n\=ama ida\d m k\=atu\d m va\d t\d tati, ida\d m na va\d t\d tati}\footnote{Dhp-a\,3.36} (Is doing this in Buddhism suitable, or not?)\par
- \pali{Kicch\=a vutti no itar\=itareneva, yutta\d m cintetu\d m satatamaniccata\d m}\footnote{Thag\,1.111} (Our livelihood is difficult, [so] it is suitable to think about impermanence constantly.)\par
- \pali{deva, sace imasmi\d m k\=ara\d ne da\d n\d da\d m gahetu\d m yutta\d m, ga\d nhatha}\footnote{Dhp-a\,4.58} (Your Majesty, if monetary penalty is suitable in this case, impose it on me.)\par
- \pali{anucchaviko bhava\d m dh\=ana\d m pa\d tiggahetu\d m.}\footnote{R\=upa\,638} (You are suitable to receive alms.)\par
- \pali{kalla\d m nu tena tadabhinanditu\d m}\footnote{D2\,128 (DN\,15)} (Is it worth rejoicing by that?)\par
- \pali{Ya\d m pan\=anicca\d m dukkha\d m vipari\d n\=amadhamma\d m, kalla\d m nu ta\d m samanupassitu\d m -- eta\d m mama, esohamasmi, eso me att\=ati}\footnote{Mv\,1.21} (Which nature is impermanent, unbearable, and changing, is that suitable to see that [nature] as ``This is mine, I am this, this is myself''?)\par
- \pali{arah\=a tva\d m vattu\d m.}\footnote{Kacc\,637} (You are suitable to say)\footnote{Perhaps, this means like ``You have the right to say that'' or ``You should say that.''}\par
- \pali{ida\d m k\=atu\d m anur\=upa\d m}\footnote{R\=upa\,638} (This [action] is suitable to do.)\par
- \pali{Ala\d m samakkh\=atu\d m saddhammassa}\footnote{D3\,173 (DN\,29)} (Enough to announce the true teaching)\par
- \pali{ala\d m k\=atu\d m ala\d m sa\d mvidh\=atu\d m}\footnote{A8\,76} (Suitable to do, suitable to arrange)\par

\paragraph*{Using with `be able to'} A common verb to use in this sense is \pali{sakkoti}.\footnote{Kacc\,562, R\=upa\,638, Sadd\,1149; Kacc\,637, Sadd\,1246} Sometimes particle \pali{sakk\=a} is used instead. Another term having the same meaning is \pali{bhabba}.\footnote{By its form, it is said to be a future passive participle \citep[p.~111]{collins:grammar}. In Kacc\,543, R\=upa\,555, and Sadd\,1128, it is the product of \pali{bh\=u + \d nya}. Its meaning is equal to \pali{bhavitabbo}. See also page \pageref{pacck3:dnya}.} This is used like an adjective. Sometimes the distinction between `be suitable to' and `be able to' is not clear. In some contexts, they can be used interchangeably. And sometimes they all are more or less equal to `be possible.' Another verb rarely found in this use is \pali{pahoti}.\par
- \pali{Gil\=an\=a n\=ama bhikkhun\=i na sakkoti ov\=ad\=aya v\=a sa\d mv\=as\=aya v\=a gantu\d m.}\footnote{Buv2\,161} (A nun [who] is not able to go for instruction or meeting is called sick [person].)\par
- \pali{na sakkhissasi y\=avaj\=iva\d m paripu\d n\d na\d m parisuddha\d m brahmacar\-iya\d m caritu\d m}\footnote{Buv1\,38} ([You] will not be able to practice the religious life completely and purely.)\par
- \pali{Na c\=api mantayuddhena, sakk\=a jetu\d m dhanena v\=a.}\footnote{S1\,136 (SN\,3)} (One cannot win [death] even by spell-battling or by wealth.)\par
- \pali{Imesa\d m pana, br\=ahma\d na, pa\~ncanna\d m a\.ng\=ana\d m sakk\=a eka\d m a\.nga\d m \d thapayitv\=a cat\=uha\.ngehi samann\=agata\d m br\=ahma\d n\=a br\=ahma\d na\d m pa\~n\~napetu\d m}\footnote{D1\,311 (DN\,4)} (In these five qualities, brahman, (if) one quality has been set aside, are brahmans (still) able to declare one endowed with four qualities as brahman?)\par
- \pali{puriso s\=isacchinno abhabbo tena sar\=irabandhanena j\=ivitu\d m}\footnote{Buv1\,55} (A person, having the head cut, is not able to live with that head tied to the body.)\par
- \pali{bhabbo nu kho, bhante, m\=atug\=amo \ldots arahattaphala\d m v\=a sacchik\=atu\d m?}\footnote{Cv\,10.402} (Is it possible, sir, that a woman [going forth] \ldots is able to realize the arhant result, etc.?)\par
- \pali{pahoti c\=ayasm\=a mah\=akacc\=ano imassa bhagavat\=a sa\d mkhittena uddesassa uddi\d t\d thassa vitth\=arena attha\d m avibhattassa vitth\=arena attha\d m vibhajitu\d m.}\footnote{M3\,280 (MN\,133)} (Ven.\,Mah\=akacc\=ayana is able to explain succinctly the meaning given by the Buddha comprehensively, [and] explain thoroughly the meaning which is not.)\par

\paragraph*{Using with \pali{labbh\=a}} This is an idiomatic use. Here \pali{labbh\=a} is indeclinable meaning `possible' or `allowable' or `may be obtained.'\par
- \pali{Labbh\=a, t\=ata sudinna, h\=in\=ay\=avattitv\=a bhog\=a ca bhu\~njitu\d m pu\~n\-\~n\=ani ca k\=atu\d m.}\footnote{Buv1\,34} (It is possible [to you], Sudinna, [when] having disrobed, to enjoy the wealth and make merit.)\par
- \pali{may\=a ca na labbh\=a ekik\=aya vatthu\d m, a\~n\~n\=aya ca bhikkhuniy\=a na labbh\=a d\=arakena saha vatthu\d m, katha\d m nu kho may\=a pa\d tipajjitabba\d m}\footnote{Cv\,10.432. Note carefully on this passive structure, when \pali{labbh\=a} is used. This form can happen to \pali{sakk\=a} as well.} (Living alone is not possible to be done by me. Living with the child is not possible to be done by other nun (either). How should be done by me?)\par

\paragraph*{Using with `to want'} We can say that someone wants to do something by using \pali{icchati} or similar verbs with infinitives.\par
- \pali{ayy\=a icchati teka\d tulay\=agu\d m p\=atu\d m}\footnote{Buv1\,157} (The venerable wants to drink rich-gruel with three ingredients.)\par
- \pali{bhikkhu \=apatti\d m \=apajjitv\=a na icchati \=apatti\d m passitu\d m}\footnote{Mv\,9.415} (A monk, having transgressed an offense, does not want to see the offense.)\par
- \pali{icch\=amaha\d m, bhante, kesamassu\d m oh\=aretv\=a k\=as\=ay\=ani vatth\=ani acch\=adetv\=a ag\=arasm\=a anag\=ariya\d m pabbajitu\d m.}\footnote{Buv1\,25. Using \pali{tv\=a} verbs here is noteworthy. They give us a sense of order.} (I, sir, want to shave hair and beard, wear yellow robes, [then] go forth from household to homelessness.)\par

\paragraph*{Using with `to intend'} I find that \pali{ma\~n\~nati} (to think, to deem) can be used in this sense, for example:\par
- \pali{So tva\d m, bhante, tena lesena d\=ar\=uni adinna\d m haritu\d m ma\~n\~nasi!}\footnote{Buv1\,88} ([What I mean is] you, sir, intend to take these ungiven pieces of wood by that trick.)\par
- \pali{handa maya\d m, \=avuso, gih\=ina\d m kammanta\d m adhi\d t\d thema, eva\d m te amh\=aka\d m d\=atu\d m ma\~n\~nissanti.}\footnote{Buv1\,193} (Let us, venerables, undertake the work of householders, so that they will consider giving [food] to us.)\par

\paragraph*{Using as a noun} Sometimes in English, infinitives can be a noun, like ``to err is human.'' In P\=ali, we can also use in that way. Moreover, an equivalent infinitive can be used alternatively to an action noun in dative case, for example, instead of using \pali{dassan\=aya} we can roughly use \pali{passitu\d m}. This use is the general case of some other uses mentioned earlier and below, because as a noun infinitives can be a patient (object) of other verbs. Here are examples in both forms:\par
- \pali{Janetti y\=api te m\=at\=a, na ta\d m iccheyya \textbf{passitu\d m}}\footnote{Ja\,16:184} (Even the mother who bore you might not want to see you.)\par
- \pali{Ak\=alo kho, \=avuso, bhagavanta\d m \textbf{dassan\=aya}, pa\d tisall\=ino bhagav\=a}\footnote{D1\,360 (DN\,6)} (It is not a [proper] time, Venerable, to see the Blessed One. He has withdrawn into seclusion.)\par
- \pali{na sukar\=a u\~nchena paggahena \textbf{y\=apetu\d m}}\footnote{Buv1\,30} (To support oneself with alms is not easy.)\par
- \pali{Yo vo may\=a pi\d n\d dap\=ato anu\~n\~n\=ato, ala\d m vo so y\=avadeva imassa k\=ayassa \d thitiy\=a \textbf{y\=apan\=aya}}\footnote{D3\,182 (DN\,29)} (Which food was allowed for you [all] by me, that food is enough as much for sustaining this body, for supporting oneself.)\par

\paragraph*{Using as a modifier} If terms in \pali{tu\d m} form can be used as a noun, logically it can be used as a modifier in dative sense, for example:\par
- \pali{k\=alo bhu\~njitu\d m}\footnote{R\=upa\,638. This is equal to \pali{bhu\~njan\=aya k\=alo}.} (time to eat)\par
- \pali{p\=alibh\=asa\d m sikkhitu\d m potthaka\d m}\footnote{This is equivalent to \pali{p\=alibh\=as\=aya sikkh\=aya potthaka\d m}.} (a book for learning P\=ali)\par

\paragraph*{Using with other verbs} As all illustrations go, it is reasonable that we can use \pali{tu\d m} with other verbs if its meaning allows, like we do in English. Here are some examples that I can think of:\par
- \pali{d\=atu\d m vattu\~nca labhati.}\footnote{R\=upa\,638} ([One] gets to give and to say.)\par
- \pali{anuj\=an\=ami, bhikkhave, m\=atug\=amassa chappa\~ncav\=ac\=ahi dhamma\d m desetu\d m.}\footnote{Buv1\,61} (I allow you, monks, to teach the Dhamma to a woman with 5--6 words.)\par
- \pali{Anuj\=an\=ami, bhikkhave, t\=ani pa\~nca bhesajj\=ani k\=ale pa\d tiggahetv\=a k\=ale paribhu\~njitu\d m}\footnote{Mv\,6.260} (I allow you, monks, to take in time those five medicines which having been received in time.)\par
- \pali{Sabbakammajahassa bhikkhuno, \ldots\ Attho natthi jana\d m lapetave}\footnote{Ud\,3.21} (For a monk who discards all actions, there is no use to ask people [for help].)\par
- \pali{nadi\d m gamiss\=ama sin\=ayitu\d m}\footnote{M2\,283 (MN\,81)} (Let us go to the river to bathe.)\par
- \pali{aha\d m p\=alibh\=asa\d m sikhitu\d m p\=a\d thas\=ala\d m gacch\=ami.} (I go to school to study P\=ali.)\par
- \pali{aha\d m tva\d m j\=an\=apetu\d m ima\d m likh\=ami.} (I write this to make you know.)\par

\paragraph*{Using with other particles} Not only do certain verbs require infinitives, some particles, or terms functioning as an adverb, are also found being accompanied with infinitives, apart from the frequently found ones already mentioned above such as \pali{sakk\=a}.\par
- \pali{\textbf{atippago} kho t\=ava s\=avatthiya\d m pi\d n\d d\=aya caritu\d m}\footnote{M1\,163 (MN\,13)} (It is too early to go for alms in S\=avatth\=i.)\par

\paragraph*{Using in compounds} Without the final nasal consonant, verbs in \pali{tu\d m} can be found in compounds. As far as I know, \pali{k\=ama} (desire) is found as a part in compounds.\footnote{\citealp[p.~374]{perniola:grammar}}\par
- \pali{bhagav\=a kira s\=avatthi\d m gantuk\=amo}\footnote{Cv\,10.410} (The Buddha [is one who] wishes to go to S\=avatth\=i.)\par
- \pali{Tena kho pana samayena a\~n\~nataro sattho r\=ajagah\=a pa\d tiy\=aloka\d m gantuk\=amo hoti.}\footnote{Buv2\,231. In PTSD, \pali{pa\d tiy\=aloka} means `the south.' But in the commentary (Sp2\,407), it means the direction against the sun, \pali{pa\d tiy\=alokanti s\=uriy\=alokassa pa\d timukha\d m}, hence the west.} (In that time, there is another caravan wishing to go from R\=ajagaha to the west.)\par
- \pali{Upasampanno upasampanna\d m khu\d msetuk\=amo vambhetuk\=amo ma\.nkukattuk\=amo h\=inena h\=ina\d m vadeti}\footnote{Buv2\,16} (An ordained person, who wishes to scold, to scorn, to humiliate [another] ordained person, speaks to the other with humiliating speech.)\par
- \pali{Atha kho ajakal\=apako yakkho bhagavato bhaya\d m chambhitatta\d m lomaha\d msa\d m upp\=adetuk\=amo yena bhagav\=a tenupasa\.nkami}\footnote{Ud\,1.7} (Then demon Ajakal\=apaka, who wishes to frighten the Buddha, approached to where he [stayed].)\par

\bigskip
Now you are ready to finish this chapter by doing our task. Saying ``Now, I can speak P\=ali'' is simple as:

\palisample{id\=ani aha\d m p\=alibh\=asa\d m bh\=asitu\d m sakkomi.}

Or to use \pali{sakk\=a}, it is fashionable to put it at the beginning to stress the meaning.

\palisample{sakk\=a aha\d m id\=ani p\=alibh\=asa\d m bh\=asitu\d m.}

\section*{Exercise \ref{chap:inf}}
Say these in P\=ali.
\begin{compactenum}
\item When I know P\=ali enough, is it possible to find the ultimate truth in the canon?
\item It is impossible.
\item Why not?
\item First, any ultimate truth, or whatever you mean by that, is not in the letters, or any signifying action. It is like a finger pointing to the moon.
\item It is miserable to hear that.
\item And second, how are you sure what you read is authentic?
\item Wasn't the canon well-preserved?
\item Yes, it was well-preserved once an edition is done. Before the compilation we cannot know for sure. Monks remembered different things even from the same event, like you see in headlines today.
\item At least, there must be an intention to preserve the real teaching.
\item In a way, it is true, and I think so. But, do you remember that in the canon itself it is said that the teaching would last only 500 years if a woman was ordained.? If not, it was just 1,000 years.\footnote{Cv\,10.403}
\item Is it not 5,000 years?
\item That number exists only in Buddhists belief and hope.\footnote{From Theravada's evidence, the process of disappearance of the teaching is described in \emph{Manorathapur\=a\d n\=i}, the commentary to A\.nguttaranik\=aya (Mnp1\,130). There are five stages of disappearing (\pali{pa\~nca antaradh\=an\=ani}), one thousand years each.} If you trust the authenticity of the canon, why do you believe in later explanation rather than in the canon?
\item That sounds depressing. What is the use of P\=ali then?
\item It is not quite so depressing. It indeed liberates us from the attachment. All scriptures should be studied, but not to be clung on to. The knowledge of P\=ali can liberate you from false belief.
\item That means I have to read it all by myself.
\item It is not necessary to do so. We have many translations so far. You can read them. With knowledge of P\=ali, you can uncover the hidden intention (agenda) of the texts as well as of the translators. That is a way to go in P\=ali studies in modern era.
\item I see. It seems there are many thing to do in the field.
\item It is not enough to just translate text in P\=ali studies. It has to be more critical and analytical.
\end{compactenum}
