\chapter{I go to school \headhl{in town}}\label{chap:loc}

We have two remaining cases to talk about. Here is the last substantial one. We are going to learn how to mark points in space and time where or when the action occurs. It is called \emph{locative} case. We normally use this a lot in conversations.

\phantomsection
\addcontentsline{toc}{section}{Declension of Locative Case}
\section*{Declension of Locative Case}

As the name implied, this case indicates the location of the action in dimensions of space and time. In English we use prepositions to achieve this function, mainly `in', `on' and `at.' The meaning of the location can be in both literal and figurative sense. Table \ref{tab:locreg} shows locative declension of regular nouns.

\begin{table}[!hbt]
\centering\small
\caption{Locative case endings of regular nouns}
\label{tab:locreg}
\bigskip
\begin{tabular}{@{}>{\bfseries}l*{5}{>{\itshape}l}@{}} \toprule
\multirow{2}{*}{G. Num.} & \multicolumn{5}{c}{\bfseries Endings} \\
\cmidrule(l){2-6}
& a & i & \=i & u & \=u\\
\midrule
m. sg. & asmi\d m & ismi\d m & \replacewith{\=i}{ismi\d m} & usmi\d m & \replacewith{\=u}{usmi\d m} \\
& amhi & imhi & \replacewith{\=i}{imhi} & umhi & \replacewith{\=u}{umhi} \\
& \texthl{\replacewith{a}{e}} & & & & \\
m. pl. & \replacewith{a}{esu} & \replacewith{i}{\=isu} & \=isu & \replacewith{u}{\=usu} & \=usu \\
\midrule
nt. sg. & asmi\d m & ismi\d m & & usmi\d m & \\
& amhi & imhi & & umhi & \\
& \texthl{\replacewith{a}{e}} & & & & \\
nt. pl. & \replacewith{a}{esu} & \replacewith{i}{\=isu} & & \replacewith{u}{\=usu} & \\
\midrule
& \=a & i & \=i & u & \=u \\
\midrule
f. sg. & \=aya & iy\=a & \replacewith{\=i}{iy\=a} & uy\=a & \replacewith{\=u}{uy\=a} \\
& \texthl{\=aya\d m} & \texthl{iya\d m} & \texthl{\replacewith{\=i}{iya\d m}} & \texthl{uya\d m} & \texthl{\replacewith{\=u}{uya\d m}} \\
f. pl. & \=asu & \replacewith{i}{\=isu} & \=isu & \replacewith{u}{\=usu} & \=usu \\
\bottomrule
\end{tabular}
\end{table}

Locative case is one in a few cases that have distinct endings. Especially the plural ending `\pali{su}' is unique and easy to recognize. Among indistinct inflected forms of f.\ sg.\ nouns, locative cases have a noticeable difference---the `\pali{a\d m}' ending. This pattern can be found also in the declension of locative case of pronouns shown in Table \ref{tab:locpron}.

\begin{table}[!hbt]
\centering
\caption{Locative case of pronouns}
\label{tab:locpron}
\bigskip
\begin{tabular}{@{}*{5}{>{\itshape}l}@{}} \toprule
\multirow{2}{*}{\bfseries\upshape Pron.} & \multicolumn{2}{c}{\bfseries\upshape m./nt.} & \multicolumn{2}{c}{\bfseries\upshape f.} \\
\cmidrule(lr){2-3} \cmidrule(lr){4-5}
& \bfseries\upshape sg. & \bfseries\upshape pl. & \bfseries\upshape sg. & \bfseries\upshape pl. \\
\midrule
amha & mayi & amhesu & mayi & amhesu \\
tumha & tayi & tumhesu & tayi & tumhesu \\
ta & tasmi\d m & tesu & t\=aya\d m & t\=asu \\
& tamhi & & tassa\d m &\\
& asmi\d m & & assa\d m & \\
eta & etasmi\d m & etesu & etassa\d m & et\=asu \\
& etamhi & & etissa\d m & \\
ima & imasmi\d m & imesu & imissa\d m & im\=asu \\
& imamhi & & assa\d m & \\
& asmi\d m & & & \\
amu & amusmi\d m & am\=usu & amussa\d m & am\=usu \\
& amumhi & & & \\
\bottomrule
\end{tabular}
\end{table}

Now we can say ``I go to school in town'' as:

\palisample{aha\d m nagarasmi\d m p\=a\d thas\=ala\d m gacch\=ami.\sampleor aha\d m nagaramhi p\=a\d thas\=ala\d m gacch\=ami.\sampleor[or more often] aha\d m nagare p\=a\d thas\=ala\d m gacch\=ami.}

For time marking, we can say ``Today I go to school in the morning'' as:

\palisample{aha\d m ajja pabh\=atasmi\d m p\=a\d thas\=ala\d m gacch\=ami.}

For `in the morning' you can also use its equivalent \pali{pubba\d nhasmi\d m} and other ending variations. The word `today' (\pali{ajja}) is normally used as indeclinable, hence the declension is not applied. You simply use as it is. We will talk more about indeclinables later. Be careful of modifiers; they have to take the same case as the noun they modified. And the obvious subject `\pali{aha\d m}' can be left out, because it is really not necessary, grammatically speaking. So, practically we say ``This morning I go to school'' as:

\palisample{imasmi\d m pabh\=atasmi\d m p\=a\d thas\=ala\d m gacch\=ami.}

We can mix place and time together as ``This morning I go to school in town.''

\palisample{imasmi\d m pabh\=atasmi\d m nagarasmi\d m p\=a\d thas\=ala\d m gacch\=ami.}

When composing a sentence, you may use different cases to convey the same idea. For example, you may change the sentence by using dative case, ``I go to town for school.'' 

\palisample{nagara\d m gacch\=ami p\=a\d thas\=al\=aya.}

Loc.\ also has other uses. Like gen.\ it can be used in the phrase ``In those, \ldots'' or ``Among those, \ldots'' For example, ``In those people, she is great'' can be said as:

\palisample{etesu janesu s\=a mahant\=a hoti.}

Like ins., abl., and gen., loc.\ can also be used to mark a cause of the action. For example, ``I have a big house because of (my) fortune'' can be:

\palisample{(mama) dhanesu mayha\d m mahanta\d m geha\d m atthi.}

Apart from acc.\ and gen., loc.\ sometimes marks the object or the destination of the action. So, to say ``I go home'' these sentences are equivalent.

\palisample{ag\=ara\d m gacch\=ami.\sampleor ag\=arassa gacch\=ami.\sampleor ag\=arasmi\d m gacch\=ami.}

It is better to use a more specific verb if you want to emphasize the manner of going. For example, \pali{pavisati} `to enter' sounds right in the sentence ``I go into a house.'' So, it is proper to say:

\palisample{ag\=arasmi\d m pavis\=ami.}

To finish the exercise below, we have to know some place-related and time-related words. I collected these in Appendix \ref{vocab:noun}. Please find unknown words there. I have some remark on months. Some months are formed as a compound ending with \pali{m\=asa} to make them unambiguous. You can also do this with other months. It is worth knowing that months in P\=ali are based on lunar calendar, so they only fit approximately to the modern months, around half a month shifted forwards. Now try this exercise.

\section*{Exercise \ref{chap:loc}}
Say these in P\=ali.
\begin{compactenum}
\item We sit on chairs in a room of our school.
\item You drive a car on that street to a market town.
\item I live in a country in a big continent.
\item Farmers work on their field in rainy season.
\item In winter leaves fall from trees.
\item December has good weather.
\item In (all) seasons, trees of spring are beautiful.
\end{compactenum}
