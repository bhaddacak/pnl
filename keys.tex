\chapter{Answer Keys}\label{chap:keys}

Every exercise has its key. It is highly reccomended to use these keys after you take an effort to tackle the exercises. Some answers have an additional comment or explanation. These will make sense only when you understand the difficulty of the matter. In early chapters, I try to present alternative words as many as possible, separated by slashes (/). It is a bit annoying, but helpful to new students.

\section*{Exercise \ref{chap:nom}}
\begin{multicols}{2}
\RaggedRight
\begin{answerkey}
\item rukkho. taru.
\item rukkh\=a. tar\=u. taravo.
\item sarab\=u.
\item sarab\=u. sarabuyo.
\item hatth\=i. kar\=i.
\item hatth\=i. hatthino. kar\=i. karino.
\item bh\=as\=a.
\item bh\=as\=a. bh\=as\=ayo.
\item nh\=ar\=u. nh\=aru. \\(nah\=ar\=u. nah\=aru.)
\item nh\=ar\=u. nh\=aruno. nh\=aravo. (nah\=ar\=u. nah\=aruno. nah\=aravo.)
\item sammajjan\=i.
\item sammajjan\=i. \\sammajjaniyo.
\item rajju.
\item rajj\=u. rajjuyo.
\item indadhanu.
\item indadhan\=uni. \\indadhan\=u.
\item a\d t\d thi.
\item a\d t\d th\=ini. a\d t\d th\=i.
\item asani.
\item asan\=i. asaniyo.
\item n\=a\d likera\d m.
\item n\=a\d liker\=ani.
\item s\=uci.
\item s\=uc\=i. s\=ucayo.
\item ka\d tacchu.
\item ka\d tacch\=u. ka\d tacchavo.
\item selo. p\=as\=a\d no. sil\=a.
\item sel\=a. p\=as\=a\d n\=a. sil\=a. sil\=ayo.
\item \=av\=aso. niv\=aso. \=alayo. geha\d m. ghara\d m. ag\=ara\d m.
\item \=av\=as\=a. niv\=as\=a. \=alay\=a. geh\=ani. geh\=a. ghar\=ani. ghar\=a. ag\=ar\=ani. ag\=ar\=a.
\end{answerkey}
\end{multicols}

\section*{Exercise \ref{chap:adj}}
\begin{answerkey}
\item dukkar\=a bh\=as\=a.
\item taru\d no/b\=alo hatth\=i.
\item bahuk\=a sarab\=u/sarabuyo.
\item sur\=up\=a/sundar\=a itth\=i/itthiyo.
\item sobhan\=ani/bh\=asur\=ani akkh\=i/akkh\=ini. sobhan\=ani/bh\=asur\=ani cakkh\=u/cakkh\=uni.
\item kiso bh\=iruko sunakho.
\item mahant\=a garuk\=a sel\=a/p\=as\=a\d n\=a. mahant\=a garuk\=a sil\=a/sil\=ayo.
\item pa\~n\~nav\=a k\=aru\d niko \=acariyo.
\item sandar\=ani/sur\=up\=ani ratt\=ani/lohit\=ani pupph\=ani/kusum\=ani.
\item s\=igho d\=igho/\=ayato dh\=umaratho.
\end{answerkey}

\section*{Exercise \ref{chap:pron-demon}}
\begin{answerkey}
\item eso aggi.
\item asu/asuk\=a vijju.
\item te jan\=a.
\item eso/aya\d m hatth\=i th\=ulo. so ucco.\footnote{In Chapter \ref{chap:yata} we will learn that a proper way to put this is ``\pali{yo eso hatth\=i th\=ulo, so ucco}.''}
\item aya\d m utu u\d nh\=a. so gimh\=ano.\footnote{A better version is ``\pali{y\=a aya\d m utu u\d nh\=a, so gimh\=ano}.''}
\item et\=a sarab\=u/sarabuyo bahuk\=a. t\=a asundar\=a/vir\=up\=a.\footnote{A better version is ``\pali{y\=a et\=a sarab\=u bahuk\=a, t\=a asundar\=a}.''}
\item ime s\=igh\=a pas\=u/pasavo ass\=a/hay\=a.
\item am\=uni/asuk\=ani bahuk\=ani phal\=ani. (am\=uni/asuk\=ani phal\=ani bahuk\=ani.)
\item aya\d m mahallako puriso pa\~n\~nav\=a. (aya\d m puriso mahallako pa\~n\~nav\=a.)
\item et\=a taru\d n\=a videsik\=a/vij\=atik\=a ka\~n\~n\=a(yo) sur\=up\=a/sundar\=a.
\end{answerkey}

\section*{Exercise \ref{chap:pron-person}}
\begin{answerkey}
\item tumhe ar\=i/arayo p\=ap\=a mahant\=a.
\item tva\d m puriso ucco sur\=upo kusalo.
\item maya\d m mahant\=a cam\=u/sen\=a pabal\=a v\=ir\=a/nibbhay\=a.
\item (ye) ete jan\=a bhikkh\=u/bhikkhavo. te kis\=a dubbal\=a.
\item (ya\d m) ida\d m vatthu mahaggha\d m. so n\=ilo a\d n\d d\=ak\=aro ma\d ni.\footnote{It might be better to rephrase the sentence to ``This precious gem is blue, oval.'' Hence ``\pali{aya\d m mahaggho ma\d ni n\=ilo a\d n\d d\=ak\=aro}.''}
\end{answerkey}

\section*{Exercise \ref{chap:verb-be}}
\begin{answerkey}
\item Mozart-n\=amo\footnote{See some treatment on name in Chapter \ref{chap:nom}, page \pageref{par:foreignname}.} mahanto v\=adako hoti/bhavati/atthi.
\item maya\d m v\=a\d nij\=a homa/bhav\=ama/amha/asma pabal\=a dhanavanto/dhanavant\=a.
\item tumhe kapa\d n\=a/y\=acak\=a hotha/bhavatha/attha mahallak\=a dubbal\=a da\d lidd\=a.
\item aha\d m mahiso homi/bhav\=ami/amhi/asmi k\=a\d lo mahanto ghoro/ca\d n\d do/d\=aru\d no.
\item tva\d m thoko k\=i\d to hosi/bhavasi/asi vir\=upo n\=ico appaggho.
\end{answerkey}

\section*{Exercise \ref{chap:gen}}
\begin{answerkey}
\item mayha\d m/amha\d m/mama/mama\d m ida\d m dhana\d m atthi.
\item tuyha\d m/tumha\d m/tava sundar\=a/sur\=up\=a a\.nguliyo/a\.ngul\=i santi.
\item im\=asa\d m dha\~n\~n\=ana\d m/subhag\=ana\d m itth\=ina\d m analas\=a/\\atandit\=a s\=amino/s\=am\=i santi.
\item etesa\d m ma\d n\d d\=uk\=ana\d m th\=ul\=a mahant\=ani akkh\=ini/cakkh\=uni santi.
\item imesa\d m rukkh\=ana\d m/tar\=una\d m phal\=ani santi bahuk\=ani. t\=ani etesa\d m jan\=ana\d m honti.
\item mayha\d m/amha\d m/mama/mama\d m bh\=at\=a\footnote{This term (\pali{bh\=atu}) has irregular declension, see page \pageref{decl:pitu}.}/anujo atthi, bhagin\=i/anuj\=a natthi.
\end{answerkey}

\section*{Exercise \ref{chap:irrn}}
\begin{answerkey}
\item aya\d m sundaro candim\=a rasmiv\=a/jutim\=a/bh\=a\d num\=a hoti.
\item tuyha\d m c\=agava(n)t\=i m\=at\=a saddh\=ava(n)t\=i hoti. 
\item imassa yuvassa/yuvino r\=ajino/ra\~n\~no gu\d nav\=a/s\=ilav\=a mano atthi.
\item mama je\d t\d thabh\=atu/je\d t\d thabh\=atuno/je\d t\d thabh\=atussa sakh\=a dhanav\=a/vasum\=a hoti.
\item mama m\=atu/m\=atuy\=a/m\=atussa bhaginiy\=a\footnote{Or you can use \pali{m\=atucch\=aya} instead of \pali{m\=atu bhaginiy\=a}.} bhatt\=a balav\=a hoti.
\end{answerkey}

\section*{Exercise \ref{chap:verb-go}}
\begin{answerkey}
\item asu/amu dh\=umaratho hoti. so dh\=umarathanivattana\d m gacchati.
\item imassa vih\=arassa sus\=il\=a/gu\d navant\=a/s\=ilavant\=a bhikkh\=u santi. jan\=a ima\d m gacchanti.
\item tva\d m mahanta\d m \=apa\d na\d m gacchasi. tassa bahuk\=ani \\bha\d n\d d\=ani santi.
\item etassa ara\~n\~nassa/vanassa bahuk\=ani rukkh\=ani santi. \\aha\d m sundara\d m ta\d m gacch\=ami.
\item maya\d m bahupupph\=ar\=ama\d m\footnote{This is a reason why compounds are widely used in P\=ali. They make things easier. To learn more about compounds, see Appendix \ref{chap:samasa}.} gacch\=ama.
\end{answerkey}

\section*{Exercise \ref{chap:abl}}
\begin{answerkey}
\item (aha\d m) mama g\=amasm\=a/g\=amamh\=a/g\=am\=a vijj\=alaya\d m gacch\=ami.
\item eso mah\=aratho tass\=a gehasm\=a/geh\=a amh\=aka\d m nagara\d m \=agacchati.
\item tesa\d m da\d liddehi ra\d t\d thehi, bahuk\=a videsik\=a kammak\=ar\=a Ame\-rica-desa\d m\footnote{See some treatment for foreign country and city names in Sentence No.\,\ref{conv:bangkok}, page \pageref{conv:bangkok}.} gacchanti.
\item ete th\=ul\=a jan\=a \=arogyasm\=a ta\d m \=arogyas\=ala\d m gacchanti.
\item asundarasm\=a/vir\=upasm\=a tumhe nah\=apitas\=ala\d m \\gacchatha.
\item imehi bi\d l\=alehi eso s\=ukaro garuko hoti.
\end{answerkey}

\section*{Exercise \ref{chap:ins}}
\begin{answerkey}
\item aha\d m ka\d n\d nehi su\d n\=ami, cakkh\=uhi pass\=ami, mukhena bhu\~nj\=ami.
\item aha\d m tay\=a vin\=a vas\=ami\footnote{To live here means to dwell not to subsist, so \pali{vasati} or \pali{viharati} is the proper word, not \pali{j\=ivati}.} da\d liddena.
\item dh\=umarathena et\=a itthiyo t\=asa\d m g\=amasm\=a ta\d m nagara\d m gacchanti.
\item aha\d m bahuk\=ani vatth\=uni ki\d n\=ami etasm\=a v\=a\d nijasm\=a mayha\d m m\=ulena.
\item te tesa\d m cakkh\=uhi ida\d m sundara\d m r\=upa\d m passanti.
\item aha\d m sah\=ayehi saddhi\d m/saha naccas\=ala\d m gacch\=ami mama khuddakena rathena.
\item tva\d m kusal\=a \=acariy\=an\=i hatthena mahanta\d m rukkha\d m \\harasi tava sissehi bahukehi kum\=arehi saddhi\d m/saha.
\end{answerkey}

\section*{Exercise \ref{chap:dat}}
\begin{answerkey}
\item tva\d m se\d t\d th\=i kassak\=aya/kassakattha\d m/kassakassa vatthu\d m desi/dad\=asi.
\item aha\d m mama k\=aya\d m har\=ami may\=a saddhi\d m mama hit\=aya/ atth\=aya.
\item da\d liddasm\=a g\=amasm\=a ete kammak\=ar\=a dhan\=aya ta\d m \\nagara\d m \=agacchanti.
\item \=arogyas\=al\=ana\d m\footnote{Genitive meaning is better.} vejj\=a tesa\d m sippena kamma\d m karonti bahu\-k\=ana\d m jan\=ana\d m \=arogy\=aya/\=arogyattha\d m/\=arogyassa.
\item mahantassa bhojan\=ag\=arassa\footnote{Genitive case is used.} s\=ud\=a bhojan\=ani pacanti im\=aya p\=a\d thas\=al\=aya siss\=ana\d m.
\end{answerkey}

\section*{Exercise \ref{chap:loc}}
\begin{answerkey}
\item maya\d m amh\=aka\d m p\=a\d thas\=al\=aya gabbhasmi\d m/gabbhamhi/ gabbhe p\=i\d thesu nis\=id\=ama.
\item tva\d m t\=aya\d m/tassa\d m racch\=aya/racch\=aya\d m rathena\footnote{It is better to use `car' as instrumental. So, the sentence is reformed as ``You drive to a market town on that street by car.''} \\nigama\d m s\=aresi/gacchasi.
\item aha\d m mahantasmi\d m/mahantamhi/mahante \\mah\=ad\=ipasmi\d m/mah\=ad\=ipamhi/mah\=ad\=ipe\footnote{You can use gen.\ here as ``\pali{mahantassa mah\=ad\=ipassa}'' denoting ``of a big continent.''}\\ra\d t\d thasmi\d m/ra\d t\d thamhi/ra\d t\d the vas\=ami.
\item kassak\=a vass\=anasmi\d m/vass\=anamhi/vass\=ane tesa\d m \\ked\=arasmi\d m/ked\=aramhi/ked\=are kamma\d m karonti.
\item hemantasmi\d m/hemantamhi/hemante pa\d n\d n\=ani \\rukkhehi/rukkhebhi patanti.
\item m\=agasirasmi\d m/m\=agasiramhi/m\=agasire\footnote{It is alright to use gen.\ too if you want to go literal.} sundaro utu atthi.
\item ut\=usu vasantassa rukkh\=a sundar\=a honti.
\end{answerkey}

\section*{Exercise \ref{chap:vockim}}
\begin{answerkey}
\item ka\d m purisa\d m sallapasi. (acc.)\\kassa purisassa sallapasi. (gen.)
\item ko raccha\d m/v\=ithi\d m tarati, kena saddhi\d m.
\item s\=a kasmi\d m ida\d m vatthu\d m ki\d n\=ati. (loc.)\\s\=a kasm\=a ida\d m vatthu\d m ki\d n\=ati. (abl., better)
\item (tva\d m) kena mah\=arathena p\=a\d thas\=ala\d m gacchasi.
\item (tva\d m) kena/kasm\=a/kasmi\d m ajja p\=a\d thas\=ala\d m na \\gacchasi.
\item kimatth\=aya/kassa/kena/kasm\=a/kasmi\d m ta\d m potthaka\d m pa\d thanti.
\item kasm\=a tiracch\=an\=a bh\=ayasi.\footnote{Idiomatically, \pali{bh\=ayati} takes ablative case (see Chapter \ref{chap:abl}).} tiracch\=anesu kasm\=a bh\=ayasi.\footnote{In (all) animals, what do you fear?}
\item kassa mittena saddhi\d m naccas\=ala\d m gacchasi?
\item imasmi\d m k\=ale katha\d m/kena tava j\=ivita\d m pavattati.
\item j\=an\=asi k\=idiso tuyha\d m sampar\=ayo.\footnote{When a verb is put at the beginning, it can mark a yes-no question (see Chapter \ref{chap:ques}).}
\end{answerkey}

\section*{Exercise \ref{chap:yata}}
\begin{answerkey}
\item ya\d m potthaka\d m tuyha\d m hoti, ta\d m pa\d th\=ami.\footnote{You might be tempted to put it simply as ``\pali{tuyha\d m potthaka\d m pa\d th\=ami}.'' This sentence is not good because of ambiguity. It can also mean ``I read a book for you.''}
\item yasmi\d m mama m\=at\=apitaro vasanti, tasmi\d m aha\d m vas\=ami.
\item ya\d m \=acariyo vadati, ta\d m siss\=a vadanti.
\item yasm\=a ra\d t\d th\=a s\=a \=agacchati, tasm\=a tva\d m \=agacchasi.
\item tva\d m mayha\d m ya\d m ratha\d m desi, tena nagara\d m gacch\=ami.
\item yassa mahanta\d m geha\d m atthi, coro tassa ratha\d m coreti.
\end{answerkey}

\section*{Exercise \ref{chap:ind-intro}}
\begin{answerkey}
\item aha\d m ta\d m kum\=ari\d m pucch\=ami `kinn\=am\=as\=i'ti.
\item amh\=aka\d m nagarassa kammantas\=al\=a atthi dhan\=ag\=ar\=ani ca, \=arogyas\=al\=a pana naccas\=al\=a v\=a natthi.
\item mama d\=urabh\=asanayanta\d m na upalabh\=ami, coro ta\d m \=ad\=ati v\=a ta\d m vinassati v\=a.
\item \=acariyo p\=a\d thas\=ala\d m gacchati d\=arakehi saddhi\d m \\mah\=arathena v\=a, mittena saddhi\d m rathena v\=a.
\item bi\d l\=alo v\=a sunakho v\=a ima\d m k\=acatumba\d m bhindati, na aha\d m tva\d m ca d\=arak\=a v\=a.
\end{answerkey}

\section*{Exercise \ref{chap:adjcomp}}
\begin{answerkey}
\item aha\d m tay\=a dha\~n\~nataro/dha\~n\~niyo/dha\~n\~nisiko [m.] \\(dha\~n\~natar\=a/dha\~n\~niy\=a/dha\~n\~nisik\=a [f.]) homi, so puriso pana dha\~n\~natamo/dha\~n\~ni\d t\d tho hoti. \\aha\d m tay\=a uttaro dha\~n\~no [m.] (uttar\=a dha\~n\~n\=a [f.]) homi, so puriso pana uttamo dha\~n\~no hoti.
\item mama je\d t\d thabh\=at\=a may\=a baliyo hoti. \\aha\d m tasm\=a ka\d niyo/ka\d niy\=a homi.
\item so kiso s\=ukaro th\=ulatam\=a/th\=uli\d t\d th\=a bi\d l\=al\=a \\garukataro/garukiyo/garukisiko hoti. \\so kiso s\=ukaro uttam\=a th\=ul\=a bi\d l\=al\=a uttaro karuko hoti.
\item satim\=a kha\d no mahagghatamo/mahagghi\d t\d tha/uttamo \\mahaggho k\=alo hoti amh\=aka\d m j\=ivite.\footnote{It is, perhaps, better to say ``\pali{satimantassa kha\d no \ldots}'' (A moment of a mindful one \ldots).}
\item p\=alibh\=as\=a pa\d than\=a sall\=apena sukaratar\=a hoti. \\p\=alibh\=as\=a sall\=apena sukaratar\=a hoti na pa\d thanena.\footnote{``P\=ali is easier by conversation not by reading.'' (This sentence is easier to understand.) Also, Antonio Costanzo suggests \pali{suggahitatar\=a} for \pali{sukaratar\=a}. The meaning is clearer but the word is a bit advanced.}
\end{answerkey}

\section*{Exercise \ref{chap:past}}
\begin{answerkey}
\item kasm\=a hiyyo p\=a\d thas\=ala\d m na agacchi/agaccho?
\item mama rogo \=asi, aha\d m ca/pi \=arogyas\=ala\d m agacchi\d m.
\item vejjo tuyha\d m ki\d m vadi?
\item so mayha\d m \=arocesi ``na patir\=upa\d m hoti p\=a\d thas\=ala\d m \\gamana\d m'' iti.
\item ak\=asi tuyha\d m gehasmi\d m sikkha\d m? \\ki\d m tuyha\d m gehasmi\d m sikkha\d m ak\=asi?\footnote{Putting a verb at the beginning can form a yes-no question. Or you can put \pali{ki\d m} at the beginning, but this can make the sentence ambiguous because \pali{ki\d m} can be seen as a modifier of other words. For more detail on questioning, see Chapter \ref{chap:ques}.}
\item so vejjo puna ca vadi ``seyya\d m sayana\d m'' iti.\footnote{Here, \pali{puna} means `again.' Hence, \pali{puna ca} means like `also.' Antonio Costanzo suggests this translation: \pali{``sayane setu\d m seyyo''ti bhisakko kathesipi.} This makes use of an infinitive which we have not yet learned.}
\end{answerkey}

\section*{Exercise \ref{chap:fut}}
\begin{answerkey}
\item kasmi\d m sve gamissasi?
\item sve \=apa\d nasmi\d m nav\=ani vatth\=ani ki\d niss\=ami.
\item tava bahuk\=ani \=asi. kassa t\=ani lacchasi/labhissasi.
\item mama bhaginiy\=a t\=ani dass\=ami. s\=a nav\=ani vatth\=ani \\icchi, tass\=a \=apa\d nasmi\d m ki\d n\=anassa k\=alo pana natthi.\footnote{Formed by primary derivation, \pali{ki\d n\=ana} is a product of \pali{yu} or \pali{ana} (see Appendix \ref{chap:kita}, page \pageref{pacck4:yu}). The term is an action noun meaning `buying.'}
\item piy\=ayissanti t\=ani tuyha\d m bhagin\=i? \\(rucchissanti t\=ani tuyha\d m bhaginiy\=a?)\footnote{See Chapter \ref{chap:dat} for the use of \pali{ruccati} (satisfy, delight). This verb takes a dative object.}
\item \=ama, maya\d m sama\d m/samena \=ak\=arena niv\=asema. \\s\=a t\=ani acch\=adessati.\footnote{To use verb `to dress' we have two choices. First, if there is something to put on, we use \pali{acch\=adeti} (v.t.), otherwise we use \pali{niv\=aseti} or \pali{paridahati} (v.i.).}
\end{answerkey}

\section*{Exercise \ref{chap:imp}}
\begin{answerkey}
\item vad\=ahi, bho, potthak\=alayassa\footnote{This word can be of gen.\ or dat.\ sense. If dat.\ is intentional, you can also use \pali{potthak\=alay\=aya} or \pali{potthak\=alayattha\d m}.} magga\d m.
\item imasm\=a imin\=a maggena gacch\=ahi yasmi\d m dutiya\d m \\maggasandhi\d m, tasmi\d m gacch\=ahi dakkhi\d na\d m.\footnote{For more about ordinal number, see Chapter \ref{chap:num}. It is more suitable to use \pali{ito} instead of \pali{imasm\=a} (see Chapter \ref{chap:ind-to}). And \pali{y\=ava--t\=ava} can be used instead of \pali{yasmi\d m--tasmi\d m}.}
\item pass\=ami.
\item tasm\=a lohita\d m ag\=ara\d m passasissasi. ta\d m atigacch\=ahi. potthak\=alayo v\=amasmi\d m ti\d t\d thati.
\item \=acikkha me potthak\=alayassa os\=anak\=ala\d m.
\item pa\~ncaggha\d tik\=aya\d m, tena s\=igha\d m gaccha.\footnote{Here \pali{s\=igha\d m} (quickly) is used as adverb (see Chapter \ref{chap:adv}). For more about time telling, see Sentence No.\,\ref{conv:time}, page \pageref{conv:time}.}
\item upagacch\=ami ta\d m pure tasm\=a.
\item m\=a sa\~ncara. dh\=ava.
\end{answerkey}

\section*{Exercise \ref{chap:opt}}
\begin{answerkey}
\item gaccheyy\=asi samosara\d na\d m Liza-n\=am\=aya gehasmi\d m \\imissa\d m rattiya\d m.
\item k\=idisa\d m samosara\d na\d m?\footnote{For \pali{k\=idisa} (what kind?), see Appendix \ref{chap:kita}, page \pageref{par:kiidisa}.}
\item j\=atadivasassa samosara\d na\d m siy\=a.
\item (yasm\=a) ta\d m na parij\=an\=ami, tasm\=a aha\d m na gaccheyya\d m./ \\asanthavasm\=a aha\d m na gaccheyya\d m.\footnote{Other words that can do the same job as \pali{santhava} (familiarity) is \pali{paricaya} and \pali{viss\=asa}. By prefixing the terms with \pali{a}, you can make them negative (see page \pageref{nip:a}).}
\item santhavassa punappuna\d m ta\d m sam\=agaccheyy\=asi. \\tasm\=a may\=a saddhi\d m gacche.
\item hareyya\d m nu pa\d n\d n\=ak\=ara\d m?\footnote{Particle \pali{nu} can mark a yes-no question (see Chapter \ref{chap:ques}).}
\item yo j\=atadivasasamosara\d nassa s\=aro, so pa\d n\d n\=ak\=aro siy\=a.
\end{answerkey}

\section*{Exercise \ref{chap:cond}}
\begin{answerkey}
\item siy\=a nu amh\=aka\d m sambh\=asana\d m, \=acariya? tava k\=alo ce atthi.
\item \=ama, sace na aticira\d m. sikkh\=apana\d m me atthi im\=aya\d m a\d d\d dhagha\d tik\=aya\d m.\footnote{A general positive response is \pali{\=ama} (see Chapter \ref{chap:ques}, and Appendix \ref{chap:nipata}, page \pageref{nip:aama}). And \pali{ati-} is used as `too' or `excessive.' You can learn more about \pali{upasagga} in Appendix \ref{chap:upasagga}.}
\item k\=idis\=a asundar\=a mama visesalipi, kasm\=a D-va\d n\d na\d m me ad\=asi?
\item sace tva\d m me s\=adhuka\d m su\d neyy\=asi/asu\d nisse vijj\=agabbhe, aj\=anisse `janasammatap\=alanan'ti na `janassa matap\=alanan'ti.
\item hoti nanu ta\d m `janassa matap\=alana\d m'?
\item na eva\d m. kasm\=a tava mitte na pucchi?
\item maya\d m ekato/ekadh\=a/ekattena j\=aneyy\=ama.\footnote{For \pali{ekato}, see Chapter \ref{chap:ind-to}. For \pali{ekadh\=a}, see Appendix \ref{chap:taddhita}, page \pageref{pacct13:dhaa}. And \pali{ekatta} (nt.) is a noun meaning `unity' or `agreement.'} atthi nu me niddosassa kicca\d m, sace tva\d m anuj\=an\=asi.\footnote{In this sentence, ``\pali{atthi nu me niddosassa kicca\d m}'' means like ``Is there anything to do with my correction?'' A more practical way to say this is ``\pali{sakkomi nu ta\d m niddosa\d m k\=atu\d m?},'' but this uses an infinitive which we have not yet learned (see Chapter \ref{chap:inf}). Or you can use optative mood, like ``\pali{niddosa\d m kareyya\d m nu ta\d m?}'' (May/Should I fix that?). But it is not quite a right way to do, because using future passive participle (see Chapter \ref{chap:pass}) is more fashionable. Hence, it should be put as ``\pali{ki\d m ta\d m niddosa\d m k\=atabba\d m?}'' (Should it be fixed?).}
\item sace tva\d m icchasi, `janasammatap\=alanan'ti ta\d m puna likh\=ahi. tena hi sve mayha\d m ta\d m dehi.\footnote{In practice, the absolutive (see Chapter \ref{chap:pp}) is a more suitable solution here, hence, ``\pali{\ldots ta\d m puna likhitv\=a sve mayha\d m dehi}.''}
\item thuti te atthu, \=acariya.
\end{answerkey}

\section*{Exercise \ref{chap:pron-misc}}
\begin{answerkey}
\item j\=an\=asi ekanta\d m nu tva\d m naccas\=al\=aya magga\d m?\footnote{By `surely,' we can use, among several others, \pali{ekanta} (\pali{eka+anta}). Literally, this means `one end.' Figuratively, it means `no other alternative' or `absolute,' hence `sure.' In the sentence, the term is used as an adverb (see Chapter \ref{chap:adv}). Other several particles can be used likewise, in a way, are, for example, \pali{addh\=a, a\~n\~nadatthu, dhuva\d m, n\=una, khalu,} and so on (see Appendix \ref{chap:nipata}).}
\item \=ama, j\=aneyya\d m.\footnote{Optative mood can express supposition (see Chapter \ref{chap:opt}).} ekakkhattu\d m ta\d m agami\d m.\footnote{For \pali{ekakkhattu\d m}, see Appendix \ref{chap:nipata}, page \pageref{nip:kkhattudm}.} \\ki\d m \=acikkhati GPS-upakara\d na\d m?\footnote{On neologism, see notes on page \pageref{sec:neologism}.}
\item GPS-upakara\d na\d m eva\d m \=acikkhati, \\`abhimukhe maggasandhiya\d m dakkhi\d nena gacch\=ah\=i'ti.\footnote{In P\=ali it is very common to use direct speech. So, we change indirect speech to \pali{iti} structure (see Chapter \ref{chap:iti}).}
\item micch\=a magga\d m \=acikkheyya.\footnote{Alternative to optative mood that marks a surmise, we can form the sentence using direct speech, like ``\pali{micch\=a magga\d m \=acikkhat\=i'ti ma\~n\~n\=ami}'' (I think it tells [us] a wrong way).} \\t\=a racch\=a dakkhina\d m upanagara\d m nayati. \\gacch\=ama nanu nagarassa majjha\d m?\footnote{Imperative mood is used in this sentence.}
\item sacca\d m. tato pana ya\d m a\~n\~na\d m/para\d m \=ava\d t\d tana\d m hoti, \\ta\d m nagarassa majjhe nayati.\footnote{For \pali{tato} (from there), see Chapter \ref{chap:ind-to}.} yasm\=a ga\d nakayanta\d m sabbad\=a vipajjati, tasm\=a maya\d m GPS-upakara\d na\d m \\anugaccheyy\=ama.\footnote{For \pali{sabbad\=a}, see Chapter \ref{chap:ind-to}. Or you can use, as we have learned so far, \pali{sabbasmi\d m k\=ale}.}
\item s\=adhu, ta\d m amh\=aka\d m sabba\d m hoti. abhimukhe maggassa dve s\=akh\=a santi. katarasmi\d m magge gaccheyyu\d m?
\item id\=ani `v\=amasmi\d m gaccheyyun'ti ta\d m \=acikkhati.
\item n\=una koci doso atthi. t\=a racch\=a uttara\d m nayati. amh\=aka\d m vijj\=alayassa maggo'ti sar\=ami. k\=idisena/kena gatiniy\=amena tva\d m GPS-upakara\d na\d m \d thapesi.
\item passeyya\d m. kham\=ahi me.\footnote{Verb \pali{khamati} means `to forgive.' So, this sentence means ``Forgive me.'' That is a way to say `sorry' in P\=ali.} t\=a have vijj\=alaya\d m nayati.\footnote{For \pali{have}, an emphatic particle, see page \pageref{nip:have}.}
\item ga\d nakayanta\d m na kad\=aci vipajjati, manuss\=a n\=ama pana khalanti.\footnote{Interestingly, \pali{n\=ama}, among other particles, can be used in blaming (see page \pageref{nip:naama}), as we see in this sarcasm. Also note that, \pali{na kad\=aci} and \pali{vipajjati} are Antonio Costanzo's creative ideas.}
\item khama.
\end{answerkey}

\section*{Exercise \ref{chap:num}}
\begin{answerkey}
\item kati jan\=a etarahi COVID-rog\=i honti?\footnote{For \pali{etarahi}, a locative particle, see page \pageref{nip:intime}. COVID-rog\=i simply means `COVID patients.'}
\item sattadasamadivase $(17th)$ m\=agham\=ase $(Feb)$ \\sa\d mvacchar\=ana\d m ekav\=isatayuttaradvisahasse $(2021)$ \\pa\~ncav\=isajanuttarasattasat\=adhik\=ani $(725)$ \\pa\~ncatti\d msajanasahass\=adhik\=ani $(35,000)$ \\jan\=ana\d m ek\=adasako\d ti $(11 \times 10^{7})$ honti.
\item kesu ra\d t\d thesu bahukatam\=a/bahuki\d t\d th\=a rog\=i santi?\footnote{This simply means ``In what contries do the most numerous patients exist?'' For adjective comparison, see Chapter \ref{chap:adjcomp}.}
\item pathama\d m America-ra\d t\d the dviko\d ti a\d t\d thadasasatasahassa\d m ca pam\=a\d nena, dutiya\d m Jambud\=ipa-ra\d t\d the ekako\d ti \\ekadasasatasahassa\d m ca, tatiya\d m Brazil-ra\d t\d the ekako\d ti.
\item ki\d m/k\=idisa\d m C\=inara\d t\d tha\d m?
\item etarahi tassa pa\~ncanavutijanuttarasattasat\=adhik\=ani $(795)$ \\ek\=unanavutijanasahass\=ani $(89,000)$ santi, n\=amavaliy\=a catur\=as\=itima\d m $(84th)$.
\item ki\d m id\=ani mara\d nassa a\~n\~nama\~n\~nappam\=a\d na\d m?
\item pam\=a\d nato satabh\=agavasena dve hoti. so n\=una bhay\=anako rogo.
\item k\=iva cira\d m maya\d m imasmi\d m sa\.nkantikarogabh\=avasmi\d m vasiss\=ama?
\item yasm\=a id\=ani gopanassa antopavesana\d m\footnote{To make it simple, I use \pali{gopanassa antopavesana} to mean ``injection of protection.'' I found that \pali{gop\=uya} (m.) can be used for `vaccine.'} atthi, tasm\=a ta\d m dvetayavasse/katipayavasse pavatteyya/bhaveyya.
\item lokassa vin\=asana\d m siy\=a, dhammat\=aya veran\=iy\=atana\d m.
\item k\=iva abhi\d nha\d m tva\d m adhun\=a naccach\=ayar\=up\=ani passi?
\item dv\=adasa pam\=a\d nena imasmi\d m satt\=ahe.
\item t\=ani atibahuk\=ani siyu\d m.
\end{answerkey}

\section*{Exercise \ref{chap:ind-to}}
\begin{answerkey}
\item ima\d m sabbad\=a vissuta\d m pa\d tima\d m passatha, d\=arak\=a. \\pa\d n\d narasa-satavaccharato Michelangelo-n\=amassa David-n\=amo hoti.
\item ki\d m s\=a akittim\=a hoti, \=acariya?\footnote{This means ``Is it not artificial?'' Pronoun \pali{s\=a} relates with \pali{pa\d tim\=a} (f.).}
\item yato/yatra s\=a m\=ulabh\=utavatthuto pa\d tir\=upak\=a hoti, tato/ tatra s\=a na tena sama\d m sundar\=a.
\item atthi nu David-n\=amo saccato/tathato, \=acariya?\footnote{This question means ``Does David really exist?'' Or you can ask more literally ``ki\d m David-n\=amo sacco/tatho hoti?}
\item \=ama, so Israel-ra\d t\d thassa dutiyo r\=aj\=a abhavi at\=ite cirak\=alato.
\item tad\=a passi nu tato Michelangelo ta\d m?
\item na ekad\=a. y\=adisa\d m tassa ma\~n\~nanato r\=upa\d m hoti, t\=adis\=a aya\d m pa\d tim\=a.\footnote{Thinking in terms of \pali{ya-ta} structure often yields a better translation of complex sentences. In here, \pali{ma\~n\~nanato} = \pali{ma\~n\~nan\=a + to}.}
\item tato so/s\=a asacco/asacc\=a.\footnote{By `it' in this question, it can mean the statue (\pali{s\=a}) or David himself (\pali{so}).}
\item \=ama, pana passa \ldots
\item so saccato atimahanto siy\=a.\footnote{Optative mood can express a hypothesis, and ablative case or \pali{-to} particles in this case can mark a cause or reason.} katarato so naggo?
\item a\~n\~n\=ani vatth\=uni passeyyu\d m, d\=arak\=a.
\end{answerkey}

\section*{Exercise \ref{chap:ques}}
\begin{answerkey}
\item t\=ata, kasm\=a gagana\d m n\=ila\d m?
\item ta\d m dujj\=ana\d m, putta.
\item ta\d m samuddena pacc\=avattat\=i'ti m\=at\=a vadi.\footnote{Here, \pali{pacc\=avattati} (to reflect) = \pali{pati + vattati}, for \pali{pati} see Appendix \ref {chap:upasagga}.} sacca\d m nu hoti?
\item na sacca\d m. m\=a t\=adisa\d m kassaci \=arocehi.
\item ki\d m b\=ahiro ok\=aso n\=ilo siy\=a?
\item na siy\=a. b\=ahiro ok\=aso ka\d nho hoti.
\item tena hi \=arocehi me kasm\=a gagana\d m n\=ilan'ti.
\item suriyara\d msi \=ak\=asassa a\d n\=u paharati. ra\d msiy\=a vikiranena n\=ilava\d n\d na\d m a\~n\~n\=ani abhibhavati adhik\=aya abhi\d nhasiddhiy\=a.
\item visajjana\d m te mogha\d m. m\=atussa pucchana\d m seyyo.\footnote{For \pali{seyyo}, see Chapter \ref{chap:adjcomp}.}
\item ki\d m/k\=idisa\d m indadhanu, t\=ata, kuto ta\d m?
\item ta\d m dis\=ama\d n\d dale niddhikumbh\=ihi hoti.
\item ta\d m attharahita\d m.
\end{answerkey}

\section*{Exercise \ref{chap:adv}}
\begin{answerkey}
\item sace sabbesa\d m pubbahetu atthi, mayha\d m serit\=a \\tathato/saccato santi v\=a na v\=a?
\item yo tava serit\=aya attho, so nissito hoti.
\item yath\=ak\=ama\d m maya\d m kamm\=ani kara\d n\=aya sakkoma iti attho.\footnote{It is more typical to use infinitive in this sentence, hence \pali{k\=atu\d m} rather than \pali{kara\d n\=aya}. See Chapter \ref{chap:inf} for more detail.}
\item kattuno attano dassan\=a, yasm\=a attan\=a attan\=a eva\d m \\ma\~n\~nati, tasm\=a mayha\d m serit\=a siyu\d m.
\item eva\d m bahutamabh\=ag\=a jan\=a ta\d m pa\~nha\d m passeyyu\d m.
\item dhammat\=aya dassan\=a a\~n\~nato pana sabb\=ani a\~n\~n\=ani \\nissayanti. serit\=aya have sa\~n\~n\=avipall\=aso hoti. \\Benjamin Libet-n\=amassa \=avikara\d nasm\=a mayha\d m \\matthalu\.nga\d m s\=ighatara\d m pi j\=an\=ati mayha\d m cetan\=aya.
\item tasm\=a p\=apa\d m kara\d n\=aya sakkomi, yasm\=a na mama \\t\=ira\d na\d m hoti tathato.
\item ta\d m eka\d msena a\~n\~na\d m pa\~nha\d m hoti. katha\~ncipi tava attano vipall\=asajanik\=aya serit\=aya kusal\=ani kamm\=ani k\=atabb\=ani.\footnote{An easy way to say the last sentence is to use future passive participle (see Chapter \ref{chap:pass}). Alternatively to \pali{katha\~ncipi} (however), you can also use \pali{kenaci \=ak\=arena} (by whatever manner).} 
\end{answerkey}

\section*{Exercise \ref{chap:pi}}
\begin{answerkey}
\item kasm\=a, bhante, aha\d m na dhanav\=a homi, \\bahuk\=ani pu\~n\~n\=ani kato pi.\footnote{It is necessary to use past participle here (see Chapter \ref{chap:pp}) because we already use \pali{homi} as the main verb. Otherwise you have to split the sentence.}
\item seyyath\=ida\d m, gahapati?
\item aha\d m d\=ana\d m ad\=asi\d m imasmi\d m \=ar\=ame katipay\=ana\d m \\ku\d t\=ina\d m.
\item imin\=a s\=asanena tva\d m niyata\d m dhanav\=a bhavissasi \\sampar\=aye na tava k\=amena pi.
\item api ca aha\d m itthatte dhanavantassa icch\=ami, bhante.\footnote{Typically, desiring to do something in P\=ali \pali{icchati} is used with the infinitive (see Chapter \ref{chap:inf}). If the action is a noun, dative case is expected. For desiring certain objects, we use accusative case as usual.}
\item tassa tva\d m dakkha\d m viriyena kamm\=ani karohi. tena pi yattakassa dhanavantassa icchasi, tattako na bhaveyy\=asi.
\item tena hi ki\d m d\=anassa hita\d m imasmi\d m s\=asane?
\item ekanta\d m tava d\=anas\=are vipatti atthi.\footnote{This literally means ``There is definitely your failure in the essence of giving.''}
\end{answerkey}

\section*{Exercise \ref{chap:prp}}
\begin{answerkey}
\item ki\d m kari, ayy\=a, tava gehasmi\d m corassa bhindantassa?
\item corassa \=agaccham\=anassa, s\=amarakkhi, uparitale aha\d m sayi.
\item j\=anant\=a hosi, ki\d m nassi?
\item passeyya\d m, na p\=aka\d ta\d m siy\=a. pubba\d nhe adhotala\d m \\\=agaccham\=anassa abhimukha\d m dv\=ara\d m viva\d ta\d m iti j\=ani\d m, s\=ital\=ikara\d nama\~nj\=us\=a viva\d t\=a iti ca.\footnote{Here \pali{viva\d ta} is used as passive past participle (see Chapter \ref{chap:pass}). For refrigerator, it is \pali{s\=ital\=i + kara\d na + ma\~nj\=us\=a} (f.), a box that makes coolness.}
\item so ch\=ato siy\=a.
\item ta\d m upahasan\=iya\d m. na aha\d m kassaci geha\d m bhindiss\=ami, ki\d mci eva kh\=adan\=aya iccham\=an\=aya.
\item koci te j\=anake siy\=a.\footnote{I avoid using passive voice by using a noun here, it is read ``Maybe someone [is] in your knowledge.''} kattha tava s\=am\=i abhavi pavattiy\=a bhavam\=an\=aya?
\item so me eva\d m \=arocesi sabbarattiya\d m kamma\d m kurum\=ano na geha\d m pacc\=agamiss\=am\=i'it.\footnote{Try to make it direct speech. It is easier to handle. See more in Chapter \ref{chap:iti}.} \\sace so hoti, kasm\=a dv\=ara\d m viva\d ta\d m. n\=una coro hoti.
\item (a\~n\~no s\=amarakkh\=i) maya\d m eka\d m purisa\d m sadisa\d m tava s\=amin\=a upalabh\=ama, ayy\=a, sur\=amatto so sayanto \\rathas\=al\=aya\d m.\footnote{Alternatively to \pali{sadisa\d m tava s\=ami\d m} (like your husband), you can say ``\pali{sama\d m tava s\=amin\=a}.'' For the use of \pali{sama\d m} with ins., see page \pageref{nip:samadm}. Yet another way to say this is ``\pali{tava s\=ami\d m iva/viya}.''}
\item (pa\d thamo s\=amarakkh\=i) ima\d m [\=arocana\d m] sabbe va\d n\d neti.
\end{answerkey}

\section*{Exercise \ref{chap:pp}}
\begin{answerkey}
\item ko doso tava rathassa, bho kayika?
\item idha aha\d m \=agacchanto katipayakkhattu\d m yanta\d m nivatta\d m (hoti).\footnote{Here, \pali{\=agacchanto} relates to \pali{aha\d m} (supposed to be a male), not \pali{yanta\d m}. For \pali{-kkhattu\d m}, see page \pageref{nip:kkhattudm}.}
\item acir\=at\=ite ta\d m s\=aretv\=a koci upaddavo v\=a ki\d mci as\=atatika\d m payojana\d m v\=a bh\=uta\d m.\footnote{In P\=ali, verbs normally agree with the last subject (see page \pageref{par:multiactors}), thus \pali{bh\=uta\d m} not \pali{bh\=uto}.}
\item na garuka\d m, eka\d m pabbata\d m gantv\=a aha\d m katipayesu jalasotesu ta\d m s\=arito.
\item na patir\=upa\d m tava rathassa t\=adise pade hoti. \\ratho te sabbena vibhajanena v\=ima\d msana\d m k\=atabbo.\footnote{It is typical to use future passive participle in the last sentence (see Chapter \ref{chap:pass}). Alternatively, you can say it in active form, like ``\pali{aha\d m ratha\d m te sabbena vibhajanena v\=ima\d msiss\=ami}'' (I will overhaul your car).}
\item ta\d m [v\=ima\d msana\d m] me bahuka\d m m\=ula\d m aggheyya.\footnote{Using optative mood is better here.} \\ki\d m tva\d m nanu ta\d m pakati\d m eva karohi? atthi nanu tassa p\=aka\d to doso yante?
\item tena hi, bho, aha\d m ta\d m yanta\d m rathasm\=a u\d t\d th\=apetv\=a, ta\d m dhovitv\=a anto bahiddh\=a ca, saka\d t\d th\=ane ta\d m \d th\=apetv\=a, ta\d m s\=aress\=ami.
\item tasm\=a katha\~ncipi bahuka\d m m\=ula\d m te dadeyy\=ami.
\item ta\d m amh\=aka\d m kicca\d m, bho kayika.
\end{answerkey}

\section*{Exercise \ref{chap:pass}}
\begin{answerkey}
\item Buddhassa purimaj\=atiya\d m Vessantarassa k\=ale, \\tassa putto dh\=it\=a ca a\~n\~nassa dinn\=a ca da\d n\d dit\=a. \\ta\d m ki\d m adhammika\d m kamma\d m?
\item na sakk\=a Buddhavisayo amhehi j\=anitu\d m.\footnote{It is typical to use infinitive in this sentence (see Chapter \ref{chap:inf}). In practice, you may convert this to active structure and use a dative action noun instead of the infinitive, hence ``We are not capable for knowing the Buddha's vision'' (\pali{maya\d m Buddhavisaya\d m \~n\=a\d nassa na sakkoma}).} eva\d m ta\d m by\=akariyati `sabbhodhi padh\=an\=a hoti attano pariggahehi, puttehi bhariy\=aya c\=a'ti.\footnote{In dictionaries you may find \pali{vy\=akaroti} (to explain) instead. In traditional texts, we normally use \pali{by\=akaroti}. Other words beginning with \pali{vy-} will be \pali{by-} as well, e.g.\ \pali{byaggha} (tiger) not \pali{vyaggha}. In this sentence, the verb is in passive form, \pali{by\=akara + i + ya}.}
\item na bodhito so tasmi\d m k\=ale. katha\d m so ta\d m j\=ani? \\tassa micch\=amati siy\=a. sace tasmi\d m k\=ale saccato sabbe tena \~n\=at\=a, puna j\=ati na bhavitabb\=a.
\item s\=asanassa dassanato t\=adiso vitakko na kattabbo. \\a\~n\~nath\=a s\=asanassa m\=ulapati\d t\d th\=a umm\=ulitabb\=a.
\item sace ta\d m kamma\d m paccuppannak\=ale vijjati, \\ta\d m adhammika\d m bhavissati, yasm\=a putt\=a ca bhariy\=a puri\-sena na pariggahit\=a. na sakk\=a t\=a a\~n\~nassa dinn\=a purisassa attano hit\=aya eva.
\item paccuppannak\=alassa niy\=am\=a at\=itak\=alasm\=a asam\=an\=a santi. na adhammika\d m tasmi\d m k\=ale ta\d m kamma\d m siy\=a.\footnote{Optative mood is tense-blind (imperative also). So, it can be used regardless of time. To stress certain idea, put it at the beginning.}
\item ki\d m s\=ilassa dhammo ak\=aliko hoti v\=a na v\=a? \\ud\=ahu visesas\=ilo atthi visi\d t\d th\=aya puggal\=aya?
\item na vivecitabba\d m Buddhassa t\=ira\d na\d m.
\item tva\d m va\d t\d tula\d m takkesi.
\item tva\d m ta\d m saddah\=ahi \~n\=a\d n\=aya.
\item v\=ima\d msan\=iya\d m Buddhas\=asana\d m iti ma\~n\~n\=ami.
\end{answerkey}

\section*{Exercise \ref{chap:caus}}
\begin{answerkey}
\item (tasm\=a) aha\d m ma\d m att\=ana\d m vikkhip\=apemi, \\yasm\=a eva\d m ma\~n\~n\=ami `sace kassaci att\=a natthi, \\ko sa\d ms\=are sa\d msarat\=i'ti?
\item aya\d m pa\~nho pur\=a\d no nirantaro ca hoti. tasmi\d m viv\=ado Buddhak\=alasm\=a pi paccuppannak\=ale vattati.
\item duttakkana\d m eva siy\=a iti ma\~n\~n\=ami.\footnote{For \pali{du} (bad, poor, difficult), see Appendix \ref{chap:upasagga}, page \pageref{upasagga:du}.}
\item ki\~nc\=api tassa pa\~nhassa vijj\=avisayak\=a s\=akacch\=a vijjeyya, api ca Buddhabhattike mah\=ajane so pa\~nho natthi.\footnote{This sentence is a concession (see Chapter \ref{chap:pi}).} ya\d m ki\~nci vuccati jan\=ana\d m, ta\d m janehi pa\d tigga\d nhiyati takkanena vin\=a.\footnote{This sentence is put in passive form. You can use \pali{pa\d tiggahetabba\d m} instead. This sounds more speculative.}
\item ko pana ekassa ek\=ibh\=ava\d m sa\d ms\=arassa antare p\=avatt\=apeti?\footnote{In causative form, \pali{p\=avatt\=apeti = pavatta + \d n\=ape + ti}.}
\item tassa bahuk\=a va\d n\d nan\=a ett\=avat\=a dinn\=a. \\sace \~n\=a\d n\=aya icchasi, t\=ani potthak\=ani pa\d th\=ahi. api ca aha\d m eva\d m ma\~n\~n\=ami `yasm\=a so pa\~nho ajjhattavijj\=aya na hoti, tasm\=a tassa vitth\=arena va\d n\d nan\=a na siy\=a'ti.\footnote{Between the quotes, it is literally read ``Because that problem is not metaphysical, its explanation in detail might/should not be exist.''} \\saccena n\=utan\=aya\d m cetasikavijj\=aya\d m pi amh\=aka\d m attasa\~n\~n\=a amhe dukkh\=apeti. `ta\d m hi Buddho \=arocan\=aya ussahito'ti ma\~n\~n\=ami.
\item di\d t\d tho amhi.\footnote{This means `I understood.' If the speaker is a female, it will be \pali{di\d t\d th\=a}.} t\=a di\d t\d thi sa\.ngha\d m p\=avatt\=apeti, sukhena pi p\=alana\d m sand\=apeti. tass\=a [di\d t\d thiy\=a] mah\=ajanika\d m kicca\d m atthi, a\~n\~nato anattav\=adassa cittavisayaka\d m kicca\d m puggal\=aya puggal\=aya atthi.\footnote{For repetition, see Chapter \ref{chap:adv}. You may use one \pali{paccatta\d m} (individually) instead.}
\end{answerkey}

\section*{Exercise \ref{chap:inf}}
\begin{answerkey}
\item yasmi\d m pahonaka\d m P\=alibh\=asa\d m j\=an\=ami, sakk\=a nu kho tasmi\d m antima\d m sacca\d m upalabhitu\d m tipi\d take?
\item na \d th\=ana\d m tena vijjati.\footnote{This is a way to say ``It is (not) possible'' (see Chapter \ref{chap:opt}).}
\item ki\d m nanu?
\item pa\d thama\d m, ki\~nci antima\d m sacca\d m, tena sama\d m a\~n\~na\d m v\=a, akkharesu natthi, kasmi\~nci pa\~n\~nattikamme v\=a. \\candassa s\=ucana\d m a\.nguliy\=a viya hoti.
\item socan\=iya\d m ta\d m sutv\=a.
\item dutiyampi, ya\d m tva\d m pa\d thasi, ta\d m katha\d m ekantena yath\=atatha\d m j\=an\=asi?
\item s\=adhuka\d m rakkhita\d m nanu tipi\d taka\d m?
\item \=ama, ta\d m s\=adhuka\d m rakkhita\d m, ya\d m eka\d m sa\d msodhana\d m. pure sa\.nga\d nhane maya\d m ekanta\d m na j\=an\=ama. \\ek\=a pavatti pi bhikkh\=uhi n\=an\=ak\=arehi sarito, paccuppannak\=alassa pavattipattesu viya.\footnote{The last sentence is formed in passive voice. It is read ``Even one event was remembered by monks in different manners, like in newspapers nowadays.''}
\item antamaso [te] yath\=atatha\d m s\=asana\d m rakkhitu\d m \\sa\~ncetan\=aya ma\~n\~nit\=a santi.\footnote{Slightly different in structure, this is read as ``At least, they thought with intention to preserve the real teaching.'' You can leave out \pali{sa\~ncetan\=aya} because it is redundant.}
\item ekadh\=a ta\d m sacca\d m, aha\d m eva\d m ma\~n\~n\=ami ca. \\api ca ki\d m tva\d m eva\d m sarasi `tipi\d take hi sace m\=atug\=amo pabbajito, pan\~ca eva vassasat\=ani saddhammo \d thassat\=i'ti? no ce tena vassasahassa\d m eva abhavi.
\item nanu pa\~nca vassasahass\=ani?
\item t\=a ga\d nan\=a Buddhabhattik\=ana\d m jan\=ana\d m di\d t\d thiya\d m eva \=aka\.nkh\=aya\d m ca atthi. sace tva\d m tipi\d takassa tathata\d m saddahasi, kasm\=a n\=utanatara\d m va\d n\d nana\d m ga\d nh\=asi, \\na tipi\d take?
\item ta\d m socan\=iya\d m. tena hi ko attho P\=alibh\=as\=aya?
\item na hi t\=adisa\d m socan\=iya\d m. ta\d m pana amhe up\=ad\=an\=a \\pamoceti. sabbe ganth\=a sikkhan\=iy\=a/uggahitabb\=a, \\na laggitabb\=a. P\=alibh\=as\=aya vijj\=a tva\d m atacch\=a di\d t\d thiy\=a pamocitu\d m sakkoti.
\item tasm\=a sabba\d m [tipi\d taka\d m] may\=a attan\=a pa\d thitabba\d m.
\item na avassa\d m eva t\=adisa\d m. bahuk\=ani parivattan\=ani ett\=avat\=a santi. tva\d m t\=ani pa\d thitu\d m sakkosi. ganth\=ana\d m ca \\parivattak\=ana\d m ca pa\d ticchanna\d m sa\~ncetana\d m vivaritu\d m sakkosi P\=alibh\=as\=avijj\=aya. t\=adiso a\~n\~nataro maggo \\P\=alisikkh\=aya paccuppannak\=ale.
\item di\d t\d tha\d m me atthi.\footnote{This means ``It is understood by me.''} bahukicc\=ani tassa\d m vijj\=aya\d m siyu\d m.
\item na ala\d m P\=alisikkh\=aya\d m gantha\d m eva parivattitu\d m. \\t\=a vic\=ara\d nayuttatar\=a vibhajanayuttatar\=a ca bhavitabb\=a.
\end{answerkey}

\section*{Exercise \ref{chap:iti}}
\begin{answerkey}
\item So kosakimi Alis\=a\footnote{To make it easier, instead of using \pali{Alice-n\=am\=a}, I use \pali{Alis\=a} for Alice.} ca a\~n\~nama\~n\~na\d m olokit\=a ki\~nci k\=alantara\d m abh\=asanena: ante, so kosakimi dh\=uman\=a\d lik\=aya\footnote{Hookah = \pali{dh\=uma + n\=a\d lik\=a} (smoke tube/bottle).} tassa mukha\d m n\=iharitv\=a, Alisa\d m \=amantesi nidd\=alu\d m ol\=ina\d m.
\item ``K\=a tva\d m as\=i''ti? kosakimi vadi.
\item Na ida\d m sall\=ap\=aya man\=apa\d m \=arabhana\d m ahosi. \\Alis\=a eva\d m vissajjesi \=isaka\d m salajja\d m ``aha\d m---aha\d m \\kicchena j\=an\=ami, bho, id\=ani eva---antamaso yasmi\d m u\d t\d thit\=a amhi pubba\d nhe, tasmi\d m aha\d m `k\=a amh\=i'ti j\=an\=ami. `Apica tato katipayakkhattu\d m vipari\d n\=amit\=a'ti ma\~n\~n\=am\=i''ti.
\item ``Tena ki\d m attho hot\=i''ti? kosakimi vadi, atida\d lha\d m. \\``Att\=ana\d m tva\d m by\=akaroh\=i''ti.
\item ``Aha\d m att\=ana\d m by\=ak\=atu\d m na sakkomi (bhayena\footnote{Perhaps, it is better not to translate `be afraid.' It confuses the sense.}), \\bho''ti Alis\=a vadi, ``Yasm\=a aha\d m mama att\=a na homi, passas\=i''ti.
\item ``Aha\d m na pass\=am\=i''ti kosakimi vadi.
\item Alis\=a eva\d m vissajjesi ativin\=ita\d m ``\ldots ekasmi\d m divase n\=an\=a pam\=a\d nehi bhavitv\=a vikhepak\=a amh\=i''ti.
\item ``Na hot\=i''ti kosakimi vadi.
\item S\=a att\=ana\d m u\d t\d th\=apetv\=a atigaruk\=aya eva\d m vadi, \\``tva\d m pure ko as\=i'ti \=aroceyy\=as\=i'ti ma\~n\~n\=am\=i''ti.\footnote{The last part has three layers of speech.}
\item ``Kasm\=a''ti? kosakimi vadi.
\item Yasmi\d m Alis\=a ki\~nci s\=attha\d m\footnote{`Good reason' is hard to translate. I use \pali{s\=attha\d m} (useful) for `good.' The term is formed by \pali{sa + attha} (with benefit). See page \pageref{upasagga:sadm} for how \pali{sa} comes. Antonio Costanzo suggests a more straight term, \pali{kusala\d m k\=ara\d na\d m}. It sounds too morality-loaded to me.} hetu\d m cintetu\d m na \\asakkhi, kosakimi pi accantasmi\d m ani\d t\d thasmi\d m \\cittasabh\=avasmi\d m bhaveyya, tasmi\d m s\=a pa\d tikkami.
\item ``Pacc\=agaccha!'' iti kosakimi ta\d m pakkosi. \\``Mama garuk\=a v\=ac\=a atthi!'' iti
\item Alis\=a parivattetv\=a puna \=agacchi.
\item ``Upasamehi tava cittasabh\=avan''ti kosakimi vadi.
\item ``Atthi nu ta\d m sabban''ti? Alis\=a vadi, pasahit\=a tass\=a \\kodha\d m.
\item ``No''ti kosakimi vadi. So tassa b\=ah\=a pas\=aretv\=a, puna dh\=uman\=a\d lik\=aya tassa mukha\d m n\=iharitv\=a, vadi, \\``Ma\~n\~nasi nanu tva\d m t\=adisa\d m `vipari\d n\=amit\=a'ti'' iti.
\item ``Eva\d m, bho''ti Alis\=a vadi. ``Na sakkomi mama \\purimasabh\=ava\d m saritu\d m---ek\=ipam\=a\d na\d m na \d th\=apemi dasavigha\d tik\=ayan''ti.\footnote{For `minute' (\pali{vigha\d tik\=a}), see Sentence No.\,\ref{conv:time}, page \pageref{conv:time}.}
\item ``Ki\d m pam\=a\d na\d m bhavitu\d m icchas\=i''ti? kosakimi pucchi.
\item ``Aho, na visi\d t\d tha\d m pam\=a\d na\d m hot\=i''ti Alis\=a vegena \\vissajjesi, ``eka\d m pam\=a\d na\d m na anekad\=a vipari\d n\=amita\d m, passasi. Aha\d m \=isaka\d m uttar\=a mahant\=a bhavitu\d m \\iccheyy\=ami, bho, no ce tva\d m kopito''ti Alis\=a vadi. \\``T\=i\d ni a\.ngul\=ani duggat\=a ucc\=a hom\=i''ti.
\item ``Ta\d m have atisundara\d m pam\=ana\d m!'' iti kosakimi ujuka\d m k\=aya\d m \d thapetv\=a kuddho vadi (so yath\=abh\=uta\d m t\=i\d ni a\.ngul\=ani ucco).
\item Ekadv\=isu vigha\d tik\=asu, kosakimi ahicchattakasm\=a oruhitv\=a ti\d nagumbe sa\d msappitv\=a, eva\d m vadam\=ano gacchi, \\``Eko anto ta\d m uccatara\d m va\d d\d dhessati, a\~n\~no anto pi ta\d m n\=icatara\d m va\d d\d dhessat\=i''ti.\footnote{Antonio Costanzo suggests \pali{anto} (an end) for `side' here.}
\item ``Kassa eko anto, kassa a\~n\~no anto''ti? \\Alis\=a attano ma\~n\~ni.
\item ``Ahicchattakass\=a''ti kosakimi vadi, uccassarena s\=a pucchi iva; a\~n\~natare kha\d ne, so vigacchi.
\end{answerkey}
