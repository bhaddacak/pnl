\chapter{We \headhl{always} walk \headhl{from home} to school \headhl{here}}\label{chap:ind-to}

\phantomsection
\addcontentsline{toc}{section}{Suffixed Indeclinables}
\section*{Suffixed Indeclinables}

Occasionally, in previous lessons and exercises I mentioned some particles that have \pali{-to} ending. Because its prevailing uses, it should be introduced earlier. I present these particles late, because we can use other alternatives, such as nouns with a suitable case. So, it is not urgent to know. Now we will learn this kind of particles. There are more than \pali{-to} as we shall see. When you understand all these, you are encouraged to use them. Because they are very handy to use, and in some situations they can solve ambiguity problems.

As I count by myself, there are 19 suffixes when they are added to certain nouns or pronouns, the whole words become indeclinable. These suffixes are \pali{to, tra, tha, ha, dha, dhi, hi\d m, ha\d m, hi\~ncana\d m, hi\~nci, va, d\=a, d\=ani, rahi, raha, dhun\=a, d\=acana\d m, ajja,} and \pali{ajju}.

The first one seems to be the most used, because it enables us to make new words endlessly. When words are annexed by \pali{to} they can perform functions of three cases mainly, i.e.\ ins., abl., and loc., and in lesser extent, gen.\footnote{Sadd\,493; Kacc\,248; R\=upa\,260; Mogg\,4.95--8; Payo\,472--5; Niru\,275--8.} I list some examples given by traditional textbooks in Table \ref{tab:indto}.

\bigskip
\begin{longtable}[c]{@{}%
	>{\itshape\raggedright\arraybackslash}p{0.2\linewidth}%
	>{\raggedright\arraybackslash}p{0.5\linewidth}%
	>{\raggedright\arraybackslash}p{0.2\linewidth}@{}}
\caption{Some particles suffixed with \pali{to}}\label{tab:indto}\\
\toprule
\bfseries\upshape Particle & \bfseries\upshape Meaning& \bfseries\upshape Case\\ \midrule
\endfirsthead
\multicolumn{3}{c}{\tablename\ \thetable: Some particles suffixed with \pali{to} (contd\ldots)}\\
\toprule
\bfseries\upshape Particle & \bfseries\upshape Meaning& \bfseries\upshape Case\\ \midrule
\endhead
\bottomrule
\ltblcontinuedbreak{3}
\endfoot
\bottomrule
\endlastfoot
%
aniccato & by impermanent nature & ins. \\
dukkhato & by unsatisfactory nature & ins. \\
rogato & by sickness & ins. \\
purisato & from man & abl. \\
ithito & from woman & abl. \\
r\=ajato & from king & abl. \\
g\=amato & from home & abl. \\
corato & from thief & abl. \\
aggito & from fire & abl. \\
sabbato & from all & abl. \\
a\~n\~nato & from other & abl. \\
a\~n\~natarato & from further other & abl. \\
itarato & from other & abl. \\
ekato & by/from/on one side & ins./abl./loc. \\
ubhato & by/from/on both sides & ins./abl./loc. \\
parato & by/from/on other side & ins./abl./loc. \\
aparato & by/from/on further other side & ins./abl./loc. \\
purato & by/from/on front side & ins./abl./loc. \\
pacchato & by/from/on rear side, from behind & ins./abl./loc. \\
dakkhi\d nato & by/from/on right/southern side & ins./abl./loc. \\
uttarato & \mbox{by/from/on upper/northern side} & ins./abl./loc. \\
v\=amato & by/from/on left side & ins./abl./loc. \\
tato & from that & abl. \\
etto, ato & from this/that & abl. \\
ito & from this & abl. \\
yato & from where & abl. \\
kuto & from where? & abl. \\
katarato & from which? & abl. \\
\=adito & at first, from the beginning & abl./loc. \\
majjhato & in the middle, amid & loc. \\
s\=isato & on the head side & loc. \\
p\=adato & on the foot side & loc. \\
passato & on the flank/side & loc. \\
pi\d t\d thito & on the back side & loc. \\
mukhato & on the front side & loc. \\
aggato & at the top & loc. \\
m\=ulato & at the root & loc. \\
he\d t\d thato & in the lower, beneath & loc. \\
abhito & round about, on both sides & loc. \\
parito & on every side & loc. \\
antato & at the end & loc. \\
\end{longtable}

The use of gen.\ by \pali{to} particles is rare. Here is an example suggested by Sadd\,493: ``\pali{ya\d m parato d\=anapaccay\=a}.''\footnote{Ja\,14:212--3} This is equivalent to ``\pali{ya\d m parassa d\=anapaccay\=a}'' meaning ``which (thing obtained) by supportive gift \emph{of other}.'' In practice, if you do not have a very good reason to do likewise, I suggest you to avoid such a use. Aggava\d msa, in Sadd\,496, also says that \pali{to} particles sometimes have nom.\ meaning as an alternative to \pali{iti}. I will ignore this too in our entire course. At the stage of making a firm foundation, you should avoid any wildly ambiguous usage. However, cases suggested in the table are not absolute, you can use in other proper senses as long as the meaning allows. And by no means it is a complete list. You can make your own words if you think it is sensible for others to understand. I can give you one contemporary example: ``\pali{aha\d m hadayato vad\=ami}'' (I speak from/by the heart). This might make no sense in the traditional way, but it sounds fashionable.

The use of these particles is simple as it sounds. For example, ``\pali{g\=amato \=agacch\=ami}'' (I come (here) from home), ``\pali{corato bh\=ayati}''\footnote{Verb \pali{bh\=ayati} takes abl., see Chapter \ref{chap:abl}, page \pageref{sec:ablverbs}.} (he/she fears (from) thiefs). As indeclinables, they can be used in both singular and plural sense.\footnote{In Mogg\,4.95, examples go self-explained like this: ``\pali{G\=amato \=agacchati g\=amasm\=a \=agacchati, corato bh\=ayati corehi bh\=ayati.}''}

We can use \palibf{ekato} and \palibf{parato} or \palibf{a\~n\~nato} as we say ``On one side \ldots, on the other side \ldots'' in English. For example, ``\pali{ekato vir\=upo homi, parato k\=aru\d niko}'' (On one side I am ugly, on the other side I am kind). \pali{Ekato} can also mean `together,' e.g.\ \pali{ekato karoti} (to put together, to collect).

Other terms worth mentioning here, for its frequent uses, is \pali{kuto} and \pali{yato}/\pali{tato}. We use \palibf{kuto} to make a question, for instance, ``\pali{kuto \=agacchasi}'' (From where do you come?). A pair of \palibf{yato}/\palibf{tato} can form a correlative sentence, as we have seen in Chapter \ref{chap:yata}. For example, ``\pali{yato \=agacchasi, tato \=agacch\=ami}'' means ``I come from where you come.'' Other words in this group can be used with no difficulty, so I leave them to you.

Apart from the terms listed in the table, in Payogasiddhi some others are given as examples. I list the rest here so that you can get more idea: \pali{hatthito} (from elephant), \pali{hetuto} (from cause), \pali{yuttito} (from justice), \pali{bhikkhunito} (from nun), \pali{y\=aguto} (from rice-gruel), \pali{jambuto} (from rose-apple), \pali{cittato} (from mind), \pali{\=ayuto} (from age).\footnote{Payo\,472} As you may notice, long ending of nouns is usually shortened before being composed with \pali{to}. Here are more examples from Niruttid\=ipan\=i: \pali{ka\~n\~nato, vadhuto} (from girl), \pali{rattito} (from night), \pali{m\=atito} (from maternal side), \pali{pitito} (from paternal side), \pali{bhikkhuto} (from monk), \pali{satth\=arato} (from the master), \pali{kattuto} (from doer).\footnote{Niru\,275}

Let us move to other group of suffixes. The next ten, namely \pali{tra, tha, ha, dha, dhi, hi\d m, ha\d m, hi\~ncana\d m, hi\~nci} and \pali{va}, are added to pronouns to make them loc.\ in space.\footnote{Kacc\,249--255; R\=upa\,266--275; Sadd\,494, 499--503; Mogg\,4.99--103; Payo\,476--480, Niru\,279--284} The list of these particles is shown in Table \ref{tab:indtra}.

\bigskip
\begin{longtable}[c]{@{}%
	>{\itshape\raggedright\arraybackslash}p{0.45\linewidth}%
	>{\raggedright\arraybackslash}p{0.45\linewidth}@{}}
\caption{Particles suffixed with \pali{tra}, etc.}\label{tab:indtra}\\
\toprule
\bfseries\upshape Particle & \bfseries\upshape Meaning\\ \midrule
\endfirsthead
\multicolumn{2}{c}{\tablename\ \thetable: Particles suffixed with \pali{tra}, etc. (contd\ldots)}\\
\toprule
\bfseries\upshape Particle & \bfseries\upshape Meaning\\ \midrule
\endhead
\bottomrule
\ltblcontinuedbreak{2}
\endfoot
\bottomrule
\endlastfoot
%
sabbatra & in all \\
sabbattha & in all \\
sabbadhi & in all \\
a\~n\~natra & in other \\
a\~n\~nattha & in other \\
yatra & in which, where \\
yattha & in which, where \\
yahi\d m & in which, where \\
yaha\d m & in which, where \\
tatra & in that \\
tattha & in that \\
tahi\d m & in that \\
taha\d m & in that \\
katra & in which?, where? \\
kattha & in which?, where? \\
kuhi\d m & in which?, where? \\
kuha\d m & in which?, where? \\
kaha\d m & in which?, where? \\
kuhi\~ncana\d m & in which?, where? \\
kuhi\~nci\footnote{Sadd\,500, Mogg\,4.104} & in which?, where?\\
kva\footnote{This can become \pali{ko}, e.g.\ ``\pali{Ko te bala\d m mah\=ar\=aja}'' (Great king, sir, where is your power?), Ja\,22:1880. See also Sadd-Pad Ch.\,12.} & in which?, where? \\
kuva\d m\footnote{Niru\,280} & in which?, where? \\
atra & in this/that \\
attha & in this/that \\
ettha & in this/that \\
idha & in this \\
iha & in this \\
amutra & in such and such a place \\
amuttha & in such and such a place \\
ubhayattha & in both \\
\end{longtable}

The rest eight of suffixes, namely \pali{d\=a, d\=ani, rahi, raha, dhun\=a, d\=acana\d m, ajja,} and \pali{ajju} are also added to pronouns to make them loc.\ in time.\footnote{Kacc\,257--9; R\=upa\,276--9; Sadd\,505--7, 1167 (for \pali{ajja, ajju}); Mogg\,4.105--7; Payo\,482--4; Niru\,285--7} I summarize these particles in Table \ref{tab:indda}.

\bigskip
\begin{longtable}[c]{@{}%
	>{\itshape\raggedright\arraybackslash}p{0.45\linewidth}%
	>{\raggedright\arraybackslash}p{0.45\linewidth}@{}}
\caption{Particles suffixed with \pali{d\=a}, etc.}\label{tab:indda}\\
\toprule
\bfseries\upshape Particle & \bfseries\upshape Meaning\\ \midrule
\endfirsthead
\multicolumn{2}{c}{\tablename\ \thetable: Particles suffixed with \pali{d\=a}, etc. (contd\ldots)}\\
\toprule
\bfseries\upshape Particle & \bfseries\upshape Meaning\\ \midrule
\endhead
\bottomrule
\ltblcontinuedbreak{2}
\endfoot
\bottomrule
\endlastfoot
%
sabbad\=a & in all time \\
sad\=a & in all time \\
a\~n\~nad\=a & in other time \\
ekad\=a & in one time, once \\
yad\=a & in what time, when \\
tad\=a & in that time \\
tad\=ani & in that time \\
kad\=a & in what time?, when? \\
kud\=a\footnote{Sadd\,505, Mogg\,4.106} & in what time?, when? \\
karaha\footnote{Mogg\,4.107} & in what time?, when? \\
kad\=aci & in certain time, sometimes \\
id\=ani & in this time \\
etarahi & in this time \\
adhun\=a & in this time \\
kud\=acana\d m & in any time \\
ajja\footnote{Mogg\,4.107, Sadd\,1167--8. In Sadd\,1168, the term is formed by \pali{ima + ajja}, but \pali{ima} is changed to \pali{a}. In Sadd\,1167, this means `in this time' (\pali{imasmi\d m k\=ale ajja}).} & on this day, today \\
sajju\footnote{Mogg\,4.107, Niru\,287, Sadd\,1167. Mogg gives us a vague explanation, ``\pali{sam\=ane ahani sajju}'' (in the same/existing day). In Niru it is clearer, ``\pali{tattha `sajj\=u'ti tasmi\d m divase}'' (in that sense, \pali{sajju} means ``on that day''). So, it seems to mean `on the day mentioned.' However, Sadd\,1167 suggests that \pali{sam\=anak\=ale sajju} means \pali{tasmi\d m kha\d ne} (in that moment). In Sadd\,1169, it is shown that \pali{s} is truncated from \pali{sam\=ana}. PTSD seems to follow this when ``instantly, speedily, quickly'' is given as meaning of the term. That is familiar to us to use it as an adverb in conversations.} & on that day \\
aparajju\footnote{Mogg\,4.107, Sadd\,1167} & on other day \\
\end{longtable}

Using these particles is straightforward like you do with other locative cases, for example ``\pali{kad\=a gacchasi}'' (When do you go?), ``\pali{yad\=a gacchasi, tad\=a gacch\=ami}'' (I go when yo go), ``\pali{kad\=aci \=agacchati}'' (Sometimes he/she comes).

Now we can finish our heading task ``We always walk to school here.'' Analyzing the sentence and figuring out what particles we can use here, we find that `always' means `in all time,' hence \pali{sabbad\=a}. The phrase `from home' can be a \pali{to} word, hence \pali{gehato}. Another term is `here' meaning `around this area.' We can use \pali{idha} for this. Therefore, rearranging words properly, we get this sentence:

\palisample{idha maya\d m sabbad\=a gehato p\=a\d thas\=ala\d m p\=adena gacch\=ama.}

Now it is your turn to do the exercise.

\section*{Exercise \ref{chap:ind-to}}
Say these in P\=ali. This fictitious dialogue between a teacher and young children takes place in a local museum.
\begin{compactenum}
\item Children, look at this all-time famous statue. It is David of Michelangelo from the 15th century.
\item Is it real, teacher?
\item It is a copy from the original piece, so it is not equally beautiful as that.
\item Is David real, teacher?
\item Yes, he was the second king of Israel from the long past.
\item Did Michelangelo see him in that time?
\item No, not even once. It is from his imagination that this statue should look like.
\item So, it is not real.
\item Yes, but look \ldots
\item He must look very big, if it is real. And why does he get naked?
\item Let us see other objects, children.
\end{compactenum}
