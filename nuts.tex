\chapter{Nuts and Bolts}\label{chap:nuts}

\phantomsection
\addcontentsline{toc}{section}{Introduction to P\=ali Letters}
\section*{Introduction to P\=ali Letters}

Before speaking a language we have to learn its basic parts first. In this chapter I will summarize briefly the letters used in P\=ali. By the fact that the language was long dead, we by no means know its exact phonetics. Guide to pronunciation here is just a reasonable reconstruction. Unlike Sanskrit that normally uses Devanagari to represent its alphabet, P\=ali uses local scripts, e.g.\ Sinhala, Myanmar, Khmer, and Thai. When Westerners come to study P\=ali, they use Roman script. In this book we use the modern application of Roman script to P\=ali. It is just some of English alphabet with a few diacritical marks.

P\=ali letters are divided into 8 vowels\footnote{Kacc\,3, R\=upa\,3, Sadd\,3, but Mogg\,1.2 asserts that there are 10 vowels (\pali{das\=ado sar\=a}) including short \pali{e} and short \pali{o} when they are followed by double consonants. We will not follow Moggall\=ana's view.} (\pali{sara}) and 33 consonants (\pali{vya\~njana/bya\~njana}). Here are the vowels:

\begin{center}
\large \itshape a\ \ \ \=a\ \ \ i\ \ \ \=i\ \ \ u\ \ \ \=u\ \ \ e\ \ \ o
\end{center}

There are three pairs of short-long (\pali{rassa-d\=igha}) sounds. The top bar (macron) marks the long sounds. Other two, \pali{e} and \pali{o} are usually long but can be short when preceding double consonants. A guide for vowel pronunciation is shown in Table \ref{tab:vowels}.\footnote{adapted from \citealp[p.~2]{tilbe:grammar}}

\begin{table}[!hbt]
\centering
\caption{Pronunciation of P\=ali vowels}
\label{tab:vowels}
\bigskip
\begin{tabular}{@{}>{\itshape}cl@{}} \toprule
\bfseries\upshape Vowel & \bfseries Sounds like\\ \midrule
a & u in but\\
\=a & a in father\\
i & i in pin\\
\=i & ee in seen\\
u & oo in foot\\
\=u & oo in food\\
e & a in mate\\
o & o in note\\
\bottomrule
\end{tabular}
\end{table}

The P\=ali consonants in typical order are:

\begin{center}
\large \itshape k\ \ kh\ \ g\ \ gh\ \ \.n\ \ c\ \ ch\ \ j\ \ jh\ \ \~n\ \ \d t\ \ \d th\ \ \d d\ \ \d dh\ \ \d n\ \ t\ \ \\th\ \ d\ \ dh\ \ n\ \ p\ \ ph\ \ b\ \ bh\ \ m\ \ y\ \ r\ \ l\ \ v\ \ s\ \ h\ \ \d l\ \ \d m
\end{center}

These consonants can be grouped corresponding to their place of articulation in the mouth, whether they are voiceless or voiced, and whether they are aspirated or non-aspirated. Scholars classifies \pali{\d m}\footnote{In old texts \pali{\ng}, sometimes \pali{\.m}, is used.} (\pali{niggah\=ita}) as a vowel because it is just the sign of nasalization of \pali{a}, \pali{i}, and \pali{u}.\footnote{\citealp[p.~2]{geiger:grammar}; \citealp[p.~1]{collins:grammar}} However, traditional grammarians count \pali{\d m} as a consonant.\footnote{Kacc\,6, R\=upa\,8, Sadd\,6, Mogg\,1.6, Niru\,6} The reason is that, by traditional definition vowels can make sounds by themselves\footnote{In Sadd\,3, \pali{saya\d m r\=ajant\=iti sar\=a} (self-shining are vowels).}, but consonants cannot.\footnote{Vowels are those on which others depend (\pali{nissaya}), whereas consonants are those dependent on others (\pali{nissita}), as stated in R\=upa\,2: \pali{sar\=a nissay\=a, itare nissit\=a}.} Following this definition, it is reasonable to put \pali{\d m} in consonant group because it has to follow vowels \pali{a, i, u} to make sound, unlike other consonants which depend on succeeding vowels. They all cannot produce any sound by themselves.

\begin{table}[!hbt]
\centering
\caption{Grouping of P\=ali consonants}
\label{tab:grouping}
\bigskip\footnotesize
\begin{tabular}{@{}l*{10}{>{\itshape}c}@{}} \toprule
\multirow{2}{*}{\ } &
\multicolumn{2}{c}{\textbf{voiceless}} &
\multicolumn{5}{c}{\textbf{voiced}} &
\multicolumn{1}{c}{\textbf{vl.}} &
\multicolumn{1}{c}{} & \\
\cmidrule(lr){2-3}\cmidrule(lr){4-8}\cmidrule(lr){9-9}
&
\begin{sideways}\bfseries\upshape unaspirated\,\end{sideways} &
\begin{sideways}\bfseries\upshape aspirated\end{sideways} &
\begin{sideways}\bfseries\upshape unaspirated\end{sideways} &
\begin{sideways}\bfseries\upshape aspirated\end{sideways} & 
\begin{sideways}\bfseries\upshape nasal\end{sideways} &
\begin{sideways}\bfseries\upshape semivowel\end{sideways} &
\begin{sideways}\bfseries\upshape spirant\end{sideways} &
\begin{sideways}\bfseries\upshape sibilant\end{sideways} &
\begin{sideways}\bfseries\upshape nasal\end{sideways} &
\begin{sideways}\bfseries\upshape vowels\end{sideways} \\
\midrule
\textbf{guttural} & k & kh & g & gh & \.n & & h & & & a, \=a, e, o \\
\textbf{palatal} & c & ch & j & jh & \~n & y & & & & i, \=i, e \\
\textbf{retroflex} & \d t & \d th & \d d & \d dh & \d n & r, \d l & & & & \\
\textbf{dental} & t & th & d & dh & n & l, v & & s & & \\
\textbf{labial} & p & ph & b & bh & m & v & & & & u, \=u, o\\
\textbf{nasal} & & & & & & & & & \d m & \\
\bottomrule
\end{tabular}
\end{table}

The consonant grouping is summarized in Table \ref{tab:grouping}. Here are some explanation including what unable to put in the table. \emph{Gutturals} are pronounced in the throat (\pali{ka\d n\d thaja}). \emph{Palatals} are pronounced in the palate (\pali{t\=aluja}) using the middle of the tongue (\pali{jivh\=amajjha}). \emph{Retroflexes}\footnote{Some old texts use \emph{cerebral}. \citealp[See also][p.~3]{warder:intro}.} are pronounced with the tongue curled round touching the top of the mouth, the back of the ridge behind the teeth (\pali{muddhaja}). This is done by the area near the tip of the tongue (\pali{jivhopagga}). \emph{Dentals} are pronounced with the teeth (\pali{dantaja}) using the tip of the tongue (\pali{jivhagga}). \emph{Labials} are pronounced with the lips (\pali{o\d t\d thaja}). \pali{Niggah\=ita} (\pali{\d m}) is pronounced with the nose (\pali{n\=asika}). The first 25 consonants that can be classified by their source of production are called \pali{vagga}.\footnote{Kacc\,7, R\=upa\,9, Sadd\,7, Mogg\,1.7} The remaining of that, including \pali{niggah\=ita}, are hence called \pali{avagga}.

Most consonants are generated in one place, except \pali{\.n, \~n, \d n, n, m} are nasal plus their own sources mentioned above, and \pali{v} is dental-labial. The last column shows the place of articulation of corresponding vowels. It is worth noting that \pali{e} and \pali{o} are generated from two sources, guttural-palatal and guttural-labial respectively. \emph{Voiced} (\pali{ghosa}) sounds are produced with vibrating vocal cords, whereas \emph{voiceless} (\pali{aghosa}) sounds are produced with open, nonvibrating vocal cords. \emph{Aspirated} (\pali{dhanita}) sounds are produced with additional puffing air, like blending with `h,' whereas \emph{unaspirated} (\pali{sithila}) sounds are absent of that air. By itself \pali{h} is generated from the throat, but when it combines with \pali{\.n, \~n, \d n, n, m, y, r, l, v,} and \pali{\d l} it is generated from the chest (\pali{urasija})\footnote{See the explanation of Sadd\,23.}, for example, \pali{ta\~nhi, ta\~nh\=a, nh\=as\=a, asumha, muyhate, vulhate, avhito,} and \pali{r\=u\d lhi}. The hissing sound of \pali{s} is dental and voiceless. There is no voiced hiss like \emph{z} in P\=ali.

A pronunciation guideline of P\=ali consonants is shown in Table \ref{tab:consonants}.

\bigskip
\begin{longtable}[c]{@{}>{\centering\itshape}p{0.25\linewidth}p{0.65\linewidth}@{}}
\caption{Pronunciation of P\=ali consonants}\label{tab:consonants}\\
\toprule
\bfseries\upshape Consonant & \bfseries Sounds like\\ \midrule
\endfirsthead
\multicolumn{2}{c}{\tablename\ \thetable: Pronunciation of P\=ali consonants (contd\ldots)}\\
\toprule
\bfseries\upshape Consonant & \bfseries Sounds like\\ \midrule
\endhead
\bottomrule
\ltblcontinuedbreak{2}
\endfoot
\bottomrule
\endlastfoot
%
k & k in king\\
g & g in gun\\
\.n & ng in sing\\
c & ch in choose, church\\
j & j in jump\\
\~n & n(y) in minion\\
t, d, n & in English\\
th & t + air, not like thin or then\\
\d t, \d th, \d d, \d dh, \d n & dentals but the tongue touches the top of the mouth not teeth\\
p, b, m & in English\\
ph & p + air, not like phone\\
y, r, l, s, h & in English\\
v & w in English; v when standing alone\footnote{\citealp[p.~3]{warder:intro}}\\
\d l & l but aspirated and the tongue touches the top of the mouth \\
-\d m & ng in sung, sing, (soong)\\
\end{longtable}

\phantomsection
\addcontentsline{toc}{section}{Vowel Gradation}
\section*{Vowel Gradation}

A related topic I want to add here for referencing in the future is vowel gradation or strength. There are three levels of this. I summarize it in Table \ref{tab:vstrength}.\footnote{This is adapted from \citealp[p.~5]{collins:grammar}. See also \citealp[p.~12]{warder:intro}.} 

\begin{table}[!hbt]
\centering
\caption{Vowel gradation}
\label{tab:vstrength}
\bigskip
\begin{tabular}{@{}*{3}{>{\itshape}c}@{}} \toprule
\bfseries\upshape (zero) & \bfseries gu\d na & \bfseries vuddhi \\ \midrule
a & a & a + a = \=a \\
i \=i y & a + i/y = ay or \u e & a + a + i/y = \=ay or e \\
u \=u v & a + u/v = av or \u o & a + a + u/v = \=av or o \\
\bottomrule
\end{tabular}
\end{table}

If you cannot understand the thing, just ignore it for now. You will find it useful when we come to relevant topics. In the table, the plus (+) sign denotes the conjunction of two vowels. It does not entail any order, so it is the same when \pali{a} meets \pali{i} or vice versa. Slash (/) means `or' here. In \pali{gu\d na} strength, \pali{\u e} and \pali{\u o} denote short sounds of the vowels. In practice, these and their long sound are not much different. So, we normally do not use the notation of short sounds.

As you have seen previously, \pali{y} and \pali{v} are called `semivowel' because they are produced similarly to certain vowels, i.e.\ \pali{i} and \pali{u} respectively. When you learn about word joining (Sandhi) (Appendix \ref{chap:sandhi}), you will see that these semivowels and their equivalents can be interchanged. You will see \pali{gu\d na} strength mostly in Sandhi. And when you learn about \pali{paccaya} processing, particularly \pali{\d na} and its kin, such as in Appendix \ref{chap:taddhita}, you will see \pali{vuddhi} strength there.

