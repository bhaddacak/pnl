\chapter{Critical Introduction to P\=ali}\label{chap:intro}

P\=ali is one of old languages of India used exclusively within Theravada Buddhist traditions, mainly to write religious scriptures. Philologically speaking, together with Pr\=akrit in Asoka's inscriptions, P\=ali is grouped into early Middle Indo-Aryan language.\footnote{\citealp[p.~14]{cardonajain:intro}} The Old Indo-Aryan is represented by Sanskrit. These languages belong to the bigger tree---Indo-European. That is why Indic languages and European languages, such as those which are rooted in Greek and Latin, have common characteristics, e.g.\ the use of inflection (much more about this in due course).

\phantomsection
\addcontentsline{toc}{section}{Did the Buddha speak P\=ali?}
\section*{Did the Buddha speak P\=ali?}
A quick answer can be simply ``Yes, of course'' from the tradition's point of view. But answering this question thoughtfully is more complicated than it seems. The coming discussion may be difficult to follow by new P\=ali learners. You just skip the quoted passages for now, and come to them again when ready. The point I try to make here is so important that it determines my approach to P\=ali and Buddhism as a whole.

Before we deal with the question, we have to tackle an equally tricky question first: ``Where does P\=ali come from?'' The name is relatively new to the language itself. The term \pali{p\=ali} means ``line, row, series'' which denotes a series of books in Buddhist scriptures.\footnote{\citealp[p.~vii]{childers:dict}} At first, \pali{p\=ali} is used to differentiate itself from non-canonical literature. That is to say, commentaries and beyond are not \pali{p\=ali} in this sense.\footnote{Thai tradition still follows this notion to some extent.} At last, it comes to mean any text in the scriptures or a portion of it. When the language of the scriptures is mentioned, it is called \pali{p\=alibh\=as\=a} meaning literally ``language (\pali{bh\=as\=a}) of the texts (\pali{p\=ali}).'' This language is equivalent to, as Robert Childers maintains, M\=agadh\=i or Magadhese, the language of Magadha, the area where the Buddha lived for many years. The English use of P\=ali as the language's name comes from the way the Sinhalese use the term.\footnote{\citealp[p.~vii]{childers:dict}} Sometimes we see \pali{P\=a\d li} is used instead, but this spell is of late introduction by the Sinhalese.\footnote{\citealp[p.~322]{childers:dict}}

The problem of the language's name is so easy that no one seems to argue about this. Then a more difficult riddle comes: ``Is P\=ali really Magadhese?'' The controversy about this issue is complex and perennial. Let us consider textual evidence first. In C\=u\d lavagga (minor collections) of the Vinaya, there is an incident that two brother monks said to the Buddha that monks coming from various cultures corrupt the Buddha's words by repeating it in \emph{one's own} dialect (\pali{sak\=aya niruttiy\=a buddhavacana\d m d\=usenti}).\footnote{Cv\,5.285} The problematic word here is \pali{sak\=aya} (by one's own). It can be interpreted as monks' own dialect\footnote{Rhys Davids and Oldenberg follow this line of translation \citep[p.~150]{rhys:vinaya3}. \d Th\=anissaro Bhikkhu also follows this because it is more understandable \citep[p.~745]{thanissaro:code}. Richard Gombrich shows us that in Ara\d ni-vibha\.nga Sutta (MN 139) the Buddha allowed the use of local dialects \citep[p.~147]{gombrich:what}. Or, as the text goes, the Buddha suggested not to insist on (only one) local language for it can lead to a conflict.} which makes more sense in this context, or as the Buddha's own dialect which is Magadhese.\footnote{This is the standard view of the tradition. Buddhaghosa states clearly in the commentary: ``\pali{Sak\=aya niruttiy\=ati ettha sak\=a nirutti n\=ama samm\=asambuddhena vuttappak\=aro m\=agadhiko voh\=aro.}'' In sum, ``one's own dialect'' means the Buddha's dialect or Magadhese. In I.\,B.\ Horner's translation of the Vinaya, this line of translation is used \citep[p.~2171]{horner:discipline}. Buddhaghosa even thinks this Magadhese or M\=agadh\=i is the basic language of all humans (\pali{m\=agadhik\=aya sabbasatt\=ana\d m m\=ulabh\=as\=aya}), \citealp[see][p.~437:\S XIV.25]{buddhaghosa:visud} (Vism\,14.428). But this view is simply wrong according to today well-established knowledge. }.

Let us go back to the story. When the two monks complained that monks from various clans corrupted the Buddha's words. They proposed a solution by putting the teaching into (Sanskrit) verse (\pali{buddhavacana\d m chandaso \=aropema}).\footnote{Sanskrit is not explicitly indicated in the text, but implied by \pali{chandaso} (of verse). \citealp[See][p.~150]{rhys:vinaya3}. In the commentary, Buddhaghosa specifies it as Veda-like (\pali{veda\d m viya sakkatabh\=as\=aya}).} The Buddha, however, declined the request and forbade so doing. Then he allowed monks to learn the teaching in, again, ``one's own'' dialect.\footnote{\pali{Anuj\=an\=ami, bhikkhave, sak\=aya niruttiy\=a buddhavacana\d m pariy\=apunitu\d m} (Cv\,5.285).} We have two competing ideas here. First, the Buddha allowed the teaching to be rendered into local languages. In other words, the meaning is more important than its form. This is the widely accepted view among scholars of Buddhism. And second, following the tradition, the Buddha allowed the teaching to be kept and learned in its original form.\footnote{I have checked Thai translations of the P\=ali canon on this issue. I found that the old translations make clear that ``one's own dialect'' means ``the original language.'' Whereas a recent translation of MCU edition puts the commentary's explanation in a footnote, and leaves the text to be read as ``one's own language.'' This looks more straight, but confusing to the readers. Some traditional adherents, such as Ven.\,\d Th\=anissaro as mentioned above, do not go with the traditional commentary.}

Let us think carefully about this. The main point is not about how the monks learn the teaching. Can anyone understand anything in other language? There must be a kind of translation, otherwise the learning will never happen. It really makes no sense that the Buddha gives a permission or prohibition to use any language at all in learning situation. The very point is that whether the Buddha's words (\pali{buddhavacana}) should be kept intact or left behind and rendered into new language. This is not a trivial question. It sits upon a fundamental assumption: whether meaning is independent to its medium. If you think it is, rendering words to a new form does not matter much as long as the spirit of the words is still there. If you think it is not, like many modern linguists and philosophers, words are not always or fully translatable, so it is better to keep the original. Unsurprisingly, the tradition follows the latter view, but I think it was not always so and Buddhist communities might hold different stances on this.

To the problem of the permission to learn the Buddha's words in ``one's own dialect,'' Wilhelm Geiger suggests us to stick to the explanation given by Buddhaghosa, i.e.\ in ``the Buddha's own dialect.'' He gives us this account: ``The real meaning of this injunction is, as is also best in consonance with Indian spirit, that there can be no other form of the words of Buddha than in which the Master himself had preached.''\footnote{\citealp[p.~7]{geiger:literature}} Let us keep this issue in mind for a while and consider evidence outside the scriptures.

There is a way to find out whether the P\=ali language we have today looks similar to those used in the ancient time. Comparing with Asoka's inscriptions (around 300 years after the Buddha's death) is the most viable method used by scholars, because Asoka's rock edicts spread all over India with different use of dialects for easing local understanding. Unfortunately to the traditional mind, the closest form of language to P\=ali is not found in the north, but found in Girn\=ar at the far west of India.\footnote{\citealp[pp.~182--3]{oberlies:asokan}; \citealp[p.~3]{geiger:literature}} K.\,R.\ Norman casts some doubt on this, ``since it is possible that it represents, in part at least, the scribe's attempt to convert the Eastern dialect he must have received from P\=a\d taliputra into what he thought was appropriate to the region in which the edict was being promulgated, rather than the actual dialect of that region.''\footnote{\citealp[p.~4]{norman:literature}} However, by the fact that inscriptions in this area are closer to our P\=ali than those from the north, the view that P\=ali is not M\=agadh\=i but rather a dialect of western India is somewhat justified. How is it so? One possible scenario is when Buddhism spread to the west, it assimilated to that local culture. Then this version of Buddhist teaching went to Sri Lanka.

Hermann Oldenberg thinks that the transmission of Buddhism from the mainland India to Sri Lanka was not a one-time dispatch as the story of Mahinda's missionary goes. Moreover, he has reasons to think that Mahinda did not brought the canon with Ujjen\=i dialect to Sri Lanka.\footnote{\citealp[pp.~l--li]{oldenberg:vinaya1}} There was continuous interaction between the island and the southern India. It is possible that, according to Oldenberg, the P\=ali canon and the P\=ali language itself are brought to Sri Lanka from the kingdoms of Andhra or Kali\.nga.\footnote{\citealp[p.~liv]{oldenberg:vinaya1}} From this view, P\=ali is by no means M\=agadh\=i by a different reason.

Another reason to reject P\=ali as the language originally spoken by the Buddha is the incongruous nature of the language we have it. Wilhelm Geiger enumerates four stages of development of the language as follows: (1) the language of the G\=ath\=a or poetry that is very heterogenous; (2) the language of the canonical prose that is governed by more rigid rules; (3) the later prose of the post-canonical literature that looks artificial and erudite; and (4) the language of later artificial poetry that imitates Sanskrit syntax and archaic styles.\footnote{\citealp[pp.~1--2]{geiger:literature}} This shows that the language underwent changes and mixing---``a compromise of various dialects.''\footnote{\citealp[p.~2]{geiger:literature}; \citealp[p.~1]{geiger:grammar}} Geiger also gives us reasons why the P\=ali canon looks so incongruous: 

\begin{quote}
The peculiarities of its language may be fully\linebreak explained on the hypothesis of (a) a gradual development and integra­tion from different parts of India, (b) a long oral tradition extending over several centuries, and (c) the fact that the texts were written down in a different country.\footnote{\citealp[p.~5]{geiger:literature}}
\end{quote}

Let us take another meticulous thought on this matter. Could P\=ali change? The question sounds naive but let us start with a simple mind. If we hold that the Buddha allowed his words to be translated into local dialects, then P\=ali definitely underwent changes. There is no reason to keep what is no longer understood. So, what we have today is far from the original form, but the intended meaning is still with us. That is one line of thought sitting on an assumption of translatability of texts. On the other hand, if we hold that the Buddha really allowed monks to keep and learn his words as they are, unfortunately changes are still inevitable. As we know that monks committed the teaching to their memory for several centuries and across locations, the original language gradually lost its sense. When words or phrases are no longer understood anymore, they cannot be kept in memory intact for long. They are easily changed to a more intelligible form, like a game of Chinese whispers. Or they may assume new meaning completely. Even the best effort cannot keep the original intact. And even the teaching is written down (around the 1st century B.C.), it still can be changed to be comprehensible. K.\,R.\ Norman tells us that ``the P\=ali of the canon as we have it now is a reflection of the P\=ali of the twelfth century, when the influence of the P\=ali grammarians was at its highest.''\footnote{\citealp[p.~6]{norman:literature}}

Considering the physical evidence might give us a clearer picture. ``The continuous manuscript tradition with complete texts begins only during the late 15th century.''\footnote{\citealp[p.~4]{hinuber:literature}; \citealp[See also][p.~xxv]{geiger:grammar}} This means P\=ali as we have it today is not old as the tradition holds it.

To conclude, if the question is ``Did the Buddha speak P\=ali (as we have it now)?'' The straight answer is ``No.'' You may add ``but close enough,'' but we do not really know how close it is. If the question normatively implies as ``Should the Buddha speak P\=ali?'' I choose to follow the tradition by answering ``Yes, of course.''\footnote{Speaking the language and saying things presented in the canon are different stories. The latter is harder to defend as we will see below.} This keeps me from a lot of headache and enables me to focus on more important things.

\phantomsection
\addcontentsline{toc}{section}{How reliable is the P\=ali canon?}
\section*{How reliable is the P\=ali canon?}
This question seems irrelevant to the content of this book. I include this problem here to reflect my attitudes that determine the approach of the book. Undoubtedly, the tradition gives a positive answer to this question. Hence, reliability of the canon is out of question. From the first council (3 months after the Buddha's death) onwards, the teachings was settled and finalized. Monks recited and memorized the outcome ``as accurately, purely and completely as possible---in short, pristinely and perfectly.''\footnote{\citealp[p.~19]{payutto:canon}} To the traditional mind, what we obtained from the first council is the final teachings. The task afterward is only to keep it as such, both by remembering the original as perfectly as possible and preventing spurious teachings to creep in. Ideally, the results of the subsequent councils should be more or less the same. As a matter of fact, however, the structure of the canon as well as the content were changed continuously. For example, after the third council (around 300 years after the first one) \pali{Kath\=avatthu} was added to the Abhidhamma. Recently, three books, namely \pali{Nettippakara\d na}, \pali{Pe\d takopadesa}, and \pali{Milindapa\~nh\=a} were included to the canon by a council in Myanmar. This shows that if some good treatises are old enough, they can be candidates for canonical promotion. I supposed that \pali{Visuddhimagga} might be a next one. This textual evidence clearly tells us that new materials can be added to the canon if they agree with the tradition's `spirit.'

If the canon is continuously changed by adding new materials, correcting the unfitted, or deleting anomalies, what do we really mean by reliability? It can mean if any change occurs it has to correspond with the existing canon which was preserved from the first compilation. But if we know exactly what is the original, why changes are allowed at all? That means we are not really sure what counts as original in the first place. There must be a kind of approving process to include or exclude particular ideas or events. That is to say, the direction of the canon is determined mostly by the authority. The canon has to be normalized before it gets `published.' That is the main reason why the whole canon is so congruous.\footnote{In Steven Collin's words, ``remarkably stable in content'' \citeyearpar[p.~41]{collins:nirvana}} Richard Gombrich also notes on this point: ``[A] sacred tradition is at least as likely to iron out inconsistencies as to introduce them.''\footnote{\citealp[p.~11; see also p.~19]{gombrich:how}} To iron out is to make the terrain of ideas looks even. So, ``the banal reading is more likely to replace the oddity than vice versa.''\footnote{\citealp[pp.~11--2]{gombrich:how}}

From the traditional account, the canon is accurate because of the process of ``communal recitation''\footnote{\citealp[pp.~13--4]{payutto:canon}} as the term \emph{sa\.ng\=ayana} literally means. Simultaneous chanting is more accurate than writing\footnote{\citealp[p.~22]{payutto:canon}} because when reciting a sutta together, if one monk chants only a different word, the error can be detected easily. When the correction is made, the process of recital repeats again until no single mistake is found. Then monks memorize this impeccable version. The process explained can address accuracy and inconsistency problem but not reliability. When a picture looks flat, it is unlikely to be real. Real life is more colorful and hectic than that. It is reasonable to see that the recital process is just the final action of approval.\footnote{I think communal recitation is a ritual to make things done, like a stamp. I also think chanting together does not guarantee accuracy, only it sounds harmonious. From my experience nowadays, even from the same source, monks chant \textit{parittas} (certain magical suttas), which are supposed to be well-memorized, in a variety of ways corresponding to the practice of their senior members. Hence, monks from different groups chant slightly different pronunciations. Sometimes the chanting goes wrong against the text. And some monks, even who know P\=ali, recite wrongly all the time. I speculate that if we have all monks write down what they chant regularly, we will have numerous versions of \textit{parittas}. No one ever conducts a research on this, as far as I know.} We have overlooked a more important process than the communal recitation: ``How do all memorized stories come?''

How do monks who have a memory of the same sutta hold exactly the same word sequences? It is unlikely that they had listened to the same source and remembered exactly the same things. No news reports of the same event are alike. When the Buddha preaches to a group of people, do the audience hear and understand the same thing? That is impossible. The same arrangement of words must come form only one source. The origins of the story may have many narrations, but the formal outcome must come from a single source who has a decision power. The tradition ascribes Ven.\,\=Ananda as the source of the Suttanta (the collection of the suttas). As the process goes, I suppose, not everything Ven.\,\=Ananda heard was accepted by the Sa\.ngha. There must be processes of cross-checking, compromising, and unifying until the final version was reached. I suspect democratic atmosphere in such a situation. I think the most powerful person won the arguments. The authority therefore played a major role on producing suttas to be remembered. And religious authority always ties to political authority who sponsors/sanctions the event.\footnote{I do not want to bring politics to our discussion. But from my background of religious studies, considering power relation in religious affairs often bring us a more accurate picture of what is going on or what is really behind the scene.}

Many Buddhists now may feel uncomfortable and contend that monks who are qualified to do the compilation job were all arhants who are unbiased and honest. Being an arhant does not mean one has a perfect memory, or knowledge beyond one's sphere, or a better critical thinking skill, or a better idea of `justice.' Sometimes arhants can do wrong conventionally, be ill-mannered, and be short-sighted.\footnote{An interesting example is about Ven.\,Pi\d n\d dola Bh\=aradv\=aja who displayed psychic power and being rebuked by the Buddha (Cv\,5.252; \citealp[p.~790]{thanissaro:code}).} So, honesty does not help to make the task more reliable. Sometimes people go honestly wrong. We can attribute this as a fallacy of appeal to authority. Arhants are more like just a high-quality stamp in this context.

Modern scientific knowledge can shed some light to this issue, particularly from cognitive science. Studies of the nature of memory can change the way we look at the traditional account. From the common sense widely held by the tradition, memory is like a recorder. When someone hears or sees something with attention, the data are kept in the mind like a video recorder. The story can be recounted or replayed with reliable accuracy. Memory studies suggest that we should give up that naive view.\footnote{The claim that our memory is not like tape or video recorder is made by Elizabeth Loftus, a leading researcher in memory studies. For a quick grasp of her work, see her TED talk ``How reliable is your memory?'' by searching Elizabeth Loftus in \url{www.ted.com}.} In fact, our memory is an active process that ourselves also play a part in memorizing. When we have an experience, ``instead of \emph{reproducing} the original event or story, we derive a \emph{reconstruction} based on our existing presuppositions, expectations and our `mental set'.''\footnote{\citealp[p.~12]{foster:memory} (emphasis in original)} That is to say, our perception is highly selective. Put it bluntly, we hear, see, and remember what we want to hear, see, and remember.

The problem of reliability therefore does not lie on the accuracy of chanting together but rather the acquisition of individual accounts before that. It can be questionable whether monks' memory reflects the real events, or their selective remembering, their wishful accounts, or just their misunderstanding. The tradition explains that Ven.\,\=Ananda memorized all events that are the source of the Suttanta when he was not fully awakened---by definition still has some degree of partiality. How then did Ven.\,\=Ananda get them all right? Normally, when something is said about the one we love dearly, the story usually goes extolled, if not slightly exaggerated.\footnote{In religious studies, there is a notion of attitude towards one's religion that can be either \emph{exclusivist} (my religion is true, yours is false), \emph{inclusivist} (your religion is a part of mine), or \emph{pluralist} (mine and yours are equally true). When a religious canon is read, exclusivist stance is clearly seen. Even in grammatical text like Saddan\=iti, Aggava\d msa states strongly that only words from the Buddha, i.e.\ P\=ali, can lead to the salvation, not from other languages like Sanskrit (\pali{P\=aramit\=anubh\=avena, mahes\=ina\d mva dehato; Santi nipph\=adan\=a, neva, sakkat\=adivaco viya}, Sadd-Pad Ch.\,1; \citealp[p.~8]{smith:sadd1}). If it is so, how about partiality of the narrators?}

If memory is not so creditable as we think, writing down seems better. Unfortunately, writing is not blunder-proof either, because ``every time a text is copied out, errors occurs.''\footnote{\citealp[p.~xxvi]{geiger:grammar}} Nevertheless, it is really better on the point that writing leaves traces on material objects that enable us to do a comparative study as long as the media are not completely destroyed. If everything is in the memory, we can have only the latest version.

To conclude this section, I have to say that I do not want to debunk the authenticity of the P\=ali canon and throw Buddhists into despair. I just apply my critical thinking carefully upon the subject. It is better to know it in all respects, not just believe it and put aside the peculiarities. I think reliability does not matter much, because the P\=ali canon is the best textual material we possess. It is the only thing we have that identifies the world of Theravada Buddhism. Without this we have nothing to say about. The canon is a platform that every Buddhist stands on. It provides a fundamental normative component of numerous Buddhist cultures, a wealth of teaching materials, and an essential source of the answers to existential problems (a kind of who- am-I riddle). It is like a matrix that all Buddhists live in. Steven Collins calls this matrix \emph{P\=ali imaginaire}: ``a mental universe created by and within Pali texts.''\footnote{\citealp[p.~41; see also p.~1]{collins:nirvana}}

\phantomsection
\addcontentsline{toc}{section}{Why do we study P\=ali then?}
\section*{Why do we study P\=ali then?}
I will close this chapter with this question to lead the readers to the coming lessons. If you do not care about P\=ali and see the spiritual aspect of Buddhism is more important, I endorse your view and suggest that you go practicing and do not hold any belief seriously. Do not argue with anyone over words. Just be mindful and keep quiet.\footnote{I have no elaborate system of practice to suggest. My own method is downright simple, ``Shut up, and sit down.''} Once you have a strong belief about a particular concept and want to justify your correctness, you get trapped in a discursive labyrinth. This potentially does harm to your practice.

If you normally deal with texts, studying P\=ali definitely broaden and sharpen your perspective. There is no better way to study ancient texts than reading them in the original language. Translation of the canon is a good place to start learning the religion. But keep in mind that not everything is translatable, and translation needs some personal judgement. Understanding why translators put it in such a way is far more important. The only way to do is to understand P\=ali yourself. I encourage Buddhists to go back to the P\=ali scriptures every time they have a problem with explanations or engage in argumentations. Do not rely totally on any translation, but it can be used as a guideline. I often found that when a translation makes clear in a particular point, the P\=ali itself is uncertain and open to many interpretations. Translation, to me, is a kind of \emph{discourse}\footnote{``a strongly bounded area of social knowledge, a system of statements within which the world can be known'' \citep[p.~83]{ashcroft:postcolonial}} making process, which has things to do with promotion of certain ideology. If you do not want to be a subject of manipulation, learning P\=ali is the best choice.

To put it another way, if you want to understand textual dimension of Buddhism, you have to do some research on the P\=ali canon. I do not claim that you will find the ultimate truth in the text or you will uncover the original message of the Buddha. The only way to find the truth, from any Buddhist tradition, is in your mindful body not in the text. That is outside the scope of this book. Doing research here I mean applying deep analysis and critical thinking over the text. If you want to do textual study, do it rigorously. This is the way you can get real knowledge from the text. I do not say ``don't believe the tradition,'' but rather be careful of logical fallacies, such as appeal to (false) authority, appeal to faith, jumping to conclusions, \textit{non sequitur}, wishful thinking, and many more.\footnote{Concerning the canon study, the Venerable Payutto reminds Buddhists to be careful of ``academic freedom'' under the guise of ``academic research'' \citep[p.~68]{payutto:canon}. I resist this admonition, because academic research is more or less equal to critical analysis, which always brings us some knowledge. Even we may do not like it. The real problem is the hidden agenda behind the research and the quality of the process.} We should think critically why or how the tradition or anyone has certain conclusion about something. By `critical' here I do not mean `criticizing' or `fault finding,' but rather `reasonable' thinking which determines what we believe and do.\footnote{\citealp[p.~32]{ennis:critical}}

Is it will be difficult? If you are very new to the language, certainly it is. But fortunately, nowadays we have several tools to speed up the learning process. We do not need to remember many things like traditional students do. Essential materials, like the texts and dictionaries, are now easily accessed by electronic devices. It takes some time to get the fundamental ideas. Once you grasp the nature of the language, the process of learning will go effortlessly and joyfully. I have never passed any formal course or examination of the language in any level. If I can learn by myself, so can you all.
