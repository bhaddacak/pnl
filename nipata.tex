\chapter{\headhl{Nip\=ata} (Particles)}\label{chap:nipata}

Together with \pali{upasagga}, \pali{nip\=ata} (particle) is counted as indeclinable class of words. Unlike \pali{upasagga} which is normally attached to other words, \pali{nip\=ata} can stand by its own. The good part of it is we do not need to worry about its inflection. We just use it. The bad part of it is many of words in this class have multiple meanings, ranging from very specific to no meaning at all. I have written an introductory part of particles in Chapter \ref{chap:ind-intro}. Please go to that first if you have not yet.

In this appendix, I will summarize particles mentioned in the textbooks, particularly in R\=upa and Sadd. This is meant to be for referencing like other part in the Appendices. In R\=upa, the material is found at the end of \pali{N\=amaka\d n\d da} after sutta no.\,282. In Sadd, it is found in chapter 27 of Sadd-Sut. It is clear that Aggava\d msa elaborates on R\=upasiddhi's material. In my writing I follow mainly the explanations in Sadd with a better arrangement and some additions. Some examples which are not matched well in the scriptures are changed properly.

The content presented by Aggava\d msa on \pali{nip\=ata} is somewhat disorganized, although an attempt to order things can be seen (but it fails nonetheless). To ease the learners, I reorganize the whole things and order them in a more systematic manner. However, some original grouping is still discernable. I very hard try to include all things presented by Aggava\d msa, but some really make no sense to me. So, I neglect some trivial accounts, particularly the uses without any testimony in the canon.

The list presented here is by no means exhaustive, but it has a good coverage. Sometimes the line between whether a term is indeclinable or not is really thin, particularly when it is used as an adverb with acc.\ or ins.\ form. Such a term is used in the same way every time. That makes it looks like an indeclinable one.

I list all groups below, for you can see the big picture first. Towards the end of this Appendix, I also list all particles mentioned to ease your finding.
\bigskip
\begin{longtable}[c]{%
	>{\raggedleft\arraybackslash}p{0.02\linewidth}%
	>{\raggedright\arraybackslash}p{0.5\linewidth}%
	>{\raggedleft\arraybackslash}p{0.2\linewidth}}
\caption*{Groups of particles}\\
\toprule
& \bfseries\upshape Group & \bfseries\upshape Page \\ \midrule
\endfirsthead
\multicolumn{3}{c}{Groups of particles (contd\ldots)}\\
\toprule
& \bfseries\upshape Group & \bfseries\upshape Page \\ \midrule
\endhead
\bottomrule
\ltblcontinuedbreak{3}
\endfoot
\bottomrule
\endlastfoot
%
\ref{nipgrp1}. & Particles with case implied & \pageref{nipgrp1} \\
\ref{nipgrp2}. & General-purpose particles & \pageref{nipgrp2} \\
\ref{nipgrp3}. & Negation, prohibition & \pageref{nipgrp3} \\
\ref{nipgrp4}. & Questioning & \pageref{nipgrp4} \\
\ref{nipgrp5}. & Marking causes & \pageref{nipgrp5} \\
\ref{nipgrp6}. & Expressing doubt & \pageref{nipgrp6} \\
\ref{nipgrp7}. & Emphasizing & \pageref{nipgrp7} \\
\ref{nipgrp8}. & Illustrating & \pageref{nipgrp8} \\
\ref{nipgrp9}. & Setting a boundary & \pageref{nipgrp9} \\
\ref{nipgrp10}. & Responding & \pageref{nipgrp10} \\
\ref{nipgrp11}. & Comparing & \pageref{nipgrp11} \\
\ref{nipgrp12}. & Conditional marking & \pageref{nipgrp12} \\
\ref{nipgrp13}. & Praising, blaming & \pageref{nipgrp13} \\
\ref{nipgrp14}. & Urging & \pageref{nipgrp14} \\
\ref{nipgrp15}. & Repeating & \pageref{nipgrp15} \\
\ref{nipgrp16}. & Disgust & \pageref{nipgrp16} \\
\ref{nipgrp17}. & Fast movement & \pageref{nipgrp17} \\
\ref{nipgrp18}. & Miscellaneous particles & \pageref{nipgrp18} \\
\ref{nipgrp19}. & Fillers & \pageref{nipgrp19} \\
\end{longtable}

\setcounter{nipgrp}{0}
\refstepcounter{nipgrp}\label{nipgrp1}
\section*{\arabic{nipgrp}. Particles with case implied}

As its definition goes, a particle can be used without declension. That is true at the apparent level. However, as I explained elsewhere (see Chapter \ref{chap:ind-intro}), from the tradition's point of view, \pali{nip\=ata}s indeed decline but their declension is deleted. In practice, we do not care about that. We just use particles as their meaning guides us. However, some particles are more case-oriented than others. We will start with these first.

\subsection*{(1) Nominative particles}

They are \pali{atthi} (be existent, productive, effective), \pali{sakk\=a} (be possible, capable), and \pali{labh\=a} (be possible, allowable; may be obtained) which have nominative meaning embedded. Examples are shown below.

\subsection*{\fbox{\pali{Atthi}}}\label{nip:atthi}
It is logical to treat \pali{natthi} in the same way.\par
- \pali{\textbf{atthi} dinna\d m \textbf{atthi} yi\d t\d tha\d m}\footnote{M1\,441 (MN\,41)} (Giving is productive, sacrificing is productive.) \par

\subsection*{\fbox{\pali{Sakk\=a}}}\label{nip:sakkaa}
It is worth noting that this particle is normally used with infinitives or \pali{tu\d m} verbs.\par
- \pali{\textbf{Sakk\=a}, bhikkhave, akusala\d m pajahitu\d m \ldots kusala\d m bh\=avetu\d m}\footnote{A2\,19} (Monks, one is able to abandon unwholesomeness, \ldots to cultivate wholesomeness.) \par

\subsection*{\fbox{\pali{Labbh\=a}}}\label{nip:labbhaa}
This is also used with \pali{tu\d m} verbs.\par
- \pali{\textbf{Labbh\=a} pathav\=i ketu\d m vikketu\d m \=a\d thapetu\d m ocinitu\d m vicinitu\d m}\footnote{Kv\,492} (One may be obtained a land to buy, to sell, to mortgage, to collect, to select.) \par

\subsection*{(2) Accusative particles}
In this group, they are \pali{div\=a} (day), \pali{bhiyyo} (in a higher degree, exceedingly, more), and \pali{namo} (be my adoration to, homage). All these can also have nominative meaning in some contexts.

\subsection*{\fbox{\pali{Div\=a}}}\label{nip:divaa}
\paragraph*{\pali{Div\=a} as nominatives} For example:\par
- \pali{ratti\d myeva sam\=ana\d m \textbf{div\=a}ti sa\~nj\=ananti}\footnote{M1\,50 (MN\,4)} ([They] recognize the night as the day.) \par
\paragraph*{\pali{Div\=a} as accusatives} For example:\par
- \pali{\textbf{div\=a}yeva sam\=ana\d m ratt\=iti sa\~nj\=ananti}\footnote{M1\,50 (MN\,4)} ([They] recognize the day as the night.) \par

\subsection*{\fbox{\pali{Bhiyyo}}}\label{nip:bhiyyo}
\paragraph*{\pali{Bhiyyo} as nominatives} For example:\par
- \pali{uppajjati sukha\d m, sukh\=a \textbf{bhiyyo} somanassa\d m}\footnote{D2\,288 (DN\,18)} (Happiness arises, more delight than happiness [arises].) \par
\paragraph*{\pali{Bhiyyo} as accusatives} For example:\par
- \pali{\textbf{bhiyyo} pallomam\=ap\=adi\d m ara\~n\~ne vih\=ar\=aya}\footnote{M1\,35 (MN\,4)} ([I am] firmly not made the hair stand for staying in the forest.) \par

\subsection*{\fbox{\pali{Namo}}}\label{nip:namo}
\paragraph*{\pali{Namo} as nominatives} For example:\par
- \pali{\textbf{Namo} te buddha v\=iratthu, vippamuttosi sabbadhi}\footnote{S1\,90 (SN\,2)} (Be my adoration to the Buddha, the Brave One, who is set free [from defilement] in all [objects].) \par
\paragraph*{\pali{Namo} as accusatives} For example:\par
- \pali{\textbf{namo} karohi n\=agassa}\footnote{M1\,249 (MN\,23)} (Pay a homage to the serpent.) \par

\subsection*{(3) Instrumental particles}

We have around a dozen of these, i.e.\ \pali{saha, vin\=a, saddhi\d m, saya\d m, sama\d m, s\=ama\d m, samm\=a, micch\=a, sakkhi, paccatta\d m, kinti,} and \pali{to, so, dh\=a} as indeclinable suffixes. Also \pali{rite} and \pali{rahit\=a} in the meaning of \pali{vin\=a} are mentioned later. Some of particles in this group, we have met earlier in Chapter \ref{chap:ins}.

\subsection*{\fbox{\pali{Saha}}}\label{nip:saha}
This means `with' or `together with' or `in the presence of.'\par
- \pali{sa\.ngho \textbf{saha} v\=a gaggena vin\=a v\=a gaggena uposatha\d m kareyya}\footnote{Mv\,2.167} (The Sangha should do the recitation together with monk Gagga, or without monk Gagga.) \par
- \pali{\textbf{saha} vatthebhi sobhati}\footnote{M2\,302 (MN\,82)} (He looks beautiful with cloth.) \par

\subsection*{\fbox{\pali{Saddhi\d m}}}\label{nip:saddhidm}
This means exactly the same as \pali{saha} and can be used interchangeably.\par 
- \pali{mahat\=a bhikkhusa\.nghena \textbf{saddhi\d m}}\footnote{passim, e.g.\ Buv1\,1} (together with a big group of monks) \par

\subsection*{\fbox{\pali{Vin\=a}}}\label{nip:vinaa}
This means `without' or `in the absence of,' or in the meaning of `with the exception of' like `other than' or `besides.' Also \pali{a\~n\~natra} can have this meaning (see the miscellaneous group).\par
- \pali{sa\.ngho saha v\=a gaggena \textbf{vin\=a} v\=a gaggena uposatha\d m kareyya} \par
- \pali{\textbf{vin\=a} saddhamm\=a nattha\~n\~no koci n\=atho loke vijjati}\footnote{Kacc\,272. It is said in this sutta that by this sense it can be used with abl.\ or acc.\ or ins. Thus, \pali{vin\=a saddhamma\d m} and \pali{vin\=a saddhammena} are also valid.} (Besides the true teaching, any other protector in the world does not exist.) \par
In rare cases, \pali{vin\=a} takes the ablative, for example:\par
- \pali{\~N\=atisa\.ngh\=a vin\=a hoti}\footnote{Snp\,3.594} ([One] is departed from relatives.)\par

\subsection*{\fbox{\pali{Rite}}}\label{nip:rite}
This is a synonym of \pali{vin\=a}.\par
- \pali{\textbf{Rite} saddhamm\=a kuto sukha\d m labhati}\footnote{Kacc\,272. Also \pali{rite saddhamma\d m} and \pali{rite saddhammena} can be used.} (Without the true teaching, where does one get happiness?) \par

\subsection*{\fbox{\pali{Rahit\=a}}}\label{nip:rahitaa}
This is another synonym of \pali{vin\=a}.\par
- \pali{\textbf{Rahit\=a} m\=atuj\=a pu\~n\~na\d m katv\=a d\=ana\d m deti}\footnote{Kacc\,272. Also \pali{rahit\=a m\=atuja\d m} and \pali{rahit\=a m\=atujena} can be used.} (The one without mother, having made merit, gives alms.) \par

\subsection*{\fbox{\pali{Saya\d m}}}\label{nip:sayadm}
This means `by oneself.'\par
- \pali{\textbf{saya\d m} abhi\~n\~n\=aya kamuddiseyya\d m}\footnote{Dhp\,24.353} (By knowing by myself, who should I point out [as my teacher]?) \par

\subsection*{\fbox{\pali{Sama\d m}}}\label{nip:samadm}
This means `equally.'\par
- \pali{sahassena \textbf{sama\d m} mit\=a}\footnote{S1\,32 (SN\,1); Ja\,10:131} ([verses] equally counted by a thousand) \par

\subsection*{\fbox{\pali{S\=ama\d m}}}\label{nip:saamadm}
This means, like \pali{saya\d m}, `by oneself.'\par
- \pali{\textbf{S\=ama\d m} sacc\=ani bujjhitv\=a}\footnote{Ap1\,1:341} (having known the [noble] truths by himself) \par

\subsection*{\fbox{\pali{Samm\=a}}}\label{nip:sammaa}
This means `properly' or `rightly' or `thoroughly.'\par
- \pali{Ye eva\d m j\=ananti, te \textbf{samm\=a} j\=ananti;}\footnote{M3\,301 (MN\,136)} (Those who know in this way know it rightly.) \par

\subsection*{\fbox{\pali{Micch\=a}}}\label{nip:micchaa}
In contrast with \pali{samm\=a}, this means `wrongly.'\par
- \pali{ye a\~n\~nath\=a j\=ananti, \textbf{micch\=a} tesa\d m \~n\=a\d na\d m}\footnote{M3\,301 (MN\,136)} (Those who know in a different way know it wrongly.) \par

\subsection*{\fbox{\pali{Sakkhi}}}\label{nip:sakkhi}
This means ``before one's eyes'' or ``by one's own eyes'' or generally like ``by oneself.''\par
- \pali{S\=aha\d m d\=ani \textbf{sakkhi} j\=an\=ami, munino desayato sugatassa}\footnote{S1\,39 (SN\,1)} (Now I know by myself [the teaching] of the Buddha preached.) \par

\subsection*{\fbox{\pali{Paccatta\d m}}}\label{nip:paccattadm}
This means `individually' or `separately' or generally like ``by oneself.''\par
- \pali{\textbf{paccatta\d m} veditabbo vi\~n\~n\=uhi}\footnote{D2\,290 (DN\,18)} ([This teaching] shall be known by the wise individually.) \par

\subsection*{\fbox{\pali{Kinti}}}\label{nip:kinti}
This means `how' or `by which way.'\par
- \pali{\textbf{kinti}me saddh\=aya va\d d\d dheyyu\d m}\footnote{D3\,224 (DN\,30)} (By which way should my [disciples] grow by faith?) \par

\subsection*{\fbox{\pali{To}}}\label{nip:to1}
This is not an independent word. It is an indeclinable suffix added to nouns to make them having instrumental meaning. Other cases can also be the case as well, see Chapter \ref{chap:ind-to} for more detail.\par
- \pali{anicca\textbf{to}} (by impermanent quality) \par
- \pali{dukkha\textbf{to}} (by suffering) \par
- \pali{roga\textbf{to}} (by illness) \par

\subsection*{\fbox{\pali{So}}}\label{nip:so1}
Like \pali{-to}, for some nouns \pali{-so} is preferred, for example:\par
- \pali{sutta\textbf{so}} (by discourse) \par
- \pali{pada\textbf{so}} (by term) \par

\subsection*{\fbox{\pali{Dh\=a}}}\label{nip:dhaa}
Yet some nouns work well with \pali{-dh\=a}, for example:\par
- \pali{eka\textbf{dh\=a}} (by one side/part) \par
- \pali{dvi\textbf{dh\=a}} (by two sides/parts) \par

\subsection*{(4) Dative particles}\label{nip:tudm}\label{nip:tave}

In this group, there is no individual particle mentioned, but two indeclinable suffixes are given, i.e.\ \pali{tu\d m} and \pali{tave}. We can see these two \pali{paccaya}s in primary derivation (see Appendix \ref{chap:kita}, page \pageref{par:kitatum}). In English point of view, the products of these sound like infinitive (see Chapter \ref{chap:inf}). These forms of verbs normally come with certain verbs, e.g.\ \pali{sakkoti} (be able to). Some quick examples are:\par
- \pali{k\=a\textbf{tave} sakkoti.} ([One] is able to do.) \par
- \pali{d\=a\textbf{tave} sakkoti.} ([One] is able to give.) \par
- \pali{d\=a\textbf{tu\d m} sakkoti.} ([One] is able to give.) \par
- \pali{vivece\textbf{tu\d m} sakkoti.} ([One] is able to separate oneself [from].) \par
- \pali{vivec\=ape\textbf{tu\d m} sakkoti.} ([One] is able to cause [someone] to separate oneself [from].) \par
- \pali{vinode\textbf{tu\d m} sakkoti.} ([One] is able to dispel.) \par
- \pali{vinod\=ape\textbf{tu\d m} sakkoti.} ([One] is able to cause [someone] to dispel.) \par

\subsection*{(5) Ablative particles}

For this group, only two indeclinable suffixes are given, i.e.\ \pali{to} and \pali{so}.

\subsection*{\fbox{\pali{To}}}\label{nip:to2}
This suffix is shared with instrumental meaning. The context will guide you what the proper case should be, for example:\par
- \pali{m\=ati\textbf{to} ca piti\textbf{to} ca sa\d msuddhagaha\d niko}\footnote{Buv2\,499; D1\,303 (DN\,4)} ([He is] of pure descent from [both] maternal side and paternal side.) \par

\subsection*{\fbox{\pali{So}}}\label{nip:so2}
Apart from instrumental meaning, this suffix can be used as ablatives, for example:\par
- \pali{d\=igha\textbf{so}} (from the long side) \par
- \pali{ora\textbf{so}} (from the near shore) \par

\subsection*{(6) Locative particles}

Suffixes used for locative meaning are several, including \pali{to, tra, tha,} etc. There are also a number of individual terms used in locative sense as well. For more detail on these indeclinable suffixes, see Chapter \ref{chap:ind-to}.\par

\subsection*{\fbox{\pali{To}, etc.}}\label{nip:to3}
For example:\par
- \pali{eka\textbf{to}} (in one side) \par
- \pali{pura\textbf{to}} (in the front) \par
- \pali{paccha\textbf{to}} (behind) \par
- \pali{passa\textbf{to}} (in the side) \par
- \pali{pi\d t\d thi\textbf{to}} (in the back) \par
- \pali{s\=isa\textbf{to}} (in the head) \par
- \pali{p\=ada\textbf{to}} (in the foot) \par
- \pali{agga\textbf{to}} (in the top) \par
- \pali{m\=ula\textbf{to}} (in the root) \par
- \pali{ya\textbf{tra}/ya\textbf{tha}/ya\textbf{hi\d m}} (in which place) \par
- \pali{ta\textbf{tra}/ta\textbf{tha}/ta\textbf{hi\d m}} (in that place) \par
- \pali{k\textbf{va}/ku\textbf{hi\d m}/ku\textbf{ha\d m}/ka\textbf{ha\d m}/ku\textbf{hi\~ncana\d m}/} (in where?) \par

\subsection*{\fbox{\pali{Ko}}}\label{nip:ko}
This can be used as `where' regardless of its apparent form.\par
- \pali{\textbf{Ko} te bala\d m mah\=ar\=aja, \textbf{ko} nu te rathama\d n\d dala\d m}\footnote{Ja\,22:1880} (Your Majesty, where is your strength? Where is your chariot?) \par

\subsection*{\fbox{In place}}\label{nip:inplace}
There are a number of particles used in locative sense of place, for example:\par
- \palibf{samant\=a} (everywhere) \par
- \palibf{s\=amant\=a} (in a near place) \par
- \palibf{samantato} (in the surrounding area) \par
- \palibf{parito} (in the surrounding area) \par
- \palibf{abhito} (in the inner area) \par
- \palibf{ekajjha\d m} (in one part/area) \par
- \palibf{ekamanta\d m} (in one proper area) \par
- \palibf{he\d t\d th\=a} (beneath) \par
- \palibf{upari} (on the upper part) \par
- \palibf{uddha\d m} (on the upper part) \par
- \palibf{adho} (in the lower part) \par
- \palibf{tiriya\d m} (in a crosswise direction) \par
- \palibf{sammukh\=a} (in the front, in face to face) \par
- \palibf{paramukh\=a} (in one's absence) \par
- \palibf{\=avi} (in open area, in visible manner) \par
- \palibf{raho} (in close area, in secret) \par
- \palibf{ucca\d m} (in a high place) \par
- \palibf{n\=ica\d m} (in a low place) \par
- \palibf{tiro} (in the outer side) \par
- \palibf{anto} (in the inner side) \par
- \palibf{antar\=a} (in between) \par
- \palibf{ajjhatta\d m} (inside oneself) \par
- \palibf{bahiddh\=a/bahi} (outside oneself) \par
- \palibf{b\=ahir\=a/b\=ahira\d m} (in outside) \par
- \palibf{ora\d m} (in the near shore) \par
- \palibf{p\=ara\d m} (in the far shore) \par
- \palibf{\=ar\=a/\=arak\=a} (in a far place) \par
- \palibf{pacch\=a} (behind) \par
- \palibf{pure} (in the front) \par
- \palibf{hura\d m} (in other world) \par
- \palibf{pecca} (in the next being) \par
- \palibf{ap\=ac\=ina\d m} (in the sounth) \par

\subsection*{\fbox{In time}}\label{nip:intime}
Comparable to the sense of place, these particles are used in time, for example:\par
- \palibf{sampati} (in the present time, now) \par
- \palibf{\=ayati\d m} (in next time) \par
- \palibf{ajju} (on today) \par
- \palibf{sajja/sajju} (on today) \par
- \palibf{aparajju} (on the other day [= tomorrow]) \par
- \palibf{sve/suve} (on tomorrow) \par
- \palibf{uttarasve/uttarasuve/parasuve} (on the day after tomorrow) \par
- \palibf{hiyyo} (on yesterday) \par
- \palibf{pare} (on other day) \par
- \palibf{s\=aya\d m} (in the evening) \par
- \palibf{p\=ato} (in the morning) \par
- \palibf{k\=ala\d m} (in the early morning) \par
- \palibf{div\=a} (in the day) \par
- \palibf{ratti} (in the night) \par
- \palibf{nicca\d m} (in a perpetual manner, always) \par
- \palibf{satata\d m} (in a perpetual manner, always) \par
- \palibf{abhi\d nha\d m} (in frequent time, often) \par
- \palibf{abhikkha\d na\d m} (in frequent time, often) \par
- \palibf{muhu\d m} (in frequent time, often) \par
- \palibf{muhutta\d m} (in a moment) \par
- \palibf{bh\=utapubba\d m} (in the past) \par
- \palibf{pur\=a} (in the past) \par
- \palibf{yad\=a} (in which time) \par
- \palibf{tad\=a/tad\=ani} (in that time) \par
- \palibf{etarahi} (in this time) \par
- \palibf{adhun\=a} (in this time, in just a moment ago) \par
- \palibf{id\=ani} (in this time) \par
- \palibf{kad\=a} (in what time?) \par
- \palibf{kud\=acana\d m} (in whatever time) \par
- \palibf{sabbad\=a} (in all time) \par
- \palibf{sad\=a} (in all time) \par
- \palibf{a\~n\~nad\=a} (in other time) \par
- \palibf{ekad\=a} (in one time, sometimes) \par

\subsection*{(7) Vocative particles}\label{nip:voc}

Some particles are use only in addressing like vocative nouns. Most of them can be used in both singular and plural sense. For more detail, see Chapter \ref{chap:vockim}. Here are some examples:\par
- \palibf{\=avuso} --- used to address an equal or inferior person \par
- \palibf{ambho/hambho} --- used to address an equal or inferior person \par
- \pali{\textbf{ambho} purisa, ki\d m tuyhimin\=a p\=apakena dujj\=ivitena, mata\d m te j\=ivit\=a seyyo}\footnote{Buv1\,171} (Man, what [is the use] for you with this evil, miserable life? Dying from life is better for you.) \par
- \palibf{bha\d ne} --- used to address an equal or inferior person \par
- \palibf{hare/are/re} --- used to address an equal or inferior person in a less polite way \par
- \pali{\textbf{hare} sakh\=a kissa nu ma\d m jah\=asi}\footnote{Ja\,6:94} (Hey! friend, why do you abandon me?) \par
- \palibf{je} --- used to address a female servant \par
- \pali{Sace, \textbf{je}, tva\d m sacca\d m bha\d nasi, ad\=asi\d m ta\d m karomi}\footnote{Buv1\,31} (Slave, if what you say is true, I will free you.) \par
- \palibf{he} --- used to address an inferior person, animal, or thing \par
- \palibf{bho} --- general term to address people or things \par
- \pali{\textbf{bho} puriso} (Man sir) \par
- \pali{\textbf{bho} dhutt\=a} (Hey rascals) \par
- \pali{\textbf{bho} yakkh\=a} (Hey demons) \par
- \pali{ummujja, \textbf{bho} puthusile}\footnote{S4\,358 (SN\,42)} (You dense stone! rise up.) \par
- \pali{gacchatha \textbf{bho} ghara\d niyo}\footnote{In this case \pali{bho} is indeclinable, so it can be in both m.\ and f, sg.\ and pl.} (House-wives, you may go.) \par
- \pali{Ehi \textbf{samma}\footnote{It is explained that \pali{samma, samm\=a, m\=arisa, m\=aris\=a} are counted as particles, because other forms of these terms are not found.} nivattassu}\footnote{Ja\,2:5} (Sir, come and turn back.) \par
- \pali{M\=a \textbf{samm}eva\d m avacuttha}\footnote{Ja\,22:2321. Aggava\d msa analyzes this as \pali{m\=a samm\=a eva\d m \dots}} (Sirs, do not say that.) \par
- \pali{Sace, \textbf{m\=aris\=a}, dev\=ana\d m sa\.ng\=amagat\=ana\d m uppajjeyya bhaya\d m v\=a chambhitatta\d m v\=a lomaha\d mso v\=a.}\footnote{S1\,249 (SN\,11)} (Sirs, if fear, shock, or hair-raising happen to deities in the war.) \par

\refstepcounter{nipgrp}\label{nipgrp2}
\section*{\arabic{nipgrp}. General-purpose particles}

In this group, frequently used particles with a variety of application are described.

\subsection*{\fbox{\pali{Atha}}}\label{nip:atha}
This particle can be used for several things as follows:
\paragraph*{\pali{Atha} in questioning} For example:\par
\begin{quote}
\pali{Atha tva\d m kena va\d n\d nena, kena v\=a pana hetun\=a;}\\
\pali{Kena v\=a atthaj\=atena, att\=ana\d m parimocayi.}\footnote{Ja\,22:774}\\[1.5mm]
``With what reason, you [all] are set free?''
\end{quote}
\paragraph*{\pali{Atha} as `then'} This means `after that' (without intermission). Also \pali{atho} can be used in this way, for example:\par
- \pali{\textbf{atha} na\d m \=aha} (Then [he] said to that [person].) \par
\paragraph*{\pali{Atha} as `continuously'} For example:\par
- \pali{\textbf{Atha} kho bhagav\=a rattiy\=a pa\d thama\d m y\=ama\d m pa\d ticcasamupp\=ada\d m anulomapa\d tiloma\d m manas\=ak\=asi}\footnote{Mv\,1.1} (The Buddha reflected on dependent origination in forward and backward direction throughout the first third of the night.) \par
\paragraph*{\pali{Atha} as `another section'} For example:\par
- \pali{\textbf{atha} pubbassaralopo} ([another] section on deletion of preceding vowel) \par
\paragraph*{\pali{Atha} as `next, later'} For example:\par
- \pali{\textbf{Atha} dakkhisi bhaddante, nigrodha\d m madhupipphala\d m}\footnote{Ja\,22:1906} (Sir, next you will see a banyan tree with sweet fruits.) \par
\paragraph*{\pali{Atha} as a filler} For example:\par
- \pali{\textbf{atha} puriso \=agaccheyya}\footnote{M3\,156 (MN\,119)} (A person should come.) \par

\subsection*{\fbox{\pali{Eva\d m}}}\label{nip:evadm}
\paragraph*{\pali{Eva\d m} in illustrating} For example:\par
- \pali{\textbf{evam}pi te mano}\footnote{D1\,485 (DN\,11)} (Your mind also thinks in this way.) \par
\paragraph*{\pali{Eva\d m} in responding} For example:\par
- \pali{``\textbf{Eva\d m}, bhante''ti kho te bhikkh\=u bhagavato pa\d tissu\d nitv\=a}\footnote{Mv\,8.349} (Having agreed with the Buddha, those monks [say] ``Yes, sirs.'') \par
\paragraph*{\pali{Eva\d m} in comparing} For example:\par
- \pali{\textbf{Eva\d m} vijitasa\.ng\=ama\d m, satthav\=aha\d m anuttara\d m}\footnote{S1\,215 (SN\,8)} ([Disciples tend the Buddha,] the Incomparable One who like a caravan leader who win the war.) \par
\begin{quote}
\pali{Yath\=api pupphar\=asimh\=a, kayir\=a m\=al\=agu\d ne bah\=u;}\\
\pali{\textbf{Eva\d m} j\=atena maccena, kattabba\d m kusala\d m bahu\d m.}\footnote{Dhp\,4.53}\\[1.5mm]
``Like [a florist] makes many garlands from a heap of flowers,''\\
``Thus [one], with birth and death, should do many wholesome [deeds].''
\end{quote}
\paragraph*{\pali{Eva\d m} in instructing} For example:\par
- \pali{\textbf{eva\d m} te abhikkamitabba\d m, \textbf{eva\d m} te pa\d tikkamitabba\d m} (Stepping forward should be done by you in this way, stepping backward should be done by you in this way.) \par
\paragraph*{\pali{Eva\d m} in encouraging} For example:\par
- \pali{\textbf{evam}eta\d m, bhagav\=a, \textbf{evam}eta\d m, sugata}\footnote{D1\,241--2, 2.8.357 (DN\,9, 21)} (Sir, that is so, the Blessed One, that is so.) \par
\paragraph*{\pali{Eva\d m} in blaming} For example:\par
- \pali{\textbf{evam}eva\d m pan\=aya\d m vasal\=i yasmi\d m v\=a tasmi\d m v\=a tassa mu\d n\d da\-kassa sama\d nassa va\d n\d na\d m bh\=asati.}\footnote{S1\,187 (SN\,7)} (This outcast [woman] talks about quality of that bald ascetic in everywhere as such.) \par
\paragraph*{\pali{Eva\d m} in manner (\pali{\=ak\=are})} For example:\par
- \pali{\textbf{Eva\d m}by\=akho aha\d m, \=avuso, bhagavat\=a dhamma\d m desita\d m \=aj\=an\-\=ami}\footnote{M1\,234 (MN\,22). There is a note that \pali{eva\d mby\=akho} may be in fact \pali{eva\d m kho}. In Buv1\,24, for instance, \pali{yath\=a} is used in stead of \pali{eva\d m}.} (Like this, friend, I understand the teaching pointed out by the Blessed One.) \par
\paragraph*{\pali{Eva\d m} in showing an example (\pali{nidassane})} For example:\par
- \pali{\textbf{Eva\~n}ca vadehi, `s\=adhu kira bhava\d m \=anando yena subhassa m\=a\d navassa todeyyaputtassa nivesana\d m tenupasa\.nkamatu anukampa\d m up\=ad\=ay\=a'ti.}\footnote{D1\,445 (DN\,10)} ([You] say like this [to Ven.\ \=Ananda], ``Venerable \=Ananda please does me a favor by helping me approach to the place of Subha, the young son of Todeyya.'') \par
\paragraph*{\pali{Eva\d m} as \pali{avadh\=ara\d na}} This means like ``only this, not others'' (see also particle \pali{no} below), for example:\par
- \pali{Samatt\=a, bhante, sam\=adinn\=a ahit\=aya dukkh\=aya sa\d mvattant\=iti. \textbf{Eva\d m} no ettha hoti.}\footnote{A3\,66} (Sir, ``all these [practices] taken upon leads to no benefit, to suffering.'' In this, we [understand] thus.) \par

\subsection*{\fbox{\pali{Ca}}}\label{nip:ca}
This is one of the most used particles. It is mainly used to denote conjunction, or connecting two things together, like `and' in English. It will be never at the beginning of sentences. It can be used in a variety of ways as explained below.
\paragraph*{\pali{Ca} in connecting words} For example: \par
- \pali{Mitt\=amacc\=a \textbf{ca} bhacc\=a \textbf{ca}, puttad\=ar\=a \textbf{ca} bandhav\=a.}\footnote{Ja\,21:31} \\(Friends \& colleagues, dependants, children \& wife, and relatives.) \par
\paragraph*{\pali{Ca} in connecting sentences with different verbs} For example: \par
- \pali{d\=ana\~n\textbf{ca} dehi, s\=ila\~n\textbf{ca} rakkh\=ahi.} (Give alms and observe the precept.) \par
\paragraph*{\pali{Ca} in connecting sentences with the same verb} For example: \par
- \pali{sama\d no \textbf{ca} ti\d t\d thati, br\=ahma\d no \textbf{ca} ti\d t\d thati.} (An ascetic stands, also a brahman stands.) \par
\paragraph*{\pali{Ca} in connecting words in analytic part of compounds} For example: \par
- \pali{s\=ita\~n\textbf{ca} u\d nha\~n\textbf{ca} s\=itu\d nha\d m.} (cool and hot [thus] \pali{s\=itu\d nha}.) \par
\paragraph*{\pali{Ca} in contrasting} In some cases, \pali{ca} is used to contrast two situations, like `but' or `however' in English. This use is equivalent to \pali{pana}. For example:\par
\begin{quote}
\pali{Na ve kadariy\=a devaloka\d m vajanti,}\\
\pali{b\=al\=a have nappasa\d msanti d\=ana\d m;}\\
\pali{Dh\=iro \textbf{ca} d\=ana\d m anumodam\=ano,}\\
\pali{teneva so hoti sukh\=i parattha.}\footnote{Dhp\,13.177}\\[1.5mm]
``Misers do not go to heaven,''\\
``Foolish persons indeed do not praise giving;''\\
``Wise persons, however, rejoicing in giving,''\\
``Become happy in the afterlife.''\\
\end{quote}
\paragraph*{\pali{Ca} as a filler} Sometimes \pali{ca} means nothing, just a space filler, for example.\par
- \pali{Ki\~n\textbf{ca}, bhikkhave, r\=apa\d m vadetha?}\footnote{S3\,79 (SN\,22)} (Monks, why do you call `form'?) \par

\subsection*{\fbox{\pali{Pana}}}\label{nip:pana}
This one is also frequently used, but pinning down what it exactly means is difficult. It is used in various ways, often with other particles. I show you only some common uses below.
\paragraph*{\pali{Pana} in contrasting} This can be equivalent to `but' or `whereas' or `on the other hand' in English, for example:\par
- \pali{Sudassa\d m vajjama\~n\~nesa\d m, attano \textbf{pana} duddasa\d m}\footnote{Dhp\,18.252} (Others' fault is easily seen, but one's own [fault] is hard to see.) \par
- \pali{Duss\=ilo \textbf{pana} mittehi, dha\d msate p\=apam\=acara\d m.}\footnote{Thag\,12.610} (On the other hand, an immoral person, usually doing evil things, breaks from friends.) \par
- \pali{atthakath\=aya\d m \textbf{pana} vutta\d m khal\=uti eko saku\d no}\footnote{Vism\,2.23} \\([Whereas] in the commentary, it is said that `\pali{khalu}' means a kind of bird.) \par
\paragraph*{\pali{Pana} as a filler} For example:\par
- \pali{Ki\d m \textbf{pana} bhava\d m gotamo daharo ceva j\=atiy\=a, navo ca pabbajj\=aya.}\footnote{S1\,112 (SN\,3)} (Why does Gotama, [as he is] young by birth, and new by ordination, [say he is the Buddha]?) \par
- \pali{Accantasant\=a \textbf{pana} y\=a, aya\d m nibb\=anasampad\=a;}\footnote{Vism\,1.21} (This attainment of nirvana [is] the absolute peace.) \par

\subsection*{\fbox{\pali{Pi, api}}}\label{nip:pi}\label{nip:api}
It is explained that we normally do not start a sentence with \pali{pi}, but we can with \pali{api}. Sometimes this comes with other particles and is used in an idiomatic way, for example, \pali{api nu} is used in questioning, just means like simple \pali{nu}; \pali{api ca} means `but.'
\paragraph*{\pali{Pi, api} as `even'} We find these quite often, for example:\par
- \pali{Bahum\textbf{pi} ce sa\d mhita bh\=asam\=ano, na takkaro hoti naro pamatto}\footnote{Dhp\,1.19} (A careless person, even reciting a lot of teaching, does not become the doer [of that teaching].) \par
- \pali{Dutiyam\textbf{pi} kho sudinno kalandaputto m\=at\=apitaro etadavoca}\footnote{Buv1\,26} (Even in the second time, Sudinna, the son of Kalanda, said to the parents.) \par
- \pali{chinno\textbf{pi} rukkho punareva r\=uhati}\footnote{Dhp\,24.338} (Even being cut, a tree grows again.) \par
- \pali{aham\textbf{pi} kho, bhikkhu, na j\=an\=ami, yatthime catt\=aro mah\=abh\=ut\=a aparises\=a nirujjhanti}\footnote{D1\,491 (DN\,11)} (Even I, monk, do not know where these four great elements completely cease.) \par
\paragraph*{\pali{Pi, api} in conjunction} This means `also' or `too.' Sometimes it sounds like \pali{ca}, for example:\par
- \pali{Bhikkh\=u uposatha\d m \=agacchant\=a uddissam\=ane\textbf{pi} p\=atimokkhe \=agacchanti, uddi\d t\d thamatte\textbf{pi} \=agacchanti, antar\=a\textbf{pi} parivasanti.}\footnote{Mv\,2.140} (Monks, coming to the Vinaya recitation, [some] come while reciting, [some] come at the end, and [some] are [still] on the way.) \par
\paragraph*{\pali{Pi, api} in contrasting} This is often accompanied with \pali{ca}, for example:\par
- \pali{Ahampi kho te, bha\d ne j\=ivaka, m\=atara\d m na j\=an\=ami; \textbf{api} c\=aha\d m te pit\=a; may\=asi pos\=apito}\footnote{Mv\,8.328} (My dear J\=ivaka, even though I do not know your mother, but I am your father, [because you was] fed by me.) \par

\subsection*{\fbox{\pali{Yath\=a}}}\label{nip:yathaa1}
We can find this particle in a variety of use as described below. When it comes together with \pali{tath\=a}, the pair can be used in comparing. See also the group of comparing below.
\paragraph*{\pali{Yath\=a} as `very much'} This sounds like a stress, for example:\par
- \pali{\textbf{Yath\=a} aya\d m nimir\=aj\=a, pa\d n\d dito kusalatthiko}\footnote{Ja\,22:442} (This Nimir\=aja is very much of a wise man, seeking wholesomeness.) \par
\paragraph*{\pali{Yath\=a} as `properly'} For example:\par
- \pali{\textbf{yath\=a}r\=upa\d m upasa\d mharati}\footnote{This example is given by Aggava\d msa. It seems that \pali{yath\=ar\=upa\d m} is used as a unit meaning like ``in the way mentioned.''} ([One] concentrates properly.) \par
\paragraph*{\pali{Yath\=a} in repeating (\pali{vicch\=a})} For example:\par
- \pali{ye ye vu\d d\d dh\=a v\=a \textbf{yath\=a}vu\d d\d dha\d m}\footnote{Kacc\,319} (Whoever, whoever are elderly, thus \pali{yath\=avu\d d\d dha}.) \par
\paragraph*{\pali{Yath\=a} in succession} For example:\par
- \pali{vu\d d\d dh\=ana\d m pa\d tip\=a\d ti \textbf{yath\=a}vu\d d\d dha\d m}\footnote{Kacc\,319} (Order of the elderly is \pali{yath\=avu\d d\d dha}) \par
\paragraph*{\pali{Yath\=a} as `respectively'} For example:\par
- \pali{\=Aki\~nca\d m nevasa\~n\~na\~nca, sam\=apajji \textbf{yath\=a}kkama\d m}\footnote{Ap2\,2:245} \\([Then she] engaged in the 3rd and the 4th formless state respectively.) \par
\paragraph*{\pali{Yath\=a} as a filler} 
This normally means `like, in relation to, according to, in whatever way.' In the example given by Aggava\d msa below, it is hard to say the term is just a filler.\par
- \pali{\textbf{yath\=a} katha\d m pana bhante bhagavati brahmacariya\d m vussati?} (How, sir, [one is allowed] to practice the religious life in [the guidance of] the Buddha?) \par
\paragraph*{\pali{Yath\=a} in illustrating} This use is found in grammar textbooks, for example:\par
- \pali{Ko gassa, \textbf{yath\=a}? Kul\=upako}\footnote{Kacc\,20} (For `\pali{g}' [change it to] `\pali{k},' like what? \pali{Kul\=upako}.) \par

\subsection*{\fbox{\pali{V\=a}}}\label{nip:vaa}
This one is also a top-five particle. It is used mainly for disjunction, or alternative options. Like \pali{ca}, we do not start a sentence with \pali{v\=a}. There is also a nuaunce of meaning explained below.
\paragraph*{\pali{V\=a} in disjunction} This means you have to choose only one option from many, for example:\par
- \pali{so gandhabbo khattiyo \textbf{v\=a} br\=ahma\d no \textbf{v\=a} vesso \textbf{v\=a} suddo \textbf{v\=a}?}\footnote{M3\,411 (MN\,93)} (Is that spirit of the warrior caste, the priestly caste, the merchant caste, or the worker caste?) \par
\paragraph*{\pali{V\=a} in conjunction} This works like `and.' It can be seen as inclusive or, so multiple options can be chosen, for example:\par
- \pali{P\=a\d taliputtassa kho, \=ananda, tayo antar\=ay\=a bhavissanti, aggito \textbf{v\=a} udakato \textbf{v\=a} mithubhed\=a \textbf{v\=a}.}\footnote{D2\,152 (DN\,16)} (\=Ananda, three dangers will happen to P\=a\d taliputta, from fire, from flood, and from breaking of alliance.) \par
\paragraph*{\pali{V\=a} in simile} When no option is presented, it can mean `like,' for example:\par
- \pali{Madhu\textbf{v\=a} ma\~n\~nati b\=alo, y\=ava p\=apa\d m na paccati;}\footnote{Dhp\,5.69} (A foolish person deems [evil] as honey, as far as the evil result is not yielded.) \par
\paragraph*{\pali{V\=a} as a filler} Like \pali{ca}, in some contexts this can mean nothing, for example.\par
- \pali{aya\d m \textbf{v\=a} so mah\=an\=ago}\footnote{This instance is suspicious. I find ``\pali{ayameva so mah\=an\=ago}'' in M1\,291 (MN\,27).} (That big elephant [is] this one.) \par

\refstepcounter{nipgrp}\label{nipgrp3}
\section*{\arabic{nipgrp}. Negation, prohibition}\label{nip:neg}

Particles in this group are \pali{na, no, m\=a, a, ala\d m,} and \pali{hala\d m}. They normally make things negative. Later \pali{ya\~nce} is mentioned to be one of these. There are also other particles that can be used in this sense, e.g.\ \pali{khalu} (see Miscellaneous group below).

\subsection*{\fbox{\pali{Na}}}\label{nip:na}
This one is also a top-five. We frequently use this, if not always, to negate the meaning of almost everything. Aggava\d msa says that \pali{na} is placed either at the beginning or the end of sentences. This account is questionable to me. Here are some examples:\par
- \pali{\textbf{Na} c\=aha\d m pa\d n\d na\d m bhu\~nj\=ami, \textbf{na} heta\d m mayha bhojana\d m;}\footnote{Ja\,22:86} (I will not eat the leaf, because this is not my food.) \par
\paragraph*{\pali{Na} in simile} Beside negating function, \pali{na} can be used in simile like \pali{viya}, for example:\par
- \pali{Ya\d m \textbf{na} ka\~ncanadepi\~ncha, andhena tamas\=a gata\d m;}\footnote{Ja\,21:7. In this example, \pali{na} is related to \pali{kata\d m}.} ([Sumu\-kha] who has golden wings, which action is done as if by a blind person doing in the dark.) \par

\subsection*{\fbox{\pali{No}}}\label{nip:no}
This can be used in negation but less often. This particle normally appears either at the beginning or the end of sentences, not in the middle.\footnote{This is understandable because 1st person pronoun also has \pali{no} as its plural form which never appears in the first position. If so, it will be very confusing with this \pali{no}.} Some examples are:\par
- \pali{subh\=asita\d mva bh\=aseyya, \textbf{no} ca dubbh\=asita\d m bha\d ne} (One should say good speech, should not say bad speech.) \par
\paragraph*{\pali{No} in questioning} Another use of \pali{no} is in questions. It is equal to \pali{nu}, for example:\par
- \pali{Abhij\=an\=asi \textbf{no} tva\d m, mah\=ar\=aja, ima\d m pa\~nha\d m a\~n\~ne sama\d nabr\-\=ahma\d ne pucchit\=a.}\footnote{D1\,164 (DN\,2)} (Do you remember, Your Majesty, [you have ever] asked this question to other ascetics and brahmans?) \par
\paragraph*{\pali{No} as \pali{avadh\=ara\d na}} This peculiar word means like simile, but it treats the object as the only one of its class. We can find this use in compounds (see page \pageref{par:samasa-avadh}). Here is an example:\par
- \pali{Na \textbf{no} sama\d m atthi tath\=agatena}\footnote{Khp\,6:3} (There is no jewel equal to/by the Buddha)\footnote{\pali{sama\d m ratana\d m natthev\=ati attho} (Sadd-Sut Ch.\,27). The Buddha is the only jewel, so to speak.} \par

\subsection*{\fbox{\pali{M\=a}}}\label{nip:maa}
This particle is mainly used for prohibition. In prose, it appears only in the first position. For more examples, see page \pageref{par:vclassma}.\par
- \pali{kha\d no vo \textbf{m\=a} upaccag\=a.}\footnote{Dhp\,22.315. This instance is in verse.} (Don't let the moment run away.)\par

\subsection*{\fbox{\pali{A}}}\label{nip:a}
We can see this quite often, but it is normally attached in front of words to negate their meaning. Duplication of a character can be seen. And when the first character of the words is a vowel, it becomes \pali{an}.\par
- \pali{May\=a ceta\d m, bhikkhave, \textbf{a\~n}\~n\=ata\d m abhavissa \textbf{a}di\d t\d tha\d m \textbf{a}vidita\d m \textbf{a}sacchikata\d m \textbf{a}phassita\d m pa\~n\~n\=aya}\footnote{M2\,178 (MN\,70)} (Monks, [suppose] I had not known, not seen, not understood, not realized, not experienced [this] with wisdom.) \par
\paragraph*{Other uses of \pali{a}} It is said that \pali{a} has ten meanings, including negation mentioned above. It can also denote state of lacking of something. Other unexpected meanings can be exemplified below:\par
- \pali{\textbf{a}manusso} (human-like) \par
- \pali{\textbf{a}r\=aj\=a} (blameworthy king) \par
- \pali{\textbf{an}udar\=a ka\~n\~n\=a} (a small-bellied girl [or in a good shape in modern sense]) \par
- \pali{\textbf{a}nantaka\d m}\footnote{This \pali{a} means nothing. The term is exactly equal to \pali{nantaka\d m} (= \pali{pilotika\d m}).} (a rag, old clothe) \par

\subsection*{\fbox{\pali{Ala\d m, Hala\d m}}}\label{nip:aladm}\label{nip:haladm}
This particle is used for prohibition in the sense of ``It is not suitable to do such and such things.'' In a way, it is like to say ``That's enough'' in English. These both only appear either in the first or the last position of sentences, not in between.\par
- \pali{\textbf{ala\d m} me buddhen\=a'ti vadati vi\~n\~n\=apeti.}\footnote{Buv1\,52} ([He] says, makes know, ``What [the use] with the Buddha for me! [= Enough! with the Buddha]'') \par
- \pali{\textbf{hala\d m} d\=ani pak\=asitu\d m.}\footnote{D2\,65 (DN\,14)} (It is not suitable to say now.) \par
\paragraph*{\pali{Ala\d m} in positive sense} In some context, this particle can mean like `suitable' or `enough' in positive meaning, for example:\par
- \pali{\textbf{alam}eta\d m sabba\d m}\footnote{Buv2\,237} (All these rules [are] enough [to make no violation]) \par

\subsection*{\fbox{\pali{Ya\~nce}}}\label{nip:yaynce}
This can be used to introduce an analogy,\footnote{Thanks to Antonio Costanzo for this observation.} for example:\par
- \pali{Seyyo amitto medh\=av\=i, \textbf{ya\~nce} b\=al\=anukampako}\footnote{Ja\,1:45} (A wise enemy is better, not a foolish supporter.) \par
- \pali{Da\d n\d dova kira me seyyo, \textbf{ya\~nce} putt\=a anassav\=a}\footnote{S1\,200 (SN\,7)} (Even my walking stick is better, not disobedient sons.) \par
- \pali{Tadeva mara\d na\d m seyyo, \textbf{ya\~nce} j\=ive tay\=a vin\=a}\footnote{Ja\,21:3} (Death is better, living without you is not.) \par

\refstepcounter{nipgrp}\label{nipgrp4}
\section*{\arabic{nipgrp}. Questioning}\label{nip:ques}

Some particles are helpful in marking questions. There is a handful of them that we can use in questioning, i.e.\ \pali{kacci, nu, nanu, katha\d m, ki\d msu, ki\d m} and \pali{kasm\=a}. Also a combination, \pali{kinnu}, can be in this list. Some other minor particles can also mark a question, such as \pali{atha} (see above). Aggava\d msa does not mention \pali{ud\=ahu} which should be grouped here, so I add this too. I also have a dedicated lesson on questioning, see Chapter \ref{chap:ques} for more detail.

\subsection*{\fbox{\pali{Kacci}}}\label{nip:kacci}
\paragraph*{\pali{Kacci} in questioning} For example:\par
- \pali{\textbf{kacci}, bhikkhu, khaman\=iya\d m; \textbf{kacci} y\=apan\=iya\d m.}\footnote{Mv\,10.465} \\(Monk, is it bearable? Is it sufficient for your life?) \par

\subsection*{\fbox{\pali{Nu}}}\label{nip:nu}
\paragraph*{\pali{Nu} in questioning} For example:\par
- \pali{ko \textbf{nu} kho hetu, ko paccayo bhagavato sitassa p\=atukamm\=aya?}\footnote{M2\,282 (MN\,81)} (What is the cause, what is the reason of the Buddha's making his smile visible?) \par
\paragraph*{\pali{Nu} as \pali{avadh\=ara\d na}} This sounds like using `eva,' for example:\par
- \pali{m\=ara di\d t\d thigata\d m \textbf{nu} te}\footnote{S1\,171 (SN\,5)} (Demon, [that is] only your [wrong] view.) \par
\paragraph*{\pali{Nu} as \pali{n\=ama}} For example:\par
- \pali{Ya\d m \textbf{nu} gijjho yojanasata\d m, ku\d nap\=ani avekkhati}\footnote{Ja\,2:27} \\(Which [called] vulture, [that creature can] see corpses from a distance of 100 yojanas.) \par

\subsection*{\fbox{\pali{Nanu}}}\label{nip:nanu}
\paragraph*{\pali{Nanu} in questioning} This has negative meaning (\pali{na + nu}), for example:\par
- \pali{\textbf{Nanu} tva\d m, phagguna, kulaputto saddh\=a ag\=arasm\=a anag\=ari\-ya\d m pabbajito}\footnote{M1\,223 (MN\,21)} (Phagguna, a son of the family, didn't you go forth from household life to homelessness because of faith?) \par

\subsection*{\fbox{\pali{Katha\d m}}}\label{nip:kathadm}
\paragraph*{\pali{Katha\d m} in questioning} This sounds like `how' in English, for example:\par
- \pali{\textbf{Katha\d m} su tarati ogha\d m, \textbf{katha\d m} su tarati a\d n\d nava\d m}\footnote{Snp\,1.185; S1\,246 (SN\,10)} (How does [one] cross the flood? How does [one] cross the ocean?) \par

\subsection*{\fbox{\pali{Ki\d msu, ki\d m}}}\label{nip:kidm}\label{nip:kidmsu}
\paragraph*{\pali{Ki\d msu, ki\d m} in questioning} If this is used as an indeclinable, it sounds like `what' in general. For more information on this see page \pageref{par:kim}.\par
- \pali{\textbf{Ki\d msu} chetv\=a sukha\d m seti}\footnote{S1\,71 (SN\,1)} (What is to be cut, [for one can] sleep happily?) \par
- \pali{\textbf{ki\d m} sevam\=ano labhat\=idha pa\~n\~na\d m}\footnote{Ja\,17:82} (In this world, what to get, to make use of, [for one can have] wisdom?) \par

\subsection*{\fbox{\pali{Kasm\=a}}}\label{nip:kasmaa}
\paragraph*{\pali{Kasm\=a} in questioning} As you may guess, this is used to ask about cause or reason, for example:\par
\begin{quote}
\pali{\textbf{Kasm\=a} bhava\d m vijanamara\~n\~namassito,}\\
\pali{Tapo idha kubbasi brahmapattiy\=a.}\footnote{S1\,204 (SN\,7)}\\[1.5mm]
``Why does the Venerable [Gotama] live in the deserted forest?''\\
``Do you practice to attain the excellent life?'' \par
\end{quote}

\subsection*{\fbox{\pali{Kinnu}}}\label{nip:kinnu}
\paragraph*{\pali{Kinnu} in questioning} This comes from \pali{ki\d m + nu}. The unit means `why' or it just marks a reflective question, for example:\par
- \pali{\textbf{kinnu} tva\d m, br\=ahma\d na, l\=ukho l\=ukhap\=avura\d no}\footnote{S1\,200 (SN\,7)} (Brahman, why do you look poor, using ragged cloth?) \par

\subsection*{\fbox{\pali{Ud\=ahu}}}\label{nip:udaahu}
\paragraph*{\pali{Ud\=ahu} in questioning} This term is normally translated as `or' but in interrogative sense. It is meant to ask whether one of the options is the case or not, like ``Is this good or bad?'' When it is at the beginning, it means like ``Or [might this be the case that]?'' Here are some examples:\par
- \pali{appa\d tiggahit\=ani nu kho \textbf{ud\=ahu} pa\d tiggahetabb\=ani}\footnote{Mv\,6.268} (Are these of ungiven matters or [they should be] given?) \par
- \pali{parinibbuto nu kho me upajjh\=ayo \textbf{ud\=ahu} no parinibbuto}\footnote{Snp\,2.345} (Was my preceptor dead or not?)\par
- \pali{Ki\d m nu te, va\.ng\=isa, im\=a g\=ath\=ayo pubbe parivitakkit\=a, \textbf{ud\=ahu} \d th\=anasova ta\d m pa\d tibhanti}\footnote{S1\,126 (SN\,8)} (Va\.ng\=isa, are these verses reflected by you previously, or [they just] come into your mind?)\par
- \pali{\textbf{Ud\=ahu}, eva\d m su te bhagavanto ara\~n\~navanapatth\=ani pant\=ani sen\=asan\=ani pa\d tisevanti}\footnote{D3\,76 (DN\,25)} (Or [you have heard] that those buddhas use lodging in secluded jungles?)\par

\refstepcounter{nipgrp}\label{nipgrp5}
\section*{\arabic{nipgrp}. Marking causes}

Particles in this group are \pali{yasm\=a, tasm\=a, tath\=a hi,} and \pali{tena}. This function, in its full expression, is used with \pali{ya-ta} structure. In grammar textbooks, \pali{iti} can also be used to give a reason.

\subsection*{\fbox{\pali{Yasm\=a--tasm\=a}}}\label{nip:yasmaa}\label{nip:tasmaa}
For example:\par
- \pali{\textbf{Yasm\=a} ca kho, bhikkhave, r\=upa\d m anatt\=a, \textbf{tasm\=a} r\=upa\d m \=ab\=adh\=aya sa\d mvattati}\footnote{S3\,59 (SN\,22)} (From which reason, monks, form is not-self; from that reason, form leads to illness.) \par

\subsection*{\fbox{\pali{Tath\=a hi}}}\label{nip:tathaahi}
For example:\par
- \pali{\textbf{Tath\=a hi} pana me, ayyaputt\=a, bhagav\=a nimantito sv\=atan\=aya bhatta\d m saddhi\d m bhikkhusa\.nghena}\footnote{D2\,161 (DN\,16)} (Form that reason, Venerables, the Buddha is invited by me to have a meal tomorrow together with monks.) \par

\subsection*{\fbox{\pali{Tena}}}\label{nip:tena}
For example:\par
- \pali{su\~n\~na\d m me ag\=ara\d m pavisitabba\d m ahosi, \textbf{tena} p\=avisi\d m}\footnote{M2\,229 (MN\,76)} (The empty house was worth entering, then I entered.) \par

\refstepcounter{nipgrp}\label{nipgrp6}
\section*{\arabic{nipgrp}. Expressing doubt}

To show some doubt, these are used: \pali{appeva, appeva n\=ama,} and \pali{nu kho}. In using \pali{appeva} or \pali{appeva n\=ama}, the doubt normally comes from whether something should be done or not. So, they are normally used with optative mood.

\subsection*{\fbox{\pali{Appeva}}}\label{nip:appeva}
For example:\par
- \pali{\textbf{appeva} ma\d m bhagav\=a a\d t\d thita\d m ovadeyya}\footnote{Snp\,5.1064} (Is it the case if the Buddha will teach me with care?) \par

\subsection*{\fbox{\pali{Appeva n\=ama}}}\label{nip:appevanaama}
For example:\par
- \pali{\textbf{Appeva n\=ama} ayam\=ayasm\=a anulomik\=ani sen\=asan\=ani pa\d tisevam\=ano}\footnote{A7\,56} (Is this will be good if this venerable having use proper lodging \ldots?) \par

\subsection*{\fbox{\pali{Nu kho}}}\label{nip:nukho}
For example:\par
- \pali{aha\d m \textbf{nu kho}smi? No \textbf{nu kho}smi? Ki\d m \textbf{nu kho}smi? Katha\d m \textbf{nu kho}smi?}\footnote{M1\,18 (MN\,2)} (I am, or not? What am I? How am I?) \par

\refstepcounter{nipgrp}\label{nipgrp7}
\section*{\arabic{nipgrp}. Emphasizing}

The function of emphasizing is near to mean nothing in particular. It just strengthens the meaning of terms or the sentence. In P\=ali, it is called \pali{eka\d msatthe} (in one meaning). That means other meaning is excluded, so the intended meaning is stressed. In English, we can use `really' or 'surely' or 'indeed' to perform a similar function. There are six particles mentioned exclusively for this use, namely \pali{addh\=a, a\~n\~nadatthu, taggha, j\=atu, k\=ama\d m, sasakka\d m,} and \pali{j\=atucche}. In addition, \pali{tu} is mentioned later. I also move \pali{assu}, \pali{n\=una}, and \pali{vata} from other group to the list. Outside this group, several other particles can also be used in this way.

\subsection*{\fbox{\pali{Addh\=a}}}\label{nip:addhaa}
For example:\par
- \pali{\textbf{Addh\=a}, \=avuso kacc\=ana, bhagav\=a j\=ana\d m j\=an\=ati passa\d m passati}\footnote{A10\,172} (Venerable Kacc\=ana, [it is true that] the Buddha [when] knows, [he says I] know, [when] sees, [he says I] see.) \par

\subsection*{\fbox{\pali{A\~n\~nadatthu}}}\label{nip:aynynadatthu}
\paragraph*{\pali{A\~n\~nadatthu} in emphasizing} For example:\par
- \pali{\textbf{a\~n\~nadatthu} m\=a\d navak\=ana\d myeva sutv\=a}\footnote{S4\,132 (SN\,35)} ([He] surely having listened to the young man's [words] \ldots) \par
\paragraph*{\pali{A\~n\~nadatthu} as `except'} For example:\par
- \pali{Atha kho \=ayasm\=a ra\d t\d thap\=alo sakapitu nivesane neva d\=ana\d m alattha na paccakkh\=ana\d m; \textbf{a\~n\~nadatthu} akkosameva alattha.}\footnote{M2\,299 (MN\,82)} (At that time the Venerable Ra\d t\d thap\=ala did not get alms in his own father's house, did not get response, except only contempt.)\footnote{Translating the last part as ``indeed he got only contempt'' is also probable. Thus the term is used for exphasizing.} \par

\subsection*{\fbox{\pali{Taggha}}}\label{nip:taggha}
For example:\par
- \pali{\textbf{Taggha}, bhagav\=a, bojjha\.ng\=a}\footnote{S5\,195 (SN\,46)} (The Blessed One sir, [these are] indeed factors of wisdom.) \par

\subsection*{\fbox{\pali{J\=atu}}}\label{nip:jaatu}
For example:\par
- \pali{Ida\~nhi \textbf{j\=atu} me di\d t\d tha\d m, nayida\d m itih\=itiha\d m}\footnote{S1\,184 (SN\,6)} (This [arhant\-ship] is seen indeed by me, this is not a hearsay.) \par

\subsection*{\fbox{\pali{K\=ama\d m}}}\label{nip:kaamadm}
For example:\par
- \pali{\textbf{K\=ama\d m} caj\=ama asuresu p\=a\d na\d m}\footnote{S1\,252 (SN\,11)} (I surely have to give up my life in these demons.) \par

\subsection*{\fbox{\pali{Sasakka\d m}}}\label{nip:sasakkadm}
For example:\par
- \pali{evar\=upa\d m te, r\=ahula, k\=ayena kamma\d m \textbf{sasakka\d m} na kara\d n\-\=iya\d m}\footnote{M2\,109 (MN\,61)} (R\=ahula, such an action is indeed should not be done by you.) \par

\subsection*{\fbox{\pali{J\=atucche}}}\label{nip:jaatucche}
For example:\par
\begin{quote}
\pali{Na mig\=ajina \textbf{j\=atucche} aha\d m ka\~nci kud\=acana\d m;}\\
\pali{Adhammena jine \~n\=ati\d m, na c\=api \~n\=atayo mama\d m.}\footnote{Ja\,22:264}\\[1.5mm]
``Mig\=ajina, sir, I indeed do not win unfairly [= take advantage of] my any relative, and they do not do that to me as well.''
\end{quote}

\subsection*{\fbox{\pali{Tu}}}\label{nip:tu}
\paragraph*{\pali{Tu} in emphasizing} For example:\par
- \pali{Seyyo amitto matiy\=a upeto, na \textbf{tv}eva mitto mativippah\=ino}\footnote{Ja\,1:44. In this instance, \pali{tveva = tu + eva}.} (An enemy having wisdom is better, a friend without wisdom is really not [good].) \par
\paragraph*{\pali{Tu} as a filler} For example:\par
- \pali{vedan\=ad\=isupekasmi\d m khandhasaddo \textbf{tu} ru\d lhiy\=a}\footnote{from the 6th verse of Saccasa\.nkhepa} (The term `\pali{khanda}' is raised to show one part of feeling, etc.) \par

\subsection*{\fbox{\pali{Assu}}}\label{nip:assu}
For example:\par
- \pali{n\textbf{\=assu}dha koci bhagavanta\d m upasa\.nkamati}\footnote{Buv1\,162} (Indeed, no one here approaches the Buddha.) \par

\subsection*{\fbox{\pali{N\=una}}}\label{nip:nuuna}
By the term, \pali{n\=una} means `surely' or `indeed.' It has a sense of exphasizing, but with nuances as described below.\par
\paragraph*{\pali{N\=una} in speculating} This is like making an assumption, for example:\par
- \pali{na hi \textbf{n\=una} so orako dhammavinayo, na s\=a orak\=a pabbajj\=a}\footnote{Mv\,1.30} (That teaching and discipline surely is not bad. That going forth is not bad.) \par
\paragraph*{\pali{N\=una} in reflecting} This means recollecting something in the past, for example:\par
- \pali{S\=a \textbf{n\=una}s\=a kapa\d nik\=a, andh\=a apari\d n\=ayik\=a}\footnote{Ja\,11:4} ([What a pity!,] that female elephant, blind, without a leader.) \par
\paragraph*{\pali{N\=una} in thinking} This is like reflecting, but the target is in the future. Normally it comes with \pali{ya\d m} and is used in optative mood, for example:\par
- \pali{Ya\d m\textbf{n\=un\=a}ha\d m anupakhajja j\=ivit\=a voropeyya}\footnote{S3\,85 (SN\,22)} (Which person I should take away the life.) \par

\subsection*{\fbox{\pali{Vata}}}\label{nip:vata}
\paragraph*{\pali{Vata} in emphasizing} For example:\par
- \pali{Acchera\d m \textbf{vata} lokasmi\d m, uppajjanti vicakkha\d n\=a}\footnote{Ja\,22:421. Aggava\d msa explain this use as `\pali{attheka\d mse}'' (in one meaning). This stresses the certainty because other meaning is prevented.} (Amazing indeed, wise men arise in the world.) \par
\paragraph*{\pali{Vata} in weariness (\pali{khede})} This can mean, I think, like `unfortunately' or `too bad!' or `poor man!' or `alas!,' for example:\par
- \pali{kiccha\d m \textbf{vat\=a}ya\d m loko \=apanno}\footnote{D2\,57 (DN\,14)} (Alas!, this worldling falls into difficulty.) \par
\paragraph*{\pali{Vata} in sympathy} For example:\par
\begin{quote}
\pali{Kapa\d no \textbf{vata}ya\d m bhikkhu, mu\d n\d do sa\.ngh\=a\d tip\=aruto;}\\
\pali{Am\=atiko apitiko, rukkham\=alasmi jh\=ayati.}\footnote{Ja\,19:8}\\[1.5mm]
``Poor man! this miserable monk, bald, wearing a robe,\\
no mother, no father, meditates under the tree.''
\end{quote}
\paragraph*{\pali{Vata} in thinking} For example:\par
- \pali{aho \textbf{vat\=aya\d m} nasseyya} (This [man] should perish [how can it be?].) \par
- \pali{aho \textbf{vata} me dhamma\d m su\d neyyu\d m}\footnote{S2\,146 (SN\,16)} ([They] should listen the teaching from me [how can it be?].) \par
\paragraph*{\pali{Vata} as a filler} For example:\par
- \pali{abbhuta\d m \textbf{vata}, bho}\footnote{S2\,202 (SN\,19)} (Sir, that's wonderful.) \par

\refstepcounter{nipgrp}\label{nipgrp8}
\section*{\arabic{nipgrp}. Illustrating}

There are three in this group: \pali{eva\d m, ittha\d m,} and \pali{iti}. In English, they can be `thus' or `in this way' or `as such.' In direct speech, \pali{iti} is used extensively in the canon. For more information on \pali{iti}, see page \pageref{par:iti}. Since \pali{eva\d m} can be used in a variety of ways, I group it as a general-purpose particle (see above). Also \pali{yath\=a} can be used in this sense, see in the general group too.

\subsection*{\fbox{\pali{Ittha\d m}}}\label{nip:itthadm}
For example:\par
- \pali{\textbf{ittham}pi te mano}\footnote{D1\,485 (DN\,11)} (Your mind also think in this way.) \par

\subsection*{\fbox{\pali{Iti}}}\label{nip:iti}
\paragraph*{\pali{Iti} in illustrating} For example:\par
- \pali{\textbf{iti}pi te citta\d m}\footnote{D1\,485 (DN\,11)} (Your mind also think in this way.) \par
\paragraph*{\pali{Iti} in marking a cause} For example:\par
- \pali{S\=asat\textbf{\=iti} satth\=a}\footnote{Kacc\,566} (Because one teaches, thus `teacher.') \par
\paragraph*{\pali{Iti} in finishing} This is mostly used in textbooks, for example:\par
- \pali{\textbf{Iti} padar\=upasiddhiya\d m n\=amaka\d n\d do dutiyo}\footnote{R\=upa ch.\,2} (The section of noun, chapter 2 in Padar\=upasiddhi, thus [ends].) \par

\refstepcounter{nipgrp}\label{nipgrp9}
\section*{\arabic{nipgrp}. Setting a boundary}

Particles in this group are \pali{y\=ava, t\=ava, y\=avat\=a, t\=avat\=a, kitt\=avat\=a,} and \pali{ett\=avat\=a}. Also \pali{k\=iva} can be added to the list. In \pali{ya-ta} structure (see Chapter \ref{chap:yata}), they normally come in pair, i.e.\ \pali{y\=ava} with \pali{t\=ava}, \pali{y\=avat\=a} with \pali{t\=avat\=a}. These pairs roughly mean `as far as' or `as much as.' However, the pairs are not necessarily well-matched. They sometimes come unpaired. Please see examples below for more understanding.

\subsection*{\fbox{\pali{Y\=ava(t\=a)--t\=ava(t\=a)}}}\label{nip:yaava}\label{nip:taava}
For example:\par
- \pali{\textbf{Y\=ava}ssa k\=ayo \d thassati \textbf{t\=ava} na\d m dakkhanti devamanuss\=a.}\footnote{A7\,56} (As far as the body of that [Buddha] will last, [by that stretch] humans and deities will see that [body].) \par
- \pali{\textbf{Y\=avat\=a}, bhikkhave, k\=asikosal\=a, \ldots, r\=aj\=a tattha pasenadi kosalo aggamakkh\=ayati.}\footnote{A10\,29} (Monks, as far as K\=as\=i and Kosala last, [in that period] King Pasenadi Kosala is said to be the top.) \par
- \pali{Na tena pa\d n\d dito hoti, \textbf{y\=avat\=a} bahu bh\=asati}\footnote{Dhp\,19.258} (When one speaks a lot, it is not with that [reason to make] one become a wise man.) \par
- \pali{Na \textbf{t\=avat\=a} dhammadharo, \textbf{y\=avat\=a} bahu bh\=asati}\footnote{Dhp\,19.259} (As much as one speaks a lot, one does not become a teaching holder.) \par
- \pali{\textbf{t\=avat\=a} tva\d m bhavissasi isi v\=a isitth\=aya v\=a pa\d tipanno}\footnote{D1\,285 (DN\,3)} (As much that you will become a seer or a practitioner for being a seer.) \par

\subsection*{\fbox{\pali{Kitt\=avat\=a}}}\label{nip:kittaavataa}
This is used for questioning in the sense of `how far' or `in what respect' or `to what extent,' for example:\par
- \pali{\textbf{kitt\=avat\=a} nu kho, bhante, up\=asako hoti.}\footnote{A8\,25} (In what respect, sir, does one become [= can be called] a lay devotee?) \par

\subsection*{\fbox{\pali{Ett\=avat\=a}}}\label{nip:ettaavataa}
This means `to this extent' or `by this much,' for example:\par
- \pali{\textbf{ett\=avat\=a} kho, mah\=an\=ama, up\=asako hoti}\footnote{A8\,25} (Mah\=an\=ama, one becomes a lay devotee by this much.) \par

\subsection*{\fbox{\pali{K\=iva}}}\label{nip:kiiva}
This means `how much' or `how long' or `how far,' for example:\par
- \pali{\textbf{K\=iva} d\=uro, bhante, ito p\=a\d taliputtanagara\d m}\footnote{Mil\,1.17} (How far is, sir, from here to P\=a\d taliputta?) \par
With \pali{y\=ava} and \pali{ca}, \pali{y\=avak\=iva\~nca} as a unit means `as long as,' for example:\par
- \pali{Y\=ava\textbf{k\=iva}\~nca me, bhikkhave, imesu cat\=usu ariyasaccesu}\footnote{Mv\,1.16} (Monks, as long as [I did not realize] these four noble truths) \par

\refstepcounter{nipgrp}\label{nipgrp10}
\section*{\arabic{nipgrp}. Responding}\label{nip:resp}

This group of particles is used for answering a question or the like. They are \pali{eva\d m, s\=ahu, lahu, op\=ayika\d m, patir\=upa\d m, \=ama,} and \pali{\=amo}. We have met \pali{eva\d m} in a section above. The rest of these are described below. Apart from these, \pali{s\=adhu} can also be used in this sense (see Praising group below). Another one that can be in this group is \pali{evameta\d m}.

\subsection*{\fbox{\pali{S\=ahu}}}\label{nip:saahu}
This means like `good,' for example:\par
- \pali{\textbf{S\=ah\=u}ti v\=a lah\=uti v\=a op\=ayikanti v\=a patir\=upanti v\=a \ldots vi\~n\~n\=apeti}\footnote{Mv\,1.65} ([One] makes know [by saying] `good' or `never mind' or `suitable' or `proper.') \par

\subsection*{\fbox{\pali{Lahu}}}\label{nip:lahu}
This sounds like `never mind' in English, for example:\par
- \pali{S\=ah\=uti v\=a \textbf{lah\=u}ti v\=a op\=ayikanti v\=a patir\=upanti v\=a \ldots vi\~n\~n\=apeti} \par

\subsection*{\fbox{\pali{Op\=ayika\d m}}}\label{nip:opaayikadm}
This means `suitable,' for example:\par
- \pali{S\=ah\=uti v\=a lah\=uti v\=a \textbf{op\=ayikan}ti v\=a patir\=upanti v\=a \ldots vi\~n\~n\=apeti} \par

\subsection*{\fbox{\pali{Patir\=upa\d m}}}\label{nip:patiruupadm}
This means `proper' or exactly as \pali{op\=ayika\d m}, for example:\par
- \pali{S\=ah\=uti v\=a lah\=uti v\=a op\=ayikanti v\=a \textbf{patir\=upan}ti v\=a \ldots vi\~n\~n\=apeti} \par

\subsection*{\fbox{\pali{\=Ama}}}\label{nip:aama}
This is the most used one for an agreeable response. It means simply `Yes,' for example:\par
- \pali{ap\=avuso, amh\=aka\d m satth\=ara\d m j\=an\=asi? \textbf{Am\=a}vuso, j\=an\=ami.}\footnote{Mv\,3.231} (Venerable, do you know our teacher? Yes, Venerable, I know.) \par

\subsection*{\fbox{\pali{\=Amo}}}\label{nip:aamo}
This is an alternative form of \pali{\=ama}. It is less frequently seen.\par
- \pali{\textbf{\=amo}'ti pa\d tij\=ananti}\footnote{D3\,37 (DN\,24)} ([They] acknowledge, `Yes.') \par

\subsection*{\fbox{\pali{Evameta\d m}}}\label{nip:evametadm}
This is used to express agreement, for example:\par
- \pali{\textbf{Evameta\d m}, mah\=ar\=aja, \textbf{evameta\d m}, mah\=ar\=aja! Sabbe satt\=a mara\d nadhamm\=a mara\d napariyos\=an\=a}\footnote{S1\,133 (SN\,3)} (That's right, Your Majesty, that's right. All beings have death by nature, have death as the end.) \par

\refstepcounter{nipgrp}\label{nipgrp11}
\section*{\arabic{nipgrp}. Comparing}\label{sec:nip-comparing}

P\=ali makes use of comparing quite a lot, including figures of speech like simile. Particles in this group are \pali{yath\=a, tath\=a, yatheva, tatheva, eva\d m, evameva, evameva\d m, evampi, yath\=api, seyyath\=api, seyyath\=api n\=ama, viya, iva, yathariva,} and \pali{tathariva}. As an example tells us, \pali{yadeva} and \pali{tadeva} can be added to this list. Some of these are used in pair as \pali{ya-ta} structure, but it is not always so. For \pali{eva\d m}, see the general-purpose group above. It is worth noting that \pali{yath\=a--tath\=a} can be composed with other particles to achieve the same effect, for example, \pali{yath\=an\=ama--tath\=an\=ama}, \pali{yath\=ahi--tath\=ahi}, and \pali{yath\=aca--tath\=aca}.

\subsection*{\fbox{\pali{Yath\=a--tath\=a}}}\label{nip:yathaa2}\label{nip:tathaa}
For examples:\par
- \pali{Nagara\d m \textbf{yath\=a} paccanta\d m, gutta\d m santarab\=ahira\d m; Eva\d m gopetha att\=ana\d m}\footnote{Dhp\,22.315. It is worth noting that \pali{eva\d m} can be used instead of \pali{tath\=a} to form \pali{ya-ta} structure.} (In which way a bordering town is protected inside and outside, protect yourself in that way.) \par

\subsection*{\fbox{\pali{Yatheva--tatheva, yadeva--tadeva}}}\label{nip:yatheva}\label{nip:tatheva}\label{nip:yadeva}\label{nip:tadeva}
For examples:\par
\begin{quote}
\pali{Yadeva ty\=aha\d m vacana\d m, akara\d m bhaddamatthu te;}\\
\pali{Tadeva me tva\d m vacana\d m, y\=acito kattumarahasi.}\footnote{Ja\,22:45. Aggava\d msa gives us \pali{yatheva} and \pali{tatheva} instead.}\\[1.5mm]
``In which way I do what you tell me.\\
(May luck be with you.)\\
In that way you do what I have asked you to do.''
\end{quote}

\subsection*{\fbox{\pali{Evameva}}}\label{nip:evameva}
This comes from \pali{eva\d m + eva} meaning ``just like that.'' We can find that it is also used together with \pali{seyyath\=api} (see below).\par
- \pali{\textbf{Evameva} tvampi pamu\~ncassu saddha\d m}\footnote{Snp\,5.1152} (May you make the faith arise just like that.) \par

\subsection*{\fbox{\pali{Evameva\d m}}}\label{nip:evamevadm}
This comes from \pali{eva\d m + eva\d m} meaning ``exactly like that,'' for examples:\par
- \pali{\textbf{evameva\d m} bhot\=a gotamena anekapariy\=ayena dhammo \\pak\=asito}\footnote{D1\,354 (DN\,5)} (Exactly like that, the teaching preached by the Buddha in various ways.) \par

\subsection*{\fbox{\pali{Evampi}}}\label{nip:evampi}
For examples:\par
- \pali{\textbf{Evampi} yo vedagu bh\=avitatto}\footnote{Snp\,2.324} ([One is] just like [a person] who attained the highest knowledge, well-developed.) \par

\subsection*{\fbox{\pali{Yath\=api}}}\label{nip:yathaapi}
For examples:\par
- \pali{\textbf{Yath\=api} sel\=a vipul\=a, nabha\d m \=ahacca pabbat\=a}\footnote{S1\,136 (SN\,3)} (Like a huge rock mountain reaching the sky) \par

\subsection*{\fbox{\pali{Seyyath\=api}}}\label{nip:seyyathaapi}
To form a simile, this is often used with \pali{evameva}, for examples:\par
\begin{quote}
\pali{\textbf{Seyyath\=api}, bhikkhave, mah\=arukkho \ldots So ta\d m rukkha\d m m\=ule chindeyya \ldots\ \textbf{Evameva} kho, bhikkhave, up\=ad\=aniyesu dhammesu \=ad\=inav\=anupassino viharato ta\d nh\=a nirujjhati.}\footnote{S2\,55 (SN\,12)}\\
``Just like a big tree, monks, \ldots\ That man might cut the tree in the root. \ldots\ In the same way, monks, [when] one constantly contemplates the disadvantage of causes of attachments, craving vanishes.''\\
\end{quote}

\subsection*{\fbox{\pali{Seyyath\=api n\=ama}}}\label{nip:seyyathaapinaama}
For examples:\par
- \pali{\textbf{Seyyath\=api} n\=ama mahat\=i na\.ngal\=is\=a}\footnote{S1\,138 (SN\,4)} ([It is] like a big beam of a plough.) \par

\subsection*{\fbox{\pali{Viya}}}\label{nip:viya}
For examples:\par
- \pali{hatthippabhinna\d m \textbf{viya} a\.nkusaggaho}\footnote{Dhp\,23.326} (Like [an elephant trainer] controls a broken elephant [being in rut].) \par

\subsection*{\fbox{\pali{Iva}}}\label{nip:iva}
For examples:\par
- \pali{t\=ula\d m bha\d t\d tha\d m\textbf{va} m\=aluto}\footnote{S1\,161 (SN\,4)} (Like wind blows cotton away.) \par

\subsection*{\fbox{\pali{Yathariva}}}\label{nip:yathariva}
For examples:\par
- \pali{\textbf{yathariva} bhot\=a gotamena}\footnote{D1\,263 (DN\,3)} (Like by Gotama) \par

\subsection*{\fbox{\pali{Tathariva}}}\label{nip:tathariva}
For examples:\par
- \pali{\textbf{tathariva} bhagav\=a}\footnote{This example is given by Aggava\d msa. The only instance of \pali{tathariva} found in the canon is in Dh\=atukath\=a 7.316, Abhidhammapi\d taka, but it seems not to have this meaning, just a filler.} (Like the Buddha) \par

\refstepcounter{nipgrp}\label{nipgrp12}
\section*{\arabic{nipgrp}. Conditional marking}

This group helps us to form a conditional or hypothetical statement. They are \pali{ce, sace,} and \pali{yadi}.

\subsection*{\fbox{\pali{Ce, sace}}}\label{nip:ce}\label{nip:sace}
For examples:\par
- \pali{ma\d m \textbf{ce} tva\d m nikha\d na\d m vane}\footnote{Ja\,22:5} (If you bury me in the forest.) \par
- \pali{Tato piva mah\=ar\=aja, \textbf{sace} tva\d m abhik.nkhasi}\footnote{Ja\,22:344} (Your Majesty, you may drink the water [that I took from that place] if you wish.) \par

\subsection*{\fbox{\pali{Yadi}}}\label{nip:yadi}
Apart from being used in conditionals, \pali{yadi} can mean other things as well, as shown in examples below.
\paragraph*{\pali{Yadi} in conditional marking} For examples:\par
- \pali{\textbf{Yadi}massa lokan\=athassa, virajjhiss\=ama s\=asana\d m}\footnote{Bv\,2:72} (If we fail in the teaching of the World's Protector.) \par
\paragraph*{\pali{Yadi} as `or'} For examples:\par
- \pali{ya\~n\~nadeva parisa\d m upasa\.nkamati, \textbf{yadi} khattiyaparisa\d m, \textbf{yadi} br\=ahma\d naparisa\d m, \textbf{yadi} gahapatiparisa\d m}\footnote{Mv\,6.285} ([One] approaches to any company: of the Warrior Caste, of the Priestly Caste, or of the Merchant Caste.) \par
\paragraph*{\pali{Yadi} as `in which time' (\pali{yad\=a})} For examples:\par
- \pali{\textbf{Yadi} passanti pavane, d\=arak\=a phaline dume}\footnote{Cp\,1:100} (In which time, the children see fruitful trees in the forest.) \par

\refstepcounter{nipgrp}\label{nipgrp13}
\section*{\arabic{nipgrp}. Praising, blaming}

Terms in this group is hard to translate into English. They are like interjections that express certain emotion rather than a sensible meaning. The uses of these are typically idiomatic in P\=ali. Particles in this group can also be used in other meaning, not just praising or blaming.

\subsection*{\fbox{\pali{Aho}}}\label{nip:aho}
\paragraph*{\pali{Aho} in praising} For example:\par
- \pali{\textbf{aho} buddho, \textbf{aho} dhammo, \textbf{aho} dhammassa sv\=akkh\=atat\=a!}\footnote{M2\,345 (MN\,85)} (Oh! the Buddha, the Dhamma, the well-preached teaching.) \par
- \pali{\textbf{aho} d\=ana\d m paramad\=ana\d m kassape suppati\d t\d thita\d m!}\footnote{Ud\,3.27} \\(Oh! the giving, the excellent giving, to Ven.\,Kassapa is well-established.) \par
\paragraph*{\pali{Aho} in blaming} For example:\par
- \pali{\textbf{aho} vata re amh\=aka\d m pa\d n\d ditaka, \textbf{aho} vata re amh\=aka\d m bahussutaka, \textbf{aho} vata re amh\=aka\d m tevijjaka}\footnote{D1\,291 (DN\,3)} (Shame! our wisemanship, our learnedness, our knowledge of the three vedas.) \par
\paragraph*{\pali{Aho} in wishing} For example:\par
- \pali{\textbf{aho} vata ma\d m rajje abhisi\~nceyyu\d m}\footnote{Mv\,1.57} (May people consecrate/anoint me as the king.) \par

\subsection*{\fbox{\pali{N\=ama}}}\label{nip:naama}
\paragraph*{\pali{N\=ama} in praising} For example:\par
- \pali{Yatra hi \textbf{n\=ama} tath\=agato eva\d mmahiddhiko eva\d mmah\=anubh\=avo}\footnote{D3\,162 (DN\,28)} (Oh! even the Buddha's [disciple] has such a great power.) \par
\paragraph*{\pali{N\=ama} in blaming} For example:\par
- \pali{atthi \textbf{n\=ama}, \=ananda, thera\d m bhikkhu\d m vihesiyam\=ana\d m ajjhupekkhissatha}\footnote{A5\,166} ([It is not good,] \=Ananda, when a senior monk is being harassed, you [all] just look indifferently.) \par

\subsection*{\fbox{\pali{S\=adhu}}}\label{nip:saadhu}
This particle sounds much like we exclaim `Good' in English. Apart from the use in praising, it can also be used in some other ways.
\paragraph*{\pali{S\=adhu} in praising} For example:\par
- \pali{\textbf{S\=adhu s\=adhu}, \=ananda, yath\=a ta\d m s\=ariputto samm\=a by\=akara\-m\=ano by\=akareyya}\footnote{S2\,24 (SN\,12)} (That is good, \=Ananda, in the way S\=ariputta answers that, he does it rightly.) \par
\paragraph*{\pali{S\=adhu} in requesting} For example:\par
- \pali{\textbf{s\=adhu} me, bhante, bhagav\=a sa\d mkhittena dhamma\d m desetu}\footnote{S5\,369 (SN\,47)} (Sir, may the Blessed One briefly teach me the Dhamma.) \par
\paragraph*{\pali{S\=adhu} in responding} For example:\par
- \pali{\textbf{S\=adh\=u}ti vatv\=ana pah\=utak\=amo, pakk\=ami yakkho vidhurena saddhi\d m}\footnote{Ja\,22:1461} (Having said ``That's good,'' the wealthy demon went away with Vidhura.) \par
\paragraph*{\pali{S\=adhu} in appreciating} For example:\par
- \pali{\textbf{s\=adhu} te kata\d m} (The action done by you is good.) \par

\subsection*{\fbox{\pali{Su\d t\d thu}}}\label{nip:sudtdthu}
\paragraph*{\pali{Su\d t\d thu} in responding} For example:\par
- \pali{s\=adhu \textbf{su\d t\d thu} bhante sa\d mvariss\=ami} (Right!, sir, I will restrain well.) \par
\paragraph*{\pali{Su\d t\d thu} in appreciating} For example:\par
- \pali{\textbf{su\d t\d thu} tay\=a kata\d m} (The action done by you is good.) \par

\subsection*{\fbox{\pali{Ki\~nc\=api, ki\~nci}}}\label{nip:kiyncaapi}
\paragraph*{\pali{Ki\~nc\=api} in praising} For example:\par
- \pali{\textbf{ki\~nc\=api} me, bhante, bhagav\=a saddh\=ayiko paccayiko}\footnote{A10\,89} (Even, sir, the Buddha [is] trustworthy and reliable to me.) \par
\paragraph*{\pali{Ki\~nci} in blaming} For example:\par
- \pali{A\~n\~nepi devo poseti, \textbf{ki\~nci} devo saka\d m paja\d m}\footnote{In Ja\,1:7, it is \pali{ki\~nca}.} (The king yet takes care of other people, [why he can't do as such with] his own offspring.) \par
\paragraph*{\pali{Ki\~nc\=api} as `although'} For example:\par
- \pali{aya\d m, bhante, \=ayasm\=a \=anando \textbf{ki\d nc\=api} sekkho, abhabbo chand\=a dos\=a moh\=a bhay\=a agati\d m gantu\d m}\footnote{Cv\,11.437} ([Mah\=akassapa], sir, this Ven.\,\=Ananda, althought he is [still] not enlightened, is unable to be biased from liking, disliking, delusion, and fear.) \par

\subsection*{\fbox{\pali{Dh\=iratthu}}}\label{nip:dhiiratthu}
\paragraph*{\pali{Dh\=iratthu} in blaming} For example:\par
- \pali{\textbf{Dhiratthu} ka\d n\d dina\d m salla\d m}\footnote{Ja\,1:13. In fact, no single instance of \pali{dh\=iratthu} is found in the canon.} ([It is blameworthy,] the sharpened arrow.) \par

\subsection*{\fbox{\pali{Dh\=i}}}\label{nip:dhii}
\paragraph*{\pali{Dh\=i} in blaming} For example:\par
- \pali{\textbf{Dh\=i} br\=ahma\d nassa hant\=ara\d m}\footnote{Dhp\,26.389} ([It is blameworthy,] one who kills a brahman.) \par

\subsection*{\fbox{\pali{Kismi\d m viya}}}\label{nip:kismidmviya}
\paragraph*{\pali{Kismi\d m viya} as ``it is a shame!''} This is an idiom, for example:\par
- \pali{\textbf{kismi\d m viya} rittahattha\d m gantu}\footnote{Buv2\,230} (It is a shame! to go empty-handed.) \par

\refstepcounter{nipgrp}\label{nipgrp14}
\section*{\arabic{nipgrp}. Urging}

This group is normally used to urge others to do something. It is difficult to render these into English. They are \pali{i\.ngha} and \pali{handa} here.

\subsection*{\fbox{\pali{I\.ngha}}}\label{nip:ingha}
For examples:\par
- \pali{\textbf{i\.ngha} me tva\d m, \=ananda, p\=an\=iya\d m \=ahara}\footnote{D2\,191 (DN\,16)} (Go!, \=Ananda, bring me water.) \par

\subsection*{\fbox{\pali{Handa}}}\label{nip:handa}
For examples:\par
- \pali{\textbf{handa}d\=ani, bhikkhave, \=amantay\=ami vo}\footnote{D2\,185 (DN\,16)} (Now, monks, I remind you \ldots) \par

\refstepcounter{nipgrp}\label{nipgrp15}
\section*{\arabic{nipgrp}. Repeating}

This group is used in the sense of `again.' They are \pali{puna, puno, puna\d m,} and \pali{punappuna\d m} here.

\subsection*{\fbox{\pali{Puna, puno, puna\d m}}}\label{nip:puna}\label{nip:puno}\label{nip:punadm}
For examples:\par
- \pali{\textbf{puna} vad\=ami} (I will say it again.) \par
- \pali{\textbf{Puno}pi dhamma\d m deseti}\footnote{Ap1\,54:60} ([One] teaches the Dhamma again.) \par
- \pali{Na \textbf{puno} amat\=ak\=ara\d m, passiss\=ami mukha\d m tava}\footnote{Ap2\,2:235} (I will not see the face of the Maker of Deathlessness again.) \par
- \pali{N\=aha\d m \textbf{puna\d m} na ca \textbf{puna\d m}, na c\=api apunappuna\d m; Hatthibondi\d m pavekkh\=ami}\footnote{Ja\,1:148} (Not again, not again, I will not see the [dead] elephant's body again.) \par

\subsection*{\fbox{\pali{Punappuna\d m}}}\label{nip:punuppunadm}
This means `frequently' or `again and again,' for examples:\par
- \pali{dukkh\=a j\=ati \textbf{punappuna\d m}}\footnote{Dhp\,11.153} (Being reborn again and again is suffering.) \par

\refstepcounter{nipgrp}\label{nipgrp16}
\section*{\arabic{nipgrp}. Disgust}

They are \pali{du\d t\d thu} and \pali{ku} mentioned in this group.

\subsection*{\fbox{\pali{Du\d t\d thu}}}\label{nip:dudtdthu}
For examples:\par
- \pali{\textbf{du\d t\d thu}lla\d m} (a disgusting thing) \par

\subsection*{\fbox{\pali{Ku}}}\label{nip:ku}
For examples:\par
- \pali{\textbf{ku}putto} (a terrible son) \par

\refstepcounter{nipgrp}\label{nipgrp17}
\section*{\arabic{nipgrp}. Fast movement}

All particles in this group mean `quickly,' normally used as an adverb. They are \pali{khippa\d m, lahu\d m, acira\d m, tuva\d ta\d m,} and \pali{su}.\footnote{In textbooks also \pali{ara\d m, \=asu\d m} and \pali{tu\d n\d na\d m} are mentioned, but I find no use in the canon, at least in this sense, so I drop them. Likewise, \pali{su} should be treated as such, but Aggava\d msa gives us a clear example, despite its peculiarity. So, I retain it.}

\subsection*{\fbox{\pali{Khippa\d m}}}\label{nip:khippadm}
For examples:\par
\begin{quote}
\pali{Etamatthavasa\d m \~natv\=a, pa\d n\d dito s\=ilasa\d mvuto;}\\
\pali{Nibb\=anagamana\d m magga\d m, \textbf{khippam}eva visodhaye.}\footnote{Dhp\,20.289}\\[1.5mm]
``Having known this truth, a wise person who morally restrains oneself;''\\
``Quickly purify oneself on the path to nirvana.''\\
\end{quote}

\subsection*{\fbox{\pali{Lahu\d m}}}\label{nip:lahudm}
For examples:\par
- \pali{tehi, bhikkhave, \=av\=asikehi bhikkh\=uhi \textbf{lahu\d m lahu\d m} sannipatitv\=a pav\=aretabba\d m}\footnote{Mv\,4.240} (Monks, having come together, the Invitation [Pav\=ara\d n\=a] should be done quickly by bhikkhus living in that [quarter].) \par

\subsection*{\fbox{\pali{Acira\d m}}}\label{nip:aciradm}
As the opposite of \pali{cira\d m}, this means `in a short time' or `quickly,' for example:\par
- \pali{\textbf{Acira\d m} vataya\d m k\=ayo, pathavi\d m adhisessati}\footnote{Dhp\,3.41} (In a short time, this body will lie on the ground.) \par

\subsection*{\fbox{\pali{Tuva\d ta\d m}}}\label{nip:tuvadtadm}
For examples:\par
- \pali{\textbf{tuva\d ta\d m} kho, ayyaputta, \=agaccheyy\=asi}\footnote{Ud\,3.22} (Master's son, please come back quickly.) \par

\subsection*{\fbox{\pali{Su}}}\label{nip:su}
For example:\par
- \pali{lahu\d m lahu\d m kucchita\d m gacchat\=iti \textbf{su}ddo}\footnote{This example is given by Aggava\d msa. In Sadd-Dh\=a\,329, he gives an explanation as ``\pali{Tath\=a hi su iti s\=ighatthe nip\=ato}.''} ([One] goes contemptibly, quickly, thus \pali{sudda} [a member of the S\=udra caste].) \par

\refstepcounter{nipgrp}\label{nipgrp18}
\section*{\arabic{nipgrp}. Miscellaneous particles}

To be more orderly, I group various minor particles into this, if they have a particular meaning or use. Some of them can be used in a variety of contexts. For those with little meaning or no meaning at all, I group as fillers in the last section.

\subsection*{\fbox{\pali{A\~n\~natra}}}\label{nip:aynynatra}
\paragraph*{\pali{A\~n\~natra} as `without'} This particle means more or less like \pali{vin\=a}, for example:\par
- \pali{Id\=ani ya\d m ta\d m \textbf{a\~n\~natra} buddhupp\=ad\=a appavattapubba\d m sabbatitthiy\=ana\d m avisayabh\=uta\d m tesu tesu suttantesu}\footnote{Vism\,8.178} (Now, without the arising of the Buddha, [the teaching of mindfulness with the body] which is unknown to all other schools would not exist, that [teaching] is in various discourses.) \par
\paragraph*{\pali{A\~n\~natra} as `otherwise'} This can also mean like `unless' or 'except,' for example:\par
- \pali{Yo pana bhikkhu m\=atug\=amassa uttarichappa\~ncav\=ac\=ahi \linebreak dhamma\d m deseyya, \textbf{a\~n\~natra} vi\~n\~nun\=a purisaviggahena, p\=acittiya\d m}\footnote{Buv2\,63} (A monk who preaches the Dhamma more than 5--6 words to a woman commits a P\=acitt\=i offense, unless [he is accompanied] with a knowing man.) \par

\subsection*{\fbox{\pali{\=Isaka\d m}}}\label{nip:iisakadm}
\paragraph*{\pali{\=Isaka\d m} as `little' or `small'} For example:\par
- \pali{Seyyath\=api, \=ananda , \textbf{\=isaka\d m}po\d ne padumapal\=ase udakaphusit\=ani pavattanti, na sa\d n\d thanti.}\footnote{M3\,456 (MN\,152)} (\=Ananda, it is like drops of water on a lotus's leaf, slightly slanted, do not stay [on it].) \par

\subsection*{\fbox{\pali{Eva}}}\label{nip:eva}
This particle has a limited use. It means `only' in the sense that other meaning is prevented. It can be used with adjectives, for example, ``\pali{akko tamonudo eva}'' (The sun dispels only darkness); with nouns, for example, ``\pali{buddho eva tamonudo}'' (Only the Buddha dispels darkness); with verbs, for example, ``\pali{n\=ila\d m sarojamattheva}'' (The blue lotus only exists).
\paragraph*{\pali{Eva} as `only'} There is an example from the canon:\par
- \pali{Pubb\textbf{eva} me, bhikkhave, sambodh\=a anabhisambuddhassa bodhisattass\textbf{eva} sato etadahosi}\footnote{S4\,272 (SN\,36)} (Only in the past, monks, this [thought] happened to me when I just was a Bodhisatta not yet enlightened.) \par

\subsection*{\fbox{\pali{Atho}}}\label{nip:atho}
\paragraph*{\pali{Atho} in corresponding situations} For example:\par
- \pali{Sv\=agata\d m te mah\=ar\=aja, \textbf{atho} te adur\=agata\d m;}\footnote{Ja\,20:134} (Your Majesty, good coming are done by you, so your safe coming) \par
\paragraph*{\pali{Atho} as a filler} For example:\par
- \pali{\textbf{atho} ma\d m anukampasi}\footnote{Ja\,6:120} ([You also] sympathize with me.) \par

\subsection*{\fbox{\pali{Kate}}}\label{nip:kate}
\paragraph*{\pali{Kate} as `depending on' (\pali{pa\d ticcatthe})} This may sounds like `because of,' for example:\par
- \pali{Na mano v\=a sar\=ira\d m v\=a, ma\d m-\textbf{kate} sakka kassaci}\footnote{Ja\,10:23} (Sakka [the king of the gods], sir, may I ask you that no one [will be harmed], either in the mind or the body because of me.) \par

\subsection*{\fbox{\pali{Katha\~nci}}}\label{nip:kathaynci}
\paragraph*{\pali{Katha\~nci} as `difficultly'} This can be used in the sense of `hardly,' for example:\par
- \pali{Catt\=aro vinip\=at\=a, duve ca gatiyo \textbf{katha\~nci} labbhanti}\footnote{Thig\,16.458. This is the only instance found in the canon.} ([Beings] get into 4 hells [easily], but hardly into 2 existences [heaven \& the world].) \par

\subsection*{\fbox{\pali{Kalla\d m}}}\label{nip:kalladm}
\paragraph*{\pali{Kalla\d m} as `suitable'} This is normally used with verbs in \pali{tu\d m} form, for example:\par
- \pali{Ya\d m pan\=anicca\d m dukkha\d m vipari\d n\=amadhamma\d m, \textbf{kalla\d m} nu ta\d m samanupassitu\d m}\footnote{Mv\,1.21} (Is it suitable to see a thing which is by nature impermanent, unbearable, and changing as one's own?) \par

\subsection*{\fbox{\pali{Kaha\d m}}}\label{nip:kahadm}
By the term, it means `where?' In the example below, it is used like an interjection of lament.\par
- \pali{\textbf{kaha\d m}, ekaputtaka, \textbf{kaha\d m}, ekaputtaka}\footnote{M2\,353 (MN\,87)} (Alas, the only child!, alas, the only child!) \par

\subsection*{\fbox{\pali{Kira}}}\label{nip:kira}
\paragraph*{\pali{Kira} as ``as I have heard''} This means like \pali{khalu} in one sense, for example:\par
- \pali{Assosi kho citto gahapati sambahul\=ana\d m \textbf{kira} ther\=ana\d m bhikkh\=una\d m pacch\=abhatta\d m pi\d n\d dap\=atapa\d tikkant\=ana\d m ma\d n\d dalam\=a\d le sannisinn\=ana\d m sannipatit\=ana\d m ayamantar\=akath\=a udap\=adi}\footnote{S1\,343 (SN\,7)} \\(Merchant Citta heard that this discussion, of several senior monks sitting together on a platform after alms-round and meal, happened.) \par

\subsection*{\fbox{\pali{Kkhattu\d m}}}\label{nip:kkhattudm}
\paragraph*{\pali{Kkhattu\d m} as `time'} This may be better to be counted as a \pali{paccaya}, but its products end up as indeclinables, for example:\par
- \pali{eka\textbf{kkhattu\d m}} (one time) \par
- \pali{dvi\textbf{kkhattu\d m}} (two times) \par
- \pali{ti\textbf{kkhattu\d m}} (three times) \par

\subsection*{\fbox{\pali{Khalu}}}\label{nip:khalu}
\paragraph*{\pali{Khalu} as ``as I have heard''} This can also means like ``as it is said.'' This use is shared with \pali{kira}, for example:\par
- \pali{sama\d no \textbf{khalu} bho gotamo}\footnote{D1\,301 (DN\,4)} (As I have heard, sir, ascetic Gotama \ldots) \par
\paragraph*{\pali{Khalu} in negation} Occasionally this can be used in negation like \pali{na}, for example:\par
- \pali{\textbf{khalu}pacch\=abhattiko}\footnote{Vism\,2.23. This is equal to \pali{na pacch\=abhattiko}.} (One who does not eat after meal) \par
\paragraph*{\pali{Khalu} in emphasis} This rougly means like `surely' or `really' or `indeed,' for example:\par
- \pali{s\=adhu \textbf{khalu} payaso p\=ana\d m ya\~n\~nadattena}\footnote{Sadd-Pad Ch.\,5, also partly in Niru\,115.} (Drinking [of] milk done by Ya\~n\~nadatta is really good.) \par
\paragraph*{\pali{Khalu} as a filler} For example:\par
- \pali{sama\d no \textbf{khalu} bho gotamo sakyaputto sakyakul\=a pabbajito}\footnote{D1\,301 (DN\,4)} (Ascetic Gotama, a son of Sakya, went forth from Sakya clan.) \par

\subsection*{\fbox{\pali{Kho}}}\label{nip:kho}
\paragraph*{\pali{Kho} as \pali{avadh\=ara\d na}} This explanation given by Aggava\d msa is hard to understand, because \pali{avadh\=ara\d na} is normally used in simile (see above in \pali{no}). In the example below, it is explained that ``\pali{assosi kho}'' is equal to ``\pali{assosi eva}.'' That can mean ``only heard.'' So, it is better to see this as a particle used for affirmative emphasis which is like `indeed' or `really' or `surely.'\par
- \pali{Assosi \textbf{kho} vera\~njo br\=ahma\d no}\footnote{Buv1\,1} (Brahman Vera\~nja heard) \par
\paragraph*{\pali{Kho} as a filler} For example:\par
- \pali{Atha \textbf{kho} bhagav\=a bhikkh\=u \=amantesi.}\footnote{Cv\,5.265} (Then the Buddha called monks.) \par

\subsection*{\fbox{\pali{Ci}}}\label{nip:ci}
\paragraph*{\pali{Ci} as indefinite interrogative particle} This particle normally comes together with a form of \pali{ka} or \pali{ki\d m}. It add indefinite sense, i.e.\ `any' or `some,' to the word, for example:\par
- \pali{Sayanighara\d m n\=ama yattha kattha\textbf{ci} ra\~n\~no sayana\d m pa\~n\~natta\d m hoti}\footnote{Buv2\,499} (A sleeping place of the king which is prepared in anywhere [is] called `the sleeping room.') \par
- \pali{Ki\d m pana, v\=ase\d t\d tha, atthi ko\textbf{ci} tevijj\=ana\d m br\=ahma\d n\=ana\d m ekabr\-\=ahma\d nopi, yena brahm\=a sakkhidi\d t\d tho}\footnote{D1\,525 (DN\,13)} (V\=ase\d t\d tha, among brahmans who know the three Vedas, is there even anyone who saw the Brahma face to face?) \par

\subsection*{\fbox{\pali{Cira\d m, cirassa\d m}}}\label{nip:ciradm}\label{nip:cirassadm}
\paragraph*{\pali{Cira\d m, cirassa\d m} as `for a long time'} For example:\par
- \pali{\textbf{cira\d m} tva\d m anutappissati} (You will regret for a long time.) \par
- \pali{\textbf{cira\d m} d\=ighamaddh\=ana\d m ti\d t\d thanti}\footnote{D3\,119 (DN\,27)} ([They] last for a long time.) \par
- \pali{\textbf{Cirassa\d m} vata pass\=ami, br\=ahma\d na\d m parinibbuta\d m}\footnote{S1\,99 (SN\,2)} (It is a long time, at last I see the Noble One fully liberated.) \par

\subsection*{\fbox{\pali{Tu\d nh\=i}}}\label{nip:tudnhii}
\paragraph*{\pali{Tu\d nh\=i} in silence} This means nothing is not said or done, for example:\par
- \pali{\textbf{Tu\d nh\=i}bh\=uto upekkheyya}\footnote{Ja\,22:1491} (Being in a silent state, [he] should be indifferent.) \par

\subsection*{\fbox{\pali{Tuna, tv\=ana, tv\=a}}}\label{nip:tuna}\label{nip:tvaana}\label{nip:tvaa}
In primary derivation, these three are called \pali{paccaya}. Their products, a kind of verbal \pali{kita}, are counted as indeclinable for they stay intact when used (like \pali{tu\d m} and \pali{tave} mentioned in dative particles above). For more information on these verb forms, see Chapter \ref{chap:inf}. In practice, \pali{tuna} (sometimes \pali{t\=una}), \pali{tv\=ana}, and \pali{tv\=a} can be used interchangeably, but \pali{tv\=a} is commonly seen in the texts. Sometimes alternative forms, e.g.\ \pali{-ya} form, are more fashionable for some roots. I list some of them here for you can recognize them more easily.\par
- \pali{passituna} (having seen) \par
- \pali{passitv\=a(na)} (having seen) \par
- \pali{labhitv\=a(na)} (having got) \par
- \pali{laddh\=a(na)} (having got) \par
- \pali{vijjhitv\=a(na)} (having pierced) \par
- \pali{viddh\=a(na)} (having pierced) \par
- \pali{bujjhitv\=a(na)} (having known) \par
- \pali{buddh\=a(na)} (having known) \par
- \pali{disv\=a(na)} (having seen) \par
- \pali{di\d t\d th\=a(na)} (having seen) \par
- \pali{datv\=a} (having given) \par
- \pali{up\=adh\=aya} (having grasped) \par
- \pali{vi\~n\~n\=aya} (having known) \par
- \pali{viceyya} (having chosen) \par
- \pali{vineyya} (having led) \par
- \pali{nihacca} (having destroyed) \par
- \pali{samecca} (having calmed) \par
- \pali{\=arabbha} (having begun) \par
- \pali{\=agamma} (having come) \par
- \pali{\=agaccha} (having come) \par
- \pali{katv\=a} (having done) \par
- \pali{karitv\=a} (having done) \par
- \pali{kacca} (having done) \par

\subsection*{\fbox{\pali{Dhuva\d m}}}\label{nip:dhuvadm}
\paragraph*{\pali{Dhuva\d m} as `constantly'} For example:\par
- \pali{nicco \textbf{dhuvo} sassato}\footnote{E.g.\ D1\,44 (DN\,1). This is not a good example, because \pali{dhuva} here is used as an adjective not an indeclinable.} (permanent, constant, eternal) \par
\paragraph*{\pali{Dhuva\d m} as `certainly'} For example:\par
- \pali{\textbf{dhuva\d m} buddho bhav\=amaha\d m}\footnote{Bv\,2:109} (I certainly will become a buddha.) \par

\subsection*{\fbox{\pali{N\=an\=a}}}\label{nip:naanaa}
\paragraph*{\pali{N\=an\=a} as `various'} For example:\par
- \pali{\textbf{n\=an\=a}phaladhar\=a dum\=a}\footnote{Ja\,22:1978} ([There are] trees of various fruits.) \par

\subsection*{\fbox{\pali{N\=ana\d m}}}\label{nip:naanadm}
\paragraph*{\pali{N\=ana\d m} as `different'} For example:\par
- \pali{bya\~njanameva \textbf{n\=ana\d m}}\footnote{Pvr\,354; M1\,459 (MN\,43)} (Only the alphabet [is] different.) \par

\subsection*{\fbox{\pali{P\=atu}}}\label{nip:paatu}
\paragraph*{\pali{Puthu} as `visible'} This particle is normally use with other terms. It works like an \pali{upasagga} (prefix), but it is not one of them. Here are some examples:\par
- \pali{Mahes\=i v\=a bhikkhu\d m disv\=a sita\d m \textbf{p\=atu}karoti. Bhikkhu v\=a mahesi\d m disv\=a sita\d m \textbf{p\=atukaroti}.}\footnote{Buv2\,497} (Having seen the monk, the queen makes a smile visible. Likewise, having seen the queen, the monk makes a smile visible.) \par
- \pali{obh\=aso \textbf{p\=atu}bhavati, brahm\=a \textbf{p\=atu}bhavissati}\footnote{D1\,493 (DN\,11)} ([When] light appears, the supreme god appears.) \par

\subsection*{\fbox{\pali{Puthu}}}\label{nip:puthu}
\paragraph*{\pali{Puthu} as `individually'} This is the same \pali{visu\d m}, for example:\par
- \pali{kammassak\=ase \textbf{puthu} sabbasatt\=a}\footnote{Ja\,22:1287} (All beings individually have actions as their property.) \par

\subsection*{\fbox{\pali{Mana\d m}}}\label{nip:manadm}
\paragraph*{\pali{Mana\d m} as `almost'} For example:\par
- \pali{nadi\d m taranto \textbf{mana\d m} v\=u\d lho ahosi}\footnote{Mv\,2.143} (Crossing the river, [Mah\=akassapa] was almost carried away [by the water].) \par

\subsection*{\fbox{\pali{Visu\d m}}}\label{nip:visudm}
\paragraph*{\pali{Visu\d m} as `individually'} Also \pali{puthu} has this meaning, for example:\par
- \pali{sutt\=a honti \textbf{visu\d m} a\d t\d tha}\footnote{A7\,96-622} (There are eight discourses in each [group].) \par

\subsection*{\fbox{\pali{Sacchi}}}\label{nip:sacchi}
\paragraph*{\pali{Sacchi} in experiencing} This means realizing or having a clear and direct experience. It is normally used with \pali{kara} (to do) as an idiom, for example:\par
- \pali{arahattaphala\d m \textbf{sacchi} ak\=asi} ([He] realized the fruit of arhant\-ship.) \par

\subsection*{\fbox{\pali{Sa\d nika\d m}}}\label{nip:sadnikadm}
\paragraph*{\pali{Sa\d nika\d m} as `slowly'} For example:\par
- \pali{atha na\d m kumbhi\d m oropetv\=a ubbhinditv\=a mukha\d m vivaritv\=a \textbf{sa\d nika\d m} nillokema}\footnote{D2\,421 (DN\,23)} (Then, taking down the pot, slowly opening its cover, we take a look.) \par

\subsection*{\fbox{\pali{Seyyathida\d m}}}\label{nip:seyyathidadm}
This particle is normally used before a list. It is more or less equal to `as follows' or `that is to say.' Sometimes we see it as \pali{seyyath\=ida\d m}. Aggava\d msa tells us it is equal to ``\pali{so katamo}'' or ``\pali{te katame}'' or ``\pali{s\=a katam\=a}'' or ``\pali{t\=a katam\=a}'' or ``\pali{ta\d m katama\d m}'' or ``\pali{t\=ani katam\=ani}.'' That is to say, it can be used without worrying about number and gender.\par
- \pali{\textbf{Seyyathida\d m} r\=upup\=ad\=anakkhandho}\footnote{D2\,399 (DN\,22)} ([They are] the material form as the object of attachment, etc.) \par

\subsection*{\fbox{\pali{Sotthi, suvatthi}}}\label{nip:sotthi}\label{nip:suvatthi}
\paragraph*{\pali{Sotthi, suvatthi} in blessing} For example:\par
- \pali{\textbf{sotthi} hotu sabbasatt\=ana\d m}\footnote{Aggava\d msa has a discussion about whether this term should be counted as an indeclinable or not, because it can be nom.\ used in this instance. Other forms can be found also, for example, ``\pali{[Na] sotthi\d m pass\=ami p\=a\d nina\d m}'' [S1\,98 (SN\,2)] (I do not see well-being in living beings), and ``\pali{sotthin\=amhi samu\d t\d thito}'' [Ja\,22:401] (I was lifted up safely). He conclude that for just these forms are found, the term should be counted as an indeclinable. This condition is applied to \pali{suvatthi} as well.} (May all beings be blessed.) \par
- \pali{Etena saccena \textbf{suvatthi} hotu}\footnote{Snp\,2.226} (With this truth, may well-being occur.) \par

\subsection*{\fbox{\pali{Have, ve}}}\label{nip:have}\label{nip:ve}
\paragraph*{\pali{Have, ve} in emphasis (\pali{eka\d msatthe})} This use is for strengthening the meaning, for example:\par
- \pali{Yad\=a \textbf{have} p\=atubhavanti dhamm\=a}\footnote{Ud\,1.1} (When the natural qualities appear.) \par
- \pali{Na \textbf{ve} anatthakusalena, atthacariy\=a sukh\=avah\=a}\footnote{Ja\,1:46} (Doing beneficial thing with unskillful way indeed does not bring happiness.) \par
- \pali{na \textbf{v}\=aya\d m bhaddik\=a sur\=ati}\footnote{In Ja\,1:53, it is ``\pali{na c\=aya\d m \ldots}.''} (This liqueur is really tasteless.) \par
\paragraph*{\pali{Have, ve} as a filler} For example:\par
- \pali{\textbf{have} te bhonto sama\d mabr\=ahma\d n\=a}\footnote{M1\,35 (MN\,4)} (Those ascetics and brahmans) \par
- \pali{Sa \textbf{ve} etena y\=anena, nibb\=anasseva santike}\footnote{S1\,46 (SN\,1)} (That [person goes] near nirvana by this vehicle [the noble path].) \par

\subsection*{\fbox{\pali{H\=a}}}\label{nip:haa}
\paragraph*{\pali{H\=a} in weariness} For example:\par
- \pali{\textbf{H\=a} yog\=a vippayogant\=a}\footnote{Ap2\,2:252} (Oh!, meetings [and] separations at the end.) \par

\refstepcounter{nipgrp}\label{nipgrp19}
\section*{\arabic{nipgrp}. Fillers}

There are a good number of particles that means nothing in particular. We can call these fillers (\pali{padap\=ura\d na}). They makes the sentence sound better or smoother. Here is the list given: \pali{atha, khalu, vata, vatha, atho, assu, yagghe, hi, carahi, na\d m, ca, v\=a, vo, pana, have, k\=iva, ha, tato, yath\=a, suda\d m, kho, ve, kaha\d m, ena\d m, seyyathida\d m, \=a,} and \pali{ta\d m}. Some of these which can be put certain meaning to them are grouped elsewhere. The rest of them are put here. Some are really have no meaning whatsoever. Maybe once they were used as what we call \pali{discourse markers} today, but the intented function has been lost. Some are problematic, in my view, because they really have meaning one way or another, not just a space filler.

\subsection*{\fbox{\pali{\=A}}}\label{nip:aa}
For example:\par
- \pali{Yad\textbf{\=a}na\d m ma\~n\~nati b\=alo, bhay\=a my\=aya\d m titikkhati;}\footnote{This instance is tricky. Aggava\d msa explaines that \pali{yad\=ana\d m} comes from \pali{ya\d m + \=a + na\d m}. Thus \pali{\=a} is an particle. But from S1\,250 (SN\,11), it is in fact ``\pali{yad\=a na\d m}.'' There is no \pali{\=a} here. It makes a perfect sense with \pali{yad\=a} (when). The moral of this is, we should be careful with a forgotten or misplacing space in P\=ali. We can also see the same trick, if not a mistake, done by Aggava\d msa in Chapter \ref{chap:num}.} \\(When a fool thinks this, ``this [man] puts up with me because of fear.'') \par

\subsection*{\fbox{\pali{Ena\d m}}}\label{nip:enadm}
For example:\par
- \pali{Yatv\=adhikara\d nam\textbf{ena\d m} cakkhundriya\d m asa\d mvuta\d m viharanta\d m}\footnote{D3\,310 (DN\,33)} (Because of which reason that the faculty of sight kept unrestrained \ldots) \par

\subsection*{\fbox{\pali{Carahi}}}\label{nip:carahi}
For example:\par
- \pali{katha\d m \textbf{carahi} mah\=apa\~n\~no}\footnote{I found ``\pali{Katha\d m carahi sabba\~n\~n\=u}'' in Ap2\,2:170.} (How to be a great wise [person]?) \par

\subsection*{\fbox{\pali{Tato}}}\label{nip:tato}
This particle normally means `from that' (\pali{ta + to}). In some places, this meaning is ignored, for example.\par
- \pali{\textbf{Tato} ca maghav\=a sakko, atthadass\=i purindado}\footnote{Ja\,17:62} (Magha, the king of the gods, [is] a benefit-seer, a giver in the past.) \par

\subsection*{\fbox{\pali{Ta\d m}}}\label{nip:tadm}
For example:\par
- \pali{\textbf{Ta\d m} kissa hetu?}\footnote{Buv1\,34. This instance is also questionable, for \pali{ta\d m} can mean `that.' So, it propably means ``Why that?'' or ``If that is the case, why?''} (Of what reason? [Why?]) \par

\subsection*{\fbox{\pali{Na\d m}}}\label{nip:nadm}
For example:\par
- \pali{na \textbf{na\d m} suj\=ato sama\d no gotamo}\footnote{D3\,117 (DN\,27). Using \pali{na na\d m} here looks unusual. See the note in the scripture.} (Ascetic Gotama had a good birth.) \par

\subsection*{\fbox{\pali{Yagghe}}}\label{nip:yagghe}
This particle is normally used to address a person with superior status. It means somehow like ``look here, don't you know, surely, you ought to know; now then'' (see PTSD).\par
- \pali{\textbf{yagghe}, mah\=ar\=aja, j\=aneyy\=asi}\footnote{M2\,306 (MN\,82)} (Your Majesty, you should know.) \par

\subsection*{\fbox{\pali{Vatha}}}\label{nip:vatha}
For example:\par
- \pali{ta\d m \textbf{vata} jayaseno r\=ajakum\=aro}\footnote{M3\,214 (MN\,125). In the canon, it is \pali{vata}, not \pali{vatha} as Aggava\d msa gives us. In fact, there is no place of \pali{vatha} used in the whole collection.} (Prince Jayasena will know that thing.) \par

\subsection*{\fbox{\pali{Vo}}}\label{nip:vo}
For example:\par
- \pali{ete \textbf{vo} sukhasammat\=a}\footnote{Snp\,3.765} (These [objects of five senses] are agreed upon as happpiness.) \par

\subsection*{\fbox{\pali{Suda\d m}}}\label{nip:sudadm}
This particle used in the canon in most cases has no meaning whatsoever. Some may say that it can be used as `as I have heard' like \pali{kira} and \pali{khalu}, this can be the case if the context allows like the example below.\par
- \pali{Tatra \textbf{suda\d m} bhagav\=a r\=ajagahe viharanto gijjhak\=u\d te pabbate}\footnote{D2\,142 (DN\,16)} (The Blessed One, living there, in the Vulture's Peak, R\=ajagaha) \par

\subsection*{\fbox{\pali{Ha}}}\label{nip:ha}
For example:\par
- \pali{m\=a \textbf{ha} pana me bhante bhagav\=a} (Sir, the Buddha did not say to me.)\footnote{I translated this from Thai. The example is not found in any text, at least in this form. I suspect that it is in fact \pali{m\=aha} (\pali{m\=a + \=aha}), so the translation looks probable.} \par

\subsection*{\fbox{\pali{Hi}}}\label{nip:hi}
For example:\par
- \pali{So \textbf{h}\=avuso, bhagav\=a j\=ana\d m j\=an\=ati, passa\d m passati}\footnote{M1\,203 (MN\,18)} (That Buddha, my dear, [when] knows, [he says I] know, [when] sees, [he says I] see.) \par

\section*{Index of particles}

Particles mentioned in this appendix is numerous. To help learners, I put all of them into order and add the referencing points. The table below shows the result of this effort. There are around 250 particles listed in the table.

\bigskip
\begin{longtable}[c]{%
	>{\itshape\raggedright\arraybackslash}p{0.35\linewidth}%
	>{\raggedleft\arraybackslash}p{0.25\linewidth}}
\caption*{Index of particles}\\
\toprule
\bfseries\upshape Particle & \bfseries\upshape Page \\ \midrule
\endfirsthead
\multicolumn{2}{c}{Index of particles (contd\ldots)}\\
\toprule
\bfseries\upshape Particle & \bfseries\upshape Page \\ \midrule
\endhead
\bottomrule
\ltblcontinuedbreak{2}
\endfoot
\bottomrule
\endlastfoot
%
a & \pageref{nip:a} \\
acira\d m & \pageref{nip:aciradm} \\
ajju & \pageref{nip:intime} \\
ajjhatta\d m & \pageref{nip:inplace} \\
a\~n\~natra & \pageref{nip:aynynatra} \\
a\~n\~nad\=a & \pageref{nip:intime} \\
a\~n\~nadatthu & \pageref{nip:aynynadatthu} \\
atthi & \pageref{nip:atthi} \\
atha & \pageref{nip:atha} \\
atho & \pageref{nip:atho} \\
addh\=a & \pageref{nip:addhaa} \\
adhun\=a & \pageref{nip:intime} \\
adho & \pageref{nip:inplace} \\
antar\=a & \pageref{nip:inplace} \\
anto & \pageref{nip:inplace} \\
aparajju & \pageref{nip:intime} \\
api & \pageref{nip:api} \\
ap\=ac\=ina\d m & \pageref{nip:inplace} \\
appeva & \pageref{nip:appeva} \\
appeva n\=ama & \pageref{nip:appevanaama} \\
abhikkha\d na\d m & \pageref{nip:intime} \\
abhi\d nha\d m & \pageref{nip:intime} \\
abhito & \pageref{nip:inplace} \\
ambho & \pageref{nip:voc} \\
are & \pageref{nip:voc} \\
ala\d m & \pageref{nip:aladm} \\
assu & \pageref{nip:assu} \\
aho & \pageref{nip:aho} \\
\=a & \pageref{nip:aa} \\
\=ama & \pageref{nip:aama} \\
\=amo & \pageref{nip:aamo} \\
\=ayati\d m & \pageref{nip:intime} \\
\=arak\=a & \pageref{nip:inplace} \\
\=ar\=a & \pageref{nip:inplace} \\
\=av\=i & \pageref{nip:inplace} \\
\=avuso & \pageref{nip:voc} \\
i\.ngha & \pageref{nip:ingha} \\
iti & \pageref{nip:iti} \\
id\=ani & \pageref{nip:intime} \\
ittha\d m & \pageref{nip:itthadm} \\
iva & \pageref{nip:iva} \\
\=isaka\d m & \pageref{nip:iisakadm} \\
ekajjha\d m & \pageref{nip:inplace} \\
ekad\=a & \pageref{nip:intime} \\
ekamanta\d m & \pageref{nip:inplace} \\
etarahi & \pageref{nip:intime} \\
ett\=avat\=a & \pageref{nip:ettaavataa} \\
ena\d m & \pageref{nip:enadm} \\
eva & \pageref{nip:eva} \\
evameta\d m & \pageref{nip:evametadm} \\
evameva & \pageref{nip:evameva} \\
evameva\d m & \pageref{nip:evamevadm} \\
evampi & \pageref{nip:evampi} \\
eva\d m & \pageref{nip:evadm} \\
ucca\d m & \pageref{nip:inplace} \\
uttarasuve & \pageref{nip:intime} \\
uttarasve & \pageref{nip:intime} \\
ud\=ahu & \pageref{nip:udaahu} \\
uddha\d m & \pageref{nip:inplace} \\
upari & \pageref{nip:inplace} \\
op\=ayika\d m & \pageref{nip:opaayikadm} \\
ora\d m & \pageref{nip:inplace} \\
kacci & \pageref{nip:kacci} \\
kate & \pageref{nip:kate} \\
katha\~nci & \pageref{nip:kathaynci} \\
katha\d m & \pageref{nip:kathadm} \\
kad\=a & \pageref{nip:intime} \\
kalla\d m & \pageref{nip:kalladm} \\
kasm\=a & \pageref{nip:kasmaa} \\
kaha\d m & \pageref{nip:kahadm} \\
k\=ama\d m & \pageref{nip:kaamadm} \\
k\=ala\d m & \pageref{nip:intime} \\
ki\~nc\=api & \pageref{nip:kiyncaapi} \\
kitt\=avat\=a & \pageref{nip:kittaavataa} \\
kinti & \pageref{nip:kinti} \\
kinnu & \pageref{nip:kinnu} \\
kira & \pageref{nip:kira} \\
kismi\d m viya & \pageref{nip:kismidmviya} \\
ki\d m & \pageref{nip:kidm} \\
ki\d msu & \pageref{nip:kidmsu} \\
k\=iva & \pageref{nip:kiiva} \\
ku & \pageref{nip:ku} \\
kud\=acana\d m & \pageref{nip:intime} \\
ko & \pageref{nip:ko} \\
kkhattu\d m & \pageref{nip:kkhattudm} \\
khalu & \pageref{nip:khalu} \\
khippa\d m & \pageref{nip:khippadm} \\
kho & \pageref{nip:kho} \\
ca & \pageref{nip:ca} \\
carahi & \pageref{nip:carahi} \\
ci & \pageref{nip:ci} \\
cira\d m & \pageref{nip:ciradm} \\
cirassa\d m & \pageref{nip:cirassadm} \\
ce & \pageref{nip:ce} \\
j\=atu & \pageref{nip:jaatu} \\
j\=atucche & \pageref{nip:jaatucche} \\
je & \pageref{nip:voc} \\
taggha & \pageref{nip:taggha} \\
tato & \pageref{nip:tato} \\
tathariva & \pageref{nip:tathariva} \\
tath\=a & \pageref{nip:tathaa} \\
tath\=a hi & \pageref{nip:tathaahi} \\
tatheva & \pageref{nip:tatheva} \\
tadeva & \pageref{nip:tadeva} \\
tad\=a & \pageref{nip:intime} \\
tad\=ani & \pageref{nip:intime} \\
tave & \pageref{nip:tave} \\
tasm\=a & \pageref{nip:tasmaa} \\
ta\d m & \pageref{nip:tadm} \\
t\=ava & \pageref{nip:taava} \\
t\=avat\=a & \pageref{nip:taava} \\
tiriya\d m & \pageref{nip:inplace} \\
tiro & \pageref{nip:inplace} \\
tu & \pageref{nip:tu} \\
tu\d nh\=i & \pageref{nip:tudnhii} \\
tuna & \pageref{nip:tuna} \\
tuva\d ta\d m & \pageref{nip:tuvadtadm} \\
tu\d m & \pageref{nip:tudm} \\
tena & \pageref{nip:tena} \\
to & \pageref{nip:to1}, \pageref{nip:to2}, \pageref{nip:to3} \\
tv\=a & \pageref{nip:tvaa} \\
tv\=ana & \pageref{nip:tvaana} \\
div\=a & \pageref{nip:divaa}, \pageref{nip:intime} \\
du\d t\d thu & \pageref{nip:dudtdthu} \\
dh\=a & \pageref{nip:dhaa} \\
dh\=i & \pageref{nip:dhii} \\
dh\=iratthu & \pageref{nip:dhiiratthu} \\
dhuva\d m & \pageref{nip:dhuvadm} \\
na & \pageref{nip:na} \\
nanu & \pageref{nip:nanu} \\
namo & \pageref{nip:namo} \\
na\d m & \pageref{nip:nadm} \\
n\=ama & \pageref{nip:naama} \\
naana\d m & \pageref{nip:naanadm} \\
n\=an\=a & \pageref{nip:naanaa} \\
nicca\d m & \pageref{nip:intime} \\
n\=ica\d m & \pageref{nip:inplace} \\
nu & \pageref{nip:nu} \\
nukho & \pageref{nip:nukho} \\
n\=una & \pageref{nip:nuuna} \\
no & \pageref{nip:no} \\
paccatta\d m & \pageref{nip:paccattadm} \\
pacch\=a & \pageref{nip:inplace} \\
patir\=upa\d m & \pageref{nip:patiruupadm} \\
pana & \pageref{nip:pana} \\
paramukh\=a & \pageref{nip:inplace} \\
parasuve & \pageref{nip:intime} \\
parito & \pageref{nip:inplace} \\
pare & \pageref{nip:intime} \\
pecca & \pageref{nip:inplace} \\
p\=atu & \pageref{nip:paatu} \\
p\=ato & \pageref{nip:intime} \\
p\=ara\d m & \pageref{nip:inplace} \\
puthu & \pageref{nip:puthu} \\
puna & \pageref{nip:puna} \\
punappuna\d m & \pageref{nip:punuppunadm} \\
puna\d m & \pageref{nip:punadm} \\
puno & \pageref{nip:puno} \\
pur\=a & \pageref{nip:intime} \\
pure & \pageref{nip:inplace} \\
pi & \pageref{nip:pi} \\
bahi & \pageref{nip:inplace} \\
bahiddh\=a & \pageref{nip:inplace} \\
b\=ahira & \pageref{nip:inplace} \\
b\=ahira\d m & \pageref{nip:inplace} \\
bha\d ne & \pageref{nip:voc} \\
bhiyyo & \pageref{nip:bhiyyo} \\
bh\=utapubba\d m & \pageref{nip:intime} \\
bho & \pageref{nip:voc} \\
mana\d m & \pageref{nip:manadm} \\
m\=a & \pageref{nip:maa} \\
m\=arisa & \pageref{nip:voc} \\
micch\=a & \pageref{nip:micchaa} \\
muhutta\d m & \pageref{nip:intime} \\
muhu\d m & \pageref{nip:intime} \\
yagghe & \pageref{nip:yagghe} \\
ya\~nce & \pageref{nip:yaynce} \\
yathariva & \pageref{nip:yathariva} \\
yath\=a & \pageref{nip:yathaa1}, \pageref{nip:yathaa2}\\
yath\=api & \pageref{nip:yathaapi} \\
yatheva & \pageref{nip:yatheva} \\
yadeva & \pageref{nip:yadeva} \\
yad\=a & \pageref{nip:intime} \\
yadi & \pageref{nip:yadi} \\
yasm\=a & \pageref{nip:yasmaa} \\
y\=ava & \pageref{nip:yaava} \\
y\=avat\=a & \pageref{nip:yaava} \\
ratti & \pageref{nip:intime} \\
rahit\=a & \pageref{nip:rahitaa} \\
raho & \pageref{nip:inplace} \\
rite & \pageref{nip:rite} \\
re & \pageref{nip:voc} \\
labbh\=a & \pageref{nip:labbhaa} \\
lahu & \pageref{nip:lahu} \\
lahu\d m & \pageref{nip:lahudm} \\
vata & \pageref{nip:vata} \\
vatha & \pageref{nip:vatha} \\
vin\=a & \pageref{nip:vinaa} \\
visu\d m & \pageref{nip:visudm} \\
v\=a & \pageref{nip:vaa} \\
viya & \pageref{nip:viya} \\
ve & \pageref{nip:ve} \\
vo & \pageref{nip:vo} \\
sakk\=a & \pageref{nip:sakkaa} \\
sakkhi & \pageref{nip:sakkhi} \\
sace & \pageref{nip:sace} \\
sacchi & \pageref{nip:sacchi} \\
sajja & \pageref{nip:intime} \\
sajju & \pageref{nip:intime} \\
sa\d nika\d m & \pageref{nip:sadnikadm} \\
satata\d m & \pageref{nip:intime} \\
sad\=a & \pageref{nip:intime} \\
saddhi\d m & \pageref{nip:saddhidm} \\
sabbad\=a & \pageref{nip:intime} \\
samantato & \pageref{nip:inplace} \\
samant\=a & \pageref{nip:inplace} \\
sama\d m & \pageref{nip:samadm} \\
sampati & \pageref{nip:intime} \\
samma & \pageref{nip:voc} \\
samm\=a & \pageref{nip:sammaa} \\
sammukh\=a & \pageref{nip:inplace} \\
saya\d m & \pageref{nip:sayadm} \\
sasakka\d m & \pageref{nip:sasakkadm} \\
saha & \pageref{nip:saha} \\
s\=adhu & \pageref{nip:saadhu} \\
s\=amant\=a & \pageref{nip:inplace} \\
s\=ama\d m & \pageref{nip:saamadm} \\
s\=aya\d m & \pageref{nip:intime} \\
s\=ahu & \pageref{nip:saahu} \\
su & \pageref{nip:su} \\
su\d t\d thu & \pageref{nip:sudtdthu} \\
suda\d m & \pageref{nip:sudadm} \\
suve & \pageref{nip:intime} \\
seyyath\=api & \pageref{nip:seyyathaapi} \\
seyyath\=api n\=ama & \pageref{nip:seyyathaapinaama} \\
seyyathida\d m & \pageref{nip:seyyathidadm} \\
suvatthi & \pageref{nip:suvatthi} \\
so & \pageref{nip:so1}, \pageref{nip:so2} \\
sotthi & \pageref{nip:sotthi} \\
sve & \pageref{nip:intime} \\
ha & \pageref{nip:ha} \\
handa & \pageref{nip:handa} \\
hambho & \pageref{nip:voc} \\
hare & \pageref{nip:voc} \\
hala\d m & \pageref{nip:haladm} \\
have & \pageref{nip:have} \\
h\=a & \pageref{nip:haa} \\
hi & \pageref{nip:hi} \\
hiyyo & \pageref{nip:intime} \\
he & \pageref{nip:voc} \\
he\d t\d th\=a & \pageref{nip:inplace} \\
hura\d m & \pageref{nip:inplace} \\
\end{longtable}
