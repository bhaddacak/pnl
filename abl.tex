\chapter{I go to school \headhl{from home}}\label{chap:abl}

As I said previously that P\=ali has no individual prepositions as English does, adding just a simple part such as ``from home'' to a sentence seems to have more work to do. You have to know a new case of declension---\emph{ablative}.

\phantomsection
\addcontentsline{toc}{section}{Declension of Ablative Case}
\section*{Declension of Ablative Case}

The main use of ablative case is to mark the origin, from where things move. This can be abstract as the cause or the motivation of actions. Table \ref{tab:ablreg} summarizes the declension of this case for regular nouns. Ablative case is quite easy to recognize, particularly singular forms of m.\ and nt.\ which are the same. The plural ending \pali{hi} or \pali{bhi} is a good clue to identify ablative case, but it can be confused with instrumental case (we shall see later). Singular endings of f.\ are also easy to recognize, but it can be confusing with other cases as well because most declensions of f.\ nouns have indistinct forms. For instance, genitive and ablative cases of sg.\ f.\ are all the same. However, these f.\ endings are a good clue for gender identification.

\begin{table}[!hbt]
\centering
\caption{Ablative case endings of regular nouns}
\label{tab:ablreg}
\bigskip
\begin{tabular}{@{}>{\bfseries}l*{5}{>{\itshape}l}@{}} \toprule
\multirow{2}{*}{G. Num.} & \multicolumn{5}{c}{\bfseries Endings} \\
\cmidrule(l){2-6}
& a & i & \=i & u & \=u\\
\midrule
m. sg. & asm\=a & ism\=a & \replacewith{\=i}{ism\=a} & usm\=a & \replacewith{\=u}{usm\=a} \\
& amh\=a & imh\=a & \replacewith{\=i}{imh\=a} & umh\=a & \replacewith{\=u}{umh\=a} \\
& \texthl{\replacewith{a}{\=a}} & & & & \\
m. pl. & \replacewith{a}{ehi} & \replacewith{i}{\=ihi} & \=ihi & \replacewith{u}{\=uhi} & \=uhi \\
& \replacewith{a}{ebhi} & \replacewith{i}{\=ibhi} & \=ibhi & \replacewith{u}{\=ubhi} & \=ubhi \\
\midrule
nt. sg. & asm\=a & ism\=a & & usm\=a & \\
& amh\=a & imh\=a & & umh\=a & \\
& \texthl{\replacewith{a}{\=a}} & & & & \\
nt. pl. & \replacewith{a}{ehi} & \replacewith{i}{\=ihi} & & \replacewith{u}{\=uhi} & \\
& \replacewith{a}{ebhi} & \replacewith{i}{\=ibhi} & & \replacewith{u}{\=ubhi} & \\
\midrule
& \=a & i & \=i & u & \=u\\
\midrule
f. sg. & \=aya & iy\=a & \replacewith{\=i}{iy\=a} & uy\=a & \replacewith{\=u}{uy\=a} \\
f. pl. & \=ahi & \replacewith{i}{\=ihi} & \=ihi & \replacewith{u}{\=uhi} & \=uhi \\
& \=abhi & \replacewith{i}{\=ibhi} & \=ibhi & \replacewith{u}{\=ubhi} & \=ubhi \\
\bottomrule
\end{tabular}
\end{table}

Like other previous chapters, we have to learn the declension of pronouns at the same time. The summary is shown in Table \ref{tab:ablpron}.

\begin{table}[!hbt]
\centering
\caption{Ablative case of pronouns}
\label{tab:ablpron}
\bigskip
\begin{tabular}{@{}*{5}{>{\itshape}l}@{}} \toprule
\multirow{2}{*}{\bfseries\upshape Pron.} & \multicolumn{2}{c}{\bfseries\upshape m./nt.} & \multicolumn{2}{c}{\bfseries\upshape f.} \\
\cmidrule(lr){2-3} \cmidrule(lr){4-5}
& \bfseries\upshape sg. & \bfseries\upshape pl. & \bfseries\upshape sg. & \bfseries\upshape pl. \\
\midrule
amha & may\=a & amhehi & & \\
tumha & tay\=a & tumhehi & & \\
ta & tasm\=a & tehi & t\=aya & t\=ahi \\
& tamh\=a & tebhi & & t\=abhi \\
& asm\=a & & & \\
eta & etasm\=a & etehi & et\=aya & et\=ahi \\
& etamh\=a & etebhi & & et\=abhi \\
ima & imasm\=a & imehi & im\=aya & im\=ahi \\
& imamh\=a & imebhi & & im\=abhi \\
& asm\=a & & & \\
amu & amusm\=a & am\=uhi & amuy\=a & am\=uhi \\
& amumh\=a & am\=ubhi & & am\=ubhi \\
\bottomrule
\end{tabular}
\end{table}

After you are familiar with ablative forms, now we can say ``I go to school from home.''

\palisample{aha\d m gehasm\=a p\=a\d thas\=ala\d m gacch\=ami. \sampleor aha\d m gehamh\=a p\=a\d thas\=ala\d m gacch\=ami. \sampleor[or more often] aha\d m geh\=a p\=a\d thas\=ala\d m gacch\=ami.}

Remember that when adjectives are used to modify nouns, they have to take the same case as the noun they modify. For example, ``A big man goes from a big house to a big school'' can be rendered as:

\palisample{mahanto puriso mahantasm\=a gehasm\=a mahanta\d m p\=a\d thas\=ala\d m gacchati.}

Ablative case can also denote the cause of the action. For example, we can say ``People go to the city because they are poor'' simply as:

\palisample{jan\=a da\d liddasm\=a nagara\d m gacchanti.}

Beside being used to specify the source or cause of the action, abl.\ can also be used in adjective comparison. For example, to say ``That girl is more beautiful than me'' using abl., you have to change the sentence to ``That girl is beautiful from me.'' Hence:

\palisample{may\=a es\=a sundar\=a hoti.}

We will talk about adjective comparison in detail later in Chapter \ref{chap:adjcomp}.

\phantomsection
\addcontentsline{toc}{section}{Verbs taking ablatives}
\section*{Verbs taking ablatives}\label{sec:ablverbs}

There are a number of verbs, instead of taking acc.\ as its object, taking abl. I list some of them in Table \ref{tab:ablverb}. The list does not contain verbs that require abl.\ by their meaning, e.g.\ \pali{patati} (fall), \pali{nikhamati} (go out).\footnote{It is worth seeing \citealp[pp.~90--2]{warder:intro} for some various uses of ablative case.} I list only the peculiar ones.

\begin{table}[!hbt]
\centering
\caption{Verbs taking ablatives}
\label{tab:ablverb}
\bigskip
\begin{tabular}{>{\itshape}ll} \toprule
\bfseries\upshape Verb & \bfseries Meaning \\ \midrule
bh\=ayati & fear \\
uttasati & be alarmed, be terrified \\
viramati & abstain, cease \\
\bottomrule
\end{tabular}
\end{table}

When we say we fear or are terrified by something, normally we use abl.\footnote{However, you can find this in the canon: ``\pali{na ta\d m bh\=ay\=ami \=avuso}'' (I don't fear that, man) (S1\,164, SN\,5). This is in poetic form.}---``I fear from something.'' For example, you can say ``I fear snake'' by:

\palisample{aha\d m sappasm\=a bh\=ay\=ami.}

You can replace \pali{bh\=ayati} with \pali{uttasati} because the meanings of both are close. Abstaining from something in P\=ali is like English, e.g.\ \pali{p\=a\d n\=atip\=at\=a viram\=ami} (I abstain from taking lives).

\section*{Exercise \ref{chap:abl}}
Say these in P\=ali.
\begin{compactenum}
\item From my village, I go to college.
\item That bus comes from her house to our city.
\item From their poor countries, many foreign workers go to America.
\item Those fat people, because of health, go to that hospital.
\item Because you (pl.) are ugly, you go to barber's shop.
\item That pig is heavier than those cats.
\end{compactenum}
